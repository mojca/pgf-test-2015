% Copyright 2013 by Till Tantau
%
% This file may be distributed and/or modified
%
% 1. under the LaTeX Project Public License and/or
% 2. under the GNU Public License.
%
% See the file doc/generic/pgf/licenses/LICENSE for more details.

\ProvidesFileRCS[v\pgfversion] $Header: /cvsroot/pgf/pgf/generic/pgf/modules/pgfmodulebending.code.tex,v 1.1 2013/08/29 22:35:09 tantau Exp $

%
% This file defines commands for "bending coordinate systems". This is
% needed, for instance, for bend arrow heads.
%
%
% By "bending coordinate system" I mean a curvilinear coordinate
% system in which the x-axis "goes along" a Bezier curve and the
% y-axis is always perpendicular to the (Bezier) curve at the given
% x-coordinate. Formally, given a pair (x,y), let B(x) be the point on
% the Bezier curve B at distance x from the start of the curve. Let
% T(x) be the tangent of B at B(x) and let P(x) be the (normalized)
% vector perpendicular to T(x). Then (x,y) would be mapped to B(x) +
% y*P(x). As an example, if B is a circle, then the corresponding
% bending coordinate system is (essentially, except for an offset in
% the y value) the polar coordinate system.
%
% Since it is next to impossible to quickly compute precise length
% of Bezier curves in TeX, instead of real distances along Bezier
% curves, we use the following trick: We compute precisely the "time" t_1
% (in the parametrization of the Bezier curve as a function of "time")
% needed to travel along one unit (1pt in TeX). We then act as if a
% length of, say, 20pt were then reached at time 20*t_1. In reality,
% this will not "really" be the correct point, but close enough for
% our purposes; especially when only small values of x near to the
% start/end of the curve are used.
%
% You use the bending cs by first telling pgf which Bezier curve you
% are interested in (using \pgfsetbendingline or \pgfsetbendingcurve)
% and then using \pgfpointbending. The results will be best if small
% values of x are used because of the abovementioned trick of
% substituing distance by time. 
%
% When a new bending cs is installed, some, possibly expensive,
% precomputations are done. Subsequent calls to \pgfpointbending
% should then be relatively quick.



% Returns a point in the current bending coordinate system.
% 
% #1 = distance along the curve
% #2 = distance perpendicular to the curve
% 
% Description:
% 
% Returns the computed position in \pgf@x and \pgf@y.

\def\pgfpointbending#1#2{%
  \pgf@bending@point{#1}{#2}%
}

\def\pgf@bending@point{\pgferror{No bending coordinate system set}}



% Sets B to the straight line from #1 to #2. 
%
% #1 = start point
% #2 = end point
%
% Example:
%
% \pgfsetbendingline{\pgfpoint{1cm}{1cm}}{\pgfpoint{3cm}{2cm}}
% \pgfpointbending{1pt}{0pt} % this will be 1pt along the line from
%                            % (1,1) to (3,2)

\def\pgfsetbendingline#1#2{%
  \pgf@process{#1}%
  \edef\pgf@bending@line@start{\noexpand\pgfqpoint{\the\pgf@x}{\the\pgf@y}}%
  \pgf@process{\pgfpointnormalised{\pgfpointdiff{}{#2}}}%
  \edef\pgf@bending@line@normal{\noexpand\pgfqpoint{\the\pgf@x}{\the\pgf@y}}%
  \pgf@xa-\pgf@x%
  \pgf@y=\pgf@x%
  \pgf@x=\pgf@xa%
  \edef\pgf@bending@line@orth{\noexpand\pgfqpoint{\the\pgf@x}{\the\pgf@y}}%
  \let\pgf@bending@point\pgf@bending@line@point%
}

\def\pgf@bending@line@point#1#2{%
  \pgfmathsetmacro\pgf@bending@xfactor{#1}%
  \pgfmathsetmacro\pgf@bending@yfactor{#2}%
  \pgf@process{\pgf@bending@line@normal}%
  \pgf@xa\pgf@bending@xfactor\pgf@x%
  \pgf@ya\pgf@bending@xfactor\pgf@y%
  \pgf@process{\pgf@bending@line@orth}%
  \advance\pgf@xa by\pgf@bending@yfactor\pgf@x%
  \advance\pgf@ya by\pgf@bending@yfactor\pgf@y%
  \pgf@process{\pgf@bending@line@start}%
  \advance\pgf@x by\pgf@xa%
  \advance\pgf@y by\pgf@ya%
}



% Sets B to the curve line from #1 to #4 via the control points #2 and
% #3. 
%
% #1 = start point
% #2 = first control point
% #3 = second control point
% #4 = end point
%
% Example:
%
% \pgfsetbendingline
% {\pgfpoint{0mm}{10mm}}
% {\pgfpoint{5.5mm}{10mm}}
% {\pgfpoint{10mm}{5.5mm}}
% {\pgfpoint{10mm}{0mm}} % nearly a quarter circle
% \pgfpointbending{5mm}{5mm} % should be 5mm along the circle, put at
%                            % distance 15mm from the origin (5mm fromt he circle line).

\def\pgfsetbendingcurve#1#2#3#4{%
  \pgf@process{#1}%
  \edef\pgf@bending@line@a{\noexpand\pgfqpoint{\the\pgf@x}{\the\pgf@y}}%
  \pgf@xa=-\pgf@x%
  \pgf@ya=-\pgf@y%
  \pgf@process{#2}%
  \edef\pgf@bending@line@b{\noexpand\pgfqpoint{\the\pgf@x}{\the\pgf@y}}%
  \pgf@process{#3}%
  \edef\pgf@bending@line@c{\noexpand\pgfqpoint{\the\pgf@x}{\the\pgf@y}}%
  \pgf@process{#4}%
  \edef\pgf@bending@line@d{\noexpand\pgfqpoint{\the\pgf@x}{\the\pgf@y}}%
  \advance\pgf@xa by\pgf@x%
  \advance\pgf@ya by\pgf@y%
  \pgfmathparse{veclen(\the\pgf@xa,\the\pgf@ya)}%
  \pgfmathsetmacro\pgf@bending@line@unit{1/\pgfmathresult}%
  \let\pgf@bending@point\pgf@bending@curve@point%
}

\def\pgf@bending@curve@point#1#2{%
  \pgfpointcurveattime{(#1)*\pgf@bending@line@unit}{\pgf@bending@line@a}{\pgf@bending@line@b}{\pgf@bending@line@c}{\pgf@bending@line@d}
  \pgf@xc=\pgf@x% save
  \pgf@yc=\pgf@y% save
  % compute normal:
  \advance\pgf@xb by-\pgf@x%
  \advance\pgf@yb by-\pgf@y%
  \pgf@process{\pgfpointnormalised{\pgf@x=\pgf@yb\pgf@y=-\pgf@xb}}
  \pgfmathsetmacro\pgf@bending@yfactor{#2}%
  \pgf@x=\pgf@bending@yfactor\pgf@x%
  \pgf@y=\pgf@bending@yfactor\pgf@y%
  \advance\pgf@x by\pgf@xc%
  \advance\pgf@y by\pgf@yc%
}





\endinput

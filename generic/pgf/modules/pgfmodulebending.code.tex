% Copyright 2013 by Till Tantau
%
% This file may be distributed and/or modified
%
% 1. under the LaTeX Project Public License and/or
% 2. under the GNU Public License.
%
% See the file doc/generic/pgf/licenses/LICENSE for more details.

\ProvidesFileRCS[v\pgfversion] $Header: /cvsroot/pgf/pgf/generic/pgf/modules/pgfmodulebending.code.tex,v 1.7 2013/09/17 20:22:04 tantau Exp $


%
% This file defines commands for drawing bending arrows and lines.
%

\usepgfmodule{nonlineartransformations}
\usepgflibrary{curvilinear}


% Configuration options for arrow tips:

\pgfkeys{
  /pgf/@arrows/flex/.code=\pgfarrowsaddtooptions{\pgf@bend@set@rigidity{#1}},
  /pgf/@arrows/flex/.default=1,
  /pgf/@arrows/bend/.code=\pgfarrowsaddtooptions{\pgf@bend@set@rigidity{-1}}
}

\def\pgf@bend@set@rigidity#1{\def\pgf@arrows@the@rigidity{#1}\pgf@shorten@curvedtrue}

%
%
% Handle the end of a curve
%
% 

% First, the preparation. This is, essenatially, setting up the
% nonlinear transformation

\def\pgf@prep@curveend{
  \pgftransformreset%
  \pgfsetcurvilinearbeziercurve{\pgfpointlastonpath}{\pgfpointsecondlastonpath}{\pgfpointthirdlastonpath}{\pgfpointfourthlastonpath}
}

% Second, the shortening.

\def\pgf@do@shorten@curvedend{%
  \expandafter\expandafter\expandafter\pgf@do@shorten@curvedend@unpack\expandafter\pgfpointfourthlastonpath\pgfpointthirdlastonpath%
}
\def\pgf@do@shorten@curvedend@unpack\pgfqpoint#1#2\pgfqpoint#3#4{%
  \pgf@process{\pgfcurvilineardistancetotime{\pgf@path@shortening@distance}}%
  % Compute new curve:
  \pgfpointcurveattime{\pgf@x}{\pgfpointlastonpath}{\pgfpointsecondlastonpath}{\pgfpointthirdlastonpath}{\pgfpointfourthlastonpath}%
  % \pgfx/y and \pgf xa/xb already correct. Need to setup remaining vector...
  \pgf@xb#3%
  \pgf@yb#4%
  \pgf@xb\pgf@time@t\pgf@xb%
  \pgf@yb\pgf@time@t\pgf@yb%
  \advance\pgf@xb by\pgf@time@s\pgf@xc%
  \advance\pgf@yb by\pgf@time@s\pgf@yc%
  \edef\pgfprocessresultsubpathsuffix{%
    \expandafter\noexpand\pgfsubpathfourthlasttoken{#1}{#2}%
    \noexpand\pgfsyssoftpath@curvetosupportatoken{\the\pgf@xb}{\the\pgf@yb}%
    \noexpand\pgfsyssoftpath@curvetosupportbtoken{\the\pgf@xa}{\the\pgf@ya}%
    \noexpand\pgfsyssoftpath@curvetotoken{\the\pgf@x}{\the\pgf@y}}%
  \expandafter\expandafter\expandafter\def%
  \expandafter\expandafter\expandafter\pgfprocessresultpathsuffix%
  \expandafter\expandafter\expandafter{\expandafter\pgfprocessresultsubpathprefix\pgfprocessresultsubpathsuffix}%
}

% Third, the drawing

\def\pgf@do@draw@curvedend{%
  \pgf@do@draw@curved\pgf@end@tip@sequence%
}

\def\pgf@do@draw@curved#1{%
  {%
    \pgftransformreset%
    \pgf@xb\pgf@arrow@tip@total@length% xb will keep track of the total length
    \let\pgf@arrow@handle\pgf@draw@curved%
    \let\pgf@arrow@handle@dot\relax%
    #1%
  }%
}

\def\pgf@draw@curved#1#2{%
  % Prepare:
  {%
    \pgfarrows@getid{#1}{#2}%
    % Do shift:
    \expandafter\expandafter\expandafter\pgf@arrow@drawer@rigid@shift\csname pgf@ar@ends@\pgf@arrow@id\endcsname%
    % do swap:
    {%
      \csname pgf@ar@saves@\pgf@arrow@id\endcsname%
      \expandafter\expandafter\expandafter\pgf@arrow@drawer@rigid@or@bend\csname pgf@ar@rigid@\pgf@arrow@id\endcsname%      
      % hull points inside the above
    }%
  \expandafter}%
  % Transform to next tip:
  \expandafter\pgf@xb\the\pgf@xb%
}

\def\pgf@arrow@drawer@rigid@shift#1#2#3#4{% tip end, back end, line end, sep
  % Let xa be the actual back end of the current arrow plus the back end:
  \advance\pgf@xb by#2%
  \pgf@xa\pgf@xb%
  % Update the xb:
  \pgf@x#1%
  \advance\pgf@x by#4%
  \advance\pgf@xb by-\pgf@x%
}


\def\pgf@arrow@drawer@rigid@or@bend#1#2{% rigid start, rigid end, flexing
  \ifdim\pgf@arrows@the@rigidity pt<0pt%
    \pgftransformnonlinear{\pgfpointcurvilinearbezierorthogonal{\pgf@x}{\pgf@y}}
    \pgftransformscale{-1}%
    \pgftransformxshift{-\pgf@xa}%
    \ifpgfarrowswap%
      \pgftransformyscale{-1}%
    \fi%
    \pgf@relevantforpicturesizefalse%
    \pgfscope\csname pgf@ar@code@\pgf@arrow@id\endcsname\endpgfscope%
    \pgf@arrows@bending@hull%
  \else%
    % Compute the rigid positions:
    \pgf@x#1% start
    \pgf@y#2% flexibility*(end-start) + start
    \advance\pgf@y by-\pgf@x%
    \pgf@y\pgf@arrows@the@rigidity\pgf@y%
    \advance\pgf@y by\pgf@x%
    % Are they equal?
    \ifdim\pgf@x=\pgf@y%
      % Yes, do tangent:
      \pgf@bending@do@tangent%
    \else%
      \pgf@bending@do@flex%
    \fi%
    \pgf@xc#1%
    \pgf@xc-\pgf@xc%
    \pgftransformxshift{+\pgf@xc}%
    \ifpgfarrowswap%
      \pgftransformyscale{-1}%
    \fi%
    \pgfscope\pgflowlevelsynccm\csname pgf@ar@cache@\pgf@arrow@id\endcsname\endpgfscope%
    \pgf@arrows@rigid@hull%
  \fi%
}

\def\pgf@bending@do@tangent{%
  \pgf@xb\pgf@xa%
  \advance\pgf@xb by-\pgf@x%
  \pgfcurvilineardistancetotime{\pgf@xb}%
  \pgf@xb\pgf@x%
  % Compute points:
  \pgfpointcurveattime{\pgf@xb}{\pgf@curvilinear@line@a}{\pgf@curvilinear@line@b}{\pgf@curvilinear@line@c}{\pgf@curvilinear@line@d}%
  \pgf@xc\pgf@x% save
  \pgf@yc\pgf@y% save
  % compute normal:
  \advance\pgf@xb by-\pgf@xa%
  \advance\pgf@yb by-\pgf@ya%
  \pgf@process{\pgfpointnormalised{\pgf@x=\pgf@xb\pgf@y=\pgf@yb}}
  \pgf@ya-\pgf@y%
  \pgfsettransformentries%
    {\pgf@sys@tonumber\pgf@x}{\pgf@sys@tonumber\pgf@y}%
    {\pgf@sys@tonumber\pgf@ya}{\pgf@sys@tonumber\pgf@x}{\pgf@xc}{\pgf@yc}%
}%
  
\def\pgf@bending@do@flex{
  \pgf@xb\pgf@xa%
  \pgf@xc\pgf@xa%
  \advance\pgf@xc by-\pgf@x%
  \advance\pgf@xb by-\pgf@y%
  \pgfcurvilineardistancetotime{\pgf@xc}%
  \pgf@xc\pgf@x%
  \pgfcurvilineardistancetotime{\pgf@xb}%
  \pgf@xb\pgf@x%
  % Compute points:
  \pgf@process{\pgfpointcurveattime{\pgf@xc}{\pgf@curvilinear@line@a}{\pgf@curvilinear@line@b}{\pgf@curvilinear@line@c}{\pgf@curvilinear@line@d}}%
  \pgf@xc\pgf@x\pgf@yc\pgf@y%
  \pgf@process{\pgfpointcurveattime{\pgf@xb}{\pgf@curvilinear@line@a}{\pgf@curvilinear@line@b}{\pgf@curvilinear@line@c}{\pgf@curvilinear@line@d}}%
  \pgf@xb\pgf@x\pgf@yb\pgf@y%
  \pgftransformarrow{\pgfqpoint{\pgf@xb}{\pgf@yb}}{\pgfqpoint{\pgf@xc}{\pgf@yc}}%
}






% Convex hull stuff:

\def\pgf@arrows@bending@hull{%
  \ifpgfoverlayarrowtips\else%
    % Do bb update:
    \expandafter\let\expandafter\pgf@temp\csname pgf@ar@hull@\pgf@arrow@id\endcsname%
    \ifx\pgf@temp\pgfutil@empty\else%
      \let\pgf@arrow@hull@point\pgf@arrow@bending@hull@point@first%
      \pgf@temp%
      \pgf@arrow@bending@update@bb%
    \fi%
  \fi%  
}

\def\pgf@arrow@bending@hull@point@first#1#2{%
  \pgfpointtransformednonlinear{\pgfqpoint{#1}{#2}}%
  \pgf@xa\pgf@x\pgf@xb\pgf@x%
  \pgf@ya\pgf@y\pgf@yb\pgf@y%
  \let\pgf@arrow@hull@point\pgf@arrow@bending@hull@point@other%
}
\def\pgf@arrow@bending@hull@point@other#1#2{%
  \pgfpointtransformednonlinear{\pgfqpoint{#1}{#2}}%
  \ifdim\pgf@x<\pgf@xa\pgf@xa\pgf@x\else\ifdim\pgf@x>\pgf@xb\pgf@xb\pgf@x\fi\fi%
  \ifdim\pgf@y<\pgf@ya\pgf@ya\pgf@y\else\ifdim\pgf@y>\pgf@yb\pgf@yb\pgf@y\fi\fi%
}

\def\pgf@arrow@bending@update@bb{%
  \ifdim\pgf@picmaxx=-16000pt\relax%
    \global\pgf@picminx\pgf@xa%
    \global\pgf@picmaxx\pgf@xb%
    \global\pgf@picminy\pgf@ya%
    \global\pgf@picmaxy\pgf@yb%
  \else%
    \ifdim\pgf@xa<\pgf@picminx\global\pgf@picminx\pgf@xa\fi%
    \ifdim\pgf@ya<\pgf@picminy\global\pgf@picminy\pgf@ya\fi%
    \ifdim\pgf@xb>\pgf@picmaxx\global\pgf@picmaxx\pgf@xb\fi%
    \ifdim\pgf@yb>\pgf@picmaxy\global\pgf@picmaxy\pgf@yb\fi%
  \fi%
}




%
%
% Handle the start of a curve
%
% 


% First, the preparation. 

\def\pgf@prep@curvedstart{
  \pgftransformreset%
  \pgfsetcurvilinearbeziercurve{\pgfpointfirstonpath}{\pgfpointsecondonpath}{\pgfpointthirdonpath}{\pgfpointfourthonpath}%
}

% Second, the shortening.

\def\pgf@do@shorten@curvedstart{%
  \expandafter\expandafter\expandafter\pgf@do@shorten@curvedstart@unpack\expandafter\pgfpointthirdonpath\pgfpointfourthonpath%
}
\def\pgf@do@shorten@curvedstart@unpack\pgfqpoint#1#2\pgfqpoint#3#4{%
  \pgf@process{\pgfcurvilineardistancetotime{\pgf@path@shortening@distance}}%
  % Compute new curve:
  \pgfpointcurveattime{\pgf@x}{\pgfpointfirstonpath}{\pgfpointsecondonpath}{\pgfpointthirdonpath}{\pgfpointfourthonpath}%
  % \pgfx/y and \pgf xa/xb already correct. Need to setup remaining vector...
  \pgf@xb#1%
  \pgf@yb#2%
  \pgf@xb\pgf@time@t\pgf@xb%
  \pgf@yb\pgf@time@t\pgf@yb%
  \advance\pgf@xb by\pgf@time@s\pgf@xc%
  \advance\pgf@yb by\pgf@time@s\pgf@yc%
  \edef\pgfpointfirstonpath{\noexpand\pgfqpoint{\the\pgf@x}{\the\pgf@y}}%
  \edef\pgfpointsecondonpath{\noexpand\pgfqpoint{\the\pgf@xa}{\the\pgf@ya}}%
  \edef\pgfprocessresultpathsuffix{%
    \noexpand\pgfsyssoftpath@movetotoken{\the\pgf@x}{\the\pgf@y}%
    \noexpand\pgfsyssoftpath@curvetosupportatoken{\the\pgf@xa}{\the\pgf@ya}%
    \noexpand\pgfsyssoftpath@curvetosupportbtoken{\the\pgf@xb}{\the\pgf@yb}%
    \noexpand\pgfsyssoftpath@curvetotoken{#3}{#4}%
  }
  \expandafter\expandafter\expandafter\def%
  \expandafter\expandafter\expandafter\pgfprocessresultpathsuffix%
  \expandafter\expandafter\expandafter{\expandafter\pgfprocessresultpathsuffix\pgfsubpathend}%
}


% Third, the drawing

\def\pgf@do@draw@curvedstart{%
  \pgf@do@draw@curved\pgf@start@tip@sequence%
}





\endinput

% Copyright 2013 by Till Tantau
%
% This file may be distributed and/or modified
%
% 1. under the LaTeX Project Public License and/or
% 2. under the GNU Public License.
%
% See the file doc/generic/pgf/licenses/LICENSE for more details.

\ProvidesFileRCS[v\pgfversion] $Header: /cvsroot/pgf/pgf/generic/pgf/modules/pgfmodulebending.code.tex,v 1.2 2013/08/29 22:51:58 tantau Exp $

%
% This file defines commands for "bending coordinate systems". This is
% needed, for instance, for bend arrow heads.
%
%
% By "bending coordinate system" I mean a curvilinear coordinate
% system in which the x-axis "goes along" a Bezier curve and the
% y-axis is always perpendicular to the (Bezier) curve at the given
% x-coordinate. Formally, given a pair (x,y), let B(x) be the point on
% the Bezier curve B at distance x from the start of the curve. Let
% T(x) be the tangent of B at B(x) and let P(x) be the (normalized)
% vector perpendicular to T(x). Then (x,y) would be mapped to B(x) +
% y*P(x). As an example, if B is a circle, then the corresponding
% bending coordinate system is (essentially, except for an offset in
% the y value) the polar coordinate system.
%
% The length along Bezier curves are approximated using a lookup table
% of four time/length points. For this, an approximate time for length
% 1pt is computed first, and then the length for twice this time, four
% times this time, and eight times this time are approximated. A
% distance-to-time conversion is then done by a linear interpolation
% of distance-to-time for these four points. Note that all of these
% computations are not particularly precise, but a compromise trading
% speed against precision. Also note that the results will only be
% best near the start of the curve and may be far off near the end if
% that end is degenerate (second control point very near to end
% point). 
%
% You use the bending cs by first telling pgf which Bezier curve you
% are interested in (using \pgfsetbendingline or \pgfsetbendingcurve)
% and then using \pgfpointbending. The results will be best if small
% values of x are used because of the abovementioned trick of
% substituing distance by time. 
%
% When a new bending cs is installed, some, possibly expensive,
% precomputations are done. Subsequent calls to \pgfpointbending
% should then be relatively quick.



% Returns a point in the current bending coordinate system.
% 
% #1 = distance along the curve
% #2 = distance perpendicular to the curve
% 
% Description:
% 
% Returns the computed position in \pgf@x and \pgf@y.

\def\pgfpointbending#1#2{%
  \pgf@bending@point{#1}{#2}%
}

\def\pgf@bending@point{\pgferror{No bending coordinate system set}}



% Sets B to the straight line from #1 to #2. 
%
% #1 = start point
% #2 = end point
%
% Example:
%
% \pgfsetbendingline{\pgfpoint{1cm}{1cm}}{\pgfpoint{3cm}{2cm}}
% \pgfpointbending{1pt}{0pt} % this will be 1pt along the line from
%                            % (1,1) to (3,2)

\def\pgfsetbendingline#1#2{%
  \pgf@process{#1}%
  \edef\pgf@bending@line@start{\noexpand\pgfqpoint{\the\pgf@x}{\the\pgf@y}}%
  \pgf@process{\pgfpointnormalised{\pgfpointdiff{}{#2}}}%
  \edef\pgf@bending@line@normal{\noexpand\pgfqpoint{\the\pgf@x}{\the\pgf@y}}%
  \pgf@xa-\pgf@x%
  \pgf@y=\pgf@x%
  \pgf@x=\pgf@xa%
  \edef\pgf@bending@line@orth{\noexpand\pgfqpoint{\the\pgf@x}{\the\pgf@y}}%
  \let\pgf@bending@point\pgf@bending@line@point%
}

\def\pgf@bending@line@point#1#2{%
  \pgfmathsetmacro\pgf@bending@xfactor{#1}%
  \pgfmathsetmacro\pgf@bending@yfactor{#2}%
  \pgf@process{\pgf@bending@line@normal}%
  \pgf@xa\pgf@bending@xfactor\pgf@x%
  \pgf@ya\pgf@bending@xfactor\pgf@y%
  \pgf@process{\pgf@bending@line@orth}%
  \advance\pgf@xa by\pgf@bending@yfactor\pgf@x%
  \advance\pgf@ya by\pgf@bending@yfactor\pgf@y%
  \pgf@process{\pgf@bending@line@start}%
  \advance\pgf@x by\pgf@xa%
  \advance\pgf@y by\pgf@ya%
}



\newif\ifpgf@bendinging@crawling@start
\newdimen\pgf@bending@time@a
\newdimen\pgf@bending@length@a
\newdimen\pgf@bending@length@b
\newdimen\pgf@bending@length@c
\newdimen\pgf@bending@length@d

% Sets B to the curve line from #1 to #4 via the control points #2 and
% #3. 
%
% #1 = start point
% #2 = first control point
% #3 = second control point
% #4 = end point
%
% Example:
%
% \pgfsetbendingline
% {\pgfpoint{0mm}{10mm}}
% {\pgfpoint{5.5mm}{10mm}}
% {\pgfpoint{10mm}{5.5mm}}
% {\pgfpoint{10mm}{0mm}} % nearly a quarter circle
% \pgfpointbending{5mm}{5mm} % should be 5mm along the circle, put at
%                            % distance 15mm from the origin (5mm fromt he circle line).

\def\pgfsetbendingcurve#1#2#3#4{%
  \pgf@process{#1}%
  \edef\pgf@bending@line@a{\noexpand\pgfqpoint{\the\pgf@x}{\the\pgf@y}}%
  \pgf@xa=-\pgf@x%
  \pgf@ya=-\pgf@y%
  \pgf@process{#2}%
  \edef\pgf@bending@line@b{\noexpand\pgfqpoint{\the\pgf@x}{\the\pgf@y}}%
  \pgf@xb=-\pgf@x%
  \pgf@yb=-\pgf@y%
  \advance\pgf@x by\pgf@xa%
  \advance\pgf@y by\pgf@ya%
  \pgfmathsetmacro\pgf@bending@lenab{veclen(\the\pgf@x,\the\pgf@y)}%
  \pgf@process{#3}%
  \edef\pgf@bending@line@c{\noexpand\pgfqpoint{\the\pgf@x}{\the\pgf@y}}%
  \pgf@xc=-\pgf@x%
  \pgf@yc=-\pgf@y%
  \advance\pgf@x by\pgf@xb%
  \advance\pgf@y by\pgf@yb%
  \pgfmathsetmacro\pgf@bending@lenbc{veclen(\the\pgf@x,\the\pgf@y)}%
  \pgf@process{#4}%
  \edef\pgf@bending@line@d{\noexpand\pgfqpoint{\the\pgf@x}{\the\pgf@y}}%
  \advance\pgf@x by\pgf@xc%
  \advance\pgf@y by\pgf@yc%
  \pgfmathsetmacro\pgf@bending@lencd{veclen(\the\pgf@x,\the\pgf@y)}%
  %
  \pgf@x=\pgf@bending@lenab pt%
  \advance\pgf@x by\pgf@bending@lenbc pt%
  \advance\pgf@x by\pgf@bending@lencd pt%
  \pgfmathreciprocal@{\pgf@x}%
  \pgf@bending@time@a\pgfmathresult pt%
  \pgf@process{\pgfpointcurveattime{\pgf@bending@time@a}{\pgf@bending@line@a}{\pgf@bending@line@b}{\pgf@bending@line@c}{\pgf@bending@line@d}}%
  \pgf@xb=-\pgf@x%
  \pgf@yb=-\pgf@y%
  \advance\pgf@x by\pgf@xa%
  \advance\pgf@y by\pgf@ya%
  \pgfmathsetlength\pgf@bending@length@a{veclen(\the\pgf@x,\the\pgf@y)}%
  \ifdim\pgf@bending@length@a>1pt\relax%
    % Ok, too large, let us make this smaller
    \pgfmathsetlength\pgf@bending@time@a{\pgf@bending@time@a/\pgf@bending@length@a}%
    \pgf@process{\pgfpointcurveattime{\pgf@bending@time@a}{\pgf@bending@line@a}{\pgf@bending@line@b}{\pgf@bending@line@c}{\pgf@bending@line@d}}
    \pgf@xb=-\pgf@x%
    \pgf@yb=-\pgf@y%
    \advance\pgf@x by\pgf@xa%
    \advance\pgf@y by\pgf@ya%
    \pgfmathsetlength\pgf@bending@length@a{veclen(\the\pgf@x,\the\pgf@y)}%
  \fi%
  % Compute three positions:
  \pgf@process{\pgfpointcurveattime{2\pgf@bending@time@a}{\pgf@bending@line@a}{\pgf@bending@line@b}{\pgf@bending@line@c}{\pgf@bending@line@d}}
  \pgf@xa=-\pgf@x%
  \pgf@ya=-\pgf@y%
  \advance\pgf@x by\pgf@xb%
  \advance\pgf@y by\pgf@yb%
  \pgfmathsetlength\pgf@bending@length@b{veclen(\the\pgf@x,\the\pgf@y)+\pgf@bending@length@a}%
  \pgf@process{\pgfpointcurveattime{4\pgf@bending@time@a}{\pgf@bending@line@a}{\pgf@bending@line@b}{\pgf@bending@line@c}{\pgf@bending@line@d}}
  \pgf@xb=-\pgf@x%
  \pgf@yb=-\pgf@y%
  \advance\pgf@x by\pgf@xa%
  \advance\pgf@y by\pgf@ya%
  \pgfmathsetlength\pgf@bending@length@c{veclen(\the\pgf@x,\the\pgf@y)+\pgf@bending@length@b}%
  \pgf@process{\pgfpointcurveattime{8\pgf@bending@time@a}{\pgf@bending@line@a}{\pgf@bending@line@b}{\pgf@bending@line@c}{\pgf@bending@line@d}}
  \advance\pgf@x by\pgf@xb%
  \advance\pgf@y by\pgf@yb%
  \pgfmathsetlength\pgf@bending@length@d{veclen(\the\pgf@x,\the\pgf@y)+\pgf@bending@length@c}%
  \let\pgf@bending@comp@a\pgf@bending@comp@a@initial%
  \let\pgf@bending@comp@b\pgf@bending@comp@b@initial%
  \let\pgf@bending@comp@c\pgf@bending@comp@c@initial%
  \let\pgf@bending@comp@d\pgf@bending@comp@d@initial%
  \let\pgf@bending@comp@e\pgf@bending@comp@e@initial%
  \let\pgf@bending@point\pgf@bending@curve@point%
}

\def\pgf@bending@comp@a@initial{%
  \pgfmathsetmacro{\pgf@bending@quot@a}{\pgf@bending@time@a/\pgf@bending@length@a}%
  \let\pgf@bending@comp@a\pgf@bending@comp@a@cont%
  \pgf@bending@comp@a@cont%
}
\def\pgf@bending@comp@a@cont{%
  \pgf@x\pgf@bending@quot@a\pgf@x%
}

\def\pgf@bending@comp@b@initial{%
  \pgf@y=\pgf@bending@length@b%
  \advance\pgf@y by-\pgf@bending@length@a%
  \pgfmathreciprocal@{\the\pgf@y}%
  \let\pgf@temp\pgfmathresult%
  \pgfmathsetmacro\pgf@bending@quot@b{\pgfmathresult\pgf@bending@time@a}%
  \pgfmathsetmacro\pgf@bending@correct@b{-\pgf@bending@quot@b\pgf@bending@length@a+\pgf@bending@time@a}%
  \let\pgf@bending@comp@b\pgf@bending@comp@b@cont%
  \pgf@bending@comp@b@cont%
}
\def\pgf@bending@comp@b@cont{%
  \pgf@x\pgf@bending@quot@b\pgf@x%
  \advance\pgf@x by\pgf@bending@correct@b pt%
}

\def\pgf@bending@comp@c@initial{%
  \pgf@y=\pgf@bending@length@c%
  \advance\pgf@y by-\pgf@bending@length@b%
  \pgfmathreciprocal@{\the\pgf@y}%
  \let\pgf@temp\pgfmathresult%
  \pgfmathsetmacro\pgf@bending@quot@c{2*\pgfmathresult\pgf@bending@time@a}%
  \pgfmathsetmacro\pgf@bending@correct@c{-\pgf@bending@quot@c\pgf@bending@length@b+2*\pgf@bending@time@a}%
  \let\pgf@bending@comp@c\pgf@bending@comp@c@cont%
  \pgf@bending@comp@c@cont%
}
\def\pgf@bending@comp@c@cont{%
  \pgf@x\pgf@bending@quot@c\pgf@x%
  \advance\pgf@x by\pgf@bending@correct@c pt%
}

\def\pgf@bending@comp@d@initial{%
  \pgf@y=\pgf@bending@length@d%
  \advance\pgf@y by-\pgf@bending@length@c%
  \pgfmathreciprocal@{\the\pgf@y}%
  \let\pgf@temp\pgfmathresult%
  \pgfmathsetmacro\pgf@bending@quot@d{4*\pgfmathresult\pgf@bending@time@a}%
  \pgfmathsetmacro\pgf@bending@correct@d{-\pgf@bending@quot@d\pgf@bending@length@c+4*\pgf@bending@time@a}%
  \let\pgf@bending@comp@d\pgf@bending@comp@d@cont%
  \pgf@bending@comp@d@cont%
}
\def\pgf@bending@comp@d@cont{%
  \pgf@x\pgf@bending@quot@d\pgf@x%
  \advance\pgf@x by\pgf@bending@correct@d pt%
}

\def\pgf@bending@comp@e@initial{%
  \pgfmathsetmacro{\pgf@bending@quot@e}{8*\pgf@bending@time@a/\pgf@bending@length@d}%
  \let\pgf@bending@comp@e\pgf@bending@comp@e@cont%
  \pgf@bending@comp@e@cont%
}
\def\pgf@bending@comp@e@cont{%
  \pgf@x\pgf@bending@quot@e\pgf@x%
}

\def\pgf@bending@curve@point#1#2{%
  \pgfbendingdistancetotime{#1}%
  \pgfpointcurveattime{\pgf@x}{\pgf@bending@line@a}{\pgf@bending@line@b}{\pgf@bending@line@c}{\pgf@bending@line@d}
  \pgf@xc=\pgf@x% save
  \pgf@yc=\pgf@y% save
  % compute normal:
  \advance\pgf@xb by-\pgf@xa%
  \advance\pgf@yb by-\pgf@ya%
  \pgf@process{\pgfpointnormalised{\pgf@x=\pgf@yb\pgf@y=-\pgf@xb}}
  \pgfmathsetmacro\pgf@bending@yfactor{#2}%
  \pgf@x=\pgf@bending@yfactor\pgf@x%
  \pgf@y=\pgf@bending@yfactor\pgf@y%
  \advance\pgf@x by\pgf@xc%
  \advance\pgf@y by\pgf@yc%
}


% Convert a distance into a time
% 
% #1 = a distance
% 
% Description:
%
% After having called \pgfsetbendingcurve, you can use this macro to
% convert a distance into a time along the curve set in that
% command. The result will be stored in \pgf@x.

\def\pgfbendingdistancetotime#1{%
  \pgfmathsetlength{\pgf@x}{#1}%
  \ifdim\pgf@x<\pgf@bending@length@c\relax%
    \ifdim\pgf@x<\pgf@bending@length@a\relax%
      \pgf@bending@comp@a%
    \else\ifdim\pgf@x<\pgf@bending@length@b\relax%
      \pgf@bending@comp@b%
    \else%
      \pgf@bending@comp@c%
    \fi\fi%
  \else\ifdim\pgf@x<\pgf@bending@length@d\relax%
    \pgf@bending@comp@d%
  \else%
    \pgf@bending@comp@e%
  \fi\fi%
}



\endinput

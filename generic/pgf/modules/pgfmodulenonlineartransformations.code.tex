% Copyright 2013 by Till Tantau
%
% This file may be distributed and/or modified
%
% 1. under the LaTeX Project Public License and/or
% 2. under the GNU Public License.
%
% See the file doc/generic/pgf/licenses/LICENSE for more details.

\ProvidesFileRCS[v\pgfversion] $Header: /cvsroot/pgf/pgf/generic/pgf/modules/pgfmodulebending.code.tex,v 1.3 2013/08/30 11:44:10 tantau Exp $


%
% This file defines commands for nonlinear coordinate systems. This is
% needed, for instance, for bend arrow heads.
%
%
% Non-linear coordinate systems transform coordinates, in addition to
% the normal linear transformations, in a, well, non-linear way. An
% example are polar coordinates.
%
% Like the linear transformations, non-linear transformations are
% applied on a low level and they are integrated into the whole path
% construction mechanism. So, while \pgfpointpolar also "does
% something nonlinear", it just provides a way of computing a
% coordinate. In contrast, installing a polar transformation will
% transform everything being drawn using the \pgfpath... commands an
% will even turn a straight line into a circle. 
%
% Since computing non-linear transformations is expensive, they are
% only applied "if really necessary".




% Adds a non-linear transformation to the current list of
% transformations. 
%
% #1 = code
%
% Description:
%
% When the code #1 is called, \pgf@x and \pgf@y will be to some
% values. The result should be modified values of \pgf@x and
% \pgf@y. The macro should not do any global assignments except to
% \pgf@x and \pgf@y, but may do arbitrary local assignments.
% 
% The effect of installing a new transformation is the following: As
% long as there is at least one nonlinear transformation installed,
% whenever the path construction commands add a curve or line to the
% softpath, the line or curve will be "bend" by applying all of the
% nonlinear transformation installed. In particular, straight lines
% will be replaced by curves.
% 
% Note that the normal linear transformation are always applied before
% any nonlinear transformations are applied.

\def\pgftransformnonlinear#1{%
  \expandafter\def\expandafter\pgf@nlt@list\expandafter{\pgf@nlt@list#1}%
  \let\pgf@nlt@moveto\pgf@nlt@moveto@nlt
  \let\pgf@nlt@lineto\pgf@nlt@lineto@nlt
  \let\pgf@nlt@curveto\pgf@nlt@curveto@nlt
  \let\pgf@nlt@closepath\pgf@nlt@closepath@nlt
}

\def\pgf@nlt@infos{{0pt}{0pt}{0pt}{0pt}{0pt}{0pt}{0pt}{0pt}}
% last coordinate, last coordinate transformed, last moveto, last moveto transformed

\def\pgf@nlt@moveto@nlt#1#2{%
  {%
    \edef\pgf@restore{\pgf@x\the\pgf@x\pgf@y\the\pgf@y}%
    \pgfutil@tempdima#1%
    \pgfutil@tempdimb#2%
    \pgf@x\pgfutil@tempdima%
    \pgf@y\pgfutil@tempdimb%
    \pgf@process{\pgf@nlt@list}%
    \pgf@protocolsizes{\pgf@x}{\pgf@y}%
    \pgfsyssoftpath@moveto{\the\pgf@x}{\the\pgf@y}%
    \expandafter\pgf@nlt@redef@info@moveto\pgf@nlt@infos%
  \expandafter}%
  \pgf@restore
}
\def\pgf@nlt@redef@info@moveto#1#2#3#4#5#6#7#8{%
  \xdef\pgf@nlt@infos{%
    {\the\pgfutil@tempdima}{\the\pgfutil@tempdimb}%last coordinate
    {\the\pgf@x}{\the\pgf@y}%last coordinate transformed
    {\the\pgfutil@tempdima}{\the\pgfutil@tempdimb}% last moveto
    {\the\pgf@x}{\the\pgf@y}%last moveto transformed
  }%
}

\def\pgf@nlt@lineto@nlt#1#2{%
  {%
    \edef\pgf@restore{\pgf@x\the\pgf@x\pgf@y\the\pgf@y}%
    \edef\pgf@temp{\pgf@xc\the#1\pgf@yc\the#2}%
    \pgf@temp%
    \expandafter\pgf@nlt@get@last@coord\pgf@nlt@infos%
    % Test, whether the points are quite near:
    \pgf@xa\pgf@x\advance\pgf@xa by-\pgf@xc%
    \pgf@ya\pgf@y\advance\pgf@ya by-\pgf@yc%
    \pgfutil@tempswafalse%
    \ifdim\pgf@xa<0.1pt\relax\ifdim\pgf@xa>-0.1pt\relax\ifdim\pgf@ya<0.1pt\relax\ifdim\pgf@ya>-0.1pt\relax\pgfutil@tempswatrue\fi\fi\fi\fi%
    \ifpgfutil@tempswa%
      \pgf@x\pgf@xc\pgf@y\pgf@yc%
      \pgf@process{\pgf@nlt@list}%
      \pgf@protocolsizes{\pgf@x}{\pgf@y}%
      \pgfsyssoftpath@lineto{\the\pgf@x}{\the\pgf@y}%
      \pgfutil@tempdima\pgf@x\pgfutil@tempdimb\pgf@y%
      \expandafter\pgf@nlt@redef@info\pgf@nlt@infos%
    \else%
      % Compute support points
      \pgf@xb=.33333\pgf@x%
      \advance\pgf@xb by.66666\pgf@xc%
      \pgf@yb=.33333\pgf@y%
      \advance\pgf@yb by.66666\pgf@yc%
      \pgf@xa=.33333\pgf@xc%
      \advance\pgf@xa by.66666\pgf@x%
      \pgf@ya=.33333\pgf@yc%
      \advance\pgf@ya by.66666\pgf@y%
      \pgf@nlt@inner@curve%
    \fi%
  \expandafter}%
  \pgf@restore
}

\def\pgf@nlt@get@last@coord#1#2#3#4#5#6#7#8{%
  \pgf@x#1\pgf@y#2%
  \pgfutil@tempdima#3\pgfutil@tempdimb#4%
}

\def\pgf@nlt@curveto@nlt#1#2#3#4#5#6{%
  {%
    \edef\pgf@restore{\pgf@x\the\pgf@x\pgf@y\the\pgf@y}%
    \edef\pgf@temp{\pgf@xa\the#1\pgf@ya\the#2\pgf@xb\the#3\pgf@yb\the#4\pgf@xc\the#5\pgf@yc\the#6}%
    \pgf@temp%
    \expandafter\pgf@nlt@get@last@coord\pgf@nlt@infos%
    \pgf@nlt@inner@curve%
  \expandafter}%
  \pgf@restore
}


\newdimen\pgf@nlt@x
\newdimen\pgf@nlt@y

% Compute a curve from \pgf@x/pgf@y to \pgf@xc/\pgf@yc via the
% controls \pgf@xa/\pgf@ya and \pgf@xb/\pgf@yb.
\def\pgf@nlt@inner@curve{%
  % Replace supports by relative supports
  \advance\pgf@xa by-\pgf@x%
  \advance\pgf@ya by-\pgf@y%
  \advance\pgf@xb by-\pgf@xc%
  \advance\pgf@yb by-\pgf@yc%
  % tempdim a/b = transformed x/y
  \advance\pgf@x by0.25pt%
  \pgf@process{\pgf@nlt@list}%
  \pgf@nlt@x\pgf@x%
  \pgf@nlt@y\pgf@y%
  \advance\pgf@nlt@x by-\pgfutil@tempdima% nlt@x/y = 1/10 of tangent vector in transformed cs
  \advance\pgf@nlt@y by-\pgfutil@tempdimb%
  \pgf@nlt@x=4\pgf@nlt@x% nlt = tangent
  \pgf@nlt@y=4\pgf@nlt@y%
  \pgf@process{\pgfpointnormalised{\pgfqpoint{-\pgf@nlt@y}{\pgf@nlt@x}}}% \pgf@x/y = normal vector in transformed cs
  % Multiply normal vector by \pgf@ya and add 4*tangent*\pgf@xa and add start
  \pgf@x\pgf@sys@tonumber\pgf@ya\pgf@x%
  \pgf@y\pgf@sys@tonumber\pgf@ya\pgf@y%
  \advance\pgf@x by\pgf@sys@tonumber\pgf@xa\pgf@nlt@x%
  \advance\pgf@y by\pgf@sys@tonumber\pgf@xa\pgf@nlt@y%
  \advance\pgf@x by\pgfutil@tempdima%
  \advance\pgf@y by\pgfutil@tempdimb%
  \pgf@xa\pgf@x% Save!
  \pgf@ya\pgf@y%
  % Now same thing for end:
  \pgf@x\pgf@xc\pgf@y\pgf@yc%
  \pgf@process{\pgf@nlt@list}%
  \pgfutil@tempdima\pgf@x%
  \pgfutil@tempdimb\pgf@y% tempdim = transformed end
  \pgf@x\pgf@xc%
  \pgf@y\pgf@yc%
  \advance\pgf@x by0.25pt%
  \pgf@process{\pgf@nlt@list}%
  \pgf@nlt@x\pgf@x%
  \pgf@nlt@y\pgf@y%
  \advance\pgf@nlt@x by-\pgfutil@tempdima% nlt@x/y = 1/4 of tangent vector in transformed cs
  \advance\pgf@nlt@y by-\pgfutil@tempdimb%
  \pgf@nlt@x=4\pgf@nlt@x% nlt = tangent
  \pgf@nlt@y=4\pgf@nlt@y%
  \pgf@process{\pgfpointnormalised{\pgfqpoint{-\pgf@nlt@y}{\pgf@nlt@x}}}% \pgf@x/y = normal vector in transformed cs
  % Multiply normal vector by \pgf@ya and add 10*tangent*\pgf@xa
  \pgf@x\pgf@sys@tonumber\pgf@yb\pgf@x%
  \pgf@y\pgf@sys@tonumber\pgf@yb\pgf@y%
  \advance\pgf@x by\pgf@sys@tonumber\pgf@xb\pgf@nlt@x%
  \advance\pgf@y by\pgf@sys@tonumber\pgf@xb\pgf@nlt@y%
  \advance\pgf@x by\pgfutil@tempdima%
  \advance\pgf@y by\pgfutil@tempdimb%
  \pgf@xb\pgf@x%
  \pgf@yb\pgf@y%
  \pgf@protocolsizes{\pgf@xa}{\pgf@ya}%
  \pgf@protocolsizes{\pgf@xb}{\pgf@yb}%
  \pgf@protocolsizes{\pgfutil@tempdima}{\pgfutil@tempdimb}%
  \pgfsyssoftpath@curveto{\the\pgf@xa}{\the\pgf@ya}{\the\pgf@xb}{\the\pgf@yb}{\the\pgfutil@tempdima}{\the\pgfutil@tempdimb}
  \expandafter\pgf@nlt@redef@info\pgf@nlt@infos%
}

\def\pgf@nlt@redef@info#1#2#3#4#5#6#7#8{%
  \xdef\pgf@nlt@infos{%
    {\the\pgf@xc}{\the\pgf@yc}%last coordinate
    {\the\pgfutil@tempdima}{\the\pgfutil@tempdimb}%last coordinate transformed
    {#5}{#6}% last moveto
    {#7}{#8}%last moveto transformed
  }
}


\def\pgf@nlt@closepath@nlt{%
  {%
    \expandafter\pgf@nlt@get@closepath\pgf@nlt@infos%
    \pgfutil@tempswatrue%
    \ifdim\pgf@xc<0.01pt\relax\ifdim\pgf@xc>-0.01pt\relax\ifdim\pgf@yc<0.01pt\relax\ifdim\pgf@yc>-0.01pt\relax\pgfutil@tempswafalse\fi\fi\fi\fi%
    \ifpgfutil@tempswa%
      \pgf@nlt@lineto@nlt{\pgf@x}{\pgf@y}%
    \fi%
    \pgfsyssoftpath@closepath%
  }%
}

\def\pgf@nlt@get@closepath#1#2#3#4#5#6#7#8{%
  \pgf@x#5\pgf@y#6%
  \pgf@xc#1\pgf@yc#2%
  \advance\pgf@xc by-\pgf@x%
  \advance\pgf@yc by-\pgf@y%
}


\endinput

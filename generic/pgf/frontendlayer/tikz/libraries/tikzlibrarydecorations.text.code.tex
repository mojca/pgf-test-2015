% Copyright 2013 by Mark Wibrow
%
% This file may be distributed and/or modified
%
% 1. under the LaTeX Project Public License and/or
% 2. under the GNU Public License.
%
% See the file doc/generic/pgf/licenses/LICENSE for more details.

\usetikzlibrary{decorations}
\usepgflibrary{decorations.text}


\newif\iftikz@lib@dec@te@kerning
\newif\iftikz@lib@dec@te@pathfromtext
\newif\iftikz@lib@dec@te@segmentfromwidth


\def\tikz@lib@dec@te@addto@macro#1#2{%
	\expandafter\def\expandafter#1\expandafter{#1#2}%
}

\let\tikz@lib@dec@te@textoptions=\pgfutil@empty
\let\tikz@lib@dec@te@widthoptions=\pgfutil@empty
\let\tikz@lib@dec@te@charactercountvar=\pgfutil@empty
\let\tikz@lib@dec@te@charactertotalvar=\pgfutil@empty
\let\tikz@lib@dec@te@lettercountvar=\pgfutil@empty
\let\tikz@lib@dec@te@wordcountvar=\pgfutil@empty
\let\tikz@lib@dec@te@transformdecorationtext=\pgfutil@empty%

\def\tikz@lib@dec@te@spacetext{space}

\pgfkeys{/pgf/decoration/text effects/.cd,
	  .unknown/.code={\let\tikz@lib@dec@te@key@tmp=\pgfkeyscurrentname%
	   \pgfkeysalso{/tikz/\tikz@lib@dec@te@key@tmp={#1}}},
	  path from text/.is if=tikz@lib@dec@te@pathfromtext,
	  path from text angle/.store in=\tikz@lib@dec@te@pathfromtextangle,
	  path from text angle=0,
		characters/.style={/pgf/decoration/text effects/every character/.style={#1}},
		characters/.append/.code={\pgfkeysalso{/pgf/decoration/text effects/every character/.append style={#1}}},
		character widths/.code={\pgfkeysalso{/pgf/decoration/text effects/every character width/.style={#1}}},
		character widths/.append/.code={\pgfkeysalso{/pgf/decoration/text effects/every character width/.append style={#1}}},
		character variable/.store in=\tikz@lib@dec@te@charactercountvar,
		character variable=,
		character total variable/.store in=\tikz@lib@dec@te@charactertotalvar,
		character total variable=,
		letter variable/.store in=\tikz@lib@dec@te@lettercountvar,
		letter variable=,
		word variable/.store in=\tikz@lib@dec@te@wordcountvar,
		word variable=,
		%
		style characters/.code args={#1 with #2}{\tikz@lib@dec@te@parse@stylecharacters{#1}{#2}},
		text along path/.style={
			inner xsep=0pt,
			anchor=base,
			transform shape
		},
		segment from width/.is if=tikz@lib@dec@te@segmentfromwidth,
		segment from width=true,
		word separator/.code={%
			\def\tikz@lib@dec@te@tmp{#1}%
			\ifx\tikz@lib@dec@te@tmp\tikz@lib@dec@te@spacetext%
				\def\tikz@lib@dec@te@wordsep{ }%
			\else%
				\def\tikz@lib@dec@te@wordsep{#1}
			\fi},
		word separator=space,
		character command/.store in=\tikz@lib@dec@te@charactercommand,
		character command=\tikz@lib@dec@te@charactercommandignore,
		every character width/.style={/pgf/decoration/text effects/every character/.try},
		every character/.style={},
		reverse text/.code={\tikz@lib@dec@te@addto@macro\tikz@lib@dec@te@transformdecorationtext%
		 	{\tikz@lib@dec@te@reversetext}},
		group letters into words/.code={\tikz@lib@dec@te@addto@macro\tikz@lib@dec@te@transformdecorationtext%
				 	{\tikz@lib@dec@te@letterstowords}},
	  replace characters/.code args={#1 with #2}{\tikz@lib@dec@te@parse@replacecharacters{#1}{#2}}
}


\def\tikz@lib@dec@te@charactercommandignore#1{#1}

\tikzset{%
	text effects/.code={\pgfkeysalso{/pgf/decoration/text effects/.cd,#1}},
}



\newtoks\tikz@lib@dec@te@toks

\def\tikz@lib@dec@te@advancecountmacro#1#2{%
	\pgfutil@tempcnta=#1\relax%
	\advance\pgfutil@tempcnta by#2\relax%
	\edef#1{\the\pgfutil@tempcnta}%
}

\def\tikz@lib@dec@te@advancedimenmacro#1#2{%
	\pgf@x=#1\relax%
	\advance\pgf@x by#2\relax%
	\edef#1{\the\pgf@x}%
}

% Scan the first character from \tikz@lib@dec@te@rest to \tikz@lib@dec@te@token
% and set \tikz@lib@dec@te@rest to the remaining characters.
%
% Sounds easy, right? But we must take into account
% the automatic stripping of braces { } and space 
% characters. 
%
% Without changing category codes, all consecutive spaces
% will be subsumed into one space. This is unavoidable.
%
\def\tikz@lib@dec@te@scanchar{%
	\let\tikz@lib@dec@te@token@last=\tikz@lib@dec@te@token%
	\ifx\tikz@lib@dec@te@rest\pgfutil@empty%
		\let\tikz@lib@dec@te@token=\pgfutil@empty%
	\else%
		\let\tikz@lib@dec@te@@rest=\tikz@lib@dec@te@rest%
		\let\tikz@lib@dec@te@rest=\pgfutil@empty%
		\expandafter\tikz@lib@dec@te@@scanchar\tikz@lib@dec@te@@rest\tikz@lib@dec@te@%
	\fi%
}

\def\tikz@lib@dec@te@@scanchar{%
	\tikz@lib@dec@te@toks={}%
	\futurelet\tikz@lib@dec@te@token\tikz@lib@dec@te@@@scanchar%
}

\def\tikz@lib@dec@te@@@scanchar{%
	\ifx\tikz@lib@dec@te@token\pgfutil@sptoken%
		\let\tikz@lib@dec@te@next=\tikz@lib@dec@te@@@scanchar@space%
	\else%
		\ifx\tikz@lib@dec@te@token\bgroup%
			\let\tikz@lib@dec@te@next\tikz@lib@dec@te@@@scanchar@bgroup%
		\else%
			\let\tikz@lib@dec@te@next\tikz@lib@dec@te@@@scanchar@char%
		\fi%
	\fi%
	\tikz@lib@dec@te@next%
}

% Might want this to be a hard space.
\def\tikz@lib@dec@te@space@char{ }

% The character is a space.
\def\tikz@lib@dec@te@@@scanchar@space{%
	\def\tikz@lib@dec@te@token{ }%
	\afterassignment\tikz@lib@dec@te@@@scanchar@@space\futurelet\tikz@lib@dec@te@@token}

\def\tikz@lib@dec@te@@@scanchar@@space{%
	\ifx\tikz@lib@dec@te@@token\bgroup%
		% If there is a \bgroup following the space.
		% So this must be saved as a group.
		\let\tikz@lib@dec@te@next=\tikz@lib@dec@te@@@scanchar@@space@bgroup%
	\else%
		\let\tikz@lib@dec@te@next=\tikz@lib@dec@te@@@scanchar@@space@char%
	\fi%
	\tikz@lib@dec@te@next%
}

\def\tikz@lib@dec@te@@@scanchar@@space@bgroup#1{%
	\expandafter\tikz@lib@dec@te@toks\expandafter=\expandafter{\the\tikz@lib@dec@te@toks{#1}}%
	\tikz@lib@dec@te@@@scanchar@toend%
}

\def\tikz@lib@dec@te@@@scanchar@@space@char#1{%
	\tikz@lib@dec@te@@@scanchar@@@toend#1%
}

% The character is a \bgroup.
% This one is easy. Normal TeX parsing
% gets rid of the braces.
\def\tikz@lib@dec@te@@@scanchar@bgroup#1{%
	\def\tikz@lib@dec@te@token{#1}%
	\tikz@lib@dec@te@@@scanchar@toend}

% Get `macro' as category code 12 text
{\def\|#1{\catcode`#1=12}\let\:=\expandafter\let\[=\csname\let\]=\endcsname
\|\m\|\a\|\c\|\r\|\o\:\gdef\[tikz@lib@dec@te@macro@text\]{macro}%
\|\t\|\h\:\gdef\[tikz@lib@dec@te@math@text\]{math}}

% The character is not a space or a \bgroup.
% But it might be an expandable macro.
% This is `sort of' easy, but must remember to take
% character out of stream as the original test used
% \futurelet.
\def\tikz@lib@dec@te@@@scanchar@char{%
	\def\tikz@lib@dec@te@marshal{\expandafter\pgfutil@in@\expandafter{\tikz@lib@dec@te@macro@text}}%
	\expandafter\tikz@lib@dec@te@marshal\expandafter{\meaning\tikz@lib@dec@te@token}%
	\ifpgfutil@in@%
		\let\tikz@lib@dec@te@next=\tikz@lib@dec@te@@@scanchar@@char%
	\else%
		\let\tikz@lib@dec@te@next=\tikz@lib@dec@te@@@scanchar@toend%
	\fi%
	\tikz@lib@dec@@@scanchar@extract}%

\def\tikz@lib@dec@@@scanchar@extract#1{%
	\def\tikz@lib@dec@te@token{#1}%
	\tikz@lib@dec@te@next%
}
\def\tikz@lib@dec@te@@@scanchar@@char{%
	% Hope it is expanable...
	\expandafter\tikz@lib@dec@te@@scanchar\tikz@lib@dec@te@token}%

% When collecting the rest of the characters we must
% guard against {...} exisiting as the last character as
% TeX will happily strip the braces off.
\def\tikz@lib@dec@te@@@scanchar@toend{%
	\futurelet\tikz@lib@dec@te@@token\tikz@lib@dec@te@@@scanchar@@toend%
}

\def\tikz@lib@dec@te@@@scanchar@@toend{%
	\ifx\tikz@lib@dec@te@@token\bgroup%
		\def\tikz@lib@dec@te@marshal{\expandafter\def\expandafter\tikz@lib@dec@te@rest\expandafter}%
		\expandafter\tikz@lib@dec@te@marshal\expandafter{\expandafter\tikz@lib@dec@te@rest\the\tikz@lib@dec@te@toks}%
		\let\tikz@lib@dec@te@next=\tikz@lib@dec@te@@@scanchar@@toend@bgroup%
	\else%
		\let\tikz@lib@dec@te@next=\tikz@lib@dec@te@@@scanchar@@@toend%
	\fi%
	\tikz@lib@dec@te@next%
}

\def\tikz@lib@dec@te@@@scanchar@@toend@bgroup#1{%
	\expandafter\tikz@lib@dec@te@toks\expandafter{\the\tikz@lib@dec@te@toks{#1}}%
	\tikz@lib@dec@te@@@scanchar@toend}

\def\tikz@lib@dec@te@@@scanchar@@@toend{%
	\futurelet\tikz@lib@dec@te@nexttoken\tikz@lib@dec@te@@@scanchar@@@@toend}
\def\tikz@lib@dec@te@@@scanchar@@@@toend#1\tikz@lib@dec@te@{%
	\ifx\tikz@lib@dec@te@nexttoken\pgfutil@sptoken%
		\let\tikz@lib@dec@te@nexttoken=\tikz@lib@dec@te@space@char%
	\fi%
	\expandafter\def\expandafter\tikz@lib@dec@te@rest\expandafter{\the\tikz@lib@dec@te@toks#1}%
}



% Calculate the width of the decoration text.
\def\tikz@lib@dec@te@gettextwidth{%
	\let\tikz@lib@dec@te@rest=\pgfdecorationremainingtext%
	\let\tikz@lib@dec@te@lasttoken=\pgfutil@empty%
	\def\tikz@lib@dec@te@charactercount{0}%
	\def\tikz@lib@dec@te@wordcount{0}%
	\def\tikz@lib@dec@te@lettercount{0}%
	\def\tikz@lib@dec@te@textwidth{0pt}%
	\tikz@lib@dec@te@@gettextwidth%
}

% Actually we curently ignore math stuff.
{\catcode`\|=3\relax\gdef\tikz@lib@dec@te@mathshift@char{|}}

\def\tikz@lib@dec@te@nameadd#1#2{%
	\expandafter\expandafter\expandafter\tikz@lib@dec@te@toks\expandafter\expandafter\expandafter{\csname#1\endcsname}%
	\expandafter\edef\csname#1\endcsname{\the\tikz@lib@dec@te@toks#2}
}

\newif\iftikz@lib@dec@te@finalletter

\def\tikz@lib@dec@te@@gettextwidth{%
	\tikz@lib@dec@te@scanchar%
	\ifx\tikz@lib@dec@te@token\pgfutil@empty%
		\let\tikz@lib@dec@te@next=\relax%
	\else%
		\def\tikz@lib@dec@te@characterwidth{0pt}%
		%
		\tikz@lib@dec@te@advancecountmacro{\tikz@lib@dec@te@charactercount}{1}%
		\ifx\tikz@lib@dec@te@token\tikz@lib@dec@te@wordsep%
			\ifnum\tikz@lib@dec@te@lettercount>0\relax%
				\def\tikz@lib@dec@te@lettercount{0}%
			\fi%
			\tikz@lib@dec@te@getcharacterwidth%
			\expandafter\edef\csname tikz@lib@dec@te@character@\tikz@lib@dec@te@charactercount\endcsname{%
				\noexpand\def\noexpand\tikz@lib@dec@te@characterwidth{\tikz@lib@dec@te@characterwidth}%
				\noexpand\def\noexpand\tikz@lib@dec@te@wordcount{0}%
				\noexpand\def\noexpand\tikz@lib@dec@te@lettercount{0}%
			}%	
		\else%
			\ifx\tikz@lib@dec@te@nexttoken\tikz@lib@dec@te@wordsep%
				\tikz@lib@dec@te@finallettertrue%
			\else%
				\ifx\tikz@lib@dec@te@nexttoken\tikz@lib@dec@te@%
					\tikz@lib@dec@te@finallettertrue%
				\else%
					\tikz@lib@dec@te@finalletterfalse%
				\fi%
			\fi%
			\ifnum\tikz@lib@dec@te@lettercount=0\relax%
				\tikz@lib@dec@te@advancecountmacro\tikz@lib@dec@te@wordcount{1}%
			\fi%
			\tikz@lib@dec@te@advancecountmacro\tikz@lib@dec@te@lettercount{1}%
			\tikz@lib@dec@te@getcharacterwidth%
			\expandafter\edef\csname tikz@lib@dec@te@character@\tikz@lib@dec@te@charactercount\endcsname{%
				\noexpand\def\noexpand\tikz@lib@dec@te@characterwidth{\tikz@lib@dec@te@characterwidth}%
				\noexpand\def\noexpand\tikz@lib@dec@te@wordcount{\tikz@lib@dec@te@wordcount}%
				\noexpand\def\noexpand\tikz@lib@dec@te@lettercount{\tikz@lib@dec@te@lettercount}%
				\noexpand\csname tikz@lib@dec@te@finalletter\iftikz@lib@dec@te@finalletter true\else false\fi\noexpand\endcsname%
			}%	
		\fi%
		%
		\tikz@lib@dec@te@advancedimenmacro{\tikz@lib@dec@te@textwidth}{\tikz@lib@dec@te@characterwidth}%
		%
		\expandafter\let\csname tikz@lib@dec@te@character@\tikz@lib@dec@te@charactercount @width\endcsname=\tikz@lib@dec@te@characterwidth%
		\let\tikz@lib@dec@te@next=\tikz@lib@dec@te@@gettextwidth%
	\fi%
	\tikz@lib@dec@te@next%
}



\def\tikz@lib@dec@te@addoptions#1{%
	\expandafter\def\expandafter\tikz@lib@dec@te@options\expandafter{\tikz@lib@dec@te@options,#1}%
}

% Define 'count' variables (if required) and get current options.
% Options are not applied here, but stored in \tikz@lib@dec@te@options.
\def\tikz@lib@dec@te@getoptions#1{%
	\let\tikz@lib@dec@te@lastcharactercommand=\tikz@lib@dec@te@charactercommand%
	\def\tikz@lib@dec@te@options{/pgf/decoration/text effects/.cd}%
	\tikz@lib@dec@te@addoptions{#1}%
	%
	% Define the count variables (if required)
	\ifx\tikz@lib@dec@te@charactercountvar\pgfutil@empty%
	\else%
		\expandafter\let\tikz@lib@dec@te@charactercountvar=\tikz@lib@dec@te@charactercount%
	\fi%
	%
	\ifx\tikz@lib@dec@te@charactertotalvar\pgfutil@empty%
	\else%
		\expandafter\let\tikz@lib@dec@te@charactertotalvar=\tikz@lib@dec@te@charactertotal%
	\fi%
	%
	\ifx\tikz@lib@dec@te@wordcountvar\pgfutil@empty%
	\else%
		\expandafter\let\tikz@lib@dec@te@wordcountvar=\tikz@lib@dec@te@wordcount%
	\fi%
	%
	\ifx\tikz@lib@dec@te@lettercountvar\pgfutil@empty%
	\else%
		\expandafter\let\tikz@lib@dec@te@lettercountvar=\tikz@lib@dec@te@lettercount%
	\fi%
	%
	\edef\tikz@lib@dec@te@tmp{character \tikz@lib@dec@te@charactercount/.try,%
		character {\meaning\tikz@lib@dec@te@token}/.try}%
	\expandafter\tikz@lib@dec@te@addoptions\expandafter{\tikz@lib@dec@te@tmp}%
	\ifnum\tikz@lib@dec@te@wordcount>0\relax%
		\edef\tikz@lib@dec@te@tmp{every word/.try, word \tikz@lib@dec@te@wordcount/.try}%
		\expandafter\tikz@lib@dec@te@addoptions\expandafter{\tikz@lib@dec@te@tmp}%
		\ifnum\tikz@lib@dec@te@lettercount>0\relax%
			\edef\tikz@lib@dec@te@tmp{every letter/.try,every \tikz@lib@dec@te@lettercount\space letter/.try}%
			\expandafter\tikz@lib@dec@te@addoptions\expandafter{\tikz@lib@dec@te@tmp}%
			\ifnum\tikz@lib@dec@te@lettercount=1\relax%
				\edef\tikz@lib@dec@te@tmp{every first letter/.try}%
				\expandafter\tikz@lib@dec@te@addoptions\expandafter{\tikz@lib@dec@te@tmp}%
			\fi%
			\iftikz@lib@dec@te@finalletter%
				\tikz@lib@dec@te@addoptions{every final letter/.try}%
			\fi%
		\fi%
	\fi%
	\ifx\tikz@lib@dec@te@token\tikz@lib@dec@te@wordsep%
		\tikz@lib@dec@te@addoptions{/pgf/decoration/text effects/every word separator/.try}%
	\fi%
}

% Get the width of the node containing the current character 
\def\tikz@lib@dec@te@getcharacterwidth{%
	\tikz@lib@dec@te@getoptions{every character width/.try}%
	\csname tikz@lib@dec@te@character@\tikz@lib@dec@te@charactercount\endcsname%
	%
	% Does this kerning calculating really work...?
	\iftikz@lib@dec@te@kerning
		% Get width of node containing the last character and the current character.
		\pgfpositionnodelater\tikz@lib@dec@te@@getcharacterwidth%
		\expandafter\node\expandafter[\tikz@lib@dec@te@options]{%
			\hbox{\tikz@lib@dec@te@lastcharactercommand{\tikz@lib@dec@te@lasttoken}%
			\tikz@lib@dec@te@charactercommand{\tikz@lib@dec@te@token}}%
		};%
		\let\tikz@lib@dec@te@characterswidth=\tikz@lib@dec@te@characterwidth@tmp%
	\fi%
	% Get width of node containing only the current character.
	\let\tikz@lib@dec@te@lastcharacterwidth=\tikz@lib@dec@te@characterwidth%
	\pgfpositionnodelater\tikz@lib@dec@te@@getcharacterwidth%
	\pgfutil@ifundefined{tikz@lib@dec@te@character@replacements@\meaning\tikz@lib@dec@te@token @code}{%
		\expandafter\node\expandafter[\tikz@lib@dec@te@options]{\hbox{\tikz@lib@dec@te@charactercommand{\tikz@lib@dec@te@token}}};%
		\let\tikz@lib@dec@te@characterwidth=\tikz@lib@dec@te@characterwidth@tmp%
	}{%
	  \tikz@lib@dec@te@getcharacter@replacementwidth{\tikz@lib@dec@te@token}%
	  \let\tikz@lib@dec@te@characterwidth=\tikz@lib@dec@te@character@replacementwidth%
	}%
	%
	% Compensate for kerning.
	\iftikz@lib@dec@te@kerning%
		\pgf@x=\tikz@lib@dec@te@characterswidth\relax%
		\advance\pgf@x by-\tikz@lib@dec@te@characterwidth\relax%
		\advance\pgf@x by-\tikz@lib@dec@te@lastcharacterwidth\relax%
		\ifdim\pgf@x>0pt\relax%
			\pgf@x=0pt\relax%
		\fi%		
		\advance\pgf@x by\tikz@lib@dec@te@characterwidth\relax%
		\edef\tikz@lib@dec@te@@characterwidth{\the\pgf@x}%
	\fi%		
}


% Called by `late positioning' of the node
\def\tikz@lib@dec@te@@getcharacterwidth{%
	\iftikz@lib@dec@te@segmentfromwidth%
		\pgf@x=\pgfpositionnodelatermaxx\relax%
		\advance\pgf@x by-\pgfpositionnodelaterminx\relax%
	\else%
		\pgf@x=\pgfpositionnodelatermaxy\relax%
		\advance\pgf@x by-\pgfpositionnodelaterminy\relax%
	\fi%
	\xdef\tikz@lib@dec@te@characterwidth@tmp{\the\pgf@x}%
}


% Actually draw a character node.
\def\tikz@lib@dec@te@drawcharacter{%
	\csname tikz@lib@dec@te@character@\tikz@lib@dec@te@charactercount\endcsname%
	\tikz@lib@dec@te@getoptions{every character/.try}%
	\pgfpositionnodelater\relax%
	\pgfutil@ifundefined{tikz@lib@dec@te@character@replacements@\meaning\tikz@lib@dec@te@token @code}{%
		\expandafter\node\expandafter[\tikz@lib@dec@te@options]{\hbox{\tikz@lib@dec@te@charactercommand{\tikz@lib@dec@te@token}}};%
	}{\tikz@lib@dec@te@drawcharacter@replacement{\tikz@lib@dec@te@token}}%
}






% Evil hack into the decoration code.
%
% If the path contains single move to and the `path from text'
% key is set to true. The width of the text is calcuated here
% and the (straight line) path automatically calculated.
%
\def\pgf@decorate@path@check@moveto#1{%
	\ifx\tikz@lib@dec@te@transformdecorationtext\pgfutil@empty%
	\else%
	  \tikz@lib@dec@te@transformdecorationtext%
	\fi%
  \expandafter\pgf@decorate@path@@check@moveto#1\pgf@decorate@stop\pgf@decorate@@stop}

\def\pgf@decorate@token@stop{\pgf@decorate@stop}%
\def\pgf@decorate@path@@check@moveto#1#2#3#4\pgf@decorate@@stop#5#6{%
  \def\pgf@decorate@temp{#4}%
  \pgf@x=#2\relax%
  \pgf@y=#3\relax%
  \ifx\pgf@decorate@temp\pgf@decorate@token@stop%
		 \iftikz@lib@dec@te@pathfromtext%
		 	% Get the start position...
			\pgfextract@process\tikz@lib@dec@te@pathstart{}%
			\tikz@lib@dec@te@getcharactercount%
			\let\pgfdecorationremainingtext=\pgfdecorationtext%			
			% ...and the width of the text...
			\tikz@lib@dec@te@gettextwidth%
			\let\tikz@lib@dec@te@charactertotal=\tikz@lib@dec@te@charactercount%
			% ... and create a lineto path of the correct length.
			\pgfsetpath\pgfutil@empty%
			\pgfpathmoveto{\tikz@lib@dec@te@pathstart}%
			\pgfpathlineto{\pgfpointadd{\tikz@lib@dec@te@pathstart}%
				{\pgfpointpolar{\tikz@lib@dec@te@pathfromtextangle}{\tikz@lib@dec@te@textwidth}}}%
			\pgfgetpath\pgfdecoratedpath%
			\pgfsetpath\pgfutil@empty%
  	\else%
    	#5%
    \fi%
  \else%
  	\tikz@lib@dec@te@pathfromtextfalse%
    #6%
  \fi%
}

\def\tikz@lib@dec@align@left@text{left}
\def\tikz@lib@dec@align@right@text{right}
\def\tikz@lib@dec@align@center@text{center}

% Fianlly the actual decoration.
%
\pgfdeclaredecoration{text effects along path}{set-up}{
	\state{set-up}[width=+0pt, next state=shift, persistent precomputation={%
		\iftikz@lib@dec@te@pathfromtext%
			% The width of the text has already been calculated.
		\else%
			\tikz@lib@dec@te@getcharactercount%
			\let\pgfdecorationremainingtext=\pgfdecorationtext%
			\tikz@lib@dec@te@gettextwidth%
			\let\tikz@lib@dec@te@charactertotal=\tikz@lib@dec@te@charactercount%
		\fi%
		% Calculate the distance to move along the path for alignment.
		\pgfkeysgetvalue{/pgf/decoration/text align/align}{\tikz@lib@dec@align}%
	  \ifx\tikz@lib@dec@align\tikz@lib@dec@align@center@text%
		  \pgf@x=\pgfdecoratedpathlength\relax%
		  \advance\pgf@x by-\tikz@lib@dec@te@textwidth\relax%
		  \divide\pgf@x by2\relax%
			\else%
				\ifx\tikz@lib@dec@align\tikz@lib@dec@align@right@text%
					\pgf@x=\pgfdecoratedpathlength\relax%
					\advance\pgf@x by-\tikz@lib@dec@te@textwidth\relax%
				\else%
					\pgf@x=0pt\relax%
				\fi%
			\fi%
	  \ifdim\pgf@x<0pt\relax%
	  	\pgf@x=0pt\relax%
		\fi%
		\edef\tikz@lib@te@alignshift{\the\pgf@x}%
	  %
	 	\def\tikz@lib@dec@te@charactercount{0}%	  
		\let\pgfdecorationremainingtext=\pgfdecorationtext%
	}]{}
	%
\state{shift}[width=+\tikz@lib@te@alignshift, next state=scan]{}
%
\state{scan}[width=+0pt, next state=pre token, persistent precomputation={
	\let\tikz@lib@dec@te@rest=\pgfdecorationremainingtext%
	\tikz@lib@dec@te@scanchar%
	\ifx\tikz@lib@dec@te@token\pgfutil@empty%
		\def\pgf@decorate@next@state{final}%
	\else%
		\let\pgfdecorationremainingtext=\tikz@lib@dec@te@rest%
		\tikz@lib@dec@te@advancecountmacro\tikz@lib@dec@te@charactercount{1}%
		\csname tikz@lib@dec@te@character@\tikz@lib@dec@te@charactercount\endcsname%
	\fi%
}]{}
%
\state{pre token}[width=\tikz@lib@dec@te@characterwidth/2, next state=token]{}
\state{token}[width=+0pt, next state=post token]
{%
	\tikz@lib@dec@te@drawcharacter%
}
\state{post token}[width=\tikz@lib@dec@te@characterwidth/2, next state=scan]{}
}

\def\tikz@lib@dec@te@getcharactercount{%
	\let\tikz@lib@dec@te@rest=\pgfdecorationtext%
	\pgfmathloop%
	\tikz@lib@dec@te@scanchar%
	\ifx\tikz@lib@dec@te@token\pgfutil@empty%
	\else%
	\repeatpgfmathloop%
	\let\tikz@lib@dec@te@charactertotal=\pgfmathcounter%
}



% Parse styles for individual characters
% #1 a string of characters (e.g., aieou{\"U}{\"a"})
% #2 the options for to apply to each character.
%
\def\tikz@lib@dec@te@parse@stylecharacters#1#2{%
	\let\tikz@lib@dec@te@rest@copy=\tikz@lib@dec@te@rest%
	\def\tikz@lib@dec@te@rest{#1}%
	\pgfmathloop%
		\tikz@lib@dec@te@scanchar%
		\ifx\tikz@lib@dec@te@token\pgfutil@empty%
		\else%
			\edef\tikz@lib@dec@te@marshal{/pgf/decoration/text effects/character \meaning\tikz@lib@dec@te@token}%
			\pgfkeysalso{/pgf/decoration/text effects/character {\meaning\tikz@lib@dec@te@token}/.style={#2}}%
		\repeatpgfmathloop%
		\let\tikz@lib@dec@te@rest=\tikz@lib@dec@te@rest@copy%
}

% Parse replacment code for for individual characters
% #1 a string of characters (e.g., aieou{\"U}{\"a"})
% #2 the code to execute to each character.
%
\def\tikz@lib@dec@te@parse@replacecharacters#1#2{%
	\let\tikz@lib@dec@te@rest@copy=\tikz@lib@dec@te@rest%
	\def\tikz@lib@dec@te@rest{#1}%
	\pgfmathloop%
		\tikz@lib@dec@te@scanchar%
		\ifx\tikz@lib@dec@te@token\pgfutil@empty%
		\else%
			\expandafter\def\csname tikz@lib@dec@te@character@replacements@\meaning\tikz@lib@dec@te@token @code\endcsname{#2}%
		\repeatpgfmathloop%
		\let\tikz@lib@dec@te@rest=\tikz@lib@dec@te@rest@copy%
}

	  
% Reverse the characters in \pgfdecorationtext
% 
% So, `text effects along path!` becomes `!htap gnola stceffe txet'
%
\def\tikz@lib@dec@te@reversetext{%
  \let\tikz@lib@dec@te@rest=\pgfdecorationtext%
	\let\tikz@lib@dec@te@text=\pgfutil@empty%
	\pgfmathloop%
	\tikz@lib@dec@te@scanchar%
	\ifx\tikz@lib@dec@te@token\pgfutil@empty%
	\else%
		\expandafter\expandafter\expandafter\def%
		  \expandafter\expandafter\expandafter\tikz@lib@dec@te@text%
		    \expandafter\expandafter\expandafter{\expandafter\tikz@lib@dec@te@token\tikz@lib@dec@te@text}%
	\repeatpgfmathloop%
	\let\pgfdecorationtext=\tikz@lib@dec@te@text%
}	

% Group characters (letters) in \pgfdecorationtext
%
% So, `text effects along path!` becomes `{text} {effects} {along} {path!}`
%
\def\tikz@lib@dec@te@letterstowords{%
  \let\tikz@lib@dec@te@rest=\pgfdecorationtext%
	\let\tikz@lib@dec@te@text=\pgfutil@empty%
	\let\tikz@lib@dec@te@word=\pgfutil@empty%
	\pgfmathloop%
	\tikz@lib@dec@te@scanchar%
	\ifx\tikz@lib@dec@te@token\pgfutil@empty%
	\else%
	  \ifx\tikz@lib@dec@te@token\tikz@lib@dec@te@wordsep%
	  	\ifx\tikz@lib@dec@te@text\pgfutil@empty%
	  	\else%
	  	  \expandafter\tikz@lib@dec@te@addto@macro\expandafter\tikz@lib@dec@te@text%
	  	  	\expandafter{\tikz@lib@dec@te@wordsep}%
	  	\fi%
	  	\expandafter\tikz@lib@dec@te@addto@macro\expandafter\tikz@lib@dec@te@text%
	  		 \expandafter{\expandafter{\tikz@lib@dec@te@word}}%
	  	\let\tikz@lib@dec@te@word=\pgfutil@empty%
	  \else%
		 	\expandafter\tikz@lib@dec@te@addto@macro\expandafter\tikz@lib@dec@te@word%
		 		    \expandafter{\tikz@lib@dec@te@token}%
		\fi%
	\repeatpgfmathloop%
	\ifx\tikz@lib@dec@te@word\pgfutil@empty%
	\else%
		\ifx\tikz@lib@dec@te@text\pgfutil@empty%
	  \else%
 	  	\expandafter\tikz@lib@dec@te@addto@macro\expandafter\tikz@lib@dec@te@text%
	  	  \expandafter{\expandafter{\tikz@lib@dec@te@wordsep}}%
  	\fi%
  	\expandafter\tikz@lib@dec@te@addto@macro\expandafter\tikz@lib@dec@te@text%
				\expandafter{\expandafter{\tikz@lib@dec@te@word}}%
	\fi%
	\let\pgfdecorationtext=\tikz@lib@dec@te@text%
}	

% Replace the character #1 with a character@replacement
%
\def\tikz@lib@dec@te@drawcharacter@replacement#1{%
	\pgfscope%
	\expandafter\tikzset\expandafter{\tikz@lib@dec@te@options}%
	\csname tikz@lib@dec@te@character@replacements@\meaning#1@code\endcsname%
	\endpgfscope%
}

% Get the width of the character@replacement associated with
% the character #1
%
\newbox\tikz@lib@dec@te@box
\def\tikz@lib@dec@te@getcharacter@replacementwidth#1{%
	\pgfinterruptpicture%
	\setbox\tikz@lib@dec@te@box=\hbox{%
	\pgfpicture
	\expandafter\tikzset\expandafter{\tikz@lib@dec@te@options}%
	\csname tikz@lib@dec@te@character@replacements@\meaning#1@code\endcsname%
	\endpgfpicture}%
	\xdef\tikz@lib@dec@te@character@replacementwidth{\the\wd\tikz@lib@dec@te@box}%
	\endpgfinterruptpicture%
}


\endinput

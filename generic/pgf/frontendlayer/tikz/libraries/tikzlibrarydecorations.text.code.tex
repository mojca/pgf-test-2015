% Copyright 2013 by Mark Wibrow
%
% This file may be distributed and/or modified
%
% 1. under the LaTeX Project Public License and/or
% 2. under the GNU Public License.
%
% See the file doc/generic/pgf/licenses/LICENSE for more details.

\usetikzlibrary{decorations}
\usepgflibrary{decorations.text}


% Code for the `text effects along path' decoration

% Some text constants
% To do: merge these with existing text constants
\def\tikz@lib@dec@te@space@text{space}
\def\tikz@lib@dec@align@left@text{left}
\def\tikz@lib@dec@align@right@text{right}
\def\tikz@lib@dec@align@center@text{center}

% Some ifs
\newif\iftikz@lib@dec@te@pathfromtext
\newif\iftikz@lib@dec@te@segmentfromwidth
\newif\iftikz@lib@dec@te@scaletexttopath
\newif\iftikz@lib@dec@te@fittexttopath
\newif\iftikz@lib@dec@te@wordsep
\newif\iftikz@lib@dec@te@finalletter


% Utility macros
% To do: integrate with exisiting pgf macros
\def\tikz@lib@dec@te@addto@macro#1#2{%
  \expandafter\def\expandafter#1\expandafter{#1#2}%
}

\def\tikz@lib@dec@te@namelet#1#2{%
  \expandafter\expandafter\expandafter\let\expandafter\expandafter\csname\expandafter#1\expandafter\endcsname\expandafter=\csname#2\endcsname%
}

\def\tikz@lib@dec@te@advancecountmacro#1#2{%
  \pgfutil@tempcnta=#1\relax%
  \advance\pgfutil@tempcnta by#2\relax%
  \edef#1{\the\pgfutil@tempcnta}%
}

\def\tikz@lib@dec@te@advancedimenmacro#1#2{%
  \pgf@x=#1\relax%
  \advance\pgf@x by#2\relax%
  \edef#1{\the\pgf@x}%
}

% Macros for defining and updating text parameters
\def\tikz@lib@dec@te@start@parameters#1{%
 \expandafter\let\csname tikz@lib@dec@te@character@#1\endcsname=\pgfutil@empty%
}

\def\tikz@lib@dec@te@addto@parameters#1#2{%
 \expandafter\tikz@lib@dec@te@addto@macro\expandafter%
   {\csname tikz@lib@dec@te@character@#1\endcsname}{#2}%
}
\def\tikz@lib@dec@te@addmacroto@parameters#1#2{%
 \def\tikz@lib@dec@te@marshal{\expandafter\tikz@lib@dec@te@addto@macro\expandafter%
   {\csname tikz@lib@dec@te@character@#1\endcsname}}%
  \expandafter\tikz@lib@dec@te@marshal\expandafter{\expandafter\def\expandafter#2\expandafter{#2}}%
}

\def\tikz@lib@dec@te@getparameters#1{\csname tikz@lib@dec@te@character@#1\endcsname}



\let\tikz@lib@dec@te@textoptions=\pgfutil@empty
\let\tikz@lib@dec@te@widthoptions=\pgfutil@empty
\let\tikz@lib@dec@te@charactercountvar=\pgfutil@empty
\let\tikz@lib@dec@te@charactertotalvar=\pgfutil@empty
\let\tikz@lib@dec@te@lettercountvar=\pgfutil@empty
\let\tikz@lib@dec@te@wordcountvar=\pgfutil@empty
\let\tikz@lib@dec@te@transformtext=\pgfutil@empty%


% Keys... lots of them...
\pgfkeys{/pgf/decoration/text effects/.cd,
   .unknown/.code={\let\tikz@lib@dec@te@key@tmp=\pgfkeyscurrentname%
    \pgfkeysalso{/tikz/\tikz@lib@dec@te@key@tmp={#1}}},
   path from text/.is if=tikz@lib@dec@te@pathfromtext,
   path from text angle/.store in=\tikz@lib@dec@te@pathfromtextangle,
   path from text angle=0,
   fit text to path/.is if=tikz@lib@dec@te@fittexttopath,
   characters/.style={/pgf/decoration/text effects/every character/.style={#1}},
   characters/.append/.code={\pgfkeysalso{/pgf/decoration/text effects/every character/.append style={#1}}},
   character widths/.code={\pgfkeysalso{/pgf/decoration/text effects/every character width/.style={#1}}},
   character widths/.append/.code={\pgfkeysalso{/pgf/decoration/text effects/every character width/.append style={#1}}},
   character count/.store in=\tikz@lib@dec@te@charactercountvar,
   character count=,
   character total/.store in=\tikz@lib@dec@te@charactertotalvar,
   character total=,
   letter count/.store in=\tikz@lib@dec@te@lettercountvar,
   letter count=,
   letter total/.store in=\tikz@lib@dec@te@lettertotalcountvar,
   letter total=,
   word count/.store in=\tikz@lib@dec@te@wordcountvar,
   word count=,
   word total/.store in=\tikz@lib@dec@te@wordtotalvar,
   word total=,
   %
   style characters/.code args={#1 with #2}{\tikz@lib@dec@te@parse@stylecharacters{#1}{#2}},
   text along path/.style={
     inner xsep=0pt,
     anchor=base,
     transform shape
   },
   segment from width/.is if=tikz@lib@dec@te@segmentfromwidth,
   segment from width=true,
   word separator/.code={%
     \def\tikz@lib@dec@te@tmp{#1}%
     \ifx\tikz@lib@dec@te@tmp\tikz@lib@dec@te@space@text%
       \let\tikz@lib@dec@te@wordsep=\tikz@lib@dec@te@spacetoken%
     \else%
       \def\tikz@lib@dec@te@wordsep{#1}
     \fi},
   word separator=space,
   character command/.store in=\tikz@lib@dec@te@charactercommand,
   character command=\tikz@lib@dec@te@charactertypesetonly,
   every character width/.style={/pgf/decoration/text effects/every character/.try},
   every character/.style={},
   reverse text/.code={\tikz@lib@dec@te@addto@macro\tikz@lib@dec@te@transformtext%
     {\tikz@lib@dec@te@reversetext}},
   group letters into words/.code={\tikz@lib@dec@te@addto@macro\tikz@lib@dec@te@transformtext%
     {\tikz@lib@dec@te@groupletters}},
   group letters/.code={\tikz@lib@dec@te@addto@macro\tikz@lib@dec@te@transformtext%
     {\tikz@lib@dec@te@groupletters}},
   replace characters/.code args={#1 with #2}{\tikz@lib@dec@te@parse@replacecharacters{#1}{#2}},
   scale text to path/.is if=tikz@lib@dec@te@scaletexttopath,
   fit text to path/.is if=tikz@lib@dec@te@fittexttopath,
   text repeat/.store in=\tikz@lib@dec@te@repeattext,
   text repeat/.default=-1,
   text repeat=0,
}


\def\tikz@lib@dec@te@charactertypesetonly#1{#1}

\tikzset{%
  text effects/.code={\pgfkeysalso{/pgf/decoration/text effects/.cd,#1}},
}






% Prepare the decoration text
%
% This sets up all the character/letter/word, counts/totals
% and caluclates the default width of the characters/text 
% (which may be modified later).
%
\def\tikz@lib@dec@te@preparetext{%
  % Step 1: scan all the characters in the decoration text
  \let\tikz@lib@dec@te@text=\pgfdecorationtext%
  \tikz@lib@dec@te@scancharacters{%
    \expandafter\let\csname tikz@lib@dec@te@character@\tikz@lib@dec@te@scancount @token\endcsname=%
      \tikz@lib@dec@te@character%
  }%
  \let\tikz@lib@dec@te@charactertotalcount=\tikz@lib@dec@te@scancount%
  % Step 2: reverse and/or group characters/letters.
  \ifx\tikz@lib@dec@te@transformtext\pgfutil@empty%
  \else%
    \tikz@lib@dec@te@transformtext%
  \fi%
  % Step 3: get the character/letter/word counts
  \def\tikz@lib@dec@te@charactercount{0}%
  \def\tikz@lib@dec@te@wordcount{0}%
  \def\tikz@lib@dec@te@wordsepcount{0}%
  \def\tikz@lib@dec@te@lettercount{0}%
  \def\tikz@lib@dec@te@lastcharacter{0}%
  \pgfmathloop%
  \ifnum\pgfmathcounter>\tikz@lib@dec@te@charactertotalcount%
    \ifnum\tikz@lib@dec@te@lettercount>0\relax%
      \tikz@lib@dec@te@addto@parameters{\tikz@lib@dec@te@lastcharactercount}{\tikz@lib@dec@te@finallettertrue}%
    \fi
  \else%
    \expandafter\let\expandafter\tikz@lib@dec@te@character\expandafter=\csname tikz@lib@dec@te@character@\pgfmathcounter @token\endcsname%
    \tikz@lib@dec@te@advancecountmacro{\tikz@lib@dec@te@charactercount}{1}%
    \tikz@lib@dec@te@start@parameters{\tikz@lib@dec@te@charactercount}%
    \tikz@lib@dec@te@addmacroto@parameters{\tikz@lib@dec@te@charactercount}{\tikz@lib@dec@te@charactercount}%
    \ifx\tikz@lib@dec@te@character\tikz@lib@dec@te@wordsep%
      \tikz@lib@dec@te@advancecountmacro{\tikz@lib@dec@te@wordsepcount}{1}%
      \ifnum\tikz@lib@dec@te@lettercount>0\relax%
        \tikz@lib@dec@te@addto@parameters{\tikz@lib@dec@te@lastcharactercount}{\tikz@lib@dec@te@finallettertrue}%
      \fi
      \def\tikz@lib@dec@te@lettercount{0}%
      \tikz@lib@dec@te@addto@parameters{\tikz@lib@dec@te@charactercount}{\tikz@lib@dec@te@wordseptrue}%    
    \else
      \ifnum\tikz@lib@dec@te@lettercount=0\relax%
        \tikz@lib@dec@te@advancecountmacro\tikz@lib@dec@te@wordcount{1}%
      \fi%
      \tikz@lib@dec@te@addto@parameters{\tikz@lib@dec@te@charactercount}{\tikz@lib@dec@te@finalletterfalse}%
      \tikz@lib@dec@te@advancecountmacro{\tikz@lib@dec@te@lettercount}{1}%            
      \tikz@lib@dec@te@addto@parameters{\tikz@lib@dec@te@charactercount}{\tikz@lib@dec@te@wordsepfalse}%
       \expandafter\let\csname tikz@lib@dec@te@word \tikz@lib@dec@te@wordcount @lettercount\endcsname=\tikz@lib@dec@te@lettercount%
    \fi%
    \tikz@lib@dec@te@addmacroto@parameters{\tikz@lib@dec@te@charactercount}{\tikz@lib@dec@te@lettercount}%
    \tikz@lib@dec@te@addmacroto@parameters{\tikz@lib@dec@te@charactercount}{\tikz@lib@dec@te@wordcount}%
    \tikz@lib@dec@te@addmacroto@parameters{\tikz@lib@dec@te@charactercount}{\tikz@lib@dec@te@character}%
    \let\tikz@lib@dec@te@lastcharactercount=\pgfmathcounter%
  \repeatpgfmathloop%
  \let\tikz@lib@dec@te@wordtotalcount=\tikz@lib@dec@te@wordcount%
  % Step 4: add character/letter/word totals
  \pgfmathloop
  \ifnum\pgfmathcounter>\tikz@lib@dec@te@charactertotalcount%
  \else%
    \tikz@lib@dec@te@addmacroto@parameters{\pgfmathcounter}{\tikz@lib@dec@te@charactertotalcount}%
    \tikz@lib@dec@te@addmacroto@parameters{\pgfmathcounter}{\tikz@lib@dec@te@wordtotalcount}%
    \expandafter\let\expandafter\tikz@lib@dec@te@lettertotalcount\expandafter=\csname tikz@lib@dec@te@word \tikz@lib@dec@te@wordcount @lettercount\endcsname%
    \tikz@lib@dec@te@addmacroto@parameters{\pgfmathcounter}{\tikz@lib@dec@te@lettertotalcount}%
  \repeatpgfmathloop%
  % Step 5: calculate the the text and character width
  % but do not add to the parameters yet.
  \def\tikz@lib@dec@te@textwidth{0pt}%
  \pgfmathloop%
  \ifnum\pgfmathcounter>\tikz@lib@dec@te@charactertotalcount%
  \else%
    \def\tikz@lib@dec@te@characterwidth{0pt}%
    \tikz@lib@dec@te@getcharacterwidth{\pgfmathcounter}%
    \expandafter\let\csname tikz@lib@dec@te@character@\pgfmathcounter @width\endcsname=\tikz@lib@dec@te@characterwidth%
    \pgf@x=\tikz@lib@dec@te@characterwidth\relax%
    \pgf@x=0.5\pgf@x%
    \expandafter\edef\csname tikz@lib@dec@te@character@\pgfmathcounter @prewidth\endcsname{\the\pgf@x}%
    \expandafter\edef\csname tikz@lib@dec@te@character@\pgfmathcounter @postwidth\endcsname{\the\pgf@x}%
    \tikz@lib@dec@te@advancedimenmacro{\tikz@lib@dec@te@textwidth}{\tikz@lib@dec@te@characterwidth}%
  \repeatpgfmathloop%
}

% Does the category code of the space matter?
\def\tikz@lib@dec@te@spacetoken{ }

\let\tikz@lib@dec@te@wordsep=\tikz@lib@dec@te@spacetoken


% Scan and count the characters in \tikz@lib@dec@te@text
%
% Create macros \tikz@lib@dec@te@character@<n>@token containing
% the nth character.
%
% Total number of characters is store in 
% \tikz@lib@dec@te@chractertotal@count
%
\def\tikz@lib@dec@te@scancharacters#1{%
  \def\tikz@lib@dec@te@@@scancharacters@process{#1}%
  \def\tikz@lib@dec@te@scancount{0}%
  \expandafter\tikz@lib@dec@te@@scancharacters\tikz@lib@dec@te@text\tikz@lib@dec@te@scancharacters}%

\def\tikz@lib@dec@te@@scancharacters{%
  \futurelet\tikz@lib@dec@te@token\tikz@lib@dec@te@@@scancharacters}

\def\tikz@lib@dec@te@@@scancharacters{%
  \ifx\tikz@lib@dec@te@token\tikz@lib@dec@te@scancharacters%
    \let\tikz@lib@dec@te@next=\tikz@lib@dec@te@@@@scancharacters%
  \else%
    \tikz@lib@dec@te@advancecountmacro\tikz@lib@dec@te@scancount{1}%
    \ifx\tikz@lib@dec@te@token\pgfutil@sptoken%
      \let\tikz@lib@dec@te@next=\tikz@lib@dec@te@@@scancharacters@space%
    \else%
      \ifx\tikz@lib@dec@te@token\bgroup%
        \let\tikz@lib@dec@te@next=\tikz@lib@dec@te@@@scancharacters@bgroup%
      \else%
        \let\tikz@lib@dec@te@next=\tikz@lib@dec@te@@@scancharacters@normal%
      \fi%
    \fi%
  \fi%
  \tikz@lib@dec@te@next%
}

\def\tikz@lib@dec@te@@@scancharacters@space{%
  \let\tikz@lib@dec@te@character=\tikz@lib@dec@te@spacetoken%
  \tikz@lib@dec@te@@@scancharacters@process%
  \pgfutil@ifnextchar x\tikz@lib@dec@te@@scancharacters\tikz@lib@dec@te@@scancharacters%
}

\def\tikz@lib@dec@te@@@scancharacters@bgroup#1{%
  \def\tikz@lib@dec@te@character{{#1}}%
  \tikz@lib@dec@te@@@scancharacters@process%
  \tikz@lib@dec@te@@scancharacters%
}

\def\tikz@lib@dec@te@@@scancharacters@normal#1{%
  \def\tikz@lib@dec@te@character{#1}%
  \tikz@lib@dec@te@@@scancharacters@process%
  \tikz@lib@dec@te@@scancharacters%
}


\def\tikz@lib@dec@te@@@@scancharacters#1\tikz@lib@dec@te@scancharacters{}


% Reverse the order of the scanned chracters in the text
% (Actually reverse the ordering of the macros containing the characters)
\def\tikz@lib@dec@te@reversetext{%
  \c@pgf@counta=\tikz@lib@dec@te@charactertotalcount%
  \pgfmathloop%
  \ifnum\pgfmathcounter>\c@pgf@counta%
  \else%
    \tikz@lib@dec@te@namelet{tikz@lib@dec@te@tmp}{tikz@lib@dec@te@character@\pgfmathcounter @token}%
    \tikz@lib@dec@te@namelet{tikz@lib@dec@te@character@\pgfmathcounter @token}{tikz@lib@dec@te@character@\the\c@pgf@counta @token}%
    \tikz@lib@dec@te@namelet{tikz@lib@dec@te@character@\the\c@pgf@counta @token}{tikz@lib@dec@te@tmp}%
    \advance\c@pgf@counta by-1\relax%
  \repeatpgfmathloop%
}

% Group letters (i.e., anything that isn't the word separator)
% as single characters (words).
\def\tikz@lib@dec@te@groupletters{%
  \c@pgf@counta=0\relax%
  \let\tikz@lib@dec@te@tmp=\pgfutil@empty%
  \let\tikz@lib@dec@te@@tmp=\pgfutil@empty%
  \pgfmathloop%
  \ifnum\pgfmathcounter>\tikz@lib@dec@te@charactertotalcount%
    \ifx\tikz@lib@dec@te@@tmp\pgfutil@empty%
    \else%
      \advance\c@pgf@counta by1\relax%
      \tikz@lib@dec@te@namelet{tikz@lib@dec@te@character@\the\c@pgf@counta @token}{tikz@lib@dec@te@@tmp}%
    \fi%
  \else%
    \tikz@lib@dec@te@namelet{tikz@lib@dec@te@tmp}{tikz@lib@dec@te@character@\pgfmathcounter @token}%
    \ifx\tikz@lib@dec@te@tmp\tikz@lib@dec@te@wordsep%
      \advance\c@pgf@counta by1\relax%
      \tikz@lib@dec@te@namelet{tikz@lib@dec@te@character@\the\c@pgf@counta @token}{tikz@lib@dec@te@@tmp}%
      \let\tikz@lib@dec@te@@tmp=\pgfutil@empty%
      \advance\c@pgf@counta by1\relax%
      \tikz@lib@dec@te@namelet{tikz@lib@dec@te@character@\the\c@pgf@counta @token}{tikz@lib@dec@te@wordsep}%
    \else%
      \expandafter\tikz@lib@dec@te@addto@macro\expandafter{\expandafter\tikz@lib@dec@te@@tmp\expandafter}\expandafter{\tikz@lib@dec@te@tmp}%
    \fi%
  \repeatpgfmathloop%
  \edef\tikz@lib@dec@te@charactertotalcount{\the\c@pgf@counta}%
}


% Get width of node containing only the current character.
\def\tikz@lib@dec@te@getcharacterwidth#1{%
  \csname tikz@lib@dec@te@character@#1\endcsname%
  \tikz@lib@dec@te@getoptions{every character width/.try}%
  \let\tikz@lib@dec@te@lastcharacterwidth=\tikz@lib@dec@te@characterwidth%
  \pgfpositionnodelater\tikz@lib@dec@te@@getcharacterwidth%
  \pgfutil@ifundefined{tikz@lib@dec@te@character@replacements@\meaning\tikz@lib@dec@te@character @code}{%
    \expandafter\node\expandafter[\tikz@lib@dec@te@options,scale=\tikz@lib@dec@scaledcharacterwidthfactor]%
    {\hbox{\tikz@lib@dec@te@charactercommand{\tikz@lib@dec@te@character}}};%
    \let\tikz@lib@dec@te@characterwidth=\tikz@lib@dec@te@characterwidth@tmp%
  }{%
    \tikz@lib@dec@te@getcharacter@replacementwidth{\tikz@lib@dec@te@character}%
    \let\tikz@lib@dec@te@characterwidth=\tikz@lib@dec@te@character@replacementwidth%
  }%
}

% Update the character parameters with the widths.
\def\tikz@lib@dec@te@addwidthstoparameters{%
  \pgfmathloop%
  \ifnum\pgfmathcounter>\tikz@lib@dec@te@charactertotalcount%
  \else%
    \tikz@lib@dec@te@namelet{tikz@lib@dec@te@characterwidth}{tikz@lib@dec@te@character@\pgfmathcounter @width}%
    \tikz@lib@dec@te@namelet{tikz@lib@dec@te@characterprewidth}{tikz@lib@dec@te@character@\pgfmathcounter @prewidth}%
    \tikz@lib@dec@te@namelet{tikz@lib@dec@te@characterpostwidth}{tikz@lib@dec@te@character@\pgfmathcounter @postwidth}%
    \tikz@lib@dec@te@addmacroto@parameters{\pgfmathcounter}{\tikz@lib@dec@te@characterwidth}%
    \tikz@lib@dec@te@addmacroto@parameters{\pgfmathcounter}{\tikz@lib@dec@te@characterprewidth}%
    \tikz@lib@dec@te@addmacroto@parameters{\pgfmathcounter}{\tikz@lib@dec@te@characterpostwidth}%
  \repeatpgfmathloop%
}

% Get the options to applied to a character node.

\def\tikz@lib@dec@te@addoptions#1{%
  \expandafter\def\expandafter\tikz@lib@dec@te@options\expandafter{\tikz@lib@dec@te@options,#1}%
}

\def\tikz@lib@dec@te@getoptions#1{%
  \let\tikz@lib@dec@te@lastcharactercommand=\tikz@lib@dec@te@charactercommand%
  \def\tikz@lib@dec@te@options{/pgf/decoration/text effects/.cd}%
  \tikz@lib@dec@te@addoptions{#1}%
  \let\tikzdecorationcharactercount=\tikz@lib@dec@te@charactercount%
  \let\tikzdecorationcharactertotal=\tikz@lib@dec@te@charactertotalcount%
  \let\tikzdecorationlettercount=\tikz@lib@dec@te@lettercount%
  \let\tikzdecorationlettertotal=\tikz@lib@dec@te@lettertotalcount%
  \let\tikzdecorationwordcount=\tikz@lib@dec@te@wordcount%
  \let\tikzdecorationwordtotal=\tikz@lib@dec@te@wordtotalcount%
  \let\tikzdecorationcharacter=\tikz@lib@dec@te@character%
  %
  % Define the count variables (if required)
  \ifx\tikz@lib@dec@te@charactercountvar\pgfutil@empty%
  \else%
    \expandafter\let\tikz@lib@dec@te@charactercountvar=\tikz@lib@dec@te@charactercount%
  \fi%
  %
  \ifx\tikz@lib@dec@te@charactertotalvar\pgfutil@empty%
  \else%
    \expandafter\let\tikz@lib@dec@te@charactertotalvar=\tikz@lib@dec@te@charactertotalcount%
  \fi%
  %
  \ifx\tikz@lib@dec@te@wordcountvar\pgfutil@empty%
  \else%
    \expandafter\let\tikz@lib@dec@te@wordcountvar=\tikz@lib@dec@te@wordcount%
  \fi%
  %
  \ifx\tikz@lib@dec@te@lettercountvar\pgfutil@empty%
  \else%
    \expandafter\let\tikz@lib@dec@te@lettercountvar=\tikz@lib@dec@te@lettercount%
  \fi%
  %
  \edef\tikz@lib@dec@te@tmp{character \tikz@lib@dec@te@charactercount/.try,%
    character {\meaning\tikz@lib@dec@te@character}/.try}%
  \expandafter\tikz@lib@dec@te@addoptions\expandafter{\tikz@lib@dec@te@tmp}%
  \ifnum\tikz@lib@dec@te@wordcount>0\relax%
    \edef\tikz@lib@dec@te@tmp{every word/.try, word \tikz@lib@dec@te@wordcount/.try}%
    \expandafter\tikz@lib@dec@te@addoptions\expandafter{\tikz@lib@dec@te@tmp}%
    \ifnum\tikz@lib@dec@te@lettercount>0\relax%
      \edef\tikz@lib@dec@te@tmp{every letter/.try,every \tikz@lib@dec@te@lettercount\space letter/.try}%
      \expandafter\tikz@lib@dec@te@addoptions\expandafter{\tikz@lib@dec@te@tmp}%
      \ifnum\tikz@lib@dec@te@lettercount=1\relax%
        \edef\tikz@lib@dec@te@tmp{every first letter/.try}%
        \expandafter\tikz@lib@dec@te@addoptions\expandafter{\tikz@lib@dec@te@tmp}%
      \fi%
      \iftikz@lib@dec@te@finalletter%
        \tikz@lib@dec@te@addoptions{every final letter/.try}%
      \fi%
    \fi%
  \fi%
  \ifx\tikz@lib@dec@te@character\tikz@lib@dec@te@wordsep%
    \tikz@lib@dec@te@addoptions{/pgf/decoration/text effects/every word separator/.try}%
  \fi%
}

% Called by `late positioning' of the node
% to get the width of the bounding box
% of a character.
\def\tikz@lib@dec@te@@getcharacterwidth{%
  \iftikz@lib@dec@te@segmentfromwidth%
    \pgf@x=\pgfpositionnodelatermaxx\relax%
    \advance\pgf@x by-\pgfpositionnodelaterminx\relax%
  \else%
    \pgf@x=\pgfpositionnodelatermaxy\relax%
    \advance\pgf@x by-\pgfpositionnodelaterminy\relax%
  \fi%
  \xdef\tikz@lib@dec@te@characterwidth@tmp{\the\pgf@x}%
}


\def\tikz@lib@dec@te@drawcharacter{%
  \csname tikz@lib@dec@te@character@\tikz@lib@dec@te@charactercount\endcsname%
  \tikz@lib@dec@te@getoptions{every character/.try}%
  \pgfpositionnodelater\relax%
  \pgfutil@ifundefined{tikz@lib@dec@te@character@replacements@\meaning\tikz@lib@dec@te@character @code}{%
    \expandafter\node\expandafter[\tikz@lib@dec@te@options, scale=\tikz@lib@dec@scaledcharacterwidthfactor]{\hbox{\tikz@lib@dec@te@charactercommand{\tikz@lib@dec@te@character}}};%
  }{\tikz@lib@dec@te@drawcharacter@replacement{\tikz@lib@dec@te@character}}%
}

% Replace the character #1 with a character@replacement
%
\def\tikz@lib@dec@te@drawcharacter@replacement#1{%
  \pgfscope%
  \expandafter\tikzset\expandafter{\tikz@lib@dec@te@options}%
  \csname tikz@lib@dec@te@character@replacements@\meaning#1@code\endcsname%
  \endpgfscope%
}

% Get the width of the character@replacement associated with
% the character #1
%
\newbox\tikz@lib@dec@te@box
\def\tikz@lib@dec@te@getcharacter@replacementwidth#1{%
  \pgfinterruptpicture%
  \setbox\tikz@lib@dec@te@box=\hbox{%
  \pgfpicture
  \expandafter\tikzset\expandafter{\tikz@lib@dec@te@options}%
  \csname tikz@lib@dec@te@character@replacements@\meaning#1@code\endcsname%
  \endpgfpicture}%
  \xdef\tikz@lib@dec@te@character@replacementwidth{\the\wd\tikz@lib@dec@te@box}%
  \endpgfinterruptpicture%
}

\def\tikz@lib@dec@scaledcharacterwidthfactor{1}%

% Set up the decoration
\def\tikz@lib@dec@te@state@setup@precomputation{%
  \def\tikz@lib@te@alignshift{0pt}%
  \iftikz@lib@dec@te@pathfromtext%
    % The width of the text and characters has already been calculated.
  \else%
    \let\tikz@lib@dec@te@text=\pgfdecorationtext%
    % Get the chracter tokens and the total number of characters.
    \tikz@lib@dec@te@preparetext%
    \iftikz@lib@dec@te@scaletexttopath%
      % OK, *scaling* text to path is easy.
      % Simply divide the path length by the text width to
      % obtain the scaling factor which will be applied 
      % "at the last minute"...
      \pgfmathdivide{\pgfdecoratedpathlength}{\tikz@lib@dec@te@textwidth}%
      \let\tikz@lib@dec@scaledcharacterwidthfactor=\pgfmathresult%
      \expandafter\pgfkeys\expandafter{\tikz@lib@dec@te@tmp}%
      % ... and recalulate the (pre and post) widths.
      \pgfmathloop
      \ifnum\pgfmathcounter>\tikz@lib@dec@te@charactertotalcount%
      \else%
        \pgf@x=\csname tikz@lib@dec@te@character@\pgfmathcounter @prewidth\endcsname\relax%
        \pgf@x=\tikz@lib@dec@scaledcharacterwidthfactor\pgf@x%                      
        \expandafter\edef\csname tikz@lib@dec@te@character@\pgfmathcounter @prewidth\endcsname{\the\pgf@x}%
        %
        \pgf@x=\csname tikz@lib@dec@te@character@\pgfmathcounter @postwidth\endcsname\relax%
        \pgf@x=\tikz@lib@dec@scaledcharacterwidthfactor\pgf@x%                      
        \expandafter\edef\csname tikz@lib@dec@te@character@\pgfmathcounter @postwidth\endcsname{\the\pgf@x}%
      \repeatpgfmathloop%
      % Update parameters with thewidths.
      \tikz@lib@dec@te@addwidthstoparameters%
      \let\tikz@lib@dec@te@textwidth=\pgfdecoratedpathlength%
    \else%
      \iftikz@lib@dec@te@fittexttopath%
        % *Fitting* text is less easy.
        % The pre and post width of all except the first and last characters 
        % must be recalculated.
        % Only the post width of the first character and the pre width
        % of the final character are scaled as these characters should be
        % fixed and the start and end of the path, respectively.
        %
        % We hope that the decoration engine is accurate enough to position
        % the last word correctly.
        \pgf@x=\pgfdecoratedpathlength%
        \pgf@xa=\csname tikz@lib@dec@te@character@1@postwidth\endcsname\relax%
        \pgf@xb=\csname tikz@lib@dec@te@character@\tikz@lib@dec@te@charactertotalcount @prewidth\endcsname\relax%
        \advance\pgf@x by-\pgf@xa%
        \advance\pgf@x by-\pgf@xb%
        \pgf@y=\tikz@lib@dec@te@textwidth\relax%
        \advance\pgf@y by-\pgf@xa%
        \advance\pgf@y by-\pgf@xb%
        %
        \pgfmathdivide{\the\pgf@x}{\the\pgf@y}%
        \let\tikz@lib@dec@characterwidthscale=\pgfmathresult%      
        \pgfmathloop
        \ifnum\pgfmathcounter>\tikz@lib@dec@te@charactertotalcount\relax%
        \else%
        \ifnum\pgfmathcounter>1\relax%
          \pgf@x=\csname tikz@lib@dec@te@character@\pgfmathcounter @prewidth\endcsname\relax%
          \pgf@x=\tikz@lib@dec@characterwidthscale\pgf@x%  
            \expandafter\edef\csname tikz@lib@dec@te@character@\pgfmathcounter @prewidth\endcsname{\the\pgf@x}%
          \fi%
          \ifnum\pgfmathcounter<\tikz@lib@dec@te@charactertotalcount\relax%
            \pgf@x=\csname tikz@lib@dec@te@character@\pgfmathcounter @postwidth\endcsname\relax%
            \pgf@x=\tikz@lib@dec@characterwidthscale\pgf@x%  
            \expandafter\edef\csname tikz@lib@dec@te@character@\pgfmathcounter @postwidth\endcsname{\the\pgf@x}%
          \fi%
        \repeatpgfmathloop%
        % Update parameters with widths.
        \tikz@lib@dec@te@addwidthstoparameters%
        \let\tikz@lib@dec@te@textwidth=\pgfdecoratedpathlength%
      \else%
        % Calculate the distance to move along the path for alignment.
        \pgfkeysgetvalue{/pgf/decoration/text align/align}{\tikz@lib@dec@align}%
        \ifx\tikz@lib@dec@align\tikz@lib@dec@align@center@text%
           \pgf@x=\pgfdecoratedpathlength\relax%
           \advance\pgf@x by-\tikz@lib@dec@te@textwidth\relax%
           \ifdim\pgf@x<0pt\relax%
             \pgf@x=0pt\relax%
           \fi%
           \divide\pgf@x by2\relax%
           % Adjust the pre-width of the first character...
           \pgf@xa=\csname tikz@lib@dec@te@character@1@prewidth\endcsname\relax%
           \advance\pgf@xa by\pgf@x%
           \expandafter\edef\csname tikz@lib@dec@te@character@1@prewidth\endcsname{\the\pgf@xa}%
           % ...and the post-width of the last character.
           \pgf@xa=\csname tikz@lib@dec@te@character@\tikz@lib@dec@te@charactertotalcount @postwidth\endcsname\relax%
           \advance\pgf@xa by\pgf@x%
           \expandafter\edef\csname tikz@lib@dec@te@character@\tikz@lib@dec@te@charactertotalcount @postwidth\endcsname{\the\pgf@xa}%
        \else%
          \ifx\tikz@lib@dec@align\tikz@lib@dec@align@right@text%
            \pgf@x=\pgfdecoratedpathlength\relax%
            \advance\pgf@x by-\tikz@lib@dec@te@textwidth\relax%
            \ifdim\pgf@x<0pt\relax%
              \pgf@x=0pt\relax%
            \fi%
            % Adjust the pre-width of the first character...
            \pgf@xa=\csname tikz@lib@dec@te@character@1@prewidth\endcsname\relax%
            \advance\pgf@xa by\pgf@x%
            \expandafter\edef\csname tikz@lib@dec@te@character@1@prewidth\endcsname{\the\pgf@xa}%
          \fi%
        \fi%
        \tikz@lib@dec@te@addwidthstoparameters%
      \fi%
    \fi%
  \fi%
}


% Evil hack into the decoration code.
%
% If the path contains single move to and the `path from text'
% key is set to true. The width of the text is calcuated here
% and the (straight line) path automatically calculated.
%
\def\pgf@decorate@path@check@moveto#1{%
  \expandafter\pgf@decorate@path@@check@moveto#1\pgf@decorate@stop\pgf@decorate@@stop}

\def\pgf@decorate@token@stop{\pgf@decorate@stop}%
\def\pgf@decorate@path@@check@moveto#1#2#3#4\pgf@decorate@@stop#5#6{%
  \def\pgf@decorate@temp{#4}%
  \pgf@x=#2\relax%
  \pgf@y=#3\relax%
  \ifx\pgf@decorate@temp\pgf@decorate@token@stop%
     \iftikz@lib@dec@te@pathfromtext%
       % Get the start position...
      \pgfextract@process\tikz@lib@dec@te@pathstart{}%
      \tikz@lib@dec@te@preparetext%
      \tikz@lib@dec@te@addwidthstoparameters%
      % ... and create a lineto path of the correct length.
      \pgfsetpath\pgfutil@empty%
      \pgfpathmoveto{\tikz@lib@dec@te@pathstart}%
      \pgfpathlineto{\pgfpointadd{\tikz@lib@dec@te@pathstart}%
        {\pgfpointpolar{\tikz@lib@dec@te@pathfromtextangle}{\tikz@lib@dec@te@textwidth}}}%
      \pgfgetpath\pgfdecoratedpath%
      \pgfsetpath\pgfutil@empty%
    \else%
      #5%
    \fi%
  \else%
    \tikz@lib@dec@te@pathfromtextfalse%
    #6%
  \fi%
}


\pgfdeclaredecoration{text effects along path}{setup}{
  \state{setup}[width=+0pt, next state=scan, persistent precomputation={%
    \tikz@lib@dec@te@state@setup@precomputation%
    \def\tikz@lib@dec@te@charactercount{0}% 
  }]{}
  %
\state{scan}[width=+0pt, next state=pre token, persistent precomputation={%
  \tikz@lib@dec@te@advancecountmacro\tikz@lib@dec@te@charactercount{1}%
  % Usual switch to final state is based on distance.
  % Here we also need it to be on the number of the 
  % current character.
  \ifnum\tikz@lib@dec@te@charactercount>\tikz@lib@dec@te@charactertotalcount\relax%
    \ifnum\tikz@lib@dec@te@repeattext=0\relax%
      \def\pgf@decorate@next@state{final}%
    \else%
      \tikz@lib@dec@te@advancecountmacro\tikz@lib@dec@te@repeattext{-1}%
      \def\tikz@lib@dec@te@charactercount{1}%
    \fi%
  \fi%
  % Install parameters for this character.
  \csname tikz@lib@dec@te@character@\tikz@lib@dec@te@charactercount\endcsname%
}]{}
%
\state{pre token}[width=+\tikz@lib@dec@te@characterprewidth, next state=token]{}
\state{token}[width=+0pt, next state=post token]
{%
  \tikz@lib@dec@te@drawcharacter%
}
\state{post token}[width=+\tikz@lib@dec@te@characterpostwidth, next state=scan]{}
}


% Parse styles for individual characters
% #1 a string of characters (e.g., aieou{\"U}{\"a"})
% #2 the options for to apply to each character.
%
\def\tikz@lib@dec@te@parse@stylecharacters#1#2{%
  \def\tikz@lib@dec@te@text{#1}%
  \tikz@lib@dec@te@scancharacters{%
    \pgfkeysalso{/pgf/decoration/text effects/character {\meaning\tikz@lib@dec@te@character}/.style={#2}}%
  }%
}

% Parse replacment code for for individual characters
% #1 a string of characters (e.g., aieou{\"U}{\"a"})
% #2 the code to execute to each character.
%
\def\tikz@lib@dec@te@parse@replacecharacters#1#2{%
  \def\tikz@lib@dec@te@text{#1}%
  \tikz@lib@dec@te@scancharacters{%
    \expandafter\def\csname tikz@lib@dec@te@character@replacements@\meaning\tikz@lib@dec@te@character @code\endcsname{#2}%
  }%
}

\endinput

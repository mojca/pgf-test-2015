% Copyright 2008 by Till Tantau and others Wibrow
%
% This file may be distributed and/or modified
%
% 1. under the LaTeX Project Public License and/or
% 2. under the GNU Public License.
%
% See the file doc/generic/pgf/licenses/LICENSE for more details.

\usetikzlibrary{decorations.markings,calc}


%
% Indicate that the picture contains a circuit
%
\tikzset{
  circuit/.style={
    /utils/exec=\tikz@lib@circ@on@tofalse,
    execute at begin to={
      \tikz@lib@circ@on@totrue
      \let\tikz@lib@circ@start@node\pgfutil@empty
      \def\tikz@lib@circ@end{-- (\tikztotarget) \tikztonodes}
    }
  }
}

\newif\iftikz@lib@circ@on@to


%
% General management
%

\tikzset{
  to path with symbol on it/.style={
    to path={
      \pgfextra{\tikz@lib@circ@on@tofalse}
      decorate [decoration={markings,mark connection node=mark node}]
      {
        \tikz@lib@circ@start@node
        \tikz@lib@circ@end
      }
    },
  },
  handle circuit symbol/.code={%
    \global\advance\tikz@lib@circ@count by1\relax%
    \iftikz@lib@circ@on@to
      \pgfkeysalso{to path with symbol on it}
      \ifx\tikz@time\tikz@zero@text%
        \def\tikz@lib@circ@start@node{%
          {\tikz@lib@circ@compute@direction{\tikz@lib@circ@start}%
          node[alias=tikz@lib@circ@node@start,at={(0,0)},#1]{}}(tikz@lib@circ@node@start)}%
      \else%
        \ifx\tikz@time\tikz@one@text%
          \def\tikz@lib@circ@end{
            {\tikz@lib@circ@compute@direction{\tikz@lib@circ@target}%
              node[alias=tikz@lib@circ@node\the\tikz@lib@circ@count,at={(0,0)},#1]{}}
            --(tikz@lib@circ@node\the\tikz@lib@circ@count)\tikztonodes
          }%
        \else%
          \edef\tikz@marshal{mark=at position \ifx\tikz@time\pgfutil@empty0.5\else\tikz@time\fi\space with}%
          \def\tikz@marshala{decoration=}%
          \expandafter\expandafter\expandafter\tikzset%
          \expandafter\expandafter\expandafter{\expandafter\tikz@marshala\expandafter{\tikz@marshal{\node[name=mark node,#1]{};}}}
        \fi%
      \fi%
    \else
      \pgfkeysalso{#1}
    \fi
  }
}
\newcount\tikz@lib@circ@count
\def\tikz@zero@text{0}
\def\tikz@one@text{1}

\def\tikz@lib@circ@compute@direction#1{%
  \pgfextra{%
    \tikz@scan@one@point\tikz@lib@circ@save@start(\tikztostart)%
    \tikz@scan@one@point\tikz@lib@circ@save@target(\tikztotarget)%
    \pgf@process{\pgfpointnormalised{\pgfpointdiff{\tikz@lib@circ@start}{\tikz@lib@circ@target}}}%
    \pgf@ya=-\pgf@y%
    \pgftransformcm{\the\pgf@x}{\the\pgf@y}{\the\pgf@ya}{\the\pgf@x}{#1}%
  }%
}
\def\tikz@lib@circ@save@start#1{\def\tikz@lib@circ@start{#1}}
\def\tikz@lib@circ@save@target#1{\def\tikz@lib@circ@target{#1}}

%
% Labels
%

\tikzset{
  label'/.code={\pgfutil@ifnextchar[\tikz@lib@circ@labp@plain{\tikz@lib@circ@labp@plain[]}#1\pgf@stop},%}
  label sloped/.code={\pgfutil@ifnextchar[\tikz@lib@circ@lab@sloped@plain{\tikz@lib@circ@lab@sloped@plain[]}#1\pgf@stop},%}
  label' sloped/.code={\pgfutil@ifnextchar[\tikz@lib@circ@lab@slopedp@plain{\tikz@lib@circ@lab@slopedp@plain[]}#1\pgf@stop},%}
  circuit declare unit/.style 2 args={
    #1/.code={\pgfutil@ifnextchar[\tikz@lib@circ@lab{\tikz@lib@circ@lab[]}##1\pgf@stop{#2}},%}
    #1 sloped/.code={\pgfutil@ifnextchar[\tikz@lib@circ@lab@sloped{\tikz@lib@circ@lab@sloped[]}##1\pgf@stop{#2}},%}
    #1'/.code={\pgfutil@ifnextchar[\tikz@lib@circ@labp{\tikz@lib@circ@labp[]}##1\pgf@stop{#2}},%} 
    #1' sloped/.code={\pgfutil@ifnextchar[\tikz@lib@circ@lab@slopedp{\tikz@lib@circ@lab@slopedp[]}##1\pgf@stop{#2}}%}
  },
  circuit declare unit={ampere}{A},
  circuit declare unit={volt}{V},
  circuit declare unit={ohm}{\Omega},
  circuit declare unit={siemens}{S},
  circuit declare unit={second}{s},
  circuit declare unit={henry}{H},
  circuit declare unit={farad}{F},
  circuit declare unit={coulomb}{C},
  circuit declare unit={voltampere}{VA},
  circuit declare unit={watt}{W},
  circuit declare unit={hertz}{Hz},
  circuit declare unit={radian}{rad},
  circuit declare unit={radsec}{\frac{\radian}{\second}},
}

\def\tikz@lib@circ@lab[#1]#2\pgf@stop#3{\tikzset{label={[#1]$\mathrm{#2#3}$}}}
\def\tikz@lib@circ@lab@sloped[#1]#2\pgf@stop#3{\tikzset{label={[transform shape,#1]$\mathrm{#2#3}$}}}
\def\tikz@lib@circ@labp[#1]#2\pgf@stop#3{\tikzset{label={[#1]below:$\mathrm{#2#3}$}}}
\def\tikz@lib@circ@lab@slopedp[#1]#2\pgf@stop#3{\tikzset{label={[transform shape,#1]below:$\mathrm{#2#3}$}}}

\def\tikz@lib@circ@lab@sloped@plain[#1]#2\pgf@stop{\tikzset{label={[transform shape,#1]#2}}}
\def\tikz@lib@circ@labp@plainp[#1]#2\pgf@stop{\tikzset{label={[#1]below:#2}}}
\def\tikz@lib@circ@lab@slopedp@plain[#1]#2\pgf@stop{\tikzset{label={[transform shape,#1]below:#2}}}




%
% Basic theming
%

\tikzset{
  every circuit symbol/.style={thick},
  open symbol/.style={draw},
  filled symbol/.style={draw,fill=black},
}



\endinput


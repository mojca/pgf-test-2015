% Copyright 2010 by Till Tantau
%
% This file may be distributed and/or modified
%
% 1. under the LaTeX Project Public License and/or
% 2. under the GNU Public License.
%
% See the file doc/generic/pgf/licenses/LICENSE for more details.

\ProvidesFileRCS[v\pgfversion] $Header: /cvsroot/pgf/pgf/generic/pgf/frontendlayer/tikz/libraries/graphs/tikzlibrarygraphs.code.tex,v 1.1 2010/10/23 16:35:41 tantau Exp $


% 
% Interface keys 
%

\tikzset{
  graphs/new edge directed/.code n args={3}{\path (#1) edge[->,#3] (#2);},
  graphs/new edge undirected/.code n args={3}{\path (#1) edge[-,#3] (#2);},
  graphs/new edge bidirected/.code n args={3}{\path (#1) edge[<->,#3] (#2);},
  graphs/new edge back directed/.style n args={3}{new edge directed={#2}{#1}{#3}},
  graphs/default edge kind/.initial=undirected
}


%
% Keys for using nodes declared outside a graph inside a graph as if
% it were declared there
% 

\tikzset{
  graphs/use existing nodes/.code=\tikz@lib@graph@alltrue
}

\tikzset{
  new set/.code={
    \expandafter\global\expandafter\let\csname tikz@lg@node@set #1\endcsname\pgfutil@empty%
  },
  set/.code={
    \expandafter\def\expandafter\tikz@alias\expandafter{\tikz@alias%
      \expandafter\def\expandafter\pgf@temp\expandafter{\csname tikz@lg@node@set #1\endcsname}%
      \expandafter\expandafter\expandafter\pgfutil@g@addto@macro\expandafter\pgf@temp\expandafter{\expandafter\tikz@lg@do\expandafter{\tikz@fig@name}}%
    }%
  },%
}

\newif\iftikz@lib@graph@all


% 
% Basic options 
%

\tikzset{
  graphs/@nodes styling/.style=,
  graphs/nodes/.style={/tikz/graphs/@nodes styling/.append style={,#1}},
  graphs/@edges styling/.initial=,
  graphs/edges/.style={/tikz/graphs/@edges styling/.append={,#1}},
  graphs/join/.initial=match,
  graphs/connect as/.initial=,
}


% 
% The parser 
%

\def\tikz@lib@graph@parser{%
  \pgfutil@ifnextchar[{\tikz@lib@graph@parser@}{\tikz@lib@graph@parser@[]}%]
}

\long\def\tikz@lib@graph@parser@[#1]#2{%
  \scope[graphs/.cd,every graph/.try,#1]%
    \let\tikz@lib@graph@options\pgfutil@empty%
    \let\tikz@lib@graph@node@list\pgfutil@empty%
    \pgfkeyslet{/tikz/graphs/hints/stack}\pgfutil@empty%
    \tikz@lib@graph@start@hint@group%
      \tikz@lib@graph@parse@group{#2}%
    \tikz@lib@graph@end@hint@group
    \let\tikz@lg@do=\tikz@lib@graph@cleanup%
    \tikz@lib@graph@node@list%
  \endscope%
  \tikz@lib@graph@parser@done%
}




\long\def\tikz@lib@graph@parse@group#1{%
  \let\tikz@lib@graph@group@c\pgfutil@empty%
  \let\tikz@lib@graph@group@cont\pgfutil@empty%
  \let\tikz@lib@graph@group@conta\pgfutil@empty%
  \tikz@lib@graph@par#1\par\pgf@stop@eogroup%
}

\def\tikz@lib@graph@push@hints{
  \pgfkeysgetvalue{/tikz/graphs/hints/stack}\tikz@temp@stack%
  \pgfkeysgetvalue{/tikz/graphs/hints/chain}\tikz@temp@chain%
  \pgfkeysgetvalue{/tikz/graphs/hints/group}\tikz@temp@group%
  \expandafter\expandafter\expandafter\def\expandafter\expandafter\expandafter\tikz@temp@addme%
  \expandafter\expandafter\expandafter{\expandafter\expandafter\expandafter{\expandafter%
      \tikz@temp@chain\expandafter}\expandafter/\expandafter{\tikz@temp@group}}%
  \expandafter\expandafter\expandafter\def%
  \expandafter\expandafter\expandafter\tikz@temp%
  \expandafter\expandafter\expandafter{%
    \expandafter\tikz@temp@stack\expandafter,\tikz@temp@addme}
  \pgfkeyslet{/tikz/graphs/hints/stack}\tikz@temp%
}

\def\tikz@lib@graph@start@hint@group{%
  \pgfkeysgetvalue{/tikz/graphs/hints/group/initial}\tikz@temp%
  \pgfkeyslet{/tikz/graphs/hints/group}\tikz@temp%
}

\def\tikz@lib@graph@end@hint@group{%
  \pgfkeys{/tikz/graphs/hints/last/update from group}%
  \pgfkeysgetvalue{/tikz/graphs/hints/last}\tikz@lib@graph@last@hint%
  \global\let\tikz@lib@graph@last@hint\tikz@lib@graph@last@hint%
}

\def\tikz@lib@graph@hint@aftergroup{%
  \pgfkeyslet{/tikz/graphs/hints/last}\tikz@lib@graph@last@hint%
  \pgfkeys{/tikz/graphs/hints/chain/update}%
}


% 
% Remove \par 
%

\long\def\tikz@lib@graph@par#1\par{%
  \pgfutil@ifnextchar\pgf@stop@eogroup{%
    \expandafter\tikz@lib@graph@encloser\tikz@lib@graph@group@c#1[}{%
    \expandafter\def\expandafter\tikz@lib@graph@group@c\expandafter{\tikz@lib@graph@group@c#1}%
    \tikz@lib@graph@par%
  }%
}


% 
% Replace ...[...]... by ...[{...}]...
% 
\def\tikz@lib@graph@encloser#1[{%
  \pgfutil@ifnextchar\pgf@stop@eogroup{%
    \expandafter\tikz@lib@graph@semi\tikz@lib@graph@group@cont#1;%
  }{%
    \expandafter\def\expandafter\tikz@lib@graph@group@cont\expandafter{\tikz@lib@graph@group@cont#1[}%]
    \tikz@lib@graph@encloser@cont%
  }%
}

\def\tikz@lib@graph@encloser@cont#1]#2[{%
  \pgfutil@ifnextchar\pgf@stop@eogroup{%
    \expandafter\tikz@lib@graph@semi\tikz@lib@graph@group@cont{#1}]#2;}{%
    \expandafter\def\expandafter\tikz@lib@graph@group@cont\expandafter{\tikz@lib@graph@group@cont{#1}]#2[}%
    \tikz@lib@graph@encloser@cont}%
}


% 
% Replace ; by , 
%

\def\tikz@lib@graph@semi#1;{%
  \pgfutil@ifnextchar\pgf@stop@eogroup{%
    \expandafter\tikz@lib@graph@start@group\tikz@lib@graph@group@conta#1,}{%
    \expandafter\def\expandafter\tikz@lib@graph@group@conta\expandafter{\tikz@lib@graph@group@conta#1,}%
    \tikz@lib@graph@semi%
  }%
}



% 
% Start of a group 
%
\def\tikz@lib@graph@start@group{%
  \pgfutil@ifnextchar[\tikz@lib@graph@start@group@opt{\tikz@lib@graph@start@group@opt[]}%]
}

\def\tikz@lib@graph@start@group@opt[#1]{%
  \pgfkeys{/tikz/graphs/.cd,connect as=,every group/.try,#1}%
  \tikz@lib@graph@main@parser%
}


% 
% Main parse 
%

\def\tikz@lib@graph@main@parser#1,{%
  \begingroup%
    \pgfkeysgetvalue{/tikz/graphs/hints/chain/initial}\tikz@temp%
    \pgfkeyslet{/tikz/graphs/hints/chain}\tikz@temp%
    \let\tikz@lib@graph@stored@actions\pgfutil@empty%
    \let\tikz@lib@graph@node@list\pgfutil@empty% reset
    \tikz@lib@graph@parse@one#1-\pgf@stop@eodashes%
}


\def\tikz@lib@graph@parse@one{%
  \pgfutil@ifnextchar\bgroup\tikz@lib@graph@scope\tikz@lib@graph@node%
}




% A normal node

\def\tikz@lib@graph@node#1-{%
  % Detect trailing <
  \tikz@lib@graph@@node#1<\pgf@stop%
}

\def\tikz@lib@graph@@node#1<#2\pgf@stop%
{
  % 
  % #1 will be a node (not a group)
  % 
  % Syntax: node name [options]
  % 
  % Grab node name
  \tikz@lib@graph@grab@name#1[\pgf@stop%
  \tikz@lib@graph@stored@actions%
  \pgfutil@ifnextchar\pgf@stop@eodashes{%
    \tikz@lib@graph@graph@done%
  }{%
    % 
    % Now, get arrow kind 
    % 
    \def\pgf@test{#2}%
    \ifx\pgf@test\pgfutil@empty%
      \expandafter\tikz@lib@graph@no@back@arrow%
    \else%
      \expandafter\tikz@lib@graph@back@arrow%
    \fi%
  }%
}

\def\tikz@lib@graph@no@back@arrow{%
  \pgfutil@ifnextchar>\tikz@lib@graph@forward@arrow{%
    \pgfutil@ifnextchar-\tikz@lib@graph@undirected@arrow{%
      \PackageError{graphs library}{One of the arrow types <-, --, ->, or <-> expected}{}%
      \tikz@lib@graph@undirected@arrow%
    }%
  }%
}

\def\tikz@lib@graph@undirected@arrow-{%
  \def\tikz@lib@graph@arrow@type{undirected}%
  \tikz@lib@graph@after@arrow%
}

\def\tikz@lib@graph@forward@arrow>{%
  \def\tikz@lib@graph@arrow@type{directed}%
  \tikz@lib@graph@after@arrow%
}

\def\tikz@lib@graph@bi@arrow>{%
  \def\tikz@lib@graph@arrow@type{bidirected}%
  \tikz@lib@graph@after@arrow%
}

\def\tikz@lib@graph@back@arrow{%
  \pgfutil@ifnextchar>{\tikz@lib@graph@bi@arrow}{%
    \def\tikz@lib@graph@arrow@type{back directed}%
    \tikz@lib@graph@after@arrow%
  }%
}

\def\tikz@lib@graph@after@arrow{%
  \pgfutil@ifnextchar[{\tikz@lib@graph@after@arrow@opt}{\tikz@lib@graph@after@arrow@opt[]}%]
}

\def\tikz@lib@graph@after@arrow@opt[#1]{%
  % 
  % Ok, first recolor 
  % 
  \pgfkeys{/tikz/graph node/recolor in by=in''}
  \pgfkeys{/tikz/graph node/recolor out by=out'}
  % Save action for next node
  \expandafter\def\expandafter\tikz@lib@graph@stored@actions\expandafter{%
    \expandafter\tikz@lib@graph@joiner\expandafter{\tikz@lib@graph@arrow@type}{#1}}%
  \tikz@lib@graph@parse@one%
}

\def\tikz@lib@graph@joiner#1#2{%
  \pgfkeys{/tikz/graph node/recolor in by=in'}
  \pgfkeys{/tikz/graph node/recolor in'' by=in}
  {%
    \pgfkeyssetvalue{/tikz/graphs/default edge kind}{#1}%
    \pgfkeys{/tikz/graphs/.cd,#2}%
    \pgfkeysgetvalue{/tikz/graphs/join}\pgf@temp%
    \expandafter\tikz@lg@do@connector\expandafter{\pgf@temp}%
  }%
  \pgfkeys{/tikz/graph node/!in',/tikz/graph node/!out'}
}

\def\tikz@lib@graph@graph@done\pgf@stop@eodashes{%
    % Get hints outside
    \pgfkeysgetvalue{/tikz/graphs/hints/chain}\tikz@temp%
    \global\let\tikz@temp@chain@hint\tikz@temp%
    % Get node list outside...
    \expandafter%  
  \endgroup%
  \expandafter\expandafter\expandafter\def%
  \expandafter\expandafter\expandafter\tikz@lib@graph@node@list%
  \expandafter\expandafter\expandafter{\expandafter\tikz@lib@graph@node@list\tikz@lib@graph@node@list}%
  \pgfkeyslet{/tikz/graphs/hints/chain}\tikz@temp@chain@hint%
  \pgfkeys{/tikz/graphs/hints/group/update}%
  \pgfutil@ifnextchar\pgf@stop@eogroup%
  \tikz@lib@graph@graph@group@done%
  \tikz@lib@graph@main@parser%
}

\def\tikz@lib@graph@graph@group@done\pgf@stop@eogroup{%
  \pgfkeysgetvalue{/tikz/graphs/connect as}\tikz@lib@graph@group@connector
  \expandafter\tikz@lg@do@connector\expandafter{\tikz@lib@graph@group@connector}  
}

\def\tikz@lg@do@connector#1{%
  \tikzset{/tikz/connectors/.cd,#1}
}



%
% Handle node
%
\def\tikz@lib@graph@grab@name{%
  \pgfutil@ifnextchar\foreach\tikz@lib@graph@do@foreach\tikz@lib@graph@grab@name@@%
}

\def\tikz@lib@graph@do@foreach\foreach#1in#2#3[\pgf@stop{%
  % Ok, we do a parse on a foreach loop.
  \begingroup
    \let\tikz@lib@graph@node@list@saved\pgfutil@empty%
    \foreach #1 in {#2}%
    {%
      \let\tikz@lib@graph@node@list\tikz@lib@graph@node@list@saved%
      \tikz@lib@graph@parse@group{#3}%
      % TODO: Need to also save hints!
      \global\let\tikz@lib@graph@node@list@saved\tikz@lib@graph@node@list%
    }%
    \expandafter%  
  \endgroup%
  \expandafter\expandafter\expandafter\def%
  \expandafter\expandafter\expandafter\tikz@lib@graph@node@list%
  \expandafter\expandafter\expandafter{\expandafter\tikz@lib@graph@node@list\tikz@lib@graph@node@list@saved}%  
}


\def\tikz@lib@graph@grab@name@@#1[{%
  \tikz@lib@graph@read#1\pgf@stop%
  \let\tikz@lib@graph@as\tikz@lib@graph@as@default%
  \pgfutil@ifnextchar\pgf@stop{%
    \ifx\tikz@lib@graph@name\pgfutil@empty%
    \else
      \expandafter\tikz@lib@graph@noskip%
    \fi%
  }{\tikz@lib@graph@node@opt[}%
}
\def\tikz@lib@graph@noskip{\tikz@lib@graph@node@opt[][}%]

\def\tikz@lib@graph@as@default{%
  \let\tikzgraphnodename\tikz@lib@graph@name@only%
  \let\tikzgraphnodepath\tikz@lib@graph@path%
  \let\tikzgraphnodefullname\tikz@lib@graph@name%
  \tikz@lib@graph@typesetter%
}

\def\tikz@lib@graph@read#1\pgf@stop{%
  \pgfkeys@spdef\tikz@lib@graph@name@only{#1}%
  \edef\tikz@lib@graph@name{\tikz@lib@graph@path\tikz@lib@graph@name@only}%
}


\def\tikz@lib@graph@node@opt[#1]#2[\pgf@stop{%
  \ifx\tikz@lib@graph@name@only\tikz@lib@graph@usetext
    % Ok, make a list of the nodes stored in #1:
    \let\tikz@lg@temp\pgfutil@empty%
    \foreach \tikz@lg@node@name in {#1} {\expandafter\tikz@lib@graph@handle@use\expandafter{\tikz@lg@node@name}}
    % Ok, now add the nodes to the node list
    \expandafter\expandafter\expandafter\def%
    \expandafter\expandafter\expandafter\tikz@lib@graph@node@list%
    \expandafter\expandafter\expandafter{%
      \expandafter\tikz@lib@graph@node@list\tikz@lg@temp}%
    % Then color and initialize them:
    \let\tikz@lg@do\tikz@lib@graph@do@use%
    \tikz@lg@temp%
  \else%
    \expandafter\ifx\csname tikz@lib@graph@def@\tikz@lib@graph@name@only\endcsname\relax%
      {%
        \edef\tikz@lib@graph@node@list{\noexpand\tikz@lg@do{\tikz@lib@graph@name}}%
        \tikz@lg@if@local@node{\tikz@lib@graph@name}%
        {\tikzset{/tikz/graph node/.cd,in,out,#1}}%
        {%
          \tikz@lg@init@color{\tikz@lib@graph@name}{\tikz@lgc@all@true\tikz@lgc@in@true\tikz@lgc@out@true}%
          \iftikz@lib@graph@all%
            \tikzset{/tikz/graph node/.cd,#1}%
          \else%
            \pgfkeys{/tikz/graphs/hints/place}
            \node [name=\tikz@lib@graph@name,/tikz/graph node/.cd,/tikz/graphs/@nodes styling,#1]
              {\tikz@lib@graph@as};% 
          \fi
        }%
      }%
      \pgfkeys{/tikz/graphs/hints/last/update from node/.expand once=\tikz@lib@graph@name}
      \pgfkeys{/tikz/graphs/hints/chain/update}
      \expandafter\expandafter\expandafter\def%
      \expandafter\expandafter\expandafter\tikz@lib@graph@node@list%
      \expandafter\expandafter\expandafter{%
        \expandafter\tikz@lib@graph@node@list\expandafter\tikz@lg@do\expandafter{\tikz@lib@graph@name}}%
    \else
      % The name of the node is a graph name
      \tikz@lib@graph@handle@graph{#1}%      
    \fi
  \fi%  
}

\def\tikz@lib@graph@handle@use#1{%
  % Is #1 the name of a node set?
  \expandafter\let\expandafter\pgf@temp\csname tikz@lg@node@set #1\endcsname
  \ifx\pgf@temp\relax
    \pgfutil@g@addto@macro\tikz@lg@temp{\tikz@lg@do{#1}}    
  \else%
    \expandafter\pgfutil@g@addto@macro\expandafter\tikz@lg@temp\expandafter{\pgf@temp}
  \fi  
}

\def\tikz@lib@graph@do@use#1{%
  \tikz@lg@init@color{#1}{\tikz@lgc@all@true\tikz@lgc@in@true\tikz@lgc@out@true}%
}

\def\tikz@lib@graph@usetext{use}

\def\tikz@lib@graph@typeset@def#1 as #2\pgf@stop{
  \def\tikz@lib@graph@typesetter#1\pgf@stop{#2}
}

\tikzset{
  graphs/typeset/.store in=\tikz@lib@graph@typesetter,
  graphs/math nodes/.style={/tikz/graphs/typeset=$\tikzgraphnodename$},
  graphs/empty nodes/.style={/tikz/graphs/typeset=},
  graphs/typeset=\tikzgraphnodename
}


% 
% Handle scope 
%
\def\tikz@lib@graph@scope#1{
  \begingroup
    \let\tikz@lib@graph@node@list\pgfutil@empty%
    \tikz@lib@graph@push@hints%
    \tikz@lib@graph@start@hint@group%
      \tikz@lib@graph@parse@group{#1}%
    \tikz@lib@graph@end@hint@group%
    \expandafter%  
  \endgroup%
  \expandafter\expandafter\expandafter\def%
  \expandafter\expandafter\expandafter\tikz@lib@graph@node@list%
  \expandafter\expandafter\expandafter{\expandafter\tikz@lib@graph@node@list\tikz@lib@graph@node@list}%
  \tikz@lib@graph@hint@aftergroup%
  \tikz@lib@graph@stored@actions%
  \pgfutil@ifnextchar-{\tikz@lib@graph@scope@minus}{%
    \pgfutil@ifnextchar<{\tikz@lib@graph@scope@less}{%
      \PackageError{graphs library}{One of the arrow types <-, --, ->,
        or <-> expected}{}%
    }%
  }%
}

\def\tikz@lib@graph@scope@minus-{
  \pgfutil@ifnextchar>\tikz@lib@graph@forward@arrow{%
    \pgfutil@ifnextchar-\tikz@lib@graph@undirected@arrow{%
      \pgfutil@ifnextchar\pgf@stop@eodashes\tikz@lib@graph@graph@done{%
        \PackageError{graphs library}{One of the arrow types <-, --, ->, or <-> expected}{}%
        \tikz@lib@graph@undirected@arrow%
      }%
    }%
  }%
}

\def\tikz@lib@graph@scope@less<-{\tikz@lib@graph@back@arrow}%




% 
% Predefining graphs 
% 

\def\tikzdefgraph#1#2{%
  \expandafter\def\csname tikz@lib@graph@def@#1\endcsname{\tikz@lib@graph@do@graph{#2}}%
}

\def\tikz@lib@graph@handle@graph#1{%
  \begingroup%
    \let\tikz@lib@graph@node@list\pgfutil@empty%
    \tikzset{graphs/.cd,#1}%
    \tikz@lib@graph@push@hints%
    \tikz@lib@graph@start@hint@group%
      \csname tikz@lib@graph@def@\tikz@lib@graph@name@only\endcsname%
    \tikz@lib@graph@end@hint@group%
    \expandafter%  
  \endgroup%
  \expandafter\expandafter\expandafter\def%
  \expandafter\expandafter\expandafter\tikz@lib@graph@node@list%
  \expandafter\expandafter\expandafter{\expandafter\tikz@lib@graph@node@list\tikz@lib@graph@node@list}% 
  \tikz@lib@graph@hint@aftergroup%
}

\def\tikz@lib@graph@do@graph#1{%
  \tikz@lib@graph@parse@group{#1}%
}

\let\tikz@lib@graph@path\pgfutil@empty

\tikzset{graphs/name/.code={%
    \edef\tikz@lib@graph@path{#1\space\tikz@lib@graph@path}%
  }%
}


%
% Colors
% 

\tikzset{
  graph node/.unknown/.code={%
    \let\tikz@key\pgfkeyscurrentname% 
    \pgfkeys{/tikz/\tikz@key={#1}}%
  },
  graph node/as/.code=\def\tikz@lib@graph@as{#1}%
}


\tikzset{%
  new graph color/.style={%
    /utils/exec=\expandafter\newif\csname iftikz@lgc@#1@\endcsname,
    graph node/#1/.code={%
      \edef\tikz@lg@col{\expandafter\noexpand\csname tikz@lgc@#1@true\endcsname}%
      \let\tikz@lg@do\tikz@lg@colorize%
      \tikz@lib@graph@node@list%
    },
    graph node/!#1/.code={%
      \edef\tikz@lg@old@col{\expandafter\noexpand\csname tikz@lgc@#1@true\endcsname}%
      \def\tikz@lg@new@col{}%
      \let\tikz@lg@do\tikz@lg@change@color%
      \tikz@lib@graph@node@list%
    },
    graph node/recolor #1 by/.code={%
      \edef\tikz@lg@old@col{\expandafter\noexpand\csname tikz@lgc@#1@true\endcsname}%
      \edef\tikz@lg@new@col{\expandafter\noexpand\csname tikz@lgc@##1@true\endcsname}%
      \let\tikz@lg@do\tikz@lg@change@color%
      \tikz@lib@graph@node@list%
    },
    graph node/not #1/.style=!#1,
  },
  new graph color=in,
  new graph color=in',
  new graph color=in'',
  new graph color=out,
  new graph color=out',
  new graph color=all
}

\def\tikz@lg@init@color#1#2{%
  \expandafter\gdef\csname tikz@lgc@#1\endcsname{#2}%
}

\def\tikz@lib@graph@cleanup#1{%
  \expandafter\global\expandafter\let\csname tikz@lgc@#1\endcsname\relax%
}

\def\tikz@lg@colorize#1{%
  \expandafter\let\expandafter\pgf@temp\csname tikz@lgc@#1\endcsname%
  \expandafter\expandafter\expandafter\def%
  \expandafter\expandafter\expandafter\pgf@temp%
  \expandafter\expandafter\expandafter{%
    \expandafter\tikz@lg@col\pgf@temp}%
  \expandafter\global\expandafter\let\csname tikz@lgc@#1\endcsname\pgf@temp%
}

\def\tikz@lg@change@color#1{%
  \def\tikz@lg@temp@save{#1}%
  \let\tikz@lg@collect\pgfutil@empty%
  \expandafter\let\expandafter\pgf@temp\csname tikz@lgc@#1\endcsname%
  \expandafter\tikz@lg@change@check\pgf@temp\pgf@stop%
}
\def\tikz@lg@change@check#1{%
  \ifx#1\pgf@stop%
    \tikz@lg@change@write@back%
  \else%
    \def\pgf@temp{#1}%
    \ifx\pgf@temp\tikz@lg@old@col%
      \expandafter\tikz@lg@change@add\expandafter{\tikz@lg@new@col}% Found!
    \else%
      \tikz@lg@change@add{#1}%
    \fi%
    \expandafter\tikz@lg@change@check
  \fi%
}
\def\tikz@lg@change@add#1{%
  \expandafter\def\expandafter\tikz@lg@collect\expandafter{\tikz@lg@collect#1}%
}

\def\tikz@lg@change@write@back{%
  \expandafter\global\expandafter\let\csname tikz@lgc@\tikz@lg@temp@save\endcsname\tikz@lg@collect%
}



\def\tikz@lg@if@has@color#1#2#3#4{%
  {%
    \csname tikz@lgc@#1\endcsname%
    \expandafter\let\expandafter\pgf@temp\csname iftikz@lgc@#2@\endcsname%
    \ifx\pgf@temp\relax%
      \tikz@lib@reset@temp%
    \fi%
    \pgf@temp%
      \global\tikz@color@testtrue%
    \else%
      \global\tikz@color@testfalse%
    \fi%
  }%
  \iftikz@color@test#3\else#4\fi%
}
\newif\iftikz@color@test

\def\tikz@lg@if@local@node#1#2#3{\expandafter\ifx\csname tikz@lgc@#1\endcsname\relax#3\else#2\fi}

\def\tikz@lib@reset@temp{\let\pgf@temp\iffalse}



%
% Color functions
%

% Do something for all nodes having a certain color
%
% #1 = the color name
% #2 = a macro
% 
% Description:
% 
% For each node having color #1, the macro #2 will be called. This
% macro should take a single parameter, which will be set 
% to the node's name.

\def\tikzlibgraphforeachcolorednode#1#2{%
  \expandafter\def\expandafter\iftikz@lib@graph@color@picker\expandafter{\csname iftikz@lgc@#1@\endcsname}%
  \let\tikz@lib@graph@action#2%
  \let\tikz@lg@do\tikz@lg@pick%
  \tikz@lib@graph@node@list%  
}
\def\tikz@lg@pick#1{
  {%
    \csname tikz@lgc@#1\endcsname%
    \iftikz@lib@graph@color@picker
      \global\tikz@color@testtrue%
    \else%
      \global\tikz@color@testfalse%
    \fi%
  }%
  \iftikz@color@test\tikz@lib@graph@action{#1}\fi%
}


% Prepare a color 
%
% #1 is the color name
% #2 is a counter
% #3 is a prefix
% 
% Description:
% 
% You can call this function inside a connector. It will do the
% following: First, its counts how many nodes exist that have color
% #1. This number is stored in the counter passed as #2. Furthermore,
% let <name> be the name of the <i>-th vertex that has color #1. Then, a
% macro called \#3<i> will store <name>. For instance, if #1 is "red"
% and the third red node is called foo and if #3 is "bar", then a
% macro called "\bar3" is set to "foo" as if you had said
% "\expandafter\def\csname bar3\endcsname{foo}".
% 
% The bottom line of all this is that after a preparation you can
% easily iterate over nodes having a certain color. If you wish to
% iterate over a single color, it will be quicker and easier to call
% \tikzlibgraphforeachcolorednode, but if you need to iterate over two
% colors simultaneously, it will be better to first prepare the color.

\def\tikzlibgraphpreparecolor#1#2#3{%
  \let\tikz@lib@graph@count#2%
  \tikz@lib@graph@count0\relax
  \def\tikz@lib@graph@prefix{#3}%
  \tikzlibgraphforeachcolorednode{#1}\tikz@lib@graph@prepare%
}

\def\tikz@lib@graph@prepare#1{%
  \advance\tikz@lib@graph@count by1\relax%
  \expandafter\def\csname\tikz@lib@graph@prefix\the\tikz@lib@graph@count\endcsname{#1}%
}




% 
% The bipartite connector 
%

\tikzset{
  connectors/bipartite/.code 2 args={
    \def\tikz@lg@shoreb{#2}%
    \tikzlibgraphforeachcolorednode{#1}\tikz@lib@graph@bipartite@outer
  },
  connectors/bipartite/.default={out'}{in'}
}

\def\tikz@lib@graph@bipartite@outer#1{%
  \def\tikz@lib@graph@from{#1}%
  {%
    \tikzlibgraphforeachcolorednode{\tikz@lg@shoreb}\tikz@lib@graph@bipartite@inner%
  }%
}

\def\tikz@lib@graph@bipartite@inner#1{%
  \def\pgf@temp{#1}%
  \ifx\pgf@temp\tikz@lib@graph@from\else%
    \tikz@lib@graph@default@new@edge{\tikz@lib@graph@from}{#1}%
  \fi%
}

\def\tikz@lib@graph@default@new@edge{%
  \pgfkeysgetvalue{/tikz/graphs/@edges styling}\pgf@temp
  \expandafter\tikz@lib@graph@default@new@edge@\expandafter{\pgf@temp}%
}
\def\tikz@lib@graph@default@new@edge@#1#2#3{%
  \pgfkeys{/tikz/graphs/.cd,new edge \pgfkeysvalueof{/tikz/graphs/default edge kind}={#2}{#3}{#1}}%
}

  
% 
% The clique connector 
%

\tikzset{
  connectors/clique/.code={
    \tikzlibgraphpreparecolor{#1}\c@pgf@counta{tikz@lg}%
    \tikz@lg@clique@loop%
  },
  connectors/clique/.default=all
}

\def\tikz@lg@clique@loop{%
  \ifnum\c@pgf@counta=0\relax%
  \else
    \c@pgf@countb=\c@pgf@counta\relax%
    \tikz@lg@clique@loop@inner%
    \advance\c@pgf@counta by-1\relax%
    \expandafter\tikz@lg@clique@loop%
  \fi%
}

\def\tikz@lg@clique@loop@inner{%
  \advance\c@pgf@countb by-1\relax%
  \ifnum\c@pgf@countb>0\relax%
    \tikz@lib@graph@default@new@edge{\csname tikz@lg\the\c@pgf@counta\endcsname}{\csname tikz@lg\the\c@pgf@countb\endcsname}%
    \expandafter\tikz@lg@clique@loop@inner%
  \fi%
}


% 
% The path connector 
%

\tikzset{
  connectors/path/.code={%
    \let\tikz@lg@prev\relax%
    \tikzlibgraphforeachcolorednode{#1}\tikz@lib@graph@path@do%
  },
  connectors/path/.default=all
}

\def\tikz@lib@graph@path@do#1{%
  \ifx\tikz@lg@prev\relax%
  \else%
    \tikz@lib@graph@default@new@edge{\tikz@lg@prev}{#1}%
  \fi
  \def\tikz@lg@prev{#1}%
}


% 
% The cycle connector 
%

\tikzset{
  connectors/cycle/.code={%
    \let\tikz@lg@prev\relax%
    \let\tikz@lg@first\relax%
    \tikzlibgraphforeachcolorednode{#1}\tikz@lib@graph@cycle@do%
    \ifx\tikz@lg@first\relax%
    \else%
      \tikz@lib@graph@default@new@edge{\tikz@lg@prev}{\tikz@lg@first}%
    \fi%
  },
  connectors/cycle/.default=all
}

\def\tikz@lib@graph@cycle@do#1{%
  \ifx\tikz@lg@prev\relax%
    \def\tikz@lg@prev{#1}%
    \let\tikz@lg@first\tikz@lg@prev%
  \else%
    \tikz@lib@graph@default@new@edge{\tikz@lg@prev}{#1}%
    \def\tikz@lg@prev{#1}%
  \fi%
}




% 
% The match connector 
%

\tikzset{
  connectors/match/.code 2 args={
    {%
      \tikzlibgraphpreparecolor{#1}\c@pgf@counta{tikz@lg}
      \c@pgf@countb=0\relax%
      \let\tikz@lg@prev\relax
      \tikzlibgraphforeachcolorednode{#2}\tikz@lib@graph@match@do%
      \tikz@lib@graph@match@rest%
    }%
  },
  connectors/match/.default={out'}{in'}
}

\def\tikz@lib@graph@match@do#1{%
  \advance\c@pgf@countb by1\relax%
  \ifnum\c@pgf@countb>\c@pgf@counta\relax%
    \c@pgf@countb=\c@pgf@counta\relax%
  \fi%
  \tikz@lib@graph@default@new@edge{\csname tikz@lg\the\c@pgf@countb\endcsname}{#1}%
  \def\tikz@lg@prev{#1}%
}

\def\tikz@lib@graph@match@rest{%
  \ifnum\c@pgf@countb<\c@pgf@counta\relax%
    \advance\c@pgf@countb by1\relax%
    \tikz@lib@graph@default@new@edge{\csname tikz@lg\the\c@pgf@countb\endcsname}{\tikz@lg@prev}%
    \expandafter\tikz@lib@graph@match@rest%
  \fi%
}


% Positioning
%
% It is not the job of the graph library to compute good positions for
% nodes in a graph. However, some basic support is provided for simple
% cases.
% 
% The idea is at follows: Graphs are specified hierarchically. For
% instance, consider the following graph specification:
% 
% graph { a, b, c -> d -> {e -> f -> g, h} -> i, j -> k }
% 
% Here, we have the *group* {e->f->g,h} inside the larger graph
% specification. Each group consists of sequence of *chains* like
% e->f->g or j->k.
% 
% In order to facilitate the automatic positioning of nodes, the graph
% library will provide you with information about the position of
% nodes inside their groups and chains. A general approach is taken
% that allows you to plug in your own favorite placement mechanism.
% 
% The graph system keeps track of what I call *node placement hints*
% for the nodes that have already been positioned.  
%
% Let us start with a single node. Whenever such a node has been
% created, the key "/tikz/graphs/hints/last/update from node" is
% called with the node's name as a parameter. The code stored in this
% key should set the key hints/last to some text like a number of a
% sequence of dimensions like its width and height or the name of the
% node or whatever. This is the "node placement hint" for the node.   
%
% Next, conside a chain of nodes. For each node a placement hint
% can be computed. Each time it has been computed, this information
% is added to the current "chain placement hints". More precisely, the
% key hints/chain/update is called, which should update the current
% value of hints/chain, taking the value of hints/last into account.
% 
% Now, when the next node is created, it can take the value of
% hints/chain into account. For instance, the node
% placement hint might be the width of the node and the chain
% placement hint might always store the sum of the width of the node
% placement hints. Then, we could place a new node at the position
% that is given by this sum.
% 
% Following chains, consider a group, which consists of a list of such
% chains. Similar to the situation with chains and nodes, each time a
% chain is finished, the key hints/group/update is called that should
% update hints/group and it can take hints/chain into account.
% 
% As another example, a placement hint for a node might not only be
% the width of the node, but also its height. Then, the chain
% placement hint might store the sum of the width of the nodes in the
% chain, but also the maximum of the heights of the nodes. The group
% placement hint might now store the sum of the heights, which we
% could use to place each new chain below the previous chain.
% 
% Finally, a whole group may be nested inside a larger graph
% specification. In this case, two things happen:
% 
% First, once a group specification is done, the key
% "hints/last/update from group" is called that should set hints/last
% to some value and takes hints/last/group into account. This
% placement hint should store a placement hint that "represents the
% whole group as if it were a single node".
% 
% Second, when a group starts, the current chain and group placement
% hints are "pushed" onto the current hint stack. This allows you to
% access the current value of the hint stack when computing a good
% placement for the current node. In detail, the keys
% hints/stack stores a comma-separated list of values of the form
% {chain hint}/{group hint} of all parent graph specifications.
% 
% For both the chain hints and the group hints, whenever they are
% reset at the beginning of a chain or group, they are set to the
% current values of hints/chain/initial and hints/group/initial,
% respectively. 

\tikzset{graphs/hints/.cd,
  last/.initial=,
  last/update from node/.code=,
  last/update from group/.code=,
  chain/.initial=,
  chain/initial/.initial=,
  chain/update/.code=,
  group/.initial=,
  group/initial/.initial=,
  group/update/.code=,
  stack/.initial=,
  place/.code=,
}



% A grid-based placement strategy 
% 
% This strategy works as follows: You specify a "chain shift vector"
% and a "group shift vector". Then each new element on a chain is
% shifted by the chain shift vector relative to the previous element
% on the chain. Similarly for each new element of a group. 

\tikzset{graphs/.cd,
  grid placement/.style={
    hints/last/update from node/.code={
      \pgfkeyssetvalue{/tikz/graphs/hints/last}{1/1}
      % A single node has logical unit height and width.
    },
    hints/last/update from group/.code={
      \pgfkeysgetvalue{/tikz/graphs/hints/group}\tikz@temp
      \pgfkeyslet{/tikz/graphs/hints/last}\tikz@temp      
    },
    hints/chain/initial=0/0,
    hints/chain/update/.code={
      \pgfkeysgetvalue{/tikz/graphs/hints/last}\tikz@lib@graph@hint@last
      \pgfkeysgetvalue{/tikz/graphs/hints/chain}\tikz@lib@graph@chain@last
      \expandafter\expandafter\expandafter%
      \tikz@lib@graph@chain@hint@up\expandafter\tikz@lib@graph@hint@last\expandafter/\tikz@lib@graph@chain@last\relax%
    },
    hints/group/initial=0/0,
    hints/group/update/.code={
      \pgfkeysgetvalue{/tikz/graphs/hints/chain}\tikz@lib@graph@chain@last
      \pgfkeysgetvalue{/tikz/graphs/hints/group}\tikz@lib@graph@group@last
      \expandafter\expandafter\expandafter%
      \tikz@lib@graph@group@hint@up\expandafter\tikz@lib@graph@chain@last\expandafter/\tikz@lib@graph@group@last\relax%
    },
    hints/place/.code={%
      \pgfkeysgetvalue{/tikz/graphs/hints/stack}\tikz@temp%
      \expandafter\tikz@lib@graph@place@group@grid\tikz@temp\relax%
      \pgfkeysgetvalue{/tikz/graphs/hints/chain}\tikz@lib@graph@chain@last
      \pgfkeysgetvalue{/tikz/graphs/hints/group}\tikz@lib@graph@group@last
      \expandafter\expandafter\expandafter%
      \tikz@lib@graph@grid@place\expandafter\tikz@lib@graph@chain@last\expandafter/\tikz@lib@graph@group@last\relax%
    }
  },
  chain shift/.initial={(0,1)},
  group shift/.initial={(1,0)}
}

\def\tikz@lib@graph@chain@hint@up#1/#2/#3/#4\relax{%
  \c@pgf@counta=#3\relax%
  \advance\c@pgf@counta by#1\relax%
  \c@pgf@countb=#4\relax%
  \ifnum\c@pgf@countb<#2\relax%
    \c@pgf@countb=#2\relax%
  \fi%
  \edef\tikz@temp{\the\c@pgf@counta/\the\c@pgf@countb}%
  \pgfkeyslet{/tikz/graphs/hints/chain}\tikz@temp%
}

\def\tikz@lib@graph@group@hint@up#1/#2/#3/#4\relax{%
  \c@pgf@counta=#3\relax%
  \ifnum\c@pgf@counta<#1\relax%
    \c@pgf@counta=#1\relax%
  \fi%
  \c@pgf@countb=#4\relax%
  \advance\c@pgf@countb by#2\relax%
  \edef\tikz@temp{\the\c@pgf@counta/\the\c@pgf@countb}%
  \pgfkeyslet{/tikz/graphs/hints/group}\tikz@temp%
}

\def\tikz@lib@graph@grid@place#1/#2/#3/#4\relax{%
  \pgfkeysgetvalue{/tikz/graphs/chain shift}\tikz@temp
  \expandafter\tikz@scan@one@point\expandafter\pgf@process\tikz@temp
  \pgf@process{\pgfpointscale{#1}{}}%
  \pgf@xa=\pgf@x%
  \pgf@ya=\pgf@y%
  \pgfkeysgetvalue{/tikz/graphs/group shift}\tikz@temp
  \expandafter\tikz@scan@one@point\expandafter\pgf@process\tikz@temp
  \pgf@process{\pgfpointscale{#4}{}}%
  \advance\pgf@xa by\pgf@x%
  \advance\pgf@ya by\pgf@y%
  \edef\tikz@lib@graph@shift{(\the\pgf@xa,\the\pgf@ya)}
  \pgfkeys{/tikz/graphs/nodes/.expanded={shift={\tikz@lib@graph@shift}}}
}


\def\tikz@lib@graph@place@group@grid{%
  \pgfutil@ifnextchar\relax{\pgfutil@gobble}{\tikz@lib@graph@place@group@grid@cont}%
}

\def\tikz@lib@graph@place@group@grid@cont,#1/#2{%
  \tikz@lib@graph@grid@place#1/#2\relax%
  \tikz@lib@graph@place@group@grid%
}


\endinput

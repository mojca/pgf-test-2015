\ProvidesPackageRCS $Header: /cvsroot/latex-beamer/pgf/generic/pgf/systemlayer/pgfsys.code.tex,v 1.1 2005/06/11 17:13:10 tantau Exp $

% Copyright 2005 by Till Tantau <tantau@cs.tu-berlin.de>.
%
% This program can be redistributed and/or modified under the terms
% of the GNU Public License, version 2.




%
% The following conversion functions are used to convert from TeX
% dimensions to postscript/pdf points.
% 
{\catcode`\p=12\catcode`\t=12\gdef\Pgf@geT#1pt{#1}}

\def\pgf@sys@tonumber#1{\expandafter\Pgf@geT\the#1}

\def\pgf@sys@bp@correct#1{#1=0.99627#1}

\def\pgf@sys@bp#1{%
  {%
    \pgf@x=#1\relax%
    \pgf@x=0.99627\pgf@x%
    \edef\temp{\expandafter\Pgf@geT\the\pgf@x\space}%
    \toks@\expandafter\expandafter\expandafter{\expandafter\pgfsysprotocol@currentprotocol\temp}%
    \xdef\pgfsysprotocol@currentprotocol{\the\toks@}%
  }%
}

% internal dimensions that are always present when pgsys is loaded.
\newdimen\pgf@x
\newdimen\pgf@y
\newdimen\pgf@xa
\newdimen\pgf@ya
\newdimen\pgf@xb
\newdimen\pgf@yb
\newdimen\pgf@xc
\newdimen\pgf@yc


% The following \if decides, whether even-odd filling and clipping is
% to be used or normal the non-zero winding number rule. See the
% pdf-documentation for details on what these rules are.

\newif\ifpgfsys@eorule



% In the following dummy definitions of the pgf system commands are
% given. The individual drivers must overwrite these definitions,
% preferably all of them. If some system command is used that has not
% been overwritten, the command fails with a warning.
%
% In some cases, some pgf system commands are just shorthands for a
% series of more complicated commands. In this case, a default
% implementation is given in this file, but a driver may choose to
% overwrite this default implementation with a more effecient
% version. For example, a closestroke is the same as a closepath
% followed by a stroking, but in pdf there is a special shorthand for
% this. So, the pdf driver overwrites \pgfsys@closestroke, while the
% postscript driver does not.
%
% The commands for which default implementations are available are
% given at the end.

\def\pgf@sys@fail{\PackageWarning{pgfsys}{An unsupported pgf system was command ignored}}



\def\pgfsys@invoke{\pgf@sys@fail}
% This command is used whenever some literal text needs to be inserted
% into the resulting .pdf, .ps, or .dvi file. Note that most
% \pgfsys@xxxx commands will not directly call this command, but,
% rather, use the protocolling mechanism to record their literal
% text. It is then left to the discretion of the protocolling
% mechansim to insert the literal text at some appropriate, later
% time. 
%
% This command should only be called when \pgfsys@beginpicture has
% been called previously and when the picture has not been suspended
% using \pgfsys@beginhbox.


\def\pgfsys@outerinvoke{\pgfsys@invoke}
% This command can be used to insert protocolled graphic commands
% outside any picture. This is extremely dangerous, but sometimes
% useful. By default, this command does the same as inside pictures. 



%
%
% Path construction commands
%
%

\def\pgfsys@moveto#1#2{\pgf@sys@fail}
% This command is used to start a path at a specific point (#1bp,#2bp)
% or to move the current point of the current path to (#1bp,#2bp)
% without drawing anything upon stroking (the current path is
% `interrupted'). 
%
% Here (and in the following) the coordinates are in TeX's big points
% (bp), unless the coordinate system has been scaled. A big point is
% 1/72 inches. Note that TeX's point pt is 1/72.27 inches.
%
% Example:
%
% \pgfsys@moveto{10}{10}
% \pgfsys@lineto{0}{0}
% \pgfsys@stroke
%
% This will draw a line from (10bp,10bp) to the origin of the picture.

\def\pgfsys@lineto#1#2{\pgf@sys@fail}
% Continue the current path to (#1bp,#2bp) with a line.

\def\pgfsys@curveto#1#2#3#4#5#6{\pgf@sys@fail}
% Continue the current path with a bezier curver to (#5bp,#6bp). The
% control points of the curve are at (#1bp,#2bp) and (#3bp,#4bp).

\def\pgfsys@rect#1#2#3#4{\pgf@sys@fail}
% Append a rectangle to the current path whose lower left corner is at
% (#1bp,#2bp) and whose width/height is given by (#3bp,#4bp).
%
% This command can be `mapped back' to moveto and lineto commands, but
% it is included since pdf has a special, quick version of this command.

\def\pgfsys@closepath{\pgf@sys@fail}
% Close the current path. This results in joining the current point of
% the path with the point specified by the last moveto
% operation. Typically, this is preferable over using lineto to the
% last point specified by a moveto, since the line starting at this
% point and the line ending at this point will be smoothly joined by
% closepath. 
%
% Example:
%
% \pgfsys@moveto{0}{0}
% \pgfsys@lineto{10}{10}
% \pgfsys@lineto{0}{10}
% \pgfsys@closepath
% \pgfsys@stroke
%
% versus
%
% \pgfsys@moveto{0}{0}
% \pgfsys@lineto{10}{10}
% \pgfsys@lineto{0}{10}
% \pgfsys@lineto{0}{0}
% \pgfsys@stroke
% 
% The difference between the above will be that in the second triangle
% the corner at the origin will be wrong; it will just be the overlay
% of two lines going in different directions, not a sharp pointed
% corner.






%
%
% Stroking and filling commands
%
%

\def\pgfsys@stroke{\pgf@sys@fail}
% Stroke the current path (as if it were drawn with a pen). A number
% of so-called graphic state parameters influence this, which can be
% set in different ways:
%
% - linewidth
%   The `thickness' of the line. A width of 0 is the thinnest width
%   renderable on the device. On a high-resolution printer this may
%   become invisible and should be avoided. A good choice is 0.4pt,
%   which is the default.
%
% - stroke color
%   This special color is used for stroking. If it is not set, the
%   current color is used.
%
% - cap
%   The cap describes how the endings of lines a drawn. A round cap
%   adds a little half circle to these endings. A butt cap ends the
%   lines exactly at the end (or start) point without anything
%   added. A rectangular cap ends the lines like the butt cap, but the
%   lines protrude over the endpoint by the line thickness. (See also
%   the pdf or postscript manual). If the path has been closed, no cap
%   is drawn.
%
% - join
%   This describes how a bend (a join) in a path is rendered. A round
%   join draws bends using small arcs. A bevel join just draws the two
%   lines and then fills the join minimally so that it becomes
%   convex. A miter join extends the lines so that they form a single
%   sharp corner, but only up to a certain miter limit. (See the pdf
%   or postscript manual once more).
%
% - dash
%   The line may be dashed according to a dashing pattern.
%
% - clipping area
%   If a clipping area is established, only those parts of the path
%   that are inside the clipping area will be drawn.
%
% In addition to stroking a path, the path may also be used for
% clipping after it has been stroked. This will happen if the
% \pgfsys@clipnext is used prior to this command, see there for
% details.

\def\pgfsys@fill{\pgf@sys@fail}
% This command fills the area surrounded by the current path. If the
% path has not yet been closed, it is closed prior to filling. The
% path itself is not stroked. For self-intersecting paths or paths
% consisting of multiple parts, the nonzero winding number rule is
% used to determine whether a point is inside or outside the
% path, except if \ifpgfsys@eorule holds -- in which case the even-odd
% rule should be used. (See the pdf or postscript manual for details.) 
%
% The following graphic state parameters influence the filling:
%
% - fill color
%   If the fill color is not especially set, the current color is
%   used. 
%
% - clipping area
%   If a clipping area is established, only those parts of the filling
%   area that are inside the clipping area will be drawn.
%
% In addition to filling the path, the path will also be used for
% clipping if \pgfsys@clipnext is used prior to this command.

\def\pgfsys@fillstroke{\pgf@sys@fail}
% First, the path is filled, then the path is stroked. If the fill and
% stroke colors are the same (or if they are not specified and the
% current color is used), this yields almost the same as a
% \pgfsys@fill. However, due to the line thickness of the stroked
% path, the fillstroked area will be slightly larger.
%
% In addition to stroking and filling the path, the path will also be
% used for clipping if \pgfsys@clipnext is used prior to this command.

\def\pgfsys@clipnext{\pgf@sys@fail}
% This command should be issued after a path has been constructed, but
% before it has been stroked and/or filled or discarded. When the
% command is used, the next stroking/filling/discarding command will
% first be executed normally. Then, afterwards, the just-used path
% will be used for subsequent clipping. If there has already been a
% clipping region, this region is intersected with the new clipping
% path (the clipping cannot get bigger). The nonzero winding number
% rule is used to determine whether a point is inside or outside the
% clipping area or the even-odd rule, depending on whether
% \ifpgfsys@eorule holds.

\def\pgfsys@discardpath{\pgf@sys@fail}
% Normally, this command should `throw away' the current path.
% However, after \pgfsys@clipnext has been called, the current path
% should subsequently be used for clipping. See \pgfsys@clipnext for
% details. 





%
%
% Stroking options
%
%
\def\pgfsys@setlinewidth#1{\pgf@sys@fail}
% Sets the width of lines, when stroked, to #1bp.

\def\pgfsys@buttcap{\pgf@sys@fail}
% Sets the cap to a butt cap. See \pgfsys@stroke.

\def\pgfsys@roundcap{\pgf@sys@fail}
% Sets the cap to a round cap. See \pgfsys@stroke.

\def\pgfsys@rectcap{\pgf@sys@fail}
% Sets the cap to a rectangular cap. See \pgfsys@stroke.

\def\pgfsys@miterjoin{\pgf@sys@fail}
% Sets the join to a miter join. See \pgfsys@stroke.

\def\pgfsys@setmiterlimit#1{\pgf@sys@fail}
% Sets the miter limit of lines to #1. See the pdf or postscript
% documentation.

\def\pgfsys@roundjoin{\pgf@sys@fail}
% Sets the join to a round join. See \pgfsys@stroke.

\def\pgfsys@beveljoin{\pgf@sys@fail}
% Sets the join to a bevel join. See \pgfsys@stroke.

\def\pgfsys@setdash#1#2{\pgf@sys@fail}
% Sets the dashing patter. #1 should be a list of lengths given in big
% points (bp) separated by spaces. #2 should be a single value given
% in bp.
%
% The list of values in #1 is used to determine the lengths of the
% `on' phases of the dashing and of the `off' phases. For example, if
% #1 is `3 4', then the dashing pattern is `3bp on followed by 4bp
% off, followed by 3bp on, followed by 4bp off, and so on'. A pattern
% of `.5 4 3 1.5' means `.5bp on, 4bp off, 3bp on, 1.5bp off, .5bp on,
% ...'. If the number of entries is odd, the last one is used twice,
% so `3' means `3bp on, 3bp off, 3bp on, 3bp off,...'. An empty list
% means `always on'.
%
% The second argument determines the `phase' of the pattern. For
% example, for a pattern of `3 4' and a phase of `1', the pattern
% would start: `2bp on, 4bp off, 3bp on, 4bp off, 3bp on, 4bp off...'.



%
%
% Color stuff
%
%

\def\pgfsys@color@rgb@stroke#1#2#3{\pgf@sys@fail}
% Sets the color used for stroking operations to the given rgb tuple
% (numbers between 0 and 1). 
%
% Example: Make stroked text dark red: \pgfsys@color@rgb@stroke{0.5}{0}{0}
%
% The special stroking color is only used if the stroking color has
% been set since the last \color command. Thus, each \color command
% should reset both the stroking and filling colors.

\def\pgfsys@color@rgb@fill#1#2#3{\pgf@sys@fail}
% Sets the color used for filling operations to the given rgb tuple
% (numbers between 0 and 1). This color may be different from the
% stroking color.

\def\pgfsys@color@rgb#1#2#3{\pgfsys@color@rgb@stroke{#1}{#2}{#3}\pgfsys@color@rgb@fill{#1}{#2}{#3}}
% Sets the color used for filling and stroking operations. This is a
% ``low-level'' version of the \color command. It is currently only
% used by the plain tex version to support colors.

\def\pgfsys@color@cmyk@stroke#1#2#3#4{\pgf@sys@fail}
% Sets the color used for stroking operations to the given cymk tuple
% (numbers between 0 and 1). 

\def\pgfsys@color@cmyk@fill#1#2#3#4{\pgf@sys@fail}
% Sets the color used for filling operations to the given cymk tuple
% (numbers between 0 and 1). 

\def\pgfsys@color@cmyk#1#2#3#4{\pgfsys@color@cmyk@stroke{#1}{#2}{#3}{#4}\pgfsys@color@cmyk@fill{#1}{#2}{#3}{#4}}
% Sets the color used for filling and stroking operations.

\def\pgfsys@color@cmy@stroke#1#2#3{\pgf@sys@fail}
% Sets the color used for stroking operations to the given cym tuple
% (numbers between 0 and 1). 

\def\pgfsys@color@cmy@fill#1#2#3{\pgf@sys@fail}
% Sets the color used for filling operations to the given cym tuple
% (numbers between 0 and 1). 

\def\pgfsys@color@cmy#1#2#3{\pgfsys@color@cmy@stroke{#1}{#2}{#3}\pgfsys@color@cmy@fill{#1}{#2}{#3}}
% Sets the color used for filling and stroking operations.

\def\pgfsys@color@gray@stroke#1{\pgf@sys@fail}
% Sets the color used for stroking operations to the given gray level 
% (number between 0 and 1). 

\def\pgfsys@color@gray@fill#1{\pgf@sys@fail}
% Sets the color used for filling operations to the given gray level 
% (number between 0 and 1). 

\def\pgfsys@color@gray#1{\pgfsys@color@gray@stroke{#1}\pgfsys@color@gray@fill{#1}}
% Sets the color used for filling and stroking operations.

\newif\ifpgfsys@color@reset@inorder
% This if decides whether the color resetting of the
% \pgfsys@color@reset command may assume that all color commands are
% given ``in order.'' This means that this command will not be called,
% say, at the beginning of a box that is later inserted at a point
% where a fill color was actively set. This may allow the command to
% produce more compact code when it ``knows'' that no special fill or
% stroke color has been set.
\pgfsys@color@reset@inordertrue


\def\pgfsys@color@reset{}
% This command will be called when the \color command is used. It
% should purge any internal settings of stroking and filling
% color. After this call, till the next use of a command like
% \pgfsys@color@rgb@fill, the current color installed by the \color
% command should be used.

\def\pgfsys@color@unstacked#1{}
% This slightly obscure command causes the color stack to be
% tricked. When called, this command should set the current color to
% #1 without causing any change in the color stack. 



%
%
% Coordinate system transformations
%
%

\def\pgfsys@transformcm#1#2#3#4#5#6{\pgf@sys@fail}
% Perform a concatenation of the low-level current transformation
% matrix with the matrix given by the values #1 to #6. The
% transformation matrix is a transformation on a homogeneous
% 2D-coordinate system. See the pdf or postscript manual for details. 




%
%
% Graphic state saving and restoring.
%
%

\def\pgfsys@beginscope{\pgf@sys@fail}
% Saves the current graphic state on a graphic state stack. All
% changes to the graphic state parameters mentioned for \pgfsys@stroke
% and \pgfsys@fill will be local to the current graphic state and will
% the old values will be restored after endscope is used.
%
% WARNING: pdf and postscript differ with respect to the question of
% whether the current path is part of the graphic state or not. For
% this reason, you should never use this command unless the path is
% currently empty. For example, it might be a good idea to use
% discardpath prior to calling this command. 

\def\pgfsys@endscope{\pgf@sys@fail}
% Restores the last saved graphic state.




%
%
% Special commands
%
%

\def\pgfsys@imagesuffixlist{}
% A list of suffixes, separated by `:', that will be tried when
% searching for an image.



%
%
% Special commands
%
%


\def\pgfsys@definemask{\pgf@sys@fail}
% This command declares a mask for usage with images. It works similar
% to \pgfsys@defineimage: Certain macros are set when the command is
% called. The result should be to set the macro \pgf@mask to a pdf
% object count that can subsequently be used as a soft mask. The
% following macros will be set when this command is invoked:
%
% - \pgf@filename
%   File name of the mask to be defined.
%
% - \pgf@maskmatte
%   The so-called matte of the mask (see the pdf documentation for
%   details). The matte is a color specification consisting of 1, 3 or
%   4 numbers between 0 and 1. The number of numbers depends on the
%   number of color channels in the image (not in the mask!). It will
%   be assumed that the image has been preblended with this color.




%
%
% Shading commands
%
%

\def\pgfsys@horishading#1#2#3{\pgf@sys@fail}
% Declares a horizontal shading for later use. The effect of this
% command should be the definition of a macro called `@pgfshading#1!'
% (or \csname @pdfshading#1!\endcsname, to be precise). When invoked,
% this new macro should insert a shading at the current position.
%
% #1 is the name of the shading, which is also used in the output
% macro name. #2 is the height of the shading and must be given as a
% TeX dimension like `2cm' or `10pt'. #3 is a shading color
% specification as specified in the pgfuserguide. The shading
% specification implicitly fixes the width of the shading. 
%
% When `@pgfshading#1!' is invoked, it should insert a box of height
% #2 and the width implicit in the shading declaration.

\def\pgfsys@vertshading#1#2#3{\pgf@sys@fail}
% Like horishading, only for vertical shadings. This time, the height
% of the shading is implicit in #3 and the width is given as #2.

\def\pgfsys@radialshading#1#2#3{\pgf@sys@fail}
% Declares a radial shading. Like the previous macros, this command
% should setup the macro `@pgfshading#1!', which upon invocation
% should insert a radial shading whose size is implicit in #3. 
%
% The parameter #2 is a pgf point specification if the starting point
% of the shading. See the pgfuserguide for details.












%
%
% Commands having a default implementation. These commands only need
% not be implemented by drivers if this results in a better
% preformance. 
%
%


\def\pgfsys@closestroke{\pgfsys@closepath\pgfsys@stroke}
% This command should have the same effect as first closing the path
% and then stroking it.


\def\pgfsys@transformshift#1#2{\pgfsys@transformcm{1}{0}{0}{1}{#1}{#2}}
% This command will change the origin of the low-level coordinate
% system to (#1bp,#2bp).

\def\pgfsys@transformxyscale#1#2{\pgfsys@transformcm{#1}{0}{0}{#2}{0bp}{0bp}}
% This command will scale the low-level coordinate system (and
% everything that is drawn) by a factor of #1 in x-directed and #2 in
% y-direction. Note that this applies to everything, including
% lines. So a scaled line will have a different width and may even
% have a different width when going along the x-axis and when going
% along the y-axis, if the scaling is different in these directions.

\def\pgfsys@beginhbox{\pgfsys@beginscope}
% Called before a TeX hbox is typeset inside a pgfpicture. By default,
% this just opens a scope.

\def\pgfsys@endhbox{\pgfsys@discardpath\pgfsys@endscope}
% Called after a TeX hbox has been typeset inside a pgfpicture. By
% default, this discards the path and closes the scope.

\def\pgfsys@beginpicture{\pgfsys@beginscope}
% Called at the beginning of a pgfpicture. By default,
% this just opens a scope.

\def\pgfsys@endpicture{\pgfsys@discardpath\pgfsys@endscope}
% Called at the end of a pgfpicture.  By
% default, this discards the path and closes the scope.

\def\pgfsys@beginpurepicture{\pgfsys@beginpicture}
% This version of the beginpicture command can be used for pictures
% that are guaranteed not to contain any hboxes. In this case, a
% driver might provide a more compact version of the command.

\def\pgfsys@endpurepicture{\pgfsys@endpicture}
% Called at the end of a ``pure'' pgfpicture.

\def\pgfsys@defineimage{%
  \ifx\pgf@imagewidth\@empty\else\edef\pgf@imagewidth{width=\pgf@imagewidth,}\fi%
  \ifx\pgf@imageheight\@empty\else\edef\pgf@imageheight{height=\pgf@imageheight,}\fi%
  \ifx\pgf@imagepage\@empty\else\edef\pgf@imagepage{page=\pgf@imagepage,}\fi%
  \edef\pgf@image{\noexpand\includegraphics[\pgf@imageheight\pgf@imagewidth]{\pgf@filename}}%
}
% Called, when an image should be defined. 
%
% This command does not take any parameters. Instead, certain macros
% will be preinstalled with appropriate values when this command is
% invoked. These are:
%
% - \pgf@filename
%   File name of the image to be defined.
%
% - \pgf@imagewidth
%   Will be set to the desired (scaled) width of the image.
%
% - \pgf@imageheight
%   Will be set to the desired (scaled) height of the image.
%
%   If this macro and also the height macro are empty, the image
%   should have its `natural' size.
%
%   If exactly only of them is specified, the undefined value the
%   image is scaled so that the aspect ratio is kept.
%
%   If both are set, the image is scaled in both directions
%   independently, possibly changing the aspect ratio.
%
%
% The following macros presumable mostly make sense for drivers that
% can handle pdf: 
%
% - \pgf@imagepage
%   The desired page number to be extracted from a multi-page
%   `image'. 
%
% - \pgf@imagemask
%   If set, it will be set to `/SMask x 0 R', where x is the pdf
%   object number of a soft mask to be applied to the image.
%
% - \pgf@imageinterpolate
%   If set, it will be set to `/Interpolate true' or  `/Interpolate
%   false', indicating whether the image should be interpolated in
%   pdf. 
%
%
% The command should now setup the macro \pgf@image such that calling
% this macro will result in typesetting the image. Thus, \pgf@image is
% the `return value' of the command.


\def\pgfsys@defobject#1#2#3#4{%
  \pgfsysprotocol@getcurrentprotocol\pgfsys@temp%
  {%
    \pgfsysprotocol@setcurrentprotocol\@empty%
    \pgfsysprotocol@bufferedtrue%
    #4%
    \pgfsysprotocol@getcurrentprotocol\pgfsys@@temp%
    \expandafter\global\expandafter\let\csname #1\endcsname=\pgfsys@@temp%
  }%
  \pgfsysprotocol@setcurrentprotocol\pgfsys@temp%
}
% Declares an object for later use. The idea is that the object can be
% precached in some way and then be rendered more quickly when used
% several times. For example, an arrow head might be defined and
% prerendered in this way.
%
% The parameter #1 is the name for later use. #2 and #3 are pgf points
% specifying a bounding box for the object. #4 is the code for the
% object. The code may only contain protocollable code.


\def\pgfsys@useobject#1#2{%
  \pgfsysprotocol@getcurrentprotocol\pgfsys@temp%
  {%
    \pgfsysprotocol@setcurrentprotocol\@empty%
    \pgfsysprotocol@bufferedfalse%
    #2%
    \expandafter\pgfsysprotocol@setcurrentprotocol\csname #1\endcsname%
    \pgfsysprotocol@invokecurrentprotocol%
  }%
  \pgfsysprotocol@setcurrentprotocol\pgfsys@temp%
}
% Renders a previously declared object. The first parameter is the
% name of the the object. The second parameter is extra code that
% should be executed right *before* the object is rendered. Typically,
% this will be some transformation code. 


\def\pgfsys@begininvisible{\pgfsys@transformcm{1}{0}{0}{1}{2000bp}{2000bp}}
% Between this command and the closing endinvisible, all output should
% be suppressed. Nothing should be drawn at all, which includes all
% paths, images and shadings.

\def\pgfsys@endinvisible{\pgfsys@transformcm{1}{0}{0}{1}{-2000bp}{-2000bp}}
% Ends the invisibilty section, unless invisibility blocks have been
% nested. In this case, only the `last' one restores visibility.




% Discern the driver:
\input pgf.cfg

\PackageInfo{pgfsys}{Driver file for pgf: \pgfsysdriver}
\InputIfFileExists{\pgfsysdriver}{}{%
  \PackageError{pgfsys}{Driver file ``\pgfsysdriver'' not found.}{}}

\endinput

% Copyright 2006 by Till Tantau
%
% This file may be distributed and/or modified
%
% 1. under the LaTeX Project Public License and/or
% 2. under the GNU Public License.
%
% See the file doc/generic/pgf/licenses/LICENSE for more details.

\ProvidesFileRCS[v\pgfversion] $Header: /cvsroot/pgf/pgf/generic/pgf/libraries/pgflibraryshapes.multipart.code.tex,v 1.1 2007/06/08 11:24:59 tantau Exp $

\newbox\pgfnodepartlowerbox

%
% A circle that is split in the middle into an upper and a lower part.
%
% This node consists of two parts: The upper (main) part is shown in
% the upper half of the circle. The second part is the (optional)
% lower part.
%
% Parts: text, lower

\pgfdeclareshape{circle split}
{
  %
  % Node parts
  %
  \nodeparts{text,lower}

  %
  % Anchors
  %
  \savedanchor\centerpoint{%
    \pgf@x=.5\wd\pgfnodeparttextbox%
    \pgfmathsetlength{\pgf@y}{\pgfkeysvalueof{/pgf/inner ysep}}%
    \pgf@y=-\pgf@y%
    \advance\pgf@y by-\dp\pgfnodeparttextbox%
    \advance\pgf@y by-.5\pgflinewidth%
  }%
  \savedanchor\loweranchor{%
    \pgf@x=-.5\wd\pgfnodepartlowerbox%
    \advance\pgf@x by.5\wd\pgfnodeparttextbox%
    \pgfmathsetlength{\pgf@y}{\pgfkeysvalueof{/pgf/inner ysep}}%
    \pgf@y=-2\pgf@y%
    \advance\pgf@y by-\ht\pgfnodepartlowerbox%
    \advance\pgf@y by-.5\pgflinewidth%
    \advance\pgf@y by-\dp\pgfnodeparttextbox%
    \advance\pgf@y by-.5\pgflinewidth%
  }
    
  \saveddimen\radius{%
    % 
    % Caculate ``height radius''
    % 
    \pgf@ya=.5\ht\pgfnodeparttextbox%
    \advance\pgf@ya by.5\dp\pgfnodeparttextbox%
    \advance\pgf@ya by.5\ht\pgfnodepartlowerbox%
    \advance\pgf@ya by.5\dp\pgfnodepartlowerbox%
    \advance\pgf@ya by.5\pgflinewidth%
    \pgfmathsetlength\pgf@yb{\pgfkeysvalueof{/pgf/inner ysep}}%
    \advance\pgf@ya by2\pgf@yb%
    % 
    % Caculate ``width radius''
    % 
    \pgf@xa=.5\wd\pgfnodeparttextbox%
    \ifdim\pgf@xa<.5\wd\pgfnodepartlowerbox%
      \pgf@xa=.5\wd\pgfnodepartlowerbox%
    \fi%
    \pgfmathsetlength\pgf@xb{\pgfkeysvalueof{/pgf/inner xsep}}%
    \advance\pgf@xa by\pgf@xb%
    % 
    % Calculate length of radius vector:
    % 
    \pgf@process{\pgfpointnormalised{\pgfqpoint{\pgf@xa}{\pgf@ya}}}%
    \ifdim\pgf@x>\pgf@y%
        \c@pgf@counta=\pgf@x%
        \ifnum\c@pgf@counta=0\relax%
        \else%
          \divide\c@pgf@counta by 255\relax%
          \pgf@xa=16\pgf@xa\relax%
          \divide\pgf@xa by\c@pgf@counta%
          \pgf@xa=16\pgf@xa\relax%
        \fi%
      \else%
        \c@pgf@counta=\pgf@y%
        \ifnum\c@pgf@counta=0\relax%
        \else%
          \divide\c@pgf@counta by 255\relax%
          \pgf@ya=16\pgf@ya\relax%
          \divide\pgf@ya by\c@pgf@counta%
          \pgf@xa=16\pgf@ya\relax%
        \fi%
    \fi%
    \pgf@x=\pgf@xa%
    % 
    % If necessary, adjust radius so that the size requirements are
    % met: 
    % 
    \pgfmathsetlength{\pgf@xb}{\pgfkeysvalueof{/pgf/minimum width}}%  
    \pgfmathsetlength{\pgf@yb}{\pgfkeysvalueof{/pgf/minimum height}}%  
    \ifdim\pgf@x<.5\pgf@xb%
        \pgf@x=.5\pgf@xb%
    \fi%
    \ifdim\pgf@x<.5\pgf@yb%
        \pgf@x=.5\pgf@yb%
    \fi%
    % 
    % Now, add larger of outer sepearations.
    % 
    \pgfmathsetlength{\pgf@xb}{\pgfkeysvalueof{/pgf/outer xsep}}%  
    \pgfmathsetlength{\pgf@yb}{\pgfkeysvalueof{/pgf/outer ysep}}%  
    \ifdim\pgf@xb<\pgf@yb%
      \advance\pgf@x by\pgf@yb%
    \else%
      \advance\pgf@x by\pgf@xb%
    \fi%
  }

  %
  % Anchors
  % 
  \inheritanchorborder[from=circle]
  \inheritanchor[from=circle]{north}
  \inheritanchor[from=circle]{north west}
  \inheritanchor[from=circle]{north east}
  \inheritanchor[from=circle]{center}
  \inheritanchor[from=circle]{west}
  \inheritanchor[from=circle]{east}
  \inheritanchor[from=circle]{mid}
  \inheritanchor[from=circle]{mid west}
  \inheritanchor[from=circle]{mid east}
  \inheritanchor[from=circle]{base}
  \inheritanchor[from=circle]{base west}
  \inheritanchor[from=circle]{base east}
  \inheritanchor[from=circle]{south}
  \inheritanchor[from=circle]{south west}
  \inheritanchor[from=circle]{south east}
  \anchor{lower}{\loweranchor}

  %
  % Background path
  %
  \inheritbackgroundpath[from=circle]
  \beforebackgroundpath{
    \pgfutil@tempdima=\radius%
    \pgfmathsetlength{\pgf@xb}{\pgfkeysvalueof{/pgf/outer xsep}}%  
    \pgfmathsetlength{\pgf@yb}{\pgfkeysvalueof{/pgf/outer ysep}}%  
    \ifdim\pgf@xb<\pgf@yb%
      \advance\pgfutil@tempdima by-\pgf@yb%
    \else%
      \advance\pgfutil@tempdima by-\pgf@xb%
    \fi%
    \advance\pgfutil@tempdima by-.5\pgflinewidth%  
    \pgfsetshortenstart{0pt}%
    \pgfsetshortenend{0pt}%
    \pgfsetarrows{-}%  
    \pgfpathmoveto{\pgfpointadd{\centerpoint}{\pgfqpoint{-1\pgfutil@tempdima}{0pt}}}%
    \pgfpathlineto{\pgfpointadd{\centerpoint}{\pgfqpoint{\pgfutil@tempdima}{0pt}}}%
    \pgfusepath{stroke}%
  }
}



\endinput

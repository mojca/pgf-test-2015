% Copyright 2006 by Till Tantau
%
% This file may be distributed and/or modified
%
% 1. under the LaTeX Project Public License and/or
% 2. under the GNU Public License.
%
% See the file doc/generic/pgf/licenses/LICENSE for more details.

\ProvidesFileRCS[v\pgfversion] $Header: /cvsroot/pgf/pgf/generic/pgf/libraries/pgflibrarytikzcalendar.code.tex,v 1.4 2007/02/05 11:23:22 tantau Exp $


\pgfutil@usemodule{pgfcalendar}


%
% General options (these options affect all calendars typeset unsind \calendar):
%
% Start, end, and code executed for each current date
%

\tikzoption{start date}{\def\tikz@lib@cal@start{#1}}
\tikzoption{end date}{\def\tikz@lib@cal@end{#1}}
\tikzoption{execute at begin day}{\expandafter\def\expandafter\tikz@atbegin@day\expandafter{\tikz@atbegin@day#1}}
\tikzoption{execute at end day}{\def\pgf@temp{#1}\expandafter\expandafter\expandafter\def\expandafter\expandafter\expandafter\tikz@atend@day\expandafter\expandafter\expandafter{\expandafter\pgf@temp\tikz@atend@day}}

\let\tikz@atbegin@day=\pgfutil@empty
\let\tikz@atend@day=\pgfutil@empty



% 
% Options affecting some calendars:
%

% General day offset

\tikzoption{day offset}{\def\tikz@lib@cal@offset{#1}}


% Templates for typesetting days, month, years

\tikzoption{day template}{\def\tikzdaytemplate{#1}}
\tikzoption{month template}{\def\tikzmonthtemplate{#1}}
\tikzoption{year template}{\def\tikzyeartemplate{#1}}


% Day sequence

\tikzstyle{day sequence}=[%
  execute at begin day={\tikzstyle{day}=[]},
  execute at end day={%
    \tikzdaytemplate%
    \expandafter\tikz@scan@one@point\expandafter\pgftransformshift\tikz@lib@cal@offset\relax%
  },
  day template={\node[anchor=base east,name=\pgfcalendarsuggestedname,style=day]{\pgfcalendarnoleadingzero{\pgfcalendarcurrentday}};}
]

\tikzstyle{day sequence down}=[day sequence,day offset={(0pt,-1.2em)}]
\tikzstyle{month end markers}=[execute at end day={
   \ifdate{day of month=1,equals=\pgfcalendarbeginiso}
   {\draw [thick,cap=round,black!50] (-3pt,0.9em) -- (-3em,0.9em);}{}
}]




%
% Code of the actual \calendar command (tikz.code.tex contains \let\calendar=\tikz@lib@cal@calendar):
%

\def\tikz@lib@cal@calendar{%
  \begingroup%
    \let\tikz@lib@cal@ifs=\pgfutil@empty%
    \tikz@expandcount=1000\relax%
    \tikz@setkeys{name=,at={(0,0)}}%
    \tikz@lib@cal@scanner%
}

\def\tikz@lib@cal@scanner{%
  \afterassignment\tikz@lib@cal@handle\let\@let@token=%
}

\def\tikz@lib@cal@handle{%
  \let\@next=\tikz@lib@cal@expand%
  \ifx\@let@token;%
    \let\@next=\tikz@lib@cal@stop%
  \else%
    \ifx\@let@token(%)
      \let\@next=\tikz@lib@cal@name%
    \else%
      \ifx\@let@token a%
        \let\@next=\tikz@lib@cal@at%
      \else%
        \ifx\@let@token[%
          \let\@next=\tikz@lib@cal@option%
        \else%
          \ifx\@let@token i%
            \let\@next=\tikz@lib@cal@if%
          \fi%
        \fi%
      \fi%
    \fi%
  \fi%
  \@next%
}
\def\tikz@lib@cal@expand{%
  \advance\tikz@expandcount by -1%
  \ifnum\tikz@expandcount<0\relax%
    \PackageError{tikz}{Giving up on this calendar}{}%
    \let\@next=\tikz@lib@cal@end%
  \else%
    \let\@next=\tikz@lib@cal@@expand
  \fi%
  \@next}

\def\tikz@lib@cal@@expand{\expandafter\tikz@lib@cal@scanner\@let@token}


\def\tikz@lib@cal@name#1){%
  \tikz@setkeys{name=#1}%
  \tikz@lib@cal@scanner%
}
\def\tikz@lib@cal@at t#1(#2){%
  \tikz@setkeys{at={(#2)}}%
  \tikz@lib@cal@scanner%
}
\def\tikz@lib@cal@option#1]{%
  \tikz@setkeys{#1}%
  \tikz@lib@cal@scanner%
}
\long\def\tikz@lib@cal@if f#1(#2)#3{%
  \pgfutil@ifnextchar e{\tikz@lib@cal@else{#2}{#3}}%
  {\expandafter\def\expandafter\tikz@lib@cal@ifs\expandafter{\tikz@lib@cal@ifs\ifdate{#2}{#3}{}}\tikz@lib@cal@scanner}%
}
\long\def\tikz@lib@cal@else#1#2else#3{%
  \expandafter\def\expandafter\tikz@lib@cal@ifs\expandafter{\tikz@lib@cal@ifs\ifdate{#1}{#2}{#3}}%
  \tikz@lib@cal@scanner%
}
\def\tikz@lib@cal@stop{%
    \pgftransformshift{\tikz@node@at}%
    \expandafter\pgfcalendar\expandafter{\tikz@fig@name}{\tikz@lib@cal@start}{\tikz@lib@cal@end}%
    {%
      \tikz@atbegin@day%
      \tikz@lib@cal@ifs%
      \tikz@atend@day%
    }%
  \endgroup%
}




\endinput

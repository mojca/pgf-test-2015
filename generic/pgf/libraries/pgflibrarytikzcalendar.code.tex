% Copyright 2006 by Till Tantau
%
% This file may be distributed and/or modified
%
% 1. under the LaTeX Project Public License and/or
% 2. under the GNU Public License.
%
% See the file doc/generic/pgf/licenses/LICENSE for more details.

\ProvidesFileRCS[v\pgfversion] $Header: /cvsroot/pgf/pgf/generic/pgf/libraries/pgflibrarytikzcalendar.code.tex,v 1.2 2007/02/02 13:21:21 tantau Exp $



% Typeset a calendar
%
% #1 = prefix for the calendar nodes
% #2 = start date in ISO format
% #3 = end date in ISO format
% #4 = anchor date (start date is used when empty)
% #5 = date pass check code
% #6 = date rendering code
%
% Description:
%
% A calendar is typeset as follows: You specify a range of dates (by
% providing a start and an end date). Then, for each date in this
% range two things happen: First, a test is applied whether the date
% should be considered. If this test fails, the next date is
% used. Otherwise, a function is called that gets the date and lots
% of information about the date (like its day of week) and information
% about an anchor date. The job of this funtion is to compute a
% position where the date should be put (it must setup a
% transformation matrix) and to put the date there. Using the
% information about the date, the funtion is invited to change graphic
% parameters as needed.
%
% Inside the check and rendering code, different macros will be setup
% an can be access:
%
% 

\def\pgfcalendar#1#2#3#4#5#6{%
}





%
%
% Date conversion functions
%
%


% Translation stuff

\ifx\translate\@undefined
  \def\translate#1{#1}
\fi

% Load month dictionary, if possible

\ifx\usedictionary\@undefined
\else
  \usedictionary{translator-months-dictionary}
\fi


% Convert a date to the Julian day number (number of days since
% January 1st, -4712)
%
% #1 = date in the ISO format (like 2006-11-10)
% #2 = counter that should be set to the number of days
%
% Description:
%
% The conversion is taken from the Wikipedia entry on Julian days. 
%
% Example:
%
% \pgfcalendardatetojulian{2006-01-10}{\mycount}

\def\pgfcalendardatetojulian#1#2{\edef\pgf@temp{#1}\expandafter\pgfcalendar@datetojulian\pgf@temp/{#2}}
\def\pgfcalendar@datetojulian#1-#2-#3/#4{%
  {%
    %
    % Store year, month and days.
    % 
    \count1=#1\relax%
    \count2=#2\relax%
    \count3=#3\relax%
    %
    % Store final result in \c@pgf@countd
    %
    %
    % 4) a = \lfloor (14-month) /12 \rfloor
    %
    \ifnum\count2<3\relax%
      \count4=1\relax%
    \else%
      \count4=0\relax%
    \fi%
    %
    % 5) y = year + 4800 - a
    %
    \count5=\count1\relax%
    \advance\count5 by 4800\relax%
    \advance\count5 by-\count4\relax%
    %
    % 6) m = month + 12a - 3
    %
    \count6=#2\relax%
    \count0=\count4\relax%
    \multiply\count0 by12\relax
    \advance\count6 by\count0\relax%
    \advance\count6 by-3\relax%
    %
    % 7) jdn = day + \floor{(153 m+2)/5} + 365y + \floor{y/4} -
    % \floor{y/100} + \florr{y/400} - 32045
    %
    \count7=\count3\relax%
    % + \floor{(153 m+2)/5} :
    \count0=\count6\relax
    \multiply\count0 by 153\relax%
    \advance\count0 by 2\relax%
    \divide\count0 by 5\relax%
    \advance\count7 by \count0\relax%
    % + 365y :
    \count0=\count5%
    \multiply\count0by365\relax%
    \advance\count7 by\count0\relax%
    % + \floor{y/4}
    \count0=\count5\relax%
    \divide\count0 by 4\relax%
    \advance\count7 by\count0%
    % - \floor{y/100}
    \count0=\count5\relax%
    \divide\count0 by 100\relax%
    \advance\count7 by-\count0\relax%
    % + \floor{y/400}
    \count0=\count5\relax%
    \divide\count0 by 400\relax%
    \advance\count7 by\count0%
    %  - 32045
    \advance\count7 by-32045\relax%
    %
    \xdef\pgf@temp{\the\count7}%
  }%
  #4=\pgf@temp\relax%
}


% Convert Julian day number date. 
%
% #1 = the number of Julian days
% #2 = a macro in which the year should be stored.
% #3 = a macro in which the month should be stored.
% #4 = a macro in which the day should be stored.
%
% Formula used:
%
% 1) J = Julian day number
% 2) j = J + 32044
% 3) g = j div 146097
% 4) dg = j mod 146097
% 5) c = (dg div 36524 + 1) * 3 div 4
% 6) dc = dg - c * 36524
% 7) b = dc div 1461
% 8) db = dc mod 1461
% 9) a = (db div 365 + 1) * 3 div 4
% 10)da = db - a * 365
% 11)y = g * 400 + c * 100 + b * 4 + a
% 12)m = (da * 5 + 308) div 153 - 2
% 13)d = da - (m + 4) * 153 div 5 + 122
% 14)Y = y - 4800 + (m + 2) div 12
% 15)M = (m + 2) mod 12 + 1
% 16)D = d + 1
%
% Example
%
% \pgfcalendarjuliantodate{\mynumber}{\myyear}{\mymonth}{\myday}

\def\pgfcalendarjuliantodate#1#2#3#4{%
  {%
    % 1) J = Julian day number
    \count1=#1\relax%
    % 2) j = J + 32044     
    \count2=\count1\relax%
    \advance\count2 by 32044\relax%
    % 3) g = j div 146097  
    \count3=\count2\relax%
    \divide\count3 by 146097\relax%
    % 4) dg = j mod 146097 
    \count4=\count3\relax%
    \multiply\count4 by-146097\relax%
    \advance\count4 by\count2\relax%
    % 5) c = (dg div 36524 + 1) * 3 div 4 
    \count5=\count4\relax%
    \divide\count5 by36524\relax%
    \advance\count5 by1\relax%
    \multiply\count5 by3\relax%
    \divide\count5 by4\relax%
    % 6) dc = dg - c * 36524 
    \count6=\count4\relax%
    \count0=\count5\relax%
    \multiply\count0 by-36524\relax%
    \advance\count6 by\count0\relax%
    % 7) b = dc div 1461
    \count7=\count6\relax%
    \divide\count7 by1461\relax%
    % 8) db = dc mod 1461
    \count8=\count7\relax%
    \multiply\count8 by-1461\relax%
    \advance\count8 by\count6\relax%
    % 9) a = (db div 365 + 1) * 3 div 4
    \count9=\count8\relax%
    \divide\count9 by 365\relax%
    \advance\count9 by 1\relax%
    \multiply\count9 by3\relax%
    \divide\count9 by4\relax%
    % 10)da = db - a * 365
    \count10=\count8\relax%
    \count0=\count9\relax%
    \multiply\count0 by-365\relax%
    \advance\count10 by\count0\relax%
    % 11)y = g * 400 + c * 100 + b * 4 + a
    \count11=\count3\relax%
    \multiply\count11 by400\relax%
    \count0=\count5\relax%
    \multiply\count0 by100\relax%
    \advance\count11 by\count0\relax%
    \count0=\count7\relax%
    \multiply\count0 by4\relax%
    \advance\count11 by\count0\relax%
    \advance\count11 by\count9\relax%
    % 12)m = (da * 5 + 308) div 153 - 2
    \count12=\count10\relax%
    \multiply\count12 by5\relax%
    \advance\count12 by 308\relax%
    \divide\count12 by 153\relax%
    \advance\count12 by -2\relax%
    % 13)d = - (m + 4) * 153 div 5 + 122 + da
    \count13=\count12\relax%
    \advance\count13 by 4\relax%
    \multiply\count13 by153\relax%
    \divide\count13 by5\relax%
    \count13=-\count13\relax%
    \advance\count13 by 122\relax%
    \advance\count13 by \count10\relax%
    % 14)Y =  (m + 2) div 12 + y - 4800
    \count14=\count12\relax%
    \advance\count14 by 2\relax%
    \divide\count14 by12\relax%
    \advance\count14 by\count11\relax%
    \advance\count14 by-4800\relax%
    % 15)M = (m + 2) mod 12 + 1
    \count15=\count12\relax%
    \advance\count15 by2\relax%
    \count0=\count15\relax%
    \divide\count0 by12\relax%
    \multiply\count0 by12\relax%
    \advance\count15 by-\count0\relax%
    \advance\count15 by1\relax%
    % 16)D = d + 1
    \count16=\count13%
    \advance\count16by 1\relax%    
    %
    \xdef\pgf@temp@year{\the\count14}%
    \xdef\pgf@temp@month{\ifnum\count15<10 0\fi\the\count15}%
    \xdef\pgf@temp@day{\ifnum\count16<10 0\fi\the\count16}%
  }%
  \let#2=\pgf@temp@year%
  \let#3=\pgf@temp@month%  
  \let#4=\pgf@temp@day%  
}



% Returns the day of week as a number between 0 = Monday and 6 =
% Sunday
%
% #1 = a Julian day number
% #2 = a counter into which the weekday should be put.
%
% Example:
%
% \pgfcalendardatetojulian{2006-01-10}{\mycount}
% \pgfcalendarjuliantoweekday{\mycount}{\myweekday}

\def\pgfcalendarjuliantoweekday#1#2{%
  \c@pgf@counta=#1\relax%
  #2=\c@pgf@counta%
  \divide#2by7\relax%
  \multiply#2by-7\relax%
  \advance#2by\c@pgf@counta\relax%
}



% Converts a day of week into a weekday name (long or short)
%
% #1 = a number representing a weekday (0=Monday)
%
% Example:
%
% \pgfcalendardatetojulian{2006-01-10}{\mycount}
% \pgfcalendarjuliantoweekday{\mycount}{\myweekday}
% October 1st, 2006 was a \pgfcalendarweekdayname{\myweekday}

\def\pgfcalendarweekdayname#1{%
  \translate{\ifcase#1Monday\or Tuesday\or Wednesday\or Thursday\or Friday\or Saturday\or Sunday\fi}%
}

\def\pgfcalendarweekdayshortname#1{%
  \translate{\ifcase#1Mon\or Tue\or Wed\or Thu\or Fri\or Sat\or Sun\fi}%
}


% Converts a month of year number into a month name (long or short)
%
% #1 = a number representing a month (1=January)
%
% Example:
%
% \pgfcalendarmonthname

\def\pgfcalendarmonthname#1{%
  \translate{\ifcase#1\or January\or February\or March\or April\or
    May\or June\or July\or August\or September\or October\or
    November\or December\fi}%
}

\def\pgfcalendarmonthshortname#1{%
  \translate{\ifcase#1\or Jan\or Feb\or Mar\or Apr\or
    May\or Jun\or Jul\or Aug\or Sep\or Oct\or
    Nov\or Dec\fi}%
}



\endinput

% Copyright 2008 by Till Tantau
%
% This file may be distributed and/or modified
%
% 1. under the LaTeX Project Public License and/or
% 2. under the GNU Public License.
%
% See the file doc/generic/pgf/licenses/LICENSE for more details.

\ProvidesFileRCS[v\pgfversion] $Header: /cvsroot/pgf/pgf/generic/pgf/libraries/datavisualization/pgflibrarydatavisualization.formats.functions.code.tex,v 1.2 2008/11/07 19:02:39 tantau Exp $

\usepgfmodule{datavisualization}

% This file defines code for doing plots involving polar
% coordinates. This includes not only standard polar plots, but also
% pie charts. 




\pgfooclass{polar mapper}
{
  % Class plor mapper
  %
  % A polar mapper is used to map attributes given as (2d) polar
  % coordinates to Cartesian coordinates. Note that no special ranges
  % can be specified and that the angle must be given in degrees. You
  % can, however, use a standard mapper to change these things.
  
  \attribute angle;
  % The attribute from which the angle is read. If the angle attribute/const
  % is set, this value will be used instead.
  
  \attribute radius;
  % The attribute from which the radius is read. If the radius attribute/const
  % is set, this value will be used instead.
  
  \attribute x;
  % The attribute to which the resulting x coordinate is written.
  
  \attribute y;
  % The attribute to which the resulting y coordinate is written. 

  
  % Constructor
  %
  % #1 = angle attribute. Example: rotation.
  % #2 = radius attribute. Example: height.
  % #3 = x attribute. Example: x.
  % #4 = y attribute. Example: y.
  %
  \method polar mapper(#1,#2,#3,#4) {
    \pgfooset{angle}{#1}
    \pgfooset{radius}{#2}
    \pgfooset{x}{#3}
    \pgfooset{y}{#4}
  } 

  % Method
  \method default connects() {
    \pgfoothis.get handle(\pgf@dv@me)
    \pgfkeysvalueof{/pgf/data visualization/obj}.connect(\pgf@dv@me,map,map datapoint signal)
    \pgfkeysvalueof{/pgf/data visualization/obj}.connect(\pgf@dv@me,path,path signal)
  }

  % Slot
  %
  % Mapping is done only if both the angle and the radius attribute
  % have non-empty values.
  \method map() {
    \pgfkeysgetvalue{/data point/\pgfoovalueof{angle}/const}\pgf@dv@angle%
    \ifx\pgf@dv@angle\relax%
      \pgfkeysgetvalue{/data point/\pgfoovalueof{angle}}\pgf@dv@angle%
    \fi%
    \ifx\pgf@dv@angle\relax%
    \else%
      \ifx\pgf@dv@angle\pgfutil@empty%
      \else%
        \pgfkeysgetvalue{/data point/\pgfoovalueof{radius}/const}\pgf@dv@radius%
        \ifx\pgf@dv@radius\relax%
          \pgfkeysgetvalue{/data point/\pgfoovalueof{radius}}\pgf@dv@radius%
        \fi%
        \ifx\pgf@dv@radius\pgfutil@empty%
        \else%
          \ifx\pgf@dv@radius\relax%
          \else%
            \pgfmathsincos{\pgf@dv@angle}%
            \pgfmathmultiply{\pgfmathresultx}{\pgf@dv@radius}%
            \pgfkeyslet{/data point/\pgfoovalueof{x}}\pgfmathresult%
            \pgfmathmultiply{\pgfmathresulty}{\pgf@dv@radius}%
            \pgfkeyslet{/data point/\pgfoovalueof{y}}\pgfmathresult%
          \fi%
        \fi%
      \fi%
    \fi%
  }

  % Getters
  \method get angle attribute(#1) {
    \pgfooget{angle}{#1}
  }

  \method get radius attribute(#1) {
    \pgfooget{radius}{#1}
  }

  % Reaction to the path signal
  %
  % We try to be smart, which may well go wrong...
  %
  % Here is what we do: When a pgfdvpathmoveto is done, the radius and
  % the angle are recorded. When a pgfdvpathlineto is done, the saved
  % radius and angle are compared with the new angle and radius. If,
  % and only if, the radius is the same, but the angle is different,
  % the lineto is caught and replaced by an arc from the previous
  % position to the new position.

  \attribute prev angle;
  \attribute prev radius;
  
  \method path(#1) {%
    \ifpgfdvpathcommandhandled%
    \else%
      \ifx#1\pgfdvpathlinetotoken
        % Ok, let us have a look...
        \pgfooget{prev radius}\pgf@dv@prev@rad
        \pgfkeysgetvalue{/data point/\pgfoovalueof{radius}}\pgf@dv@current@rad
        \ifx\pgf@dv@current@rad\relax
        \else
          \ifdim\pgf@dv@current@rad pt=\pgf@dv@prev@rad pt%
            \pgfooget{prev angle}\pgf@dv@prev@angle
            \pgfkeysgetvalue{/data point/\pgfoovalueof{angle}}\pgf@dv@current@angle
            \ifx\pgf@dv@current@angle\relax
            \else
              \ifdim\pgf@dv@current@angle pt=\pgf@dv@prev@angle pt%
              % do nothing
              \else%
                \pgf@lib@dv@handle@arc%
              \fi%
            \fi%
          \fi%
        \fi%
      \fi%
    \fi%
    % No matter what, record the current position
    \pgfooeset{prev angle}{\pgfkeysvalueof{/data point/\pgfoovalueof{angle}}}
    \pgfooeset{prev radius}{\pgfkeysvalueof{/data point/\pgfoovalueof{radius}}}
  }

  \def\pgf@lib@dv@handle@arc{%
    {%
      % Compute the main axes
      % Compute the first axis:
      {%
        \pgfkeyssetvalue{/data point/\pgfoovalueof{angle}/const}{0}%
        \pgfpointdvlocaldatapoint%
        \xdef\pgf@dv@veczero{\noexpand\pgfqpoint{\the\pgf@x}{\the\pgf@y}}%
      }%
      % Compute the second axis:
      {%
        \pgfkeyssetvalue{/data point/\pgfoovalueof{angle}/const}{90}%
        \pgfpointdvlocaldatapoint%
        \xdef\pgf@dv@vecninety{\noexpand\pgfqpoint{\the\pgf@x}{\the\pgf@y}}%
      }%
      % Compute origin:
      {%
        \pgfkeyssetvalue{/data point/\pgfoovalueof{radius}/const}{0}%
        \pgfpointdvlocaldatapoint%
        \xdef\pgf@dv@veccenter{\noexpand\pgfqpoint{\the\pgf@x}{\the\pgf@y}}%
      }%
      % Do arc:
      \pgftransformtriangle{\pgf@dv@veccenter}{\pgf@dv@veczero}{\pgf@dv@vecninety}%
      \pgfpatharc{\pgf@dv@prev@angle}{\pgf@dv@current@angle}{1pt}%
    }%
    \pgfdvpathcommandhandledtrue%
  }
}




\endinput



%   % Shifts the coordinate system by the given amount
%   %
%   % This command will offset the coordinate system by the given
%   % dimension in the direction of the unit vector.
%   \method shift(#1) {
%     \pgf@process{\pgfpointnormalised{\pgfoovalueof{unit vector}}}
%     \pgfmathparse{#1}
%     \pgftransformshift{\pgfqpoint{\pgfmathresult\pgf@x}{\pgfmathresult\pgf@y}}
%   }
  
%   % Shifts the coordinate system by the given amount orthogonally
%   %
%   % This command will offset the coordinate system by the given
%   % dimension orthogonally to the direction of the unit vector.
%   \method shift orthogonally(#1) {
%     \pgf@process{\pgfpointnormalised{\pgfoovalueof{unit vector}}}
%     \pgfmathparse{#1}
%     \pgf@x=-\pgf@x
%     \pgftransformshift{\pgfqpoint{\pgfmathresult\pgf@y}{\pgfmathresult\pgf@x}}
%   }
  

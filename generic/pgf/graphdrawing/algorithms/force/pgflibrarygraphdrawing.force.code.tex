% Copyright 2011 by Jannis Pohlmann
%
% This file may be distributed and/or modified
%
% 1. under the LaTeX Project Public License and/or
% 2. under the GNU Public License.
%
% See the file doc/generic/pgf/licenses/LICENSE for more details.

\ProvidesFileRCS[v\pgfversion] $Header: /cvsroot/pgf/pgf/generic/pgf/graphdrawing/algorithms/force/pgflibrarygraphdrawing.force.code.tex,v 1.3 2011/05/14 16:49:10 jannis-pohlmann Exp $



\usepgflibrary{graphdrawing}


%
% Dispatcher for deciding on a default spring layout
% 


\pgfgddeclareforwardedkeys{/graph drawing}{
  spring layout/.forward to={/graph drawing/\pgfkeysvalueof{/graph drawing/spring layout/default algorithm}}
}

%\pgfkeys{/graph drawing/spring layout/default algorithm/.initial=Walshaw2000 spring electrical}
\pgfkeys{/graph drawing/spring layout/default algorithm/.initial=Hu2006 spring electrical}




%
% Definition of spring and spring-electrical algorithms as well as
% options to configure them.
%

\pgfgdset{
  spring layout/.cd,
  %
  iterations/.graph parameter,
  iterations/.parameter initial=500,
  % 
  random seed/.graph parameter,
  random seed/.parameter initial=42,
  % 
  natural spring dimension/.graph parameter=evaluate math expression,
  natural spring dimension/.parameter initial=1cm,
  %
  initial step dimension/.graph parameter=evaluate math expression,
  initial step dimension/.parameter initial=0,
  % 
  spring constant/.graph parameter,
  % 
  approximate repulsive forces/.graph parameter,
  approximate repulsive forces/.parameter initial=false,
  approximate repulsive forces/.default=true,
  % 
  cooling factor/.graph parameter,
  % 
  convergence tolerance/.graph parameter,
  convergence tolerance/.parameter initial=0.01,
  % 
  % Parameters related to the multilevel approach where the input graph
  % is iteratively coarsened into graphs with fewer nodes, until either
  % the number of nodes is small enough or the number of nodes in the 
  % could not be reduced by at least a user-defined ratio.
  %
  coarsen/.graph parameter,
  coarsen/.parameter initial=true,
  coarsen/.default=true,
  %
  coarsening/.code=\pgfkeys{
    /graph drawing/spring layout/coarsening/.cd,
    #1
  },
  %
  coarsening/minimum graph size/.graph parameter,
  coarsening/minimum graph size/.parameter initial=2,
  %
  coarsening/downsize ratio/.graph parameter,
  coarsening/downsize ratio/.parameter initial=0.25,
  %
  coarsening/collapse independent edges/.graph parameter,
  coarsening/collapse independent edges/.parameter initial=true,
  coarsening/collapse independent edges/.default=true,
  %
  coarsening/connect independent nodes/.graph parameter,
  coarsening/connect independent nodes/.parameter initial=false,
  coarsening/connect independent nodes/.default=true,
  %
  % Node parameters for spring and spring-electrical algorithms.
  %
  electric charge/.node parameter,
  electric charge/.parameter initial=1,
  %
  % Edge parameters for spring and spring-electrical algorithms.
  %
}

%
% Spring-electrical algorithm based on the paper
% 
%   "Efficient, High-Quality Force-Directed Graph Drawing"
%   Yifan Hu, 2006
%
\pgfgddeclarealgorithmkey
  {Hu2006 spring electrical}
  {spring layout}
  {
    algorithm=Hu2006 spring,
    spring layout/cooling factor/.parameter initial=0.95,
    spring layout/spring constant/.parameter initial=0.2,
  }



%
% Spring-electrical algorithm based on the paper
%
%   "A Multilevel Algorithm for Force-Directed Graph Drawing"
%   C. Walshaw, 2000
%
\pgfgddeclarealgorithmkey
  {Walshaw2000 spring electrical}
  {spring layout}
  {
    algorithm=Walshaw2000 spring,
    spring layout/cooling factor/.parameter initial=0.95,
    spring layout/spring constant/.parameter initial=0.01,
    spring layout/convergence tolerance/.parameter initial=0.001,
  }

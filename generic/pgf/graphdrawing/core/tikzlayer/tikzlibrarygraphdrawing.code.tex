% Copyright 2010 by Renée Ahrens, Olof Frahm, Jens Kluttig, Matthias Schulz, Stephan Schuster
%
% This file may be distributed and/or modified
%
% 1. under the LaTeX Project Public License and/or
% 2. under the GNU Public License.
%
% See the file doc/generic/pgf/licenses/LICENSE for more details.

\ProvidesFileRCS[v\pgfversion] $Header: /cvsroot/pgf/pgf/generic/pgf/graphdrawing/core/tikzlayer/tikzlibrarygraphdrawing.code.tex,v 1.3 2011/05/15 14:15:45 jannis-pohlmann Exp $

\usepgflibrary{graphdrawing}




%
% Setup graph drawing for tikz
% 

\tikzset{
  graph drawing scope/.style={
    % 
    % General stuff
    %
    /tikz/execute at begin scope={
      \pgfgdsetedgecallback{\pgfgdtikzedgecallback}
      \pgfgdbeginscope%
    },
    /tikz/execute at end scope={
      \pgfgdendscope
    },
    % 
    % Setup for plain syntax
    % 
    /utils/exec=\pgfgdaddspecificationhook{\tikz@lib@gd@spec@hook},%
    % 
    % Setup for the graphs syntax 
    %
    /tikz/--/.style={arrows=-},
    /tikz/graphs/new ->/.code n args={4}{\pgfgdedge{##1}{##2}{->}{/tikz,##3}{##4}},
    /tikz/graphs/new <-/.code n args={4}{\pgfgdedge{##1}{##2}{<-}{/tikz,##3}{##4}},
    /tikz/graphs/new --/.code n args={4}{\pgfgdedge{##1}{##2}{--}{/tikz,##3}{##4}},
    /tikz/graphs/new <->/.code n args={4}{\pgfgdedge{##1}{##2}{<->}{/tikz,##3}{##4}},
    /tikz/graphs/new -!-/.code n args={4}{\pgfgdedge{##1}{##2}{-!-}{/tikz,##3}{##4}},
    /tikz/graphs/placement/compute position/.code=,%
    % 
    % Setup for the tree syntax
    %
    /tikz/growth function=\relax,
    /tikz/edge from parent macro=\tikz@gd@edge@from@parent@macro,
  },
  % Link in...
  /graph drawing/at begin scope/.forward to=/tikz/graph drawing scope
}

\pgfgdappendtoforwardinglist{/tikz/,/tikz/graphs/}

\def\tikz@lib@gd@spec@hook{%
  \tikzset{
    % 
    % Catch edges 
    % 
    edge macro=\tikz@gd@plain@edge@macro,
    % 
    % Catch after and predix node commands 
    % 
    /tikz/append after command/.code=\tikz@lib@gd@aac{##1},
    /tikz/prefix after command/.code=\tikz@lib@gd@pac{##1},
    /utils/exec=\def\pgfpositionnodelaterpostsetup{\tikzlibgdpostsetup\tikzlibgdpostsetupend}
  }
}

\def\pgfgdtikzedgecallback#1#2#3#4#5#6#7{%
  \begin{scope}
    [to path={#7 -- (\tikztotarget) \tikztonodes},#3,#4,/graph drawing/.cd,#6]
    \draw
     (#1\tikzparentanchor)
       to
       #5
       (#2\tikzchildanchor);%
  \end{scope}
}

\def\tikz@gd@edge@from@parent@macro#1#2{
  [/utils/exec=\pgfgdedge{\tikzparentnode}{\tikzchildnode}{--}{/tikz,#1}{#2}]
}

\def\tikz@gd@plain@edge@macro#1#2{
  \pgfgdedge{\tikztostart}{\tikztotarget}{--}{/tikz,#1}{#2}
}


% 
% Do append after command and prefix after command by creating late options 
%

\def\tikz@lib@gd@aac#1{%
  \expandafter\tikz@lib@gd@aac@unpack\pgfpositionnodelaterpostsetup{#1}
}
\def\tikz@lib@gd@aac@unpack\tikzlibgdpostsetup#1\tikzlibgdpostsetupend#2{%
  \def\pgfpositionnodelaterpostsetup{\tikzlibgdpostsetup#1\tikzlibgdlateoptions{#2}\tikzlibgdpostsetupend}
}
\def\tikz@lib@gd@pac#1{%
  \expandafter\tikz@lib@gd@pac@unpack\pgfpositionnodelaterpostsetup{#1}
}
\def\tikz@lib@gd@ppac@unpack\tikzlibgdpostsetup#1\tikzlibgdpostsetupend#2{%
  \def\pgfpositionnodelaterpostsetup{\tikzlibgdpostsetup\tikzlibgdlateoptions{#2}#1\tikzlibgdpostsetupend}
}

\def\tikzlibgdpostsetup{%
  \begingroup%
  \expandafter\tikz@lib@gd@post@setup\pgfpositionnodelatername\relax%
}
\def\tikz@lib@gd@post@setup not yet positionedPGFINTERNAL#1\relax{%
  \def\tikz@temp{name=#1}%
}
\def\tikzlibgdlateoptions#1{%
  \expandafter\def\expandafter\tikz@temp\expandafter{\tikz@temp,after node path={#1}}
}
\def\tikzlibgdpostsetupend{%
  \path[late options/.expand once=\tikz@temp];
  \endgroup%
}



% 
% Conversions: Convert coordinates into pairs of values surrounded by
% braces. 
%

\pgfgdset{
  conversions/coordinate/.code={%
    \tikz@scan@one@point\pgf@process#1%
    \pgfmathsetmacro{\tikz@gd@temp@x}{\pgf@x}
    \pgfmathsetmacro{\tikz@gd@temp@y}{\pgf@y}
    \edef\pgfgdresult{{\tikz@gd@temp@x}{\tikz@gd@temp@y}}
  }
}





\endinput

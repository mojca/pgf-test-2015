% Copyright 2010 by Renée Ahrens, Olof Frahm, Jens Kluttig, Matthias Schulz, Stephan Schuster
% Copyright 2011 by Till Tantau
%
% This file may be distributed and/or modified
%
% 1. under the LaTeX Project Public License and/or
% 2. under the GNU Public License.
%
% See the file doc/generic/pgf/licenses/LICENSE for more details.

\ProvidesFileRCS[v\pgfversion] $Header: /cvsroot/pgf/pgf/generic/pgf/graphdrawing/core/basiclayer/pgflibrarygraphdrawing.code.tex,v 1.10 2011/05/30 17:09:26 jannis-pohlmann Exp $



% check if luatex is running

\ifx\directlua\relax%
  \PackageError{pgf}{You need to run LuaTeX to use the graph drawing library}{}
  \expandafter\endinput
\fi
\ifx\directlua\undefined%
  \PackageError{pgf}{You need to run LuaTeX to use the graph drawing library}{}
  \expandafter\endinput
\fi




%
% Basic Interface to the Graph Drawing Engine
%

% 
% In order to use the graph drawing engine, inside a pgfpicture you
% need to create a graph drawing scope using the following commands;
%



% Begins a new graph drawing scope
% 
% Description:
%
% Inside a graph drawing scope, all pgf nodes that are newly created
% are automatically passed to the graph drawing engine. In contrast,
% edges have to be indicated explicitly using the macro \pgfgdedge
% (this is because it is somewhat unclear what, exactly, should count
% as an edge). Naturally, users typically will not call \pgfgdedge
% explicitly, but have styles and keys invoke it internally.
%
% Usage:
% 
% \pgfgdset{algorithm=somealgorithm}
% \pgfgdbeginscope
%   \pgfnode{rectangle}{center}{A}{A}{}
%   \pgfnode{rectangle}{center}{B}{B}{}
%   \pgfnode{rectangle}{center}{C}{C}{}
%   \pgfgdedge{A}{B}{->}{}{}
%   \pgfgdedge{B}{C}{->}{}{}
%   \pgfgdedge{C}{A}{->}{}{}
% \pgfgdendscope
% 
% Naturally, users will typically use TikZ's somewhat simpler syntax: 
% 
% \tikz \graph [some algorithm] { A -> B -> C -> A };

\def\pgfgdbeginscope{%
  \begingroup
    % get options
    \pgfkeysgetvalue{/graph drawing/graph parameters}\pgf@gd@graph@parameters%
    \pgfgdlogmessage{GD:SYS: graph drawing scope started with graph parameters: \pgf@gd@graph@parameters}
    \directlua{
      pgf.graphdrawing.Interface:newGraph("\luaescapestring{\pgf@gd@graph@parameters}")
    }%
    \begingroup
      % Switch on late positioning
      \pgfpositionnodelater{\pgf@gd@positionnode@callback}
      % Switch on late edges
      \pgfgd@latecallback%
      % Kill transformations (will be added by the position now
      % macros)
      \pgftransformreset
}


% Ends a graph drawing scope
% 
% Description:
%
% This macro invokes the selected graph drawing algorithm and
% ships out all nodes within this scope
%
% See \pgfgdbeginscope

\def\pgfgdendscope{%
      \endgroup
    % Late positioning has ended
    \directlua{
      pgf.graphdrawing.Interface:drawGraph()
      pgf.graphdrawing.Interface:finishGraph()
    }
  \endgroup
}


% Hook into graph specification
% 
% #1 = code
% 
% Description:
% 
% Allows you to specify code that should be active while the graph
% drawing engine collects the information concerning the graph, but
% which should no longer be active when the graph is rendered.

\def\pgfgdaddspecificationhook#1{
  \expandafter\def\expandafter\pgfgd@latecallback\expandafter{\pgfgd@latecallback#1}
}
\let\pgfgd@latecallback\pgfutil@empty

% 
% All graph drawing keys normally live in the following namespace:
% /graph drawing. 
%

\def\pgfgdset{\pgfqkeys{/graph drawing}}




% Passing options to the graph drawing engine.
% 
% #1 = option's name
% #2 = option's value
% 
% Description:
%
% When a graph drawing algorithm starts working, a set of options,
% called "graph drawing parameters" in the following, can influence the
% way the algorithm works. For instance, an graph drawing parameter
% might be the average distance between vertices which the algorithm
% should take into account. Another example might be the fact the
% certain nodes are special nodes and that a certain edge should have
% a large label.
%
% These graph drawing parameters are different from "usual" pgf
% options: An alogrithmic parameter influenced the way the algorithm
% works, while usual options normally only influence how the result
% looks like. For instance, the fact that a node is red is not an
% graph drawing parameter (usually, at least), while the shape of a node
% might be an graph drawing parameter.
%
% There are three kinds of graph drawing parameters:
%
% 1) graph drawing graph parameters 
%    These parameters influence "the whole graph". An example
%    is the distance between vertices on the same level of a tree. 
%
% 2) graph drawing node parameters
%    These parameters are "attached" to a single node. This includes
%    options that are only meaningful in the context of a graph
%    drawing algorithm (like, say, the "mass" of a node in a
%    force-base algorith), but also hybrid attributed like the shape
%    of a node. The shape is important for pgf when it typesets the
%    node, but it may also be important for the graph drawing
%    algorithm since it might position circles differently from, say,
%    rectangles.
%    
% 3) graph drawing edge parameters
%    Similarly to nodes, edges can also have graph drawing
%    parameters. Also similarly to nodes, there can be purely
%    graph drawing parameters and also options that are hybrid.
%    
% You have to "declare" a graph drawing parameter similarly to a normal
% key, but instead of using the /.code, you use /.graph
% parameter, /.node parameter, and /.edge
% parameter. More details on how these handlers work is given below.
%
% Specifying the set of graph drawing parameters for a given graph or
% node or edge works as follows: When the graph drawing engine is
% started for a graph (using \pgfgdbeginscope), a snapshot is taken of
% all graph drawing graph parameters currently setup at this
% point. Similarly, when a node is created inside such a scope, a
% snapshot is taken of the set of all graph drawing node parameters in
% force at this point is taken and stored together with the
% node. Finally, when an edge is created, a snapshot of the setting of
% the graph drawing edge parameters is taken.
% 
% All of these option sets can easily be accessed inside the graph 
% drawing algorithms, see the documentation of the lua layer.

\def\pgfgdgraphparameter#1#2{%
  \pgfkeysaddvalue{/graph drawing/graph parameters}{}{{#1}{#2}}%
}
\def\pgfgdnodeparameter#1#2{%
  \pgfkeysaddvalue{/graph drawing/node parameters}{}{{#1}{#2}}%
}
\def\pgfgdedgeparameter#1#2{%
  \pgfkeysaddvalue{/graph drawing/edge parameters}{}{{#1}{#2}}%
}

\pgfgdset{
  graph parameters/.initial=,
  node parameters/.initial=,
  edge parameters/.initial=,
}  



% Key handler /.graph parameter
% 
% Description:
%
% When this key hanlder is applied to a key, this key becomes a graph
% drawing graph parameter (as explained above). Subsequently, setting
% this key will cause special internals to be setup so that graph
% drawing algorithms can access the value of this key easily and
% directly inside the lua layer.
% 
% A typical usage would be
%
%   /some path/my key/.graph parameter
%  
% Now, when people write /some path/my key=foo in their code, inside
% the algorithm the parameter "my key" would be set to "foo".
% 

\pgfkeys{
  /handlers/.graph parameter/.code=\pgf@gd@parameter{#1}{\pgfgdgraphparameter},
  /handlers/.node parameter/.code=\pgf@gd@parameter{#1}{\pgfgdnodeparameter},
  /handlers/.edge parameter/.code=\pgf@gd@parameter{#1}{\pgfgdedgeparameter}
}
\def\pgf@gd@parameter#1#2{%
  \def\pgfgd@temp{#1}%
  \ifx\pgfgd@temp\pgfkeysnovalue@text\def\pgfgd@temp{no conversion}\fi%
  \edef\pgf@marshal{\noexpand\pgf@gd@@parameter{\pgfkeyscurrentpath}{\pgfgd@temp}{\noexpand#2}}
  \pgf@marshal%
}

\def\pgf@gd@@parameter#1#2#3{%
  \pgfkeysdef{#1}{
    \pgfkeys{/graph drawing/conversions/#2={##1}}%
    \expandafter\pgf@gd@@@parameter\expandafter{\pgfgdresult}{#3}{#1}%
  }%
}
\def\pgf@gd@@@parameter#1#2#3{#2{#3}{#1}}





% Key handler /.parameter initial
%
% #1 = an initial value
%
% Description:
%
% Use this key handler instead of /.initial to assign an initial value
% to a graph parameter. What will happen is that on the Lua layer, the passed
% value is stored in a special table. Whenever the key is not
% explicitly set by the user, the value #1 will be used.
% 
% Note that
% 
%   /foo/.graph parameter, /foo=5
% 
% and
% 
%   /foo/.graph parameter, /foo/.parameter initial=5
%   
% have the same effect insofar as, in both cases, getOption ('/foo')
% will yield 5. The difference is that in the second case the TeX
% layer does not need to pass around huge lists of options and will,
% thus, be faster. In general, it is recommendable that initial values
% for graph drawing graph parameters are setup using this handler.
% 
% #1 is passed through the conversion of the key before being stored.

\pgfkeys{
  /handlers/.parameter initial/.code={
    {
      % Hack into setting of edge/node/graph parameters
      \let\pgfgdgraphparameter\pgf@gd@steal
      \let\pgfgdnodeparameter\pgf@gd@steal
      \let\pgfgdedgeparameter\pgf@gd@steal
      \expandafter\pgfkeys\expandafter{\pgfkeyscurrentpath={#1}}%
    }
  }
}

\def\pgf@gd@steal#1#2{%
  \directlua{pgf.graphdrawing.Interface:setGraphParameterDefault('\luaescapestring{#1}','\luaescapestring{#2}')}%
}



% In many cases, when you specify a graph parameter, you will not wish
% the "actual" setting of the key to be passed to the algorithm. For
% instance, suppose we write 
% 
%   /some path/width/.graph parameter
% 
% Now we could write /some path/width=20pt+2pt, and inside the
% algorithm the parameter "width" would equal the string
% "20pt+2pt". However, inside the algorithm, it would be somewhat
% preferable to have access to the value "22" rather than the string
% "20pt+2pt". Similary, when the width is "1in", the algorithm will
% prefer to get passed the number "72.27" instead of "1in".
% 
% For this reason, when you define a graph drawing parameter, you can
% optionally specify a conversion that will be applied to the
% key's value before it is passed to the algorithm. To do so, 
% write
% 
% /some path/my key/.graph parameter=conversion key
% 
% The conversion key will be executed with path prefix /graph
% drawing/conversions/. It gets passed the value that was
% assigned to the parameter key. It should store the "result" of the
% conversion in the macro \pgfgdresult. The contents of this macro is
% the text that will actually be passed down to the algorithm.
% 
% As an example, let us start with the default conversion: It simply
% stores its parameter as the result:
% 
% /graph drawing/conversions/no conversion/.code=\def\pgfgdresult{#1}
% 
% A more complicated example is the math conversion:
% 
% /graph drawing/conversions/math/.code=
%    \pgfmathparse{#1}\let\pgfgdresult\pgfmathresult

\pgfgdset{
  conversions/no conversion/.code=\def\pgfgdresult{#1},
  conversions/evaluate math expression/.code=\pgfmathparse{#1}\let\pgfgdresult\pgfmathresult,
  conversions/coordinate/.code args={(#1pt,#2pt)}{\edef\pgfgdresult{{#1}{#2}}}
}



% Graph drawing graph parameter "algorithm"
%  
% Description:
% 
% Used to decide which algorithm is used in the graph drawing engine
% for drawing the next graph.

\pgfgdset{
  algorithm/.graph parameter
}


% Adds an edge to the graph
%
% #1 = first node
% #2 = second node
% #3 = edge direction
% #4 = edge options (will be executed in a protected environment)
% #5 = aux stuff (curtesy for TikZ -- edge nodes)
%
% Description:
%
% Creating an edge means that you tell the graph drawing algorithm
% that #1 and #2 are connected. The "kind" of connection is indicated
% by #3, which may be one of the following:
% 
% ->  = a directed edge (also known as an arc) from #1 to #2
% --  = an undirected edge between #1 and #2
% <-  = a directed edge from #2 to #1, but with the "additional hint"
%       that this is a "backward" edge. A graph drawing algorithm may
%       or may not take this hint into account
% <-> = a bi-directed edge between #1 and #2.
% 
%
% The parameters #4 and #5 are a bit more tricky. When an edge between
% two vertices of a graph is created via \pgfgdedge, nothing is
% actually done immediately. After all, without knowing the final
% positions of the nodes #1 and #2, there is no way of
% creating the actual drawing commands for the edge. Thus, the actual
% drawing of the edge is done only when the graph drawing algorithm is
% done (namely in the macro \pgfgdedgecallback, see later).
% 
% Because of this "delayed" drawing of edges, options that influence
% the edge must be retained until the moment when the edge is actually
% drawn. Parameters #4 and #5 store such options.
% 
% Let us start with #4. This parameter should be set to a list of
% key-value pairs like
% 
%   /tikz/.cd, color=red, very thick, this edge must be vertical
% 
% Some of these options may be of interest to the graph drawing
% algorithm (like the last option) while others will 
% only be important during the drawing of edge (like the first
% option). The options that are important for the graph drawing
% algorithm must be passed to the algorithm via setting keys that have
% been declared using the handler .edge parameter (see 
% above).
% 
% The tricky part is that options that are of interest to the graph
% drawing algorithm must be executed *before* the algorithm starts,
% but the options as a whole are usually only executed during the
% drawing of the edges, which is *after* the algorithm has finished.
% 
% To overcome this problem, the following happens:
% 
% The options in #4 are executed "tentatively" inside
% \pgfgdedge. However, this execution is done in a "heavily guarded
% sandbox" where all effects of the options (like changing the
% color or the line width) do not propagate beyond the sandbox. Only
% the changes of the graph drawing edge parameters leave the
% sandbox. These parameters are then passed down to the graph drawing
% engine.
% 
% Later, when the edge is drawn using \pgfgdedgecallback, the options #4
% are available once more and then they are executed normally.
%
% Note that when the options in #4 are executed, no path is
% preset. Thus, you typically need to start it with, say, /tikz/.cd.
%
%
% The text in #5 is some "auxilliary" text that is simply stored away
% and later directly to \pgfgdedgecallback. This is a curtesy to TikZ,
% which can use it to store its node labels. 
%
% Example:
%
% \pgfgdedge{A}{B}{->}{red}{}
%
\def\pgfgdedge#1#2#3#4#5{%
  % Ok, setup sandbox
  \begingroup%
    \setbox0=\hbox{{%
        \pgfinterruptpath%
          \pgfkeys{#4}%
          \pgfkeysgetvalue{/graph drawing/edge parameters}\pgf@gd@edge@options%
          % create edge in Lua
          \toks0={#4}%
          \toks1={#5}%
          \directlua{
            pgf.graphdrawing.Interface:addEdge(
            "\luaescapestring{#1}","\luaescapestring{#2}","\luaescapestring{#3}",
            "\luaescapestring{\pgf@gd@edge@options}",
            "\luaescapestring{\the\toks0}","\luaescapestring{\the\toks1}")
          }%
        \endpgfinterruptpath%
      }}%
  \endgroup%
}




%
% Begins the ship out of nodes
%
\def\pgfgdbeginshipout{%
  \pgfscope
}

%
% End the shop out of nodes
%
\def\pgfgdendshipout{%
  \endpgfscope
}



% Define a callback for rendering edges
%
% #1 = macro name
% 
% Descriptions:
% 
% This is a callback from the graph drawing engine. At the end of the
% creation of a graph, when the nodes have been positioned, this macro
% is called once for each edge. The macro should take the following
% parameters:
% 
% #1 = from node
% #2 = to node
% #3 = direction (<-, --, ->, or <->)
% #4 = original options
% #5 = aux text (typically edge nodes)
% #6 = algorithm-generated options
% #7 = bend information
%
% The first five parameters are the original values that were passed
% down to the \pgfgdedge command.
% 
% #6 contains options that have been "computed by the algorithm". For
% instance, an algorithm might have determined, say, flow capacities
% for edges and it might now wish to communicate this information back
% to the upper layers. These options should be executed with the path
% /graph drawing.
% 
% #7 contains algorithmically-computed information concerning how the
% edge should bend. Currently, this will be a text like
% "--(10pt,20pt)--(30pt,40pt)" in tikz-syntax, but this may change to
% make things more portable.
% 
% By default, a simple line is drawn between the nodes. Usually, you
% will wish to install a more "fancy" callback, here.

\def\pgfgdsetedgecallback#1{\let\pgfgdedgecallback=#1}

\def\pgfgddefaultedgecallback#1#2#3#4#5#6#7{%
  {%
    \pgfscope
      \pgfpathmoveto{\pgfpointshapeborder{#1}{\pgfpointanchor{#2}{center}}}
      \pgfpathlineto{\pgfpointshapeborder{#2}{\pgfpointanchor{#1}{center}}}
      \pgfusepath{stroke}
    \endpgfscope
  }
}

\pgfgdsetedgecallback{\pgfgddefaultedgecallback}




% 
%
% Logging
% 
%


% New if for debugging mode
%
\newif\ifpgf@gd@verbose

% 
% Set verbose debugging mode
%
% #1 should be zero (0) or one (1) for false and true
%
\def\pgf@gd@set@verbose@mode#1{
  \ifodd#1
    \directlua{ pgf.graphdrawing.Sys:setVerbose(true) }
    \pgf@gd@verbosetrue
  \else
    \directlua{ pgf.graphdrawing.Sys:setVerbose(false) }
    \pgf@gd@verbosefalse
  \fi
}

%
% Enables the verbose logging of the graph drawing library
%
\def\pgfgdenableverboselogging{
  \pgf@gd@set@verbose@mode1%
}

%
% Disables the verbose logging of the graph drawing library
\def\pgfgddisableverboselogging{
  \pgf@gd@set@verbose@mode0%
}

%
% Prints a given message to the TeX output
%
% Note: logging must be enabled by \pgfgdenableverboselogging
%
% Example:
%
%  \pgfgdenableverboselogging
%  \pgfgdlogmessage{Hello world}
%
\def\pgfgdlogmessage#1{
 \ifpgf@gd@verbose\directlua{ texio.write_nl("\luaescapestring{#1}") }\fi
}






% Helper keys for defining new graph drawing algorithms and their
% options  
% 
% Description:
% 
% When you hanve written a new graph drawing algorithm in lua (see the
% documentation for how this works), you need to make the higher
% levels aware of the algorithm and its parameters.
% 
% In principle, users could just write /graph drawing/algorithm=your
% algorithm's name and set options using for instance /graph
% drawing/your options=foo, but this would result in numerous
% repetitions of the prefix "/graph drawing/" in the code. Indeed, it
% would be somewhat preferable that users can just write things like
% 
% \tikz \graph [planar] { ... };
% 
% and have the option "planar" launch some fancy algorithm. However,
% "planar" will then have to have the path /tikz/graphs/planar, while
% the graph drawing algorithms expect their keys in  /graph drawing.
% 
% To overcome these difficulties, key forwarding is used. You declare
% you keys with the /graph drawing/ path, which is the "correct way"
% of declaring these keys, and then additionally setup keys in the
% /tikz/ and the /tikz/graphs/ paths that forward their values to the
% /graph drawing/ path. The helper functions below are intended to make
% this reasonably easy.
% 
%
% Graph parameters come in too flavours: "common" and
% "family-specific". A "common" graph parameter can be used by
% several graph drawing algorithms. An example are orientation keys,
% which can actually be applied to any graph in a postprocessing
% step. In contrast, "family-specific" keys are only important for
% one algorithm or only for algorithms from a small family of
% algorithms. For instance, a "stiffness" for spring layout algorithms
% would only apply to, well, spring layout algorithms.
% 
% The common graph parameters reside in the path /graph drawing/,
% while the family-specific graph parameters reside in the path /graph
% drawing/<family name>/. For common graph parameters, forwarding will
% be setup in paths like /tikz/ or /tikz/graphs, so you can use these
% keys directly. In contrast, no forwarding will be setup for
% family-specific keys. Rather, these keys can be passed to the
% algorithm's key, which will in turn executed the keys with the
% prefix /graph drawing/<family name>/.


% Declare keys for a new algorithm
% 
% #1 = algorithm's key name
% #2 = algorithm's family name
% #3 = options that configure the algorithm
% 
% Description:
% 
% This will setup the following keys:
% 
% /graph drawing/#1 is setup so that, when executed, the graph drawing
% algorithm is configured by the options in #3 and the value passed to
% are executed as keys with the path prefix /graph
% drawing/#2. Furthermore, the key /graph 
% drawing/at begin scope will be executed.This key does nothing by
% default, but it could be used to setup things (as is done in tikz)
% to start the graph drawing engine for the current scope.
% 
% Next, forwarding fill be setup for each path in the current
% forwarding list (set using \pgfgdsetkeyforwardinglist). If <path> is
% in this list, <path>/#1 is forwarded to /graph drawing/#1.
% 
% All keys in #4 are setup with the path prefix #2.
% 
% Note that when several algorithms share keys, they should share the
% family.
%
%
% Example:
%
% \pgfgddeclarealgorithmkey
% {AhrensFKSS2011 minimize crossings}
% {AhrensFKSS2011 minimize crossings}
% {algorithm=localsearchgraph}

\def\pgfgddeclarealgorithmkey#1#2#3{%
  \pgfkeys{
    /graph drawing/#1/.code=\pgfgdset{%
      #3,%
      #2/.cd,%
      ##1,%
      /graph drawing/.cd,%
      at begin scope
    }
  }%
  \pgf@gd@setup@forwards{#1}
  \pgfutil@g@addto@macro\pgf@gd@forwards@list{\pgf@gd@setup@forwards{#1}}
}
\def\pgf@gd@setup@forwards#1{
  \let\pgf@temp\pgfutil@empty
  \foreach \pgf@gd@path in \pgf@gd@forwarding@list {
    \ifx\pgf@gd@path\pgfutil@empty\else%
      \expandafter\pgfutil@g@addto@macro\expandafter\pgf@temp\expandafter{\expandafter\pgfkeys\expandafter{\pgf@gd@path#1/.forward to=/graph drawing/#1}}
    \fi%
  }
  \pgf@temp
}

\pgfgdset{/graph drawing/at begin scope/.code=} % does nothing by default.

\let\pgf@gd@forwards@list\pgfutil@empty




% Append to the forwarding list:
% 
% #1 = paths to append to the forwarding list
% 
% Description:
% 
% Append the paths in #1 (with trailing slashes) to the forwarding
% list.
% 
% If algorithms have already been declared, forwarding will also be
% setup for them (using a bit of magic...).

\def\pgfgdappendtoforwardinglist#1{%
  \let\pgf@gd@forwarding@list@orig\pgf@gd@forwarding@list
  \def\pgf@gd@forwarding@list{#1}%
  \pgf@gd@forwards@list%
  \expandafter\def\expandafter\pgf@gd@forwarding@list\expandafter{\pgf@gd@forwarding@list@orig,#1}
}
\let\pgf@gd@forwarding@list\pgfutil@empty



% Declare a new common key
% 
% #1 = an optional path prefix 
% #2 = a list of key declarations
% 
% Description:
% 
% Each <element> of #2 should have the form <key name>/<action>.
% This will cause the following to happen:
% 
% #1/<key name>/<action> is executed
% 
% If #1/<key name> was not defined before, for each <path> in the path
% list, a forward from <path>/<key name> to /graph drawing/<key name>
% is setup.
%
% Example:
%
% \pgfgddeclarekeys{/graph drawing}
% {
%   height/.graph parameter=evaluate math expression,
%   height/.parameter initial=5cm, % no forwarding since already defined.
%   width/.graph parameter=evaluate math expression,
%   width/.parameter initial=5cm,
% }

\def\pgfgddeclareforwardedkeys#1#2{%
  \def\pgf@gd@prefix{#1}%
  \let\pgf@temp\pgfutil@empty
  \foreach \pgf@gd@temp/\pgf@gd@dec in {#2} {%
    \ifx\pgf@gd@temp\pgfutil@empty\else%
    \pgfkeysifdefined{\pgf@gd@prefix/\pgf@gd@temp/.@def}{%
    \expandafter\expandafter\expandafter\def%
    \expandafter\expandafter\expandafter\content%
    \expandafter\expandafter\expandafter{%
    \expandafter\expandafter\expandafter\pgf@gd@parse@onlykey%
    \expandafter\expandafter\expandafter{\expandafter\pgf@gd@temp\expandafter}\expandafter{\pgf@gd@dec}}}{%
    \expandafter\expandafter\expandafter\def%
    \expandafter\expandafter\expandafter\content%
    \expandafter\expandafter\expandafter{%
    \expandafter\expandafter\expandafter\pgf@gd@parse@commonkey%
    \expandafter\expandafter\expandafter{\expandafter\pgf@gd@temp\expandafter}\expandafter{\pgf@gd@dec}}}%
    \expandafter\pgfutil@g@addto@macro\expandafter\pgf@temp\expandafter{\content}%
    \fi
  }
  \pgf@temp
}
\def\pgf@gd@parse@commonkey#1#2{%
  \pgfkeys{\pgf@gd@prefix/#1/#2}
  \pgf@gd@setup@forwards{#1}
  \pgfutil@g@addto@macro\pgf@gd@forwards@list{\pgf@gd@setup@forwards{#1}}
}
\def\pgf@gd@parse@onlykey#1#2{%
  \pgfkeys{\pgf@gd@prefix/#1/#2}
}



% 
% Orientation and transformation keys 
%


% Orientation
% 
% #1 = a node name
% #2 = another node name
% #3 = an angle
% #4 = "normal" or "swapped"
% 
% Description:
%
% When a layout is computed for a graph, the "origin" of the graph and
% its rotation typically are not fixed a priori. Indeed, shifting the
% graph arbitrarily will not change the layout and, for most layouts,
% neither will a rotation of the graph.
% 
% For this reason, it may be useful to specify positions of some node
% explicitly and have the rest of the graph be shifted and rotated so
% that these nodes lie at the specified positions.
% 
% When the graph parameter "orientation" is set, it should be set to
% {#1}{#2}{#3}{#4}. Then, the graph will be transformed such that #1
% is at the origin, #2 is on a line at an angle of #3 from #1 and
% depending on the value of #4, the rest of the points will or will
% not be mirrored along the line from #1 to #2.
% 
% The transformation is done automatically, the graph drawing
% algorithm need not worry about it...
%
%
% The orient and orient' keys are provide a simpler interface to the
% orientation key:
%
% orient = first node-second node
%   sets /graph drawing/orientation to {first node}{second node}{0}{normal}
% 
% orient = first node|second node
%   sets /graph drawing/orientation to {first node}{second node}{90}{normal}
% 
% orient = first node:angle:second node
%   sets /graph drawing/orientation to {first node}{second node}{angle}{normal}
%   
% orient' works as the above, but with "normal" replaced by "swapped"


\pgfgdset{
  orientation/.graph parameter,
  normal orientation/.code={\tikz@gd@parse@orient{#1}{normal}},
  swapped orientation/.code={\tikz@gd@parse@orient{#1}{swapped}},
}

\pgfgddeclareforwardedkeys{/graph drawing}{
  orient/.forward to={/graph drawing/normal orientation},
  orient'/.forward to={/graph drawing/swapped orientation}
}

\def\tikz@gd@parse@orient#1#2{
  \pgfutil@in@:{#1}%
  \ifpgfutil@in@%
    \tikz@gd@parse@orient@angle#1\pgf@stop{#2}%
  \else%
    \pgfutil@in@-{#1}%
    \ifpgfutil@in@%
      \tikz@gd@parse@orient@horiz#1\pgf@stop{#2}%
    \else%
      \edef\pgf@temp{#1}%
      \expandafter\tikz@gd@parse@orient@vert\pgf@temp\pgf@stop{#2}%
    \fi%
  \fi%
}

\def\tikz@gd@parse@orient@horiz#1-#2\pgf@stop#3{
  \pgfkeys{/graph drawing/orientation={#1}{#2}{0}{#3}}%
}

\def\tikz@gd@parse@orient@vert#1|#2\pgf@stop#3{
  \pgfkeys{/graph drawing/orientation={#1}{#2}{-90}{#3}}%
}

\def\tikz@gd@parse@orient@angle#1:#2:#3\pgf@stop#4{
  \pgfmathsetmacro{\tikz@gd@temp}{#2}%
  \edef\tikz@gd@temp{/graph drawing/orientation={#1}{#3}{\tikz@gd@temp}{#4}}%
  \expandafter\pgfkeys\expandafter{\tikz@gd@temp}%
}




% Desired positiong for a node
% 
% #1 = an x-position
% #2 = a y-position
% 
% Description:
% 
% This node parameter stores two numbers in the format {#1}{#2}. When
% set, the layout algorithm is "requested" to try an place the node at
% this position.

\pgfgddeclareforwardedkeys{/graph drawing}{
  desired at/.node parameter=coordinate
}








%
% INTERNAL MACROS
%-------------------------------------------------------------------------------
%


%
% This file defines the basic interactions between LUA and PGF
%

%
% This box is used for passing pgf elements to lua and vice versa 
%
% After the invocation of \pgfpositionnodelater the caller should copy the contents of 
% \box\pgfpositionnodelaterbox to another box register (for this refer to the manual).
%
\newbox\pgf@gd@box

% 
% Initialize the lua graph drawing environment
%
\directlua{
  local file = 'pgflibrarygraphdrawing-loader.lua'
  local format = 'tex'
  % Use either resolvers or kpse to locate files.
  if resolvers then
    dofile(assert(resolvers.find_file(file, format), "couldn't find file " .. file))
  else
    dofile(assert(kpse.find_file(file, format), "couldn't find file " .. file))
  end
  % set the box number for saving nodes by pgf
  pgf.graphdrawing.Sys:setBoxNumber(\the\pgf@gd@box)
}



%
% Callback method for \pgf@gd@positionnodelater
% 
% Pipes the box which contains the nodes to lua.
% The called lua method then saves the data.
%
\def\pgf@gd@positionnode@callback{%
  \pgfgdlogmessage{GD:SYS: positionnode callback invoked for box \the\pgf@gd@box}
  {%
    % save options to macro \pgf@gd@node@options
    \pgfkeysgetvalue{/graph drawing/node parameters}\pgf@gd@node@options
    % save pgfpositionnodelaterbox (see manual)
    \setbox\pgf@gd@box=\box\pgfpositionnodelaterbox\relax
    % evaluate coordinates
    \pgfmathsetmacro{\pgf@gd@node@minx}{\pgfpositionnodelaterminx}%
    \pgfmathsetmacro{\pgf@gd@node@miny}{\pgfpositionnodelaterminy}%
    \pgfmathsetmacro{\pgf@gd@node@maxx}{\pgfpositionnodelatermaxx}%
    \pgfmathsetmacro{\pgf@gd@node@maxy}{\pgfpositionnodelatermaxy}%
    % call lua system library to create a lua node object
    \directlua{
      %% NOTE: options parameter has to be the {key1}{value1}{key2}{value2} string
      pgf.graphdrawing.Interface:addNode(
        "\luaescapestring{\pgfpositionnodelatername}",
        "\luaescapestring{\pgf@gd@node@minx}",
        "\luaescapestring{\pgf@gd@node@miny}",
        "\luaescapestring{\pgf@gd@node@maxx}",
        "\luaescapestring{\pgf@gd@node@maxy}",
        "\luaescapestring{\pgf@gd@node@options}") }
  }%
}

%
% Shipout a node
%
% #1 = name of the node
% #2 = x min of the bounding box
% #3 = x max of the bounding box
% #4 = y min of the bounding box
% #5 = y max of the bounding box
% #6 = desired x pos of the node
% #7 = desired y pos of the node
% #8 = box register number of the TeX node
%
% This is an internal function and will be called by the Sys-class of the Lua part
%
% Example
%
%  \pgfgdinternalshipoutnode{not yet positionedPGFINTERNALnodename}{10}{10}{20}{20}{10pt}{10pt}0
% 
\def\pgfgdinternalshipoutnode#1#2#3#4#5#6#7#8{%
  {%
    \pgfgdlogmessage{GD:TEX: positioning node #1 (#2,#3,#4,#5) to #6,#7 from register #8}\relax
    \def\pgfpositionnodelatername{#1}
    \def\pgfpositionnodelaterminx{#2}
    \def\pgfpositionnodelatermaxx{#3}
    \def\pgfpositionnodelaterminy{#4}
    \def\pgfpositionnodelatermaxy{#5}
    \setbox\pgf@gd@box=\box\pgfutil@voidb@x
    \directlua{
      texnode = pgf.graphdrawing.TeXBoxRegister:getBox(#8)
      assert(texnode,"GD:SYS:TEX: tex node was nil")
      tex.box[\the\pgf@gd@box] = texnode
    }
    \setbox\pgfpositionnodelaterbox=\box\pgf@gd@box
    \pgfpositionnodenow{\pgfqpoint{#6pt}{#7pt}}
  }%
}

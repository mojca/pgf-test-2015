% Copyright 2010 by Renée Ahrens, Olof Frahm, Jens Kluttig, Matthias Schulz, Stephan Schuster
%
% This file may be distributed and/or modified
%
% 1. under the LaTeX Project Public License and/or
% 2. under the GNU Public License.
%
% See the file doc/generic/pgf/licenses/LICENSE for more details.

\ProvidesFileRCS[v\pgfversion] $Header: /cvsroot/pgf/pgf/generic/pgf/graphdrawing/tex/tikzlibrarygraphdrawing.code.tex,v 1.10 2012/06/28 19:31:46 tantau Exp $

\usepgflibrary{graphdrawing}

\def\tikz@lib@ensure@gd@loaded{}


% Patch the level and sibling distances so that gd and plain tikz are
% in sync
\tikzset{level distance=1cm, sibling distance=1cm}



%
% Setup graph drawing for tikz
% 

\def\tikz@gd@request@callback#1#2{%
  \tikzset{
    execute at begin scope={
      \tikzset{ 
        --/.style={arrows=-},
      }   
      \pgfgdsetedgecallback{\pgfgdtikzedgecallback}%
      % 
      % Setup for plain syntax
      % 
      \pgfgdaddspecificationhook{\tikz@lib@gd@spec@hook}%
      #1
      \let\tikzgdeventcallback\pgfgdevent%
      \let\tikzgdeventgroupcallback\pgfgdeventgroup%
      \let\tikzgdlatenodeoptionacallback\pgfgdsetlatenodeoption%
    },
    execute at end scope={#2}
  }%
}

\pgfgdsetrequestcallback\tikz@gd@request@callback



\pgfgdappendtoforwardinglist{/tikz/,/tikz/graphs/}

\def\tikz@lib@gd@spec@hook{%
  \tikzset{
    edge macro=\tikz@gd@plain@edge@macro,
    /tikz/at/.style={/graph drawing/desired at={##1}},
    % 
    % Setup for the tree syntax (do this late so that grow options
    % are still valid after a layout has been chosen)
    % 
    /tikz/growth function=\relax,
    /tikz/edge from parent macro=\tikz@gd@edge@from@parent@macro,
    % 
    % Setup for the graphs syntax 
    % 
    /tikz/graphs/new ->/.code n args={4}{\pgfgdedge{##1}{##2}{->}{/tikz,##3}{##4}},
    /tikz/graphs/new <-/.code n args={4}{\pgfgdedge{##1}{##2}{<-}{/tikz,##3}{##4}},
    /tikz/graphs/new --/.code n args={4}{\pgfgdedge{##1}{##2}{--}{/tikz,##3}{##4}},
    /tikz/graphs/new <->/.code n args={4}{\pgfgdedge{##1}{##2}{<->}{/tikz,##3}{##4}},
    /tikz/graphs/new -!-/.code n args={4}{\pgfgdedge{##1}{##2}{-!-}{/tikz,##3}{##4}},
    /tikz/graphs/placement/compute position/.code=,%
  }
}

\pgfgdaddprepareedgehook{
  \tikz@enable@edge@quotes%
  \let\tikz@transform=\pgfutil@empty%
  \let\tikz@options=\pgfutil@empty%
  \let\tikz@tonodes=\pgfutil@empty%
}


\def\pgfgdtikzedgecallback#1#2#3#4#5#6#7{%
  \draw[edge macro=]
    (#1\tikzparentanchor)
    edge[to path={#7 -- (\tikztotarget) \tikztonodes},#3,#4,/graph drawing/.cd,#6]
    #5
    (#2\tikzchildanchor);%
}

\def\tikz@gd@edge@from@parent@macro#1#2{
  [/utils/exec=\pgfgdedge{\tikzparentnode}{\tikzchildnode}{--}{/tikz,#1}{#2}]
}

\def\tikz@gd@plain@edge@macro#1#2{
  \pgfgdedge{\tikztostart}{\tikztotarget}{--}{/tikz,#1}{#2}
}



% 
% Conversions: Convert coordinates into pairs of values surrounded by
% braces. 
%

\pgfgdset{
  conversions/coordinate/.code={%
    \tikz@scan@one@point\pgf@process#1%
    \pgfmathsetmacro{\tikz@gd@temp@x}{\pgf@x}
    \pgfmathsetmacro{\tikz@gd@temp@y}{\pgf@y}
    \edef\pgfgdresult{{\tikz@gd@temp@x,\tikz@gd@temp@y}}
  }
}



%
% Overwrite node callback
%

\def\pgfgdgeneratenodecallback#1#2#3#4{%
  { \node[every generated node/.try,name={#1},shape={#2},/graph drawing/.cd,#3]{#4}; }
}


% 
% Layout handling 
%

\pgfgdset{start layout node/.style={/tikz/.cd,every layout node/.try,@layout styling}}
\tikzset{
  @layout styling/.style=,
  layout nodes/.style={/tikz/@layout styling/.append style={,#1}},
  graphs/layout nodes/.style={/tikz/layout nodes={#1}},
}


% "General" text placement for layout nodes that works with all node
% kinds.

\tikzset{
  layout text above inside/.code=\tikz@lg@simple@contents@top{#1},%
  layout text above inside/.default=text ragged,
  layout text below inside/.code=\tikz@lg@simple@contents@bottom{#1},%
  layout text below inside/.default=text ragged,
  layout text sep/.initial=.1em,
  layout text none/.code={
    \def\pgfgdlayoutnodecontents{%
      \hbox to \pgfkeysvalueof{/graph drawing/layout bounding box width}{%
        \vrule width0pt height\pgfkeysvalueof{/graph drawing/layout bounding box height}\hfil}%
    }%
    \tikzset{anchor=center}
  },
  @layout text label or pin/.code 2 args={
    \pgfkeysgetvalue{/graph drawing/layout text}\pgf@temp%
    \tikzset{layout text none}%
    \def\pgf@tempa{#1:}%
    \expandafter\expandafter\expandafter\def\expandafter\expandafter\expandafter\pgf@tempa\expandafter\expandafter\expandafter{\expandafter\expandafter\expandafter{\expandafter\pgf@tempa\pgf@temp}}%
    \tikzset{#2/.expand once=\pgf@tempa}%
  },%
  layout text label/.style={@layout text label or pin={#1}{label}},
  layout text label/.default=\tikz@label@default@pos,
  layout text pin/.style={@layout text label or pin={#1}{pin}},
  layout text pin/.default=\tikz@pin@default@pos,
}

\def\tikz@lg@simple@contents@top#1{%
  \tikzset{anchor=base}%
  \def\pgfgdlayoutnodecontents{%
    \vbox{%
      \pgfkeysgetvalue{/graph drawing/layout text}\pgf@temp%
      \ifx\pgf@temp\pgfutil@empty%
      \else%
        \ifx\pgf@temp\pgfutil@sptoken%
        \else%
            \hbox to \pgfkeysvalueof{/graph drawing/layout bounding box width}{%
              \hsize=\pgfkeysvalueof{/graph drawing/layout bounding box width}\relax%
              \vbox{\noindent\strut\tikzset{#1}\tikz@text@action\pgf@temp}%
            }%
        \fi%
      \fi%
      \pgfmathparse{\pgfkeysvalueof{/tikz/layout text sep}}%
      \kern\pgfmathresult pt\relax%  
      \pgf@y=\pgfkeysvalueof{/graph drawing/layout bounding box height}\relax%
      \hbox to \pgfkeysvalueof{/graph drawing/layout bounding box width}{%
      \vrule width0pt height.5\pgf@y depth.5\pgf@y\hfil}%
    }%
  }%
}

\def\tikz@lg@simple@contents@bottom#1{%
  \tikzset{anchor=base}%
  \def\pgfgdlayoutnodecontents{%
    {%
      \pgf@y=\pgfkeysvalueof{/graph drawing/layout bounding box height}\relax%
      \setbox0=\vbox{%
        \hbox to \pgfkeysvalueof{/graph drawing/layout bounding box width}{\vrule width0pt height\pgf@y\hfil}%
        \pgfmathparse{\pgfkeysvalueof{/tikz/layout text sep}}%
        \kern\pgfmathresult pt\relax%  
        \pgfkeysgetvalue{/graph drawing/layout text}\pgf@temp%
        \ifx\pgf@temp\pgfutil@empty%
        \else%
          \ifx\pgf@temp\pgfutil@sptoken%
          \else%
            \hbox to \pgfkeysvalueof{/graph drawing/layout bounding box width}{%
              \hsize=\pgfkeysvalueof{/graph drawing/layout bounding box width}\relax%
              \vbox{\noindent\strut\tikzset{#1}\tikz@text@action\pgf@temp}%
            }%
          \fi%
        \fi%
      }%
      \pgf@ya=\ht0\relax%
      \advance\pgf@ya by-.5\pgf@y\relax%
      \ht0=.5\pgf@y\relax%
      \dp0=\pgf@ya\relax%
      \box0\relax%
    }%
  }%
}

\tikzset{every layout node/.style=layout text above inside}

%
% Simple shorthand
%

\tikzset{
  layout/.style={/graph drawing/request scope and layout,/graph drawing/algorithm={#1}},
  graphs/layout/.style={/tikz/layout={#1}},
}


\endinput

% Copyright 2010 by Renée Ahrens, Olof Frahm, Jens Kluttig, Matthias Schulz, Stephan Schuster
% Copyright 2011 by Till Tantau
%
% This file may be distributed and/or modified
%
% 1. under the LaTeX Project Public License and/or
% 2. under the GNU Public License.
%
% See the file doc/generic/pgf/licenses/LICENSE for more details.

\ProvidesFileRCS[v\pgfversion] $Header: /cvsroot/pgf/pgf/generic/pgf/graphdrawing/tex/pgflibrarygraphdrawing.trees.code.tex,v 1.2 2012/04/19 13:49:07 tantau Exp $


\usepgflibrary{graphdrawing}
\usepgflibrary{graphdrawing.layered}


%
% Common tree options
%

%
% 
% Spanning tree method
%
%

% Some algorithms work only on trees. When the to-be-drawn graph is
% not a tree, a spanning tree is computed and passed to the algorithm,
% allowing the algorithm to work on a tree rather than on the original
% graph. 

\pgfgddeclareforwardedkeys{/graph drawing}{
  spanning tree algorithm/.graph parameter,
  spanning tree algorithm/.parameter initial=pgf.gd.trees.BreadthFirst,
  root is first node/.graph parameter,
  root/.node parameter,
  minimum number of children/.node or graph parameter=evaluate math expression,
  minimum number of children/.parameter initial=0,
  desired child index/.node parameter=evaluate math expression,
  edge priority method/.graph parameter,
  edge priority method/.parameter initial=forward first,
  edge priority/.edge parameter=evaluate math expression,
  spanning tree edge/.style={/graph drawing/edge priority=1},
  no spanning tree edge/.style={/graph drawing/edge priority=5}
}

% Spanning tree algorithms:

\pgfgddeclareforwardedkeys{/graph drawing}{
  breadth first spanning tree/.style={spanning tree algorithm=pgf.gd.trees.BreadthFirst},
  depth first spanning tree/.style={spanning tree algorithm=pgf.gd.trees.DepthFirst},
}


% Specify the index of a child for a node

\pgfgddeclareforwardedkeys{/graph drawing}{
  first/.style={/graph drawing/desired child index=1},
  second/.style={/graph drawing/desired child index=2},
  third/.style={/graph drawing/desired child index=3},
  fourth/.style={/graph drawing/desired child index=4},
}



\pgfgdset{
  tree layout/.cd,
  %
  missing nodes get space/.graph parameter,
  missing nodes get space/.parameter initial=false,
  missing nodes get space/.default=true,
  significant sep/.graph parameter=evaluate math expression,
  significant sep/.parameter initial=1em
}


%
% A tree layout that implementes the Reingold-Tilford algorithm 
%
\pgfgddeclarealgorithmkey
  {tree layout}
  {tree layout}
  {
    algorithm=pgf.gd.trees.ReingoldTilford1981,
  }

%
% A tree layout that implementes the Reingold-Tilford algorithm 
%
\pgfgddeclarealgorithmkey
  {binary tree layout}
  {tree layout}
  {
    algorithm=pgf.gd.trees.ReingoldTilford1981,
    minimum number of children=2
  }

%
% A tree layout that implementes the Reingold-Tilford algorithm 
%
\pgfgddeclarealgorithmkey
  {extended binary tree layout}
  {tree layout}
  {
    algorithm=pgf.gd.trees.ReingoldTilford1981,
    minimum number of children=2,
    tree layout/missing nodes get space,
    tree layout/significant sep=0
  }

%
% A tree layout that implementes the Reingold-Tilford algorithm 
%
\pgfgddeclarealgorithmkey
  {ternary tree layout}
  {tree layout}
  {
    algorithm=pgf.gd.trees.ReingoldTilford1981,
    minimum number of children=3
  }

%
% A tree layout that implementes the Reingold-Tilford algorithm 
%
\pgfgddeclarealgorithmkey
  {extended ternary tree layout}
  {tree layout}
  {
    algorithm=pgf.gd.trees.ReingoldTilford1981,
    minimum number of children=3,
    tree layout/missing nodes get space,
    tree layout/significant sep=0
  }



% No algorithms yet...


\endinput

% Copyright 2010 by Renée Ahrens, Olof Frahm, Jens Kluttig, Matthias Schulz, Stephan Schuster
% Copyright 2011 by Till Tantau
%
% This file may be distributed and/or modified
%
% 1. under the LaTeX Project Public License and/or
% 2. under the GNU Public License.
%
% See the file doc/generic/pgf/licenses/LICENSE for more details.

\ProvidesFileRCS[v\pgfversion] $Header: /cvsroot/pgf/pgf/generic/pgf/graphdrawing/tex/pgflibrarygraphdrawing.trees.code.tex,v 1.1 2012/04/16 13:23:56 tantau Exp $


\usepgflibrary{graphdrawing}
\usepgflibrary{graphdrawing.layered}


%
% Common tree options
%

% Declare a root
%
% Description:
%
% Normally, the first node encountered in a graph drawing scope is
% used as the root of the tree. However, you can also pass the "root"
% key to a node to specify that this node should actually be used as
% the root of the tree.
% 
% Example:
% 
% \graph [some tree layout] {
%   a -- b [root] -- c
% };

\pgfgddeclareforwardedkeys{/graph drawing}{
  first/.style={/graph drawing/desired child index=1},
  second/.style={/graph drawing/desired child index=2},
  third/.style={/graph drawing/desired child index=3},
  fourth/.style={/graph drawing/desired child index=4},
}



\pgfgdset{
  tree layout/.cd,
  %
  missing nodes get space/.graph parameter,
  missing nodes get space/.parameter initial=false,
  missing nodes get space/.default=true,
  significant sep/.graph parameter=evaluate math expression,
  significant sep/.parameter initial=1em
}


%
% A tree layout that implementes the Reingold-Tilford algorithm 
%
\pgfgddeclarealgorithmkey
  {tree layout}
  {tree layout}
  {
    algorithm=pgf.gd.trees.ReingoldTilford1981,
  }

%
% A tree layout that implementes the Reingold-Tilford algorithm 
%
\pgfgddeclarealgorithmkey
  {binary tree layout}
  {tree layout}
  {
    algorithm=pgf.gd.trees.ReingoldTilford1981,
    minimum number of children=2
  }

%
% A tree layout that implementes the Reingold-Tilford algorithm 
%
\pgfgddeclarealgorithmkey
  {extended binary tree layout}
  {tree layout}
  {
    algorithm=pgf.gd.trees.ReingoldTilford1981,
    minimum number of children=2,
    tree layout/missing nodes get space,
    tree layout/significant sep=0
  }

%
% A tree layout that implementes the Reingold-Tilford algorithm 
%
\pgfgddeclarealgorithmkey
  {ternary tree layout}
  {tree layout}
  {
    algorithm=pgf.gd.trees.ReingoldTilford1981,
    minimum number of children=3
  }

%
% A tree layout that implementes the Reingold-Tilford algorithm 
%
\pgfgddeclarealgorithmkey
  {extended ternary tree layout}
  {tree layout}
  {
    algorithm=pgf.gd.trees.ReingoldTilford1981,
    minimum number of children=3,
    tree layout/missing nodes get space,
    tree layout/significant sep=0
  }



% No algorithms yet...


\endinput

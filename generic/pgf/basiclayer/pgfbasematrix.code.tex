% Copyright 2006 by Till Tantau
%
% This file may be distributed and/or modified
%
% 1. under the LaTeX Project Public License and/or
% 2. under the GNU Public License.
%
% See the file doc/generic/pgf/licenses/LICENSE for more details.

\ProvidesPackageRCS[v\pgfversion] $Header: /cvsroot/pgf/pgf/generic/pgf/basiclayer/pgfbasematrix.code.tex,v 1.16 2006/10/16 22:32:58 tantau Exp $


\newif\ifpgfmatrix

\newcount\pgf@matrix@currentrow
\newcount\pgf@matrix@currentcolumn
\newcount\pgf@matrix@numberofcolumns
\let\pgf@matrix@numberofrows=\pgf@matrix@currentrow % alias

\newbox\pgf@matrix@cell
\let\pgf@matrix@box=\pgf@matrix@cell % alias



% The pgfmatrix environment, to be used inside a pgfpicture
% environment. 
%
% #1 = a position inside the resulting matrix
%
% Description:
%
% Inside this environment, a table/matrix is typeset. This environment
% must be used inside a pgfpicture. For each cell of the matrix,
% drawing commands may be given. These commands will create a small
% picture and a bounding box is computed for this picture. The small
% picture is then placed inside the cell. The position inside the cell
% is computed according to the following rules:
%
% 1. In each column, the origins of the small pictures are all on a
%    vertical line.
% 2. In each row, the origins of the small pictures are all on a
%    horizontal line.
% 3. All cells in the same row have the same height and depth.
% 4. All cells in the same column have the same width.
% 5. The size of cells are chosen minimal such that the boxes fit into
%    the cells and still meet the above requirement.
%
% Next, a temporary coordinate system is established for the resulting
% matrix. The origin of the matrix is at the upper left corner. Inside
% this coordinate system, the position #1 is interpreted. Then, the
% matrix is shifted around such that this position #1 lies at the
% origin of the real main picture. Typically, you should setup the
% main coordinate system such that this position is somewhere where
% you need it.
%
% At the beginning and at the end of each cell, the macros
% \pgfmatrixbegincode and \pgfmatrixendcode are called. They get the
% row and column number of the current cell as paramters.
%
% Example:
%
% \begin{tikzpicture}
%
%   \begin{pgfmatrix}{\pgfpointorigin}
%     \draw (0,0) circle (1cm) & \node (a) [draw] {Hallo} \\
%     \node (b) [draw] {Welt}  & \draw (0,0) circle (5mm)
%   \end{pgfmatrix}
%
%   \draw [->] (a) -- (b);
%
% \end{tikzpicture}

\def\pgfmatrix#1{%
  \ifpgfmatrix%
    \PackageError{pgf}{You cannot nest pgfmatrix environments, yet}{}
  \fi%
  \begingroup%  
    \def\pgf@matrix@origin{#1}%
    \pgf@matrix@currentrow=0\relax%
    \pgf@matrix@currentcolumn=0\relax%
    \pgf@matrix@numberofrows=0\relax%
    \pgf@matrix@numberofcolumns=0\relax%
    \def\\{&\cr}%
    \tabskip=0pt%
    \offinterlineskip%
    \pgfmatrixtrue%
    \let\pgf@matrix@node@list=\pgfutil@empty%
    \setbox\pgf@matrix@box=\vbox\bgroup%
    \halign\bgroup%
    \pgf@matrix@init@row%
    \pgf@matrix@step@column%
    \pgf@matrix@startcell%
    {##}%
    \pgf@matrix@endcell%
    &%
    ##\pgf@matrix@padding&&%
    \pgf@matrix@step@column%
    \pgf@matrix@startcell%
    {##}%
    \pgf@matrix@endcell&%
    ##\pgf@matrix@padding%
    \cr%
}

\def\endpgfmatrix{%
      &\crcr%
      \egroup%
    \egroup%
    %
    % If there are named nodes, we must adjust their coordinates.
    %
    \ifx\pgf@matrix@node@list\pgfutil@empty%
    \else%
      % Ok, this means we got a lot of work to do...
      \pgf@matrix@compute@origin%
      \pgf@matrix@shift@nodes@initial%
    \fi%
    %
    % Shift origin
    % 
    \pgftransformresetnontranslations%
    \pgftransformshift{\pgf@matrix@origin\pgf@x=-\pgf@x\pgf@y=-\pgf@y}%
    \pgf@process{\pgfpointtransformed{\pgfpoint{0pt}{0pt}}}%
    \pgf@protocolsizes{\pgf@x}{\pgf@y}%
    \pgf@process{\pgfpointtransformed{\pgfpoint{\wd\pgf@matrix@box}{-\dp\pgf@matrix@box-\ht\pgf@matrix@box}}}%
    \pgf@protocolsizes{\pgf@x}{\pgf@y}%
    %
    % If there are named nodes, we must adjust their coordinates, again.
    %
    \ifx\pgf@matrix@node@list\pgfutil@empty%
    \else%
      \xdef\pgf@matrix@offset{\noexpand\pgfqpoint{\the\pgf@pt@x}{\the\pgf@pt@y}}%
      \pgf@matrix@shift@nodes@secondary{\pgf@matrix@offset}%
    \fi%
    %
    % Insert box
    %
    \wd\pgf@matrix@box=0pt%
    \ht\pgf@matrix@box=0pt%
    \dp\pgf@matrix@box=0pt%
    \setbox\pgf@matrix@box=\hbox{%
      \hskip\pgf@pt@x%
      \raise\pgf@pt@y\box\pgf@matrix@box%
      \hss%
    }%
    \wd\pgf@matrix@box=0pt%
    \ht\pgf@matrix@box=0pt%
    \dp\pgf@matrix@box=0pt%
    \box\pgf@matrix@box%
  \endgroup%
}

\let\startpgfmatrix=\pgfmatrix
\let\stoppgfmatrix=\endpgfmatrix



% Init a line

\def\pgf@matrix@init@row{%
  \relax%
  \global\advance\pgf@matrix@currentrow by 1\relax%
  \global\pgf@matrix@currentcolumn=0\relax%
  \expandafter\gdef\csname pgf@matrix@maxy\the\pgf@matrix@currentrow\endcsname{0pt}%
  \expandafter\gdef\csname pgf@matrix@miny\the\pgf@matrix@currentrow\endcsname{0pt}%
}



% Step the row column number

\def\pgf@matrix@step@column{%
  \relax%
  \global\advance\pgf@matrix@currentcolumn by1\relax%
}



% Start a cell

\def\pgf@matrix@startcell{%
  %
  % Step 1: Init the list of nodes for this cell
  %
  \let\pgf@nodecallback=\pgf@matrix@nodecallback%
  %
  % Step 2: Setup the bounding box
  %
  \pgfinterruptboundingbox%
  %
  % Step 3: Reset the transformation matrix
  %
  \pgftransformreset%
  %
  % Step 4: Collect everything in a cell box
  %
  \setbox\pgf@matrix@cell=\hbox\bgroup\bgroup%
    \pgfmatrixbegincode{\the\pgf@matrix@currentrow}{\the\pgf@matrix@currentcolumn}%
}


% End a  cell

\def\pgf@matrix@endcell{%
    \pgfmatrixendcode{\the\pgf@matrix@currentrow}{\the\pgf@matrix@currentcolumn}%
  \egroup\egroup%
  % Special case of the empty picture
  \ifdim\pgf@picmaxx=-16000pt\relax%
    \pgf@picmaxx=0pt\relax%
    \pgf@picminx=0pt\relax%
    \pgf@picmaxy=0pt\relax%
    \pgf@picminy=0pt\relax%
  \fi%
  % 
  % Step 4: Protocol and adjust height and depth
  %
  % Step 4.1: Protocol...
  %
  \ifdim\pgf@picmaxy>\csname pgf@matrix@maxy\the\pgf@matrix@currentrow\endcsname%
    \expandafter\xdef\csname pgf@matrix@maxy\the\pgf@matrix@currentrow\endcsname{\the\pgf@picmaxy}%
  \fi%  
  \ifdim\pgf@picminy<\csname pgf@matrix@miny\the\pgf@matrix@currentrow\endcsname%
    \expandafter\xdef\csname pgf@matrix@miny\the\pgf@matrix@currentrow\endcsname{\the\pgf@picminy}%
  \fi%  
  %
  % Step 4.2: and setup.
  %  
  \ht\pgf@matrix@cell=\pgf@picmaxy%
  \dp\pgf@matrix@cell=-\pgf@picminy%
  %
  % Step 5: Protocol and adjust left and right width
  %
  % Step 5.1: Protocol...
  %
  \ifnum\pgf@matrix@currentcolumn>\pgf@matrix@numberofcolumns\relax%
    \expandafter\xdef\csname pgf@matrix@maxx\the\pgf@matrix@currentcolumn\endcsname{\the\pgf@picmaxx}%
    \expandafter\xdef\csname pgf@matrix@minx\the\pgf@matrix@currentcolumn\endcsname{\the\pgf@picminx}%
    \global\pgf@matrix@numberofcolumns=\pgf@matrix@currentcolumn\relax%
  \else%
    \ifdim\pgf@picmaxx>\csname pgf@matrix@maxx\the\pgf@matrix@currentcolumn\endcsname%
      \expandafter\xdef\csname pgf@matrix@maxx\the\pgf@matrix@currentcolumn\endcsname{\the\pgf@picmaxx}%
    \fi%  
    \ifdim\pgf@picminx<\csname pgf@matrix@minx\the\pgf@matrix@currentcolumn\endcsname%
      \expandafter\xdef\csname pgf@matrix@minx\the\pgf@matrix@currentcolumn\endcsname{\the\pgf@picminx}%
    \fi%  
  \fi%
  %
  % Step 5.2: and setup.
  %  
  \xdef\pgf@matrix@paddingskip{\the\pgf@picmaxx}%
  % 
  % Step 6: Put in the box now
  %
  \hfil\hskip-\pgf@picminx%
  \wd\pgf@matrix@cell=0pt%
  \box\pgf@matrix@cell%
  %
  % Step 7: End bounding box
  \endpgfinterruptboundingbox%
}

% Default code for the start and end code
\def\pgfmatrixbegincode#1#2{}
\def\pgfmatrixendcode#1#2{}


% The following callback is called for every node that is produced
% inside a cell

\def\pgf@matrix@nodecallback#1{%
  \def\pgf@temp{#1}%
  \ifx\pgf@temp\pgfutil@empty%
  \else%
    \xdef\pgf@matrix@node@list{\pgf@matrix@node@list,{#1}}%
    \expandafter\xdef\csname pgf@matrix@node@location@#1\endcsname{{\the\pgf@matrix@currentrow}{\the\pgf@matrix@currentcolumn}}%
  \fi%
}



% Padding code

\def\pgf@matrix@padding{%
  \hskip\pgf@matrix@paddingskip\hfil%
}


% Compute the real positions of the origins

% We must now compute the real positions of the origins of all the
% small pictures. To this end, we need to compute prefix sums. After
% the procedure is done, the minx and the maxy will contain the origin
% positions. 

\def\pgf@matrix@compute@origin{%
  %
  % Prefix sum on the vertical positions
  %
  {%
    \c@pgf@counta=1\relax%
    \loop%
      \ifnum\c@pgf@counta<\pgf@matrix@numberofrows\relax%
        \pgf@y=\csname pgf@matrix@maxy\the\c@pgf@counta\endcsname\relax%
        \pgf@ya=\csname pgf@matrix@miny\the\c@pgf@counta\endcsname\relax
        \advance\pgf@y by-\pgf@ya\relax%
        \advance\c@pgf@counta by1\relax%
        \advance\pgf@y by\csname pgf@matrix@maxy\the\c@pgf@counta\endcsname\relax%
        \expandafter\xdef\csname pgf@matrix@maxy\the\c@pgf@counta\endcsname{\the\pgf@y}%
    \repeat%
  }%
  %
  % Prefix sum on the horizontal positions
  %
  {%
    \ifnum\pgf@matrix@numberofcolumns>0\relax%
      \pgf@x=\csname pgf@matrix@minx1\endcsname\relax%
      \pgf@x=-\pgf@x%
      \expandafter\xdef\csname pgf@matrix@minx1\endcsname{\the\pgf@x}%
    \fi
    \c@pgf@counta=1\relax%
    \loop%
      \ifnum\c@pgf@counta<\pgf@matrix@numberofcolumns\relax%
        \pgf@x=\csname pgf@matrix@minx\the\c@pgf@counta\endcsname\relax%
        \advance\pgf@x by\csname pgf@matrix@maxx\the\c@pgf@counta\endcsname\relax%
        \advance\c@pgf@counta by1\relax%
        \pgf@xa=\csname pgf@matrix@minx\the\c@pgf@counta\endcsname\relax%
        \advance\pgf@x by-\pgf@xa\relax%
        \expandafter\xdef\csname pgf@matrix@minx\the\c@pgf@counta\endcsname{\the\pgf@x}%
    \repeat%
  }%
}


% Shift the nodes to their origins

% The following procedure shifts all nodes in
% \pgf@matrix@node@list to their location inside a temporary
% picture. This picture will later be shifted again to its final
% position in the real picture.

\def\pgf@matrix@shift@nodes@initial{%
  \pgfutil@for\pgf@matrix@node@name:=\pgf@matrix@node@list\do{%
    \ifx\pgf@matrix@node@name\pgfutil@empty%
    \else%
      \pgf@shift@node{\pgf@matrix@node@name}{%
        \pgf@x=\csname pgf@matrix@minx%
          \expandafter\expandafter\expandafter\pgfutil@secondoftwo\csname pgf@matrix@node@location@\pgf@matrix@node@name\endcsname\endcsname%
        \pgf@y=\csname pgf@matrix@maxy%
          \expandafter\expandafter\expandafter\pgfutil@firstoftwo\csname pgf@matrix@node@location@\pgf@matrix@node@name\endcsname\endcsname%
          \pgf@y=-\pgf@y%
        }%
    \fi%
  }%
}


% The second shifting, done in the following procedure, shifts all
% nodes to their real positions inside the real picture.

\def\pgf@matrix@shift@nodes@secondary#1{%
  \pgfutil@for\pgf@matrix@node@name:=\pgf@matrix@node@list\do{%
    \ifx\pgf@matrix@node@name\pgfutil@empty%
    \else%
      \pgf@shift@node{\pgf@matrix@node@name}{#1}%
    \fi%
  }%
}


\endinput

\ProvidesPackageRCS[v\pgfversion] $Header: /cvsroot/pgf/pgf/generic/pgf/basiclayer/pgfbasepatterns.code.tex,v 1.3 2006/03/07 23:28:06 tantau Exp $

% Copyright 2006 by Till Tantau <tantau@cs.tu-berlin.de>.
%
% This program can be redistributed and/or modified under the terms
% of the GNU Public License, version 2.


% Creates a new uncolored pattern
%
% #1 = pattern name
% #2 = lower left of bounding box
% #3 = upper right of bounding box
% #4 = step vector
% #5 = pattern code
%
% Description:
%
% Declares a new uncolored pattern. An uncolored pattern is a pattern
% that has no inherent color. When the pattern is used, a color for
% the pattern must be supplied.
%
% See the pdf-manual for more details on uncolored patterns.
%
% The bounding box must be a bounding box of the pattern cell.
%
% The step vector must provide the x- and y-step by which the cells
% are spaced.
%
% The pattern code should be pgf code that can be protocolled.
%
% The transformation matrix will be reset prior to invoking the
% pattern code.
%
% Example:
%
% \pgfdeclarepatternuncolored{stars}{\pgfpointorigin}{\pgfpoint{1cm}{1cm}}
% {\pgfpoint{1cm}{1cm}}{
%   \pgftransformshift{\pgfpoint{.5cm}{.5cm}}
%   \pgfpathmoveto{\pgfpointpolar{0}{4mm}}
%   \pgfpathlineto{\pgfpointpolar{144}{4mm}}
%   \pgfpathlineto{\pgfpointpolar{288}{4mm}}
%   \pgfpathlineto{\pgfpointpolar{72}{4mm}}
%   \pgfpathlineto{\pgfpointpolar{216}{4mm}}
%   \pgfpathclose%
%   \pgfusepath{stroke}
% }

\def\pgfdeclarepatternuncolored#1#2#3#4#5{\pgf@declarepattern{#1}{#2}{#3}{#4}{#5}{0}}

% Creates a new colored pattern
%
% #1 = pattern name
% #2 = lower left of bounding box
% #3 = upper right of bounding box
% #4 = step vector
% #5 = pattern code
%
% Description:
%
% Declares a new colored pattern. Such patterns have a one or more
% fixed inherent colors. See the pdf-manual for more details on
% uncolored patterns.
%
% The parameters have the same effect as for uncolored patterns.
%
% Example:
%
% \pgfdeclarepatterncolored{red stars}{\pgfpointorigin}{\pgfpoint{1cm}{1cm}}
% {\pgfpoint{1cm}{1cm}}{
%   \pgfsetfillcolor{red}
%   \pgfsetstrokecolor{black}
%   \pgftransformshift{\pgfpoint{.5cm}{.5cm}}
%   \pgfpathmoveto{\pgfpointpolar{0}{4mm}}
%   \pgfpathlineto{\pgfpointpolar{144}{4mm}}
%   \pgfpathlineto{\pgfpointpolar{288}{4mm}}
%   \pgfpathlineto{\pgfpointpolar{72}{4mm}}
%   \pgfpathlineto{\pgfpointpolar{216}{4mm}}
%   \pgfpathclose%
%   \pgfusepath{stroke,fill}
% }

\def\pgfdeclarepatterncolored#1#2#3#4#5{\pgf@declarepattern{#1}{#2}{#3}{#4}{#5}{1}}

\def\pgf@declarepattern#1#2#3#4#5#6{%
  \@ifundefined{pgf@pattern@name@#1}{%
  \pgfsysprotocol@getcurrentprotocol\pgf@pattern@temp%
  {%
    \pgfinterruptpath%
      \pgfpicturetrue%
      \pgf@relevantforpicturesizefalse%
      \pgftransformreset%
      \pgfsysprotocol@setcurrentprotocol\@empty%
      \pgfsysprotocol@bufferedtrue%
      \pgfsys@beginscope%
      #5%
      \pgfsys@endscope%
      \pgfsysprotocol@getcurrentprotocol\pgf@pattern@code%
      \global\let\pgf@pattern@code=\pgf@pattern@code%
    \endpgfinterruptpath%
    \pgf@process{#2}%
    \pgf@xa=\pgf@x%
    \pgf@ya=\pgf@y%
    \pgf@process{#3}%
    \pgf@xb=\pgf@x%
    \pgf@yb=\pgf@y%
    \pgf@process{#4}%
    \pgf@xc=\pgf@x%
    \pgf@yc=\pgf@y%
    % Now, build a name for the pattern
    \@tempcnta=\pgf@pattern@number%
    \advance\@tempcnta by1\relax%
    \xdef\pgf@pattern@number{\the\@tempcnta}%
    \expandafter\xdef\csname pgf@pattern@name@#1\endcsname{\the\@tempcnta}%
    \expandafter\gdef\csname pgf@pattern@type@#1\endcsname{#6}%
    \xdef\pgf@marshal{\noexpand\pgfsys@declarepattern
      {\csname pgf@pattern@name@#1\endcsname}
      {\the\pgf@xa}{\the\pgf@ya}{\the\pgf@xb}{\the\pgf@yb}{\the\pgf@xc}{\the\pgf@yc}{\pgf@pattern@code}{#6}}%
  }%
  \pgfsysprotocol@setcurrentprotocol\pgf@pattern@temp%
  \expandafter\global\expandafter\let\csname pgf@pattern@instantiate@#1\endcsname=\pgf@marshal%
  }
  {%
    \PackageError{pgfbasepatterns}{Pattern `#1' already defined}{}%
  }%
}

\def\pgf@pattern@number{0}%



% Check whether a pattern is colored
%
% #1 = name of the pattern
% #2 = code to be executed when the pattern is colored
% #3 = code to be executed when the pattern is uncolored
%
% Example:
%
% \pgfifpatterniscolored{\somepattern}
%   {\pgfsetfillpatterncolored{\somepattern}}
%   {\pgfsetfillpatternuncolored{\somepattern}{\defaultpatterncolor}}

\def\pgfifpatterniscolored#1#2#3{%
  \expandafter\ifnum\csname pgf@pattern@type@#1\endcsname=1\relax#2\else#3\fi%
}



% Set an pattern as fill color
%
% #1 = Name of the pattern
% #2 = Color to be used for the pattern. If the pattern is already
%      colored, this parameter is ignored.
%
% Description:
%
% Sets the pattern for subsequent filling operations.
%
% Example:
%
% \pgfsetfillpatternuncolored{stars}{red}

\def\pgfsetfillpattern#1#2{%
  \@ifundefined{pgf@pattern@name@#1}{%
    \PackageError{pgfbasepatterns}{Undefined pattern `#1'}{}}
  {%
    \csname pgf@pattern@instantiate@#1\endcsname%
    \expandafter\global\expandafter\let\csname pgf@pattern@instantiate@#1\endcsname=\relax%
    \pgfifpatterniscolored{#1}{%
      \pgfsys@setpatterncolored{\csname pgf@pattern@name@#1\endcsname}%
    }{%
      \colorlet{pgf@tempcolor}{#2}%
      \@ifundefined{applycolormixins}{}{\applycolormixins{pgf@tempcolor}}%
      \extractcolorspec{pgf@tempcolor}{\pgf@tempcolor}%
      \expandafter\convertcolorspec\pgf@tempcolor{rgb}{\pgf@rgbcolor}%
      \expandafter\pgf@set@fill@patternuncolored\pgf@rgbcolor\relax{#1}%
    }%
  }%
}
\def\pgf@set@fill@patternuncolored#1,#2,#3\relax#4{%
  \pgfsys@setpatternuncolored{\csname pgf@pattern@name@#4\endcsname}{#1}{#2}{#3}%
}


%
% One predefined pattern
%

\pgfdeclarepatternuncolored{dots}{\pgfpoint{-1pt}{-1pt}}{\pgfpoint{1pt}{1pt}}{\pgfpoint{3pt}{3pt}}%
{
  \pgfpathcircle{\pgfpoint{0pt}{0pt}}{.5pt}
  \pgfusepath{fill}
}



\endinput

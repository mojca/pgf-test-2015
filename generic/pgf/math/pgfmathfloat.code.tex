% Copyright 2008 by Christian Feuersaenger
%
% This file may be distributed and/or modified
%
% 1. under the LaTeX Project Public License and/or
% 2. under the GNU Public License.
%
% See the file doc/generic/pgf/licenses/LICENSE for more details.
%
% This file contains methods for basic floating point arithmetics, rounding
% to arbitrary precision and number pretty printing.
%
% Version 0.9.4 2008/03/18

\newif\ifpgfmathcomparison
\newif\ifpgfmathfloatcomparison
\newif\ifpgfmathfloatroundhasperiod


% used internally in grouped macros:
\toksdef\pgfmathfloat@tmptoks=1
\newif\ifpgfmathfloat@fixed@digits@after@period
\newif\ifpgfmathfloat@round@may@need@renormalize
\toksdef\pgfmathfloat@a@Mtok=0

%% WARNING: avoid conflicting register names!
\let\pgfmathfloat@a@S=\c@pgf@counta
\let\pgfmathfloat@a@M=\pgf@xa
\let\pgfmathfloat@a@E=\c@pgf@countb
\let\pgfmathfloat@b@S=\c@pgf@countc
\let\pgfmathfloat@b@M=\pgf@xb
\let\pgfmathfloat@b@E=\c@pgf@countd

% can be changed with options.
\def\pgfmathfloat@round@precision{2}


% Computes a normalised floating point representation for #1 of the
% form
%   [FLAGS] MANTISSE EXPONENT
% meaning
%   [+-]X.XXXXXXX*10^C
% where 
%   X.XXXXXX is a mantisse with first number != 0, C is an integer and
% FLAGS contains the sign and some other special cases.
%
% This method does NOT use TeX Registers to represent the numbers! The
% computation is COMPLETELY STRING BASED.
% This allows numbers such at 10000000 or 5.23e-10 to be represented
% properly, although TeX-registers would produce overflow/underflow
% errors in these cases. 
%
% It is to be used to compute logs, because log(X*10^Y) = log(X) + log(10)*Y
%
% Arguments:
% #1:  an arbitrary number which shall be parsed. Maybe a macro.
%     Examples:
%     [+-]XXXXX.XXXXXX
%     [+-]XXXXX.XXXXXXeXXXX
%     [+-]0.0000XXXXXX
%     [+-]0.0000XXXXXXeXXXX
%     [+-]inf
%     nan
%    Maybe a macro containing such a number.
%
% Return value:
% \pgfmathresult will be \def'ed to
%  FLAGS MANTISSE 'e' EXPONENT
% where
% FLAGS is a number in [0-5] where
% 		0 == '0' (the number is +- 0.0),
% 		1 == '+', 
% 		2 == '-',
% 		3 == 'not a number'
% 		4 == '+ infinity'
% 		5 == '- infinity'
%
% MANTISSE is a normalised number 1 <= M < 9. It always contains a
% period and at least one digit after the period.
%
% EXPONENT is an integer value.
%
% Example:
% \pgfmathfloatparsenumber{1}
% -> \pgfmathresult = 11.0e0
% \pgfmathfloatparsenumber{141.212}
% -> \pgfmathresult = 11.41212e2
\def\pgfmathfloatparsenumber#1{%
	\begingroup
	\expandafter\pgfmathfloatparsenumber@impl#1\pgfmathfloat@EOI
	\ifpgfmathfloatparsenumberpendingperiod
		\pgfmathfloat@a@Mtok=\expandafter{\the\pgfmathfloat@a@Mtok.0}%
	\fi
	\edef\pgfmathresult{%
		\the\pgfmathfloat@a@S\the\pgfmathfloat@a@Mtok e\the\pgfmathfloat@a@E}%
	\pgfmath@smuggleone\pgfmathresult
	\endgroup
}

% decomposes a lowlevel floating point representation into flags,
% mantisse and exponent.
%
% #4: integer register for the flags.
% #5: dimen registers for the mantisse.
% #6: integer register for the exponent.
\def\pgfmathfloat@decompose#1#2e#3\relax#4#5#6{%
   #4=#1\relax
   #5=#2pt%
   #6=#3\relax%
}
% overloaded, #5 needs to be a token register for the mantisse.
\def\pgfmathfloat@decompose@tok#1#2e#3\relax#4#5#6{%
   #4=#1\relax
   #5={#2}%
   #6=#3\relax%
}
% overloaded, returns only ONE of the three components into #4:
\def\pgfmathfloat@decompose@F#1#2e#3\relax#4{#4=#1\relax}%
\def\pgfmathfloat@decompose@M#1#2e#3\relax#4{#4=#2pt}%
\def\pgfmathfloat@decompose@Mtok#1#2e#3\relax#4{#4={#2}}%
\def\pgfmathfloat@decompose@E#1#2e#3\relax#4{#4=#3\relax}%


% Compares #1 with #2 and sets \pgfmathresult either to 1.0 or 0.0.
% 
% It also sets the boolean \ifpgfmathfloatcomparison.
\def\pgfmathfloatlessthan@#1#2{%
%\def\pgfmathfloatlessthan#1#2#3\and#4#5#6{%
	\pgfmathfloatcomparisonfalse
	\begingroup
	\expandafter\pgfmathfloat@decompose#1\relax\pgfmathfloat@a@S\pgfmathfloat@a@M\pgfmathfloat@a@E
	\expandafter\pgfmathfloat@decompose#2\relax\pgfmathfloat@b@S\pgfmathfloat@b@M\pgfmathfloat@b@E
	\ifcase\pgfmathfloat@a@S
		% x = 0 -> (x<y <=> y >0)
		\ifnum1=\pgfmathfloat@b@S
			\aftergroup\pgfmathfloatcomparisontrue
		\fi
	\or
		% x > 0 -> (x<y <=> ( y > 0 && |x| < |y|) )
		\ifnum1=\pgfmathfloat@b@S
			% y>0:
			\pgfmathfloatlessthan@positive
		\fi
	\or
		% x < 0 -> (x<y <=> (y >= 0 ||   |x| > |y|) )
		\ifnum2=\pgfmathfloat@b@S
			% 'y<0':
			\pgfmathfloatgreaterthan@positive
		\else
			\aftergroup\pgfmathfloatcomparisontrue
		\fi
	\else
		\pgfmath@error{Sorry, 'pgfmathfloatlessthan@{#1}{#2}' not yet supported.}%
	\fi
	\endgroup
	\ifpgfmathfloatcomparison
		\def\pgfmathresult{1.0}%
	\else
		\def\pgfmathresult{0.0}%
	\fi
}
\let\pgfmathfloatlessthan=\pgfmathfloatlessthan@

% compares \pgfmathfloat@a@[SME] < \pgfmathfloat@b@[SME]
\def\pgfmathfloatlessthan@positive{%
	\ifnum\pgfmathfloat@a@E<\pgfmathfloat@b@E
		\aftergroup\pgfmathfloatcomparisontrue
	\else
		\ifnum\pgfmathfloat@a@E=\pgfmathfloat@b@E
			\ifdim\pgfmathfloat@a@M<\pgfmathfloat@b@M
				\aftergroup\pgfmathfloatcomparisontrue
			\fi
		\fi
	\fi
}

% compares \pgfmathfloat@a@[SME] > \pgfmathfloat@b@[SME]
\def\pgfmathfloatgreaterthan@positive{%
	\ifnum\pgfmathfloat@a@E>\pgfmathfloat@b@E
		\aftergroup\pgfmathfloatcomparisontrue
	\else
		\ifnum\pgfmathfloat@a@E=\pgfmathfloat@b@E
			\ifdim\pgfmathfloat@a@M>\pgfmathfloat@b@M
				\aftergroup\pgfmathfloatcomparisontrue
			\fi
		\fi
	\fi
}


\def\pgfmathfloatmax@#1#2{%
	\pgfmathfloatlessthan{#1}{#2}%
	\ifpgfmathfloatcomparison
		\edef\pgfmathresult{#2}%
	\else
		\edef\pgfmathresult{#1}%
	\fi
}
\let\pgfmathfloatmax=\pgfmathfloatmax@

\def\pgfmathfloatmin@#1#2{%
	\pgfmathfloatlessthan{#1}{#2}%
	\ifpgfmathfloatcomparison
		\edef\pgfmathresult{#1}%
	\else
		\edef\pgfmathresult{#2}%
	\fi
}
\let\pgfmathfloatmin=\pgfmathfloatmin@

% Converts a composed floating point representation to fixed point.
%
% Example:
% \pgfmathfloattofixed{142e1}
% -> \pgfmathresult = 42.0
\def\pgfmathfloattofixed@#1{%
	\begingroup
	\expandafter\pgfmathfloat@decompose@tok#1\relax\pgfmathfloat@a@S\pgfmathfloat@a@Mtok\pgfmathfloat@a@E
	\ifcase\pgfmathfloat@a@S
		\def\pgfmathresult{0.0}%
	\or
		\expandafter\pgfmathfloattofixed@impl\the\pgfmathfloat@a@Mtok\relax
	\or
		\expandafter\pgfmathfloattofixed@impl\the\pgfmathfloat@a@Mtok\relax
		\edef\pgfmathresult{-\pgfmathresult}%
	\or\def\pgfmathresult{nan}%
	\or\def\pgfmathresult{inf}%
	\or\def\pgfmathresult{-inf}%
	\fi
	\pgfmath@smuggleone\pgfmathresult
	\endgroup
}
\let\pgfmathfloattofixed=\pgfmathfloattofixed@

% Divides or multiplies the input number by 10^#4 using an arithmetic
% left/right shift.
%
% Input:
% #1 a normalised floating point number.
% #2 a positive or negative integer number denoting the shift.
%
% Example:
% \pgfmathfloatshift{11e3}{4}%
% -> pgfmathresult = 11e7
\def\pgfmathfloatshift@#1#2{%
	\begingroup
	\expandafter\pgfmathfloat@decompose@tok#1\relax\pgfmathfloat@a@S\pgfmathfloat@a@Mtok\pgfmathfloat@a@E
	\expandafter\advance\expandafter\pgfmathfloat@a@E#2\relax
	\edef\pgfmathresult{\the\pgfmathfloat@a@S\the\pgfmathfloat@a@Mtok e\the\pgfmathfloat@a@E}%
	\pgfmath@smuggleone\pgfmathresult
	\endgroup
}
\let\pgfmathfloatshift=\pgfmathfloatshift@

% Computes a good guess for log(X.XXXX) where the input number should
% be between 0 < x < 10.
%
% If it is not in this range, the returned value will be bad.
%
% For use in the newton implementation of log(x). Please note that we
% first compute a normalized floating point representation of x, and
% the mantisse is always between 0 and 10.
\def\pgfmathlog@smaller@ten@newton@startval#1.#2\to#3{%
	\ifcase#1
	\pgfmath@error{pgfmathlog@newton: INTERNAL ERROR: should not happen! 
		[\#1=0 should have been caught before!]}%
	\or#3=0.530628pt
	\or#3=0.993pt
	\or#3=1.308pt
	\or#3=1.5pt
	\or#3=1.71pt
	\or#3=1.872pt
	\or#3=2.01pt
	\or#3=2.145pt
	\or#3=2.24pt
	\fi
}

% FIXME : eliminate \pgfmathlog@invoke@expanded ?

% Expands #2 using \edef and invokes #1 with the resulting string.
%
% Example:
%   \pgfmath@y=7.9pt
%   \pgfmathlog@invoke@expanded\pgfmathexp@{{\pgf@sys@tonumber{\pgfmath@y}}}%
% will invoke
%   \pgfmathexp@{7.9}
\def\pgfmathlog@invoke@expanded#1#2{%
	\edef\pgfmath@resulttemp{#2}%
	\expandafter#1\pgfmath@resulttemp
}

% This implementation of log(X) is done with a newton method applied to x-exp(y) = 0.
% The implementation \pgfmathexp is used for exp(y).
%
% see \pgfmathlog@
\def\pgfmathlog@newton#1{%
	\begingroup%
		% compute #1 = M*10^E with normalised mantisse M = [+-]*[1-9].XXXXX
		\pgfmathfloatparsenumber{#1}%
		\expandafter\pgfmathfloat@decompose\pgfmathresult\relax\pgfmathfloat@a@S\pgfmathfloat@a@M\pgfmathfloat@a@E
		\let\pgfmathlog@loopcount=\pgfmathfloat@b@S% this counter is not used here... use it as loop counter.
		%
		\ifnum\pgfmathfloat@a@S=1
		% Now, compute log(#1) = log(M) + E*log(10)
			\expandafter\pgfmathlog@smaller@ten@newton@startval\the\pgfmathfloat@a@M.\to\pgfmathfloat@b@M
			\pgfmathlog@loopcount=0
			\loop
			\ifnum\pgfmathlog@loopcount<3
				\pgfmathlog@invoke@expanded\pgfmathexp@{{\pgf@sys@tonumber{\pgfmathfloat@b@M}}}%
				\let\pgfmathlog@tmp=\pgfmathresult
				\pgfmathlog@invoke@expanded\pgfmathdivide@{{\pgf@sys@tonumber{\pgfmathfloat@a@M}}{\pgfmathlog@tmp}}%
				\expandafter\pgfmath@ya\pgfmathresult pt
				\advance\pgfmathfloat@b@M by\pgfmath@ya
				\advance\pgfmathfloat@b@M by-1pt
				\advance\pgfmathlog@loopcount by1
			\repeat
			\pgfmathfloat@a@M=2.302585pt% = log(10)
			\multiply\expandafter\pgfmathfloat@a@M\pgfmathfloat@a@E
			\advance\pgfmathfloat@b@M by\pgfmathfloat@a@M
			\edef\pgfmathresult{\pgf@sys@tonumber{\pgfmathfloat@b@M}}%
		\else
			\let\pgfmathresult=\pgfutil@empty%
		\fi
	\pgfmath@smuggleone\pgfmathresult
	\endgroup%
}

% Computes log(x) into \pgfmathresult.
%
% REMARK
% This method does NOT use TeX Registers to represent the numbers! The
% computation is COMPLETELY STRING BASED.
% This allows numbers such at 10000000 or 5.23e-10 to be represented
% properly, although TeX-registers would produce overflow/underflow
% errors in these cases. 
%
% The natural logarithm is computed using log(X*10^Y) = log(X) + log(10)*Y
\let\pgfmathlog@=\pgfmathlog@newton

% Public interface method. This is expected to use \pgfmathparse but
% it DOESN'T do that. \pgfmathparse naturally can't handle numbers
% which exceed TeX's numerical capabilities. However, exponentials are
% typically either too large or too small for TeX.
%
% @see \pgfmathnormalisedfloatingpoint for how we parse arguments.
% @see \pgfmathlog@
\let\pgfmathlog=\pgfmathlog@newton

% Usage:
% \pgfmathapproxequaltomacro\argone\argtwo\result
%
% -> will set \resulttrue or \resultfalse
% -> check with \ifresult
\def\pgfmathapproxequaltomacro#1#2#3{%
	\begingroup%
		\expandafter\pgfmath@x#1pt\relax%
		\expandafter\pgfmath@y#2pt\relax%
		\advance\pgfmath@x-\pgfmath@y%
		\ifdim\pgfmath@x<0pt
			\multiply\pgfmath@x by-1
		\fi
		\ifdim\pgfmath@x<0.0001pt\relax%
			\expandafter\aftergroup\csname #3true\endcsname
		\else%
			\expandafter\aftergroup\csname #3false\endcsname
		\fi%
	\endgroup%
}

% Assigns \pgfmathresult to 1.0 if #1 ~= #1.
%
% It will also set the boolean \ifpgfmathcomparison accordingly.
\def\pgfmathapproxequalto#1#2{%
	\begingroup%
		\expandafter\pgfmath@x#1pt\relax%
		\expandafter\pgfmath@y#2pt\relax%
		\advance\pgfmath@x-\pgfmath@y%
		\ifdim\pgfmath@x<0pt
			\multiply\pgfmath@x by-1
		\fi
		\ifdim\pgfmath@x<0.0001pt\relax%
			\def\pgfmathresult{1.0}%
			\aftergroup\pgfmathcomparisontrue
		\else%
			\def\pgfmathresult{0.0}%
			\aftergroup\pgfmathcomparisonfalse
		\fi%
	\pgfmath@smuggleone\pgfmathresult
	\endgroup%
}


% Rounds a fixed point number #1 to \pgfmathfloat@round@precision digits precision and returns
% the result into \pgfmathresult.
%
% Any trailing zeros after the period are discarded.
% See \pgfmathroundtozerofill if you want trailing zeros and fixed
% width.
%
% This method is PURELY text based and can work with arbirtrary
% precision (well, limited to TeX's stack size and integer size).
%
% Examples:
% \pgfmathroundto{1}
% -> \pgfmathresult = '1'
%
% \pgfmathroundto{19999.9996}
% -> \pgfmathresult = '20000'
%
% Arguments:
% #1 may be either a number or a macro expanding to a number.
%
% SIDE EFFECT: the global variable \ifpgfmathfloatroundhasperiod will be set.
\def\pgfmathroundto#1{%
	\pgfmathfloatroundhasperiodtrue
	\begingroup
	\pgfmathfloat@fixed@digits@after@periodfalse
	\pgfmathroundto@impl{#1}%
	\pgfmath@smuggleone\pgfmathresult
	\endgroup
}


% Overloaded method. 
%
% This variant always uses a FIXED number behind the period and fills
% in zeros if necessary.
% Examples:
% \pgfmathroundtozerofill{1}
% -> \pgfmathresult = '1.00'
%
% \pgfmathroundtozerofill{19999.9996}
% -> \pgfmathresult = '20000.00'
%
% SIDE EFFECT: the global variable \ifpgfmathfloatroundhasperiod will be set.
\def\pgfmathroundtozerofill#1{%
	\pgfmathfloatroundhasperiodtrue
	\begingroup
	\pgfmathfloat@fixed@digits@after@periodtrue
	\pgfmathroundto@impl{#1}%
	\pgfmath@smuggleone\pgfmathresult
	\endgroup
}


% #1 maybe a macro
\def\pgfmathprintnumber@fixed@stylePERIOD#1{%
	\pgfutilensuremath{#1}%
}

% #1 maybe a macro
\def\pgfmathprintnumber@fixed@styleCOMMA#1{%
	\pgfutilensuremath{%
	\ifpgfmathfloatroundhasperiod
		\expandafter\pgfmathprintnumber@fixed@styleCOMMA@impl#1\pgfmathfloat@EOI
	\else
		#1%
	\fi
	}%
}
\def\pgfmathprintnumber@fixed@styleCOMMA@impl#1.#2\pgfmathfloat@EOI{#1{,}#2}%

% The default style to display fixed point numbers.
%
% It does not apply numerics, but it is responsable to typeset the
% rounded number.
% It can access the \ifpgfmathfloatroundhasperiod boolean.
%
% Input: #1  the floating point number (maybe a macro).
\let\pgfmathprintnumber@fixed@style=\pgfmathprintnumber@fixed@stylePERIOD


% Rounds a normalized floating point number to \pgfmathfloat@round@precision 
% digits precision and writes the result to \pgfmathresult.
%
% This method uses \pgfmathroundto for the mantisse.
%
% @see pgfmathfloatroundzerofill
%
% SIDE EFFECT: the global variable \ifpgfmathfloatroundhasperiod will be set to
% whether the final mantisse #5 has a period or not.
\def\pgfmathfloatround#1{%
	\pgfmathfloatroundhasperiodtrue
	\begingroup
	\pgfmathfloat@fixed@digits@after@periodfalse
	\expandafter\pgfmathfloat@decompose@tok#1\relax\pgfmathfloat@a@S\pgfmathfloat@a@Mtok\pgfmathfloat@a@E
	\pgfmathfloatround@impl
	\pgfmath@smuggleone\pgfmathresult
	\endgroup
}

% Overload.
%
% This method uses a fixed width for the mantisse and fills in zeros
% if necessary.
%
% This method uses \pgfmathroundtozerofill for the mantisse.
%
% SIDE EFFECT: the global variable \ifpgfmathfloatroundhasperiod will be set to
% whether the final mantisse #5 has a period or not.
\def\pgfmathfloatroundzerofill#1{%
	\pgfmathfloatroundhasperiodtrue
	\begingroup
	\pgfmathfloat@fixed@digits@after@periodtrue
	\expandafter\pgfmathfloat@decompose@tok#1\relax\pgfmathfloat@a@S\pgfmathfloat@a@Mtok\pgfmathfloat@a@E
	\pgfmathfloatround@impl
	\pgfmath@smuggleone\pgfmathresult
	\endgroup
}

\newif\ifpgfmathfloatround@allow@empty@mantisse
\def\pgfmathfloatround@mantisse@ONE{1.0}%

\def\pgfutilensuremath#1{%
	\ifmmode#1\else$#1$\fi
}

% #1: sign
% #2: mantisse
% #3: CODE to display if the mantisse is drawn.
% #4: CODE to display if the mantisse is NOT draw.
% #5: CODE to display the exponent.
\def\pgfmathfloatrounddisplaystyle@shared@impl#1#2#3#4#5{%
	\expandafter\ifcase#1\relax
		\pgfmathprintnumber@fixed@style{#2}%
			#3%
			#5%
	\or\pgfmathprintnumber@fixed@style{#2}%
		#3%
		#5%
	\or-\pgfmathprintnumber@fixed@style{#2}#3#5%
	\or NaN
	\or\infty%
	\or-\infty%
	\fi
}

\def\pgfmathfloatrounddisplaystyle@std#1#2e#3\relax{%
	\pgfutilensuremath{\pgfmathfloatrounddisplaystyle@shared@impl{#1}{#2}{\cdot}{}{10^{#3}}}%
}
\def\pgfmathfloatrounddisplaystyle@subscript#1#2e#3\relax{%
	\pgfutilensuremath{\pgfmathfloatrounddisplaystyle@shared@impl{#1}{#2}{}{1}{_{#3}}}%
}
\def\pgfmathfloatrounddisplaystyle@e#1#2e#3\relax{%
	\pgfmathfloatrounddisplaystyle@shared@impl{#1}{#2}{}{1}{%
		e\ifnum#3<0\relax
			{\count0=#3\relax
			\multiply\count0 by-1
			{-}\the\count0
			}%
		\else
			{+}#3%
		\fi
	}%
}
\def\pgfmathfloatrounddisplaystyle@E#1#2e#3\relax{%
	\pgfmathfloatrounddisplaystyle@shared@impl{#1}{#2}{}{1}{%
		E\ifnum#3<0\relax
			{\count0=#3\relax
			\multiply\count0 by-1
			{-}\the\count0
			}%
		\else
			{+}#3%
		\fi
	}%
}

% A macro which takes three arguments, sign, mantisse and exponent and
% expands to the final TeX-representation for that floating point
% number.
%
% PRECONDITION:
%   the floating point number has already been rounded properly and
%   the mantisse has been rounded correcty.
%
% #1: SIGN
% #2: MANTISSE
% #3: EXPONENT
\let\pgfmathfloatrounddisplaystyle=\pgfmathfloatrounddisplaystyle@std

\pgfkeys{%
	/pgf/number format/.is family,
	/pgf/number format,
	fixed/.code=			\pgfmath@set@number@printer{pgfmathprintnumber@FIXED},
	fixed zerofill/.code=	\pgfmath@set@number@printer{pgfmathprintnumber@FIXED@ZEROFILL},
	sci/.code=				\pgfmath@set@number@printer{pgfmathprintnumber@SCI},
	sci zerofill/.code=		\pgfmath@set@number@printer{pgfmathprintnumber@SCI@ZEROFILL},
	std/.code=				\pgfmath@set@number@printer{pgfmathprintnumber@STD},
	int detect/.code=		\pgfmath@set@number@printer{pgfmathprintnumber@INT@DETECT},
	int trunc/.code=		\pgfmath@set@number@printer{pgfmathprintnumber@INT@TRUNC},
	precision/.store in=	\pgfmathfloat@round@precision,
	use period/.code=		{\let\pgfmathprintnumber@fixed@style=\pgfmathprintnumber@fixed@stylePERIOD},
	use comma/.code=		{\let\pgfmathprintnumber@fixed@style=\pgfmathprintnumber@fixed@styleCOMMA},
	sci 10^e/.code=			{\let\pgfmathfloatrounddisplaystyle=\pgfmathfloatrounddisplaystyle@std},
	sci 10e/.code=			{\let\pgfmathfloatrounddisplaystyle=\pgfmathfloatrounddisplaystyle@std},
	sci e/.code=			{\let\pgfmathfloatrounddisplaystyle=\pgfmathfloatrounddisplaystyle@e},
	sci E/.code=			{\let\pgfmathfloatrounddisplaystyle=\pgfmathfloatrounddisplaystyle@E},
	sci subscript/.code=	{\let\pgfmathfloatrounddisplaystyle=\pgfmathfloatrounddisplaystyle@subscript},
%	sci may skip mantisse/.is if=pgfmathfloatround@allow@empty@mantisse,
%	sci may skip mantisse/.default=true,
}






\def\pgfmathprintnumber@STD#1{%
	\begingroup
	% parse the input:
	\pgfmathfloatparsenumber{#1}%
	\expandafter\pgfmathprintnumber@STD@issci\pgfmathresult\relax
	\endgroup
}

\def\pgfmathprintnumber@STD@issci#1#2e#3\relax{%
	\expandafter\ifnum#1<3
		\expandafter\ifnum#3>4
			\pgfmathprintnumber@SCI@issci#1#2e#3\relax%
		\else
			\begingroup
			\expandafter\c@pgf@counta\pgfmathfloat@round@precision
			\divide\c@pgf@counta by-2
			\expandafter\ifnum#3<\c@pgf@counta
				\pgfmathprintnumber@SCI@issci#1#2e#3\relax%
			\else
				\pgfmathprintnumber@FIXED@issci#1#2e#3\relax%
			\fi
			\endgroup
		\fi
	\else% nan or inf:
		\pgfmathfloatrounddisplaystyle#1#2e#3\relax%
	\fi
}


\def\pgfmathprintnumber@INT@TRUNC#1{%
	\begingroup
	\pgfmathfloatparsenumber{#1}%
	\expandafter\pgfmathprintnumber@INT@TRUNC@issci\pgfmathresult\relax
	\endgroup
}

\def\pgfmathprintnumber@INT@TRUNC@impl#1.#2\relax{%
	#1%
}
\def\pgfmathprintnumber@INT@TRUNC@issci#1#2e#3\relax{%
	\begingroup
	\ifnum#1<3\relax
		\pgfmathfloattofixed{#1#2e#3}%
		\expandafter\pgfmathprintnumber@INT@TRUNC@impl\pgfmathresult\relax
	\else
		\pgfmathfloatrounddisplaystyle#1#2e#3\relax%
	\fi
	\endgroup
}


\def\pgfmathprintnumber@INT@DETECT#1{%
	\begingroup
	\pgfmathfloatparsenumber{#1}%
	\expandafter\pgfmathprintnumber@INT@DETECT@issci\pgfmathresult\relax
	\endgroup
}

\def\pgfmathprintnumber@INT@DETECT@issci#1#2e#3\relax{%
	\begingroup
	\ifnum#1<3\relax
		\pgfmathfloattofixed{#1#2e#3}%
		\def\pgfmathfloat@round@precision{6}%
		\expandafter\pgfmathroundto\expandafter{\pgfmathresult}%
		\ifpgfmathfloatroundhasperiod
			\pgfmathprintnumber@SCI@issci#1#2e#3\relax
		\else
			\pgfmathresult
		\fi
	\else
		\pgfmathfloatrounddisplaystyle#1#2e#3\relax%
	\fi
	\endgroup
}

\def\pgfmathprintnumber@FIXED#1{%
	\begingroup
	\pgfmathfloatparsenumber{#1}%
	\expandafter\pgfmathprintnumber@FIXED@issci\pgfmathresult\relax
	\endgroup
}

\def\pgfmathprintnumber@FIXED@issci#1#2e#3\relax{%
	\begingroup
	\ifnum#1<3
		\pgfmathfloattofixed{#1#2e#3}%
		\expandafter\pgfmathroundto\expandafter{\pgfmathresult}%
		\expandafter\pgfmathprintnumber@fixed@style\expandafter{\pgfmathresult}%
	\else% nan or inf:
		\pgfmathfloatrounddisplaystyle#1#2e#3\relax%
	\fi
	\endgroup
}

% Expands the NUMBER (without units) to a pretty printed string argument: rounded to
% several digits, discard any suffix zeros etc.
%
% It uses \pgfmathfloat@round@precision digits after the comma.
\def\pgfmathprintnumber@FIXED@ZEROFILL#1{%
	\begingroup
	\pgfmathfloatparsenumber{#1}%
	\expandafter\pgfmathprintnumber@FIXED@ZEROFILL@issci\pgfmathresult\relax
	\endgroup
}

\def\pgfmathprintnumber@FIXED@ZEROFILL@issci#1#2e#3\relax{%
	\begingroup
	\ifnum#1<3\relax
		\pgfmathfloattofixed{#1#2e#3}%
		\expandafter\pgfmathroundtozerofill\expandafter{\pgfmathresult}%
		\expandafter\pgfmathprintnumber@fixed@style\expandafter{\pgfmathresult}%
	\else% nan or inf:
		\pgfmathfloatrounddisplaystyle#1#2e#3\relax%
	\fi
	\endgroup
}

\def\pgfmathprintnumber@SCI@ZEROFILL#1{%
	\begingroup
	\pgfmathfloatparsenumber{#1}%
	\expandafter\pgfmathprintnumber@SCI@ZEROFILL@issci\pgfmathresult\relax
	\endgroup
}

\def\pgfmathprintnumber@SCI@ZEROFILL@issci#1#2e#3\relax{%
	\begingroup
	\pgfmathfloatroundzerofill{#1#2e#3}%
	\expandafter\pgfmathfloatrounddisplaystyle\pgfmathresult\relax
	\endgroup
}
\def\pgfmathprintnumber@SCI#1{%
	\begingroup
	\pgfmathfloatparsenumber{#1}%
	\expandafter\pgfmathprintnumber@SCI@issci\pgfmathresult\relax
	\endgroup
}

\def\pgfmathprintnumber@SCI@issci#1#2e#3\relax{%
	\begingroup
	\pgfmathfloatround{#1#2e#3}%
	\expandafter\pgfmathfloatrounddisplaystyle\pgfmathresult\relax
	\endgroup
}

% Prints argument #1 using the current pretty printer environment (all
% variables in /pgf/number format).
%
% You may specify optional arguments with \pgfmathprintnumber[...].
\def\pgfmathprintnumber{%
	\pgfutil@ifnextchar[%
		{\pgfmathprintnumber@OPT}%
		{\pgfmathprintnumber@}%
}

\def\pgfmathprintnumber@OPT[#1]#2{%
	\begingroup
	\pgfqkeys{/pgf/number format}{#1}%
	\pgfmathprintnumber@{#2}%
	\endgroup
}%
	

% Changes the current number pretty printer to #1.
%
% #1 is the macro base name for the pretty print routine, without the
% leading '\'.
\def\pgfmath@set@number@printer#1{%
	\expandafter\let\expandafter\pgfmathprintnumber@\csname #1\endcsname
	\expandafter\let\expandafter\pgfmathprintnumber@issci\csname #1@issci\endcsname
}

\pgfmath@set@number@printer{pgfmathprintnumber@STD}
%\pgfmath@set@number@printer{pgfmathprintnumber@FIXED}
%\pgfmath@set@number@printer{pgfmathprintnumber@FIXED@ZEROFILL}
%\pgfmath@set@number@printer{pgfmathprintnumber@SCI@ZEROFILL}
%\pgfmath@set@number@printer{pgfmathprintnumber@SCI}

%%%%%%%%%%%%%%%%%%%%%%%%%%
% 
% IMPL
%
%%%%%%%%%%%%%%%%%%%%%%%%%%

% equals only itsself when compared with \ifx:
\def\pgfmathfloat@EOI{\pgfmathfloat@EOI}%

% Re-use counters internally.
%
% They are always grouped and only used inside of the rounding
% routines.
\let\c@pgfmathroundto@prec=\pgfmathfloat@b@S% ATTENTION: DOUBLE-USED REGISTERS!
\let\c@pgfmathroundto@offsetbehindperiod=\pgfmathfloat@b@E
\newcount\c@pgfmathroundto@lastzeros

\newif\ifpgfmathround@impl@PREPERIOD@is@negative@zero

% PRECONDITION:
%  	\ifpgfmathfloatroundhasperiod=\iftrue  holds outside of the
%  	current TeX group.
%
% POSTCONDITION:
%   \ifpgfmathfloatroundhasperiod will be set correctly AFTER the
%   current TeX group.
\def\pgfmathroundto@impl#1{%
	\pgfmathfloat@tmptoks={}%
	\let\pgfmathround@next=\pgfutil@empty
	\let\pgfmathround@cur=\pgfutil@empty
	\let\pgfmathresult=\pgfutil@empty
	\edef\pgfmathround@input{#1}%
	\expandafter\c@pgfmathroundto@prec\pgfmathfloat@round@precision\relax
	\c@pgfmathroundto@lastzeros=0
	\c@pgfmathroundto@offsetbehindperiod=-1 % means: no period found so far
	\pgfmathround@impl@PREPERIOD@is@negative@zerotrue
	\expandafter\pgfmathroundto@impl@ITERATE@NODOT@firstcall\pgfmathround@input\pgfmathfloat@EOI
}

% \pgfmathroundto implementation in WORDS:
%
% coarse idea: 
%    1. collect all digits/sign BEFORE the first period in REVERSE order
%    2. then, collect UP TO \prec digits after the period in REVERSE order
% Steps 1. and 2. lead to the digit [sign] sequence
%    "x_{-p} x_{-p+1} ... x_{-2} x_{-1} '.' x_0 ... x_r"
% where 'r' is the total number of digits. The integer 'p' denotes the
% ACTUALLY collected number of digits behind the period.
%
% Let 'k' be the desired precision.
%
% Please note that pgfmathroundto rounds the mantisse, that means |abs(x)|.
%
% There are exactly TWO cases:
% 1. The case with p<=k and x_{-p-1} = end of input.
% 2. The case with p=k and x_{-p-1} is a further, next character.
%
% Then, for case 1.):
% 	discard any unused zeros at the tail of our number (possibly
% 	including the period)
%
% and in case 2.)
%   if NEXT DIGIT < 5:
%       do exactly the same as in case 1.) above and discard any
%       following digits.
%   else
%   	let q := -p
%   	while(x_q = 9 and q<=r )
%   		if q>=0
%   			set x_q = '0'
%   		else
%   			discard digit x_q='9'
%   		fi
%   		++q
%   		if q=0
%   			discard the period 
%   		fi
%   	end while
%   	if q = r+1
%   		insert a '1'
%   	else
%   		set x_q = x_q + 1
%   	fi
%   fi
%
% All these loops have been implemented in spaghetti-code below.
% Sorry, I fear its hard to understand. In principle, everything is
% realised using more or less finite state machines (with some number
% counting logic).
%
% Some comments:
%   - The token register \pgfmathfloat@tmptoks is used to accumulate the REVERSED input number.
%	- \pgfmathfloat@EOI always denotes 'END OF INPUT'.
%	- in the second stage, we need to reverse \pgfmathfloat@tmptoks.
%	  This is -again- done with \pgfmathfloat@tmptoks.

\def\pgfmathroundto@impl@discard@period#1.#2\pgfmathfloat@EOI{%
	\pgfmathfloatroundhasperiodfalse
	\aftergroup\pgfmathfloatroundhasperiodfalse
	\def\pgfmathresult{#1}%
}

\def\pgfmathroundto@impl@gobble@rest@and@start#1\pgfmathfloat@EOI{%
	\pgfmathroundto@impl@start
}
\def\pgfmathroundto@impl@gobble@and@start\pgfmathfloat@EOI{%
	\pgfmathroundto@impl@start
}
\def\pgfmathroundto@impl@gobble\pgfmathfloat@EOI{}%
\def\pgfmathroundto@impl@gobble@rest#1\pgfmathfloat@EOI{}%

% This method will be invoked as soon as the first step, the reverse
% collection of up to PREC digits after the period, has finished.
\def\pgfmathroundto@impl@start{%
	\ifx\pgfmathround@next\pgfutil@empty
		\ifnum\c@pgfmathroundto@offsetbehindperiod<0
			% no period found.
			\ifpgfmathfloat@fixed@digits@after@period
				\ifnum\c@pgfmathroundto@prec=0\relax
					\pgfmathfloatroundhasperiodfalse
					\aftergroup\pgfmathfloatroundhasperiodfalse
					\edef\pgfmathresult{\pgfmathround@input}%
				\else
					\pgfmathfloat@tmptoks=\expandafter{\pgfmathround@input.}%
					\c@pgfmathroundto@offsetbehindperiod=\c@pgfmathroundto@prec
					\pgfmathroundto@impl@append@zeros
					\edef\pgfmathresult{\the\pgfmathfloat@tmptoks}%
				\fi
			\else
				\pgfmathfloatroundhasperiodfalse
				\aftergroup\pgfmathfloatroundhasperiodfalse
				\edef\pgfmathresult{\pgfmathround@input}%
			\fi
		\else
			%\ifnum\c@pgfmathroundto@offsetbehindperiod>\c@pgfmathroundto@prec
			%	\pgfmath@error{Internal logic error in pgfmathroundto at [I] - should not have happened!?}%
			%\fi
			\pgfmathroundto@impl@finish@with@truncation
		\fi
	\else
		%\ifnum\c@pgfmathroundto@offsetbehindperiod=\c@pgfmathroundto@prec
		%\else
		%	\pgfmath@error{Internal logic error in pgfmathroundto at [II] - should not have happened!? I have offsetbehindperiod=\the\c@pgfmathroundto@offsetbehindperiod and prec = \the\c@pgfmathroundto@prec}%
		%\fi
		\expandafter\ifnum\pgfmathround@next<5\relax
			\pgfmathroundto@impl@finish@with@truncation
		\else
			\multiply\c@pgfmathroundto@offsetbehindperiod by-1
			\expandafter\pgfmathroundto@impl@ADD@ONE\the\pgfmathfloat@tmptoks\pgfmathfloat@EOI
		\fi
	\fi
}
			
% takes the current input and decides whether trailing zeros shall be
% discarded or more zeros need to be filled in.
\def\pgfmathroundto@impl@finish@with@truncation{%
	\ifpgfmathfloat@fixed@digits@after@period
		\expandafter\pgfmathroundto@impl@REVERSE\the\pgfmathfloat@tmptoks\pgfmathfloat@EOI
		\ifnum\c@pgfmathroundto@lastzeros=\c@pgfmathroundto@offsetbehindperiod
			\ifpgfmathround@impl@PREPERIOD@is@negative@zero
				% write '0.0000' instead of '-0.0000':
				\expandafter\pgfmathroundto@impl@discard@minus\the\pgfmathfloat@tmptoks\pgfmathfloat@EOI
			\fi
		\fi
		\ifnum\c@pgfmathroundto@prec=0\relax
			\expandafter\pgfmathroundto@impl@discard@period\the\pgfmathfloat@tmptoks\pgfmathfloat@EOI
		\else
			\advance\c@pgfmathroundto@prec by-\c@pgfmathroundto@offsetbehindperiod
			\c@pgfmathroundto@offsetbehindperiod=\c@pgfmathroundto@prec
			\pgfmathroundto@impl@append@zeros
			\edef\pgfmathresult{\the\pgfmathfloat@tmptoks}%
		\fi
	\else
		\pgfmathroundto@impl@discard@suffix@zeros
	\fi
}

\def\pgfmathroundto@impl@discard@minus-#1\pgfmathfloat@EOI{\pgfmathfloat@tmptoks={#1}}%

% appends \c@pgfmathroundto@offsetbehindperiod zeros at the end of \pgfmathfloat@tmptoks.
\def\pgfmathroundto@impl@append@zeros{%
	\ifnum\c@pgfmathroundto@offsetbehindperiod>0
		\pgfmathfloat@tmptoks=\expandafter{\the\pgfmathfloat@tmptoks0}%
		\advance\c@pgfmathroundto@offsetbehindperiod by-1
		\pgfmathroundto@impl@append@zeros
	\fi
}

\def\pgfmathroundto@impl@ADD@ONE{%
	\pgfmathfloat@tmptoks={}% no longer needed because its old value will be read from input
	\pgfmathroundto@impl@ADD@ONE@ITERATE
}
\def\pgfmathroundto@impl@ADD@ONE@ITERATE{%
	\pgfutil@ifnextchar\pgfmathfloat@EOI{%
		\pgfmathfloat@round@may@need@renormalizetrue
		\edef\pgfmathresult{1\the\pgfmathfloat@tmptoks}%
		\pgfmathroundto@impl@gobble
	}{%
		\pgfutil@ifnextchar.{%
			\ifnum\c@pgfmathroundto@prec=0
				% silently discard period in special case precision=0
				\def\pgfmathround@nextcmd{\pgfmathroundto@impl@ADD@ONE@ITERATE@gobble@dot}%
			\else
				\def\pgfmathround@nextcmd{\pgfmathroundto@impl@ADD@ONE@NEXT@COLLECT}%
			\fi
			\pgfmathround@nextcmd
		}{%
			\pgfutil@ifnextchar+{%
				\pgfmathfloat@round@may@need@renormalizetrue
				\edef\pgfmathresult{1\the\pgfmathfloat@tmptoks}%
				\pgfmathroundto@impl@gobble@rest
			}{%
				\pgfutil@ifnextchar-{%
					\pgfmathfloat@round@may@need@renormalizetrue
					\edef\pgfmathresult{-1\the\pgfmathfloat@tmptoks}%
					\pgfmathroundto@impl@gobble@rest
				}{%
					\pgfmathroundto@impl@ADD@ONE@NEXT
				}%
			}%
		}%
	}%
}

\def\pgfmathroundto@impl@ADD@ONE@ITERATE@gobble@dot.{%
	\pgfmathfloatroundhasperiodfalse
	\aftergroup\pgfmathfloatroundhasperiodfalse
	\pgfmathroundto@impl@ADD@ONE@ITERATE
}

\def\pgfmathroundto@impl@ADD@ONE@NEXT@COLLECT#1{%
	\pgfmathfloat@tmptoks=\expandafter{\expandafter#1\the\pgfmathfloat@tmptoks}%
	\pgfmathroundto@impl@ADD@ONE@ITERATE
}
\def\pgfmathroundto@impl@ADD@ONE@NEXT#1{%
	\ifnum#1=9
		\ifnum\c@pgfmathroundto@offsetbehindperiod<0
			\ifpgfmathfloat@fixed@digits@after@period
				\pgfmathfloat@tmptoks=\expandafter{\expandafter0\the\pgfmathfloat@tmptoks}%
			\else
				% silently DROP digit
			\fi
		\else
			\pgfmathfloat@tmptoks=\expandafter{\expandafter0\the\pgfmathfloat@tmptoks}%
		\fi
		\advance\c@pgfmathroundto@offsetbehindperiod by1
		\ifnum\c@pgfmathroundto@offsetbehindperiod=0
			\ifpgfmathfloat@fixed@digits@after@period
				\def\pgfmathround@nextcmd{\pgfmathroundto@impl@ADD@ONE@ITERATE}%
			\else
				\def\pgfmathround@nextcmd{\pgfmathroundto@impl@ADD@ONE@ITERATE@gobble@dot}%
			\fi
		\else
			\def\pgfmathround@nextcmd{\pgfmathroundto@impl@ADD@ONE@ITERATE}%
		\fi
	\else
		% re-use this counter:
		\c@pgfmathroundto@lastzeros=#1
		\advance\c@pgfmathroundto@lastzeros by1
		\edef\pgfmathresult{\the\c@pgfmathroundto@lastzeros\the\pgfmathfloat@tmptoks}%
		\pgfmathfloat@tmptoks=\expandafter{\pgfmathresult}%
		\def\pgfmathround@nextcmd{\pgfmathroundto@impl@REVERSE@ITERATE}%
	\fi
	\pgfmathround@nextcmd
}


\def\pgfmathroundto@impl@discard@suffix@zeros{%
	\ifnum\c@pgfmathroundto@lastzeros=\c@pgfmathroundto@offsetbehindperiod
		\ifpgfmathround@impl@PREPERIOD@is@negative@zero
			\pgfmathfloatroundhasperiodfalse
			\aftergroup\pgfmathfloatroundhasperiodfalse
			\def\pgfmathresult{0}% write '0' instead of '-0'
		\else
			\expandafter\pgfmathroundto@impl@discard@period\pgfmathround@input\pgfmathfloat@EOI
		\fi
	\else
		\ifnum\c@pgfmathroundto@lastzeros=0
			\expandafter\pgfmathroundto@impl@REVERSE\the\pgfmathfloat@tmptoks\pgfmathfloat@EOI
		\else
			\expandafter\pgfmathroundto@impl@discard@suffix@zeros@ITERATE\the\pgfmathfloat@tmptoks\pgfmathfloat@EOI
		\fi
	\fi
}

% PRECONDITION: 
%    \c@pgfmathroundto@lastzeros > 0
\def\pgfmathroundto@impl@discard@suffix@zeros@ITERATE#1{%
	\advance\c@pgfmathroundto@lastzeros by-1
	\ifnum\c@pgfmathroundto@lastzeros=0
		\def\pgfmathround@nextcmd{\pgfmathroundto@impl@REVERSE}%
	\else
		\def\pgfmathround@nextcmd{\pgfmathroundto@impl@discard@suffix@zeros@ITERATE}%
	\fi
	\pgfmathround@nextcmd
}

% Usage:
% \pgfmathroundto@impl@REVERSE#1\pgfmathfloat@EOI
% -> writes #1 reversed into \pgfmathfloat@tmptoks
\def\pgfmathroundto@impl@REVERSE{%
	\pgfmathfloat@tmptoks={}%
	\pgfmathroundto@impl@REVERSE@ITERATE
}

\def\pgfmathroundto@impl@REVERSE@ITERATE{%
	\pgfutil@ifnextchar\pgfmathfloat@EOI{%
		\edef\pgfmathresult{\the\pgfmathfloat@tmptoks}%
		\pgfmathroundto@impl@gobble
	}{%
		\pgfmathroundto@impl@REVERSE@NEXT
	}%
}

\def\pgfmathroundto@impl@REVERSE@NEXT#1{%
	\pgfmathfloat@tmptoks=\expandafter{\expandafter#1\the\pgfmathfloat@tmptoks}%
	\pgfmathroundto@impl@REVERSE@ITERATE
}

\def\pgfmathroundto@impl@BEGIN@DOT.{%
	\pgfmathfloat@tmptoks=\expandafter{\expandafter.\the\pgfmathfloat@tmptoks}%
	\c@pgfmathroundto@offsetbehindperiod=0
	\pgfmathroundto@impl@ITERATE@DOT
}
\def\pgfmathroundto@impl@ITERATE@DOT{%
	\pgfutil@ifnextchar\pgfmathfloat@EOI{%
		% finished.
		\pgfmathroundto@impl@gobble@and@start
	}{%
		\pgfmathroundto@impl@NEXT@DOT
	}%
}
\def\pgfmathroundto@impl@NEXT@DOT#1{%
	\ifnum\c@pgfmathroundto@offsetbehindperiod=\c@pgfmathroundto@prec
		\def\pgfmathround@next{#1}%
		\def\pgfmathround@nextcmd{\pgfmathroundto@impl@gobble@rest@and@start}%
	\else
		\advance\c@pgfmathroundto@offsetbehindperiod by1
		\ifnum#1=0
			\advance\c@pgfmathroundto@lastzeros by1
		\else
			\c@pgfmathroundto@lastzeros=0
		\fi
		\pgfmathfloat@tmptoks=\expandafter{\expandafter#1\the\pgfmathfloat@tmptoks}%
		\def\pgfmathround@nextcmd{\pgfmathroundto@impl@ITERATE@DOT}%
	\fi
	\pgfmathround@nextcmd
}%

% in contrast to pgfmathroundto@impl@ITERATE@NODOT, this method here
% checks also for signs
\def\pgfmathroundto@impl@ITERATE@NODOT@firstcall#1{%
	\def\pgfmathround@cur{#1}%
	\ifx\pgfmathround@cur\pgfmathfloat@EOI
		\pgfmathround@impl@PREPERIOD@is@negative@zerofalse
		\def\pgfmathround@nextcmd{\pgfmathroundto@impl@start}%
	\else
		\ifx\pgfmathround@cur\pgfmathfloatparsenumber@tok@PERIOD
			\pgfmathround@impl@PREPERIOD@is@negative@zerofalse
			\def\pgfmathround@nextcmd{\pgfmathroundto@impl@BEGIN@DOT.}%
		\else
			\ifx\pgfmathround@cur\pgfmathfloatparsenumber@tok@MINUS
			\else
				\pgfmathround@impl@PREPERIOD@is@negative@zerofalse
			\fi
			\pgfmathfloat@tmptoks=\expandafter{\expandafter#1\the\pgfmathfloat@tmptoks}%
			\def\pgfmathround@nextcmd{\pgfmathroundto@impl@ITERATE@NODOT}%
		\fi
	\fi
	\pgfmathround@nextcmd
}

\def\pgfmathroundto@impl@ITERATE@NODOT#1{%
	\def\pgfmathround@cur{#1}%
	\ifx\pgfmathround@cur\pgfmathfloat@EOI
		\def\pgfmathround@nextcmd{\pgfmathroundto@impl@start}%
	\else
		\ifx\pgfmathround@cur\pgfmathfloatparsenumber@tok@PERIOD
			\def\pgfmathround@nextcmd{\pgfmathroundto@impl@BEGIN@DOT.}%
		\else
			\ifx\pgfmathround@cur\pgfmathfloatparsenumber@tok@ZERO
			\else
				\pgfmathround@impl@PREPERIOD@is@negative@zerofalse
			\fi
			\pgfmathfloat@tmptoks=\expandafter{\expandafter#1\the\pgfmathfloat@tmptoks}%
			\def\pgfmathround@nextcmd{\pgfmathroundto@impl@ITERATE@NODOT}%
		\fi
	\fi
	\pgfmathround@nextcmd
}

%--------------------------------------------
% END of pgfmathroundto implementation.
%--------------------------------------------


% flags, mantisse and exponent has already been stored into
% \pgfmathfloat@a@* [using ...@Mtok]
%
% PRECONDITION:
%   \ifpgfmathfloatroundhasperiod = \iftrue outside of the current
%   group
%
% POSTCONDITION:
%   \ifpgfmathfloatroundhasperiod will be set after the current TeX
%   group.
\def\pgfmathfloatround@impl{%
	\pgfmathfloat@round@may@need@renormalizefalse
	\expandafter\pgfmathroundto@impl\expandafter{\the\pgfmathfloat@a@Mtok}%
	\ifpgfmathfloat@round@may@need@renormalize
		\ifpgfmathfloatroundhasperiod
			\expandafter\pgfmathfloatround@impl@renormalize\pgfmathresult\pgfmathfloat@EOI%
		\else
			\expandafter\pgfmathfloatround@impl@renormalize\pgfmathresult.\pgfmathfloat@EOI%
		\fi
		\advance\pgfmathfloat@a@E by1
	\fi
	\edef\pgfmathresult{\the\pgfmathfloat@a@S\pgfmathresult e\the\pgfmathfloat@a@E}%
}

\def\pgfmathfloatround@impl@renormalize#1#2.#3\pgfmathfloat@EOI{%
	\pgfmathroundto@impl{#1.#2#3}%
}





% ATTENTION: this thing REQUIRES a period in the mantisse!
% collects everything into \pgfmathresult
\def\pgfmathfloattofixed@impl#1.#2\relax{%
	\ifnum\pgfmathfloat@a@E<0
		\pgfmathfloat@a@Mtok={0.}%
		\loop
		\ifnum\pgfmathfloat@a@E<-1
			\pgfmathfloat@a@Mtok=\expandafter{\the\pgfmathfloat@a@Mtok 0}%
			\advance\pgfmathfloat@a@E by1
		\repeat
		\def\pgfmathfloat@TMPB{#2}%
		\def\pgfmathfloat@TMPC{0}%
		\ifx\pgfmathfloat@TMPB\pgfmathfloat@TMPC
			\edef\pgfmathresult{\the\pgfmathfloat@a@Mtok #1}%
		\else
			\edef\pgfmathresult{\the\pgfmathfloat@a@Mtok #1#2}%
		\fi
	\else
		\ifnum\pgfmathfloat@a@E=0%
			\edef\pgfmathresult{#1.#2}%
		\else
			\pgfmathfloat@a@Mtok={#1}%
			\pgfmathfloattofixed@impl@collectmantisse#2\count\pgfmathfloat@a@E
			\edef\pgfmathresult{\the\pgfmathfloat@a@Mtok}%
		\fi
	\fi
}

\def\pgfmathfloattofixed@impl@collectmantisse#1#2\count#3{%
	\pgfmathfloat@a@Mtok=\expandafter{\the\pgfmathfloat@a@Mtok #1}%
	\advance\pgfmathfloat@a@E by-1%
	\def\pgfmathfloat@TMPB{#2}%
	\ifx\pgfmathfloat@TMPB\pgfutil@empty
		\loop
		\ifnum\pgfmathfloat@a@E>0%
			\pgfmathfloat@a@Mtok=\expandafter{\the\pgfmathfloat@a@Mtok 0}%
			\advance\pgfmathfloat@a@E by-1%
		\repeat
	\else
		\ifnum\pgfmathfloat@a@E=0
			\pgfmathfloat@a@Mtok=\expandafter{\the\pgfmathfloat@a@Mtok .#2}%
		\else
			\pgfmathfloattofixed@impl@collectmantisse#2\count#3%
		\fi
	\fi
}


% ============================================
%
%
% \pgfmathfloatparsenumber implementation
%
%
% ============================================


% accepted parser tokens:
\def\pgfmathfloatparsenumber@tok@ONE{1}
\def\pgfmathfloatparsenumber@tok@TWO{2}
\def\pgfmathfloatparsenumber@tok@THREE{3}
\def\pgfmathfloatparsenumber@tok@FOUR{4}
\def\pgfmathfloatparsenumber@tok@FIVE{5}
\def\pgfmathfloatparsenumber@tok@SIX{6}
\def\pgfmathfloatparsenumber@tok@SEVEN{7}
\def\pgfmathfloatparsenumber@tok@EIGHT{8}
\def\pgfmathfloatparsenumber@tok@NINE{9}
\def\pgfmathfloatparsenumber@tok@ZERO{0}
\def\pgfmathfloatparsenumber@tok@e{e}
\def\pgfmathfloatparsenumber@tok@E{E}
\def\pgfmathfloatparsenumber@tok@PLUS{+}
\def\pgfmathfloatparsenumber@tok@MINUS{-}
\def\pgfmathfloatparsenumber@tok@n{n}
\def\pgfmathfloatparsenumber@tok@N{N}
\def\pgfmathfloatparsenumber@tok@i{i}
\def\pgfmathfloatparsenumber@tok@I{I}
\def\pgfmathfloatparsenumber@tok@PERIOD{.}

\newif\ifpgfmathfloatparsenumberpendingperiod

% Starts a finite-start-machine parser to read a number.
%
% It assigns \pgfmathfloat@a@S, \pgfmathfloat@a@Mtok, \pgfmathfloat@a@E.
\def\pgfmathfloatparsenumber@impl{%
	\pgfmathfloatparsenumberpendingperiodfalse
	\pgfmathfloat@tmptoks={}% this register is used to collect PENDING ZEROS
	\pgfmathfloat@a@E=0
	\pgfmathfloatparsenumber@impl@checksign
}


\def\pgfmathfloatparsenumber@impl@checksign#1{%
	\def\pgfmathfloatparsenumber@curtok{#1}%
	\ifx\pgfmathfloatparsenumber@curtok\pgfmathfloatparsenumber@tok@PLUS
		\pgfmathfloat@a@S=1\relax%
		\def\pgfmathfloatparsenumber@NEXT{\pgfmathfloatparsenumber@impl@init}%
	\else
		\ifx\pgfmathfloatparsenumber@curtok\pgfmathfloatparsenumber@tok@MINUS
			\pgfmathfloat@a@S=2\relax%
			\def\pgfmathfloatparsenumber@NEXT{\pgfmathfloatparsenumber@impl@init}%
		\else
			\pgfmathfloat@a@S=1\relax%
			\def\pgfmathfloatparsenumber@NEXT{\pgfmathfloatparsenumber@impl@init#1}%
		\fi
	\fi
	\pgfmathfloatparsenumber@NEXT
}

% FROM: checksign.
%
% I already know the contents of #1.
\def\pgfmathfloatparsenumber@impl@init#1{%
	\def\pgfmathfloatparsenumber@curtok{#1}%
	\ifx\pgfmathfloatparsenumber@curtok\pgfmathfloatparsenumber@tok@n
		\def\pgfmathfloatparsenumber@NEXT{\pgfmathfloatparsenumber@impl@readnan}%
	\else
		\ifx\pgfmathfloatparsenumber@curtok\pgfmathfloatparsenumber@tok@i
			\def\pgfmathfloatparsenumber@NEXT{\pgfmathfloatparsenumber@impl@readinf}%
		\else
			\ifx\pgfmathfloatparsenumber@curtok\pgfmathfloatparsenumber@tok@ZERO
				\def\pgfmathfloatparsenumber@NEXT{\pgfmathfloatparsenumber@impl@leadingzero}%
			\else
				\def\pgfmathfloatparsenumber@NEXT{%
					\pgfmathfloat@a@Mtok={#1}%
					\pgfmathfloatparsenumberpendingperiodtrue
					\pgfmathfloatparsenumber@impl@positiveexp}%
				\ifx\pgfmathfloatparsenumber@curtok\pgfmathfloatparsenumber@tok@ONE
				\else
				\ifx\pgfmathfloatparsenumber@curtok\pgfmathfloatparsenumber@tok@TWO
				\else
				\ifx\pgfmathfloatparsenumber@curtok\pgfmathfloatparsenumber@tok@THREE
				\else
				\ifx\pgfmathfloatparsenumber@curtok\pgfmathfloatparsenumber@tok@FOUR
				\else
				\ifx\pgfmathfloatparsenumber@curtok\pgfmathfloatparsenumber@tok@FIVE
				\else
				\ifx\pgfmathfloatparsenumber@curtok\pgfmathfloatparsenumber@tok@SIX
				\else
				\ifx\pgfmathfloatparsenumber@curtok\pgfmathfloatparsenumber@tok@SEVEN
				\else
				\ifx\pgfmathfloatparsenumber@curtok\pgfmathfloatparsenumber@tok@EIGHT
				\else
				\ifx\pgfmathfloatparsenumber@curtok\pgfmathfloatparsenumber@tok@NINE
				\else
					\def\pgfmathfloatparsenumber@NEXT{\pgfmathfloatparsenumber@impl@error}%
				\fi\fi\fi\fi\fi\fi\fi\fi\fi
			\fi
		\fi
	\fi
	\pgfmathfloatparsenumber@NEXT
}

\def\pgfmathfloatparsenumber@impl@error#1\pgfmathfloat@EOI{\pgfmath@error{Could not read input number, sorry. The unreadable part was near '#1'.}}%

\def\pgfmathfloatparsenumber@impl@finish#1\pgfmathfloat@EOI{}

\def\pgfmathfloatparsenumber@impl@readnan an{%
	\pgfmathfloat@a@S=3\relax%
	\pgfmathfloat@a@Mtok={nan}%
	\pgfmathfloat@a@E=0%
	\pgfmathfloatparsenumber@impl@finish
}
\def\pgfmathfloatparsenumber@impl@readinf nf{%
	\ifnum\pgfmathfloat@a@S=1
		\pgfmathfloat@a@S=4\relax%
		\pgfmathfloat@a@Mtok={inf}%
	\else
		\pgfmathfloat@a@S=5\relax%
		\pgfmathfloat@a@Mtok={-inf}%
	\fi
	\pgfmathfloat@a@E=0%
	\pgfmathfloatparsenumber@impl@finish
}
\def\pgfmathfloatparsenumber@impl@leadingzero#1{%
	\def\pgfmathfloatparsenumber@curtok{#1}%
	\ifx\pgfmathfloatparsenumber@curtok\pgfmathfloatparsenumber@tok@ZERO
		\def\pgfmathfloatparsenumber@NEXT{%
			\pgfmathfloatparsenumber@impl@leadingzero}%
	\else
		\ifx\pgfmathfloatparsenumber@curtok\pgfmathfloatparsenumber@tok@PERIOD
			\def\pgfmathfloatparsenumber@NEXT{%
				\pgfmathfloat@a@E=-1
				\pgfmathfloatparsenumber@impl@leadingzero@foundperiod}%
		\else
		\ifx\pgfmathfloatparsenumber@curtok\pgfmathfloatparsenumber@tok@e
			\def\pgfmathfloatparsenumber@NEXT{%
				\pgfmathfloat@a@S=0\relax%
				\pgfmathfloat@a@Mtok={0.0}%
				\pgfmathfloat@a@E=0%
				\pgfmathfloatparsenumber@impl@readexponent}%
		\else
		\ifx\pgfmathfloatparsenumber@curtok\pgfmathfloatparsenumber@tok@E
			\def\pgfmathfloatparsenumber@NEXT{%
				\pgfmathfloat@a@S=0\relax%
				\pgfmathfloat@a@Mtok={0.0}%
				\pgfmathfloat@a@E=0%
				\pgfmathfloatparsenumber@impl@readexponent}%
		\else
		\ifx\pgfmathfloatparsenumber@curtok\pgfmathfloat@EOI
			\def\pgfmathfloatparsenumber@NEXT{%
				\pgfmathfloat@a@S=0\relax%
				\pgfmathfloat@a@Mtok={0.0}%
				\pgfmathfloat@a@E=0%
				}%
		\else
		\def\pgfmathfloatparsenumber@NEXT{%
			\pgfmathfloat@a@Mtok={#1}%
			\pgfmathfloatparsenumberpendingperiodtrue
			\pgfmathfloatparsenumber@impl@positiveexp}%
		\ifx\pgfmathfloatparsenumber@curtok\pgfmathfloatparsenumber@tok@ONE
		\else
		\ifx\pgfmathfloatparsenumber@curtok\pgfmathfloatparsenumber@tok@TWO
		\else
		\ifx\pgfmathfloatparsenumber@curtok\pgfmathfloatparsenumber@tok@THREE
		\else
		\ifx\pgfmathfloatparsenumber@curtok\pgfmathfloatparsenumber@tok@FOUR
		\else
		\ifx\pgfmathfloatparsenumber@curtok\pgfmathfloatparsenumber@tok@FIVE
		\else
		\ifx\pgfmathfloatparsenumber@curtok\pgfmathfloatparsenumber@tok@SIX
		\else
		\ifx\pgfmathfloatparsenumber@curtok\pgfmathfloatparsenumber@tok@SEVEN
		\else
		\ifx\pgfmathfloatparsenumber@curtok\pgfmathfloatparsenumber@tok@EIGHT
		\else
		\ifx\pgfmathfloatparsenumber@curtok\pgfmathfloatparsenumber@tok@NINE
		\else
			\def\pgfmathfloatparsenumber@NEXT{\pgfmathfloatparsenumber@impl@error}%
		\fi\fi\fi\fi\fi\fi\fi\fi\fi
		\fi\fi\fi
		\fi
	\fi
	\pgfmathfloatparsenumber@NEXT
}

\def\pgfmathfloatparsenumber@impl@leadingzero@foundperiod#1{%
	\def\pgfmathfloatparsenumber@curtok{#1}%
	\ifx\pgfmathfloatparsenumber@curtok\pgfmathfloatparsenumber@tok@ZERO
		\def\pgfmathfloatparsenumber@NEXT{%
			\advance\pgfmathfloat@a@E by-1
			\pgfmathfloatparsenumber@impl@leadingzero@foundperiod}%
	\else
		\ifx\pgfmathfloatparsenumber@curtok\pgfmathfloatparsenumber@tok@e
			\def\pgfmathfloatparsenumber@NEXT{%
				\pgfmathfloat@a@S=0\relax%
				\pgfmathfloat@a@Mtok={0.0}%
				\pgfmathfloat@a@E=0%
				\pgfmathfloatparsenumber@impl@readexponent}%
		\else
		\ifx\pgfmathfloatparsenumber@curtok\pgfmathfloatparsenumber@tok@E
			\def\pgfmathfloatparsenumber@NEXT{%
				\pgfmathfloat@a@S=0\relax%
				\pgfmathfloat@a@Mtok={0.0}%
				\pgfmathfloat@a@E=0%
				\pgfmathfloatparsenumber@impl@readexponent}%
		\else
		\ifx\pgfmathfloatparsenumber@curtok\pgfmathfloat@EOI
			\def\pgfmathfloatparsenumber@NEXT{%
				\pgfmathfloat@a@S=0\relax%
				\pgfmathfloat@a@Mtok={0.0}%
				\pgfmathfloat@a@E=0%
				}%
		\else
		\def\pgfmathfloatparsenumber@NEXT{%
			\pgfmathfloat@a@Mtok={#1}%
			\pgfmathfloatparsenumberpendingperiodtrue
			\pgfmathfloatparsenumber@impl@finish@number@after@period}%
		\ifx\pgfmathfloatparsenumber@curtok\pgfmathfloatparsenumber@tok@ONE
		\else
		\ifx\pgfmathfloatparsenumber@curtok\pgfmathfloatparsenumber@tok@TWO
		\else
		\ifx\pgfmathfloatparsenumber@curtok\pgfmathfloatparsenumber@tok@THREE
		\else
		\ifx\pgfmathfloatparsenumber@curtok\pgfmathfloatparsenumber@tok@FOUR
		\else
		\ifx\pgfmathfloatparsenumber@curtok\pgfmathfloatparsenumber@tok@FIVE
		\else
		\ifx\pgfmathfloatparsenumber@curtok\pgfmathfloatparsenumber@tok@SIX
		\else
		\ifx\pgfmathfloatparsenumber@curtok\pgfmathfloatparsenumber@tok@SEVEN
		\else
		\ifx\pgfmathfloatparsenumber@curtok\pgfmathfloatparsenumber@tok@EIGHT
		\else
		\ifx\pgfmathfloatparsenumber@curtok\pgfmathfloatparsenumber@tok@NINE
		\else
			\def\pgfmathfloatparsenumber@NEXT{\pgfmathfloatparsenumber@impl@error}%
		\fi\fi\fi\fi\fi\fi\fi\fi\fi
		\fi\fi\fi
	\fi
	\pgfmathfloatparsenumber@NEXT
}


\def\pgfmathfloatparsenumber@impl@positiveexp#1{%
	\def\pgfmathfloatparsenumber@curtok{#1}%
	\ifx\pgfmathfloatparsenumber@curtok\pgfmathfloatparsenumber@tok@ZERO
		\def\pgfmathfloatparsenumber@NEXT{%
			\advance\pgfmathfloat@a@E by1
			\pgfmathfloat@tmptoks=\expandafter{\the\pgfmathfloat@tmptoks0}%
			\pgfmathfloatparsenumber@impl@pendingzeros@positiveexp}%
	\else
		\ifx\pgfmathfloatparsenumber@curtok\pgfmathfloatparsenumber@tok@PERIOD
			\def\pgfmathfloatparsenumber@NEXT{\pgfmathfloatparsenumber@impl@finish@number@after@period}%
		\else
		\ifx\pgfmathfloatparsenumber@curtok\pgfmathfloatparsenumber@tok@e
			\def\pgfmathfloatparsenumber@NEXT{\pgfmathfloatparsenumber@impl@readexponent}%
		\else
		\ifx\pgfmathfloatparsenumber@curtok\pgfmathfloatparsenumber@tok@E
			\def\pgfmathfloatparsenumber@NEXT{\pgfmathfloatparsenumber@impl@readexponent}%
		\else
		\ifx\pgfmathfloatparsenumber@curtok\pgfmathfloat@EOI
			\def\pgfmathfloatparsenumber@NEXT{}%
		\else
		\def\pgfmathfloatparsenumber@NEXT{%
			\advance\pgfmathfloat@a@E by1
			\ifpgfmathfloatparsenumberpendingperiod
				\pgfmathfloat@a@Mtok=\expandafter{\the\pgfmathfloat@a@Mtok.}%
				\pgfmathfloatparsenumberpendingperiodfalse
			\fi
			\pgfmathfloat@a@Mtok=\expandafter{\the\pgfmathfloat@a@Mtok#1}%
			\pgfmathfloatparsenumber@impl@positiveexp}%
		\ifx\pgfmathfloatparsenumber@curtok\pgfmathfloatparsenumber@tok@ONE
		\else
		\ifx\pgfmathfloatparsenumber@curtok\pgfmathfloatparsenumber@tok@TWO
		\else
		\ifx\pgfmathfloatparsenumber@curtok\pgfmathfloatparsenumber@tok@THREE
		\else
		\ifx\pgfmathfloatparsenumber@curtok\pgfmathfloatparsenumber@tok@FOUR
		\else
		\ifx\pgfmathfloatparsenumber@curtok\pgfmathfloatparsenumber@tok@FIVE
		\else
		\ifx\pgfmathfloatparsenumber@curtok\pgfmathfloatparsenumber@tok@SIX
		\else
		\ifx\pgfmathfloatparsenumber@curtok\pgfmathfloatparsenumber@tok@SEVEN
		\else
		\ifx\pgfmathfloatparsenumber@curtok\pgfmathfloatparsenumber@tok@EIGHT
		\else
		\ifx\pgfmathfloatparsenumber@curtok\pgfmathfloatparsenumber@tok@NINE
		\else
			\def\pgfmathfloatparsenumber@NEXT{\pgfmathfloatparsenumber@impl@error}%
		\fi\fi\fi\fi\fi\fi\fi\fi\fi
		\fi\fi\fi
		\fi
	\fi
	\pgfmathfloatparsenumber@NEXT
}

\def\pgfmathfloatparsenumber@impl@readexponent#1\pgfmathfloat@EOI{%
	\advance\pgfmathfloat@a@E by#1\relax
}

\def\pgfmathfloatparsenumber@impl@pendingzeros@positiveexp#1{%
	\def\pgfmathfloatparsenumber@curtok{#1}%
	\ifx\pgfmathfloatparsenumber@curtok\pgfmathfloatparsenumber@tok@ZERO
		\def\pgfmathfloatparsenumber@NEXT{%
			\pgfmathfloat@tmptoks=\expandafter{\the\pgfmathfloat@tmptoks0}%
			\advance\pgfmathfloat@a@E by1
			\pgfmathfloatparsenumber@impl@pendingzeros@positiveexp}%
	\else
		\ifx\pgfmathfloatparsenumber@curtok\pgfmathfloatparsenumber@tok@PERIOD
			\def\pgfmathfloatparsenumber@NEXT{\pgfmathfloatparsenumber@impl@finish@number@after@period}%
		\else
		\ifx\pgfmathfloatparsenumber@curtok\pgfmathfloatparsenumber@tok@e
			\def\pgfmathfloatparsenumber@NEXT{\pgfmathfloatparsenumber@impl@readexponent}%
		\else
		\ifx\pgfmathfloatparsenumber@curtok\pgfmathfloatparsenumber@tok@E
			\def\pgfmathfloatparsenumber@NEXT{\pgfmathfloatparsenumber@impl@readexponent}%
		\else
		\ifx\pgfmathfloatparsenumber@curtok\pgfmathfloat@EOI
			\def\pgfmathfloatparsenumber@NEXT{}%
		\else
		\def\pgfmathfloatparsenumber@NEXT{%
			\advance\pgfmathfloat@a@E by1
			\ifpgfmathfloatparsenumberpendingperiod
				\pgfmathfloat@a@Mtok=\expandafter{\the\pgfmathfloat@a@Mtok.}%
				\pgfmathfloatparsenumberpendingperiodfalse
			\fi
			\pgfmathfloat@a@Mtok=\expandafter\expandafter\expandafter{\expandafter\the\expandafter\pgfmathfloat@a@Mtok\the\pgfmathfloat@tmptoks}%
			\pgfmathfloat@tmptoks={}%
			\pgfmathfloat@a@Mtok=\expandafter{\the\pgfmathfloat@a@Mtok#1}%
			\pgfmathfloatparsenumber@impl@positiveexp}%
		\ifx\pgfmathfloatparsenumber@curtok\pgfmathfloatparsenumber@tok@ONE
		\else
		\ifx\pgfmathfloatparsenumber@curtok\pgfmathfloatparsenumber@tok@TWO
		\else
		\ifx\pgfmathfloatparsenumber@curtok\pgfmathfloatparsenumber@tok@THREE
		\else
		\ifx\pgfmathfloatparsenumber@curtok\pgfmathfloatparsenumber@tok@FOUR
		\else
		\ifx\pgfmathfloatparsenumber@curtok\pgfmathfloatparsenumber@tok@FIVE
		\else
		\ifx\pgfmathfloatparsenumber@curtok\pgfmathfloatparsenumber@tok@SIX
		\else
		\ifx\pgfmathfloatparsenumber@curtok\pgfmathfloatparsenumber@tok@SEVEN
		\else
		\ifx\pgfmathfloatparsenumber@curtok\pgfmathfloatparsenumber@tok@EIGHT
		\else
		\ifx\pgfmathfloatparsenumber@curtok\pgfmathfloatparsenumber@tok@NINE
		\else
			\def\pgfmathfloatparsenumber@NEXT{\pgfmathfloatparsenumber@impl@error}%
		\fi\fi\fi\fi\fi\fi\fi\fi\fi
		\fi\fi\fi
		\fi
	\fi
	\pgfmathfloatparsenumber@NEXT
}

\def\pgfmathfloatparsenumber@impl@finish@number@after@period#1{%
	\def\pgfmathfloatparsenumber@curtok{#1}%
	\ifx\pgfmathfloatparsenumber@curtok\pgfmathfloatparsenumber@tok@ZERO
		\def\pgfmathfloatparsenumber@NEXT{%
			\pgfmathfloat@tmptoks=\expandafter{\the\pgfmathfloat@tmptoks0}%
			\pgfmathfloatparsenumber@impl@pendingzeros}%
	\else
		\ifx\pgfmathfloatparsenumber@curtok\pgfmathfloatparsenumber@tok@e
			\def\pgfmathfloatparsenumber@NEXT{\pgfmathfloatparsenumber@impl@readexponent}%
		\else
		\ifx\pgfmathfloatparsenumber@curtok\pgfmathfloatparsenumber@tok@E
			\def\pgfmathfloatparsenumber@NEXT{\pgfmathfloatparsenumber@impl@readexponent}%
		\else
		\ifx\pgfmathfloatparsenumber@curtok\pgfmathfloat@EOI
			\def\pgfmathfloatparsenumber@NEXT{}%
		\else
		\def\pgfmathfloatparsenumber@NEXT{%
			\ifpgfmathfloatparsenumberpendingperiod
				\pgfmathfloat@a@Mtok=\expandafter{\the\pgfmathfloat@a@Mtok.}%
				\pgfmathfloatparsenumberpendingperiodfalse
			\fi
			\pgfmathfloat@a@Mtok=\expandafter\expandafter\expandafter{\expandafter\the\expandafter\pgfmathfloat@a@Mtok\the\pgfmathfloat@tmptoks}%
			\pgfmathfloat@tmptoks={}%
			\pgfmathfloat@a@Mtok=\expandafter{\the\pgfmathfloat@a@Mtok#1}%
			\pgfmathfloatparsenumber@impl@finish@number@after@period}%
		\ifx\pgfmathfloatparsenumber@curtok\pgfmathfloatparsenumber@tok@ONE
		\else
		\ifx\pgfmathfloatparsenumber@curtok\pgfmathfloatparsenumber@tok@TWO
		\else
		\ifx\pgfmathfloatparsenumber@curtok\pgfmathfloatparsenumber@tok@THREE
		\else
		\ifx\pgfmathfloatparsenumber@curtok\pgfmathfloatparsenumber@tok@FOUR
		\else
		\ifx\pgfmathfloatparsenumber@curtok\pgfmathfloatparsenumber@tok@FIVE
		\else
		\ifx\pgfmathfloatparsenumber@curtok\pgfmathfloatparsenumber@tok@SIX
		\else
		\ifx\pgfmathfloatparsenumber@curtok\pgfmathfloatparsenumber@tok@SEVEN
		\else
		\ifx\pgfmathfloatparsenumber@curtok\pgfmathfloatparsenumber@tok@EIGHT
		\else
		\ifx\pgfmathfloatparsenumber@curtok\pgfmathfloatparsenumber@tok@NINE
		\else
			\def\pgfmathfloatparsenumber@NEXT{\pgfmathfloatparsenumber@impl@error}%
		\fi\fi\fi\fi\fi\fi\fi\fi\fi
		\fi\fi\fi
	\fi
	\pgfmathfloatparsenumber@NEXT
}

\def\pgfmathfloatparsenumber@impl@pendingzeros#1{%
	\def\pgfmathfloatparsenumber@curtok{#1}%
	\ifx\pgfmathfloatparsenumber@curtok\pgfmathfloatparsenumber@tok@ZERO
		\def\pgfmathfloatparsenumber@NEXT{%
			\pgfmathfloat@tmptoks=\expandafter{\the\pgfmathfloat@tmptoks0}%
			\pgfmathfloatparsenumber@impl@pendingzeros}%
	\else
		\ifx\pgfmathfloatparsenumber@curtok\pgfmathfloatparsenumber@tok@e
			\def\pgfmathfloatparsenumber@NEXT{\pgfmathfloatparsenumber@impl@readexponent}%
		\else
		\ifx\pgfmathfloatparsenumber@curtok\pgfmathfloatparsenumber@tok@E
			\def\pgfmathfloatparsenumber@NEXT{\pgfmathfloatparsenumber@impl@readexponent}%
		\else
		\ifx\pgfmathfloatparsenumber@curtok\pgfmathfloat@EOI
			\def\pgfmathfloatparsenumber@NEXT{}%
		\else
		\def\pgfmathfloatparsenumber@NEXT{%
			\ifpgfmathfloatparsenumberpendingperiod
				\pgfmathfloat@a@Mtok=\expandafter{\the\pgfmathfloat@a@Mtok.}%
				\pgfmathfloatparsenumberpendingperiodfalse
			\fi
			\pgfmathfloat@a@Mtok=\expandafter\expandafter\expandafter{\expandafter\the\expandafter\pgfmathfloat@a@Mtok\the\pgfmathfloat@tmptoks}%
			\pgfmathfloat@tmptoks={}%
			\pgfmathfloat@a@Mtok=\expandafter{\the\pgfmathfloat@a@Mtok#1}%
			\pgfmathfloatparsenumber@impl@finish@number@after@period}%
		\ifx\pgfmathfloatparsenumber@curtok\pgfmathfloatparsenumber@tok@ONE
		\else
		\ifx\pgfmathfloatparsenumber@curtok\pgfmathfloatparsenumber@tok@TWO
		\else
		\ifx\pgfmathfloatparsenumber@curtok\pgfmathfloatparsenumber@tok@THREE
		\else
		\ifx\pgfmathfloatparsenumber@curtok\pgfmathfloatparsenumber@tok@FOUR
		\else
		\ifx\pgfmathfloatparsenumber@curtok\pgfmathfloatparsenumber@tok@FIVE
		\else
		\ifx\pgfmathfloatparsenumber@curtok\pgfmathfloatparsenumber@tok@SIX
		\else
		\ifx\pgfmathfloatparsenumber@curtok\pgfmathfloatparsenumber@tok@SEVEN
		\else
		\ifx\pgfmathfloatparsenumber@curtok\pgfmathfloatparsenumber@tok@EIGHT
		\else
		\ifx\pgfmathfloatparsenumber@curtok\pgfmathfloatparsenumber@tok@NINE
		\else
			\def\pgfmathfloatparsenumber@NEXT{\pgfmathfloatparsenumber@impl@error}%
		\fi\fi\fi\fi\fi\fi\fi\fi\fi
		\fi\fi\fi
	\fi
	\pgfmathfloatparsenumber@NEXT
}




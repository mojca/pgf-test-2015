% Copyright 2007 by Mark Wibrow
%
% This file may be distributed and/or modified
%
% 1. under the LaTeX Project Public License and/or
% 2. under the GNU Public License.
%
% See the file doc/generic/pgf/licenses/LICENSE for more details.

% This file defines the mathematical functions and operators.
%
% Version 0.0 08/03/2007

% This file defines the mathematical functions and operators.
%
% Adding/redefining extra operators/functions:
%
% Each operator/function XXX has two forms:
%
%
% \pgfmathXXX#1...   a public version which evaluates any
%                    arguments passed to it and passes the
%                    results on to...
%
% \pgfmathXXX@#1...  a non-public version which performs 
%                    required calculation on arguments which
%                    must have already been evaluated (i.e.
%                    *without* dimensions).
% 
% If a function XXX is to be included in the parser, it is 
% recommended, for consistency, that where possible, the 
% pgfmathparser file should define the macro \pgfmath@parseXXX.
% The parser should (ideally) then call \pgfmathXXX@.
%
% It is recommend that the pgfmath versions of the pgf dimension
% and count registers be used, i.e., \pgfmath@x for \pgfmath@x, 
% \c@pgfmath@counta for c@pgfmath@counta, and so on. These are currently
% \let to their pgf equivalents, but it may be necessary to change 
% this.
%
% It is also recommened that all calculations (where necessary)
% take place within a TeX group. \pgfmath@returnone#1 makes and
% expanded version of #1 global and stores this in \pgfmathresult 
% after the group is ended.
%
% Copyright 2007 by Mark Wibrow
%
% This file may be distributed and/or modified
%
% 1. under the LaTeX Project Public License and/or
% 2. under the GNU Public License.
%
% See the file doc/generic/pgf/licenses/LICENSE for more details.

% This file defines the trigonometric functions/operations.
%
% Much of this file is based on ideas and code (particularly 
% \pgfcoremath.code.tex) due to Till Tantau
%
% Version 0.0 08/03/2007

% \pgfmathpi
%
\def\pgfmathpi{\edef\pgfmathresult{\pgfmath@pi}}
\def\pgfmath@pi{3.14159}

% \pgfmathradians
% 
% Convert #1 from radians to degrees (accurate to 1 deg).
%
\def\pgfmathradians#1{%
	\pgfmathevalautae{#1}%
	\pgfmathradians@{\pgfmathresult}}
\def\pgfmathradians@#1{%
	\begingroup%
		\pgfmath@x#1pt\relax%
		\pgfmath@x57.29577\pgfmath@x\relax% 57.29577 = 360/(2*pi)
		\pgfmathround@{\pgf@sys@tonumber{\pgfmath@x}}%
		\pgfmath@x\pgfmathresult pt\relax%
		\pgfmath@returnone\pgfmath@x%
	\endgroup%
}%

% \pgfmathsin
% 
% Calculate the sine of #1 (in degrees).
%
\def\pgfmathsin#1{%
	\pgfmathparse{#1}%
	\pgfmathsin@{\pgfmathresult}}
\def\pgfmathsin@#1{%
	\begingroup%
		\pgfmath@x#1pt\relax%
		\afterassignment\pgfmath@gobbletilpgfmath@%
		\expandafter\c@pgfmath@counta\the\pgfmath@x\relax\pgfmath@%
		\pgfmath@xb\pgfmath@x%
		\advance\pgfmath@xb-\c@pgfmath@counta pt\relax%
		\expandafter\pgfmathsin@@\expandafter{\the\c@pgfmath@counta}%
		\pgfmath@xa\pgfmath@x%
		\ifnum\c@pgfmath@counta>0\relax%
			\advance\c@pgfmath@counta-1\relax%
		\else%
			\advance\c@pgfmath@counta1\relax%
		\fi%
		\expandafter\pgfmathsin@@\expandafter{\the\c@pgfmath@counta}%
		\pgfmath@x\pgfmath@tonumber{\pgfmath@xb}\pgfmath@x%
		\pgfmath@xb-\pgfmath@xb%
		\advance\pgfmath@xb1pt\relax%
		\advance\pgfmath@x\pgfmath@tonumber{\pgfmath@xb}\pgfmath@xa%
		\pgfmath@returnone\pgfmath@x%
	\endgroup%
}
\def\pgfmathsin@@#1{% 
	% NB no \begingroup...\endgroup. I don't think \pgfmathsin@@
	% needs to be called directly. Might be wrong...
	\c@pgfmath@counta#1\relax%
	\c@pgfmath@countb#1\relax%
	\divide\c@pgfmath@countb360\relax%
	\multiply\c@pgfmath@countb-360\relax%
	\c@pgfmath@counta#1\relax%
	\advance\c@pgfmath@counta\c@pgfmath@countb%
	\c@pgfmath@countb=90\relax%
	\advance\c@pgfmath@countb by-\c@pgfmath@counta%
	\ifnum\c@pgfmath@countb>179\relax%
		\advance\c@pgfmath@countb by-360\relax%
	\else%
		\ifnum\c@pgfmath@countb<-179\relax%
			\advance\c@pgfmath@countb by360\relax%
	\fi\fi%
	\ifnum\c@pgfmath@countb<0\relax%
 		\c@pgfmath@countb=-\c@pgfmath@countb%
	\fi%
	\expandafter\pgfmath@x\csname pgfmath@cos@\the\c@pgfmath@countb\endcsname pt\relax%
}

% \pgfmathcos
% 
% Calculate the cosine of #1 (in degrees).
%
\def\pgfmathcos#1{%
	\pgfmathparse{#1}%
	\expandafter\pgfmathcos@\expandafter{\pgfmathresult}}
\def\pgfmathcos@#1{%
	\begingroup%
		\pgfmath@x#1pt\relax%
		\afterassignment\pgfmath@gobbletilpgfmath@%
		\expandafter\c@pgfmath@counta\the\pgfmath@x\relax\pgfmath@%
		\pgfmath@xb\pgfmath@x%
		\advance\pgfmath@xb-\c@pgfmath@counta pt\relax%
		\expandafter\pgfmathcos@@\expandafter{\the\c@pgfmath@counta}%
		\pgfmath@xa\pgfmath@x%
		\ifnum\c@pgfmath@counta<180\relax%
			\advance\c@pgfmath@counta1\relax%
		\else
			\advance\c@pgfmath@counta-1\relax%
		\fi
		\expandafter\pgfmathcos@@\expandafter{\the\c@pgfmath@counta}%
		\pgfmath@x\pgfmath@tonumber{\pgfmath@xb}\pgfmath@x%
		\pgfmath@xb-\pgfmath@xb%
		\advance\pgfmath@xb1pt\relax%
		\advance\pgfmath@x\pgfmath@tonumber{\pgfmath@xb}\pgfmath@xa%
		\pgfmath@returnone\pgfmath@x%
	\endgroup%
}

\def\pgfmathcos@@#1{% NB No TeX group.
	\c@pgfmath@countb#1\relax%
	\divide\c@pgfmath@countb360\relax%
	\multiply\c@pgfmath@countb-360\relax%
	\c@pgfmath@counta#1\relax%
	\advance\c@pgfmath@counta\c@pgfmath@countb%
	\ifnum\c@pgfmath@counta<0\relax%
		\c@pgfmath@counta-\c@pgfmath@counta\relax%
	\fi%
	\ifnum\c@pgfmath@counta>179\relax%
		\c@pgfmath@countb360\relax%
		\advance\c@pgfmath@countb-\c@pgfmath@counta\relax%
		\c@pgfmath@counta\c@pgfmath@countb\relax%
	\fi%
	\expandafter\pgfmath@x\csname pgfmath@cos@\the\c@pgfmath@counta\endcsname pt\relax%
}
	
% \pgfmathtan
% 
% Calculate the cotangent of #1 (in degrees).
%
\def\pgfmathtan#1{%
	\pgfmathparse{#1}%
	\pgfmathtan@{\pgfmathresult}}
\def\pgfmathtan@#1{%
	\begingroup%
		\pgfmathcos@{#1}%
		\expandafter\pgfmathreciprocal@\expandafter{\pgfmathresult}%
		\edef\pgfmath@tantemp{\pgfmathresult}%
		\pgfmathsin@{#1}%
		\pgfmath@x\pgfmathresult pt\relax%
		\pgfmath@x\pgfmath@tantemp\pgfmath@x%
		\pgfmath@returnone\pgfmath@x%
	\endgroup%
}


% \pgfmatharcsin
%
% arcsin(x) = x + 1/2 (x^3/3) + (1/2)(3/4)(x^5/5) + (1/2)(3/4)(5/6)(x^7/7) + ... 
%
\def\pgfmatharcsin#1{%
	\pgfmathparse{#1}%
	\expandafter\pgfmatharcsin@\expandafter{\pgfmathresult}}
\def\pgfmatharcsin@#1{%
	\begingroup%
		\pgfmath@x#1pt\relax%
		\pgfmath@xa\pgfmath@tonumber{\pgfmath@x}\pgfmath@x%
		\pgfmath@xb\pgfmath@x%
		\pgfmath@ya16383.99999pt\relax%
		\pgfmath@y1pt\relax%
		\c@pgfmath@counta1\relax%
		\pgfmathloop%
			\multiply\pgfmath@y\c@pgfmath@counta%
			\advance\c@pgfmath@counta1\relax%
			\divide\pgfmath@y\c@pgfmath@counta%
			\advance\c@pgfmath@counta1\relax%
			\pgfmath@xb\pgfmath@tonumber{\pgfmath@xa}\pgfmath@xb%
			\pgfmath@xc\pgfmath@xb%
			\divide\pgfmath@xc\c@pgfmath@counta%
			\advance\pgfmath@x\pgfmath@tonumber{\pgfmath@y}\pgfmath@xc%
			\ifdim\pgfmath@x=\pgfmath@ya%
			\else%
				\pgfmath@ya=\pgfmath@x%
		\repeatpgfmathloop%
		\pgfmath@returnone\pgfmath@x%
	\endgroup%
}
	
\def\pgfmatharccos#1{%
	\pgfmathparse{#1}%
	\expandafter\pgfmatharccos@\expandafter{\pgfmathresult}}
\def\pgfmatharccos@#1{%
	\begingroup%
		\pgfmath@x\pgfmath@pi pt\relax%
		\divide\pgfmath@x2\relax%
		\pgfmatharcsin@{#1}%
		\advance\pgfmath@x-\pgfmathresult pt\relax%
		\pgfmath@returnone\pgfmath@x%
	\endgroup%
}
	
\def\pgfmath@def#1#2#3{\expandafter\def\csname pgfmath@#1@#2\endcsname{#3}}
\pgfmath@def{cos}{0}{1.00000}		\pgfmath@def{cos}{1}{0.99985}
\pgfmath@def{cos}{2}{0.99939}		\pgfmath@def{cos}{3}{0.99863}
\pgfmath@def{cos}{4}{0.99756}		\pgfmath@def{cos}{5}{0.99619}
\pgfmath@def{cos}{6}{0.99452}		\pgfmath@def{cos}{7}{0.99255}
\pgfmath@def{cos}{8}{0.99027}		\pgfmath@def{cos}{9}{0.98769}
\pgfmath@def{cos}{10}{0.98481}		\pgfmath@def{cos}{11}{0.98163}
\pgfmath@def{cos}{12}{0.97815}		\pgfmath@def{cos}{13}{0.97437}
\pgfmath@def{cos}{14}{0.97030}		\pgfmath@def{cos}{15}{0.96593}
\pgfmath@def{cos}{16}{0.96126}		\pgfmath@def{cos}{17}{0.95630}
\pgfmath@def{cos}{18}{0.95106}		\pgfmath@def{cos}{19}{0.94552}
\pgfmath@def{cos}{20}{0.93969}		\pgfmath@def{cos}{21}{0.93358}
\pgfmath@def{cos}{22}{0.92718}		\pgfmath@def{cos}{23}{0.92050}
\pgfmath@def{cos}{24}{0.91355}		\pgfmath@def{cos}{25}{0.90631}
\pgfmath@def{cos}{26}{0.89879}		\pgfmath@def{cos}{27}{0.89101}
\pgfmath@def{cos}{28}{0.88295}		\pgfmath@def{cos}{29}{0.87462}
\pgfmath@def{cos}{30}{0.86603}		\pgfmath@def{cos}{31}{0.85717}
\pgfmath@def{cos}{32}{0.84805}		\pgfmath@def{cos}{33}{0.83867}
\pgfmath@def{cos}{34}{0.82904}		\pgfmath@def{cos}{35}{0.81915}
\pgfmath@def{cos}{36}{0.80902}		\pgfmath@def{cos}{37}{0.79864}
\pgfmath@def{cos}{38}{0.78801}		\pgfmath@def{cos}{39}{0.77715}
\pgfmath@def{cos}{40}{0.76604}		\pgfmath@def{cos}{41}{0.75471}
\pgfmath@def{cos}{42}{0.74314}		\pgfmath@def{cos}{43}{0.73135}
\pgfmath@def{cos}{44}{0.71934}		\pgfmath@def{cos}{45}{0.70711}
\pgfmath@def{cos}{46}{0.69466}		\pgfmath@def{cos}{47}{0.68200}
\pgfmath@def{cos}{48}{0.66913}		\pgfmath@def{cos}{49}{0.65606}
\pgfmath@def{cos}{50}{0.64279}		\pgfmath@def{cos}{51}{0.62932}
\pgfmath@def{cos}{52}{0.61566}		\pgfmath@def{cos}{53}{0.60182}
\pgfmath@def{cos}{54}{0.58779}		\pgfmath@def{cos}{55}{0.57358}
\pgfmath@def{cos}{56}{0.55919}		\pgfmath@def{cos}{57}{0.54464}
\pgfmath@def{cos}{58}{0.52992}		\pgfmath@def{cos}{59}{0.51504}
\pgfmath@def{cos}{60}{0.50000}		\pgfmath@def{cos}{61}{0.48481}
\pgfmath@def{cos}{62}{0.46947}		\pgfmath@def{cos}{63}{0.45399}
\pgfmath@def{cos}{64}{0.43837}		\pgfmath@def{cos}{65}{0.42262}
\pgfmath@def{cos}{66}{0.40674}		\pgfmath@def{cos}{67}{0.39073}
\pgfmath@def{cos}{68}{0.37461}		\pgfmath@def{cos}{69}{0.35837}
\pgfmath@def{cos}{70}{0.34202}		\pgfmath@def{cos}{71}{0.32557}
\pgfmath@def{cos}{72}{0.30902}		\pgfmath@def{cos}{73}{0.29237}
\pgfmath@def{cos}{74}{0.27564}		\pgfmath@def{cos}{75}{0.25882}
\pgfmath@def{cos}{76}{0.24192}		\pgfmath@def{cos}{77}{0.22495}
\pgfmath@def{cos}{78}{0.20791}		\pgfmath@def{cos}{79}{0.19081}
\pgfmath@def{cos}{80}{0.17365}		\pgfmath@def{cos}{81}{0.15643}
\pgfmath@def{cos}{82}{0.13917}		\pgfmath@def{cos}{83}{0.12187}
\pgfmath@def{cos}{84}{0.10453}		\pgfmath@def{cos}{85}{0.08716}
\pgfmath@def{cos}{86}{0.06976}		\pgfmath@def{cos}{87}{0.05234}
\pgfmath@def{cos}{88}{0.03490}		\pgfmath@def{cos}{89}{0.01745}
\pgfmath@def{cos}{90}{0.00000}		\pgfmath@def{cos}{91}{-0.01745}
\pgfmath@def{cos}{92}{-0.03490}		\pgfmath@def{cos}{93}{-0.05234}
\pgfmath@def{cos}{94}{-0.06976}		\pgfmath@def{cos}{95}{-0.08716}
\pgfmath@def{cos}{96}{-0.10453}		\pgfmath@def{cos}{97}{-0.12187}
\pgfmath@def{cos}{98}{-0.13917}		\pgfmath@def{cos}{99}{-0.15643}
\pgfmath@def{cos}{100}{-0.17365}		\pgfmath@def{cos}{101}{-0.19081}
\pgfmath@def{cos}{102}{-0.20791}		\pgfmath@def{cos}{103}{-0.22495}
\pgfmath@def{cos}{104}{-0.24192}		\pgfmath@def{cos}{105}{-0.25882}
\pgfmath@def{cos}{106}{-0.27564}		\pgfmath@def{cos}{107}{-0.29237}
\pgfmath@def{cos}{108}{-0.30902}		\pgfmath@def{cos}{109}{-0.32557}
\pgfmath@def{cos}{110}{-0.34202}		\pgfmath@def{cos}{111}{-0.35837}
\pgfmath@def{cos}{112}{-0.37461}		\pgfmath@def{cos}{113}{-0.39073}
\pgfmath@def{cos}{114}{-0.40674}		\pgfmath@def{cos}{115}{-0.42262}
\pgfmath@def{cos}{116}{-0.43837}		\pgfmath@def{cos}{117}{-0.45399}
\pgfmath@def{cos}{118}{-0.46947}		\pgfmath@def{cos}{119}{-0.48481}
\pgfmath@def{cos}{120}{-0.50000}		\pgfmath@def{cos}{121}{-0.51504}
\pgfmath@def{cos}{122}{-0.52992}		\pgfmath@def{cos}{123}{-0.54464}
\pgfmath@def{cos}{124}{-0.55919}		\pgfmath@def{cos}{125}{-0.57358}
\pgfmath@def{cos}{126}{-0.58779}		\pgfmath@def{cos}{127}{-0.60182}
\pgfmath@def{cos}{128}{-0.61566}		\pgfmath@def{cos}{129}{-0.62932}
\pgfmath@def{cos}{130}{-0.64279}		\pgfmath@def{cos}{131}{-0.65606}
\pgfmath@def{cos}{132}{-0.66913}		\pgfmath@def{cos}{133}{-0.68200}
\pgfmath@def{cos}{134}{-0.69466}		\pgfmath@def{cos}{135}{-0.70711}
\pgfmath@def{cos}{136}{-0.71934}		\pgfmath@def{cos}{137}{-0.73135}
\pgfmath@def{cos}{138}{-0.74314}		\pgfmath@def{cos}{139}{-0.75471}
\pgfmath@def{cos}{140}{-0.76604}		\pgfmath@def{cos}{141}{-0.77715}
\pgfmath@def{cos}{142}{-0.78801}		\pgfmath@def{cos}{143}{-0.79864}
\pgfmath@def{cos}{144}{-0.80902}		\pgfmath@def{cos}{145}{-0.81915}
\pgfmath@def{cos}{146}{-0.82904}		\pgfmath@def{cos}{147}{-0.83867}
\pgfmath@def{cos}{148}{-0.84805}		\pgfmath@def{cos}{149}{-0.85717}
\pgfmath@def{cos}{150}{-0.86603}		\pgfmath@def{cos}{151}{-0.87462}
\pgfmath@def{cos}{152}{-0.88295}		\pgfmath@def{cos}{153}{-0.89101}
\pgfmath@def{cos}{154}{-0.89879}		\pgfmath@def{cos}{155}{-0.90631}
\pgfmath@def{cos}{156}{-0.91355}		\pgfmath@def{cos}{157}{-0.92050}
\pgfmath@def{cos}{158}{-0.92718}		\pgfmath@def{cos}{159}{-0.93358}
\pgfmath@def{cos}{160}{-0.93969}		\pgfmath@def{cos}{161}{-0.94552}
\pgfmath@def{cos}{162}{-0.95106}		\pgfmath@def{cos}{163}{-0.95630}
\pgfmath@def{cos}{164}{-0.96126}		\pgfmath@def{cos}{165}{-0.96593}
\pgfmath@def{cos}{166}{-0.97030}		\pgfmath@def{cos}{167}{-0.97437}
\pgfmath@def{cos}{168}{-0.97815}		\pgfmath@def{cos}{169}{-0.98163}
\pgfmath@def{cos}{170}{-0.98481}		\pgfmath@def{cos}{171}{-0.98769}
\pgfmath@def{cos}{172}{-0.99027}		\pgfmath@def{cos}{173}{-0.99255}
\pgfmath@def{cos}{174}{-0.99452}		\pgfmath@def{cos}{175}{-0.99619}
\pgfmath@def{cos}{176}{-0.99756}		\pgfmath@def{cos}{177}{-0.99863}
\pgfmath@def{cos}{178}{-0.99939}		\pgfmath@def{cos}{179}{-0.99985}
\pgfmath@def{cos}{180}{-1.00000}	% Load the trig. stuff.
% This file defines the pesudorandom numbers.
%
% (c) 2007 Mark Wibrow
%
% but subject to the LaTeX Project Public License 
% (http://www.latex-project.org/lppl.txt)
%
% and the GNU Public License 
% (http://www.gnu.org/licenses/gpl.txt)
%
% Version 0.0 08/03/2007

% Pseudo-random number generation.
%
% See:
% \book@{pressetal1992,
%    author    = {William H. Press and Brian P. Flannery and Saul A. Teukolsky and William T. Vetterling}, 
%    title     = {Numerical Recipies in C},
%    edition   = {Second},
%    publisher = {Cambridge University Press}
% }
%
% See also, the documentation for the lcg package by Erich Janka:
% (http://www.ctan.org/tex-archive/macros/latex/contrib/lcg/lcg.pdf)
%
\def\pgfmath@rnd@m{2147483647}% LaTeX Maximum.

\begingroup
\c@pgfmath@counta=\time%
\multiply\c@pgfmath@counta by\year%
\xdef\pgfmath@rnd@z{\the\c@pgfmath@counta}% The seed. 
\endgroup

% \pgfmathsetseed
%
% Explictly set the seed for the generator
%   
% #1 -> the new seed.
%

\def\pgfmathsetseed#1{\xdef\pgfmath@rnd@z{#1}}%  

% Alternative paramaters - see Press et al (1992) p278-279, for discussion.
%
% a=16807 q=127773 r=2836
% a=48271 q=4488   r=3399
%
\def\pgfmath@rnd@a{69621}
\def\pgfmath@rnd@r{23902}
\def\pgfmath@rnd@q{30845}

% \pgfmathgeneratepseudorandomnumber
% 
% A linear congruency generator for generating
% pseudo-random numbers. Generates numbers in
% the range 1 - 2^31-1.
%
\def\pgfmathgeneratepseudorandomnumber{%
	\begingroup%
		\c@pgfmath@counta=\pgfmath@rnd@z%
		\c@pgfmath@countb=\pgfmath@rnd@z%
		\c@pgfmath@countc=\pgfmath@rnd@q%
		\divide\c@pgfmath@counta by\c@pgfmath@countc%
		\multiply\c@pgfmath@counta by-\c@pgfmath@countc%
		\advance\c@pgfmath@counta by\c@pgfmath@countb
		\c@pgfmath@countc=\pgfmath@rnd@a%
		\multiply\c@pgfmath@counta by\c@pgfmath@countc%
		\c@pgfmath@countc=\pgfmath@rnd@q%
		\divide\c@pgfmath@countb by\c@pgfmath@countc%
		\c@pgfmath@countc=\pgfmath@rnd@r%
		\multiply\c@pgfmath@countb by\c@pgfmath@countc%
		\advance\c@pgfmath@counta by-\c@pgfmath@countb%
		\ifnum\c@pgfmath@counta<0\relax%
			\c@pgfmath@countb=\pgfmath@rnd@m%
			\advance\c@pgfmath@counta by\c@pgfmath@countb%
		\fi%
		\xdef\pgfmath@rnd@z{\the\c@pgfmath@counta}%
	\endgroup%
	\edef\pgfmathresult{\pgfmath@rnd@z}%
}

% \pgfmathrnd
%
% Generates a pseudo-random number between 0 and 1.
%
\def\pgfmathrnd{%
	\begingroup%
		\pgfmathgeneratepseudorandomnumber%
		\c@pgfmath@counta\pgfmathresult%
		\c@pgfmath@countb\c@pgfmath@counta%
		\divide\c@pgfmath@countb100001\relax% To get one.
		\multiply\c@pgfmath@countb-100001\relax%
		\advance\c@pgfmath@countb\c@pgfmath@counta%
		\advance\c@pgfmath@countb1000000\relax%
		\expandafter\pgfmathrnd@\the\c@pgfmath@countb\pgfmath@%
		\pgfmath@returnone\pgfmath@x%
	\endgroup%
}%

\def\pgfmathrnd@#1#2#3\pgfmath@{%
	\edef\pgfmath@temp{#2.#3}%
	\pgfmath@x=\pgfmath@temp pt\relax%
}%

% \pgfmathrand
%
% Generates a pseudo-random number between -1 and 1.
%
\def\pgfmathrand{%
	\begingroup%
		\pgfmathgeneratepseudorandomnumber%
		\c@pgfmath@counta\pgfmathresult%
		\c@pgfmath@countb\c@pgfmath@counta%
		\divide\c@pgfmath@countb200001\relax%
		\multiply\c@pgfmath@countb-200001\relax%
		\advance\c@pgfmath@countb\c@pgfmath@counta%
		\advance\c@pgfmath@countb-100000\relax%
		\ifnum\c@pgfmath@countb<0\relax%
			\advance\c@pgfmath@countb-1000000\relax%
		\else%
			\advance\c@pgfmath@countb1000000\relax%
		\fi%
		\expandafter\pgfmathrand@\the\c@pgfmath@countb\pgfmath@%
		\pgfmath@returnone\pgfmath@x%
	\endgroup%
}%

\def\pgfmathrand@#1#2#3#4\pgfmath@{%
	\ifx#1-%
		\edef\pgfmath@temp{-#3.#4}%
	\else%
		\edef\pgfmath@temp{#2.#3#4}%
	\fi%
	\pgfmath@x=\pgfmath@temp pt\relax%
}%

% \pgfmathrandominteger
%
% Return a 'randomly' selected integer in the range #2 - #3 (inclusive).
%
% #1 - a macro to store the integer (not a count register).
% #2 - the lower limit of the range.
% #3 - the upper limit of the range.
%
\def\pgfmathrandominteger#1#2#3{%
	\begingroup%
		\pgfmathgeneratepseudorandomnumber%
		\c@pgfmath@counta#2\relax%
		\c@pgfmath@countb#3\relax%
		\c@pgfmath@countc\c@pgfmath@countb%
		% OK. Maybe #2 > #3.
		\ifnum\c@pgfmath@counta>\c@pgfmath@countb\relax%
			\c@pgfmath@countc\c@pgfmath@countb%
			\c@pgfmath@countb\c@pgfmath@counta%
			\c@pgfmath@counta\c@pgfmath@countc%
		\fi%
		\advance\c@pgfmath@countc-\c@pgfmath@counta%
		\c@pgfmath@counta\pgfmathresult\relax%
		\c@pgfmath@countb\c@pgfmath@counta%
		\divide\c@pgfmath@countb\c@pgfmath@countc%
		\multiply\c@pgfmath@countb-\c@pgfmath@countc%
		\advance\c@pgfmath@counta\c@pgfmath@countb%
		\advance\c@pgfmath@counta1\relax%
		\edef\pgfmathresult{\the\c@pgfmath@counta}%
		\pgfmath@smuggleone{\pgfmathresult}%
	\endgroup%
	\edef#1{\pgfmathresult}%
}

% \pgfmathdeclarerandomlist
%
% Create a list to be used with \pgfmathrandomelement.
%
% #1 - the name of the list
% #2 - a list of comma separated elements.
%
\def\pgfmathdeclarerandomlist#1#2{%
	\def\pgfmath@randomlistname{#1}%
	\begingroup%
		\c@pgfmath@counta=1\relax%
		% {} is a possible random element so (locally) 
		% redefine \pgfmath@empty.
		\def\pgfmath@empty{pgfmath@empty}% 
		\expandafter\pgfmath@scanrandomlist#2,pgfmath@empty,}
\def\pgfmath@scanrandomlist#1,{%
	\def\pgfmath@scanneditem{#1}%
	\ifx\pgfmath@scanneditem\pgfmath@empty%
		\expandafter\xdef\csname pgfmath@randomlist@\pgfmath@randomlistname\endcsname{\the\c@pgfmath@counta}%
		\endgroup%	
	\else%
		\expandafter\gdef\csname pgfmath@randomlist@\pgfmath@randomlistname @\the\c@pgfmath@counta\endcsname{#1}%
		\advance\c@pgfmath@counta1\relax%
		\expandafter\pgfmath@scanrandomlist%
	\fi}

% \pgfmathrandomitem
%
% Return a 'randomly' selected element from a list.
%
% #1 - a macro to store the item.
% #2 - the name of the list.
%
\def\pgfmathrandomitem#1#2{%
	\pgfmath@ifundefined{pgfmath@randomlist@#2}{\pgfmath@error{Unknown random list `#2'}}{%
		\edef\pgfmath@randomlistlength{\csname pgfmath@randomlist@#2\endcsname}%
		\pgfmathrandominteger{\pgfmath@randomtemp}{1}{\pgfmath@randomlistlength}%
		\def#1{\csname pgfmath@randomlist@#2@\pgfmath@randomtemp\endcsname}}}%  Load the random stuff.

% \pgfmathadd
%
% Add #1 and #2.
%
\def\pgfmathadd#1#2{%
	\pgfmathparse{#1}\edef\pgfmath@adda{\pgfmathresult}%
	\pgfmathparse{#2}\edef\pgfmath@addb{\pgfmathresult}%
	\pgfmathadd@{\pgfmath@adda}{\pgfmath@addb}}
\def\pgfmathadd@#1#2{%
	\begingroup%
		\expandafter\pgfmath@x#1pt\relax%
		\expandafter\pgfmath@y#2pt\relax%
		\advance\pgfmath@x by\pgfmath@y%
		\pgfmath@returnone\pgfmath@x%
	\endgroup%
}

% \pgfmathsubtract
%
% Subtract #2 from #1.
%
\def\pgfmathsubtract#1#2{%
	\pgfmathparse{#1}\edef\pgfmath@subtracta{\pgfmathresult}%
	\pgfmathparse{#2}\edef\pgfmath@subtractb{\pgfmathresult}%
	\pgfmathsubtract@{\pgfmath@subtracta}{\pgfmath@subtractb}}

\def\pgfmathsubtract@#1#2{%
	\begingroup%
		\expandafter\pgfmath@x#1pt\relax%
		\expandafter\pgfmath@y#2pt\relax%
		\advance\pgfmath@x by-\pgfmath@y%
		\pgfmath@returnone\pgfmath@x%
	\endgroup%
}

% \pgfmathmultiply
%
% Multiply #1 by #2.
%
\def\pgfmathmultiply#1#2{%
	\pgfmathparse{#1}\edef\pgfmath@multiplya{\pgfmathresult}%
	\pgfmathparse{#2}\edef\pgfmath@multiplyb{\pgfmathresult}%
	\pgfmathmultiply@{\pgfmath@multiplya}{\pgfmath@multiplyb}}
\def\pgfmathmultiply@#1#2{%
	\begingroup%
		\expandafter\pgfmath@x#1pt\relax%
		\expandafter\pgfmath@x#2\pgfmath@x%
		\pgfmath@returnone\pgfmath@x%
	\endgroup%
}

% \pgfmathdivide
%
% Divide #1 by #2.
%
\def\pgfmathdivide#1#2{%
	\pgfmathparse{#1}\edef\pgfmath@dividea{\pgfmathresult}%
	\pgfmathparse{#2}\edef\pgfmath@divideb{\pgfmathresult}%
	\pgfmathdivide@{\pgfmath@dividea}{\pgfmath@divideb}}
\def\pgfmathdivide@#1#2{%
	\begingroup%
		\expandafter\pgfmath@x#1pt\relax%
		% If #2 is an integer use TeX arithmatic.
		\expandafter\pgfmath@xa#2pt\relax%
		\afterassignment\pgfmath@xa%
		\expandafter\c@pgfmath@counta\the\pgfmath@xa\relax%
		\ifdim\pgfmath@xa=0pt\relax%
			\divide\pgfmath@x\c@pgfmath@counta%
		\else%
			\pgfmathreciprocal@{#2}%
			\pgfmath@x=\pgfmathresult\pgfmath@x%
		\fi%
		\pgfmath@returnone\pgfmath@x%
	\endgroup%
}

% \pgfmathgreaterthan
%
% 1.0 if #1 > #2. Otherwise 0.0
%
\def\pgfmathgreaterthan#1#2{%
	\pgfmathparse{#1}\edef\pgfmath@greaterthana{\pgfmathresult}%
	\pgfmathparse{#2}\edef\pgfmath@greaterthanb{\pgfmathresult}%
	\pgfmathgreaterthan@{\pgfmath@greaterthana}{\pgfmath@greaterthanb}}
\def\pgfmathgreaterthan@#1#2{%
	\begingroup%
		\expandafter\pgfmath@x#1pt\relax%
		\expandafter\pgfmath@y#2pt\relax%
		\advance\pgfmath@x-\pgfmath@y%
		\ifdim\pgfmath@x>0pt\relax%
			\pgfmath@x1pt\relax%
		\else%
			\pgfmath@x0pt\relax%
		\fi%
		\pgfmath@returnone\pgfmath@x%
	\endgroup%
}

% \pgfmathlessthan
%
% 1.0 if #1< #2. Otherwise 0.0
%
\def\pgfmathlessthan#1#2{%
	\pgfmathparse{#1}\edef\pgfmath@lessthana{\pgfmathresult}%
	\pgfmathparse{#2}\edef\pgfmath@lessthanb{\pgfmathresult}%
	\pgfmathlessthan@{\pgfmath@lessthana}{\pgfmath@lessthanb}}
\def\pgfmathlessthan@#1#2{%
	\begingroup%
		\expandafter\pgfmath@x#1pt\relax%
		\expandafter\pgfmath@y#2pt\relax%
		\advance\pgfmath@x-\pgfmath@y\relax%
		\ifdim\pgfmath@x<0pt\relax%
			\pgfmath@x1pt\relax%
		\else%
			\pgfmath@x0pt\relax%
		\fi%
		\pgfmath@returnone\pgfmath@x%
	\endgroup%
}

% \pgfmathequalto
%
% 1.0 if #1 = #2. Otherwise 0.0
%
\def\pgfmathequalto#1#2{%
	\pgfmathparse{#1}\edef\pgfmath@equaltoa{\pgfmathresult}%
	\pgfmathparse{#2}\edef\pgfmath@equaltob{\pgfmathresult}%
	\pgfmathadd@{\pgfmath@equaltoa}{\pgfmath@equaltob}}
\def\pgfmathequalto@#1#2{%
	\begingroup%
		\expandafter\pgfmath@x#1pt\relax%
		\expandafter\pgfmath@y#2pt\relax%
		\advance\pgfmath@x-\pgfmath@y%
		\ifdim\pgfmath@x=0pt\relax%
			\pgfmath@x1pt\relax%
		\else%
			\pgfmath@x0pt\relax%
		\fi%
		\pgfmath@returnone\pgfmath@x%
	\endgroup%
}

% \pgfmathreciprocal
%
% 1 / #1
%
\def\pgfmathreciprocal#1{%
	\pgfmathparse{#1}%
	\pgfmathreciprocal@{\pgfmathresult}}
\def\pgfmathreciprocal@#1{%
	\begingroup%
		\expandafter\pgfmath@x#1pt\relax%
		\edef\pgfmath@reciprocaltemp{\pgfmath@tonumber{\pgfmath@x}}%
		\expandafter\pgfmathreciprocal@@\pgfmath@reciprocaltemp00000\pgfmath@}
\def\pgfmathreciprocal@@#1.#2#3#4#5#6#7\pgfmath@{%
		\c@pgfmath@counta#2#3#4#5#6\relax%
		% If the number is an integer, use TeX arithmatic.
		\ifnum\c@pgfmath@counta=0\relax%
			\pgfmath@x1pt\relax%
			\divide\pgfmath@x#1\relax%
		\else%
			\c@pgfmath@counta#1#2#3#4#5#6\relax%
			\c@pgfmath@countb1000000000\relax%
			\divide\c@pgfmath@countb\c@pgfmath@counta%
			\c@pgfmath@counta\c@pgfmath@countb%
			\divide\c@pgfmath@counta10000\relax%
			\pgfmath@x\c@pgfmath@counta pt\relax%
			\multiply\c@pgfmath@counta-10000\relax%
			\advance\c@pgfmath@countb\c@pgfmath@counta%
			\pgfmath@y\c@pgfmath@countb pt\relax%
			\pgfmath@y.1\pgfmath@y% Yes! This way is more accurate. Go figure...
			\pgfmath@y.1\pgfmath@y%	
			\pgfmath@y.1\pgfmath@y%	
			\pgfmath@y.1\pgfmath@y%			
			\advance\pgfmath@x\pgfmath@y%
		\fi%
		\pgfmath@returnone\pgfmath@x%
	\endgroup
}

	
% \pgfmathabs
%
% Calculate |#1|
%
\def\pgfmathabs#1{%
	\pgfmathparse{#1}%
	\pgfmathabsolute@{\pgfmathresult}}
\def\pgfmathabs@#1{%
	\begingroup%
		\expandafter\pgfmath@x#1pt\relax%
		\ifdim\pgfmath@x<0pt\relax%
			\pgfmath@x=-\pgfmath@x%
		\fi%
	\pgfmath@returnone\pgfmath@x%
	\endgroup%
}

% \pgfmathmod
%
% Calculate #1 mod #2.
%
\def\pgfmathmod#1#2{%
	\pgfmathparse{#1}\edef\pgfmath@moda{\pgfmathresult}%
	\pgfmathparse{#2}\edef\pgfmath@modb{\pgfmathresult}%
	\pgfmathmod@{\pgfmath@mod@a}{\pgfmath@modb}%
}
\def\pgfmathmod@#1#2{%
	\begingroup%
		\expandafter\pgfmath@x#1pt\relax%
		\pgfmath@xa\pgfmath@x%
		\expandafter\pgfmath@xb#2pt\relax%
		\c@pgfmath@counta=\pgfmath@xa%
		\c@pgfmath@countb=\pgfmath@xb%
		\divide\c@pgfmath@counta\c@pgfmath@countb%
		\multiply\pgfmath@xb\c@pgfmath@counta%
		\advance\pgfmath@x-\pgfmath@xb%
		\pgfmath@returnone\pgfmath@x%
	\endgroup%
}

% \pgfmathsqrt
%
% Square-root of #1.
%
%
\def\pgfmathsqrt#1{%
	\pgfmathparse{#1}%
	\pgfmathsqrt@{\pgfmathresult}}
\def\pgfmathsqrt@#1{%
	\begingroup%
		\expandafter\pgfmath@x#1pt\relax%
		\pgfmath@x=.01\pgfmath@x%
		\pgfmath@xa=\pgfmath@x%
		\pgfmath@xb=\pgfmath@x%
		\pgfmathloop
			\pgfmath@xc=\pgfmath@x%
			% If pgfmath@x >= 128pt, we get an Arithmetic overflow, so...
			% If x^2 >= 16384 then 16384/x < x
			\pgfmath@y=16383.99999pt\relax%
			\c@pgfmath@counta=\pgfmath@x%
			\divide\c@pgfmath@counta by655360\relax% Can't remember why we need the extra zero.
			\ifnum\c@pgfmath@counta=0\relax%
				\c@pgfmath@counta=1\relax%
			\fi%
			\divide\pgfmath@y by\c@pgfmath@counta%
			\ifdim\pgfmath@y<\pgfmath@x%
			\else%
				\pgfmath@x=\pgfmath@tonumber{\pgfmath@x}\pgfmath@x%
				\advance\pgfmath@x by-\pgfmath@xa\relax%
				\pgfmath@ya=\pgfmath@x%
				\pgfmathreciprocal@{\pgfmath@tonumber{\pgfmath@xc}}%
				\pgfmath@x=\pgfmathresult\pgfmath@ya%
				\pgfmath@x=-.5\pgfmath@x%
				\advance\pgfmath@x by\pgfmath@xb%
			\fi%
			% If the new root equals the old root, stop.
			\ifdim\pgfmath@x=\pgfmath@xb%
			\else%
				\pgfmath@xb=\pgfmath@x%
		\repeatpgfmathloop%
		\pgfmath@x=10.0\pgfmath@x%
		\pgfmath@returnone\pgfmath@x%
	\endgroup%
}


% \pgfmathpower
%
% Calculates #1 ^ #2
%
% #2 is expected to be an integer.
%
\def\pgfmathpower#1#2{%
	\pgfmathparse{#1}\edef\pgfmath@powera{\pgfmathresult}%
	\pgfmathparse{#2}\edef\pgfmath@powerb{\pgfmathresult}%
	\pgfmathpower@{\pgfmath@powera}{\pgfmath@powerb}}
\def\pgfmathpower@#1#2{%
	\begingroup%
		\expandafter\pgfmath@xa#1pt\relax%
		\afterassignment\pgfmath@gobbletilpgfmath@%
		\expandafter\c@pgfmath@counta#2\relax\pgfmath@
		% If #2 is negative, take the reciprocal of #1
		% and the absolute value of #2, and carry on.
		%
		\ifnum\c@pgfmath@counta<0\relax%
			\c@pgfmath@counta-\c@pgfmath@counta%
			\pgfmathreciprocal@{#1}%
			\pgfmath@xa\pgfmathresult pt\relax%
		\fi%
		\pgfmath@x=1pt\relax%
		\pgfmathloop%
			\ifnum\c@pgfmath@counta>0\relax%
				\ifodd\c@pgfmath@counta%
					\pgfmath@x\pgfmath@tonumber{\pgfmath@x}\pgfmath@xa%
				\fi
				\ifnum\c@pgfmath@counta>1\relax%
					\pgfmath@xa=\pgfmath@tonumber{\pgfmath@xa}\pgfmath@xa%
				\fi%
				\divide\c@pgfmath@counta by 2\relax%
		\repeatpgfmathloop%
		\pgfmath@returnone\pgfmath@x%
	\endgroup%
}	


% \pgfmathround
% 
% Half-up rounding.
%
\def\pgfmathround#1{%
	\pgfmathparse{#1}%
	\pgfmathround@{\pgfmathresult}}
\def\pgfmathround@#1{%
	\begingroup%
		\expandafter\pgfmath@x#1pt\relax%
		\afterassignment\pgfmath@gobbletilpgfmath@%
		\expandafter\c@pgfmath@counta\the\pgfmath@x\relax\pgfmath@%
		\pgfmath@y\pgfmath@x%
		\advance\pgfmath@y-\c@pgfmath@counta pt\relax%
		\pgfmath@x\c@pgfmath@counta pt\relax%
		\ifdim\pgfmath@x<0pt\relax%
			\advance\pgfmath@x-1pt\relax%
		\fi%
		\ifdim\pgfmath@y<0.5pt\relax%
		\else%
			\advance\pgfmath@x1pt\relax%
		\fi%
		\pgfmath@returnone\pgfmath@x%
	\endgroup%
}%

% \pgfmathfloor
% 
% Floor function.
%
\def\pgfmathfloor#1{%
	\pgfmathparse{#1}%
	\expandafter\pgfmathfloor@\expandafter{\pgfmathresult}}
\def\pgfmathfloor@#1{%
	\begingroup%
		\expandafter\pgfmath@x#1pt\relax%
		\afterassignment\pgfmath@gobbletilpgfmath@%
		\expandafter\c@pgfmath@counta\the\pgfmath@x\relax\pgfmath@%
		\pgfmath@x\c@pgfmath@counta pt\relax%
		\pgfmath@returnone\pgfmath@x%
	\endgroup
}%

% \pgfmathceil
% 
% Ceiling function.
%
\def\pgfmathceil#1{%
	\pgfmathparse{#1}%
	\expandafter\pgfmathceil@\expandafter{\pgfmathresult}}
\def\pgfmathceil@#1{%
	\begingroup%
		\expandafter\pgfmath@x#1pt\relax%
		\afterassignment\pgfmath@gobbletilpgfmath@%
		\expandafter\c@pgfmath@counta\the\pgfmath@x\relax\pgfmath@%
		\pgfmath@y\pgfmath@x%
		\advance\pgfmath@y-\c@pgfmath@counta pt\relax%
		\pgfmath@x\c@pgfmath@counta pt\relax%
		\ifdim\pgfmath@y>0pt\relax%
			\advance\pgfmath@x1pt\relax%
		\fi%
	\pgfmath@returnone\pgfmath@x%
	\endgroup%
}%

% \pgfmathexp
%
% A Maclaurens expansion for e^#1.
% 0 <= #1 < ln(16384).
%
\def\pgfmathexp#1{%
	\pgfmathparse{#1}%
	\expandafter\pgfmathexp@\expandafter{\pgfmathresult}}
\def\pgfmathexp@#1{%
	\begingroup%
		\pgfmath@x1pt\relax%
		\pgfmath@xa1pt\relax%
		\pgfmath@xb\pgfmath@x%
		\pgfmathloop%
			\pgfmath@xc\pgfmathcounter pt\relax%
			\c@pgfmath@counta\pgfmath@xc%
			\divide\c@pgfmath@counta65536\relax%
			\pgfmath@xc1pt\relax%
			\divide\pgfmath@xc\c@pgfmath@counta%
			\pgfmath@xa\pgfmath@tonumber{\pgfmath@xc}\pgfmath@xa%
			\expandafter\pgfmath@xa#1\pgfmath@xa%
			\advance\pgfmath@x\pgfmath@xa%
			\ifdim\pgfmath@x=\pgfmath@xb%
			\else%
				\pgfmath@xb\pgfmath@x%
		\repeatpgfmathloop%
	\pgfmath@returnone\pgfmath@x%
	\endgroup%
}



% \pgfmathvectorlength
%
% Calcluate the Eulidean length of a 2D vector.
%
% This based on polynomial approximation co-efficents
% contributed by Rouben Rostamian.
%
% #1 - the x component of the vector.
% #2 - the y component of the vector.
%
% P(x) = c0 + x^2 * (c1 + x^2 * (c2 + x^2 * ( c3 + c4 * x^2)))
\def\pgfmath@cE{-0.01019}
\def\pgfmath@cD{0.04453}
\def\pgfmath@cC{-0.11951}
\def\pgfmath@cB{0.49936}
\def\pgfmath@cA{1.00001}

\def\pgfmathveclen#1#2{%
	\pgfmathparse{#1}\edef\pgfmath@vecx{\pgfmathresult}%
	\pgfmathparse{#2}\edef\pgfmath@vecy{\pgfmathresult}%
	\pgfmathveclen@{\pgfmath@vecx}{\pgfmath@vecy}%
}
\def\pgfmathveclen@#1#2{%
	\begingroup%
		\expandafter\pgfmath@x#1pt\relax%
		\expandafter\pgfmath@y#2pt\relax%
		\ifdim\pgfmath@x<0pt\relax\pgfmath@x-\pgfmath@x\fi%
		\ifdim\pgfmath@y<0pt\relax\pgfmath@y-\pgfmath@y\fi%
		\ifdim\pgfmath@x>\pgfmath@y%
			\pgfmath@xa\pgfmath@x%
			\pgfmath@x\pgfmath@y%
			\pgfmath@y\pgfmath@xa%
		\fi%
		% We use a scaling factor to reduce errors.
		\ifdim\pgfmath@y>10000pt\relax%
			\c@pgfmath@counta1500\relax%
		\else%
			\ifdim\pgfmath@y>1000pt\relax%
				\c@pgfmath@counta150\relax%
			\else%
				\ifdim\pgfmath@y>100pt\relax%
					\c@pgfmath@counta50\relax%
				\else%
					\c@pgfmath@counta1\relax%
				\fi%
			\fi%
		\fi
		\divide\pgfmath@x by\c@pgfmath@counta\relax%
		\divide\pgfmath@y by\c@pgfmath@counta\relax%
		\pgfmathreciprocal@{\pgfmath@tonumber{\pgfmath@y}}%
		\pgfmath@x=\pgfmathresult\pgfmath@x%
		\pgfmath@xa=\pgfmath@tonumber{\pgfmath@x}\pgfmath@x%
		\edef\pgfmath@xsq{\pgfmath@tonumber{\pgfmath@xa}}%
		\pgfmath@x=\pgfmath@cE\pgfmath@xa%
		\advance\pgfmath@x by\pgfmath@cD pt\relax%
		\pgfmath@x=\pgfmath@xsq\pgfmath@x%
		\advance\pgfmath@x by\pgfmath@cC pt\relax%
		\pgfmath@x=\pgfmath@xsq\pgfmath@x%
		\advance\pgfmath@x by\pgfmath@cB pt\relax%
		\pgfmath@x=\pgfmath@xsq\pgfmath@x%
		\advance\pgfmath@x by\pgfmath@cA pt\relax%
		\ifdim\pgfmath@y<0pt\relax%
			\pgfmath@y=-\pgfmath@y%
		\fi%
		\pgfmath@x=\pgfmath@tonumber{\pgfmath@y}\pgfmath@x%
		% Invert the scaling factor.
		\multiply\pgfmath@x by\c@pgfmath@counta\relax%
		\pgfmath@returnone\pgfmath@x%
	\endgroup%
}

% \pgfmathmax
%
% Return the maximum of #1 or #2
%
\def\pgfmathmax#1#2{%
	\pgfmathparse@{#1}\edef\pgfmath@firstoperand{\pgfmathresult}%
	\pgfmathparse@{#2}\edef\pgfmath@secondoperand{\pgfmathresult}%
	\pgfmathmax@{\pgfmath@firstoperand}{\pgfmath@secondoperand}}
\def\pgfmathmax@#1#2{%
	\begingroup
		\expandafter\pgfmath@x#1pt\relax%
		\expandafter\pgfmath@y#2pt\relax%
		\ifdim\pgfmath@x>\pgfmath@y%
			\pgfmath@returnone\pgfmath@x%
		\else%
			\pgfmath@returnone\pgfmath@y%
		\fi%
	\endgroup}

% \pgfmathmax
%
% Return the minimim of #1 or #2
%
\def\pgfmathmin#1#2{%
	\pgfmathparse@{#1}\edef\pgfmath@firstoperand{\pgfmathresult}%
	\pgfmathparse@{#2}\edef\pgfmath@secondoperand{\pgfmathresult}%
	\pgfmathmin@{\pgfmath@firstoperand}{\pgfmath@secondoperand}}
\def\pgfmathmin@#1#2{%
	\begingroup
		\expandafter\pgfmath@x#1pt\relax%
		\expandafter\pgfmath@y#2pt\relax%
		\ifdim\pgfmath@x<\pgfmath@y%
			\pgfmath@returnone\pgfmath@x%
		\else%
			\pgfmath@returnone\pgfmath@y%
		\fi%
	\endgroup
}
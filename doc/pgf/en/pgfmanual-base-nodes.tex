% Copyright 2003 by Till Tantau <tantau@cs.tu-berlin.de>.
%
% This program can be redistributed and/or modified under the terms
% of the LaTeX Project Public License Distributed from CTAN
% archives in directory macros/latex/base/lppl.txt.


\section{Nodes and Shapes}

\label{section-shapes}

This section describes the |pgfbaseshapes| package.

\begin{package}{pgfbaseshapes}
  This package defines commands both for creating nodes and for
  creating shapes. The package is loaded automatically by |pgf|, but
  you can load it manually if you have  only included |pgfcore|.  
\end{package}


\subsection{Overview}

\pgfname\ comes with a sophisticated set of commands for creating
\emph{nodes} and \emph{shapes}. A \emph{node} is a graphical object
that consists (typically) of a text label and some additional stroked
or filled paths. Each node has a certain \emph{shape}, which may be
something simple like a |rectangle| or a |circle|, but it may also be
something complicated like a |uml class diagram| (this shape is
currently not implemented, though). Different nodes that have the same
shape may look quite different, however, since shapes (need not)
specify whether the shape path is stroked or filled.

\subsubsection{Creating and Referencing Nodes}

You create a node by calling the macro |\pgfnode|. This macro takes
several parameters and draws the requested shape at a certain
position. In addition, it will ``remember'' the node's position within
the current |{pgfpicture}|. You can then, later on, refer to the
node's position. Coordinate transformations are ``fully supported,''
which means that if you used coordinate transformations to shift or
rotate the shape of a node, the node's position will still be correctly
determined by \pgfname. This is \emph{not} the case if you use canvas
transformations, instead.

\subsubsection{Anchors}

An important property of a node or a shape in general are its
\emph{anchors}. Anchors are ``important'' positions in a shape. For
example, the |center| anchor lies at the center of a shape, the
|north| anchor is usually ``at the top, in the middle'' of a shape,
the |text| anchor is the lower left corner of the shape's label, and
so on.

Anchors are important both when you create a node and when you
reference it. When you create a node, you specify the node's
``position'' by asking \pgfname\ to place the shape in such a way that
a certain anchor lies at a certain point. For example, you might ask
that the node is placed such that the |north| anchor is at the
origin. This will effectively cause the node to be placed below the
origin.

When you reference a node, you always reference an anchor of the
node. For example, when you request the ``|north| anchor of the node
just placed'' you will get the origin. However, you can also request
the ``|south| anchor of this node,'' which will give you a point
somewhere below the origin. When a coordinate transformation was in
force at the time of creation of a node, all anchors are also
transformed accordingly.

\subsubsection{Layers of a Shape}

The simplest shape, the |coordinate|, has just one anchor, namely the
|center|, and a label (which is usually empty). More complicated
shapes like the |rectangle| shape also have a \emph{background
  path}. This is a \pgfname-path that is defined by the shape. The
shape does not prescribe what should happen with the path: When a node
is created this path may be stroked (resulting in a frame around the
label), filled (resulting in a background color for the text), or just
discarded.

Although most shapes consist just of a background path plus some label
text, when a shape is drawn, up to seven different layers are drawn:

\begin{enumerate}
\item
  The ``behind the background layer.'' Unlike the background path,
  which be used in different ways by different nodes, these graphic
  commands given for this layer will always stroke or
  always fill the path they construct. They might also insert some
  text that is ``behind everything.''
\item
  The background path layer. How this path is used depends on how the
  arguments of the |\pgfnode| command.
\item
  The ``before the background path layer.'' This layer works like the
  first one, only the commands of this layer are executed after the
  background path has been used (in whatever way the creator of the
  node chose).
\item
  The label layer. This layer inserts the node's text box.
\item
  The ``behind the foreground layer.'' This layer, like the
  seventh layer, once more contain graphic commands that are ``simply
  executed.''
\item
  The foreground path layer . This path is treated in the same way as the
  background path, only it is drawn only after the label text has been
  drawn.
\item
  The ``before the foreground layer.''
\end{enumerate}

Which of these layers are actually used depends on the shape.

\subsection{Creating Nodes}

You create a node using the following command:

\begin{command}{\pgfnode\marg{shape}\marg{anchor}\marg{label
      text}\marg{name}\marg{path usage command}} 
  This command creates a new node. The \meta{shape} of the node must
  have been declared previously using |\pgfdeclareshape|.

  The shape is shifted such that the \meta{anchor} is at the
  origin. In order to place the shape somewhere else, use the
  coordinate transformation prior to calling this command.

  The \meta{name} is a name for later reference. If no name is given,
  nothing will be ``saved'' for the node, it will just be drawn.

  The \meta{path usage command} is executed for the background and the
  foreground path (if the shape defines them).

\begin{codeexample}[]
\begin{tikzpicture}
  \draw[help lines] (0,0) grid (4,3);
  {
    \pgftransformshift{\pgfpoint{1cm}{1cm}}
    \pgfnode{rectangle}{north}{Hello World}{hellonode}{\pgfusepath{stroke}}
  }
  {
    \color{red!20}
    \pgftransformrotate{10}
    \pgftransformshift{\pgfpoint{3cm}{1cm}}
    \pgfnode{rectangle}{center}
      {\color{black}Hello World}{hellonode}{\pgfusepath{fill}}
  }
\end{tikzpicture}
\end{codeexample}

  As can be seen, all coordinate transformations are also applied to
  the text of the shape. Sometimes, it is desirable that the
  transformations are applied to the point where the shape will be
  anchored, but you do not wish the shape itself to the
  transformed. In this case, you should call
  |\pgftransformresetnontranslations| prior to calling the |\pgfnode|
  command. 

\begin{codeexample}[]
\begin{tikzpicture}
  \draw[help lines] (0,0) grid (4,3);
  {
    \color{red!20}
    \pgftransformrotate{10}
    \pgftransformshift{\pgfpoint{3cm}{1cm}}
    \pgftransformresetnontranslations
    \pgfnode{rectangle}{center}
      {\color{black}Hello World}{hellonode}{\pgfusepath{fill}}
  }
\end{tikzpicture}
\end{codeexample}
\end{command}

There are a number of values that have an influence on the size of a
node. These parameters can be changed using the following commands:

\begin{command}{\pgfsetshapeminwidth\marg{dimension}}
  This command sets the macro \declare{|\pgfshapeminwidth|} to
  \meta{dimension}. This dimension is the \emph{recommended} minimum
  width of a shape. Thus, when a shape is drawn and when the shape's
  width would be smaller than \meta{dimension}, the shape's width is
  enlarged by adding some empty space.

  Note that this value is just a recommendation. A shape may choose to
  ignore the value of |\pgfshapeminwidth| altogether.
  
\begin{codeexample}[]
\begin{tikzpicture}
  \draw[help lines] (-2,0) grid (2,1);

  \pgfsetshapeminwidth{3cm}
  \pgfnode{rectangle}{center}{Hello World}{}{\pgfusepath{stroke}}
\end{tikzpicture}
\end{codeexample}
\end{command}

\begin{command}{\pgfsetshapeminheight\marg{dimension}}
  Works like |\pgfsetshapeminwidth|.
\end{command}


\begin{command}{\pgfsetshapeinnerxsep\marg{dimension}}
  This command sets the macro \declare{|\pgfshapeinnerxsep|} to
  \meta{dimension}. This dimension is the \emph{recommended} horizontal
  inner separation between the label text and the background path. As
  before, this value is just a recommendation and a shape may choose
  to ignore the value of |\pgfshapeinnerxsep|.
  
\begin{codeexample}[]
\begin{tikzpicture}
  \draw[help lines] (-2,0) grid (2,1);

  \pgfsetshapeinnerxsep{1cm}
  \pgfnode{rectangle}{center}{Hello World}{}{\pgfusepath{stroke}}
\end{tikzpicture}
\end{codeexample}
\end{command}

\begin{command}{\pgfsetshapeinnerysep\marg{dimension}}
  Works like |\pgfsetshapeinnerysep|.
\end{command}



\begin{command}{\pgfsetshapeouterxsep\marg{dimension}}
  This command sets the macro \declare{|\pgfshapeouterxsep|} to
  \meta{dimension}. This dimension is the recommended horizontal
  outer separation between the background path and the ``outer
  anchors.'' For example, if \meta{dimension} is |1cm| then the
  |north| anchor will be 1cm above the top of the background path.

  As before, this value is just a recommendation.
  
\begin{codeexample}[]
\begin{tikzpicture}
  \draw[help lines] (-2,0) grid (2,1);

  \pgfsetshapeouterxsep{.5cm}
  \pgfnode{rectangle}{center}{Hello World}{x}{\pgfusepath{stroke}}

  \pgfpathcircle{\pgfpointanchor{x}{north}}{2pt}
  \pgfpathcircle{\pgfpointanchor{x}{south}}{2pt}
  \pgfpathcircle{\pgfpointanchor{x}{east}}{2pt}
  \pgfpathcircle{\pgfpointanchor{x}{west}}{2pt}
  \pgfpathcircle{\pgfpointanchor{x}{north east}}{2pt}
  \pgfusepath{fill}
\end{tikzpicture}
\end{codeexample}
\end{command}

\begin{command}{\pgfsetshapeouterysep\marg{dimension}}
  Works like |\pgfsetshapeouterysep|.
\end{command}


\subsection{Using Anchors}

Each shape defines a set of anchors. We saw already that the anchors
are used when the shape is drawn: the shape is placed in such a way
that the given anchor is at the origin (which in turn is typically
translated somewhere else).

One has to look up the set of anchors of each shape, there is no
``default'' set of anchors, except for the |center| anchor, which
should always be present.

Once a node has been defined, you can refer to its anchors using the
following commands:

\begin{command}{\pgfpointanchor\marg{node}\marg{anchor}}
  This command is another ``point command'' like the commands
  described in Section~\ref{section-points}. It returns the coordinate
  of the given \meta{anchor} in the given \meta{node}. The command can
  be used in commands like |\pgfpathmoveto|.

\begin{codeexample}[]
\begin{pgfpicture}
  \pgftransformrotate{30}
  \pgfnode{rectangle}{center}{Hello World!}{x}{\pgfusepath{stroke}}

  \pgfpathcircle{\pgfpointanchor{x}{north}}{2pt}
  \pgfpathcircle{\pgfpointanchor{x}{south}}{2pt}
  \pgfpathcircle{\pgfpointanchor{x}{east}}{2pt}
  \pgfpathcircle{\pgfpointanchor{x}{west}}{2pt}
  \pgfpathcircle{\pgfpointanchor{x}{north east}}{2pt}
  \pgfusepath{fill}
\end{pgfpicture}
\end{codeexample}

  In the above example, you may have noticed something curious: The
  rotation transformation is still in force when the anchors are
  invoked, but it does not seem to have an effect. You might expect
  that the rotation should apply to the already rotated points once
  more.

  However, |\pgfpointanchor| returns a point that takes the current
  transformation matrix into account: \emph{The inverse transformation
    to the current coordinate transformation is applied to anchor
    point before returning it.}

  This behavior may seem a bit strange, but you will find it very
  natural in most cases. If you really want to apply a transformation
  to an anchor point (for example, to ``shift it away'' a little bit),
  you have to invoke |\pgfpointanchor| without any transformations in
  force. Here is an example:

\makeatletter
\begin{codeexample}[]
\begin{pgfpicture}
  \pgftransformrotate{30}
  \pgfnode{rectangle}{center}{Hello World!}{x}{\pgfusepath{stroke}}

  {
    \pgftransformreset
    \pgfpointanchor{x}{east}
    \xdef\mycoordinate{\noexpand\pgfpoint{\the\pgf@x}{\the\pgf@y}}
  }
    
  \pgfpathcircle{\mycoordinate}{2pt}
  \pgfusepath{fill}
\end{pgfpicture}
\end{codeexample}
\end{command}

\begin{command}{\pgfpointshapeborder\marg{node}\marg{point}}
  This command returns the point on the border of the shape that lies
  on a straight line from the center of the node to \meta{point}. For
  complex shapes it is not guaranteed that this point will actually
  lie on the border, it may be on the border of a ``simplified''
  version of the shape.

\begin{codeexample}[]
\begin{pgfpicture}
  \begin{pgfscope}
    \pgftransformrotate{30}
    \pgfnode{rectangle}{center}{Hello World!}{x}{\pgfusepath{stroke}}
  \end{pgfscope}
  \pgfpathcircle{\pgfpointshapeborder{x}{\pgfpoint{2cm}{1cm}}}{2pt}
  \pgfpathcircle{\pgfpoint{2cm}{1cm}}{2pt}
  \pgfpathcircle{\pgfpointshapeborder{x}{\pgfpoint{-1cm}{1cm}}}{2pt}
  \pgfpathcircle{\pgfpoint{-1cm}{1cm}}{2pt}
  \pgfusepath{fill}
\end{pgfpicture}
\end{codeexample}
\end{command}



\subsection{Declaring New Shapes}

Defining a shape is, unfortunately, a not-quite-trivial process. The
reason is that shapes need to be both very flexible (their size will
vary greatly according to circumstances) and they need to be
constructed reasonably ``fast.'' \pgfname\ must be able to handle
pictures with several hundreds of nodes and documents with thousands
of nodes in total. It would not do if \pgfname\ had to compute and
store, say, dozens of anchor positions for the nodes of all pictures. 

\subsubsection{What Must Be Defined For a Shape?}

In order to define a new shape, you must provide:
\begin{itemize}
\item
  a \emph{shape name},
\item
  code for computing the  \emph{saved anchors} and \emph{saved
    dimensions}, 
\item
  code for computing \emph{anchor} positions in terms of the saved anchors,
\item
  optionally code for the \emph{background path} and \emph{foreground path},
\item
  optionally code for \emph{things to be drawn before or behind} the
  background and foreground paths.
\end{itemize}


\subsubsection{Normal Anchors Versus Saved Anchors}

Anchors  are special places in shape. For example, the |north east|
anchor, which is a normal anchor, lies at the upper right corner of
the  |rectangle| shape, as does |\northeast|, which is a saved
anchor. The difference is the following: \emph{saved anchors are 
  computed and stored for each node, anchors are only computed as
  needed.} The user only has access to the normal anchors, but a
normal anchor can just ``copy'' or ``pass through'' the location of a
saved anchor. 

The idea behind all this is that a shape can declare a very large
number of normal anchors, but when a node of this shape is created,
these anchors are not actually computed. However, this causes a
problem: When we wish to reference an anchor of a node some time
later, we must still able to compute the position of the anchor. For
this, we may need a lot of information: What was the transformation
matrix that was in force when the node was created? What was the size
of the text box? What were the values of the different separation
dimensions? And so on. 

To solve this problem, \pgfname\ will always compute the locations of
all \emph{saved anchors} and store these positions. Then, when an
normal anchor position is requested later on, the anchor position can
be given just from knowing where the locations of the saved anchors.

As an example, consider the |rectangle| shape. For this shape two
anchors are saved: The |\northeast| corner and the |\southwest|
corner. A normal anchor like |north west| can now easily be expressed
in terms of these coordinates: Take the $x$-position of the
|\southwest| point  and the $y$-position of the |\northeast| point. 
The |rectangle| shape currently defines 13 normal anchors, but needs
only two saved anchors. Adding new anchors like a  |south south east|
anchor would not increase the memory and computation requirements of
pictures. 

All anchors (both saved and normal) are specified in a local
\emph{shape coordinate space}. This is also true for the background
and foreground paths. The |\pgfnode| macro will automatically apply
appropriate transformations to the coordinates so that the shape is
shifted to the right anchor or otherwise transformed. 


\subsubsection{Command for Declaring New Shapes}

The following command declares a new shape:
\begin{command}{\pgfdeclareshape\marg{shape name}\marg{shape
      specification}}
  This command declares a new shape named \meta{shape name}. The shape
  name can later be used in commands like |\pgfnode|.

  The \meta{shape specification} is some \TeX\ code containing calls
  to special commands that are only defined inside the \meta{shape
    specification} (this is like command like |\draw| that are only
  available inside the |{tikzpicture}| environment).

  \example Here is the code of the |coordinate| shape:
\begin{codeexample}[code only]
\pgfdeclareshape{coordinate}
{
  \savedanchor\centerpoint{%
    \pgf@x=.5\wd\pgfshapebox%
    \pgf@y=.5\ht\pgfshapebox%
    \advance\pgf@y by -.5\dp\pgfshapebox%
  }
  \anchor{center}{\centerpoint}
  \anchorborder{\centerpoint}
}
\end{codeexample}

  The special commands are explained next. In the examples given for
  the special commands a new shape will be constructed, which we might
  call |simple rectangle|. It should behave like the normal rectangle
  shape, only without bothering about the fine details like inner and
  outer separations. The skeleton for the shape is the following.
\begin{codeexample}[code only]
\pgfdeclareshape{simple rectangle}{
  ...
}
\end{codeexample}

  \begin{command}{\savedanchor\marg{command}\marg{code}}
    This command declares a saved anchor. \meta{command} should be a
    \TeX\ macro name like |\centerpoint|.

    The \meta{code} will be executed each time |\pgfnode| is called to
    create a node of the shape \meta{shape name}. When the \meta{code}
    is executed, the box |\pgfshapebox| will contain the text label of
    the node. Possibly, this box is void. The \meta{code} can now
    use the width, height, and depth of the box to compute the
    location of the saved anchor. In addition, the \meta{code} can
    take into account the valued of dimensions like
    |\pgfshapeminwidth| or |\pgfshapeinnerxsep|. Furthermore, the
    \meta{code} can take into consideration the values of any further
    shape-specific variables that are set at the moment when
    |\pgfnode| is called.

    The net effect of the \meta{code} should be to set the two \TeX\
    dimensions |\pgf@x| and |\pgf@y|. One way to achieve this is to
    say |\pgfpoint{|\meta{x value}|}{|\meta{y value}|}| at the end of
    the \meta{code}, but you can also just set these variables.

    The values of |\pgf@x| and |\pgf@y| have after the code has been
    executed, let us call them $x$ and $y$, will be recorded and
    stored together with the node that is created by the command
    |\pgfnode|.

    The macro \meta{command} is defined to be
    |\pgfpoint{|$x$|}{|$y$|}|. However, the \meta{command} is only
    locally defined while anchor positions are being computed. Thus,
    it is possible to use very simple names for \meta{command}, like
    |\center| or |\a|, without causing a name-clash. (To be precise,
    very simple \meta{command} names will clash with existing names,
    but only locally inside the computation of anchor positions; and
    we do not need the normal |\center| command during these
    computations.)

    For our |simple rectangle| shape, we will need only one saved
    anchor: The upper right corner. The lower left corner could either
    be the origin or the ``mirrored'' upper right corner, depending on
    whether we want the text label to have its lower left corner at
    the origin or whether the text label should be centered on the
    origin. Either will be fine, for the final shape this will make no
    difference since the shape will be shifted anyway. So, let us
    assume that the text label is centered on the origin (this will be
    specified later on using the |text| anchor). We get 
    the following code for the upper right corner:
\begin{codeexample}[code only]
\shapepoint{\upperrightcorner}{
  \pgf@y=.5\ht\pgfshapebox % height of the box, ignoring the depth
  \pgf@x=.5\wd\pgfshapebox % width of the box
}
\end{codeexample}

    If we wanted to take, say, the |\pgfshapeminwidth| into account,
    we could use the following code:
    
\begin{codeexample}[code only]
\shapepoint{\upperrightcorner}{
  \pgf@y=.\ht\pgfshapebox % height of the box
  \pgf@x=.\wd\pgfshapebox % width of the box
  \setlength{\pgf@xa}{\pgfshapeminwidth}
  \ifdim\pgf@x<.5\pgf@xa
    \pgf@x=.5\pgf@xa
  \fi
}
\end{codeexample}
    Note that we could not have written |.5\pgfshapeminwidth| since
    the minium width is stored in a ``plain text macro,'' not as a
    real dimension. So if |\pgfshapeminwidth| depth were 
    2cm, writing |.5\pgfshapeminwidth| would yield the same as |.52cm|.

    In the ``real'' |rectangle| shape the code is somewhat more
    complex, but you get the basic idea.
  \end{command}  
  \begin{command}{\saveddimen\marg{command}\marg{code}}
    This command is similar to |\savedanchor|, only instead of setting
    \meta{command} to |\pgfpoint{|$x$|}{|$y$|}|, the \meta{command} is
    set just to (the value of) $x$.

    In the |simple rectangle| shape we might use a saved dimension to
    store the depth of the shape box.
  
\begin{codeexample}[code only]
\shapedimen{\depth}{
  \pgf@x=\dp\pgfshapebox 
}
\end{codeexample}
  \end{command}  
  \begin{command}{\anchor\marg{name}\marg{code}}
    This command declares an anchor named \meta{name}. Unlike for saved
    anchors, the \meta{code} will not be executed each time a node is
    declared. Rather, the \meta{code} is only executed when the anchor
    is specifically requested; either for anchoring the node during
    its creation or as a  position in the shape referenced later on.

    The \meta{name} is a quite arbitrary string that is not ``passed
    down'' to the system level. Thus, names like |south| or |1| or
    |::| would all be fine.

    A saved anchor is not automatically also a normal anchor. If you
    wish to give the users access to a saved anchor you must declare a
    normal anchor that just returns the position of the saved anchor.

    When the \meta{code} is executed, all saved anchor macros will be
    defined. Thus, you can reference them in your \meta{code}. The
    effect of the \meta{code} should be to set the values of |\pgf@x|
    and |\pgf@y| to the coordinates of the anchor.

    Let us consider some example for the |simple rectangle|
    shape. First, we would like to make the upper right corner
    publicly available, for example as |north east|:
    
\begin{codeexample}[code only]
\anchor{north east}{\upperrightcorner}
\end{codeexample}

    The |\upperrightcorner| macro will set |\pgf@x| and |\pgf@y| to
    the coordinates of the upper right corner. Thus, |\pgf@x| and
    |\pgf@y| will have exactly the right values at the end of the
    anchor's code.

    Next, let us define a |north west| anchor. For this anchor, we can
    negate the |\pgf@x| variable:
   
\begin{codeexample}[code only]
\anchor{north west}{
  \upperrightcorner
  \pgf@x=-\pgf@x
}
\end{codeexample}

    Finally, it is a good idea to always define a |center| anchor,
    which will be the default location for a shape.

\begin{codeexample}[code only]
\anchor{center}{\pgfpointorigin}
\end{codeexample}

    You might wonder whether we should not take into consideration
    that the node is not placed at the origin, but has been shifted
    somewhere. However, the anchor positions are always specified in
    the shapes ``private'' coordinate system. The ``outer''
    transformation that has been applied to the shape upon its
    creation is applied automatically to the coordinates returned by
    the anchor's \meta{code}.

    There is one anchor that is special: The |text| anchor. This
    anchor is used upon creation of a node to determine the lower left
    corner of the text label (within the private coordinate system of
    the shape). By default, the |text| anchor is at the origin, but
    you may change this. For example, we would say
\begin{codeexample}[code only]
\anchor{text}{\pgfpoint{-.5\wd\pgfshapebox}{-.5\ht\pgfshapebox}}
\end{codeexample}
    to center the text label on the origin in the shape coordinate space. 
  \end{command}  
  \begin{command}{\anchorborder\marg{code}}
    A \emph{border anchor} is an anchor point on the border of the
    shape. What exactly is considered as the ``border'' of the shape
    depends on the shape.

    When the user request a point on the border of the shape using the
    |\pgfpointshapeborder| command, the \meta{code} will be executed
    to discern this point. When the execution of  the \meta{code}
    starts, the dimensions |\pgf@x| and |\pgf@y| will have been set to
    a location $p$ in the shape's coordinate system. It is now the job of
    the \meta{code} to setup |\pgf@x| and |\pgf@y| such that they
    specify that point on the shape's border that lies on a straight
    line from the shape's center to the point $p$. Usually, this is a
    somewhat complicated computation, involving many case distinctions
    and some basic math.

    For our |simple rectangle| we must compute a point on the border
    of a rectangle whose one corner is the origin (ignoring the depth
    for simplicity) and whose other corner is |\upperrightcorner|. The
    following code might be used:
\begin{codeexample}[code only]
\anchorborder{%
  % Call a function that computes a border point. Since this
  % function will modify dimensions like \pgf@x, we must move it to
  % other dimensions.
  \@tempdima=\pgf@x
  \@tempdimb=\pgf@y
  \pgfpointborderrectangle{\pgfpoint{\@tempdima}{\@tempdimb}}{\upperrightcorner}
}
\end{codeexample}
  \end{command}  
  \begin{command}{\backgroundpath\marg{code}}
    This command specifies the path that ``makes up'' the background
    of the shape. Note that the shape cannot prescribe what is going
    to happen with the path: It might be drawn, shaded, filled, or
    even thrown away. If you want to specify that something should
    ``always'' happen when this shape is drawn (for example, if the
    shape is a stop-sign, we \emph{always} want it to be filled with a
    red color), you can use commands like |\beforebackgroundpath|,
    explained below.

    When the \meta{code} is executed, all saved anchors will be in
    effect. The \meta{code} should contain path construction
    commands.

    For our |simple rectangle|, the following code might be used:
\begin{codeexample}[code only]
\backgroundpath{
  \pgfpathrectanglecorners
    {\upperrightcorner}
    {\pgfpointscale{-1}{\upperrightcorner}}
}  
\end{codeexample}
    As the name suggests, the background path is used ``behind'' the
    text label. Thus, this path is used first, then the text label is
    drawn, possibly obscuring part of the path.
  \end{command}  
  \begin{command}{\foregroundpath\marg{code}}
    This command works like |\backgroundpath|, only it is invoked
    after the text label has been drawn. This means that this path can
    possibly obscure (part of) the text label.
  \end{command}  
  \begin{command}{\behindbackgroundpath\marg{code}}
    Unlike the previous two commands, \meta{code} should not only
    construct a path, it should also use this path in whatever way is
    appropriate. For example, the \meta{code} might fill some area
    with a unform color.

    Whatever the \meta{code} does, it does it first. This means that
    any drawing done by \meta{code} will be even behind the background
    path.

    Note that the \meta{code} is protected with a |{pgfscope}|.
  \end{command}  
  \begin{command}{\beforebackgroundpath\marg{code}}
    This command works like |\behindbackgroundpath|, only the
    \meta{code} is executed after the background path has been used,
    but before the text label is drawn.
  \end{command}  
  \begin{command}{\behindforegroundpath\marg{code}}
    The \meta{code} is executed after the text label has been drawn,
    but before the foreground path is used.
  \end{command}  
  \begin{command}{\beforeforegroundpath\marg{code}}
    This \meta{code} is executed at the very end.
  \end{command}  
  \begin{command}{\inheritsavedanchors|[from=|\marg{another shape name}|]|}
    This command allows you to inherit the code for saved anchors from
    \meta{another shape name}. The idea is that if you wish to create
    a new shape that is just a small modification of a another shape,
    you can recycle the code used for \meta{another shape name}.

    The effect of this command is the same as if you had called
    |\savedanchor| and |\saveddimen| for each saved anchor or saved
    dimension declared in \meta{another shape name}. Thus, it is not
    possible to ``selectively'' inherit only some saved anchors, you
    always have to inherit all saved anchors from another
    shape. However, you can inherit the saved anchors of more than one
    shape by calling this command several times.
  \end{command}  
  \begin{command}{\inheritbehindbackgroundpath|[from=|\marg{another shape name}|]|}
    This command can be used to inherit the code used for the
    drawings behind the background path from \meta{another shape name}. 
  \end{command}  
  \begin{command}{\inheritbackgroundpath|[from=|\marg{another shape name}|]|}
    Inherits the background path code from \meta{another shape name}.
  \end{command}  
  \begin{command}{\inheritbeforebackgroundpath|[from=|\marg{another shape name}|]|}
    Inherits the before background path code from \meta{another shape name}.
  \end{command}  
  \begin{command}{\inheritbehindforegroundpath|[from=|\marg{another shape name}|]|}
    Inherits the behind foreground path code from \meta{another shape name}.
  \end{command}  
  \begin{command}{\inheritforegroundpath|[from=|\marg{another shape name}|]|}
    Inherits the foreground path code from \meta{another shape name}.
  \end{command}  
  \begin{command}{\inheritbeforeforegroundpath|[from=|\marg{another shape name}|]|}
    Inherits the before foreground path code from \meta{another shape name}.
  \end{command}  
  \begin{command}{\inheritanchor|[from=|\marg{another shape name}|]|\marg{name}}
    Inherits the code of one specific anchor named \meta{name} from
    \meta{another shape name}. Thus, unlike saved anchors, which must
    be inherited collectively, normal anchors can and must be
    inherited individually.
  \end{command}  
  \begin{command}{\inheritanchorborder|[from=|\marg{another shape name}|]|}
    Inherits the border anchor code from \meta{another shape name}.
  \end{command}

  The following example shows how a shape can be defined that relies
  heavily on inheritance:
\makeatletter
\begin{codeexample}[]
\pgfdeclareshape{document}{
  \inheritsavedanchors[from=rectangle] % this nearly a rectangle
  \inheritanchorborder[from=rectangle]
  \inheritanchor[from=rectangle]{center}
  \inheritanchor[from=rectangle]{north}
  \inheritanchor[from=rectangle]{south}
  \inheritanchor[from=rectangle]{west}
  \inheritanchor[from=rectangle]{east}
  % ... and possibly more
  \backgroundpath{% this is new
    % store lower right in xa/ya and upper right in xb/yb
    \southwest \pgf@xa=\pgf@x \pgf@ya=\pgf@y
    \northeast \pgf@xb=\pgf@x \pgf@yb=\pgf@y
    % compute corner of ``flipped page''
    \pgf@xc=\pgf@xb \advance\pgf@xc by-5pt % this should be a parameter
    \pgf@yc=\pgf@yb \advance\pgf@yc by-5pt
    % construct main path
    \pgfpathmoveto{\pgfpoint{\pgf@xa}{\pgf@ya}}
    \pgfpathlineto{\pgfpoint{\pgf@xa}{\pgf@yb}}
    \pgfpathlineto{\pgfpoint{\pgf@xc}{\pgf@yb}}
    \pgfpathlineto{\pgfpoint{\pgf@xb}{\pgf@yc}}
    \pgfpathlineto{\pgfpoint{\pgf@xb}{\pgf@ya}}
    \pgfpathclose
    % add little corner
    \pgfpathmoveto{\pgfpoint{\pgf@xc}{\pgf@yb}}
    \pgfpathlineto{\pgfpoint{\pgf@xc}{\pgf@yc}}
    \pgfpathlineto{\pgfpoint{\pgf@xb}{\pgf@yc}}
    \pgfpathlineto{\pgfpoint{\pgf@xc}{\pgf@yc}}
 }
}\hskip-1.2cm
\begin{tikzpicture}
  \node[shade,draw,shape=document,inner sep=2ex] (x) {Remark};
  \node[fill=yellow,draw,ellipse,double]
    at ([shift=(-80:3cm)]x) (y) {Use Case};

  \draw[dashed] (x) -- (y);  
\end{tikzpicture}
\end{codeexample}
  
\end{command}




\subsection{Predefined Shapes}

\begin{shape}{rectangle}
  This shape is a rectangle tightly fitting the text box. Use inner or
  outer separation to increase the distance between the text box and
  the border and the anchors. The following figure shows the anchors
  this shape defines; the anchors |10| and |130| are example of border
  anchors. 
\begin{codeexample}[]
\Huge
\begin{tikzpicture}
  \node[name=s,shape=rectangle,style=shape example] {Rectangle\vrule width 1pt height 2cm};
  \foreach \anchor/\placement in
    {north west/above left, north/above, north east/above right, 
     west/left, center/above, east/right, 
     mid west/right, mid/above, mid east/left, 
     base west/left, base/below, base east/right, 
     south west/below left, south/below, south east/below right, 
     text/left, 10/right, 130/above}
    \draw[shift=(s.\anchor)] plot[mark=x] coordinates {(0,0)}
      node[\placement] {\scriptsize\texttt{(s.\anchor)}};
\end{tikzpicture}
\end{codeexample}
\end{shape}



\begin{shape}{coordinate}
  The |coordinate| shape is a special shape that is mainly intended to
  be used to store locations using the node mechanism. This shape does
  not have any background path and options like |draw| have no effect on
  it. If you specify some text, this text will be typeset, but only ``a
  bit unwillingly'' since this shape is not really intended for drawing
  text.

  \tikzname\ handles this shape in a special way, see
  Section~\ref{section-tikz-coordinate-shape}. 
\end{shape}


\begin{shape}{circle}
  This shape is a circle tightly fitting the text box.
\begin{codeexample}[]
\Huge
\begin{tikzpicture}
  \node[name=s,shape=circle,style=shape example] {Circle\vrule width 1pt height 2cm};
  \foreach \anchor/\placement in
    {north west/above left, north/above, north east/above right, 
     west/left, center/above, east/right, 
     mid west/right, mid/above, mid east/left, 
     base west/left, base/below, base east/right, 
     south west/below left, south/below, south east/below right, 
     text/left, 10/right, 130/above}
     \draw[shift=(s.\anchor)] plot[mark=x] coordinates {(0,0)}
       node[\placement] {\scriptsize\texttt{(s.\anchor)}};
\end{tikzpicture}
\end{codeexample}
\end{shape}



%%% Local Variables: 
%%% mode: latex
%%% TeX-master: "pgfmanual"
%%% End: 

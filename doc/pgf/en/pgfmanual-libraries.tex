% Copyright 2003 by Till Tantau <tantau@cs.tu-berlin.de>.
%
% This program can be redistributed and/or modified under the terms
% of the LaTeX Project Public License Distributed from CTAN
% archives in directory macros/latex/base/lppl.txt.


\section{Libraries}

\subsection{Arrow Tip Library}
\label{section-library-arrows}

\begin{package}{pgflibraryarrows}
  The package defines additional arrow tips, which are described
  below. See page~\pageref{standard-arrows} for the arrows tips that
  are defined by default. Note that neither the standard packages nor
  this package defines an arrow name containing |>| or |<|. These are
  left for the user to defined as he or she sees fit.
\end{package}

\subsubsection{Arrow Tips with Differing Names for the Left and Right Ends}

\begin{tabular}{ll}
  \sarrow{[}{]} \\
  \sarrow{]}{[} \\
  \sarrow{(}{)} \\
  \sarrow{)}{(}
\end{tabular}

\subsubsection{Variants of Other Arrow Tips}

The same name is used for both the start and end arrows. Thus, for
example, to  install the first of the following arrow tips for both
the start and the end, you would say |\pgfsetarrows{latex'-latex'}|: 

\smallskip
\begin{tabular}{ll}
  \symarrow{latex'} \\
  \symarrow{latex' reversed}  \\
  \symarrow{stealth'} \\
  \symarrow{stealth' reversed}
\end{tabular}

\subsubsection{General Purpose Arrow Tips}

\begin{tabular}{ll}
  \symarrow{o} \\
  \symarrow{*} \\
  \symarrow{diamond} \\
  \symarrow{open diamond}   \\
  \symarrow{angle 90} \\
  \symarrow{angle 90 reversed}   \\
  \symarrow{triangle 90} \\
  \symarrow{triangle 90 reversed}   \\
  \symarrow{open triangle 90} \\
  \symarrow{open triangle 90 reversed}   \\
  \symarrow{left to} \\
  \symarrow{left to reversed} \\
  \symarrow{right to} \\
  \symarrow{right to reversed} \\
  \symarrow{hooks} \\
  \symarrow{hooks reversed} \\
  \symarrow{left hook} \\
  \symarrow{left hook reversed} \\
  \symarrow{right hook} \\
  \symarrow{right hook reversed}
\end{tabular}

\subsubsection{Line Caps}

\begin{tabular}{ll}
  \carrow{round cap} \\
  \carrow{butt cap} \\
  \carrow{triangle 90 cap} \\
  \carrow{triangle 90 cap reversed} \\
  \carrow{fast cap} \\
  \carrow{fast cap reversed} \\
\end{tabular}



\subsection{Plot Handler Library}
\label{section-library-plothandlers}

\begin{package}{pgflibraryplothandlers}
  This library packages defines additional plot handlers, see
  Section~\ref{section-plot-handlers} for an introduction to plot
  handlers. The additional handlers are described in the following. 
\end{package}


\subsubsection{Curve Plot Handlers}
  
\begin{command}{\pgfplothandlercurveto}
  This handler will issue a |\pgfpathcurveto| command for each point of
  the plot, \emph{except} possibly for the first. As for the line-to
  handler, what happens with the first point can be specified using
  |\pgfsetmovetofirstplotpoint| or |\pgfsetlinetofirstplotpoint|.

  Obviously, the |\pgfpathcurveto| command needs, in addition to the
  points on the path, some control points. These are generated
  automatically using a somewhat ``dumb'' algorithm: Suppose you have
  three points $x$, $y$, and $z$ on the curve such that $y$ is between
  $x$ and $z$:
\begin{codeexample}[]
\begin{tikzpicture}    
  \draw[gray] (0,0) node {x} (1,1) node {y} (2,.5) node {z};
  \pgfplothandlercurveto
  \pgfplotstreamstart
  \pgfplotstreampoint{\pgfpoint{0cm}{0cm}}
  \pgfplotstreampoint{\pgfpoint{1cm}{1cm}}
  \pgfplotstreampoint{\pgfpoint{2cm}{.5cm}}
  \pgfplotstreamend
  \pgfusepath{stroke}
\end{tikzpicture}
\end{codeexample}

  In order to determine the control points of the curve at the point
  $y$, the handler computes the vector $z-x$ and scales it by the
  tension factor (see below). Let us call the resulting vector
  $s$. Then $y+s$ and $y-s$ will be the control points around $y$. The
  first control point at the beginning of the curve will be the
  beginning itself, once more; likewise the last control point is the
  end itself.
\end{command}

\begin{command}{\pgfsetplottension\marg{value}}
  Sets the factor used by the curve plot handlers to determine the
  distance of the control points from the points they control. The
  default is $0.15$. The higher the curvature of the curve points, the
  higher this value should be.

\begin{codeexample}[]
\begin{tikzpicture}    
  \draw[gray] (0,0) node {x} (1,1) node {y} (2,.5) node {z};
  \pgfsetplottension{0.3}
  \pgfplothandlercurveto
  \pgfplotstreamstart
  \pgfplotstreampoint{\pgfpoint{0cm}{0cm}}
  \pgfplotstreampoint{\pgfpoint{1cm}{1cm}}
  \pgfplotstreampoint{\pgfpoint{2cm}{0.5cm}}
  \pgfplotstreamend
  \pgfusepath{stroke}
\end{tikzpicture}
\end{codeexample}
\end{command}


\begin{command}{\pgfplothandlerclosedcurve}
  This handler works like the curve-to plot handler, only it will
  add a new part to the current path that is a closed curve through
  the plot points.
\begin{codeexample}[]
\begin{tikzpicture}    
  \draw[gray] (0,0) node {x} (1,1) node {y} (2,.5) node {z};
  \pgfplothandlerclosedcurve
  \pgfplotstreamstart
  \pgfplotstreampoint{\pgfpoint{0cm}{0cm}}
  \pgfplotstreampoint{\pgfpoint{1cm}{1cm}}
  \pgfplotstreampoint{\pgfpoint{2cm}{0.5cm}}
  \pgfplotstreamend
  \pgfusepath{stroke}
\end{tikzpicture}
\end{codeexample}
\end{command}


\subsubsection{Comb Plot Handlers}

There are three ``comb'' plot handlers. There name stems from the fact
that the plots they produce look like ``combs'' (more or less).

\begin{command}{\pgfplothandlerxcomb}
  This handler converts each point in the plot stream into a line from
  the $y$-axis to the point's coordinate, resulting in a ``vertical
  comb.''

  This handler is useful for creating ``bar diagrams.''
  
\begin{codeexample}[]
\begin{tikzpicture}    
  \draw[gray] (0,0) node {x} (1,1) node {y} (2,.5) node {z};
  \pgfplothandlerxcomb
  \pgfplotstreamstart
  \pgfplotstreampoint{\pgfpoint{0cm}{0cm}}
  \pgfplotstreampoint{\pgfpoint{1cm}{1cm}}
  \pgfplotstreampoint{\pgfpoint{2cm}{0.5cm}}
  \pgfplotstreamend
  \pgfusepath{stroke}
\end{tikzpicture}
\end{codeexample}
\end{command}


\begin{command}{\pgfplothandlerycomb}
  This handler converts each point in the plot stream into a line from
  the $x$-axis to the point's coordinate, resulting in a ``vertical
  comb.''
  
\begin{codeexample}[]
\begin{tikzpicture}    
  \draw[gray] (0,0) node {x} (1,1) node {y} (2,.5) node {z};
  \pgfplothandlerycomb
  \pgfplotstreamstart
  \pgfplotstreampoint{\pgfpoint{0cm}{0cm}}
  \pgfplotstreampoint{\pgfpoint{1cm}{1cm}}
  \pgfplotstreampoint{\pgfpoint{2cm}{0.5cm}}
  \pgfplotstreamend
  \pgfusepath{stroke}
\end{tikzpicture}
\end{codeexample}
\end{command}

\begin{command}{\pgfplothandlerpolarcomb}
  This handler converts each point in the plot stream into a line from
  the origin to the point's coordinate.
  
\begin{codeexample}[]
\begin{tikzpicture}    
  \draw[gray] (0,0) node {x} (1,1) node {y} (2,.5) node {z};
  \pgfplothandlerpolarcomb
  \pgfplotstreamstart
  \pgfplotstreampoint{\pgfpoint{0cm}{0cm}}
  \pgfplotstreampoint{\pgfpoint{1cm}{1cm}}
  \pgfplotstreampoint{\pgfpoint{2cm}{0.5cm}}
  \pgfplotstreamend
  \pgfusepath{stroke}
\end{tikzpicture}
\end{codeexample}
\end{command}

\subsubsection{Mark Plot Handler}

\label{section-plot-marks}

\begin{command}{\pgfplothandlermark\marg{mark code}}
  This command will execute the \meta{mark code} for each point of the
  plot, but each time the coordinate transformation matrix will be
  setup such that the origin is at the position of the point to be
  plotted. This way, if the \meta{mark code} draws a little circle
  around the origin, little circles will be drawn at each point of the
  plot.
  
\begin{codeexample}[]
\begin{tikzpicture}    
  \draw[gray] (0,0) node {x} (1,1) node {y} (2,.5) node {z};
  \pgfplothandlermark{\pgfpathcircle{\pgfpointorigin}{4pt}\pgfusepath{stroke}}
  \pgfplotstreamstart
  \pgfplotstreampoint{\pgfpoint{0cm}{0cm}}
  \pgfplotstreampoint{\pgfpoint{1cm}{1cm}}
  \pgfplotstreampoint{\pgfpoint{2cm}{0.5cm}}
  \pgfplotstreamend
  \pgfusepath{stroke}
\end{tikzpicture}
\end{codeexample}

  Typically, the \meta{code} will be |\pgfuseplotmark{|\meta{plot mark
      name}|}|, where \meta{plot mark name} is the name of a
  predefined plot mark.
\end{command}

\begin{command}{\pgfuseplotmark\marg{plot mark name}}
  Draws the given \meta{plot mark name} at the origin. The \meta{plot
    mark name} must have been previously declared using
  |\pgfdeclareplotmark|. 

\begin{codeexample}[]
\begin{tikzpicture}    
  \draw[gray] (0,0) node {x} (1,1) node {y} (2,.5) node {z};
  \pgfplothandlermark{\pgfuseplotmark{pentagon}}
  \pgfplotstreamstart
  \pgfplotstreampoint{\pgfpoint{0cm}{0cm}}
  \pgfplotstreampoint{\pgfpoint{1cm}{1cm}}
  \pgfplotstreampoint{\pgfpoint{2cm}{0.5cm}}
  \pgfplotstreamend
  \pgfusepath{stroke}
\end{tikzpicture}
\end{codeexample}
\end{command}

\begin{command}{\pgfdeclareplotmark\marg{plot mark name}\marg{code}}
  Declares a plot mark for later used with the |\pgfuseplotmark|
  command.

\begin{codeexample}[]
\pgfdeclareplotmark{my plot mark}
  {\pgfpathcircle{\pgfpoint{0cm}{1ex}}{1ex}\pgfusepathqstroke}  
\begin{tikzpicture}    
  \draw[gray] (0,0) node {x} (1,1) node {y} (2,.5) node {z};
  \pgfplothandlermark{\pgfuseplotmark{my plot mark}}
  \pgfplotstreamstart
  \pgfplotstreampoint{\pgfpoint{0cm}{0cm}}
  \pgfplotstreampoint{\pgfpoint{1cm}{1cm}}
  \pgfplotstreampoint{\pgfpoint{2cm}{0.5cm}}
  \pgfplotstreamend
  \pgfusepath{stroke}
\end{tikzpicture}
\end{codeexample}
\end{command}


\begin{command}{\pgfsetplotmarksize\marg{dimension}}
  This command sets the \TeX\ dimension |\pgfplotmarksize| to
  \meta{dimension}. This dimension is a ``recommendation'' for plot
  mark code at which size the plot mark should be drawn; plot mark
  code may choose to ignore this \meta{dimension} altogether. For
  circles, \meta{dimension} should  be the radius, for other shapes it
  should be about half the width/height.

  The predefined plot marks all take this dimension into account.

\begin{codeexample}[]
\begin{tikzpicture}    
  \draw[gray] (0,0) node {x} (1,1) node {y} (2,.5) node {z};
  \pgfsetplotmarksize{1ex}
  \pgfplothandlermark{\pgfuseplotmark{*}}
  \pgfplotstreamstart
  \pgfplotstreampoint{\pgfpoint{0cm}{0cm}}
  \pgfplotstreampoint{\pgfpoint{1cm}{1cm}}
  \pgfplotstreampoint{\pgfpoint{2cm}{0.5cm}}
  \pgfplotstreamend
  \pgfusepath{stroke}
\end{tikzpicture}
\end{codeexample}
\end{command}

\begin{textoken}{\pgfplotmarksize}
  A \TeX\ dimension that is a ``recommendation'' for the size of plot
  marks.
\end{textoken}

The following plot marks are predefined (the filling color has been
set to yellow):

\medskip
\begin{tabular}{lc}
  \plotmarkentry{*}
  \plotmarkentry{x}
  \plotmarkentry{+}
\end{tabular}


\subsection{Plot Mark Library}

\begin{package}{pgflibraryplotmarks}
  When this package is loaded, the following plot marks are defined in
  addition to |*|, |x|, and |+| (the filling color has been set to yellow):

  \catcode`\|=12
  \medskip
  \begin{tabular}{lc}
    \plotmarkentry{-}
    \index{*vbar@\protect\texttt{\protect\myvbar} plot mark}%
    \index{Plot marks!*vbar@\protect\texttt{\protect\myvbar}}
    \texttt{\char`\\pgfuseplotmark\char`\{\declare{|}\char`\}} &
    \tikz\draw[color=black!25] plot[mark=|,fill=yellow,draw=black]
    coordinates {(0,0) (.5,0.2) (1,0) (1.5,0.2)};\\
    \plotmarkentry{o}
    \plotmarkentry{asterisk}
    \plotmarkentry{star}
    \plotmarkentry{oplus}
    \plotmarkentry{oplus*}
    \plotmarkentry{otimes}
    \plotmarkentry{otimes*}
    \plotmarkentry{square}
    \plotmarkentry{square*}
    \plotmarkentry{triangle}
    \plotmarkentry{triangle*}
    \plotmarkentry{diamond}
    \plotmarkentry{diamond*}
    \plotmarkentry{pentagon}
    \plotmarkentry{pentagon*}
  \end{tabular}
\end{package}

\subsection{Shape Library}

\begin{shape}{ellipse}
  This shape is an ellipse tightly fitting the text box, if not inner
  separation is given. The following figure shows the anchors this
  shape defines; the anchors |10| and |130| are example of border anchors.
\begin{codeexample}[]
\Huge
\begin{tikzpicture}
  \node[name=s,shape=ellipse,style=shape example] {Ellipse\vrule width 1pt height 2cm};
  \foreach \anchor/\placement in
    {north west/above left, north/above, north east/above right, 
     west/left, center/above, east/right, 
     mid west/right, mid/above, mid east/left, 
     base west/left, base/below, base east/right, 
     south west/below left, south/below, south east/below right, 
     text/left, 10/right, 130/above}
     \draw[shift=(s.\anchor)] plot[mark=x] coordinates{(0,0)}
       node[\placement] {\scriptsize\texttt{(s.\anchor)}};
\end{tikzpicture}
\end{codeexample}
\end{shape}

%%% Local Variables: 
%%% mode: latex
%%% TeX-master: "pgfmanual"
%%% End: 

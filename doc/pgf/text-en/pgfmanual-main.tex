% Copyright 2003 by Till Tantau <tantau@cs.tu-berlin.de>.
%
% This program can be redistributed and/or modified under the terms
% of the LaTeX Project Public License Distributed from CTAN
% archives in directory macros/latex/base/lppl.txt.

% pgf version is defined in \pgfversion in file
% generic/pgf/utilities/pgfrcs.code.tex 

\def\xcolorversion{2.00}
\def\xkeyvalversion{1.8}

\usepackage[version=0.96]{pgf}
\usepackage{tikz}
\usepackage{pgflibraryarrows}
\usepackage{pgflibraryshapes}
\usepackage{pgflibraryplotmarks}
\usepackage{pgflibrarytikzbackgrounds}
\usepackage{pgflibrarytikztrees}
\usepackage[left=2.25cm,right=2.25cm,top=2.5cm,bottom=2.5cm,nohead]{geometry}
\usepackage{amsmath,amssymb}
\usepackage{xxcolor}
\usepackage{pifont}
\usepackage{makeidx}
\usepackage[latin1]{inputenc}
\usepackage{amsmath}

% $Header: /cvsroot/latex-beamer/pgf/doc/pgf/macros/pgfmanual-macros.tex,v 1.2 2005/06/20 07:48:39 tantau Exp $

% Copyright 2003, 2004 by Till Tantau <tantau@users.sourceforge.net>.
%
% This program can be redistributed and/or modified under the terms
% of the GNU Public License, version 2.




\def\Class#1{\hbox{\small#1}}
\def\bs{$\backslash$}

\def\Environment#1{\par\bigskip\noindent\textbf{Environment \texttt{#1}}\par}
\def\Command#1{\par\bigskip\noindent\textbf{Command \texttt{#1}}\par}
\long\def\Parameters#1{\medskip\noindent Parameters:
  \begin{enumerate}\itemsep=0pt\parskip=0pt
    #1
  \end{enumerate}}
\long\def\Description#1{\unskip\medskip\noindent Description: #1}
\def\Example{\par\medskip\noindent Example: }


%\renewcommand*\descriptionlabel[1]{\hspace\labelsep\normalfont #1}


\def\pgflayout#1{\list{}{\leftmargin=2em\itemindent-\leftmargin\def\makelabel##1{\hss##1}}%
\item\strut\texttt{\string\pgfpagelayout\char`\{\declare{#1}\char`\}}\oarg{options}\par\topsep=0pt}
\def\endpgflayout{\endlist}
  


\newcommand\opt[1]{{\color{black!50!green}#1}}
\newcommand\ooarg[1]{{\ttfamily[}\meta{#1}{\ttfamily]}}

\def\opt{\afterassignment\pgfmanualopt\let\next=}
\def\pgfmanualopt{\ifx\next\bgroup\bgroup\color{black!50!green}\else{\color{black!50!green}\next}\fi}



\def\beamer{\textsc{beamer}}
\def\pdf{\textsc{pdf}}
\def\pgfname{\textsc{pgf}}
\def\tikzname{Ti\emph{k}Z}
\def\pstricks{\textsc{pstricks}}
\def\prosper{\textsc{prosper}}
\def\seminar{\textsc{seminar}}
\def\texpower{\textsc{texpower}}
\def\foils{\textsc{foils}}

{
  \makeatletter
  \global\let\myempty=\@empty
  \global\let\mygobble=\@gobble
  \catcode`\@=12
  \gdef\getridofats#1@#2\relax{%
    \def\getridtest{#2}%
    \ifx\getridtest\myempty%
      \expandafter\def\expandafter\strippedat\expandafter{\strippedat#1}
    \else%
      \expandafter\def\expandafter\strippedat\expandafter{\strippedat#1\protect\printanat}
      \getridofats#2\relax%
    \fi%
  }

  \gdef\removeats#1{%
    \let\strippedat\myempty%
    \edef\strippedtext{\stripcommand#1}%
    \expandafter\getridofats\strippedtext @\relax%
  }
  
  \gdef\stripcommand#1{\expandafter\mygobble\string#1}
}

\def\printanat{\char`\@}

\def\declare{\afterassignment\pgfmanualdeclare\let\next=}
\def\pgfmanualdeclare{\ifx\next\bgroup\bgroup\color{red!75!black}\else{\color{red!75!black}\next}\fi}

\def\command#1{\list{}{\leftmargin=2em\itemindent-\leftmargin\def\makelabel##1{\hss##1}}%
\item\extractcommand#1\@@\par\topsep=0pt}
\def\endcommand{\endlist}
\def\extractcommand#1#2\@@{\strut\declare{\texttt{\string#1}}#2%
  \removeats{#1}%
  \index{\strippedat @\protect\myprintocmmand{\strippedat}}}

\let\textoken=\command
\let\endtextoken=\endcommand

\def\myprintocmmand#1{\texttt{\char`\\#1}}

\def\example{\par\smallskip\noindent\textit{Example: }}
\def\themeauthor{\par\smallskip\noindent\textit{Theme author: }}

\def\environment#1{\list{}{\leftmargin=2em\itemindent-\leftmargin\def\makelabel##1{\hss##1}}%
\extractenvironement#1\@@\par\topsep=0pt}
\def\endenvironment{\endlist}
\def\extractenvironement#1#2\@@{%
\item{{\ttfamily\char`\\begin\char`\{\declare{#1}\char`\}}#2}%
  {\itemsep=0pt\parskip=0pt\item{\meta{environment contents}}%
  \item{\ttfamily\char`\\end\char`\{\declare{#1}\char`\}}}%
  \index{#1@\protect\texttt{#1} environment}%
  \index{Environments!#1@\protect\texttt{#1}}}

\def\plainenvironment#1{\list{}{\leftmargin=2em\itemindent-\leftmargin\def\makelabel##1{\hss##1}}%
\extractplainenvironment#1\@@\par\topsep=0pt}
\def\endplainenvironment{\endlist}
\def\extractplainenvironment#1#2\@@{%
\item{{\ttfamily\declare{\char`\\#1}}#2}%
  {\itemsep=0pt\parskip=0pt\item{\meta{environment contents}}%
  \item{\ttfamily\declare{\char`\\end#1}}}%
  \index{#1@\protect\texttt{#1} environment}%
  \index{Environments!#1@\protect\texttt{#1}}}

\def\shape#1{\list{}{\leftmargin=2em\itemindent-\leftmargin\def\makelabel##1{\hss##1}}%
\extractshape#1\@@\par\topsep=0pt}
\def\endshape{\endlist}
\def\extractshape#1\@@{%
\item{Shape {\ttfamily\declare{#1}}}%
  \index{#1@\protect\texttt{#1} shape}%
  \index{Shapes!#1@\protect\texttt{#1}}}

\def\package#1{\list{}{\leftmargin=2em\itemindent-\leftmargin\def\makelabel##1{\hss##1}}%
\item{{\ttfamily\char`\\usepackage\char`\{\declare{#1}\char`\}\space\space \char`\%\space\space  LaTeX}}
  \index{#1@\protect\texttt{#1} package}%
  \index{Packages and Files!#1@\protect\texttt{#1}}%
  \par\topsep=0pt\itemsep=0pt
\item{{\ttfamily\char`\\input \declare{#1}.tex\space\space\space \char`\%\space\space  plain TeX}}
  \par\topsep=0pt\itemsep=0pt
\item{{\ttfamily\char`\\input \declare{#1}.tex\space\space\space \char`\%\space\space  ConTeX}}
  \par\topsep=0pt
}
\def\endpackage{\endlist}

\def\filedescription#1{\list{}{\leftmargin=2em\itemindent-\leftmargin\def\makelabel##1{\hss##1}}%
\item{File {\ttfamily\declare{#1}}}
  \index{#1@\protect\texttt{#1} file}%
  \index{Packages and Files!#1@\protect\texttt{#1}}%
  \par\topsep=0pt
}
\def\endfiledescription{\endlist}


\def\packageoption#1{\list{}{\leftmargin=2em\itemindent-\leftmargin\def\makelabel##1{\hss##1}}%
\item{{\ttfamily\char`\\usepackage[\declare{#1}]\char`\{pgf\char`\}}}
  \index{#1@\protect\texttt{#1} package option}%
  \index{Package options for \textsc{pgf}!#1@\protect\texttt{#1}}%
  \par\topsep=0pt}
\def\endpackageoption{\endlist}

\def\itemoption#1{\item \declare{\texttt{#1}}%
  \index{#1@\protect\texttt{#1} option}%
  \index{Graphic options!#1@\protect\texttt{#1}}%
}
\def\itemstyle#1{\item \texttt{style=}\declare{\texttt{#1}}%
  \index{#1@\protect\texttt{#1} style}%
  \index{Styles!#1@\protect\texttt{#1}}%
}



\def\class#1{\list{}{\leftmargin=2em\itemindent-\leftmargin\def\makelabel##1{\hss##1}}%
\extractclass#1@\par\topsep=0pt}
\def\endclass{\endlist}
\def\extractclass#1#2@{%
\item{{{\ttfamily\char`\\documentclass}#2{\ttfamily\char`\{\declare{#1}\char`\}}}}%
  \index{#1@\protect\texttt{#1} class}%
  \index{Classes!#1@\protect\texttt{#1}}}

\def\partname{Part}

\makeatletter
\def\index@prologue{\section*{Index}\addcontentsline{toc}{section}{Index}
  This index only contains automatically generated entries. A good
  index should also contain carefully selected keywords. This index is
  not a good index.
  \bigskip
}
\c@IndexColumns=2
  \def\theindex{\@restonecoltrue
    \columnseprule \z@  \columnsep 35\p@
    \twocolumn[\index@prologue]%
       \parindent -30pt
       \columnsep 15pt
       \parskip 0pt plus 1pt
       \leftskip 30pt
       \rightskip 0pt plus 2cm
       \small
       \def\@idxitem{\par}%
    \let\item\@idxitem \ignorespaces}
  \def\endtheindex{\onecolumn}
\def\noindexing{\let\index=\@gobble}


\newcommand\symarrow[1]{
  \index{#1@\protect\texttt{#1} arrow tip}%
  \index{Arrow tips!#1@\protect\texttt{#1}}
  \texttt{#1}& yields thick  
  \begin{tikzpicture}[arrows={#1-#1},thick]
    \useasboundingbox (0pt,-0.5ex) rectangle (1cm,2ex);
    \draw (0,0) -- (1,0);
  \end{tikzpicture} and thin
  \begin{tikzpicture}[arrows={#1-#1},thin]
    \useasboundingbox (0pt,-0.5ex) rectangle (1cm,2ex);
    \draw (0,0) -- (1,0);
  \end{tikzpicture}
}

\newcommand\sarrow[2]{
  \index{#1@\protect\texttt{#1} arrow tip}%
  \index{Arrow tips!#1@\protect\texttt{#1}}
  \index{#2@\protect\texttt{#2} arrow tip}%
  \index{Arrow tips!#2@\protect\texttt{#2}}
  \texttt{#1-#2}& yields thick  
  \begin{tikzpicture}[arrows={#1-#2},thick]
    \useasboundingbox (0pt,-0.5ex) rectangle (1cm,2ex);
    \draw (0,0) -- (1,0);
  \end{tikzpicture} and thin
  \begin{tikzpicture}[arrows={#1-#2},thin]
    \useasboundingbox (0pt,-0.5ex) rectangle (1cm,2ex);
    \draw (0,0) -- (1,0);
  \end{tikzpicture}
}

\newcommand\carrow[1]{
  \index{#1@\protect\texttt{#1} arrow tip}%
  \index{Arrow tips!#1@\protect\texttt{#1}}
  \texttt{#1}& yields for line width 1ex
  \begin{tikzpicture}[arrows={#1-#1},line width=1ex]
    \useasboundingbox (0pt,-0.5ex) rectangle (1.5cm,2ex);
    \draw (0,0) -- (1.5,0);
  \end{tikzpicture}
}
\def\myvbar{\char`\|}
\newcommand\plotmarkentry[1]{%
  \index{#1@\protect\texttt{#1} plot mark}%
  \index{Plot marks!#1@\protect\texttt{#1}}
  \texttt{\char`\\pgfuseplotmark\char`\{\declare{#1}\char`\}} &
  \tikz\draw[color=black!25] plot[mark=#1,fill=yellow,draw=black] coordinates{(0,0) (.5,0.2) (1,0) (1.5,0.2)};\\
}
\newcommand\plotmarkentrytikz[1]{%
  \index{#1@\protect\texttt{#1} plot mark}%
  \index{Plot marks!#1@\protect\texttt{#1}}
  |mark=|\texttt{\declare{#1}} & \tikz\draw[color=black!25]
  plot[mark=#1,fill=yellow,draw=black] 
    coordinates {(0,0) (.5,0.2) (1,0) (1.5,0.2)};\\
}


\colorlet{graphicbackground}{yellow!80!black!20}
\colorlet{codebackground}{blue!20}


\ifx\scantokens\@undefined
  \PackageError{pgfmanual-macros}{You need to use extended latex
    (elatex) or (pdfelatex) to process this document}{}
\fi

\begingroup
\catcode`|=0
\catcode`[= 1
\catcode`]=2
\catcode`\{=12
\catcode `\}=12
\catcode`\\=12 |gdef|find@example#1\end{codeexample}[|endofcodeexample[#1]]
|endgroup

\begingroup
\catcode`\^=7
\catcode`\^^M=13
\catcode`\ =13%
\gdef\returntospace{\catcode`\ =13\def {\space}\catcode`\^^M=13\def^^M{}}%
\endgroup

\begingroup
\catcode`\%=13
\catcode`\^^M=13
\gdef\commenthandler{\catcode`\%=13\def%{\@gobble@till@return}}
\gdef\@gobble@till@return#1^^M{}
\gdef\typesetcomment{\catcode`\%=13\def%{\@typeset@till@return}}
\gdef\@typeset@till@return#1^^M{{\def%{\char`\%}\textsl{\char`\%#1}}\par}
\endgroup

\define@key{codeexample}{width}{\setlength\codeexamplewidth{#1}}
\define@key{codeexample}{graphic}{\colorlet{graphicbackground}{#1}}
\define@key{codeexample}{code}{\colorlet{codebackground}{#1}}
\define@key{codeexample}{execute code}{\csname code@execute#1\endcsname}
\define@key{codeexample}{code only}[]{\code@executefalse}

\newdimen\codeexamplewidth
\setlength\codeexamplewidth{4cm+7pt}
\newif\ifcode@execute
\newbox\codeexamplebox
\def\codeexample[#1]{%
  \code@executetrue
  \setkeys{codeexample}{#1}%
  \parindent0pt
  \begingroup%
  \par%
  \medskip%
  \let\do\@makeother%
  \dospecials%
  \obeylines%
  \@vobeyspaces%
  \catcode`\%=13%
  \catcode`\^^M=13%
  \find@example}
\def\endofcodeexample#1{%
  \endgroup%
  \ifcode@execute%
    \setbox\codeexamplebox=\hbox{%
      {%
        {%
          \returntospace%
          \commenthandler%
          \xdef\code@temp{#1}% removes returns and comments
        }%
        \colorbox{graphicbackground}{\ignorespaces%
          \expandafter\scantokens\expandafter{\code@temp\ignorespaces}\ignorespaces}%
      }%
    }%
    \ifdim\wd\codeexamplebox>\codeexamplewidth%
      \def\code@start{\par}%
      \def\code@flushstart{}\def\code@flushend{}%
      \def\code@mid{\parskip2pt\par\noindent}%
      \def\code@width{\linewidth-6pt}%
      \def\code@end{}%
    \else%
      \def\code@start{%
        \linewidth=\textwidth%
        \parshape \@ne 0pt \linewidth
        \leavevmode%
        \hbox\bgroup}%
      \def\code@flushstart{\hfill}%
      \def\code@flushend{\hbox{}}%
      \def\code@mid{\hskip6pt}%
      \def\code@width{\linewidth-12pt-\codeexamplewidth}%
      \def\code@end{\egroup}%
    \fi%
    \code@start%
    \noindent%
    \begin{minipage}[t]{\codeexamplewidth}\raggedright
      \hrule width0pt%
      \footnotesize\vskip-1em%
      \code@flushstart\box\codeexamplebox\code@flushend%
      \vskip-1ex
      \leavevmode%
    \end{minipage}%
  \else%
    \def\code@mid{\par}
    \def\code@width{\linewidth-6pt}
    \def\code@end{}
  \fi%
  \code@mid%  
  \colorbox{codebackground}{%
    \begin{minipage}[t]{\code@width}%
      {%
        \let\do\@makeother
        \dospecials
        \frenchspacing\@vobeyspaces
        \normalfont\ttfamily\footnotesize
        \typesetcomment%
        \@tempswafalse
        \def\par{%
          \if@tempswa
          \leavevmode \null \@@par\penalty\interlinepenalty
          \else
          \@tempswatrue
          \ifhmode\@@par\penalty\interlinepenalty\fi
          \fi}%
        \obeylines
        \everypar \expandafter{\the\everypar \unpenalty}%
        #1}
    \end{minipage}}%
  \code@end%
  \par%
  \medskip
  \endgroup
}



\makeatother


%%% Local Variables: 
%%% mode: latex
%%% TeX-master: "beameruserguide"
%%% End: 


\makeindex

\makeatletter
\renewcommand*\l@subsection{\@dottedtocline{2}{1.5em}{2.8em}}
\renewcommand*\l@subsubsection{\@dottedtocline{3}{4.3em}{3.2em}}
\makeatother

%\includeonly{pgfmanual-libraries}

% Global styles:
\tikzstyle{every plot}=[prefix=plots/pgf-]
\tikzstyle{shape example}=[color=black!30,draw,fill=yellow!30,line width=.5cm,inner xsep=2.5cm,inner ysep=0.5cm]

\index{Options for graphics|see{Graphic options}}
\index{Options for packages|see{Package options}}
\index{File|see{Packages and files}}
\index{Layout|see{Page layout}}

\begin{document}

{
  \parindent0pt
\vbox{}
\vskip 3.5cm
\Huge
\tikzname\ and \pgfname

\Large
Manual for Version \pgfversion

\vskip 3cm 

\begin{codeexample}[graphic=white]
\tikz[rotate=30]
  \foreach \x / \xcolor in {0/blue,1/cyan,2/green,3/yellow,4/red}
    \foreach \y / \ycolor in {0/blue,1/cyan,2/green,3/yellow,4/red}
      \shade[ball color=\xcolor!50!\ycolor] (\x,\y) circle (7.5mm);
\end{codeexample}
\vskip 0cm plus 1.5fill
\vbox{}         
\clearpage
}

{
  \vbox{}
  \vskip0pt plus 1fill
  F�r meinen Vater, damit er noch viele sch�ne \TeX-Graphiken erschaffen kann.
  \vskip0pt plus 3fill
  \vbox{}
  \clearpage
}


\title{The \tikzname\ and \pgfname\ Packages\\
  Manual for Version \pgfversion\\[1mm]
\large\href{http://sourceforge.net/projects/pgf}{\texttt{http://sourceforge.net/projects/pgf}}}
\author{Till Tantau\\
  \href{mailto:tantau@users.sourceforge.net}{\texttt{tantau@users.sourceforge.net}}}

\maketitle

\tableofcontents

\clearpage

\part{Getting Started}

This part is intended to help you get started with the \pgfname\
package. First, the installation process is explained; however, the
system will typically be already installed on your system, so this can
often be skipped. Next, a short tutorial is given that explains the
most often used commands and concepts of \tikzname, without going into
any of the glorious details. At the end of this section you will find
some, hopefully useful, hints on how to create ``good'' graphics in
general. The information in this section is not specific to
\pgfname. 

\vskip3cm

\begin{codeexample}[graphic=white,width=0pt]
\tikz \draw[thick,rounded corners=8pt]
  (0,0) -- (0,2) -- (1,3.25) -- (2,2) -- (2,0) -- (0,2) -- (2,2) -- (0,0) -- (2,0);
\end{codeexample}

% Copyright 2003 by Till Tantau <tantau@cs.tu-berlin.de>.
%
% This program can be redistributed and/or modified under the terms
% of the LaTeX Project Public License Distributed from CTAN
% archives in directory macros/latex/base/lppl.txt.



\section{Introduction}

The \pgfname\ package, where ``\pgfname'' is supposed to mean ``portable
graphics format'' (or ``pretty, good, functional'' if you
prefer\dots), is a package for creating graphics in an ``inline''  
manner. The package defines a number of \TeX\ commands that draw
graphics. For example, the code |\tikz \draw (0pt,0pt) -- (20pt,6pt);|
yields the line \tikz \draw (0pt,0pt) -- (20pt,6pt); and the code
|\tikz \fill[orange] (1ex,1ex) circle (1ex);| yields \tikz
\fill[orange] (1ex,1ex) circle (1ex);.

In a sense, when using \pgfname\ you ``program'' your graphics, just as you
``program'' your document when using \TeX. This means that you get 
the advantages of the ``\TeX-approach to typesetting'' also for your 
graphics: quick creation of simple graphics, precise positioning, the
use of macros, often superior typography. You also inherit all the
disadvantages: steep learning curve, no \textsc{wysiwyg}, small
changes require a long recompilation time, and the code does not
really ``show'' how things will look like. 



\subsection{Structure of the System}

The \pgfname\ system consists of different layers:

\begin{description}
\item[System layer:] This layer provides a complete abstraction of what is
  going on ``in the driver.'' The driver is a program like |dvips| or
  |dvipdfm| that takes a |.dvi| file as input and generates a |.ps| or
  a |.pdf| file. (The |pdftex| program also counts as a driver, even
  though it does not take a |.dvi| file as input. Never mind.) Each
  driver has its own syntax for the generation of graphics, causing
  headaches to everyone who wants to create graphics in a portable
  way. \pgfname's system layer ``abstracts away'' these
  differences. For example, the system command
  |\pgfsys@lineto{10pt}{10pt}| extends the current path  to the coordinate
  $(10\mathrm{pt},10\mathrm{pt})$ of the current
  |{pgfpicture}|. Depending on whether |dvips|, 
  |dvipdfm|, or |pdftex| is used to process the document, the system
  command will be converted to different |\special| commands.

  The system layer is as ``minimalistic'' as possible since each
  additional command makes it more work to port \pgfname\ to a new
  driver. Currently, only drivers that produce PostScript or
  \textsc{pdf} output are supported and only few of these (hence the
  name \emph{portable} graphics format is currently a bit
  boastful). However, in principle, the system layer could be ported
  to many different drivers quite easily. It should even be possible
  to produce, say, \textsc{svg} output in conjunction with
  \textsc{tex4ht}.

  As a user, you will not use the system layer directly.
\item[Basic layer:]
  The basic layer provides a set of basic commands that allow
  you to produce complex graphics in a much easier way than by using
  the system layer directly. For example,  the system layer provides
  no commands for creating circles since circles can be composed from
  the more basic B�zier curves (well, almost). However, as a user you
  will want to have a simple command to create circles
  (at least I do) instead of having to write down half a page of
  B�zier  curve  support coordinates. Thus, the basic layer provides a
  command |\pgfpathcircle| that generates the necessary curve
  coordinates for you.

  The basic layer is consists of a \emph{core}, which consists of
  several interdependent packages that can only be loaded \emph{en
    bloc,} and additional packages that extend the core by more
  special-purpose commands like node management or a plotting
  interface. For instance, the \textsc{beamer} package uses the core, 
  but not all of the additional packages of the basic layer.
\item[Frontend layer:]
  A frontend (of which there can be several) is a set of commands
  or a special syntax that makes using the basic layer easier. A
  problem with directly using the basic layer is that code written for
  this layer is often too ``verbose.'' For example, to draw a simple
  triangle, you may need as many as five commands when using the basic
  layer: One for beginning a path at the first corner of the triangle,
  one for extending the path to the second corner, one for going to
  the third, one for closing the path, and one for actually painting
  the triangle (as opposed to filling it). With the |tikz| frontend
  all this boils down to a single simple \textsc{metafont}-like
  command: 
\begin{verbatim}
\draw (0,0) -- (1,0) -- (1,1) -- cycle;
\end{verbatim}

  There are different frontends:
  \begin{itemize}
  \item
    The \tikzname\ frontend is the ``natural'' frontend for \pgfname. It gives
    you access to all features of \pgfname, but it is intended to be
    easy to use. The syntax is a mixture of \textsc{metafont} and
    \textsc{pstricks} and some ideas of myself. This frontend is
    \emph{neither} a complete \textsc{metafont} compatibility layer nor
    a \textsc{pstricks} compatibility layer and it is not intended to
    become either. 
  \item
    The |pgfpict2e| frontend reimplements the standard \LaTeX\
    |{picture}|  environment and commands like |\line| or |\vector|
    using the \pgfname\ basic layer. This layer is not really ``necessary''
    since the |pict2e.sty| package does at least as good a job at
    reimplementing the |{picture}| environment. Rather, the idea
    behind this package is to have a simple demonstration of how a
    frontend can be implemented.
  \end{itemize}

  It would be possible to implement a |pgftricks| frontend that maps
  \textsc{pstricks} commands to \pgfname\ commands. However, I have not
  done this and even if fully implemented, many things that work in
  \pstricks\ will not work, namely whenever some \pstricks\ command
  relies too heavily on PostScript trickery. Nevertheless, such a
  package might be useful in some situations.
\end{description}

As a user of \pgfname\ you will use the commands of a
frontend plus perhaps some commands of the basic layer. For this
reason, this manual explains the frontends first, then the basic
layer, and finally the system layer.



\subsection{Comparison with Other Graphics Packages}

There were two main motivations for creating \pgfname:
\begin{enumerate}
\item
  The standard \LaTeX\ |{picture}| environment is not powerful enough to
  create anything but really simple graphics. This is certainly not
  due to a lack of knowledge or imagination on the part of
  \LaTeX's designer(s). Rather, this is the price paid for the
  |{picture}| environment's portability: It works together with all
  backend drivers.
\item
  The |{pstricks}| package is certainly powerful enough to create
  any conceivable kind of graphic, but it is not portable at all. Most
  importantly, it does not work with |pdftex| nor with any other
  driver that produces anything but PostScript code.
\end{enumerate}

The \pgfname\ package is a trade-off between portability and expressive
power. It is not as portable as |{picture}| and not as powerful as
|{pspicture}|. However, it is more powerful than |{picture}| and
more portable than |{pspicture}|.



\subsection{Supported \TeX-Formats}

\pgfname\ can be used with any \TeX-format that is based on Donald
Knuth's original |plain| format. This includes \LaTeX\  and
Con\TeX t. If you use any format other than \LaTeX, you must  you must
say |\input tikz.tex| and |\input pgf.tex| instead of
|\usepackage{tikz}| or |\usepackage{pgf}| and you must say
|\pgfpicture| instead of |\begin{pgfpicture}| and |\endpicture|
  instead of |\end{pgfpicture}|. 

\pgfname\ was originally written for use with \LaTeX\ and this shows
in a number of places. Nevertheless, the plain \TeX\ support is
reasonably good.


\subsection{Utilities: Page Management}

The \pgfname\ package include a special subpackage called |pgfpages|,
which is used to assemble several pages into a single page. This
package is not really about creating graphics, but it is part of \pgfname\
nevertheless, mostly because its implementation uses \pgfname\ heavily.

The subpackage |pgfpages| provides commands for assembling several
``virtual pages'' into a single ``physical page.'' The idea is that
whenever \TeX\ has a page ready for ``shipout,'' |pgfpages| interrupts
this shipout and instead stores the page to be shipped out in a
special box. When enough ``virtual pages'' have been accumulated in
this way, they are scaled down and arranged on a ``physical page,''
which then \emph{really} shipped out. This mechanism allows you to
create ``two page on one page'' versions of a document directly inside
\LaTeX\ without the use of any external programs.

However, |pgfpages| can do quite a lot more than that. You can use it
to put logos and watermark on pages, print up to 16 pages on one page,
add borders to pages, and more.




\subsection{How to Read This Manual}

This manual describes both the design of the \pgfname\ system and
its usage. The organization is  very roughly according to
``user-friendliness.'' The commands and subpackages that are easiest
and most frequently used are described first, more low-level and
esoteric features are discussed later.

If you have not yet installed \pgfname, please read the installation
first. Second, it might be a good idea to read the tutorial. Finally,
you might wish to skim through the description of \tikzname. Typically, 
you will not need to read the sections on the basic layer. You will
only need to read the part on the system layer if you intend to write
your own frontend or if you wish to port \pgfname\ to a new driver.

The ``public'' commands and environments provided by the |pgf| package
are described throughout the text. In each such description, the
described command, environment or option is printed in red. Text shown
in green is optional and can be left out.



\subsection{Getting Help}

When you need help with \pgfname\ and \tikzname, please do the
following:

\begin{enumerate}
\item
  Read the manual, at least the part that has to do with your problem.
\item
  If that does not solve the problem, try having a look at the
  sourceforge development page for \pgfname\ and \tikzname\ (see the
  title of this document). Perhaps someone has already reported a
  similar problem and someone has found a solution.
\item
  On the website you will find numerous forums for getting
  help. There, you can write to help forums, file bug reports, join
  mailing lists, and so on.
\item
  Before you file a bug report, especially a bug report concerning the
  installation, make sure that this is really a bug. In particular,
  have a look at the |.log| file that results when you \TeX\ your
  files. This |.log| file should show that all the right files are
  loaded from the right directories. Nearly all installation problems
  can be resolved by looking at the |.log| file.
\item
  \emph{As a last resort} you can try to email me (the author). I do
  not mind getting emails, I simply get way too many of them. Because
  of this, I cannot guarantee that your emails will be answered timely
  or even at all. Your chances that your problem will be fixed are
  somewhat higher if you mail to the \pgfname\ mailing list
  (naturally, I read this list and answer questions when I have the
  time).
\item
  Please, do not phone me in my office. If you need a hotline, buy a
  commercial product.
\end{enumerate}


% Copyright 2003 by Till Tantau <tantau@cs.tu-berlin.de>.
%
% This program can be redistributed and/or modified under the terms
% of the LaTeX Project Public License Distributed from CTAN
% archives in directory macros/latex/base/lppl.txt.


\section{Installation}

There are different ways of installing \pgfname, depending
on your system and needs, and you may need to install other
packages as well as, see below. Before installing, you may wish to
review the \textsc{gpl} license under which the package is
distributed, see Section~\ref{section-license}. 

Typically, the package will already be installed on your
system. Naturally, in this case you do not need to worry about the
installation process at all and you can skip the rest of this
section. 


\subsection{Package and Driver Versions}

This documentation is part of version \pgfversion\ of the \pgfname\
package. In order to run \pgfname, you need a reasonably recent 
\TeX\ installation. When using \LaTeX, you need the following packages
installed (newer versions should also work):
\begin{itemize}
\item
  |xcolor| version \xcolorversion.
\item
  |xkeyval| version \xkeyvalversion, if you wish to use \tikzname.
\end{itemize}
With plain \TeX, |xcolor| is not needed, but you obviously do not
get its (full) functionality. 

Currently, \pgfname\ supports the following backend drivers:
\begin{itemize}
\item
  |pdftex| version 0.14 or higher. Earlier versions do not work.
\item
  |dvips| version 5.94a or higher. Earlier versions may also work.
\item
  |dvipdfm| version 0.13.2c or higher. Earlier versions may also work.
\item
  |tex4ht| version 2003-05-05 or higher. Earlier versions may also work.
\end{itemize}

Currently, \pgfname\ supports the following formats:
\begin{itemize}
\item
  |latex| with complete functionality.
\item
  |plain| with complete functionality, except for graphics inclusion,
  which works only for pdf\TeX.
\item
  |context| should work as |plain|, but I have not tried it.
\end{itemize}

For more details, see Section~\ref{section-formats}.



\subsection{Installing Prebundled Packages}

I do not create or manage prebundled packages of \pgfname, but,
fortunately, nice other people do. I cannot give detailed instructions
on how to install these packages, since I do not manage them, but I
\emph{can} tell you were to find them. If you have a problem with
installing, you might wish to have a look at the Debian page or the
Mik\TeX\ page first.


\subsubsection{Debian}

The command ``|aptitude install pgf|'' should do the trick. Sit back
and relax. In detail, the following packages are installed:  
\begin{verbatim}
http://packages.debian.org/pgf
http://packages.debian.org/latex-xcolor
\end{verbatim}


\subsubsection{MiKTeX}

For MiK\TeX, use the update wizard to install the (latest versions of
the) packages called |pgf|, |xcolor|, and |xkeyval|. 




\subsection{Installation in a texmf Tree}

For a permanent installation, you place the files of the
the \textsc{pgf} package in an appropriate |texmf| tree. 

When you ask \TeX\ to use a certain class or package, it usually looks
for the necessary files in so-called |texmf| trees. These trees
are simply huge directories that contain these files. By default,
\TeX\ looks for files in three different |texmf| trees:
\begin{itemize}
\item
  The root |texmf| tree, which is usually located at
  |/usr/share/texmf/| or |c:\texmf\| or somewhere similar.
\item
  The local  |texmf| tree, which is usually located at
  |/usr/local/share/texmf/| or |c:\localtexmf\| or somewhere similar.
\item
  Your personal  |texmf| tree, which is usually located in your home
  directory at |~/texmf/| or |~/Library/texmf/|.   
\end{itemize}

You should install the packages either in the local tree or in
your personal tree, depending on whether you have write access to the
local tree. Installation in the root tree can cause problems, since an
update of the whole \TeX\ installation will replace this whole tree.


\subsubsection{Installation that Keeps Everything Together}

Once you have located the right texmf tree, you must decide whether
you want to install \pgfname\ in such a way that ``all its files are
kept in one place'' or whether you want to be
``\textsc{tds}-compliant,'' where \textsc{tds} means ``\TeX\ directory
structure.''

If you want to keep ``everything in one place,'' inside the |texmf|
tree that you have chosen create a sub-sub-directory called
|texmf/tex/generic/pgf| or
|texmf/tex/generic/pgf-|\texttt{\pgfversion}, if you prefer. Then
place all files of the |pgf| package in this directory. Finally,
rebuild \TeX's filename database. This is done by running the command
|texhash| or |mktexlsr| (they are the same). In Mik\TeX, there is a
menu option to do this. 


\subsubsection{Installation that is TDS-Compliant}

While the above installation process is the most ``natural'' one and
although I would like to recommend it since it makes updating and
managing the \pgfname\ package easy, it is not
\textsc{tds}-compliant. If you want to be \textsc{tds}-compliant,
proceed as follows: (If you do not know what \textsc{tds}-compliant
means, you probably do not want to be \textsc{tds}-compliant.)

The |.tar| file of the |pgf| package contains the following files and
directories at its root: |README|, |doc|,  |generic|, |plain|, and
|latex|. You should ``merge'' each of the four directories with the
following directories |texmf/doc|, |texmf/tex/generic|,
|texmf/tex/plain|, and |texmf/tex/latex|. For example, in the |.tar|
file the |doc| directory contains just the directory |pgf|, and this
directory has to be moved to |texmf/doc/pgf|. The root |README| file
can be ignored since it is reproduced in |doc/pgf/README|.

You may also consider keeping everything in one place and using
symbolic links to point from the \textsc{tds}-compliant directories to
the central installation.

\vskip1em
For a more detailed explanation of the standard installation process
of packages, you might wish to consult
\href{http://www.ctan.org/installationadvice/}{|http://www.ctan.org/installationadvice/|}.
However, note that the \pgfname\ package does not come with a
|.ins| file (simply skip that part).


\subsection{Updating the Installation}

To update your installation from a previous version, all you need to
do is to replace everything in the directory |texmf/tex/generic/pgf|
with the files of the new version (or in all the directories where
|pgf| was installed, if you chose a \textsc{tds}-compliant
installation). The easiest way to do this is to first delete the old
version and then proceed as described above. Sometimes, there are
changes in the syntax of certain command from version to version. If
things no longer work that used to work, you may wish to have a look
at the release notes and at the change log. 


% $Header: /cvsroot/pgf/pgf/doc/pgf/en/pgfmanual-license.tex,v 1.1.1.1 2005/06/29 12:14:02 tantau Exp $

% Copyright 2003, 2004 by Till Tantau <tantau@users.sourceforge.net>.
%
% This program can be redistributed and/or modified under the terms
% of the GNU Public License, version 2.


\subsection{License: The GNU Public License, Version 2}
\label{section-license}

The \pgfname\ package is distributed under the \textsc{gnu} public
license, version 2. In detail, this means the following (the following
text is copyrighted by the Free Software Foundation):

\subsubsection{Preamble}

The licenses for most software are designed to take away your freedom to
share and change it.  By contrast, the \textsc{gnu} General Public License is
intended to guarantee your freedom to share and change free software---to
make sure the software is free for all its users.  This General Public
License applies to most of the Free Software Foundation's software and to
any other program whose authors commit to using it.  (Some other Free
Software Foundation software is covered by the \textsc{gnu} Library General Public
License instead.)  You can apply it to your programs, too.

When we speak of free software, we are referring to freedom, not price.
Our General Public Licenses are designed to make sure that you have the
freedom to distribute copies of free software (and charge for this service
if you wish), that you receive source code or can get it if you want it,
that you can change the software or use pieces of it in new free programs;
and that you know you can do these things.

To protect your rights, we need to make restrictions that forbid anyone to
deny you these rights or to ask you to surrender the rights.  These
restrictions translate to certain responsibilities for you if you
distribute copies of the software, or if you modify it.

For example, if you distribute copies of such a program, whether gratis or
for a fee, you must give the recipients all the rights that you have.  You
must make sure that they, too, receive or can get the source code.  And
you must show them these terms so they know their rights.

We protect your rights with two steps: (1) copyright the software, and (2)
offer you this license which gives you legal permission to copy,
distribute and/or modify the software.

Also, for each author's protection and ours, we want to make certain that
everyone understands that there is no warranty for this free software.  If
the software is modified by someone else and passed on, we want its
recipients to know that what they have is not the original, so that any
problems introduced by others will not reflect on the original authors'
reputations.

Finally, any free program is threatened constantly by software patents.
We wish to avoid the danger that redistributors of a free program will
individually obtain patent licenses, in effect making the program
proprietary.  To prevent this, we have made it clear that any patent must
be licensed for everyone's free use or not licensed at all.

The precise terms and conditions for copying, distribution and
modification follow.

\subsubsection{Terms and Conditions For Copying, Distribution and
  Modification}

\begin{enumerate}

\addtocounter{enumi}{-1}

\item 
This License applies to any program or other work which contains a notice
placed by the copyright holder saying it may be distributed under the
terms of this General Public License.  The ``Program'', below, refers to
any such program or work, and a ``work based on the Program'' means either
the Program or any derivative work under copyright law: that is to say, a
work containing the Program or a portion of it, either verbatim or with
modifications and/or translated into another language.  (Hereinafter,
translation is included without limitation in the term ``modification''.)
Each licensee is addressed as ``you''.

Activities other than copying, distribution and modification are not
covered by this License; they are outside its scope.  The act of
running the Program is not restricted, and the output from the Program
is covered only if its contents constitute a work based on the
Program (independent of having been made by running the Program).
Whether that is true depends on what the Program does.

\item You may copy and distribute verbatim copies of the Program's source
  code as you receive it, in any medium, provided that you conspicuously
  and appropriately publish on each copy an appropriate copyright notice
  and disclaimer of warranty; keep intact all the notices that refer to
  this License and to the absence of any warranty; and give any other
  recipients of the Program a copy of this License along with the Program.

You may charge a fee for the physical act of transferring a copy, and you
may at your option offer warranty protection in exchange for a fee.

\item
You may modify your copy or copies of the Program or any portion
of it, thus forming a work based on the Program, and copy and
distribute such modifications or work under the terms of Section 1
above, provided that you also meet all of these conditions:

\begin{enumerate}

\item 
You must cause the modified files to carry prominent notices stating that
you changed the files and the date of any change.

\item
You must cause any work that you distribute or publish, that in
whole or in part contains or is derived from the Program or any
part thereof, to be licensed as a whole at no charge to all third
parties under the terms of this License.

\item
If the modified program normally reads commands interactively
when run, you must cause it, when started running for such
interactive use in the most ordinary way, to print or display an
announcement including an appropriate copyright notice and a
notice that there is no warranty (or else, saying that you provide
a warranty) and that users may redistribute the program under
these conditions, and telling the user how to view a copy of this
License.  (Exception: if the Program itself is interactive but
does not normally print such an announcement, your work based on
the Program is not required to print an announcement.)

\end{enumerate}


These requirements apply to the modified work as a whole.  If
identifiable sections of that work are not derived from the Program,
and can be reasonably considered independent and separate works in
themselves, then this License, and its terms, do not apply to those
sections when you distribute them as separate works.  But when you
distribute the same sections as part of a whole which is a work based
on the Program, the distribution of the whole must be on the terms of
this License, whose permissions for other licensees extend to the
entire whole, and thus to each and every part regardless of who wrote it.

Thus, it is not the intent of this section to claim rights or contest
your rights to work written entirely by you; rather, the intent is to
exercise the right to control the distribution of derivative or
collective works based on the Program.

In addition, mere aggregation of another work not based on the Program
with the Program (or with a work based on the Program) on a volume of
a storage or distribution medium does not bring the other work under
the scope of this License.

\item
You may copy and distribute the Program (or a work based on it,
under Section 2) in object code or executable form under the terms of
Sections 1 and 2 above provided that you also do one of the following:

\begin{enumerate}

\item
Accompany it with the complete corresponding machine-readable
source code, which must be distributed under the terms of Sections
1 and 2 above on a medium customarily used for software interchange; or,

\item
Accompany it with a written offer, valid for at least three
years, to give any third party, for a charge no more than your
cost of physically performing source distribution, a complete
machine-readable copy of the corresponding source code, to be
distributed under the terms of Sections 1 and 2 above on a medium
customarily used for software interchange; or,

\item
Accompany it with the information you received as to the offer
to distribute corresponding source code.  (This alternative is
allowed only for noncommercial distribution and only if you
received the program in object code or executable form with such
an offer, in accord with Subsubsection b above.)

\end{enumerate}


The source code for a work means the preferred form of the work for
making modifications to it.  For an executable work, complete source
code means all the source code for all modules it contains, plus any
associated interface definition files, plus the scripts used to
control compilation and installation of the executable.  However, as a
special exception, the source code distributed need not include
anything that is normally distributed (in either source or binary
form) with the major components (compiler, kernel, and so on) of the
operating system on which the executable runs, unless that component
itself accompanies the executable.

If distribution of executable or object code is made by offering
access to copy from a designated place, then offering equivalent
access to copy the source code from the same place counts as
distribution of the source code, even though third parties are not
compelled to copy the source along with the object code.

\item
You may not copy, modify, sublicense, or distribute the Program
except as expressly provided under this License.  Any attempt
otherwise to copy, modify, sublicense or distribute the Program is
void, and will automatically terminate your rights under this License.
However, parties who have received copies, or rights, from you under
this License will not have their licenses terminated so long as such
parties remain in full compliance.

\item
You are not required to accept this License, since you have not
signed it.  However, nothing else grants you permission to modify or
distribute the Program or its derivative works.  These actions are
prohibited by law if you do not accept this License.  Therefore, by
modifying or distributing the Program (or any work based on the
Program), you indicate your acceptance of this License to do so, and
all its terms and conditions for copying, distributing or modifying
the Program or works based on it.

\item
Each time you redistribute the Program (or any work based on the
Program), the recipient automatically receives a license from the
original licensor to copy, distribute or modify the Program subject to
these terms and conditions.  You may not impose any further
restrictions on the recipients' exercise of the rights granted herein.
You are not responsible for enforcing compliance by third parties to
this License.

\item
If, as a consequence of a court judgment or allegation of patent
infringement or for any other reason (not limited to patent issues),
conditions are imposed on you (whether by court order, agreement or
otherwise) that contradict the conditions of this License, they do not
excuse you from the conditions of this License.  If you cannot
distribute so as to satisfy simultaneously your obligations under this
License and any other pertinent obligations, then as a consequence you
may not distribute the Program at all.  For example, if a patent
license would not permit royalty-free redistribution of the Program by
all those who receive copies directly or indirectly through you, then
the only way you could satisfy both it and this License would be to
refrain entirely from distribution of the Program.

If any portion of this section is held invalid or unenforceable under
any particular circumstance, the balance of the section is intended to
apply and the section as a whole is intended to apply in other
circumstances.

It is not the purpose of this section to induce you to infringe any
patents or other property right claims or to contest validity of any
such claims; this section has the sole purpose of protecting the
integrity of the free software distribution system, which is
implemented by public license practices.  Many people have made
generous contributions to the wide range of software distributed
through that system in reliance on consistent application of that
system; it is up to the author/donor to decide if he or she is willing
to distribute software through any other system and a licensee cannot
impose that choice.

This section is intended to make thoroughly clear what is believed to
be a consequence of the rest of this License.

\item
If the distribution and/or use of the Program is restricted in
certain countries either by patents or by copyrighted interfaces, the
original copyright holder who places the Program under this License
may add an explicit geographical distribution limitation excluding
those countries, so that distribution is permitted only in or among
countries not thus excluded.  In such case, this License incorporates
the limitation as if written in the body of this License.

\item
The Free Software Foundation may publish revised and/or new versions
of the General Public License from time to time.  Such new versions will
be similar in spirit to the present version, but may differ in detail to
address new problems or concerns.

Each version is given a distinguishing version number.  If the Program
specifies a version number of this License which applies to it and ``any
later version'', you have the option of following the terms and conditions
either of that version or of any later version published by the Free
Software Foundation.  If the Program does not specify a version number of
this License, you may choose any version ever published by the Free Software
Foundation.

\item
If you wish to incorporate parts of the Program into other free
programs whose distribution conditions are different, write to the author
to ask for permission.  For software which is copyrighted by the Free
Software Foundation, write to the Free Software Foundation; we sometimes
make exceptions for this.  Our decision will be guided by the two goals
of preserving the free status of all derivatives of our free software and
of promoting the sharing and reuse of software generally.

\end{enumerate}

\subsubsection{No Warranty}

\begin{enumerate}

\addtocounter{enumi}{9}

\item
Because the program is licensed free of charge, there is no warranty
for the program, to the extent permitted by applicable law.  Except when
otherwise stated in writing the copyright holders and/or other parties
provide the program ``as is'' without warranty of any kind, either expressed
or implied, including, but not limited to, the implied warranties of
merchantability and fitness for a particular purpose.  The entire risk as
to the quality and performance of the program is with you.  Should the
program prove defective, you assume the cost of all necessary servicing,
repair or correction.

\item
In no event unless required by applicable law or agreed to in writing
will any copyright holder, or any other party who may modify and/or
redistribute the program as permitted above, be liable to you for damages,
including any general, special, incidental or consequential damages arising
out of the use or inability to use the program (including but not limited
to loss of data or data being rendered inaccurate or losses sustained by
you or third parties or a failure of the program to operate with any other
programs), even if such holder or other party has been advised of the
possibility of such damages.
\end{enumerate}

%%% Local Variables: 
%%% mode: latex
%%% TeX-master: "beameruserguide"
%%% End: 


\section{Tutorial: A Picture for Karl's Students}

This tutorial is intended for new users of \pgfname\ and \tikzname. It
does not give an exhaustive account of all the features of \tikzname\ or
\pgfname, just of those that you are likely to use right away.

Karl is a math and chemistry high-school teacher. He used to create
the graphics in his worksheets and exams using \LaTeX's |{picture}|
environment. While the results were acceptable, creating the graphics
often turned out to be a lengthy process. Also, there tended to be
problems with lines having slightly wrong angles and circles also
seemed to be hard to get right. Naturally, his students could not care
less whether the lines had the exact right angles and they find
Karl's exams too difficult no matter how nicely they were drawn. But
Karl was never entirely satisfied with the result.

Karl's son, who was even less satisfied with the results (he did not
have to take the exams, after all),  told Karl that he might wish
to try out a new package for creating graphics. A bit confusingly,
this package seems to have two names: First, Karl had to download and 
install a package called \pgfname. Then it turns out that inside this
package there is another package called \tikzname, which is supposed to
stand for ``\tikzname\ ist \emph{kein}  Zeichenprogramm.'' Karl finds this
all a bit strange and \tikzname\ seems to indicate that the package
does not do what he needs. However, having used \textsc{gnu}
software for quite some time and ``\textsc{gnu} not being Unix,''
there seems to be hope yet. His son assures him that \tikzname's name is
intended to warn people that \tikzname\ is not a program that you can
use to draw graphics with your mouse or tablet. Rather, it is more
like a ``graphics language.''


\subsection{Problem Statement}

Karl wants to put a graphic on the next worksheet for his
students. He is currently teaching his students about sine and
cosine. What he would like to have is something that looks like this
(ideally):

\noindent
\begin{tikzpicture}[scale=3,cap=round]
  % Local definitions
  \def\costhirty{0.8660256}

  % Colors
  \colorlet{anglecolor}{green!50!black}
  \colorlet{sincolor}{red}
  \colorlet{tancolor}{orange!80!black}
  \colorlet{coscolor}{blue}

  % Styles
  \tikzstyle axes=[]
  \tikzstyle important line=[very thick]
  \tikzstyle information text=[rounded corners,fill=red!10,inner sep=1ex]

  % The graphic
  \draw[style=help lines,step=0.5cm] (-1.4,-1.4) grid (1.4,1.4);
  
  \draw (0,0) circle (1cm);

  \begin{scope}[style=axes]
    \draw[->] (-1.5,0) -- (1.5,0) node[right] {$x$};
    \draw[->] (0,-1.5) -- (0,1.5) node[above] {$y$};

    \foreach \x/\xtext in {-1, -.5/-\frac{1}{2}, 1}
      \draw[xshift=\x cm] (0pt,1pt) -- (0pt,-1pt) node[below,fill=white] {$\xtext$};
  
    \foreach \y/\ytext in {-1, -.5/-\frac{1}{2}, .5/\frac{1}{2}, 1}
      \draw[yshift=\y cm] (1pt,0pt) -- (-1pt,0pt) node[left,fill=white] {$\ytext$};
  \end{scope}
    
  \filldraw[fill=green!20,draw=anglecolor] (0,0) -- (3mm,0pt) arc(0:30:3mm);
  \draw (15:2mm) node[anglecolor] {$\alpha$};
    
  \draw[style=important line,sincolor]
    (30:1cm) -- node[left=1pt,fill=white] {$\sin \alpha$} +(0,-.5);
  
  \draw[style=important line,coscolor]
    (0,0) -- node[below=2pt,fill=white] {$\cos \alpha$} (\costhirty,0);
  
  \draw[style=important line,tancolor] (1,0) --
    node [right=1pt,fill=white]
    {
      $\displaystyle \tan \alpha \color{black}=
      \frac{{\color{sincolor}\sin \alpha}}{\color{coscolor}\cos \alpha}$
    } (intersection of 0,0--30:1cm and 1,0--1,1) coordinate (t);

  \draw (0,0) -- (t);
  
  \draw[xshift=1.85cm] node [right,text width=6cm,style=information text]
    {
      The {\color{anglecolor} angle $\alpha$} is $30^\circ$ in the
      example ($\pi/6$ in radians). The {\color{sincolor}sine of
        $\alpha$}, which is the height of the red line, is
      \[
      {\color{sincolor} \sin \alpha} = 1/2.
      \]
      By the Theorem of Pythagoras we have ${\color{coscolor}\cos^2 \alpha} +
      {\color{sincolor}\sin^2\alpha} =1$. Thus the length of the blue
      line, which is the {\color{coscolor}cosine of $\alpha$}, must be
      \[
      {\color{coscolor}\cos\alpha} = \sqrt{1 - 1/4} = \textstyle
      \frac{1}{2} \sqrt 2. 
      \]%
      This shows that {\color{tancolor}$\tan \alpha$}, which is the
      height of the orange line, is  
      \[
      {\color{tancolor}\tan\alpha} = \frac{{\color{sincolor}\sin
          \alpha}}{\color{coscolor}\cos \alpha} = 1/\sqrt 2.
      \]%
    };
\end{tikzpicture}


\subsection{Setting up the Environment}

In \tikzname, to draw a picture, at the start of the picture
you need to tell \TeX\ or \LaTeX\ that you want to start a picture. In
\LaTeX\ this is done using the environment |{tikzpicture}|, in plain
\TeX\ you just use |\tikzpicture| to start the picture and
|\endtikzpicture| to end it.

\subsubsection{Setting up the Environment in \LaTeX}

Karl, being a \LaTeX\ user, thus sets up his file as follows:

\begin{codeexample}[code only]
\documentclass{article} % say
\usepackage{tikz}
\begin{document}
We are working on
\begin{tikzpicture}
  \draw (-1.5,0) -- (1.5,0);
  \draw (0,-1.5) -- (0,1.5);
\end{tikzpicture}.
\end{document}
\end{codeexample}

When executed, that is, run via |pdflatex| or via |latex| followed by
|dvips|, the resulting will contain something that looks like this:

\begin{codeexample}[width=7cm]
We are working on
\begin{tikzpicture}
  \draw (-1.5,0) -- (1.5,0);
  \draw (0,-1.5) -- (0,1.5);
\end{tikzpicture}.
\end{codeexample}

Admittedly, not quite the whole picture, yet, but we
do have the axes established. Well, not quite, but we have the lines
that make up the axes drawn. Karl suddenly has a sinking feeling
that the picture is still some way off. 

Let's have a more detailed look at the code. First, the package
|tikz| is loaded. This package is a so-called ``frontend'' to the
basic \pgfname\ system. The basic layer, which is also described in this
manual, is somewhat more, well, basic and thus harder to use. The
frontend makes things easier by providing a simpler syntax.

Inside the environment there are two |\draw| commands. They mean:
``The path, which is specified following the command up to the
semicolon, should be drawn.'' The first path is specified
as |(-1.5,0) -- (0,1.5)|, which means ``a straight line from the point
at position $(-1.5,0)$ to the point at position $(0,1.5)$.'' Here, the
positions are specified within a special coordinate system in which,
initially, one unit is 1cm.

Karl is quite pleased to note that the environment automatically
reserves enough space to encompass the picture.


\subsubsection{Setting up the Environment in Plain \TeX}

Karl's wife Gerda, who also happens to be a math teacher, is not a
\LaTeX\ user, but uses plain \TeX\ since she prefers to do things
``the old way.'' She can also use \tikzname. Instead of
|\usepackage{tikz}| she has to write |\input tikz.tex| and instead of
|\begin{tikzpicture}| she writes |\tikzpicture| and  instead of
  |\end{tikzpicture}| she writes |\endtikzpicture|. 

Thus, she would use:
\begin{codeexample}[code only]
%% Plain TeX file
\input tikz.tex
\baselineskip=12pt
\hsize=6.3truein
\vsize=8.7truein
We are working on
\tikzpicture
  \draw (-1.5,0) -- (1.5,0);
  \draw (0,-1.5) -- (0,1.5);
\endtikzpicture.
\bye
\end{codeexample}

Gerda can typeset this file using either |pdftex| or |tex| together
with |dvips|. \tikzname\ will automatically discern which driver she is
using. If she wishes to use |dvipdfm| together with |tex|, she 
either needs to modify the file |pgf.cfg| or can write
|\def\pgfsysdriver{pgfsys-dvipdfm.def}| somewhere \emph{before} she
inputs |tikz.tex| or |pgf.tex|.




\subsection{Straight Path Construction}

The basic building block of all pictures in \tikzname\ is the path. 
A \emph{path} is a series of straight lines and curves that are
connected (that is not the whole picture, but let us ignore the
complications for the moment). You start a path by specifying the
coordinates of the start position as a point in round brackets, as in
|(0,0)|. This is followed by a series of ``path extension
operations.'' The simplest is |--|, which we used already. It must be
followed by another coordinate and it extends the path in a straight
line to this new position. For example, if we were to turn the two
paths of the axes into one path, the following would result:

\begin{codeexample}[]
\tikz \draw (-1.5,0) -- (1.5,0) -- (0,-1.5) -- (0,1.5);
\end{codeexample}

Karl is a bit confused by the fact that there is no |{tikzpicture}|
environment, here. Instead, the little command |\tikz| is used. This
command either takes one argument (starting with an opening brace as in
|\tikz{\draw (0,0) -- (1.5,0)}|, which yields \tikz{\draw (0,0)
 --(1.5,0);}) or collects everything up to the next semicolon and
puts it inside a |{tikzpicture}| environment. As a rule of thumb, all
\tikzname\ graphic drawing commands must occur as an argument of |\tikz|
or inside a |{tikzpicture}| environment. Fortunately, the command
|\draw| will only be defined inside this environment, so there is
little chance that you will accidentally do something wrong here. 



\subsection{Curved Path Construction}

The next thing Karl wants to do is to draw the circle. For this,
straight lines obviously will not do. Instead, we need some way to
draw curves. For this, \tikzname\ provides a special syntax. One or two
``control points'' are needed. The math behind them is not quite
trivial, but here is the basic idea: Suppose you are at point $x$ and
the first control point is $y$. Then the curve will start ``going in
the direction of~$y$ at~$x$,'' that is, the tangent of the curve at $x$
will point toward~$y$. Next, suppose the curve should end at $z$ and
the second support point is $w$. Then the curve will, indeed, end at
$z$ and the tangent of the curve at point $z$ will go through $w$.

Here is an example (the control points have been added for clarity):
\begin{codeexample}[]
\begin{tikzpicture}
  \filldraw [gray] (0,0) circle (2pt)
                   (1,1) circle (2pt)
                   (2,1) circle (2pt)
                   (2,0) circle (2pt);
  \draw (0,0) .. controls (1,1) and (2,1) .. (2,0);
\end{tikzpicture}
\end{codeexample}

The general syntax for extending a path in a ``curved'' way is
|.. controls| \meta{first control point} |and| \meta{second control
  point} |..| \meta{end point}. You can leave out the |and|
\meta{second control point}, which causes the first one to be used 
twice.

So, Karl can now add the first half circle to the picture:

\begin{codeexample}[]
\begin{tikzpicture}
  \draw (-1.5,0) -- (1.5,0);
  \draw (0,-1.5) -- (0,1.5);
  \draw (-1,0) .. controls (-1,0.555) and (-0.555,1) .. (0,1)
               .. controls (0.555,1) and (1,0.555) .. (1,0);
\end{tikzpicture}
\end{codeexample}

Karl is happy with the result, but finds specifying circles in this
way to be extremely awkward. Fortunately, there is a much simpler way.


\subsection{Circle Path Construction}

In order to draw a circle, the path construction operation |circle| can
be used. This operation is followed by a radius in round brackets as in
the following example: (Note that the previous position is used as the
\emph{center} of the circle.)

\begin{codeexample}[]
\tikz \draw (0,0) circle (10pt);
\end{codeexample}

You can also append an ellipse to the path using the |ellipse|
operation. Instead of a single radius you can specify two of them, one
for the $x$-direction and one for the $y$-direction, separated by
|and|: 

\begin{codeexample}[]
\tikz \draw (0,0) ellipse (20pt and 10pt);
\end{codeexample}

To draw an ellipse whose axes are not horizontal and vertical, but
point in an arbitrary direction (a ``turned ellipse'' like \tikz
\draw[rotate=30] (0,0) ellipse (6pt and 3pt);) you can use
transformations, which are explained later. The code for the little
ellipse is |\tikz \draw[rotate=30] (0,0) ellipse (6pt and 3pt);|, by
the way. 

So, returning to Karl's problem, he can write
|\draw (0,0) circle (1cm);| to draw the circle:

\begin{codeexample}[]
\begin{tikzpicture}
  \draw (-1.5,0) -- (1.5,0);
  \draw (0,-1.5) -- (0,1.5);
  \draw (0,0) circle (1cm);
\end{tikzpicture}
\end{codeexample}


At this point, Karl is a bit alarmed that the circle is so small when
he wants the final picture to be much bigger. He is pleased to learn
that \tikzname\ has powerful transformation options and scaling
everything by a factor of three is very easy. But let us leave the
size as it is for the moment to save some space. 




\subsection{Rectangle Path Construction}

The next things we would like to have is the grid in the background.
There are several ways to produce it. For example, one might draw lots of
rectangles. Since rectangles are so common, there is a special syntax
for them: To add a rectangle to the current path, use the |rectangle|
path construction operation. This operation should be followed by another
coordinate and will append a rectangle to the path such that the
previous coordinate and the next coordinates are corners of the
rectangle. So, let us add two rectangles to the picture:

\begin{codeexample}[]
\begin{tikzpicture}
  \draw (-1.5,0) -- (1.5,0);
  \draw (0,-1.5) -- (0,1.5);
  \draw (0,0) circle (1cm);
  \draw (0,0) rectangle (0.5,0.5);
  \draw (-0.5,-0.5) rectangle (-1,-1);
\end{tikzpicture}
\end{codeexample}

While this may be nice in other situations, this is not really leading
anywhere with Karl's problem: First, we would need an awful lot of
these rectangles and then there is the border that is not ``closed.''

So, Karl is about to resort to simply drawing four vertical and four
horizontal lines using the nice |\draw| command, when he learns that
there is a |grid| path construction operation.



\subsection{Grid Path Construction}

The |grid| path operation adds a grid to the current path. It will add
lines making up a grid that fills the rectangle whose one corner is
the current point and whose other corner is the point following the
|grid| operation. For example, the code
|\tikz \draw[step=2pt] (0,0) grid (10pt,10pt);| produces \tikz
\draw[step=2pt] (0,0) grid (10pt,10pt);. Note how the optional
argument for |\draw| can be used to specify a grid width (there are
also |xstep| and |ystep| to define the steppings independently). As
Karl will learn soon, there are \emph{lots} of things that can be
influenced using such options.

For Karl, the following code could be used:

\begin{codeexample}[]
\begin{tikzpicture}
  \draw (-1.5,0) -- (1.5,0);
  \draw (0,-1.5) -- (0,1.5);
  \draw (0,0) circle (1cm);
  \draw[step=.5cm] (-1.4,-1.4) grid (1.4,1.4);
\end{tikzpicture}
\end{codeexample}

Having another look at the desired picture, Karl notices that it would
be nice for the grid to be more subdued. (His son told him that grids
tend to be distracting if they are not subdued.) To subdue the grid,
Karl adds two more options to the |\draw| command that draws the
grid. First, he uses the color |gray| for the grid lines. Second, he
reduces the line width to |very thin|. Finally, he swaps the ordering
of the commands so that the grid is drawn first and everything else on
top. 

\begin{codeexample}[]
\begin{tikzpicture}
  \draw[step=.5cm,gray,very thin] (-1.4,-1.4) grid (1.4,1.4);
  \draw (-1.5,0) -- (1.5,0);
  \draw (0,-1.5) -- (0,1.5);
  \draw (0,0) circle (1cm);
\end{tikzpicture}
\end{codeexample}


\subsection{Adding a Touch of  Style}

Instead of the options |gray,very thin| Karl could also have
said |style=help lines|. \emph{Styles} are predefined sets of options
that can be used to organize how a graphic is drawn. By saying
|style=help lines| you say ``use the style that I (or someone else)
has set for drawing help lines.'' If Karl decides, at some later
point, that grids should be drawn, say, using the color |blue!50|
instead of |gray|, he could say the following:
\begin{codeexample}[code only]
\tikzstyle help lines=[color=blue!50,very thin]
\end{codeexample}
Alternatively, he could have said the following:
\begin{codeexample}[code only]
\tikzstyle help lines+=[color=blue!50]
\end{codeexample}
This would have added the |color=blue!50| option. The |help lines|
style would now contain \emph{two} color options, but 
the second would override the first.

Using styles makes your graphics code more flexible. You can
change the way things look easily in a consistent manner.

To build a hierarchy of styles you can have one style use
another. So in order to define a style |Karl's grid| that is based on
the |grid| style Karl could say
\begin{codeexample}[code only]
\tikzstyle Karl's grid=[style=help lines,color=blue!50]
...
\draw[style=Karl's grid] (0,0) grid (5,5);
\end{codeexample}

You can also leave out the |style=|. Thus, whenever \tikzname\ encounters
an options that it does not know about, it will check whether this
option happens to be the name of a style. If so, the style is
used. Thus, Karl could also have written:
\begin{codeexample}[code only]
\tikzstyle Karl's grid=[help lines,color=blue!50]
...
\draw[Karl's grid] (0,0) grid (5,5);
\end{codeexample}

For some styles, like the |very thin| style, it is pretty clear what
the style does and there is no need to say |style=very thin|. For
other styles, like |help lines|, it seems more natural to me to say
|style=help lines|. But, mainly, this is a matter of taste.


\subsection{Drawing Options}

Karl wonders what other options there are that influence how a path is
drawn. He saw already that the |color=|\meta{color} option can be used
to set the line's color. The option |draw=|\meta{color} does nearly
the same, only it sets the color for the lines only and a different
color can be used for filling (Karl will need this when he fills the
arc for the angle).

He saw that the style |very thin| yields very thin lines. Karl is not
really surprised by this and neither is he surprised to learn that |thin|
yields thin lines,  |thick| yields thick lines, |very thick| yields
very thick lines, |ultra thick| yields really, really thick lines and
|ultra thin| yields lines that are so thin that low-resolution printers
and displays will have trouble showing them. He wonders what gives
lines of ``normal'' thickness. It turns out that |thin| is the correct
choice. This seems strange to Karl, but his son explains him that
\LaTeX\ has two commands called |\thinlines| and |\thicklines| and
that |\thinlines| gives the line width of ``normal'' lines, more
precisely, of the thickness that, say, the stem of a letter like ``T''
or ``i'' has. Nevertheless, Karl would like to know whether there is
anything ``in the middle'' between |thin| and |thick|. There is:
|semithick|.

Another useful thing one can do with lines is to dash or dot them. For
this, the two styles |dashed| and |dotted| can be used, yielding
\tikz \draw[dashed] (0,0) -- (10pt,0pt); and \tikz \draw[dotted] (0,0)
-- (10pt,0pt);. Both options also exist in a loose and a dense
version, called |loosely dashed|, |densely dashed|, |loosely dotted|,
and |closely dotted|. If he really, really  needs to, Karl can also
define much more complex dashing patterns with the |dash pattern|
option, but his son insists that dashing is to be used with utmost
care and mostly distracts. Karl's son claims that complicated dashing
patterns are evil. Karl's students do not care about dashing patterns. 



\subsection{Arc Path Construction}

Our next obstacle is to draw the arc for the angle. For this, the
|arc| path construction operation is useful, which draws part of a
circle or ellipse. This |arc| operation must be followed by a triple in 
rounded brackets, where the components of the triple are separated by
colons. The first two components are angles, the last one is a
radius. An example would be |(10:80:10pt)|, which means ``an arc from
10 degrees to 80 degrees on a circle of radius 10pt.'' Karl obviously
needs an arc from $0^\circ$ to $30^\circ$. The radius should be
something relatively small, perhaps around one third of the circle's
radius. This gives: |(0:30:3mm)|.

When one uses the arc path construction operation, the specified arc will
be added with its starting point at the current position. So, we first
have to ``get there.'' 

\begin{codeexample}[]
\begin{tikzpicture}
  \draw[step=.5cm,gray,very thin] (-1.4,-1.4) grid (1.4,1.4);
  \draw (-1.5,0) -- (1.5,0);
  \draw (0,-1.5) -- (0,1.5);
  \draw (0,0) circle (1cm);
  \draw (3mm,0mm) arc (0:30:3mm);
\end{tikzpicture}
\end{codeexample}

Karl thinks this is really a bit small and he cannot continue unless
he learns how to do scaling. For this, he can add the |[scale=3]|
option. He could add this option to each |\draw| command, but that
would be awkward. Instead, he adds it to the whole environment, which
causes this option to apply to everything within.

\begin{codeexample}[]
\begin{tikzpicture}[scale=3]
  \draw[step=.5cm,gray,very thin] (-1.4,-1.4) grid (1.4,1.4);
  \draw (-1.5,0) -- (1.5,0);
  \draw (0,-1.5) -- (0,1.5);
  \draw (0,0) circle (1cm);
  \draw (3mm,0mm) arc (0:30:3mm);
\end{tikzpicture}
\end{codeexample}

As for circles, you can specify ``two'' radii in order to get an
elliptical arc.

\begin{codeexample}[]
  \tikz \draw (0,0) arc (0:315:1.75cm and 1cm);
\end{codeexample}


\subsection{Clipping a Path}

In order to save space in this manual, it would be nice to clip Karl's
graphics a bit so that we can focus on the ``interesting''
parts. Clipping is pretty easy in \tikzname. You can use the |\clip|
command clip all subsequent drawing. It works like |\draw|, only it
does not draw anything, but uses the given path to clip everything
subsequently. 

\begin{codeexample}[]
\begin{tikzpicture}[scale=3]
  \clip (-0.1,-0.2) rectangle (1.1,0.75);
  \draw[step=.5cm,gray,very thin] (-1.4,-1.4) grid (1.4,1.4);
  \draw (-1.5,0) -- (1.5,0);
  \draw (0,-1.5) -- (0,1.5);
  \draw (0,0) circle (1cm);
  \draw (3mm,0mm) arc (0:30:3mm);
\end{tikzpicture}
\end{codeexample}

You can also do both at the same time: Draw \emph{and} clip a
path. For this, use the |\draw| command and add the |clip|
option. (This is not the whole picture: You can also use the |\clip|
command and add the |draw| option. Well, that is also not the whole
picture: In reality, |\draw| is just a shorthand for |\path[draw]|
and |\clip| is a shorthand for |\path[clip]| and you could also say
|\path[draw,clip]|.) Here is an example: 

\begin{codeexample}[]
\begin{tikzpicture}[scale=3]
  \clip[draw] (0.5,0.5) circle (.6cm);
  \draw[step=.5cm,gray,very thin] (-1.4,-1.4) grid (1.4,1.4);
  \draw (-1.5,0) -- (1.5,0);
  \draw (0,-1.5) -- (0,1.5);
  \draw (0,0) circle (1cm);
  \draw (3mm,0mm) arc (0:30:3mm);
\end{tikzpicture}
\end{codeexample}


\subsection{Parabola and Sine Path Construction}

Although Karl does not need them for his picture, he is pleased to
learn that there are |parabola| and |sin| and |cos| path operations for
adding parabolas and sine and cosine curves to the current path. For the
|parabola| operation, the current point will lie on the parabola as
well as the point given after the parabola operation. Consider
the following example:

\begin{codeexample}[]
\tikz \draw (0,0) rectangle (1,1)  (0,0) parabola (1,1);
\end{codeexample}

It is also possible to place the bend somewhere else:

\begin{codeexample}[]
\tikz \draw[x=1pt,y=1pt] (0,0) parabola bend (4,16) (6,12);
\end{codeexample}

The operations |sin| and |cos| add a sine or cosine curve in the interval
$[0,\pi/2]$ such that the previous current point is at the start of
the curve and the curve ends at the given end point. Here are two
examples:
\begin{codeexample}[]
A sine \tikz \draw[x=1ex,y=1ex] (0,0) sin (1.57,1); curve.
\end{codeexample}

\begin{codeexample}[]
\tikz \draw[x=1.57ex,y=1ex] (0,0) sin (1,1) cos (2,0) sin (3,-1) cos (4,0)
                            (0,1) cos (1,0) sin (2,-1) cos (3,0) sin (4,1);
\end{codeexample}



\subsection{Filling and Drawing}

Returning to the picture, Karl now wants the angle to be ``filled''
with a very light green. For this he uses |\fill| instead of
|\draw|. Here is what Karl does:

\begin{codeexample}[]
\begin{tikzpicture}[scale=3]
  \clip (-0.1,-0.2) rectangle (1.1,0.75);
  \draw[step=.5cm,gray,very thin] (-1.4,-1.4) grid (1.4,1.4);
  \draw (-1.5,0) -- (1.5,0);
  \draw (0,-1.5) -- (0,1.5);
  \draw (0,0) circle (1cm);
  \fill[green!20!white] (0,0) -- (3mm,0mm) arc (0:30:3mm) -- (0,0);
\end{tikzpicture}
\end{codeexample}

The color |green!20!white| means 20\% green and 80\% white mixed
together. Such color expression are possible since \pgfname\ uses Uwe
Kern's |xcolor| package, see the documentation of that package for
details on color expressions.

What would have happened, if Karl had not ``closed'' the path using
|--(0,0)| at the end? In this case, the path is closed automatically,
so this could have been omitted. Indeed, it would even have been
better to write the following, instead:
\begin{codeexample}[code only]
  \fill[green!20!white] (0,0) -- (3mm,0mm) arc (0:30:3mm) -- cycle;
\end{codeexample}
The |--cycle| causes the current path to be closed (actually the
current part of the current path) by smoothly joining the first and
last point. To appreciate the difference, consider the following
example:

\begin{codeexample}[]
\begin{tikzpicture}[line width=5pt]
  \draw (0,0) -- (1,0) -- (1,1) -- (0,0);
  \draw (2,0) -- (3,0) -- (3,1) -- cycle;
\end{tikzpicture}
\end{codeexample}

You can also fill and draw a path at the same time using the
|\filldraw| command. This will first draw the path, then fill it. This
may not seem too useful, but you can specify different colors to be
used for filling and for stroking. These are specified as optional
arguments like this:

\begin{codeexample}[]
\begin{tikzpicture}[scale=3]
  \clip (-0.1,-0.2) rectangle (1.1,0.75);
  \draw[step=.5cm,gray,very thin] (-1.4,-1.4) grid (1.4,1.4);
  \draw (-1.5,0) -- (1.5,0);
  \draw (0,-1.5) -- (0,1.5);
  \draw (0,0) circle (1cm);
  \filldraw[fill=green!20!white, draw=green!50!black]
    (0,0) -- (3mm,0mm) arc (0:30:3mm) -- cycle;
\end{tikzpicture}
\end{codeexample}



\subsection{Shading}

Karl briefly considers the possibility of making the angle ``more
fancy'' by \emph{shading} it. Instead of filling the with a uniform
color, a smooth transition between different colors is used. For this,
|\shade| and |\shadedraw|, for shading and drawing at the same time,
can be used: 

\begin{codeexample}[]
  \tikz \shade (0,0) rectangle (2,1)  (3,0.5) circle (.5cm);
\end{codeexample}
The default shading is a smooth transition from gray to white. To
specify different colors, you can use options:

\begin{codeexample}[]
\begin{tikzpicture}[rounded corners,ultra thick]
  \shade[top color=yellow,bottom color=black] (0,0) rectangle +(2,1);
  \shade[left color=yellow,right color=black] (3,0) rectangle +(2,1);
  \shadedraw[inner color=yellow,outer color=black,draw=yellow] (6,0) rectangle +(2,1);
  \shade[ball color=green] (9,.5) circle (.5cm);
\end{tikzpicture}
\end{codeexample}

For Karl, the following might be appropriate:

\begin{codeexample}[]
\begin{tikzpicture}[scale=3]
  \clip (-0.1,-0.2) rectangle (1.1,0.75);
  \draw[step=.5cm,gray,very thin] (-1.4,-1.4) grid (1.4,1.4);
  \draw (-1.5,0) -- (1.5,0);
  \draw (0,-1.5) -- (0,1.5);
  \draw (0,0) circle (1cm);
  \shadedraw[left color=gray,right color=green, draw=green!50!black]
    (0,0) -- (3mm,0mm) arc (0:30:3mm) -- cycle;
\end{tikzpicture}
\end{codeexample}

However, he wisely decides that shadings usually only distract without
adding anything to the picture.


\subsection{Specifying Coordinates}

Karl now wants to add the sine and cosine lines. He knows already that
he can use the |color=| option to set the lines's colors. So, what is
the best way to specify the coordinates?

There are different ways of specifying coordinates. The easiest way is
to say something like |(10pt,2cm)|. This means 10pt in $x$-direction
and 2cm in $y$-directions. Alternatively, you can also leave out the
units as in |(1,2)|, which means ``one times the current $x$-vector
plus twice the current $y$-vector.'' These vectors default to 1cm in
the $x$-direction and 1cm in the $y$-direction, respectively.

In order to specify points in polar coordinates, use the notation
|(30:1cm)|, which means 1cm in direction 30 degree. This is obviously
quite useful to ``get to the point $(\cos 30^\circ,\sin 30^\circ)$ on
the circle.'' 

You can add a single |+| sign in front of a coordinate or two of
them as in |+(1cm,0cm)| or |++(0cm,2cm)|. Such coordinates are interpreted
differently: The first form means ``1cm upwards from the previous
specified position'' and the second means ``2cm to the right of the
previous specified position, making this the new specified position.''
For example, we can draw the sine line as follows:

\begin{codeexample}[]
\begin{tikzpicture}[scale=3]
  \clip (-0.1,-0.2) rectangle (1.1,0.75);
  \draw[step=.5cm,gray,very thin] (-1.4,-1.4) grid (1.4,1.4);
  \draw (-1.5,0) -- (1.5,0);
  \draw (0,-1.5) -- (0,1.5);
  \draw (0,0) circle (1cm);
  \filldraw[fill=green!20,draw=green!50!black]
    (0,0) -- (3mm,0mm) arc (0:30:3mm) -- cycle;
  \draw[red,very thick] (30:1cm) -- +(0,-0.5);
\end{tikzpicture}
\end{codeexample}

Karl used the fact $\sin 30^\circ = 1/2$. However, he very much
doubts that his students know this, so it would be nice to have a way
of specifying ``the point straight down from |(30:1cm)| that lies on
the $x$-axis.'' This is, indeed, possible using a special syntax: Karl
can write \verb!(30:1cm |- 0,0)!. In general, the meaning of
|(|\meta{p}\verb! |- !\meta{q}|)| is ``the intersection of a vertical
line through $p$ and a horizontal line through $q$.''

Next, let us draw the cosine line. One way would be to say
\verb!(30:1cm |- 0,0) -- (0,0)!. Another way is the following: we
``continue'' from where the sine ends: 

\begin{codeexample}[]
\begin{tikzpicture}[scale=3]
  \clip (-0.1,-0.2) rectangle (1.1,0.75);
  \draw[step=.5cm,gray,very thin] (-1.4,-1.4) grid (1.4,1.4);
  \draw (-1.5,0) -- (1.5,0);
  \draw (0,-1.5) -- (0,1.5);
  \draw (0,0) circle (1cm);
  \filldraw[fill=green!20,draw=green!50!black] (0,0) -- (3mm,0mm) arc
  (0:30:3mm) -- cycle;
  \draw[red,very thick]  (30:1cm) -- +(0,-0.5);
  \draw[blue,very thick] (30:1cm) ++(0,-0.5) -- (0,0);
\end{tikzpicture}
\end{codeexample}

Note the there is no |--| between |(30:1cm)| and |+(0,-0.5)|. In
detail, this path is interpreted as follows: ``First, the |(30:1cm)|
tells me to move by pen to $(\cos 30^\circ,1/2)$. Next, there comes
another coordinate specification, so I move my pen there without drawing
anything. This new point is half a unit down from the last position,
thus it is at $(\cos 30^\circ,0)$. Finally, I move the pen to the
origin, but this time drawing something (because of the |--|).''

To appreciate the difference between |+| and |++| consider the
following example:

\begin{codeexample}[]
\begin{tikzpicture}
  \def\rectanglepath{-- ++(1cm,0cm)  -- ++(0cm,1cm)  -- ++(-1cm,0cm) -- cycle}
  \draw (0,0) \rectanglepath;
  \draw (1.5,0) \rectanglepath;
\end{tikzpicture}
\end{codeexample}

By comparison, when using a single |+|, the coordinates are different:

\begin{codeexample}[]
\begin{tikzpicture}
  \def\rectanglepath{-- +(1cm,0cm)  -- +(1cm,1cm)  -- +(0cm,1cm) -- cycle}
  \draw (0,0) \rectanglepath;
  \draw (1.5,0) \rectanglepath;
\end{tikzpicture}
\end{codeexample}


Naturally, all of this could have been written more clearly and more
economically like this (either with a single of a double |+|): 
\begin{codeexample}[]
\tikz \draw (0,0) rectangle +(1,1)  (1.5,0) rectangle +(1,1);
\end{codeexample}



Karl is left with the line for $\tan \alpha$, which seems difficult to
specify using transformations and polar coordinates. For this he needs
another way of specifying coordinates: Karl can specify intersections
of lines as coordinates. The line for $\tan \alpha$ starts at $(1,0)$
and goes upward to a point that is at the intersection of a line going
``up'' and a line going from the origin through |(30:1cm)|. The syntax
for this point is the following:

\begin{codeexample}[code only]
\draw[very thick,orange] (1,0) -- (intersection of 1,0--1,1 and 0,0--30:1cm);
\end{codeexample}

In the following, two final examples of how to use relative
positioning are presented. Note that the transformation options,
which are explained later, are often more useful for shifting than
relative positioning. 

\begin{codeexample}[]
\begin{tikzpicture}[scale=0.5]
  \draw (0,0) -- (90:1cm) arc (90:360:1cm) arc (0:30:1cm) -- cycle;
  \draw (60:5pt) -- +(30:1cm) arc (30:90:1cm) -- cycle;

  \draw (3,0)  +(0:1cm) -- +(72:1cm) -- +(144:1cm) -- +(216:1cm) --
               +(288:1cm) -- cycle;
\end{tikzpicture}
\end{codeexample}



\subsection{Adding Arrow Tips}

Karl now wants to add the little arrow tips at the end of the axes. He has
noticed that in many plots, even in scientific journals, these arrow tips
seem to missing, presumably because the generating programs cannot
produce them. Karl thinks arrow tips belong at the end of axes. His
son agrees. His students do not care about arrow tips.

It turns out that adding arrow tips is pretty easy: Karl adds the option
|->| to the drawing commands for the axes:

\begin{codeexample}[]
\begin{tikzpicture}[scale=3]
  \clip (-0.1,-0.2) rectangle (1.1,1.51);
  \draw[step=.5cm,gray,very thin] (-1.4,-1.4) grid (1.4,1.4);
  \draw[->] (-1.5,0) -- (1.5,0);
  \draw[->] (0,-1.5) -- (0,1.5);
  \draw (0,0) circle (1cm);
  \filldraw[fill=green!20,draw=green!50!black] (0,0) -- (3mm,0mm) arc
  (0:30:3mm) -- cycle;
  \draw[red,very thick]    (30:1cm) -- +(0,-0.5);
  \draw[blue,very thick]   (30:1cm) ++(0,-0.5) -- (0,0);
  \draw[orange,very thick] (1,0) -- (intersection of 1,0--1,1 and 0,0--30:1cm);
\end{tikzpicture}
\end{codeexample}

If Karl had used the option |<-| instead of |->|, arrow tips would
have been put at the beginning of the path. The option |<->| puts
arrow tips at both ends of the path.

There are certain restrictions to the kind of paths to which arrow tips
can be added. As a rule of thumb, you can add arrow tips only to a
single open ``line.'' For example, you should not try to add tips to,
say, a rectangle or a circle. (You can try, but no guarantees as to what
will happen now or in future versions.) However, you can add arrow
tips to curved paths and to paths that have several segments, as in
the following examples:

\begin{codeexample}[]
\begin{tikzpicture}
  \draw [<->] (0,0) arc (180:30:10pt);
  \draw [<->] (1,0) -- (1.5cm,10pt) -- (2cm,0pt) -- (2.5cm,10pt);
\end{tikzpicture}
\end{codeexample}

Karl has a more detailed look at the arrow that \tikzname\ puts at the
end. It looks like this when he zooms it: \tikz { \useasboundingbox
  (0pt,-.5ex) rectangle (10pt,.5ex); \draw[->,line width=1pt] (0pt,0pt) --
  (10pt,0pt); }. The shape seems vaguely familiar and, indeed, this is
exactly the end of \TeX's standard arrow used in something like
$f\colon A \to B$.


Karl likes the arrow, especially since it is not ``as thick'' as the
arrows offered by many other packages. However, he expects that,
sometimes, he might need to use some other kinds of arrow.
To do so, Karl can say |>=|\meta{right arrow tip kind}, where
\meta{right arrow tip kind} is a special arrow tip specification. For
example, if Karl says |>=stealth|, then he tells \tikzname\
that he would like  ``stealth-fighter-like'' arrow tips: 

\begin{codeexample}[]
\begin{tikzpicture}[>=stealth]
  \draw [->] (0,0) arc (180:30:10pt);
  \draw [<<-,very thick] (1,0) -- (1.5cm,10pt) -- (2cm,0pt) -- (2.5cm,10pt);
\end{tikzpicture}
\end{codeexample}%>>

Karl wonders whether such a military name for the arrow type is really
necessary. He is not really mollified when his son tells him that
Microsoft's PowerPoint uses the same name. He decides to have his
students discuss this at some point.

In addition to |stealth| there are several other predefined arrow tip
kinds Karl can choose from, see
Section~\ref{section-library-arrows}. Furthermore, he can define
arrows types himself, if he needs new ones. 




\subsection{Scoping}

Karl saw already that there are numerous graphic options that affect how
paths are rendered. Often, he would like to apply certain options to
a whole set of graphic commands. For example, Karl might wish to draw
three paths using a |thick| pen, but would like everything else to
be drawn ``normally.''

If Karl wishes to set a certain graphic option for the whole picture,
he can simply pass this option to the |\tikz| command or to the
|{tikzpicture}| environment (Gerda would pass the options to
|\tikzpicture|). However, if Karl wants to apply graphic options to a
local group, he put these commands inside a |{scope}| environment
(Gerda uses |\scope| and |\endscope|). This environment takes graphic
options as an optional argument and these options apply to everything
inside the scope, but not to anything outside.

Here is an example:

\begin{codeexample}[]
\begin{tikzpicture}[ultra thick]
  \draw (0,0) -- (0,1);
  \begin{scope}[thin]
    \draw (1,0) -- (1,1);
    \draw (2,0) -- (2,1);
  \end{scope}
  \draw (3,0) -- (3,1);  
\end{tikzpicture}
\end{codeexample}

Scoping has another interesting effect: Any changes to the clipping
area are local to the scope. Thus, if you say |\clip| somewhere inside
a scope, the effect of the |\clip| command ends at the end of the
scope. This is useful since there is no other way of ``enlarging'' the
clipping area.

Karl has also already seen that giving options to commands like
|\draw| apply only to that command. In turns out that the situation is
slightly more complex. First, options to a command like |\draw| are
not really options to the command, but they are ``path options'' and
can be given anywhere on the path. So, instead of
|\draw[thin] (0,0) -- (1,0);| one can also write
|\draw (0,0) [thin] -- (1,0);| or |\draw (0,0) -- (1,0) [thin];|; all
of these have the same effect. This might seem strange since in the
last case, it would appear that the |thin| should take effect only
``after'' the line from $(0,0)$ to $(1,0)$ has been draw. However,
most graphic options only apply to the whole path. Indeed, if you say
both |thin| and |thick| on the same path, the last option given will
``win.''

When reading the above, Karl notices that only ``most'' graphic
options apply to the whole path. Indeed, all transformation options do
\emph{not} apply to the whole path, but only to ``everything following
them on the path.'' We will have a more detailed look at this in a
moment. Nevertheless, all options given during a path construction
apply only to this path. 



\subsection{Transformations}

When you specify a  coordinate like |(1cm,1cm)|, where is that
coordinate placed on the page? To determine the position, \tikzname,
\TeX, and \textsc{pdf} or PostScript all apply certain transformations
to the given coordinate in order to determine the finally position on
the page. 

\tikzname\ provides numerous options that allow you to transform
coordinates in \pgfname's private coordinate system. For example, the
|xshift| option allows you to shift all subsequent points by a certain
amount:

\begin{codeexample}[]
\tikz \draw (0,0) -- (0,0.5) [xshift=2pt] (0,0) -- (0,0.5);
\end{codeexample}

It is important to note that you can change transformation ``in the
middle of a path,'' a feature that is not supported by \pdf\
or PostScript. The reason is that \pgfname\ keeps track of its own
transformation matrix.

Here is a more complicated example:
\begin{codeexample}[]
\begin{tikzpicture}[even odd rule,rounded corners=2pt,x=10pt,y=10pt]
  \filldraw[fill=yellow]    (0,0)   rectangle (1,1)
    [xshift=5pt,yshift=5pt] (0,0)   rectangle (1,1)
                [rotate=30] (-1,-1) rectangle (2,2);
\end{tikzpicture}
\end{codeexample}

The most useful transformations are |xshift| and |yshift| for
shifting, |shift| for shifting to a given point as in |shift={(1,0)}|
or |shift={+(0,0)}| (the braces are necessary so that \TeX\ does not
mistake the comma for separating options), |rotate| for rotating by a
certain angle (there is also a |rotate around| for rotating around a
given point), |scale| for scaling by a certain factor, |xscale| and
|yscale| for scaling only in the $x$- or $y$-direction (|xscale=-1| is
a flip), and |xslant| and |yslant| for slanting. If these
transformation and those that I have not mentioned are not
sufficient,  the |cm| option allows you to apply an arbitrary
transformation matrix. Karl's students, by the way, do not know what a
transformation matrix is. 



\subsection{Repeating Things: For-Loops}

Karl's next aim is to add little ticks on the axes at positions $-1$,
$-1/2$, $1/2$, and $1$. For this, it would be nice to use some kind of
``loop,'' especially since he wishes to do the same thing at each of
these positions. There are different packages for doing this. \LaTeX\
has its own internal command for this, |pstricks| comes along with the
powerful |\mulitdo| command. All of these can be used together with
\pgfname\ and \tikzname, so if you are familiar with them, feel free to
use them. \pgfname\ introduces yet another command, called |\foreach|,
which I introduced since I could never remember the syntax of the other
packages. |\foreach| is defined in the package |pgffor| and can be used
independently of \pgfname. \tikzname\ includes it automatically.

In its basic form, the |\foreach| command is easy to use:
\begin{codeexample}[]
\foreach \x in {1,2,3} {$x =\x$, }
\end{codeexample}

The general syntax is |\foreach| \meta{variable}| in {|\meta{list of
    values}|} |\meta{commands}. Inside the \meta{commands}, the
\meta{variable} will be assigned to the different values. If the
\meta{commands} do not start with a brace, everything up to the
next semicolon is used as \meta{commands}.

For Karl and the ticks on the axes, he could use the following code:

\begin{codeexample}[]
\begin{tikzpicture}[scale=3]
  \clip (-0.1,-0.2) rectangle (1.1,1.51);
  \draw[step=.5cm,gray,very thin] (-1.4,-1.4) grid (1.4,1.4);
  \filldraw[fill=green!20,draw=green!50!black] (0,0) -- (3mm,0mm) arc
  (0:30:3mm) -- cycle;
  \draw[->] (-1.5,0) -- (1.5,0);
  \draw[->] (0,-1.5) -- (0,1.5);
  \draw (0,0) circle (1cm);

  \foreach \x in {-1cm,-0.5cm,1cm}
    \draw (\x,-1pt) -- (\x,1pt);
  \foreach \y in {-1cm,-0.5cm,0.5cm,1cm}
    \draw (-1pt,\y) -- (1pt,\y);
\end{tikzpicture}
\end{codeexample}

As a matter of fact, there are many different ways of creating the
ticks. For example, Karl could have put the |\draw ...;| inside curly
braces. He could also have used, say,
\begin{codeexample}[code only]
\foreach \x in {-1,-0.5,1}
  \draw[xshift=\x cm] (0pt,-1pt) -- (0pt,1pt);
\end{codeexample}

Karl is curious what would happen in a more complicated situation
where there are, say, 20 ticks. It seems bothersome to explicitly
mention all these numbers in the set for |\foreach|. Indeed, it is
possible to use |...| inside the |\foreach| statement to iterate over 
a large number of values (which must, however, be dimensionless
real numbers) as in the following example: 

\begin{codeexample}[]
\tikz \foreach \x in {1,...,10}
        \draw (\x,0) circle (0.4cm);
\end{codeexample}

If you provide \emph{two} numbers before the |...|, the |\foreach|
statement will use their difference for the stepping:

\begin{codeexample}[]
\tikz \foreach \x in {-1,-0.5,...,1}
       \draw (\x cm,-1pt) -- (\x cm,1pt);
\end{codeexample}

We can also nest loops to create interesting effects:

\begin{codeexample}[]
\begin{tikzpicture}
  \foreach \x in {1,2,...,5,7,8,...,12}
    \foreach \y in {1,...,5}
    {
      \draw (\x,\y) +(-.5,-.5) rectangle ++(.5,.5);
      \draw (\x,\y) node{\x,\y};
    }
\end{tikzpicture}
\end{codeexample}

The |\foreach| statement can do even trickier stuff, but the above
gives the idea.




\subsection{Adding Text}

Karl is, by now, quite satisfied with the picture. However, the most
important parts, namely the labels, are still missing! 

\tikzname\ offers an easy-to-use and powerful system for adding text and,
more generally, complex shapes to a picture at specific positions. The
basic idea is the following: When \tikzname\ is constructing a path and
encounters the keyword |node| in the middle of a path, it
reads a \emph{node specification}. The keyword |node| is typically
followed by some options and then some text between curly braces. This
text is put inside a normal \TeX\ box (if the node specification
directly follows a coordinate, which is usually the case, \tikzname\ is
able to perform some magic so that it is even possible to use verbatim
text inside the boxes) and then placed at the current position, that
is, at the last specified position (possibly shifted a bit, according
to the given options). However, all nodes are drawn only after the
path has been completely drawn/filled/shaded/clipped/whatever.  

\begin{codeexample}[]
\begin{tikzpicture}
  \draw (0,0) rectangle (2,2);
  \draw (0.5,0.5) node [fill=yellow] {Text at \verb!node 1!}
     -- (1.5,1.5) node {Text at \verb!node 2!};
\end{tikzpicture}
\end{codeexample}

Obviously, Karl would not only like to place nodes \emph{on} the last
specified position, but also to the left or the 
right of these positions. For this, every node object that you
put in your picture is equipped with several \emph{anchors}. For
example, the |north| anchor is in the middle at the upper end of the shape,
the |south| anchor is at the bottom and the |north east| anchor is in
the upper right corner. When you given the option |anchor=north|, the
text will be placed such that this northern anchor will lie on the
current position and the text is, thus, below the current
position. Karl uses this to draw the ticks as follows:

\begin{codeexample}[]
\begin{tikzpicture}[scale=3]
  \clip (-0.6,-0.2) rectangle (0.6,1.51);
  \draw[step=.5cm,style=help lines] (-1.4,-1.4) grid (1.4,1.4);
  \filldraw[fill=green!20,draw=green!50!black]
    (0,0) -- (3mm,0mm) arc (0:30:3mm) -- cycle;
  \draw[->] (-1.5,0) -- (1.5,0);   \draw[->] (0,-1.5) -- (0,1.5);
  \draw (0,0) circle (1cm);

  \foreach \x in {-1,-0.5,1}
    \draw (\x cm,1pt) -- (\x cm,-1pt) node[anchor=north] {$\x$};
  \foreach \y in {-1,-0.5,0.5,1}
    \draw (1pt,\y cm) -- (-1pt,\y cm) node[anchor=east] {$\y$};
\end{tikzpicture}
\end{codeexample}

This is quite nice, already. Using these anchors, Karl can now add
most of the other text elements. However, Karl thinks that, though
``correct,'' it is quite counter-intuitive that in order to place something
\emph{below} a given point, he has to use the \emph{north} anchor. For
this reason, there is an option called |below|, which does the
same as |anchor=north|. Similarly, |above right| does the same as
|anchor=south east|. In addition, |below| takes an optional
dimension argument. If given, the shape will additionally be shifted
downwards by the given amount. So, |below=1pt| can be used to put
a text label below some point and, additionally shift it  1pt
downwards. 

Karl is not quite satisfied with the ticks. He would like to have
$1/2$ or $\frac{1}{2}$ shown instead of $0.5$, partly to show off the
nice capabilities of \TeX\ and \tikzname, partly because for positions
like $1/3$ or $\pi$ it is certainly very much preferable to have the
``mathematical'' tick there instead of just the ``numeric'' tick.
His students, on the other hand, prefer $0.5$ over $1/2$
since they are not too fond of fractions in general.

Karl now faces a problem: For the |\foreach| statement, the position
|\x| should still be given as |0.5| since \tikzname\ will not know where
|\frac{1}{2}| is supposed to be. On the other hand, the typeset text
should really be  |\frac{1}{2}|. To solve this problem, |\foreach|
offers a special syntax: Instead of having one variable |\x|, Karl can
specify two (or even more) variables separated by a slash as in
|\x / \xtext|. Then, the elements in the set over which |\foreach|
iterates must also be of the form \meta{first}|/|\meta{second}. In
each iteration, |\x| will be set to \meta{first} and |\xtext| will be
set to \meta{second}. If no \meta{second} is given, the \meta{first}
will be used again. So, here is the new code for the ticks: 

\begin{codeexample}[]
\begin{tikzpicture}[scale=3]
  \clip (-0.6,-0.2) rectangle (0.6,1.51);
  \draw[step=.5cm,style=help lines] (-1.4,-1.4) grid (1.4,1.4);
  \filldraw[fill=green!20,draw=green!50!black]
    (0,0) -- (3mm,0mm) arc (0:30:3mm) -- cycle;
  \draw[->] (-1.5,0) -- (1.5,0); \draw[->] (0,-1.5) -- (0,1.5);
  \draw (0,0) circle (1cm);

  \foreach \x/\xtext in {-1, -0.5/-\frac{1}{2}, 1}
    \draw (\x cm,1pt) -- (\x cm,-1pt) node[anchor=north] {$\xtext$};
  \foreach \y/\ytext in {-1, -0.5/-\frac{1}{2}, 0.5/\frac{1}{2}, 1}
    \draw (1pt,\y cm) -- (-1pt,\y cm) node[anchor=east] {$\ytext$};
\end{tikzpicture}
\end{codeexample}

Karl is quite pleased with the result, but his son points out that
this is still not perfectly satisfactory: The grid and the circle
interfere with the numbers and decrease their legibility. Karl is not
very concerned by this (his students do not even notice), but his son
insists that there is an easy solution: Karl can add the
|[fill=white]| option to fill out the background of the text shape
with a white color. 

The next thing Karl wants to do is to add the labels like $\sin
\alpha$. For this, he would like to place a label ``in the middle of
line.'' To do so, instead of specifying the label
|node {$\sin\alpha$}|  directly after one of the endpoints of the line
(which would place 
the label at that endpoint), Karl can give the label directly after
the |--|, before the coordinate. By default, this places the label in
the middle of the line, but the |pos=| options can be used to modify
this. Also, options like |near start| and |near end| can be used to
modify this position:


\begin{codeexample}[]
\begin{tikzpicture}[scale=3]
  \clip (-2,-0.2) rectangle (2,0.8);
  \draw[step=.5cm,gray,very thin] (-1.4,-1.4) grid (1.4,1.4);
  \filldraw[fill=green!20,draw=green!50!black] (0,0) -- (3mm,0mm) arc
  (0:30:3mm) -- cycle;
  \draw[->] (-1.5,0) -- (1.5,0) coordinate (x axis);
  \draw[->] (0,-1.5) -- (0,1.5) coordinate (y axis);
  \draw (0,0) circle (1cm);
    
  \draw[very thick,red]
    (30:1cm) -- node[left=1pt,fill=white] {$\sin \alpha$} (30:1cm |- x axis);
  \draw[very thick,blue]
    (30:1cm |- x axis) -- node[below=2pt,fill=white] {$\cos \alpha$} (0,0);
  \draw[very thick,orange] (1,0) -- node [right=1pt,fill=white]
    {$\displaystyle \tan \alpha \color{black}=
      \frac{{\color{red}\sin \alpha}}{\color{blue}\cos \alpha}$}
    (intersection of 0,0--30:1cm and 1,0--1,1) coordinate (t);

  \draw (0,0) -- (t);
  
  \foreach \x/\xtext in {-1, -0.5/-\frac{1}{2}, 1}
    \draw (\x cm,1pt) -- (\x cm,-1pt) node[anchor=north,fill=white] {$\xtext$};
  \foreach \y/\ytext in {-1, -0.5/-\frac{1}{2}, 0.5/\frac{1}{2}, 1}
    \draw (1pt,\y cm) -- (-1pt,\y cm) node[anchor=east,fill=white] {$\ytext$};
\end{tikzpicture}
\end{codeexample}

You can also position labels on curves and, by adding the |sloped|
option, have them rotated such that they match the line's slope. Here
is an example:

\begin{codeexample}[]
\begin{tikzpicture}
  \draw (0,0) .. controls (6,1) and (9,1) ..
    node[near start,sloped,above] {near start}
    node {midway}
    node[very near end,sloped,below] {very near end} (12,0);
\end{tikzpicture}
\end{codeexample}

It remains to draw the explanatory text at the right of the
picture. The main difficulty here lies in limiting the width of the
text ``label,'' which is quite long, so that line breaking is
used. Fortunately, Karl can use the option |text width=6cm| to get the
desired effect. So, here is the full code:

\begin{codeexample}[code only]
\begin{tikzpicture}[scale=3,cap=round]
  % Local definitions
  \def\costhirty{0.8660256}

  % Colors
  \colorlet{anglecolor}{green!50!black}
  \colorlet{sincolor}{red}
  \colorlet{tancolor}{orange!80!black}
  \colorlet{coscolor}{blue}

  % Styles
  \tikzstyle{axes}=[]
  \tikzstyle{important line}=[very thick]
  \tikzstyle{information text}=[rounded corners,fill=red!10,inner sep=1ex]

  % The graphic
  \draw[style=help lines,step=0.5cm] (-1.4,-1.4) grid (1.4,1.4);
  
  \draw (0,0) circle (1cm);

  \begin{scope}[style=axes]
    \draw[->] (-1.5,0) -- (1.5,0) node[right] {$x$} coordinate(x axis);
    \draw[->] (0,-1.5) -- (0,1.5) node[above] {$y$} coordinate(y axis);

    \foreach \x/\xtext in {-1, -.5/-\frac{1}{2}, 1}
      \draw[xshift=\x cm] (0pt,1pt) -- (0pt,-1pt) node[below,fill=white] {$\xtext$};
  
    \foreach \y/\ytext in {-1, -.5/-\frac{1}{2}, .5/\frac{1}{2}, 1}
      \draw[yshift=\y cm] (1pt,0pt) -- (-1pt,0pt) node[left,fill=white] {$\ytext$};
  \end{scope}
    
  \filldraw[fill=green!20,draw=anglecolor] (0,0) -- (3mm,0pt) arc(0:30:3mm);
  \draw (15:2mm) node[anglecolor] {$\alpha$};
    
  \draw[style=important line,sincolor]
    (30:1cm) -- node[left=1pt,fill=white] {$\sin \alpha$} (30:1cm |- x axis);
  
  \draw[style=important line,coscolor]
    (30:1cm |- x axis) -- node[below=2pt,fill=white] {$\cos \alpha$} (0,0);
  
  \draw[style=important line,tancolor] (1,0) -- node[right=1pt,fill=white] {
    $\displaystyle \tan \alpha \color{black}=
    \frac{{\color{sincolor}\sin \alpha}}{\color{coscolor}\cos \alpha}$}
    (intersection of 0,0--30:1cm and 1,0--1,1) coordinate (t);

  \draw (0,0) -- (t);
  
  \draw[xshift=1.85cm]
    node[right,text width=6cm,style=information text]
    {
      The {\color{anglecolor} angle $\alpha$} is $30^\circ$ in the
      example ($\pi/6$ in radians). The {\color{sincolor}sine of
        $\alpha$}, which is the height of the red line, is
      \[
      {\color{sincolor} \sin \alpha} = 1/2.
      \]
      By the Theorem of Pythagoras ...
    };
\end{tikzpicture}
\end{codeexample}


\subsection{Nodes}

Placing text at a given position is just a special case of a more
general underlying mechanism. When you say |\draw (0,0) node{text};|,
what actually happens is that a rectangular node, anchored at its center, is
put at position $(0,0)$. On top of the rectangular node the text
|text| is drawn. Since no action is specified for the rectangle (like
|draw| or |fill|), the rectangle is actually discarded and only the
text is shown. However, by adding |fill| or |draw|, we can make the
underlying shape visible. Furthermore, we can \emph{change} the
shape using for example |shape=circle| or just |circle|. If we include
the package |pgflibraryshapes| we also get |ellipse|:


\begin{codeexample}[]
\begin{tikzpicture}
  \path (0,0)   node[ellipse,fill=yellow,draw]      (h1) {hello world}
        (0.5,2) node[circle,shade,ball color=yellow](h2) {hello world};
  \draw [->,shorten >=2pt] (h1.north) -- (h2.south);
\end{tikzpicture}
\end{codeexample}

As the above example shows, we can add the a name to a node by
putting it in parentheses between |node| and the |{|\meta{text}|}|
(you can also use the |name=| option). This will make \tikzname\ remember your node and all
its anchors. You can then refer to these anchors when specifying
coordinates. The syntax is |(|\meta{node
  name}|.|\meta{anchor}|)|. Currently, and also in the near future, 
\emph{this will not work across pictures since \tikzname\ looses track
  of the positions when it returns control to \TeX.} Magic hackery is
possible for certain drivers, but a portable implementation seems
impossible (just think of a possible \textsc{svg} driver). 

The option |shorten >| causes lines to be shortened by 2pt at the
end. Similarly, |shorten <| can be used to shorten (or even lengthen)
lines at the beginning. This is possible even if no arrow is drawn.

It is not always necessary to specify the anchor. If you do not give
an anchor, \tikzname\ will try to determine a reasonable border anchor by
itself (if \tikzname\ fails to find anything useful, it will use the
center instead). Here is a typical example:

\begin{codeexample}[]
\begin{tikzpicture}
  \begin{scope}[shape=circle,minimum size=1cm,fill=yellow]
    \tikzstyle{every node}=[draw,fill]
    \node (q_A) at (0,0) {$q_A$};
    \node (q_E) at (6,0) {$q_E$};
    \node (q_1) at (2,0) {$q_1$};
    \node (q_2) at (4,2) {$q_2$};
  \end{scope}
  \draw (q_A) -- (q_1) -- (q_2) -| (q_E);
  \draw[->,shorten >=2pt] (q_A) .. controls +(75:1.4cm) and +(105:1.4cm) .. node[above] {$x$} (q_A);
\end{tikzpicture}
\end{codeexample}

In the example, we used the |\node| command, which is an abbreviation
for |\path node|. 

% $Header: /cvsroot/pgf/pgf/doc/pgf/en/pgfmanual-guidelines.tex,v 1.1.1.1 2005/06/29 12:14:02 tantau Exp $

% Copyright 2005 by Till Tantau <tantau@cs.tu-berlin.de>.
%
% This program can be redistributed and/or modified under the terms
% of the GNU Public License, version 2.



\section{Guidelines on Graphics}

The present section is not about \pgfname\ or \tikzname, but about
general guidelines and principles concerning the creation of
graphics for scientific presentations, papers, and books.

The guidelines in this section come from different sources. Many of
them are just what I would like to claim is ``common sense,'' some
reflect my personal experience (though, hopefully, not my personal
preferences), some come from books (the bibliography is still missing,
sorry) on graphic design and typography. 
The most influential source  are the brilliant books
by Edward Tufte. While I do not agree with everything written in these
books, many of Tufte's arguments are so convincing that I decided to
repeat them in the following guidelines. 




\subsection{Should You Follow Guidelines?}

The first thing you should ask yourself when someone presents a bunch of
guidelines is: Should I really follow these guidelines? This is an
important questions, because there are good reasons not to follow
general guidelines.
\begin{itemize}
\item
  The person who setup the guidelines may have had other
  objectives than you do. For example, a guideline might say ``use the
  color red for emphasis.'' While this guideline makes perfect sense
  for, say, a presentation using a projector, red ``color'' has the
  \emph{opposite} effect of ``emphasis'' when printed using a
  black-and-white printer.

  Guidelines were almost always setup to address a specific
  situation. If you are not in this situation, following a guideline
  can do more harm than good.
\item
  The basic rule of typography is: ``Every rule can be broken, as long
  as you are \emph{aware}  that you are breaking a rule.'' This rule
  also applies to graphics. Phrased differently, the basic rule
  states: ``The only mistakes in typography are things done is
  ignorance.''

  When you are aware of a rule and when you decide that breaking the
  rule has a desirable effect, break the rule.
\item
  Some guidelines are simply \emph{wrong}, but everyone follows them
  out of tradition or is forced to do so. My favorite example is a 
  guideline a software company I used to work for has set in a big
  project: All programmers had to declare the parameters of functions
  in \emph{increasing order of size}. So, one-byte
  parameters should come first, then two-byte parameters, and so on. 

  This guideline is total nonsense. An (arguably) sensible guideline
  is ``parameters must be declared alphabetically'' so that parameters
  are easier to find. Another (arguably) sensible guideline is
  ``parameters must be declared in decreasing order of size'' so that
  less byte-alignment cache misses occur when the stack is
  accessed. The guideline the company used maximized cache misses and
  resulted in a more or less random ordering so that programmers
  constantly had to look up the parameter ordering.
\end{itemize}

So, before you apply a guideline or choose not to apply it, ask
yourself these questions: 
\begin{enumerate}
\item
  Does this guideline really address my situation?
\item
  If you do the opposite a guideline says you should do, will the
  advantages outweigh the disadvantages this guideline was supposed to
  prevent?  
\end{enumerate}



\subsection{Planning the Time Needed for the Creation of Graphics}

When you create a paper with numerous graphics, the time needed to
create these graphics becomes an important factor. How much time
should you calculate for the creation of graphics?

As a general rule, assume that a graphic will need as much time to
create as would a text of the same length. For example, when I
write a paper, I need about one hour per page for
the first draft. Later, I need between two and four hours per page
for revisions. Thus, I expect to need about half an hour for the
creation of \emph{a first draft} of a half page graphic. Later on, I
expect another one to two hours before the final graphic is finished.

In many publications, even in good journals, the authors and editors
have obviously  invested a lot of time on the text, but seem to 
have spend about five minutes to create all of the
graphics. Graphics often seem to have been added as an
``afterthought'' or look like a screen shot of whatever the authors's
statistical software shows them. As will be argued later on, the
graphics that programs like \textsc{gnuplot} produce by default are of
poor quality.

Creating informative graphics that help the reader and that fit
together with the main text is a difficult, lengthy process. 
\begin{itemize}
\item
  Treat graphics as first-class citizens of your papers. They deserve
  as much time and energy as the text does.
\item
  Arguably, the creation of graphics deserves \emph{even more} time
  than the writing of the main text since more attention will  be paid
  to the graphics and they will be looked at first. 
\item
  Plan as much time for the creation and revision of a graphic as you
  would plan for text of the same size.
\item
  Difficult graphics with a high information density may require even
  more time.
\item
  Very simple graphics will require less time, but most likely you do
  not want to have ``very simple graphics'' in your paper, anyway;
  just as you would not like to have a ``very simple text'' of the
  same size.  
\end{itemize}



\subsection{Workflow for Creating a Graphic}

When you write a (scientific) paper, you will most likely follow the
following pattern: You have some results/ideas that you would
like to report about. The creation of the paper will typically start
with compiling a rough outline. Then, the different sections are
filled with text to create a first draft. This draft is then revised
repeatedly until, often after substantial revision, a final paper
results. In a good journal paper there is typically not be a single 
sentence that has survived unmodified from the first draft.

Creating a graphics follows the same pattern:
\begin{itemize}
\item
  Decide on what the graphic should communicate. Make this a conscious
  decision, that is, determine ``What is the graphic supposed to tell
  the reader?''
\item
  Create an ``outline,'' that is, the rough overall ``shape'' of the
  graphic, containing the most crucial elements. Often, it is
  useful to do this using pencil and paper.
\item
  Fill out the finer details of the graphic to create a first
  draft.
\item
  Revise the graphic repeatedly along with the rest of the paper.
\end{itemize}




\subsection{Linking Graphics With the Main Text}

Graphics can be placed at different places in a text. Either, they can
be inlined, meaning they are somewhere ``in the middle of the text''
or they can be placed in standalone ``figures.'' Since printers (the
people) like to have their pages ``filled,'' (both for aesthetic and
economic reasons) standalone figures may traditionally be placed on
pages in the document far removed from the main text that refers to
them. \LaTeX\ and \TeX\ tend to encourage this ``drifting away'' of
graphics for technical reasons. 

When a graphic is inlined, it will more or less automatically be
linked with the main text in the sense that the labels of the graphic
will be implicitly explained by the surrounding text. Also, the main
text will typically make it clear what the graphic is about and what
is shown.

Quite differently, a standalone figure will often be viewed at a time
when the main text that this graphic belongs to either has not yet
been read or has been read some time ago. For this reason, you should
follow the following guidelines when creating standalone figures:
\begin{itemize}
\item
  Standalone figures should have a caption than should make them
  ``understandable by themselves.''

  For example, suppose a graphic shows an example of the different
  stages of a quicksort algorithm. Then the figure's caption should,
  at the very least, inform the reader that ``The figure shows the
  different stages of the quicksort algorithm introduced on page
  xyz.'' and not just ``Quicksort algorithm.''
\item
  A good caption adds as much context information as possible. For
  example, you could say: ``The figure shows the different stages of
  the quicksort algorithm introduced on page xyz. In the first line,
  the pivot element 5 is chosen. This causes\dots'' While this
  information can also be given in the main text, putting it in the
  caption will ensure that the context is kept. Do not feel afraid of
  a 5-line caption. (Your editor may hate you for this. Consider
  hating them back.)
\item
  Reference the graphic in your main text as in ``For an example of
  quicksort `in action,' see Figure~2.1 on page xyz.''
\item
  Most books on style and typography recommend that you do not use
  abbreviations as in ``Fig.~2.1'' but write ``Figure 2.1.''

  The main argument against abbreviations is that ``a period is too
  valuable to waste it on an abbreviation.'' The idea is that a period
  will make the reader assume that the sentence ends after ``Fig'' and
  it takes a ``conscious backtracking'' to realize that the sentence
  did not end after all.

  The argument in favor of abbreviations is that they save space.
  
  Personally, I am not really convinced by either argument. On the one
  hand, I have not yet seen any hard evidence that abbreviations slow 
  readers down. On the other hand,  abbreviating all ``Figure'' by
  ``Fig.''\ is most unlikely to save even a single line in  most
  documents.  

  I avoid abbreviations.
\end{itemize}



\subsection{Consistency Between Graphics and Text}

Perhaps the most common ``mistake'' people do when creating graphics
(remember that a ``mistake'' in design is always just ``ignorance'')
is to have a mismatch between the way their graphics look and the way 
their text looks.

It is quite common that authors use several different programs for
creating the graphics of a paper. An author might produce some plots
using \textsc{gnuplot}, a diagram using \textsc{xfig}, and include an
|.eps| graphic a coauthor contributed using some unknown program. All
these graphics will, most likely, use different line widths, different
fonts, and have different sizes. In addition, authors often use
options like |[height=5cm]| when including graphics to scale them to
some ``nice size.''

If the same approach were taken to writing the main text, every
section would be written in a different font at a different size. In
some sections all theorems would be underlined, in another they would
be printed all in uppercase letters, and in another in red. In
addition, the margins would be different on each page.

Readers and editors would not tolerate a text if it were written in
this fashion, but with graphics they often have to.

To create consistency between graphics and text, stick to the
following guidelines:
\begin{itemize}
\item
  Do not scale graphics.

  This means that when generating graphics using an external program,
  create them ``at the right size.''
\item
  Use the same font(s) both in graphics and the body text.
\item
  Use the same line width in text and graphics.

  The  ``line width'' for normal text is the width of the stem of
  letters like T{}. For \TeX, this is usually
  $0.4\,\mathrm{pt}$. However, some journals will not accept graphics
  with a normal line width below $0.5\,\mathrm{pt}$.
\item
  When using colors, use a consistent color coding in the text and in  
  graphics. For example, if red is supposed to alert the reader to
  something in the main text, use red also in graphics for important
  parts of the graphic. If blue is used for structural elements like 
  headlines and section titles, use blue also for structural elements
  of your graphic.

  However, graphics may also use a logical intrinsic color
  coding. For example, no matter what colors you normally use, readers
  will generally assume, say, that the color green as ``positive, go,
  ok'' and red as ``alert, warning, action.''
\end{itemize}

Creating consistency when using different graphic programs is almost
impossible. For this reason, you should consider sticking to a single
graphic program.


\subsection{Labels in Graphics}

Almost all graphics will contain labels, that is, pieces of text that
explain parts of the graphics. When placing labels, stick to the
following guidelines:

\begin{itemize}
\item
  Follow the rule of consistency when placing labels. You should do
  so in two ways: First, be consistent with the main text, that is,
  use the same font as the main text also for labels. Second, be
  consistent between labels, that is, if you format some labels in
  some particular way, format all labels in this way.
\item
  In addition to using the same fonts in text and graphics, you should
  also use the same notation. For example, if you write $1/2$ in your
  main text, also use ``$1/2$'' as labels in graphics, not
  ``0.5''. A $\pi$ is a ``$\pi$'' and not ``$3.141$''. Finally,
  $\mathrm e^{-\mathrm i \pi}$ is ``$\mathrm e^{-\mathrm i \pi}$'',
  not ``$-1$'', let alone ``-1''. 
\item
  Labels should be legible. They should not only have a reasonably
  large size, they also should not be obscured by lines or other
  text. This also applies to of lines and text \emph{behind} the
  labels.
\item
  Labels should be ``in  place.'' Whenever there is enough space,
  labels should be placed next to the thing they label. Only if
  necessary, add a (subdued) line from the label to the labeled
  object. Try to avoid labels that only reference explanations in
  external legends. Reader have to jump back and forth between the
  explanation and the object that is described. 
\item
  Consider subduing ``unimportant'' labels using, for example, a gray
  color. This will keep the focus on the actual graphic.
\end{itemize}



\subsection{Plots and Charts}

One of the most frequent kind of graphics, especially in scientific
papers, are \emph{plots}. They come in a large variety, including
simple line plots, parametric plots, three dimensional plots, pie
charts, and many more.

Unfortunately, plots are notoriously hard to get right. Partly, the
default settings of programs like \textsc{gnuplot} or Excel are to
blame for this since these programs make it very convenient to create
bad plots.

The first question you should ask yourself when creating a plot is the
following:
\begin{itemize}
\item
  Are there enough data points to merit a plot?
\end{itemize}

If the answer is ``not really,'' use a table.

A typical situation where a plot is unnecessary is when people present
a few numbers in a bar diagram. Here is a real-life example: At the
end of a seminar a lecturer asked the participants for feedback. Of
the 50 participants, 30 returned the feedback form. According to the
feedback, three participants considered the seminar ``very good,''
nine considered it  ``good,'' ten ``ok,'' eight ``bad,'' and no one thought 
that the seminar was ``very bad.''

A simple way of summing up this information is the following table:

\medskip
\begin{tabular}{lp{3.75cm}r}
  \emph{Rating given} & \raggedright\emph{Participants (out of 50) who gave this rating} &
  \emph{Percentage} \\[1.75em]
  ``very good'' & \hfil\hphantom{0}3\hfil & \hphantom{0}6\% \\
  ``good'' & \hfil\hphantom{0}9\hfil & 18\% \\
  ``ok'' & \hfil10\hfil & 20\% \\
  ``bad'' & \hfil\hphantom{0}8\hfil & 16\% \\
  ``very bad'' & \hfil\hphantom{0}0\hfil & \hphantom{0}0\% \\[2mm]
  none & \hfil20\hfil & 40\% \\
\end{tabular}

\bigskip
What the lecturer did was to visualize the data using a 3D bar
diagram. It looked like this:

\bigskip
\par
\begin{tikzpicture}[y=0.03cm,z=3mm]
  \foreach \y in {0,20,40,60,80,100}
    \draw[dashed] (0,\y,0) node[left] {\y} -- (0,\y,1)  -- (6,\y,1);

  \draw (0,0,0) -- (0,100,0)  (0,0,1) -- (0,100,1);
  \draw (0,0,0) -- (6,0,0);

  \foreach \x/\xtext/\height in {1/very good/10,2/good/30,3/ok/33,4/bad/27,5/very bad/0}
  {
    \draw (\x,0) node[rotate=90,anchor=east] {\xtext};

    \begin{scope}[xshift=\x cm]
      
    \filldraw[fill=blue!50] (-.3,0,0) rectangle (.3,\height,0);
    \filldraw[fill=blue!30] (.3,0,0) -- (.3,0,1) -- (.3,\height,1) -- (.3,\height,0) --cycle;
    \filldraw[fill=blue!20] (-.3,\height,0) -- (.3,\height,0) --
    (.3,\height,1) -- (-.3,\height,1) --cycle;
    \end{scope}
  }
\end{tikzpicture}
\bigskip

Both the table and the ``plot'' have about the same size. If your first
thought is ``the graphic looks nicer than the table,'' try to answer
the following questions based on the information in the table or in
the graphic: 
\begin{enumerate}
\item
  How many participants where there?
\item
  How many participants returned the feedback form?
\item
  What percentage of the participants returned the feedback form?
\item
  How many participants checked ``very good''?
\item
  What percentage out of all participants checked ``very good''?
\item
  Did more than a quarter of the participants check ``bad'' or ``very bad''?
\item
  What percentage of the participants that returned the form checked ``very good''?
\end{enumerate}

Sadly, the graphic does not allow us to answer \emph{a single one of these
  questions}. The table answers all of them directly, except for the last
one. In essence, the information density of the graphic is very
nearly zero. The table has a much higher information density; despite
the fact that it uses quite a lot of white space to present a few numbers.

Here is the list of things that went wrong with the 3D-bar diagram:
\begin{itemize}
\item
  The whole graphic is dominated by irritating background lines.
\item
  It is not clear what the numbers at the left mean; presumably
  percentages, but it might also be the absolute number of
  participants.
\item
  The labels at the bottom are rotated, making them hard to read.

  (In the real presentation that I saw, the text was rendered at a very 
  low resolution with about 10 by 6 pixels per letter with wrong
  kerning, making the rotated text almost impossible to read.)
\item
  The third dimension adds complexity to the graphic without adding
  information.
\item
  The three dimensional setup makes it much harder to gauge the height
  of the bars correctly. Consider the ``bad'' bar. It the number this
  bar stands for more than 20 or less? While the front of the bar is
  below the 20 line, the back of the bar (which counts) is above.
\item
  It is impossible to tell which  numbers are represented by the
  bars. Thus, the bars needlessly hide the information these bars are
  all about.
\item
  What do the bar heights add up to? Is it 100\% or 60\%?
\item
  Does the bar for ``very bad'' represent 0 or~1?
\item
  Why are the bars blue?
\end{itemize}

You might argue that in the example the exact numbers are not
important for the graphic. The important things is the ``message,''
which is that there are more ``very good'' and ``good'' ratings than
``bad'' and ``very bad.'' However, to convey this message either use a
sentence that says so or use a graphic that conveys this message more
clearly:  

\medskip
\par
\begin{tikzpicture}
  \colorlet{good}{green!75!black}
  \colorlet{bad}{red}
  \colorlet{neutral}{black!60}
  \colorlet{none}{white}

  \node[text centered,text width=3cm]{Ratings given by 50~participants};

  \begin{scope}[line width=4mm,rotate=270]
    \draw[good]          (-123:2cm) arc (-123:-101:2cm);
    \draw[good!60!white] (-36:2cm) arc (-36:-101:2cm);
    \draw[neutral]       (-36:2cm) arc (-36:36:2cm);
    \draw[bad!60!white]  (36:2cm)  arc (36:93:2cm);

    \newcount\mycount
    \foreach \angle in {0,72,...,3599}
    {
      \mycount=\angle\relax
      \divide\mycount by 10\relax
      \draw[black!15,thick] (\the\mycount:18mm) -- (\the\mycount:22mm);
    }
    
    \draw (0:2.2cm) node[below] {``ok'': 10 (20\%)};
    \draw (165:2.2cm) node[above] {none: 20 (40\%)};
    \draw (-111:2.2cm) node[left] {``very good'': 3 (6\%)};
    \draw (-68:2.2cm) node[left] {``good'': 9 (18\%)};
    \draw (65:2.2cm) node[right] {``bad'': 8 (16\%)};
    \draw (93:2.2cm) node[right] {``very bad'': 0 (0\%)};
  \end{scope}  
  \draw[gray] (0,0) circle (2.2cm) circle (1.8cm);
\end{tikzpicture}

\bigskip
The above graphic has about the same information density as the table
(about the same size and the same numbers are shown). In addition, one
can directly ``see'' that there are more good or very good ratings
than bad ones. One can also ``see'' that the number of people who gave
no rating at all is not negligible, which is quite common for feedback
forms. 

Charts are not always a good idea. Let us look at an example
that I redrew from a pie chart in \emph{Die Zeit}, June 4th, 2005:

\bigskip
\par
\begin{tikzpicture}
  \begin{scope}[xscale=3.2,yscale=1.2]

    \sffamily
    \coordinate (right border) at (2.0cm,-1.7cm);
    \coordinate (left border)  at (-2.5cm,2.1cm);

    \fill[black!25] ([xshift=-2mm,yshift=1.1cm]left border) rectangle ([xshift=2mm,yshift=-.3cm]right border);

    \node[below right,text width=10cm,inner sep=0pt] at ([yshift=.9cm,xshift=-1mm]left border)
    { {\color{black!75} \Large Kohle ist am wichtigsten}\\
      Energiemix bei der deutschen Stromerzeugung 2004};

    \filldraw[draw=gray,fill=white] ([xshift=-1mm]left border) node[below right,black]
      {\footnotesize Gesamte Netto-Stromerzeugung in Prozent, in
        Milliarden Kilowattstunden (Mrd.\ kWh)}
      rectangle ([xshift=1mm]right border);
    
    % The 3D stuff
    \pgfdeclarehorizontalshading{zeit}{100bp}
    {color(0pt)=(black);
      color(25bp)=(black);
      color(37bp)=(white);
      color(50bp)=(black);
      color(62bp)=(white);
      color(75bp)=(black);
      color(100bp)=(black)}

    \shadedraw[very thin,shading=zeit,yshift=-1.5mm] (0,0) circle (1cm);

    \fill[green!20!gray]   (0,0) -- (90:1cm) arc (90:-5:1cm);
    \fill[white!20!gray]   (0,0) -- (-5:1cm) arc (-5:-105:1cm);
    \fill[orange!20!gray]  (0,0) -- (-105:1cm) arc (-105:-180:1cm);
    \fill[orange!60!white] (0,0) -- (180:1cm) arc (180:150:1cm);
    \fill[black!75!white]  (0,0) -- (150:1cm) arc (150:145:1cm);
    \fill[blue!90!white]   (0,0) -- (145:1cm) arc (145:135:1cm);
    \fill[blue!50!white]   (0,0) -- (135:1cm) arc (135:92:1cm);
    \fill[yellow!50!black] (0,0) -- (92:1cm) arc (92:90:1cm);

    \begin{scope}[very thin]
      \draw (0,0) -- (90:1cm);
      \draw (0,0) -- (-5:1cm);
      \draw (0,0) -- (-105:1cm);
      \draw (0,0) -- (-180:1cm);
      \draw (0,0) -- (150:1cm);
      \draw (0,0) -- (145:1cm);
      \draw (0,0) -- (135:1cm);
      \draw (0,0) -- (92:1cm);
      
      \draw(0,0) circle (1cm);
    \end{scope}

    \node (Regenerative)   at (115:.75cm)  {\bfseries 9,4\%};
    \node (Kernenergie)    at (30:.5cm)   {\bfseries 27,8\%};
    \node (Braunkohle)     at (-45:.6cm)  {\bfseries 25,6\%};
    \node (Steinkohle)     at (-135:.6cm) {\bfseries 22,3\%};
    \node (Erdgas)         at (168:.75cm) {\bfseries 10,4\%};
    \coordinate (Mineral)  at (147:.9cm);
    \coordinate (Sonstige) at (140:.9cm);

    \small
    \draw (Regenerative.north) |- ([yshift=.25cm]Regenerative.north -| right border) coordinate (Regenerative label);
    \draw (91:.9cm) |- (Regenerative label);
    \node[above left] at (Regenerative label) {Regenerative\
      {\footnotesize (53,7 kWh)/davon} Wind \textbf{4,4\%}  \footnotesize (25,0 kWh)};

    \draw (Kernenergie.base east) -- (Kernenergie.base east -| right border) coordinate (Kernenergie label);
    \node[above left] at (Kernenergie label) {Kernenergie};
    \node[below left] at (Kernenergie label) {\footnotesize (158,4 kWh)};

    \draw (Braunkohle.south) |- ([yshift=-.75cm]Braunkohle.south -| right border) coordinate (Braunkohle label);
    \node[above left] at (Braunkohle label) {Braunkohle\ \ \footnotesize (146,0 kWh)};

    \draw (Steinkohle.south) |- ([yshift=-.75cm]Steinkohle.south -| left border) coordinate (Steinkohle label);
    \node[above right] at (Steinkohle label) {Steinkohle\ \ \footnotesize (127,1 kWh)};

    \draw (Erdgas.base west) -- (Erdgas.base west -| left border) coordinate (Erdgas label);
    \node[above right] at (Erdgas label) {Erdgas\ \ \footnotesize (59,2 kWh)};

    \draw (Mineral) -- (Mineral -| left border) coordinate (Mineral label);
    \node[above right] at (Mineral label) {Mineral\"olprodukte\ \
      \footnotesize (9,2 kWh) \  \ \normalsize\textbf{1,6\%}};

    \draw (Sonstige) |- (Regenerative label -| left border) coordinate (Sonstige label);
    \node[above right] at (Sonstige label) {Sonstige\ \
      \footnotesize (16,5 kWh) \hskip1.5cm\
      \normalsize\textbf{2,9\%}};
  \end{scope}    
\end{tikzpicture}

This graphic has been redrawn in \tikzname, but the original looks very
similar.

At first sight, the graphic looks  ``nice and informative,'' but there
are a lot of things that went wrong:

\begin{itemize}
\item
  The chart is three dimensional. However, the shadings add
  nothing ``information-wise,'' at best, they distract.
\item
  In a 3D-pie-chart the relative sizes are very strongly
  distorted. For example, the area taken up by the gray color of ``Braunkohle''
  is larger than the area taken up by the green color of
  ``Kernenergie'' \emph{despite the fact that the percentage of
    Braunkohle is less than the percentage of Kernenergie}.
\item
  The 3D-distortion gets worse for small areas. The area of
  ``Regenerative'' somewhat larger  than the area of ``Erdgas.''  
  The area of ``Wind'' is slightly smaller than the area of
  ``Mineral\"olprodukte'' \emph{although the percentage of Wind is
    nearly three times larger than the percentage of
    Mineral\"olprodukte.}

  In the last case, the different sizes are only partly due to
  distortion. The designer(s) of the original graphic have also made
  the ``Wind'' slice too small, even taking distortion into
  account. (Just compare the size of ``Wind'' to ``Regenerative'' in
  general.) 
\item
  According to its caption, this chart is supposed to inform us that
  coal was the most important energy source in Germany in
  2004. Ignoring the strong distortions caused by the superfluous and
  misleading 3D-setup, it takes quite a while for this message to get
  across. 

  Coal as an energy source is split up into two slices: one for
  ``Steinkohle'' and one for ``Braunkohle'' (two different kinds of
  coal). When you add them up, you see that the whole lower half of
  the pie chart is taken up by coal.

  The two areas for the different kinds of coal are not visually
  linked at all. Rather, two different colors are used, the labels are
  on different sides of the graphic. By comparison, ``Regenerative''
  and ``Wind'' are very closely linked.
\item
  The color coding of the graphic follows no logical pattern at
  all. Why is nuclear energy green? Regenerative energy is light blue,
  ``other sources'' are blue. It seems more like a joke that the area
  for ``Braunkohle'' (which literally translates to ``brown coal'') is
  stone gray, while the area for ``Steinkohle'' (which literally
  translates to ``stone coal'') is brown.
\item
  The area with the lightest color is used for ``Erdgas.'' This area
  stands out most because of the brighter color. However, for this
  chart ``Erdgas'' is not really important at all.
\end{itemize}
Edward Tufte calls graphics like the above ``chart junk.'' 

Here are a few recommendations that may help you avoid producing chart junk:
\begin{itemize}
\item
  Do not use 3D pie charts. They are \emph{evil}.
\item
  Consider using a table instead of a pie chart.
\item
  Due not apply colors randomly; use them to direct the readers's 
  focus and to group things.
\item
  Do not use background patterns, like a crosshatch or diagonal
  lines, instead of colors. They distract. Background patterns in
  information graphics are \emph{evil}.
\end{itemize}



\subsection{Attention and Distraction}

Pick up your favorite fiction novel and have a look at a typical
page. You will notice that the page is very uniform. Nothing is there
to distract the reader while reading; no large headlines, no bold
text, no large white areas. Indeed, even when the author does wish to
emphasize something, this is done using italic letters. Such letters
blend nicely with the main text---at a distance you will not be able to
tell whether a page contains italic letters, but you would notice a
single bold word immediately. The reason novels are typeset this way
is the following paradigm: Avoid distractions.

Good typography (like good organization) is something you do
\emph{not} notice. The job of typography is to make reading the text,
that is, ``absorbing'' its information content, as effortless as
possible. For a novel, readers absorb the content by reading the text
line-by-line, as if they were listening to someone telling the
story. In this situation anything on the page that distracts the eye
from  going quickly and evenly from line to line will make the text
harder to read.

Now, pick up your favorite weekly magazine or newspaper and have a
look at a typical 
page. You will notice that there is quite a lot ``going on'' on the
page. Fonts are used at different sizes and in different arrangements,
the text is organized in narrow columns, typically interleaved with
pictures. The reason magazines are typeset in this way is another
paradigm: Steer attention.

Readers will not read a magazine like a novel. Instead of reading a
magazine line-by-line, we use headlines and short abstracts to check
whether we want to read a certain article or not. The job of
typography is to steer our attention to these abstracts and headlines,
first. Once we have decided that we want to read an article, however,
we no longer tolerate distractions, which is why the main text of
articles is typeset exactly the same way as a novel.

The two principles ``avoid distractions'' and ``steer attention'' also
apply to graphics. When you design a graphic, you should eliminate
everything that will ``distract the eye.'' At the same time, you
should try to actively help the reader ``through the graphic'' by
using fonts/colors/line widths to highlight different parts.

Here is a non-exhaustive list of things that can distract readers:
\begin{itemize}
\item
  Strong contrasts will always be registered first by the eye. For
  example, consider the following two grids:

  \medskip\par
  \begin{tikzpicture}[x=40pt,y=40pt]
    \draw[step=10pt,gray] (0,0) grid +(1,1);
    \draw[step=2pt]      (2,0) grid +(1,1);
  \end{tikzpicture}

  \medskip
  Even though the left grid comes first in our normal reading order,
  the right one is much more likely to be seen first: The
  white-to-black contrast is higher than the gray-to-white
  contrast. In addition, there are more ``places'' adding to the
  overall contrast in the right grid.

  Things like grids and, more generally, help lines usually should not
  grab the attention of the readers and, hence, should be typeset with
  a low contrast to the background. Also, a loosely-spaced grid is
  less distracting than a very closely-spaced grid.
\item
  Dashed lines create many points at which there is black-to-white
  contrast. Dashed or dotted lines can be very distracting and, hence,
  should be avoided in general.

  Do not use different dashing patterns to differentiate curves in
  plots. You loose data points this way and the eye is not
  particularly good at ``grouping things according to a dashing
  pattern.'' The eye is \emph{much} better at grouping things
  according to colors.
\item
  Background patterns filling an area using  diagonal lines or
  horizontal and vertical lines or just dots are almost always
  distracting and, usually, serve no real purpose.
\item
  Background images and shadings distract and only seldom add
  anything of importance to a graphic.
\item
  Cute little cliparts can easily draw attention away from the
  data.
\end{itemize}


% Copyright 2005 by Till Tantau <tantau@cs.tu-berlin.de>.
%
% This program can be redistributed and/or modified under the terms
% of the LaTeX Project Public License Distributed from CTAN
% archives in directory macros/latex/base/lppl.txt.


\section{Input and Output Formats}

\TeX\ was designed to be a flexible system. This is true both for the
\emph{input} for \TeX\ as well as for the \emph{output}. The present
section explains which input formats there are and how they are
supported by \pgfname. It also explains which different output formats
can be produced.



\subsection{Supported Input Formats}

\TeX\ does not prescribe exactly how your input should be
formatted. While it is \emph{customary} that, say, an opening brace
starts a scope in \TeX, this is by no means necessary. Likewise, it is
\emph{customary} that environments start with |\begin|, but \TeX\
could not really care less about the exact command name.

Even though \TeX\ can be reconfigured, users can not. For this reason,
certain \emph{input formats} specify a set of commands and conventions
how input for \TeX\ should be formatted. There are currently three
``major'' formats: Donald Knuth's original |plain| \TeX\ format,
Leslie Lamport's popular \LaTeX\ format, and Hans Hangen's Con\TeX t
format.


\subsubsection{Using the  \LaTeX\ Format}

Using \pgfname\ and \tikzname\ with the \LaTeX\ format is easy: You
say |\usepackage{pgf}| or |\usepackage{tikz}|. Usually, that is all
you need to do, all configuration will be done automatically and
(hopefully) correctly.

The style files used for the \LaTeX\ format reside in the subdirectory
|latex/pgf/| of the \pgfname-system. Mainly, what these files do is to
include files in the directory |generic/pgf|. For example, here is the
content of the file |latex/pgf/frontends/tikz.sty|:
\begin{verbatim}
% Copyright 2005 by Till Tantau <tantau@users.sourceforge.net>.
%
% This program can be redistributed and/or modified under the terms
% of the GNU Public License, version 2.

\RequirePackage{pgf,calc,pgffor,pgflibraryplothandlers,xkeyval}

% Copyright 2006 by Till Tantau
%
% This file may be distributed and/or modified
%
% 1. under the LaTeX Project Public License and/or
% 2. under the GNU Public License.
%
% See the file doc/generic/pgf/licenses/LICENSE for more details.

\ProvidesPackageRCS[v\pgfversion] $Header: /cvsroot/pgf/pgf/generic/pgf/frontendlayer/tikz.code.tex,v 1.94 2007/07/31 08:21:18 tantau Exp $


% Always-present libraries:

\usepgflibrary{plothandlers}

% TikZ is a key family
\pgfkeys{/tikz/.is family}

\def\tikzset{\pgfqkeys{/tikz}}


\newdimen\tikz@lastx
\newdimen\tikz@lasty
\newdimen\tikz@lastxsaved
\newdimen\tikz@lastysaved

\newdimen\tikzleveldistance
\newdimen\tikzsiblingdistance

\newbox\tikz@figbox
\newbox\tikz@tempbox

\newcount\tikztreelevel
\newcount\tikznumberofchildren
\newcount\tikznumberofcurrentchild

\newcount\tikz@fig@count

\newif\iftikz@node@is@a@label
\newif\iftikz@snaked

\let\tikz@options=\pgfutil@empty
\def\tikz@addoption#1{\expandafter\def\expandafter\tikz@options\expandafter{\tikz@options#1}}
\def\tikz@addmode#1{\expandafter\def\expandafter\tikz@mode\expandafter{\tikz@mode#1}}
\def\tikz@addtransform#1{%
  \ifx\tikz@transform\relax%
    #1%
  \else%
    \expandafter\def\expandafter\tikz@transform\expandafter{\tikz@transform#1}%
  \fi%
}



% TikZ options:

% This command is supported for compatibility only:

\def\tikzoption#1{\pgfutil@ifnextchar[{\tikzoption@opt{#1}}{\tikzoption@noopt{#1}}}%}

\def\tikzoption@opt#1[#2]#3{\pgfkeysdef{/tikz/#1}{#3}\pgfkeyssetvalue{/tikz/#1/.@def}{#2}}
\def\tikzoption@noopt#1#2{\pgfkeysdef{/tikz/#1}{#2}\pgfkeyssetvalue{/tikz/#1/.@def}{\pgfkeysvaluerequired}}

% Baseline options
\tikzoption{baseline}[0pt]{\pgfutil@ifnextchar({\tikz@baseline@coordinate}{\tikz@baseline@simple}#1\@nil}%)
\def\tikz@baseline@simple#1\@nil{\pgfsetbaseline{#1}}
\def\tikz@baseline@coordinate#1\@nil{\pgfsetbaselinepointlater{\tikz@scan@one@point\@firstofone#1}}

% Draw options
\tikzoption{line width}{\tikz@semiaddlinewidth{#1}}%

\def\tikz@semiaddlinewidth#1{\tikz@addoption{\pgfsetlinewidth{#1}}\pgfmathsetlength\pgflinewidth{#1}}

\tikzoption{cap}{\tikz@addoption{\csname pgfset#1cap\endcsname}}
\tikzoption{join}{\tikz@addoption{\csname pgfset#1join\endcsname}}
\tikzoption{miter limit}{\tikz@addoption{\pgfsetmiterlimit{#1}}}

\tikzoption{dash pattern}{% syntax: on 2pt off 3pt on 4pt ...
  \def\tikz@temp{#1}%
  \ifx\tikz@temp\pgfutil@empty%
    \def\tikz@dashpattern{}%
    \tikz@addoption{\pgfsetdash{}{0pt}}%
  \else%
    \def\tikz@dashpattern{}%
    \expandafter\tikz@scandashon\pgfutil@gobble#1o\@nil%
    \edef\tikz@temp{{\tikz@dashpattern}{\noexpand\tikz@dashphase}}%
    \expandafter\tikz@addoption\expandafter{\expandafter\pgfsetdash\tikz@temp}%
  \fi}
\tikzoption{dash phase}{%
  \def\tikz@dashphase{#1}%
  \edef\tikz@temp{{\tikz@dashpattern}{\noexpand\tikz@dashphase}}%
  \expandafter\tikz@addoption\expandafter{\expandafter\pgfsetdash\tikz@temp}%
}%
\def\tikz@dashphase{0pt}

\def\tikz@scandashon n#1o{%
  \expandafter\def\expandafter\tikz@dashpattern\expandafter{\tikz@dashpattern{#1}}%
  \pgfutil@ifnextchar\@nil{\pgfutil@gobble}{\tikz@scandashoff}}
\def\tikz@scandashoff ff#1o{%
  \expandafter\def\expandafter\tikz@dashpattern\expandafter{\tikz@dashpattern{#1}}%
  \pgfutil@ifnextchar\@nil{\pgfutil@gobble}{\tikz@scandashon}}

\tikzoption{draw opacity}{\tikz@addoption{\pgfsetstrokeopacity{#1}}}

% Double draw options
\tikzoption{double}[]{%
  \def\tikz@temp{#1}%
  \ifx\tikz@temp\tikz@nonetext%
    \tikz@addmode{\tikz@mode@doublefalse}%
  \else%
    \ifx\tikz@temp\pgfutil@empty%
    \else%
      \def\tikz@double@color{#1}%
    \fi%
    \tikz@addmode{\tikz@mode@doubletrue}%
  \fi}
\tikzoption{double distance}{%
  \pgfmathsetlength{\pgf@x}{#1}%
  \edef\tikz@double@width@distance{\the\pgf@x}%
  \tikz@addmode{\tikz@mode@doubletrue}}

\def\tikz@double@width@distance{0.6pt}
\def\tikz@double@color{white}

% Fill options

\tikzoption{even odd rule}[]{\tikz@addoption{\pgfseteorule}}
\tikzoption{nonzero rule}[]{\tikz@addoption{\pgfsetnonzerorule}}

\tikzoption{fill opacity}{\tikz@addoption{\pgfsetfillopacity{#1}}}


% Joined fill/draw options

\tikzoption{opacity}{\tikz@addoption{\pgfsetstrokeopacity{#1}\pgfsetfillopacity{#1}}}


% Main color options
\tikzoption{color}{%
  \tikz@addoption{%
    \ifx\tikz@fillcolor\pgfutil@empty%
      \ifx\tikz@strokecolor\pgfutil@empty%
      \else%
        \pgfsys@color@reset@inorderfalse%
        \let\tikz@strokecolor\pgfutil@empty%
        \let\tikz@fillcolor\pgfutil@empty%
      \fi%
    \else%
      \pgfsys@color@reset@inorderfalse%
      \let\tikz@strokecolor\pgfutil@empty%
      \let\tikz@fillcolor\pgfutil@empty%
    \fi%
    \pgfutil@colorlet{tikz@color}{#1}%
    \pgfutil@colorlet{.}{tikz@color}%
    \pgfsetcolor{.}%
    \pgfsys@color@reset@inordertrue%
  }%
  \def\tikz@textcolor{#1}}



% Rounding options
\tikzoption{rounded corners}[4pt]{\pgfsetcornersarced{\pgfpoint{#1}{#1}}}
\tikzoption{sharp corners}[]{\pgfsetcornersarced{\pgfpointorigin}}



% Coordinate options
\tikzoption{x}{\tikz@handle@vec{\pgfsetxvec}{\tikz@handle@x}#1\relax}
\tikzoption{y}{\tikz@handle@vec{\pgfsetyvec}{\tikz@handle@y}#1\relax}
\tikzoption{z}{\tikz@handle@vec{\pgfsetzvec}{\tikz@handle@z}#1\relax}

\def\tikz@handle@vec#1#2{\pgfutil@ifnextchar({\tikz@handle@coordinate#1}{\tikz@handle@single#2}}
\def\tikz@handle@coordinate#1{\tikz@scan@one@point#1}
\def\tikz@handle@single#1#2\relax{#1{#2}}
\def\tikz@handle@x#1{\pgfsetxvec{\pgfpoint{#1}{0pt}}}
\def\tikz@handle@y#1{\pgfsetyvec{\pgfpoint{0pt}{#1}}}
\def\tikz@handle@z#1{\pgfsetzvec{\pgfpoint{#1}{#1}}}


% Transformation options
\tikzoption{scale}{\tikz@addtransform{\pgftransformscale{#1}}}
\tikzoption{xscale}{\tikz@addtransform{\pgftransformxscale{#1}}}
\tikzoption{xslant}{\tikz@addtransform{\pgftransformxslant{#1}}}
\tikzoption{yscale}{\tikz@addtransform{\pgftransformyscale{#1}}}
\tikzoption{yslant}{\tikz@addtransform{\pgftransformyslant{#1}}}
\tikzoption{rotate}{\tikz@addtransform{\pgftransformrotate{#1}}}
\tikzoption{rotate around}{\tikz@addtransform{\tikz@rotatearound{#1}}}
\def\tikz@rotatearound#1{%
  \edef\tikz@temp{#1}% get rid of active stuff
  \expandafter\tikz@rotateparseA\tikz@temp%
}%
\def\tikz@rotateparseA#1:{%
  \def\tikz@temp@rot{#1}%
  \tikz@scan@one@point\tikz@rotateparseB%
}
\def\tikz@rotateparseB#1{%
  \pgf@process{#1}%
  \pgf@xc=\pgf@x%
  \pgf@yc=\pgf@y%
  \pgftransformshift{\pgfqpoint{\pgf@xc}{\pgf@yc}}%
  \pgftransformrotate{\tikz@temp@rot}%
  \pgftransformshift{\pgfqpoint{-\pgf@xc}{-\pgf@yc}}%
}

\tikzoption{shift}{\tikz@addtransform{\tikz@scan@one@point\pgftransformshift#1\relax}}
\tikzoption{xshift}{\tikz@addtransform{\pgftransformxshift{#1}}}
\tikzoption{yshift}{\tikz@addtransform{\pgftransformyshift{#1}}}
\tikzoption{cm}{\tikz@addtransform{\tikz@parse@cm#1\relax}}
\tikzoption{reset cm}[]{\tikz@addtransform{\pgftransformreset}}
\tikzoption{shift only}[]{\tikz@addtransform{\pgftransformresetnontranslations}}

\def\tikz@parse@cm#1,#2,#3,#4,{%
  \def\tikz@p@cm{{#1}{#2}{#3}{#4}}%
  \tikz@scan@one@point\tikz@parse@cmA}
\def\tikz@parse@cmA#1{%
  \expandafter\pgftransformcm\tikz@p@cm{#1}%
}



% Grid options
\tikzoption{xstep}{\def\tikz@grid@x{#1}}
\tikzoption{ystep}{\def\tikz@grid@y{#1}}
\tikzoption{step}{\tikz@handle@vec{\tikz@step@point}{\tikz@step@single}#1\relax}
\def\tikz@step@single#1{\def\tikz@grid@x{#1}\def\tikz@grid@y{#1}}
\def\tikz@step@point#1{\pgf@process{#1}\edef\tikz@grid@x{\the\pgf@x}\edef\tikz@grid@y{\the\pgf@y}}

\def\tikz@grid@x{1cm}
\def\tikz@grid@y{1cm}


% Path usage options
\newif\iftikz@mode@double
\newif\iftikz@mode@fill
\newif\iftikz@mode@draw
\newif\iftikz@mode@clip
\newif\iftikz@mode@boundary
\newif\iftikz@mode@shade
\let\tikz@mode=\pgfutil@empty

\def\tikz@nonetext{none}

\tikzoption{path only}[]{\let\tikz@mode=\pgfutil@empty}
\tikzoption{shade}[]{\tikz@addmode{\tikz@mode@shadetrue}}
\tikzoption{fill}[]{%
  \def\tikz@temp{#1}%
  \ifx\tikz@temp\tikz@nonetext%
    \tikz@addmode{\tikz@mode@fillfalse}%
  \else%
    \ifx\tikz@temp\pgfutil@empty%
    \else%
      \tikz@addoption{\pgfsetfillcolor{#1}}%
      \def\tikz@fillcolor{#1}%
    \fi%
    \tikz@addmode{\tikz@mode@filltrue}%
  \fi%
}
\tikzoption{draw}[]{%
  \def\tikz@temp{#1}%
  \ifx\tikz@temp\tikz@nonetext%
    \tikz@addmode{\tikz@mode@drawfalse}%
  \else%
    \ifx\tikz@temp\pgfutil@empty%
    \else%
      \tikz@addoption{\pgfsetstrokecolor{#1}}%
      \def\tikz@strokecolor{#1}%
    \fi%
    \tikz@addmode{\tikz@mode@drawtrue}%
  \fi%
}
\tikzoption{clip}[]{\tikz@addmode{\tikz@mode@cliptrue}}
\tikzoption{use as bounding box}[]{\tikz@addmode{\tikz@mode@boundarytrue}}

\tikzoption{save path}{\tikz@addmode{\pgfsyssoftpath@getcurrentpath#1\global\let#1=#1}}

\let\tikz@fillcolor=\pgfutil@empty
\let\tikz@strokecolor=\pgfutil@empty


% Pattern options
\tikzoption{pattern color}{\def\tikz@pattern@color{#1}}
\tikzoption{pattern}[]{%
  \def\tikz@temp{#1}%
  \ifx\tikz@temp\tikz@nonetext%
    \tikz@addmode{\tikz@mode@fillfalse}%
  \else%
    \ifx\tikz@temp\pgfutil@empty%
    \else%
      \tikz@addoption{\pgfsetfillpattern{#1}{\tikz@pattern@color}}%
      \def\tikz@pattern{#1}%
    \fi%
    \tikz@addmode{\tikz@mode@filltrue}%
  \fi%
}
\def\tikz@pattern@color{black}
\def\tikz@pattern{dots}


% Shading options
\tikzoption{shading}{\def\tikz@shading{#1}\tikz@addmode{\tikz@mode@shadetrue}}
\tikzoption{shading angle}{\def\tikz@shade@angle{#1}\tikz@addmode{\tikz@mode@shadetrue}}
\tikzoption{top color}{%
  \pgfutil@colorlet{tikz@axis@top}{#1}%
  \pgfutil@colorlet{tikz@axis@middle}{tikz@axis@top!50!tikz@axis@bottom}%
  \def\tikz@shading{axis}\def\tikz@shade@angle{0}\tikz@addmode{\tikz@mode@shadetrue}}
\tikzoption{bottom color}{%
  \pgfutil@colorlet{tikz@axis@bottom}{#1}%
  \pgfutil@colorlet{tikz@axis@middle}{tikz@axis@top!50!tikz@axis@bottom}%
  \def\tikz@shading{axis}\def\tikz@shade@angle{0}\tikz@addmode{\tikz@mode@shadetrue}}
\tikzoption{middle color}{%
  \pgfutil@colorlet{tikz@axis@middle}{#1}%
  \def\tikz@shading{axis}\tikz@addmode{\tikz@mode@shadetrue}}
\tikzoption{left color}{%
  \pgfutil@colorlet{tikz@axis@top}{#1}%
  \pgfutil@colorlet{tikz@axis@middle}{tikz@axis@top!50!tikz@axis@bottom}%
  \def\tikz@shading{axis}\def\tikz@shade@angle{90}\tikz@addmode{\tikz@mode@shadetrue}}
\tikzoption{right color}{%
  \pgfutil@colorlet{tikz@axis@bottom}{#1}%
  \pgfutil@colorlet{tikz@axis@middle}{tikz@axis@top!50!tikz@axis@bottom}%
  \def\tikz@shading{axis}\def\tikz@shade@angle{90}\tikz@addmode{\tikz@mode@shadetrue}}
\tikzoption{ball color}{\pgfutil@colorlet{tikz@ball}{#1}\def\tikz@shading{ball}\tikz@addmode{\tikz@mode@shadetrue}}
\tikzoption{inner color}{\pgfutil@colorlet{tikz@radial@inner}{#1}\def\tikz@shading{radial}\tikz@addmode{\tikz@mode@shadetrue}}
\tikzoption{outer color}{\pgfutil@colorlet{tikz@radial@outer}{#1}\def\tikz@shading{radial}\tikz@addmode{\tikz@mode@shadetrue}}

\def\tikz@shading{axis}
\def\tikz@shade@angle{0}

\pgfdeclareverticalshading[tikz@axis@top,tikz@axis@middle,tikz@axis@bottom]{axis}{100bp}{%
  color(0bp)=(tikz@axis@bottom);
  color(25bp)=(tikz@axis@bottom);
  color(50bp)=(tikz@axis@middle);
  color(75bp)=(tikz@axis@top);
  color(100bp)=(tikz@axis@top)}

\pgfutil@colorlet{tikz@axis@top}{gray}
\pgfutil@colorlet{tikz@axis@middle}{gray!50!white}
\pgfutil@colorlet{tikz@axis@bottom}{white}

\pgfdeclareradialshading[tikz@ball]{ball}{\pgfqpoint{-10bp}{10bp}}{%
 color(0bp)=(tikz@ball!15!white);
 color(9bp)=(tikz@ball!75!white);
 color(18bp)=(tikz@ball!70!black);
 color(25bp)=(tikz@ball!50!black);
 color(50bp)=(black)}

\pgfutil@colorlet{tikz@ball}{blue}

\pgfdeclareradialshading[tikz@radial@inner,tikz@radial@outer]{radial}{\pgfpointorigin}{%
 color(0bp)=(tikz@radial@inner);
 color(25bp)=(tikz@radial@outer);
 color(50bp)=(tikz@radial@outer)}

\pgfutil@colorlet{tikz@radial@inner}{gray}
\pgfutil@colorlet{tikz@radial@outer}{white}


% Pin options
\tikzoption{pin}{\pgfutil@ifnextchar[{\tikz@parse@pin}{\tikz@parse@pin[]}#1\pgf@nil}
\tikzoption{pin distance}{\def\tikz@pin@distance{#1}}
\tikzoption{pin edge}{\def\tikz@pin@edge@style{#1}}

\tikzoption{tikz@pin@post}[]{%
  \tikz@compute@direction{\tikz@label@angle}{\tikz@pin@distance}%
  \global\let\tikz@pin@edge@style@smuggle=\tikz@pin@edge@style%
}
\tikzoption{tikz@pre@pin@edge}[]{\def\pgf@marshal{\tikzstyle{tikz@pin@options}=}
  \expandafter\pgf@marshal\expandafter[\tikz@pin@edge@style@smuggle]%
}

\def\tikz@pin@distance{3ex}
\def\tikz@pin@edge@style{}

\def\tikz@parse@pin[#1]#2:#3\pgf@nil{%
  \tikz@add@after@node@path{\bgroup
    \pgfextra{\let\tikz@save@last@node=\tikzlastnode}%
    node
    [every pin,tikz@label@angle=#2,#1,at=(\tikzlastnode.\tikz@label@angle),%
    after node path={(\tikz@save@last@node) edge[every pin edge,tikz@pre@pin@edge,tikz@pin@options] (\tikzlastnode)},
    tikz@pin@post]
    {#3} \egroup}
}


% Label and pin options

\tikzoption{label}{\pgfutil@ifnextchar[{\tikz@parse@label}{\tikz@parse@label[]}#1\pgf@nil}
\tikzoption{label distance}{\def\tikz@label@distance{#1}}

\tikzoption{tikz@label@angle}{\def\tikz@label@angle{#1}\csname tikz@label@angle@is@#1\endcsname}

\tikzoption{tikz@label@post}[]{\tikz@compute@direction{\tikz@label@angle}{\tikz@label@distance}}

\def\tikz@label@distance{0pt}

\def\tikz@parse@label[#1]#2:#3\pgf@nil{%
  \tikz@add@after@node@path{
    \bgroup
    \pgfextra{\let\tikz@save@last@fig@name=\tikz@last@fig@name}%
    node
    [every label,%
    tikz@label@angle=#2,%
    #1,%
    at=(\tikzlastnode.\tikz@label@angle),tikz@label@post]%
    {#3}%
    \pgfextra{\global\let\tikz@last@fig@name=\tikz@save@last@fig@name}%
    \egroup%
  }
}

\expandafter\def\csname tikz@label@angle@is@right\endcsname{\def\tikz@label@angle{0}}
\expandafter\def\csname tikz@label@angle@is@above right\endcsname{\def\tikz@label@angle{45}}
\expandafter\def\csname tikz@label@angle@is@above\endcsname{\def\tikz@label@angle{90}}
\expandafter\def\csname tikz@label@angle@is@above left\endcsname{\def\tikz@label@angle{135}}
\expandafter\def\csname tikz@label@angle@is@left\endcsname{\def\tikz@label@angle{180}}
\expandafter\def\csname tikz@label@angle@is@below left\endcsname{\def\tikz@label@angle{225}}
\expandafter\def\csname tikz@label@angle@is@below\endcsname{\def\tikz@label@angle{270}}
\expandafter\def\csname tikz@label@angle@is@below right\endcsname{\def\tikz@label@angle{315}}

\def\tikz@compute@direction#1#2{%
  \let\tikz@do@auto@anchor=\relax
  \c@pgf@counta=#1\relax%
  \ifnum\c@pgf@counta<0\relax
    \advance\c@pgf@counta by 360\relax%
  \fi%
  \ifnum\c@pgf@counta>359\relax
    \advance\c@pgf@counta by-360\relax%
  \fi%
  \ifnum\c@pgf@counta<4\relax%
    \def\tikz@anchor{west}%
  \else\ifnum\c@pgf@counta<87\relax%
    \def\tikz@anchor{south west}%
  \else\ifnum\c@pgf@counta<94\relax%
    \def\tikz@anchor{south}%
  \else\ifnum\c@pgf@counta<177\relax%
    \def\tikz@anchor{south east}%
  \else\ifnum\c@pgf@counta<184\relax%
    \def\tikz@anchor{east}%
  \else\ifnum\c@pgf@counta<267\relax%
    \def\tikz@anchor{north east}%
  \else\ifnum\c@pgf@counta<274\relax%
    \def\tikz@anchor{north}%
  \else\ifnum\c@pgf@counta<357\relax%
    \def\tikz@anchor{north west}%
  \else%
    \def\tikz@anchor{west}%
  \fi\fi\fi\fi\fi\fi\fi\fi%
  \tikz@addtransform{\pgftransformshift{\pgfpointpolar{#1}{#2}}}%  
}



% General shape options
\tikzoption{name}{\edef\tikz@fig@name{#1}}

\tikzoption{at}{\tikz@scan@one@point\tikz@set@at#1}
\def\tikz@set@at#1{\def\tikz@node@at{#1}}%

\tikzoption{shape}{\edef\tikz@shape{#1}}

\tikzoption{nodes}{\tikzstyle{every node}+=[#1]}


% These are /pgf options now:

%\tikzoption{inner sep}{\def\pgfshapeinnerxsep{#1}\def\pgfshapeinnerysep{#1}}
%\tikzoption{inner xsep}{\def\pgfshapeinnerxsep{#1}}
%\tikzoption{inner ysep}{\def\pgfshapeinnerysep{#1}}

%\tikzoption{outer sep}{\def\pgfshapeouterxsep{#1}\def\pgfshapeouterysep{#1}}
%\tikzoption{outer xsep}{\def\pgfshapeouterxsep{#1}}
%\tikzoption{outer ysep}{\def\pgfshapeouterysep{#1}}

%\tikzoption{minimum width}{\def\pgfshapeminwidth{#1}}
%\tikzoption{minimum height}{\def\pgfshapeminheight{#1}}
%\tikzoption{minimum size}{\def\pgfshapeminwidth{#1}\def\pgfshapeminheight{#1}}

\tikzoption{aspect}{\pgfsetshapeaspect{#1}}

\tikzoption{after node path}{\tikz@add@after@node@path{#1}}%
\def\tikz@add@after@node@path#1{\expandafter\def\expandafter\tikz@after@node\expandafter{\tikz@after@node#1}}

\def\tikzaddafternodepathoption#1{%
  #1%
  \expandafter\def\expandafter\tikz@afternodepathoptions\expandafter{\tikz@afternodepathoptions#1}}

\let\tikz@afternodepathoptions=\pgfutil@empty

\tikzoption{anchor}{\def\tikz@anchor{#1}\let\tikz@do@auto@anchor=\relax}

\tikzoption{left}[]{\def\tikz@anchor{east}\tikz@possibly@transform{x}{-}{#1}}
\tikzoption{right}[]{\def\tikz@anchor{west}\tikz@possibly@transform{x}{}{#1}}
\tikzoption{above}[]{\def\tikz@anchor{south}\tikz@possibly@transform{y}{}{#1}}
\tikzoption{below}[]{\def\tikz@anchor{north}\tikz@possibly@transform{y}{-}{#1}}
\tikzoption{above left}[]%
  {\def\tikz@anchor{south east}%
    \tikz@possibly@transform{x}{-}{#1}\tikz@possibly@transform{y}{}{#1}}
\tikzoption{above right}[]%
  {\def\tikz@anchor{south west}%
    \tikz@possibly@transform{x}{}{#1}\tikz@possibly@transform{y}{}{#1}}
\tikzoption{below left}[]%
  {\def\tikz@anchor{north east}%
    \tikz@possibly@transform{x}{-}{#1}\tikz@possibly@transform{y}{-}{#1}}
\tikzoption{below right}[]%
  {\def\tikz@anchor{north west}%
    \tikz@possibly@transform{x}{}{#1}\tikz@possibly@transform{y}{-}{#1}}

\tikzoption{node distance}{\def\tikz@node@distance{#1}}
\def\tikz@node@distance{1cm}

\tikzoption{above of}{\tikz@of{#1}{90}}%
\tikzoption{below of}{\tikz@of{#1}{-90}}%
\tikzoption{left of}{\tikz@of{#1}{180}}%
\tikzoption{right of}{\tikz@of{#1}{0}}%
\tikzoption{above left of}{\tikz@of{#1}{135}}%
\tikzoption{below left of}{\tikz@of{#1}{-135}}%
\tikzoption{above right of}{\tikz@of{#1}{45}}%
\tikzoption{below right of}{\tikz@of{#1}{-45}}%

\def\tikz@of#1#2{%
  \def\tikz@anchor{center}%
  \let\tikz@do@auto@anchor=\relax%
  \tikz@addtransform{\pgftransformshift{\pgfpointpolar{#2}{\tikz@node@distance}}}%
  \def\tikz@node@at{\pgfpointanchor{#1}{center}}}
  
\tikzoption{transform shape}[true]{%
  \csname tikz@fullytransformed#1\endcsname%
  \iftikz@fullytransformed%
    \pgfresetnontranslationattimefalse%
  \else%
    \pgfresetnontranslationattimetrue%
  \fi%
}

\newif\iftikz@fullytransformed
\pgfresetnontranslationattimetrue%

\def\tikz@anchor{center}%
\def\tikz@shape{rectangle}%

\def\tikz@possibly@transform#1#2#3{%
  \let\tikz@do@auto@anchor=\relax%
  \def\tikz@test{#3}%
  \ifx\tikz@test\pgfutil@empty%
  \else%
    \pgfmathsetlength{\pgf@x}{#3}%
    \pgf@x=#2\pgf@x\relax%
    \edef\tikz@marshal{\noexpand\tikz@addtransform{%
        \expandafter\noexpand\csname  pgftransform#1shift\endcsname{\the\pgf@x}}}% 
    \tikz@marshal%
  \fi%
}


% Inter-picture options
\tikzoption{remember picture}[true]{\csname pgfrememberpicturepositiononpage#1\endcsname}
\tikzoption{overlay}[]{\pgf@relevantforpicturesizefalse}



% Line/curve label placement options
\tikzoption{sloped}[true]{\csname pgfslopedattime#1\endcsname}
\tikzoption{allow upside down}[true]{\csname pgfallowupsidedownattime#1\endcsname}

\tikzoption{pos}{\edef\tikz@time{#1}}

\tikzoption{auto}[]{\csname tikz@install@auto@anchor@#1\endcsname}
\tikzoption{swap}[]{%
  \def\tikz@temp{left}%
  \ifx\tikz@auto@anchor@direction\tikz@temp%
    \def\tikz@auto@anchor@direction{right}%
  \else%
    \def\tikz@auto@anchor@direction{left}%
  \fi%
}

\def\tikz@time{.5}

\def\tikz@install@auto@anchor@{\let\tikz@do@auto@anchor=\tikz@auto@anchor@on}
\def\tikz@install@auto@anchor@false{\let\tikz@do@auto@anchor=\relax}
\def\tikz@install@auto@anchor@left{\let\tikz@do@auto@anchor=\tikz@auto@anchor@on\def\tikz@auto@anchor@direction{left}}
\def\tikz@install@auto@anchor@right{\let\tikz@do@auto@anchor=\tikz@auto@anchor@on\def\tikz@auto@anchor@direction{right}}

\let\tikz@do@auto@anchor=\relax%

\def\tikz@auto@anchor@on{\csname tikz@auto@anchor@\tikz@auto@anchor@direction\endcsname}

\def\tikz@auto@anchor@left{\tikz@auto@pre\tikz@auto@anchor\tikz@auto@post}
\def\tikz@auto@anchor@right{\tikz@auto@pre\tikz@auto@anchor@prime\tikz@auto@post}

\def\tikz@auto@anchor@direction{left}

% Text options
\tikzoption{text}{\def\tikz@textcolor{#1}}
\tikzoption{font}{\def\tikz@textfont{#1}}
\tikzoption{text opacity}{\def\tikz@textopacity{#1}}
\tikzoption{text width}{\def\tikz@text@width{#1}}
\tikzoption{text height}{\def\tikz@text@height{#1}}
\tikzoption{text depth}{\def\tikz@text@depth{#1}}
\tikzoption{text ragged}[]%
{\def\tikz@text@action{\raggedright\rightskip\z@ plus2em \spaceskip.3333em \xspaceskip.5em\relax}}
\tikzoption{text badly ragged}[]{\def\tikz@text@action{\raggedright\relax}}
\tikzoption{text ragged left}[]%
{\def\tikz@text@action{\raggedleft\leftskip\z@ plus2em \spaceskip.3333em \xspaceskip.5em\relax}}
\tikzoption{text badly ragged left}[]{\def\tikz@text@action{\raggedleft\relax}}
\tikzoption{text justified}[]{\def\tikz@text@action{\leftskip\z@\rightskip\z@\relax}}
\tikzoption{text centered}[]{\def\tikz@text@action{%
  \leftskip\z@ plus2em%
  \rightskip\z@ plus2em%
  \spaceskip.3333em \xspaceskip.5em%
  \parfillskip=0pt%
  \let\\=\@centercr% for latex
  \relax}}
\tikzoption{text badly centered}[]%
{\def\tikz@text@action{%
  \let\\=\@centercr% for latex
  \parfillskip=0pt%
  \rightskip\@flushglue%
  \leftskip\@flushglue\relax}}

\let\tikz@text@width=\pgfutil@empty
\let\tikz@text@height=\pgfutil@empty
\let\tikz@text@depth=\pgfutil@empty
\let\tikz@textcolor=\pgfutil@empty
\let\tikz@textfont=\pgfutil@empty
\let\tikz@textopacity=\pgfutil@empty

\def\tikz@text@action{\raggedright\rightskip\z@ plus2em \spaceskip.3333em \xspaceskip.5em\relax}





% Arrow options
\tikzoption{arrows}{\tikz@processarrows{#1}}

\tikzoption{>}{%
  \tikz@set@pointed{\csname pgf@arrows@invert#1\endcsname}{#1}%
  \expandafter\tikz@processarrows\expandafter{\tikz@current@arrows}%
}

\tikzoption{shorten <}{\pgfsetshortenstart{#1}}
\tikzoption{shorten >}{\pgfsetshortenend{#1}}

\def\tikz@set@pointed#1#2{%
  \pgfutil@ifundefined{pgf@arrow@code@tikze@>@#2}
  {%
    \pgfarrowsdeclarealias{tikzs@<@#2}{tikze@>@#2}{#1}{#2}%
    \pgfarrowsdeclarereversed{tikzs@>@#2}{tikze@<@#2}{#1}{#2}%
    \pgfarrowsdeclarecombine*{tikz@|<@#2}{tikz@>|@#2}{#1}{#2}{|}{|}%
    \pgfarrowsdeclaredouble[\pgflinewidth]{tikzs@<<@#2}{tikze@>>@#2}{#1}{#2}%
    \pgfarrowsdeclarereversed{tikzs@>>@#2}{tikze@<<@#2}{tikzs@<<@#2}{tikze@>>@#2}%
  }{}%
  \pgfutil@namedef{tikz@special@arrow@start<}{tikzs@<@#2}%
  \pgfutil@namedef{tikz@special@arrow@end>}{tikze@>@#2}%
  \pgfutil@namedef{tikz@special@arrow@start>}{tikzs@>@#2}%
  \pgfutil@namedef{tikz@special@arrow@end<}{tikze@<@#2}%
  \pgfutil@namedef{tikz@special@arrow@start|<}{tikz@|<@#2}%
  \pgfutil@namedef{tikz@special@arrow@end>|}{tikz@>|@#2}%
  \pgfutil@namedef{tikz@special@arrow@start<<}{tikzs@<<@#2}%
  \pgfutil@namedef{tikz@special@arrow@end>>}{tikze@>>@#2}%
  \pgfutil@namedef{tikz@special@arrow@start>>}{tikzs@<<@#2}%
  \pgfutil@namedef{tikz@special@arrow@end<<}{tikze@>>@#2}%
}

\def\tikz@processarrows#1{%
  \def\tikz@current@arrows{#1}%
  \def\tikz@temp{#1}%
  \ifx\tikz@temp\pgfutil@empty%
  \else%
    \tikz@@processarrows#1\@nil
  \fi%
}
\def\tikz@@processarrows#1-#2\@nil{%
  \expandafter\ifx\csname tikz@special@arrow@start#1\endcsname\relax%
    \pgfsetarrowsstart{#1}
  \else%
    \pgfsetarrowsstart{\csname tikz@special@arrow@start#1\endcsname}%
  \fi%
  \expandafter\ifx\csname tikz@special@arrow@end#2\endcsname\relax%
    \pgfsetarrowsend{#2}
  \else%
    \pgfsetarrowsend{\csname tikz@special@arrow@end#2\endcsname}%
  \fi%
}

\tikz@set@pointed{\pgf@arrows@invertto}{to}
\def\tikz@current@arrows{-}

% Parabola options
\tikzoption{bend}{\tikz@scan@one@point\tikz@set@parabola@bend#1\relax}%
\tikzoption{bend pos}{\def\tikz@parabola@bend@factor{#1}}
\tikzoption{parabola height}{%
  \def\tikz@parabola@bend@factor{.5}%
  \def\tikz@parabola@bend{\pgfpointadd{\pgfpoint{0pt}{#1}}{\tikz@last@position@saved}}}

\def\tikz@parabola@bend{\tikz@last@position@saved}
\def\tikz@parabola@bend@factor{0}

\def\tikz@set@parabola@bend#1{\def\tikz@parabola@bend{#1}}

% Axis options
\tikzoption{domain}{\def\tikz@plot@domain{#1}\expandafter\tikz@plot@samples@recalc\tikz@plot@domain\relax}
\tikzoption{range}{\def\tikz@plot@range{#1}}

% Plot options
\tikzoption{smooth}[]{\let\tikz@plot@handler=\pgfplothandlercurveto}
\tikzoption{smooth cycle}[]{\let\tikz@plot@handler=\pgfplothandlerclosedcurve}
\tikzoption{sharp plot}[]{\let\tikz@plot@handler\pgfplothandlerlineto}

\tikzoption{tension}{\pgfsetplottension{#1}}

\tikzoption{xcomb}[]{\let\tikz@plot@handler=\pgfplothandlerxcomb}
\tikzoption{ycomb}[]{\let\tikz@plot@handler=\pgfplothandlerycomb}
\tikzoption{polar comb}[]{\let\tikz@plot@handler=\pgfplothandlerpolarcomb}

\tikzoption{raw gnuplot}[true]{\csname tikz@plot@raw@gnuplot#1\endcsname}
\tikzoption{prefix}{\def\tikz@plot@prefix{#1}}
\tikzoption{id}{\def\tikz@plot@id{#1}}

\tikzoption{samples}{\def\tikz@plot@samples{#1}\expandafter\tikz@plot@samples@recalc\tikz@plot@domain\relax}
\tikzoption{samples at}{\def\tikz@plot@samplesat{#1}}
\tikzoption{parametric}[true]{\csname tikz@plot@parametric#1\endcsname}

\tikzoption{variable}{\def\tikz@plot@var{#1}}

\tikzoption{only marks}[]{\let\tikz@plot@handler\pgfplothandlerdiscard}

\tikzoption{mark}{\def\tikz@plot@mark{#1}}
\tikzoption{mark options}{\def\tikz@plot@mark@options{#1}}
\tikzoption{mark size}{\pgfsetplotmarksize{#1}}

\tikzoption{mark indices}{\def\tikz@mark@list{#1}}
\tikzoption{mark phase}{\pgfsetplotmarkphase{#1}}
\tikzoption{mark repeat}{\pgfsetplotmarkrepeat{#1}}

\let\tikz@mark@list=\pgfutil@empty

\let\tikz@plot@mark@options=\pgfutil@empty

\let\tikz@plot@handler=\pgfplothandlerlineto
\let\tikz@plot@mark=\pgfutil@empty

\def\tikz@plot@samples{25}
\def\tikz@plot@domain{-5:5}
\def\tikz@plot@var{\x}
\def\tikz@plot@samplesat{-5,-4.6,...,5}
\def\tikz@plot@samples@recalc#1:#2\relax{%
  \pgfmathparse{#1}%
  \let\tikz@temp@start=\pgfmathresult%
  \pgfmathparse{#2}%
  \let\tikz@temp@end=\pgfmathresult%
  \pgfmathparse{\tikz@temp@start+(\tikz@temp@end-\tikz@temp@start)/\tikz@plot@samples}%
  \edef\tikz@plot@samplesat{\tikz@temp@start,\pgfmathresult,...,\tikz@temp@end}%
}


\def\tikz@plot@prefix{\jobname.}
\def\tikz@plot@id{pgf-plot}

\newif\iftikz@plot@parametric
\newif\iftikz@plot@raw@gnuplot


% To options
\tikzoption{to path}{\def\tikz@to@path{#1}}

\def\tikz@to@path{-- (\tikztotarget) \tikztonodes}



% Tree options
\tikzoption{edge from parent path}{\def\tikz@edge@to@parent@path{#1}}

\tikzoption{parent anchor}{\def\tikzparentanchor{.#1}\ifx\tikzparentanchor\tikz@border@text\let\tikzparentanchor\pgfutil@empty\fi}
\tikzoption{child anchor}{\def\tikzchildanchor{.#1}\ifx\tikzchildanchor\tikz@border@text\let\tikzchildanchor\pgfutil@empty\fi}

\tikzoption{level distance}{\pgfmathsetlength\tikzleveldistance{#1}}
\tikzoption{sibling distance}{\pgfmathsetlength\tikzsiblingdistance{#1}}

\tikzoption{growth function}{\let\tikz@grow=#1}
\tikzoption{growth parent anchor}{\def\tikz@growth@anchor{#1}}
\tikzoption{grow}{\tikz@set@growth{#1}\edef\tikz@special@level{\the\tikztreelevel}}%
\tikzoption{grow'}{\tikz@set@growth{#1}\tikz@swap@growth\edef\tikz@special@level{\the\tikztreelevel}}%

\def\tikz@growth@anchor{center}

\def\tikz@special@level{-1}% never

\def\tikz@swap@growth{%
  % Swap left and right
  \let\tikz@temp=\tikz@angle@grow@right%
  \let\tikz@angle@grow@right=\tikz@angle@grow@left%
  \let\tikz@angle@grow@left=\tikz@temp%
}%

\def\tikz@set@growth#1{%
  \let\tikz@grow=\tikz@grow@direction%
  \expandafter\ifx\csname tikz@grow@direction@#1\endcsname\relax%
    \c@pgf@counta=#1\relax%
  \else%
    \c@pgf@counta=\csname tikz@grow@direction@#1\endcsname%
  \fi%
  \edef\tikz@angle@grow{\the\c@pgf@counta}%
  \advance\c@pgf@counta by-90\relax%
  \edef\tikz@angle@grow@left{\the\c@pgf@counta}%
  \advance\c@pgf@counta by180\relax%
  \edef\tikz@angle@grow@right{\the\c@pgf@counta}%
}

\def\tikz@border@text{.border}
\let\tikzparentanchor=\pgfutil@empty
\let\tikzchildanchor=\pgfutil@empty
\def\tikz@edge@to@parent@path{(\tikzparentnode\tikzparentanchor) -- (\tikzchildnode\tikzchildanchor)}

\tikzleveldistance=15mm
\tikzsiblingdistance=15mm

\def\tikz@grow@direction@down{-90}
\def\tikz@grow@direction@up{90}
\def\tikz@grow@direction@left{180}
\def\tikz@grow@direction@right{0}

\def\tikz@grow@direction@south{-90}
\def\tikz@grow@direction@north{90}
\def\tikz@grow@direction@west{180}
\def\tikz@grow@direction@east{0}

\expandafter\def\csname tikz@grow@direction@north east\endcsname{45}
\expandafter\def\csname tikz@grow@direction@north west\endcsname{135}
\expandafter\def\csname tikz@grow@direction@south east\endcsname{-45}
\expandafter\def\csname tikz@grow@direction@south west\endcsname{-135}

\def\tikz@grow@direction{%
  \pgftransformshift{\pgfpointpolar{\tikz@angle@grow}{\tikzleveldistance}}%
  \ifnum\tikztreelevel=\tikz@special@level%
  \else%
    \pgf@xc=.5\tikzsiblingdistance%
    \c@pgf@counta=\tikznumberofchildren%
    \advance\c@pgf@counta by1\relax%
    \pgfutil@tempdima=\c@pgf@counta\pgf@xc%
    \pgftransformshift{\pgfpointpolar{\tikz@angle@grow@left}{\pgfutil@tempdima}}%
    \pgftransformshift{\pgfpointpolar{\tikz@angle@grow@right}{\tikznumberofcurrentchild\tikzsiblingdistance}}%
  \fi%
}

\tikzset{grow=down}


% Snake options
\tikzoption{snake}[]{%
  \def\tikz@@snake{#1}%
  \ifx\tikz@@snake\pgfutil@empty%
    \tikz@snakedtrue%
  \else%
    \ifx\tikz@@snake\tikz@nonetext%
      \tikz@snakedfalse%
    \else%
      \tikz@snakedtrue%
      \let\tikz@snake=\tikz@@snake%
    \fi%
  \fi}

\tikzoption{segment amplitude}{\pgfmathsetlength{\pgfsnakesegmentamplitude}{#1}}
\tikzoption{segment length}{\pgfmathsetlength{\pgfsnakesegmentlength}{#1}}
\tikzoption{segment angle}{\pgfmathparse{#1}\let\pgfsnakesegmentangle=\pgfmathresult}
\tikzoption{segment aspect}{\pgfmathparse{#1}\let\pgfsnakesegmentaspect=\pgfmathresult}

\tikzoption{segment object length}{\pgfmathparse{#1}\edef\pgfsnakesegmentobjectlength{\pgfmathresult pt}}

\tikzoption{raise snake}{\def\pgf@snake@raise{\pgftransformyshift{#1}}}
\tikzoption{mirror snake}[true]{%
  \csname if#1\endcsname
    \def\pgf@snake@mirror{\pgftransformyscale{-1}}%
  \else%
    \let\pgf@snake@mirror=\pgfutil@empty%
  \fi
}

\tikzoption{gap before snake}{\def\tikz@presnake{{moveto}{#1}}}
\tikzoption{line before snake}{\def\tikz@presnake{{lineto}{#1}}}

\tikzoption{gap after snake}{\def\tikz@postsnake{{moveto}{#1}}\def\tikz@mainsnakelength{\pgfsnakeremainingdistance+-#1}}
\tikzoption{line after snake}{\def\tikz@postsnake{{lineto}{#1}}\def\tikz@mainsnakelength{\pgfsnakeremainingdistance+-#1}}

\tikzoption{gap around snake}{%
  \def\tikz@presnake{{moveto}{#1}}%
  \def\tikz@postsnake{{moveto}{#1}}%
  \def\tikz@mainsnakelength{\pgfsnakeremainingdistance+-#1}%
}
\tikzoption{line around snake}{%
  \def\tikz@presnake{{lineto}{#1}}%
  \def\tikz@postsnake{{lineto}{#1}}%
  \def\tikz@mainsnakelength{\pgfsnakeremainingdistance+-#1}%
}
\let\pgf@snake@mirror=\pgfutil@empty
\let\pgf@snake@raise=\pgfutil@empty

\pgfsetsnakesegmenttransformation{\pgf@snake@mirror\pgf@snake@raise}

\def\tikz@snake{zigzag}

\let\tikz@presnake=\pgfutil@empty
\let\tikz@postsnake=\pgfutil@empty
\def\tikz@mainsnakelength{\pgfsnakeremainingdistance}


% Matrix options
\tikzoption{matrix}[true]{\csname tikz@is@matrix#1\endcsname}

\tikzoption{matrix anchor}{\def\tikz@matrix@anchor{#1}}

\tikzoption{column sep}{\def\pgfmatrixcolumnsep{#1}}
\tikzoption{row sep}{\def\pgfmatrixrowsep{#1}}

\tikzoption{cells}{\tikzstyle{every cell}+=[#1]}

\tikzoption{ampersand replacement}{\def\tikz@ampersand@replacement{#1}}

\newif\iftikz@is@matrix
\let\tikz@matrix@anchor=\pgfutil@empty
\let\tikz@ampersand@replacement=\pgfutil@empty

% Execute option

\tikzoption{execute at begin picture}{\expandafter\def\expandafter\tikz@atbegin@picture\expandafter{\tikz@atbegin@picture#1}}
\tikzoption{execute at end picture}{\expandafter\def\expandafter\tikz@atend@picture\expandafter{\tikz@atend@picture#1}}
\tikzoption{execute at begin scope}{\expandafter\def\expandafter\tikz@atbegin@scope\expandafter{\tikz@atbegin@scope#1}}
\tikzoption{execute at end scope}{\expandafter\def\expandafter\tikz@atend@scope\expandafter{\tikz@atend@scope#1}}
\tikzoption{execute at begin to}{\expandafter\def\expandafter\tikz@atbegin@to\expandafter{\tikz@atbegin@to#1}}
\tikzoption{execute at end to}{\expandafter\def\expandafter\tikz@atend@to\expandafter{\tikz@atend@to#1}}
\tikzoption{execute at begin node}{\expandafter\def\expandafter\tikz@atbegin@node\expandafter{\tikz@atbegin@node#1}}
\tikzoption{execute at end node}{\expandafter\def\expandafter\tikz@atend@node\expandafter{\tikz@atend@node#1}}
\tikzoption{execute at begin cell}{\expandafter\def\expandafter\tikz@atbegin@cell\expandafter{\tikz@atbegin@cell#1}}
\tikzoption{execute at end cell}{\expandafter\def\expandafter\tikz@atend@cell\expandafter{\tikz@atend@cell#1}}
\tikzoption{execute at empty cell}{\expandafter\def\expandafter\tikz@at@emptycell\expandafter{\tikz@at@emptycell#1}}

\let\tikz@atbegin@picture=\pgfutil@empty
\let\tikz@atend@picture=\pgfutil@empty
\let\tikz@atbegin@scope=\pgfutil@empty
\let\tikz@atend@scope=\pgfutil@empty
\let\tikz@atbegin@to=\pgfutil@empty
\let\tikz@atend@to=\pgfutil@empty
\let\tikz@atbegin@node=\pgfutil@empty
\let\tikz@atend@node=\pgfutil@empty
\let\tikz@atbegin@cell=\pgfutil@empty
\let\tikz@atend@cell=\pgfutil@empty
\let\tikz@at@emptycell=\pgfutil@empty




% Styles
\tikzoption{set style}{\tikzstyle#1}

% Handled in a special way.
\def\tikzstyle{\pgfutil@ifnextchar\bgroup\tikz@style@parseA\tikz@style@parseB}
\def\tikz@style@parseB#1={\tikz@style@parseA{#1}=}
\def\tikz@style@parseA#1#2=#3[#4]{% check for an optional argument
  \pgfutil@in@[{#2}%]
  \ifpgfutil@in@%
    \tikz@style@parseC{#1}#2={#4}%
  \else%
    \tikz@style@parseD{#1}#2={#4}%
  \fi%
}%

\def\tikz@style@parseC#1[#2]#3=#4{%
  \pgfkeys{/tikz/#1/.default={#2}}%
  \pgfutil@in@+{#3}%
  \ifpgfutil@in@%
    \pgfkeys{/tikz/#1/.append style={#4}}%
  \else%
    \pgfkeys{/tikz/#1/.style={#4}}%
  \fi}
\def\tikz@style@parseD#1#2=#3{%
  \pgfutil@in@+{#2}%
  \ifpgfutil@in@%
    \pgfkeys{/tikz/#1/.append style={#3}}%
  \else%
    \pgfkeys{/tikz/#1/.style={#3}}%
  \fi}


%
%
% Predefined styles
%
%

\tikzstyle{help lines}=              [color=gray,line width=0.2pt]

\tikzstyle{every picture}=           []
\tikzstyle{every path}=              []
\tikzstyle{every scope}=             []
\tikzstyle{every plot}=              []
\tikzstyle{every node}=              []
\tikzstyle{every child}=             []
\tikzstyle{every child node}=        []
\tikzstyle{every to}=                []
\tikzstyle{every cell}=              []
\tikzstyle{every matrix}=            []
\tikzstyle{every edge}=              [draw]
\tikzstyle{every label}=             [draw=none,fill=none]
\tikzstyle{every pin}=               [draw=none,fill=none]
\tikzstyle{every pin edge}=          [help lines]

\tikzstyle{ultra thin}=              [line width=0.1pt]
\tikzstyle{very thin}=               [line width=0.2pt]
\tikzstyle{thin}=                    [line width=0.4pt]
\tikzstyle{semithick}=               [line width=0.6pt]
\tikzstyle{thick}=                   [line width=0.8pt]
\tikzstyle{very thick}=              [line width=1.2pt]
\tikzstyle{ultra thick}=             [line width=1.6pt]

\tikzstyle{solid}=                   [dash pattern=]
\tikzstyle{dotted}=                  [dash pattern=on \pgflinewidth off 2pt]
\tikzstyle{densely dotted}=          [dash pattern=on \pgflinewidth off 1pt]
\tikzstyle{loosely dotted}=          [dash pattern=on \pgflinewidth off 4pt]
\tikzstyle{dashed}=                  [dash pattern=on 3pt off 3pt]
\tikzstyle{densely dashed}=          [dash pattern=on 3pt off 2pt]
\tikzstyle{loosely dashed}=          [dash pattern=on 3pt off 6pt]

\tikzstyle{transparent}=             [opacity=0]
\tikzstyle{ultra nearly transparent}=[opacity=0.05]
\tikzstyle{very nearly transparent}= [opacity=0.1]
\tikzstyle{nearly transparent}=      [opacity=0.25]
\tikzstyle{semitransparent}=         [opacity=0.5]
\tikzstyle{nearly opaque}=           [opacity=0.75]
\tikzstyle{very nearly opaque}=      [opacity=0.9]
\tikzstyle{ultra nearly opaque}=     [opacity=0.95]
\tikzstyle{opaque}=                  [opacity=1]

\tikzstyle{at start}=                [pos=0]
\tikzstyle{very near start}=         [pos=0.125]
\tikzstyle{near start}=              [pos=0.25]
\tikzstyle{midway}=                  [pos=0.5]
\tikzstyle{near end}=                [pos=0.75]
\tikzstyle{very near end}=           [pos=0.875]
\tikzstyle{at end}=                  [pos=1]

\tikzstyle{bend at start}=           [bend pos=0,bend={+(0,0)}]
\tikzstyle{bend at end}=             [bend pos=1,bend={+(0,0)}]

\tikzstyle{edge from parent}=        [draw]

\tikzstyle{snake triangles 45}=      [snake=triangles,segment object length=2.41421356\pgfsnakesegmentamplitude]
\tikzstyle{snake triangles 60}=      [snake=triangles,segment object length=1.73205081\pgfsnakesegmentamplitude]
\tikzstyle{snake triangles 90}=      [snake=triangles,segment object length=\pgfsnakesegmentamplitude]


%
% Setting keys
%

\pgfkeys{/tikz/style/.style=#1}

\pgfkeys{/tikz/.unknown/.code=%
  % Is it a pgf key?
  \let\tikz@key\pgfkeyscurrentname% 
  \pgfkeys{/pgf/\tikz@key/.try=#1}%
  \ifpgfkeyssuccess%
  \else%
    \expandafter\pgfutil@in@\expandafter!\expandafter{\tikz@key}%
    \ifpgfutil@in@%
      % this is a color!
      \expandafter\tikz@addoption\expandafter{\expandafter\pgfutil@color\expandafter{\tikz@key}}%
      \edef\tikz@textcolor{\tikz@key}%
    \else%
      \pgfutil@doifcolorelse{\tikz@key}
      { %     
        \expandafter\tikz@addoption\expandafter{\expandafter\pgfutil@color\expandafter{\tikz@key}}%
        \edef\tikz@textcolor{\tikz@key}%
      }%
      {%
        % Ok, second chance: This might be an arrow specification:
        \expandafter\pgfutil@in@\expandafter-\expandafter{\tikz@key}
        \ifpgfutil@in@%
          % Ah, an arrow spec!
          \expandafter\tikz@processarrows\expandafter{\tikz@key}%
        \else%
          % Ok, third chance: A shape!
          \expandafter\ifx\csname pgf@sh@s@\tikz@key\endcsname\relax%
            \pgfkeys{/errors/unknown key={/tikz/\tikz@key}{#1}}%
          \else%
            \edef\tikz@shape{\tikz@key}%
          \fi%
        \fi%
      }%
    \fi%
  \fi%
}


%
% Main TikZ Environment
%

\def\tikzpicture{\pgfutil@ifnextchar[\tikz@picture{\tikz@picture[]}}%}
\def\tikz@picture[#1]{%
  \pgfpicture%
  \let\tikz@atbegin@picture=\pgfutil@empty%
  \let\tikz@atend@picture=\pgfutil@empty%
  \let\tikz@transform=\relax%
  \tikz@installcommands\scope[every picture,#1]%
  \tikz@atbegin@picture%
}
\def\endtikzpicture{%
    \tikz@atend@picture%
    \global\let\pgf@shift@baseline=\pgf@baseline%
    \global\let\pgf@remember@smuggle=\ifpgfrememberpicturepositiononpage%
    \endscope%
    \let\pgf@baseline=\pgf@shift@baseline%
    \let\ifpgfrememberpicturepositiononpage=\pgf@remember@smuggle%
  \endpgfpicture}

  

% Inlined picture
%
% #1 - some code to be put in a tikzpicture environment.
%
% If the command is not followed by braces, everything up to the next
% semicolon is used as argument.
%
% Example:
%
% The rectangle \tikz{\draw (0,0) rectangle (1em,1ex)} has width 1em and
% height 1ex.

\def\tikz{\pgfutil@ifnextchar[{\tikz@opt}{\tikz@opt[]}}
\def\tikz@opt[#1]{\tikzpicture[#1]\pgfutil@ifnextchar\bgroup{\tikz@}{\tikz@@}}
\def\tikz@#1{#1\endtikzpicture}
\def\tikz@@{%
  \let\tikz@next=\tikz@collectnormalsemicolon%
  \ifnum\the\catcode`\;=\active\relax%
    \let\tikz@next=\tikz@collectactivesemicolon%
  \fi%
  \tikz@next}
\def\tikz@collectnormalsemicolon#1;{#1;\endtikzpicture}
{
  \catcode`\;=\active
  \gdef\tikz@collectactivesemicolon#1;{#1;\endtikzpicture}
}



%
% Environment for scoping graphic state settings
%
\def\tikz@scope@env{\pgfutil@ifnextchar[\tikz@@scope@env{\tikz@@scope@env[]}}
\def\tikz@@scope@env[#1]{%
  \pgfscope%
  \begingroup%
  \let\tikz@atbegin@scope=\pgfutil@empty%
  \let\tikz@atend@scope=\pgfutil@empty%
  \let\tikz@options=\pgfutil@empty%
  \let\tikz@mode=\pgfutil@empty%
  \tikzset{every scope/.try,#1}%
  \tikz@options%
  \tikz@atbegin@scope%
}
\def\endtikz@scope@env{%
  \tikz@atend@scope%
  \endgroup%
  \endpgfscope%
}


%
% Install the abbreviated commands
%
\def\tikz@installcommands{%
  \ifnum\the\catcode`\;=\active\relax\expandafter\let\expandafter\tikz@origsemi\expandafter=\tikz@activesemicolon\fi%
  \ifnum\the\catcode`\:=\active\relax\expandafter\let\expandafter\tikz@origcolon\expandafter=\tikz@activecolon\fi%
  \ifnum\the\catcode`\|=\active\relax\expandafter\let\expandafter\tikz@origbar\expandafter=\tikz@activebar\fi%
  \let\tikz@origscope=\scope%
  \let\tikz@origendscope=\endscope%
  \let\tikz@origstartscope=\startscope%
  \let\tikz@origstopscope=\stopscope%
  \let\tikz@origpath=\path%
  \let\tikz@origagainpath=\againpath%
  \let\tikz@origdraw=\draw%
  \let\tikz@origpattern=\pattern%
  \let\tikz@origfill=\fill%
  \let\tikz@origfilldraw=\filldraw%
  \let\tikz@origshade=\shade%
  \let\tikz@origshadedraw=\shadedraw%
  \let\tikz@origclip=\clip%
  \let\tikz@origuseasboundingbox=\useasboundingbox%
  \let\tikz@orignode=\node%
  \let\tikz@origcoordinate=\coordinate%
  \let\tikz@origmatrix=\matrix%
  \let\tikz@origcalendar=\calendar%
  %
  \tikz@deactivatthings%
  %
  \let\scope=\tikz@scope@env%
  \let\endscope=\endtikz@scope@env%
  \let\startscope=\scope%
  \let\stopscope=\endscope%
  \let\path=\tikz@command@path%
  \let\againpath=\tikz@command@againpath%
  %
  \def\draw{\path[draw]}
  \def\pattern{\path[pattern]}
  \def\fill{\path[fill]}
  \def\filldraw{\path[fill,draw]}
  \def\shade{\path[shade]}
  \def\shadedraw{\path[shade,draw]}
  \def\clip{\path[clip]}
  \def\useasboundingbox{\path[use as bounding box]}
  \def\node{\tikz@path@overlay{node}}
  \def\coordinate{\tikz@path@overlay{coordinate}}
  \def\matrix{\tikz@path@overlay{node[matrix]}}
  \def\calendar{\tikz@lib@cal@calendar}%
}
\ifx\tikz@lib@cal@calendar\@undefined
\def\tikz@lib@cal@calendar{\PackageError{tikz}{You need to load the calendar library}{}}
\fi

\def\tikz@path@overlay#1{%
  \let\tikz@signal@path=\tikz@signal@path% for detection at begin of matrix cell
  \pgfutil@ifnextchar<{\tikz@path@overlayed{#1}}{\path #1}}
\def\tikz@path@overlayed#1<#2>{\path<#2> #1}

\def\tikz@uninstallcommands{%
  \ifnum\the\catcode`\;=\active\relax\expandafter\let\tikz@activesemicolon=\tikz@origsemi\fi%
  \ifnum\the\catcode`\:=\active\relax\expandafter\let\tikz@activecolon=\tikz@origcolon\fi%
  \ifnum\the\catcode`\|=\active\relax\expandafter\let\tikz@activebar=\tikz@origbar\fi%
  \let\scope=\tikz@origscope%
  \let\endscope=\tikz@origendscope%
  \let\startscope=\tikz@origstartscope%
  \let\stopscope=\tikz@origstopscope%
  \let\path=\tikz@origpath%
  \let\againpath=\tikz@origagainpath%
  \let\draw=\tikz@origdraw%
  \let\pattern=\tikz@origpattern%
  \let\fill=\tikz@origfill%
  \let\filldraw=\tikz@origfilldraw%
  \let\shade=\tikz@origshade%
  \let\shadedraw=\tikz@origshadedraw%
  \let\clip=\tikz@origclip%
  \let\useasboundingbox=\tikz@origuseasboundingbox%
  \let\node=\tikz@orignode%
  \let\coordinate=\tikz@origcoordinate%
  \let\matrix=\tikz@origmatrix%
  \let\calendar=\tikz@origcalendar%
}


{
  \catcode`\;=12
  \gdef\tikz@nonactivesemicolon{;}
  \catcode`\:=12
  \gdef\tikz@nonactivecolon{:}
  \catcode`\|=12
  \gdef\tikz@nonactivebar{|}
  \catcode`\;=\active
  \catcode`\:=\active
  \catcode`\|=\active
  \catcode`\"=\active
  \gdef\tikz@activesemicolon{;}%
  \gdef\tikz@activecolon{:}%
  \gdef\tikz@activebar{|}%
  \gdef\tikz@activequotes{"}%
  \gdef\tikz@deactivatthings{%
    \def;{\tikz@nonactivesemicolon}
    \def:{\tikz@nonactivecolon}
    \def|{\tikz@nonactivebar}
  }
}





% Constructs a path and draws/fills them according to the current
% settings.  

\def\tikz@command@path{%
  \let\tikz@signal@path=\tikz@signal@path% for detection at begin of matrix cell
  \pgfutil@ifnextchar[{\tikz@check@earg}%]
  {\pgfutil@ifnextchar<{\tikz@doopt}{\tikz@@command@path}}}
\def\tikz@signal@path{\tikz@signal@path}%
\def\tikz@check@earg[#1]{%
  \pgfutil@ifnextchar<{\tikz@swap@args[#1]}{\tikz@@command@path[#1]}}
\def\tikz@swap@args[#1]<#2>{\tikz@command@path<#2>[#1]}

\def\tikz@doopt{%
  \let\tikz@next=\tikz@eargnormalsemicolon%
  \ifnum\the\catcode`\;=\active\relax%
    \let\tikz@next=\tikz@eargactivesemicolon%
  \fi%
  \tikz@next}
\long\def\tikz@eargnormalsemicolon<#1>#2;{\only<#1>{\tikz@@command@path#2;}}
{
  \catcode`\;=\active
  \long\global\def\tikz@eargactivesemicolon<#1>#2;{\only<#1>{\tikz@@command@path#2;}}
}

\def\tikz@@command@path{%
  \edef\tikzscope@linewidth{\the\pgflinewidth}%
  \begingroup%
    \let\tikz@options=\pgfutil@empty%
    \let\tikz@mode=\pgfutil@empty%
    \let\tikz@moveto@waiting=\relax%
    \let\tikz@timer=\relax%
    \let\tikz@collected@onpath=\pgfutil@empty%
    \tikz@snakedfalse%
    \tikz@node@is@a@labelfalse%
    \tikz@expandcount=1000\relax%
    \tikz@lastx=0pt%
    \tikz@lasty=0pt%
    \tikz@lastxsaved=0pt%
    \tikz@lastysaved=0pt%
    \tikzset{every path/.try}%
    \tikz@scan@next@command%
}
\def\tikz@scan@next@command{%
  \ifx\tikz@collected@onpath\pgfutil@empty%
  \else%
    \tikz@invoke@collected@onpath%
  \fi%
  \afterassignment\tikz@handle\let\@let@token=%
}
\newcount\tikz@expandcount

% Central dispatcher for commands
\def\tikz@handle{%
  \let\@next=\tikz@expand%
    \ifx\@let@token(%)
      \let\@next=\tikz@movetoabs%
    \else%
      \ifx\@let@token+%
        \let\@next=\tikz@movetorel%
      \else%
        \ifx\@let@token-%
          \let\@next=\tikz@lineto%
        \else%
          \ifx\@let@token.%
            \let\@next=\tikz@dot%
          \else%
            \ifx\@let@token r%
              \let\@next=\tikz@rect%
            \else%
              \ifx\@let@token a%
                \let\@next=\tikz@arcA%
              \else%
                \ifx\@let@token[%]
                  \let\@next=\tikz@parse@options%
                \else%
                  \ifx\@let@token n%
                    \let\@next=\tikz@fig%
                  \else%
                    \ifx\@let@token\bgroup%
                      \let\@next=\tikz@beginscope%
                    \else%
                      \ifx\@let@token\egroup%
                        \let\@next=\tikz@endscope%
                      \else%
                        \ifx\@let@token;%
                          \let\@next=\tikz@finish%
                        \else%
                          \ifx\@let@token c%
                            \let\@next=\tikz@cchar%
                          \else%
                            \ifx\@let@token e%
                              \let\@next=\tikz@e@char%
                            \else%
                              \ifx\@let@token g%
                                \let\@next=\tikz@grid%
                              \else%
                                \ifx\@let@token s%
                                   \let\@next=\tikz@sine%
                                \else%
                                  \ifx\@let@token |%
                                     \let\@next=\tikz@vh@lineto%
                                  \else%
                                    \ifx\@let@token p%
                                      \let\@next=\tikz@pchar%
                                      \pgfsetmovetofirstplotpoint%
                                    \else%
                                      \ifx\@let@token t%
                                        \let\@next=\tikz@to%
                                      \else%
                                        \ifx\@let@token\pgfextra%
                                          \let\@next=\tikz@extra%
                                        \else%
                                          \ifx\@let@token\foreach%
                                            \let\@next=\tikz@foreach%
                                          \else%
                                            \ifx\@let@token\pgf@stop%
                                              \let\@next=\relax%
                                            \else%
                                              \ifx\@let@token\par%
                                                \let\@next=\tikz@scan@next@command%
                                              \fi%      
                                            \fi%      
                                          \fi%      
                                        \fi%      
                                      \fi%      
                                    \fi%  
                                  \fi%  
                                \fi%
                              \fi%
                            \fi%
                          \fi%
                        \fi%
                      \fi%  
                    \fi%  
                  \fi%  
                \fi%  
              \fi%
            \fi%
          \fi%
        \fi%
      \fi%
    \fi%
  \@next%
}

\def\tikz@pchar{\pgfutil@ifnextchar l{\tikz@plot}{\tikz@parabola}}
\def\tikz@cchar{%
  \pgfutil@ifnextchar i{\tikz@circle}%
  {\pgfutil@ifnextchar h{\tikz@children}{\tikz@cochar}}}%
\def\tikz@cochar o{%
  \pgfutil@ifnextchar o{\tikz@coordinate}{\tikz@cosine}}
\def\tikz@e@char{%
  \pgfutil@ifnextchar l{\tikz@ellipse}{\tikz@@e@char}}%
\def\tikz@@e@char dge{%
  \pgfutil@ifnextchar f{\tikz@edgetoparent}{\tikz@edge@plain}}%


\def\tikz@finish{%
  \tikz@mode@fillfalse%
  \tikz@mode@drawfalse%
  \tikz@mode@doublefalse%
  \tikz@mode@clipfalse%
  \tikz@mode@boundaryfalse%
  \edef\tikz@pathextend{%
    {\noexpand\pgfqpoint{\the\pgf@pathminx}{\the\pgf@pathminy}}%
    {\noexpand\pgfqpoint{\the\pgf@pathmaxx}{\the\pgf@pathmaxy}}%
  }%
  \tikz@mode% installs the mode settings
  % Rendering pipeline:  
  % 
  % Step 1: Setup options
  % 
  \ifx\tikz@options\pgfutil@empty%
  \else%
    \pgfsys@beginscope%
      \begingroup%
      \tikz@options%
  \fi%
  % 
  % Step 2: Do a fill if shade follows.
  %
  \iftikz@mode@fill%
    \iftikz@mode@shade%
      \pgfprocessround{\pgfsyssoftpath@currentpath}{\pgfsyssoftpath@currentpath}% change the current path
      \pgfsyssoftpath@invokecurrentpath%
      \pgfsys@fill%
      \tikz@mode@fillfalse% no more filling...
    \fi%
  \fi%
  % 
  % Step 3: Do a shade if necessary.
  %
  \iftikz@mode@shade%
    \pgfprocessround{\pgfsyssoftpath@currentpath}{\pgfsyssoftpath@currentpath}% change the current path
    \pgfshadepath{\tikz@shading}{\tikz@shade@angle}%
    \tikz@mode@shadefalse% no more shading...
  \fi%
  % 
  % Step 4: Double stroke, if necessary
  %
  \iftikz@mode@draw%
    \iftikz@mode@double%
      % Change line width
      \begingroup%
        \pgfsys@beginscope%
          \pgf@x=2\pgflinewidth%
          \advance\pgf@x by\tikz@double@width@distance%
          \pgflinewidth=\pgf@x%
          \pgfsetlinewidth{\the\pgflinewidth}%
    \fi%
  \fi%
  % 
  % Step 5: Do stroke/fill/clip as needed
  %
  \edef\tikz@temp{\noexpand\pgfusepath{%
    \iftikz@mode@fill fill,\fi%
    \iftikz@mode@draw draw,\fi%
    \iftikz@mode@clip clip,\fi%
    }}%
  \tikz@temp%
  \tikz@mode@fillfalse% no more filling
  % 
  % Step 6: Double stroke, if necessary
  %
  \iftikz@mode@draw%
    \iftikz@mode@double%
          \pgfsyssoftpath@setcurrentpath\pgf@last@used@path% reinstall
          \pgf@x=\tikz@double@width@distance%
          \pgfsetlinewidth{\the\pgf@x}%
          \pgfsetstrokecolor{\tikz@double@color}%
          \pgfsyssoftpath@flushcurrentpath%
          \pgfsys@stroke%
        \pgfsys@endscope%
        \pgf@add@arrows@as@needed
      \endgroup%
    \fi%
  \fi%
  \tikz@mode@drawfalse% no more stroking
  % 
  % Step 7: Add labels and nodes
  %
  \copy\tikz@figbox%
  \setbox\tikz@figbox=\box\voidb@x%
  %
  % Step 8: Close option brace
  %
  \ifx\tikz@options\pgfutil@empty%
  \else%
      \endgroup%
    \pgfsys@endscope%
    \iftikz@mode@clip%
      \PackageError{tikz}{Extra options not allowed for clipping path command.}{}%
    \fi%
  \fi%
  \iftikz@mode@clip%
    \aftergroup\pgf@relevantforpicturesizefalse%
  \fi%
  \iftikz@mode@boundary%
    \aftergroup\pgf@relevantforpicturesizefalse%
  \fi%
  \endgroup%
  \global\pgflinewidth=\tikzscope@linewidth%
}




\def\tikz@skip#1{\tikz@scan@next@command#1}
\def\tikz@expand{%
  \advance\tikz@expandcount by -1%
  \ifnum\tikz@expandcount<0\relax%
    \PackageError{tikz}{Giving up on this path. Did you forget a semicolon?}{}%
    \let\@next=\tikz@finish%
  \else%
    \let\@next=\tikz@@expand
  \fi%
  \@next}

\def\tikz@@expand{%
  \expandafter\tikz@scan@next@command\@let@token}



% Syntax for scopes: 
% {scoped path commands}

\def\tikz@beginscope{\begingroup\tikz@scan@next@command}
\def\tikz@endscope{%
  \global\setbox\tikz@tempbox=\copy\tikz@figbox%
  \endgroup%
  \setbox\tikz@figbox=\box\tikz@tempbox%
  \tikz@scan@next@command}


% Syntax for pgfextra: 
% \pgfextra {normal tex text}
% \pgfextra normal tex text \endpgfextra

\def\tikz@extra{\pgfutil@ifnextchar\bgroup\tikz@@extra\relax}
\long\def\tikz@@extra#1{#1\tikz@scan@next@command}
\let\endpgfextra=\tikz@scan@next@command

\def\pgfextra{pgfextra}


% Syntax for \foreach: 
% \foreach \var in {list} {path text}
%
% Example:
%
% \draw (0,0) \foreach \x in {1,2,3} {-- (\x,0) circle (1cm)} -- (5,5);

\def\tikz@foreach{%
  \def\pgffor@beginhook{\setbox\tikz@figbox=\box\tikz@tempbox\expandafter\tikz@scan@next@command\@firstofone}%
  \def\pgffor@endhook{\pgfextra{%
      \xdef\tikz@foreach@save@lastx{\the\tikz@lastx}%
      \xdef\tikz@foreach@save@lasty{\the\tikz@lasty}%
      \xdef\tikz@foreach@save@lastxsaved{\the\tikz@lastxsaved}%
      \xdef\tikz@foreach@save@lastysaved{\the\tikz@lastysaved}%
      \global\setbox\tikz@tempbox=\copy\tikz@figbox\pgfutil@gobble}}%
  \def\pgffor@afterhook{%
    \tikz@lastx=\tikz@foreach@save@lastx%
    \tikz@lasty=\tikz@foreach@save@lasty%
    \tikz@lastxsaved=\tikz@foreach@save@lastxsaved%
    \tikz@lastysaved=\tikz@foreach@save@lastysaved%
    \setbox\tikz@figbox=\box\tikz@tempbox\tikz@scan@next@command}%
  \global\setbox\tikz@tempbox=\copy\tikz@figbox%
  \foreach}

  
% Syntax for againpath: 
% \againpath \somepathname

\def\tikz@command@againpath#1{%
  \pgfextra{%
    \pgfsyssoftpath@getcurrentpath\tikz@temp%
    \expandafter\pgfutil@g@addto@macro\expandafter\tikz@temp\expandafter{#1}%
    \pgfsyssoftpath@setcurrentpath\tikz@temp%
  }
}




%
% When this if is set, a just-scanned point is a shape and its border
% position still needs to be determined, depending on subsequent
% commands. 
%

\newif\iftikz@shapeborder


% Syntax for moveto: 
% <point>
\def\tikz@movetoabs{\tikz@moveto(}
\def\tikz@movetorel{\tikz@moveto+}
\def\tikz@moveto{%
  \tikz@scan@one@point{\tikz@@moveto}}
\def\tikz@@moveto#1{%
  \tikz@make@last@position{#1}%
  \iftikz@shapeborder%
    % ok, the moveto will have to wait. flag that we have a moveto in
    % wainting:
    \edef\tikz@moveto@waiting{\tikz@shapeborder@name}%
  \else%
    \pgfpathmoveto{\tikz@last@position}%
    \let\tikz@moveto@waiting=\relax%
  \fi%
  \tikz@scan@next@command%
}

\let\tikz@moveto@waiting=\relax % normally, nothing is waiting...

\def\tikz@flush@moveto{%
  \ifx\tikz@moveto@waiting\relax%
  \else%
    \pgfpathmoveto{\tikz@last@position}%
  \fi%
  \let\tikz@moveto@waiting=\relax%
}


\def\tikz@flush@moveto@toward#1#2#3{%
  % #1 = a point towards which the last moveto should be corrected
  % #2 = a dimension to which the corrected x-coordinate should be stored
  % #3 = a dimension for the corrected y-coordinate
  \ifx\tikz@moveto@waiting\relax%
    % do nothing
  \else%
    \pgf@process{\pgfpointshapeborder{\tikz@moveto@waiting}{#1}}%
    #2=\pgf@x%
    #3=\pgf@y%
    \edef\tikz@timer@start{\noexpand\pgfqpoint{\the\pgf@x}{\the\pgf@y}}%
    \pgfpathmoveto{\pgfqpoint{\pgf@x}{\pgf@y}}%
  \fi%
  \let\tikz@moveto@waiting=\relax%
}


%
% Collecting labels on the path 
%

\def\tikz@collect@coordinate@onpath#1coordinate
\def\tikz@@collect@coordinate@opt#1[#2]{%
  \pgfutil@ifnextchar({\tikz@@collect@coordinate#1[#2]}
\def\tikz@@collect@coordinate#1[#2](#3){%
  \tikz@collect@label@onpath#1node[shape=coordinate,#2](#3){}}

\def\tikz@collect@label@onpath#1node{%
  \expandafter\def\expandafter\tikz@collected@onpath\expandafter{\tikz@collected@onpath node}%
  \tikz@collect@label@scan#1}

\def\tikz@collect@label@scan#1{%  
  \pgfutil@ifnextchar({\tikz@collect@paran#1}%
  {\pgfutil@ifnextchar[{\tikz@collect@options#1}%
    {\pgfutil@ifnextchar\bgroup{\tikz@collect@arg#1}%
      {#1}}}%
}%}}

\def\tikz@collect@paran#1(#2){%
  \expandafter\def\expandafter\tikz@collected@onpath\expandafter{\tikz@collected@onpath(#2)}%
  \tikz@collect@label@scan#1%
}
\def\tikz@collect@options#1[#2]{%
  \expandafter\def\expandafter\tikz@collected@onpath\expandafter{\tikz@collected@onpath[#2]}%
  \tikz@collect@label@scan#1%
}
\def\tikz@collect@arg#1#2{%
  \expandafter\def\expandafter\tikz@collected@onpath\expandafter{\tikz@collected@onpath{#2}}%
  #1%
}


\def\tikz@invoke@collected@onpath{%
  \tikz@node@is@a@labeltrue%
  \let\tikz@temp=\tikz@collected@onpath%
  \let\tikz@collected@onpath=\pgfutil@empty%
  \expandafter\tikz@scan@next@command\tikz@temp\pgf@stop%
  \tikz@node@is@a@labelfalse%
}




% Syntax for lineto: 
% -- <point>

\def\tikz@lineto{%
  \pgfutil@ifnextchar |%
  {\expandafter\tikz@hv@lineto\pgfutil@gobble}%
  {\expandafter\pgfutil@ifnextchar\tikz@activebar{\expandafter\tikz@hv@lineto\pgfutil@gobble}%
    {\expandafter\tikz@lineto@mid\pgfutil@gobble}}}
\def\tikz@lineto@mid{%
  \pgfutil@ifnextchar n{\tikz@collect@label@onpath\tikz@lineto@mid}%
  {%
    \pgfutil@ifnextchar c{\tikz@close}{%
      \pgfutil@ifnextchar p{\pgfsetlinetofirstplotpoint\expandafter\tikz@plot\pgfutil@gobble}%
        {\tikz@scan@one@point{\tikz@@lineto}}}}}
\def\tikz@@lineto#1{%
  % Record the starting point for later labels on the path:
  \edef\tikz@timer@start{\noexpand\pgfqpoint{\the\tikz@lastx}{\the\tikz@lasty}}
  \iftikz@shapeborder%
    % ok, target is a shape. recalculate end
    \pgf@process{\pgfpointshapeborder{\tikz@shapeborder@name}{\tikz@last@position}}%
    \tikz@make@last@position{\pgfqpoint{\pgf@x}{\pgf@y}}%
    \tikz@flush@moveto@toward{\tikz@last@position}\pgf@x\pgf@y%
    \tikz@path@lineto{\tikz@last@position}%
    \edef\tikz@timer@end{\noexpand\pgfqpoint{\the\tikz@lastx}{\the\tikz@lasty}}%
    \tikz@make@last@position{#1}%
    \edef\tikz@moveto@waiting{\tikz@shapeborder@name}%    
  \else%
    % target is a reasonable point...
    % Record the starting point for later labels on the path:
    \tikz@make@last@position{#1}%
    \tikz@flush@moveto@toward{\tikz@last@position}\pgf@x\pgf@y%
    \tikz@path@lineto{\tikz@last@position}%
    \edef\tikz@timer@end{\noexpand\pgfqpoint{\the\tikz@lastx}{\the\tikz@lasty}}%
  \fi%
  \let\tikz@timer=\tikz@timer@line%
  \tikz@scan@next@command%
}

% snake or lineto?
\def\tikz@path@lineto#1{%
  \iftikz@snaked%
    {
      \pgfsyssoftpathmovetorelevantfalse%
      \pgfpathsnakesto{\tikz@presnake,{\tikz@snake}{\tikz@mainsnakelength},\tikz@postsnake}{#1}%
    }
  \else%
    \pgfpathlineto{#1}%
  \fi%
}

% snake or lineto?
\def\tikz@path@close#1{%
  \iftikz@snaked%
    {%
      \pgftransformreset%
      \pgfpathsnakesto{\tikz@presnake,{\tikz@snake}{\tikz@mainsnakelength},\tikz@postsnake}{#1}%
    }%
    \pgfpathclose%
  \else%
    \pgfpathclose%
  \fi%
}


% Syntax for lineto horizontal/vertical: 
% -| <point>

\def\tikz@hv@lineto{%
  \pgfutil@ifnextchar n
  {\tikz@collect@label@onpath\tikz@hv@lineto}
  {\pgfutil@ifnextchar c{\tikz@collect@coordinate@onpath\tikz@hv@lineto}%
    {\tikz@scan@one@point{\tikz@@hv@lineto}}}}
\def\tikz@@hv@lineto#1{%
  \edef\tikz@timer@start{\noexpand\pgfqpoint{\the\tikz@lastx}{\the\tikz@lasty}}%
  \pgf@yc=\tikz@lasty%
  \tikz@make@last@position{#1}%
  \tikz@flush@moveto@toward{\pgfqpoint{\tikz@lastx}{\pgf@yc}}\pgf@x\pgf@yc%
  \iftikz@shapeborder%
    % ok, target is a shape. have to work now:
    {%
      \pgf@process{\pgfpointshapeborder{\tikz@shapeborder@name}{\pgfqpoint{\tikz@lastx}{\pgf@yc}}}%
      \tikz@make@last@position{\pgfqpoint{\pgf@x}{\pgf@y}}%
      \tikz@path@lineto{\pgfqpoint{\tikz@lastx}{\pgf@yc}}%
      \tikz@path@lineto{\tikz@last@position}%
      \xdef\tikz@timer@end@temp{\noexpand\pgfqpoint{\the\tikz@lastx}{\the\tikz@lasty}}% move out of group
    }%
    \let\tikz@timer@end=\tikz@timer@end@temp%
    \edef\tikz@moveto@waiting{\tikz@shapeborder@name}%    
  \else%
    \tikz@path@lineto{\pgfqpoint{\tikz@lastx}{\pgf@yc}}%
    \tikz@path@lineto{\tikz@last@position}%
    \edef\tikz@timer@end{\noexpand\pgfqpoint{\the\tikz@lastx}{\the\tikz@lasty}}% move out of group
  \fi%
  \let\tikz@timer=\tikz@timer@hvline%
  \tikz@scan@next@command%
}

% Syntax for lineto vertical/horizontal: 
% |- <point>

\def\tikz@vh@lineto-{\tikz@vh@lineto@next}
\def\tikz@vh@lineto@next{%
  \pgfutil@ifnextchar n
  {\tikz@collect@label@onpath\tikz@vh@lineto@next}
  {\pgfutil@ifnextchar c{\tikz@collect@coordinate@onpath\tikz@vh@lineto@next}%
    {\tikz@scan@one@point\tikz@@vh@lineto}}}
\def\tikz@@vh@lineto#1{%
  \edef\tikz@timer@start{\noexpand\pgfqpoint{\the\tikz@lastx}{\the\tikz@lasty}}%
  \pgf@xc=\tikz@lastx%
  \tikz@make@last@position{#1}%
  \tikz@flush@moveto@toward{\pgfqpoint{\pgf@xc}{\tikz@lasty}}\pgf@xc\pgf@y%
  \iftikz@shapeborder%
    % ok, target is a shape. have to work now:
    {%
      \pgf@process{\pgfpointshapeborder{\tikz@shapeborder@name}{\pgfqpoint{\pgf@xc}{\tikz@lasty}}}%
      \tikz@make@last@position{\pgfqpoint{\pgf@x}{\pgf@y}}%
      \tikz@path@lineto{\pgfqpoint{\pgf@xc}{\tikz@lasty}}%
      \tikz@path@lineto{\tikz@last@position}%
      \xdef\tikz@timer@end@temp{\noexpand\pgfqpoint{\the\tikz@lastx}{\the\tikz@lasty}}% move out of group
    }%
    \let\tikz@timer@end=\tikz@timer@end@temp%
    \edef\tikz@moveto@waiting{\tikz@shapeborder@name}%    
  \else%
    \tikz@path@lineto{\pgfqpoint{\pgf@xc}{\tikz@lasty}}%
    \tikz@path@lineto{\tikz@last@position}%
    \edef\tikz@timer@end{\noexpand\pgfqpoint{\the\tikz@lastx}{\the\tikz@lasty}}%
  \fi%
  \let\tikz@timer=\tikz@timer@vhline%
  \tikz@scan@next@command%
}

% Syntax for cycle: 
% -- cycle
\def\tikz@close c{%
  \pgfutil@ifnextchar o{\tikz@collect@coordinate@onpath\tikz@lineto@mid c}% oops, a coordinate
  {\tikz@@close c}}%
\def\tikz@@close cycle{%
  \tikz@flush@moveto%
  \tikz@path@close{\expandafter\pgfpoint\pgfsyssoftpath@lastmoveto}%
  \def\pgfstrokehook{}%
  \let\tikz@timer=\@undefined%
  \tikz@scan@next@command%
}


% Syntax for options: 
% [options]
\def\tikz@parse@options#1]{%
  \tikzset{#1}%
  \tikz@scan@next@command%
}

% Syntax for edges:
% edge [options] (coordinate)
% edge [options] node {node text} (coordinate)
\def\tikz@edge@plain{%
  \begingroup%
    \tikz@to@use@whom%
    \let\tikz@to@or@edge@function=\tikz@do@edge%
    \tikz@to@or@edge}

% Syntax for to paths:
% to [options] (coordinate)
% to [options] node {node text} (coordinate)
\def\tikz@to o{%
  \tikz@to@use@last@coordinate%
  \let\tikz@to@or@edge@function=\tikz@do@to%
  \tikz@to@or@edge}
  
\def\tikz@to@or@edge{\pgfutil@ifnextchar[\tikz@@to@or@edge{\tikz@@to@or@edge[]}}%}
\def\tikz@@to@or@edge[#1]{%
  \def\tikz@@to@local@options{[#1]}%
  \let\tikz@collected@onpath=\pgfutil@empty%
  \tikz@@to@collect%
}
\def\tikz@@to@collect{%
  \pgfutil@ifnextchar(\tikz@@to@or@edge@coordinate
  {\pgfutil@ifnextchar n{\tikz@collect@label@onpath\tikz@@to@collect}%
    {\pgfutil@ifnextchar c{\tikz@collect@coordinate@onpath\tikz@@to@collect}
      {\PackageError{tikz}{( expected}{}%}
        \tikz@@to@or@edge@coordinate()}}}%
}

\def\tikz@@to@or@edge@coordinate(#1){%
  \def\tikztotarget{#1}%
  \tikz@to@or@edge@function%
}

\def\tikz@do@edge{%
  \setbox\tikz@figbox=\hbox\bgroup%
    \unhbox\tikz@figbox%
    \hbox\bgroup
      \bgroup%
        \pgfinterruptpath%
          \pgfscope%
            \let\tikz@transform=\pgfutil@empty%
            \let\tikz@options=\pgfutil@empty%
            \let\tikz@tonodes=\tikz@collected@onpath%
            \def\tikztonodes{{\pgfextra{\tikz@node@is@a@labeltrue}\tikz@tonodes}}%
            \let\tikz@collected@onpath=\pgfutil@empty%
            \tikz@options%
            \tikz@transform%            
            % Typeset node:
            \tikz@atbegin@to%
            \path[style=every edge]\tikz@@to@local@options(\tikztostart)\tikz@to@path;%
            \tikz@atend@to%
          \endpgfscope%
        \endpgfinterruptpath%
      \egroup
    \egroup%
  \egroup%
    \global\setbox\tikz@tempbox=\copy\tikz@figbox%
  \endgroup%
  \setbox\tikz@figbox=\box\tikz@tempbox%  
  \tikz@scan@next@command%  
}

\def\tikz@do@to{%
  \let\tikz@tonodes=\tikz@collected@onpath%
  \def\tikztonodes{{\pgfextra{\tikz@node@is@a@labeltrue}\tikz@tonodes}}%
  \let\tikz@collected@onpath=\pgfutil@empty%
  \tikz@scan@next@command%
  \pgfextra{\tikz@atbegin@to}%
  [style=every to]\tikz@@to@local@options\tikz@to@path%
  \pgfextra{\tikz@atend@to}%
}


\def\tikz@to@use@last@coordinate{%
  \iftikz@shapeborder%
    \edef\tikztostart{\tikz@shapeborder@name}%
  \else%
    \edef\tikztostart{\the\tikz@lastx,\the\tikz@lasty}%
  \fi%
}
\def\tikz@to@use@last@fig@name{%
  \edef\tikztostart{\tikz@to@last@fig@name}%
}



% Syntax for edge from parent: 
% edge from parent [options]
\def\tikz@edgetoparent from parent{\pgfutil@ifnextchar[\tikz@@edgetoparent{\tikz@@edgetoparent[]}}%}
\def\tikz@@edgetoparent[#1]{%
  \let\tikz@edge@to@parent@needed=\pgfutil@empty%
  \tikz@node@is@a@labeltrue%
  \tikz@scan@next@command [style=edge from parent,#1] \tikz@edge@to@parent@path%
}


% Syntax for bezier curves
% .. controls(point) and (point) .. (target)
% .. controls(point) .. (target) 
% .. (target) % currently not supported

\def\tikz@dot.{\tikz@@dot}%
\def\tikz@@dot{%
  \pgfutil@ifnextchar n%
  {\tikz@collect@label@onpath\tikz@@dot}%
  {\pgfutil@ifnextchar c{\tikz@curveto@double}{\tikz@curveto@auto}}}

\def\tikz@curveto@double co{%
  \pgfutil@ifnextchar o{\tikz@collect@coordinate@onpath\tikz@@dot co}
  {\tikz@cureveto@@double}}
\def\tikz@cureveto@@double ntrols#1{%
  \tikz@scan@one@point\tikz@curveA#1%
}
\def\tikz@curveA#1{%
  \edef\tikz@timer@start{\noexpand\pgfqpoint{\the\tikz@lastx}{\the\tikz@lasty}}%
  {%
    \tikz@make@last@position{#1}%
    \xdef\tikz@curve@first{\noexpand\pgfqpoint{\the\tikz@lastx}{\the\tikz@lasty}}%
  }%
  \pgfutil@ifnextchar a
  {\tikz@curveBand}%
  {\let\tikz@curve@second\tikz@curve@first\tikz@curveCdots}%
}
\def\tikz@curveBand and{%
  \tikz@scan@one@point\tikz@curveB%
}
\def\tikz@curveB#1{%
  \def\tikz@curve@second{#1}%
  \tikz@curveCdots}
\def\tikz@curveCdots{%
  \afterassignment\tikz@curveCdot\let\@next=}
\def\tikz@curveCdot.{%
  \ifx\@next.%
  \else%
    \PackageError{tikz}{Dot expected}{}%
  \fi%
  \tikz@updatecurrenttrue%
  \tikz@curveCcheck%
}
\def\tikz@curveCcheck{%
  \pgfutil@ifnextchar n%
  {\tikz@collect@label@onpath\tikz@curveCcheck}
  {\pgfutil@ifnextchar c{\tikz@collect@coordinate@onpath\tikz@curveCcheck}
    {\tikz@scan@one@point\tikz@curveC}}%
}
\def\tikz@curveC#1{%
  \tikz@make@last@position{#1}%
  \edef\tikz@curve@third{\noexpand\pgfqpoint{\the\tikz@lastx}{\the\tikz@lasty}}%
  {%
    \tikz@lastxsaved=\tikz@lastx%
    \tikz@lastysaved=\tikz@lasty%
    \tikz@make@last@position{\tikz@curve@second}%
    \xdef\tikz@curve@second{\noexpand\pgfqpoint{\the\tikz@lastx}{\the\tikz@lasty}}%
  }%
  %
  % Start recalculating things in case start and end are shapes.
  %
  % First, the start:
  \ifx\tikz@moveto@waiting\relax%
  \else%
    \pgf@process{\pgfpointshapeborder{\tikz@moveto@waiting}{\tikz@curve@first}}%
    \edef\tikz@timer@start{\noexpand\pgfqpoint{\the\pgf@x}{\the\pgf@y}}%
    \pgfpathmoveto{\pgfqpoint{\pgf@x}{\pgf@y}}%
  \fi%
  \let\tikz@timer@cont@one=\tikz@curve@first%
  \let\tikz@timer@cont@two=\tikz@curve@second%    
  % Second, the end:
  \iftikz@shapeborder%
    % ok, target is a shape. recalculate third
    {%
      \pgf@process{\pgfpointshapeborder{\tikz@shapeborder@name}{\tikz@curve@second}}%
      \tikz@make@last@position{\pgfqpoint{\pgf@x}{\pgf@y}}%
      \edef\tikz@curve@third{\noexpand\pgfqpoint{\the\tikz@lastx}{\the\tikz@lasty}}%
      \pgfpathcurveto{\tikz@curve@first}{\tikz@curve@second}{\tikz@curve@third}%
      \global\let\tikz@timer@end@temp=\tikz@curve@third% move out of group
    }%
    \let\tikz@timer@end=\tikz@timer@end@temp%
    \edef\tikz@moveto@waiting{\tikz@shapeborder@name}%    
  \else%
    \pgfpathcurveto{\tikz@curve@first}{\tikz@curve@second}{\tikz@curve@third}%
    \let\tikz@timer@end=\tikz@curve@third
    \let\tikz@moveto@waiting=\relax%
  \fi%
  \let\tikz@timer=\tikz@timer@curve%  
  \tikz@scan@next@command%
}


% Syntax for rectangles: 
% rectangle <corner point> 
\def\tikz@rect ectangle{%
  \tikz@flush@moveto%
  \edef\tikz@timer@start{\noexpand\pgfqpoint{\the\tikz@lastx}{\the\tikz@lasty}}%
  \tikz@@rect}%
\def\tikz@@rect{%
  \pgfutil@ifnextchar n
  {\tikz@collect@label@onpath\tikz@@rect}
  {\pgfutil@ifnextchar c{\tikz@collect@coordinate@onpath\tikz@@rect}%
    {
      \pgf@xa=\tikz@lastx\relax%
      \pgf@ya=\tikz@lasty\relax%
      \tikz@scan@one@point\tikz@rectB}}}
\def\tikz@rectB#1{%
  \tikz@make@last@position{#1}%
  \edef\tikz@timer@end{\noexpand\pgfqpoint{\the\tikz@lastx}{\the\tikz@lasty}}%
  \let\tikz@timer=\tikz@timer@line%  
  \pgfpathmoveto{\pgfqpoint{\pgf@xa}{\pgf@ya}}%
  \tikz@path@lineto{\pgfqpoint{\pgf@xa}{\tikz@lasty}}%
  \tikz@path@lineto{\pgfqpoint{\tikz@lastx}{\tikz@lasty}}%
  \tikz@path@lineto{\pgfqpoint{\tikz@lastx}{\pgf@ya}}%
  \iftikz@snaked% 
    \tikz@path@lineto{\pgfqpoint{\pgf@xa}{\pgf@ya}}%
  \fi%
  \pgfpathclose%
  \pgfpathmoveto{\pgfqpoint{\tikz@lastx}{\tikz@lasty}}%
  \def\pgfstrokehook{}%
  \tikz@scan@next@command%
}



% Syntax for grids: 
% grid <corner point> 
\def\tikz@grid rid{%
  \tikz@flush@moveto%
  \pgf@xa=\tikz@lastx\relax%
  \pgf@ya=\tikz@lasty\relax%
  \pgfutil@ifnextchar[{\tikz@gridA}{\tikz@gridA[]}}%}
\def\tikz@gridA[#1]{%
  \def\tikz@grid@options{#1}%
  \tikz@scan@one@point\tikz@gridB}%
\def\tikz@gridB#1{%
  \tikz@make@last@position{#1}%
  {%
    \expandafter\tikzset\expandafter{\tikz@grid@options}
    \tikz@checkunit{\tikz@grid@x}%
    \iftikz@isdimension%
      \pgf@process{\pgfpoint{\tikz@grid@x}{0pt}}%
    \else%
      \pgf@process{\pgfpointxy{\tikz@grid@x}{0}}%
    \fi%
    \pgf@xb=\pgf@x%
    \pgf@yb=\pgf@y%
    \tikz@checkunit{\tikz@grid@y}%
    \iftikz@isdimension%
      \pgf@process{\pgfpoint{0pt}{\tikz@grid@y}}%
    \else%
      \pgf@process{\pgfpointxy{0}{\tikz@grid@y}}%
    \fi%
    \advance\pgf@xb by\pgf@x%
    \advance\pgf@yb by\pgf@y%
    \pgfpathgrid[stepx=\pgf@xb,stepy=\pgf@yb]%
      {\pgfqpoint{\pgf@xa}{\pgf@ya}}{\pgfqpoint{\tikz@lastx}{\tikz@lasty}}%
  }
  \tikz@scan@next@command%
}



% Syntax for plot: 
% plot [local options] ...    % starts with a moveto
% -- plot [local options] ... % starts with a lineto
\def\tikz@plot lot{%
  \tikz@flush@moveto%
  \pgfutil@ifnextchar[{\tikz@@plot}{\tikz@@plot[]}}%}
\def\tikz@@plot[#1]{%
  \begingroup%
    \let\tikz@options=\pgfutil@empty%
    \tikzset{every plot/.try}%
    \tikzset{#1}%
    \pgfutil@ifnextchar f{\tikz@plot@f}%
    {\pgfutil@ifnextchar c{\tikz@plot@scan@points}%
      {\pgfutil@ifnextchar ({\tikz@plot@expression}{%
      \PackageError{tikz}{Cannot parse this plotting data}{}%
       \endgroup}}}}
\def\tikz@plot@f f{\pgfutil@ifnextchar i{\tikz@plot@file}{\tikz@plot@function}}

\def\tikz@plot@file ile#1{\def\tikz@plot@data{\pgfplotxyfile{#1}}\tikz@@@plot}%
\def\tikz@plot@scan@points coordinates#1{%
  \pgfplothandlerrecord\tikz@plot@data%
  \pgfplotstreamstart%
  \pgfutil@ifnextchar\pgf@stop{\pgfplotstreamend\expandafter\tikz@@@plot\pgfutil@gobble}
  {\tikz@scan@one@point\tikz@plot@next@point}%
  #1\pgf@stop%
}
\def\tikz@plot@next@point#1{%
  \pgfplotstreampoint{#1}%
  \pgfutil@ifnextchar\pgf@stop{\pgfplotstreamend\expandafter\tikz@@@plot\pgfutil@gobble}%
  {\tikz@scan@one@point\tikz@plot@next@point}%
}  
\def\tikz@plot@function unction#1{%
  \def\tikz@plot@filename{\tikz@plot@prefix\tikz@plot@id}%  
  \iftikz@plot@raw@gnuplot%
    \def\tikz@plot@data{\pgfplotgnuplot[\tikz@plot@filename]{#1}}%
  \else%
    \iftikz@plot@parametric%   
      \def\tikz@plot@data{\pgfplotgnuplot[\tikz@plot@filename]{%
          set samples \tikz@plot@samples;
          set parametric;
          plot [t=\tikz@plot@domain] #1}}%
    \else%
      \def\tikz@plot@data{\pgfplotgnuplot[\tikz@plot@filename]{%
          set samples \tikz@plot@samples;
          plot [x=\tikz@plot@domain] #1}}%
    \fi%
  \fi%
  \tikz@@@plot%
}

\def\tikz@plot@no@resample{%
  \pgfutil@IfFileExists{\tikz@plot@filename.table}%
  {\def\tikz@plot@data{\pgfplotxyfile{\tikz@plot@filename.table}}}%
  {}%
}

\def\tikz@plot@expression(#1){%
  \edef\tikz@plot@data{\noexpand\pgfplotfunction{\expandafter\noexpand\tikz@plot@var}{\tikz@plot@samplesat}}%
  \expandafter\def\expandafter\tikz@plot@data\expandafter{\tikz@plot@data{\tikz@scan@one@point\pgfutil@firstofone(#1)}}%
  \tikz@@@plot%
}

\def\tikz@@@plot{%
    \def\pgfplotlastpoint{\pgfpointorigin}%
    \tikz@plot@handler%
    \tikz@plot@data%
    \global\let\tikz@@@temp=\pgfplotlastpoint%
    \ifx\tikz@plot@mark\pgfutil@empty%
    \else%
      % Marks are drawn after the path.
      \setbox\tikz@figbox=\hbox{%
        \unhbox\tikz@figbox%
        \hbox{{%
          \pgfinterruptpath%
            \pgfscope%
              \let\tikz@options=\pgfutil@empty%
              \let\tikz@transform=\pgfutil@empty%
              \expandafter\tikzset\expandafter{\tikz@plot@mark@options}%
              \tikz@options%
              \ifx\tikz@mark@list\pgfutil@empty%
                \pgfplothandlermark{\tikz@transform\pgfuseplotmark{\tikz@plot@mark}}%
              \else
                \pgfplothandlermarklisted{\tikz@transform\pgfuseplotmark{\tikz@plot@mark}}{\tikz@mark@list}%
              \fi
              \tikz@plot@data%
            \endpgfscope
          \endpgfinterruptpath%
        }}%
      }%
    \fi%
    \global\setbox\tikz@tempbox=\copy\tikz@figbox%
  \endgroup%
  \setbox\tikz@figbox=\box\tikz@tempbox%  
  \tikz@make@last@position{\tikz@@@temp}%  
  \tikz@scan@next@command%
}


\pgfdeclareplotmark{ball}
{%
  \def\tikz@shading{ball}%
  \shade (0,0) circle (\pgfplotmarksize);%
}




% Syntax for cosine curves:
% cos <end of quarter-period>
\def\tikz@cosine s{\tikz@scan@one@point\tikz@@cosine}
\def\tikz@@cosine#1{%
  \tikz@flush@moveto%
  \pgf@process{#1}%
  \pgf@xc=\pgf@x%
  \pgf@yc=\pgf@y%
  \advance\pgf@xc by-\tikz@lastx%
  \advance\pgf@yc by-\tikz@lasty%
  \advance\tikz@lastx by\pgf@xc%
  \advance\tikz@lasty by\pgf@yc%
  \tikz@lastxsaved=\tikz@lastx%
  \tikz@lastysaved=\tikz@lasty%
  \tikz@updatecurrenttrue%
  \pgfpathcosine{\pgfqpoint{\pgf@xc}{\pgf@yc}}%
  \tikz@scan@next@command%
}

% Syntax for sine curves:
% sin <end of quarter-period>
\def\tikz@sine in{\tikz@scan@one@point\tikz@@sine}
\def\tikz@@sine#1{%
  \tikz@flush@moveto%
  \pgf@process{#1}%
  \pgf@xc=\pgf@x%
  \pgf@yc=\pgf@y%
  \advance\pgf@xc by-\tikz@lastx%
  \advance\pgf@yc by-\tikz@lasty%
  \advance\tikz@lastx by\pgf@xc%
  \advance\tikz@lasty by\pgf@yc%
  \tikz@lastxsaved=\tikz@lastx%
  \tikz@lastysaved=\tikz@lasty%
  \tikz@updatecurrenttrue%
  \pgfpathsine{\pgfqpoint{\pgf@xc}{\pgf@yc}}%
  \tikz@scan@next@command%
}

% Syntax for parabolas: 
% parabola[options] bend <coordinate> <coordinate>
\def\tikz@parabola arabola

\def\tikz@parabola@options[#1]{%
  \def\tikz@parabola@option{#1}%
  \pgfutil@ifnextchar b{\tikz@parabola@scan@bend}{\tikz@scan@one@point\tikz@parabola@semifinal}}
\def\tikz@parabola@scan@bend bend{\tikz@scan@one@point\tikz@parabola@scan@bendB}
\def\tikz@parabola@scan@bendB#1{%
  \def\tikz@parabola@bend{#1}%
  \tikz@scan@one@point\tikz@parabola@semifinal%
}
\def\tikz@parabola@semifinal#1{%
  \tikz@flush@moveto%
  % Save original start:
  \pgf@xb=\tikz@lastx%
  \pgf@yb=\tikz@lasty%
  \tikz@make@last@position{#1}%
  \pgf@xc=\tikz@lastx%
  \pgf@yc=\tikz@lasty%
  \begingroup% now calculate bend:
    \expandafter\tikzset\expandafter{\tikz@parabola@option}%
    \tikz@lastxsaved=\tikz@parabola@bend@factor\tikz@lastx%
    \tikz@lastysaved=\tikz@parabola@bend@factor\tikz@lasty%
    \advance\tikz@lastxsaved by\pgf@xb%
    \advance\tikz@lastysaved by\pgf@yb%
    \advance\tikz@lastxsaved by-\tikz@parabola@bend@factor\pgf@xb%
    \advance\tikz@lastysaved by-\tikz@parabola@bend@factor\pgf@yb%
    \expandafter\tikz@make@last@position\expandafter{\tikz@parabola@bend}%
    % Calculate delta from bend
    \advance\pgf@xc by-\tikz@lastx%
    \advance\pgf@yc by-\tikz@lasty%
    % Ok, now calculate delta to bend
    \advance\tikz@lastx by-\pgf@xb%
    \advance\tikz@lasty by-\pgf@yb%
    \xdef\tikz@parabola@b{{\noexpand\pgfqpoint{\the\tikz@lastx}{\the\tikz@lasty}}{\noexpand\pgfqpoint{\the\pgf@xc}{\the\pgf@yc}}}%
  \endgroup%
  \expandafter\pgfpathparabola\tikz@parabola@b%
  \tikz@scan@next@command%
}


% Syntax for circles:
% circle (radius)
%
% Syntax for ellipses:
% ellipse (x-radius and y-radius)
%
% radii can be dimensionless, then they are in the xy-system
\def\tikz@circle ircle{\tikz@flush@moveto\tikz@@circle}
\def\tikz@ellipse llipse{\tikz@flush@moveto\tikz@@circle}
\def\tikz@@circle{%
  \pgfutil@ifnextchar(\tikz@@@circle{%)
    \advance\tikz@expandcount by -1%
    \ifnum\tikz@expandcount<0\relax%
      \let\@next=\tikz@@circle@scangiveup%
    \else%
      \let\@next=\tikz@@circle@scanexpand%
    \fi%
    \@next%
  }%
}
\def\tikz@@circle@scanexpand{\expandafter\tikz@@circle}
\def\tikz@@circle@scangiveup#1{\PackageError{tikz}{Cannot parse this radius}{}#1{\tikz@scan@next@command}}
\def\tikz@@@circle(#1){%
  \pgfutil@in@{ and }{#1}%
  \ifpgfutil@in@%
    \tikz@@ellipseB(#1)%
  \else%
    \tikz@@ellipseB(#1 and #1)%
  \fi%
  \tikz@scan@next@command%
}
\def\tikz@@ellipseB(#1 and #2){%
  \tikz@checkunit{#1}%
  \iftikz@isdimension%
    \pgfpathellipse{\tikz@last@position}{\pgfpoint{#1}{0pt}}{\pgfpoint{0pt}{#2}}%
  \else%
    \pgfpathellipse{\tikz@last@position}{\pgfpointxy{#1}{0}}{\pgfpointxy{0}{#2}}%
  \fi%
}

% Syntax 1 for arcs:
% arc (start angle:end angle:radius)
%
% Syntax 2 for arcs:
% arc (start angle:end angle:x-radius and y-radius)
%
% radius can be dimensionless, then the arc is in the xy-coordinate system
\def\tikz@arcA rc{%
  \tikz@flush@moveto%
  \pgfutil@ifnextchar({\tikz@@arcto}{\expandafter\tikz@arcA\expandafter r\expandafter c}}

\def\tikz@@arcto(#1){%
  \edef\tikz@temp{(#1)}%
   \expandafter\tikz@@@arcto@check@slashand\tikz@temp%
}

\def\tikz@@@arcto@check@slashand(#1:#2:#3){%
  \pgfutil@in@{ and }{#3}%
  \ifpgfutil@in@% 
    \tikz@parse@arc@and(#1:#2:#3)%
  \else%
    \tikz@parse@arc@and(#1:#2:#3 and #3)%
  \fi%
}

\def\tikz@parse@arc@and(#1:#2:#3 and #4){%
  \tikz@checkunit{#3}%
  \iftikz@isdimension%
    \tikz@@@arcfinal{\pgfpatharc{#1}{#2}{#3 and #4}}
    {\pgfpointpolar{#1}{#3 and #4}}
    {\pgfpointpolar{#2}{#3 and #4}}%
  \else%
    \tikz@@@arcfinal{\pgfpatharcaxes{#1}{#2}{\pgfpointxy{#3}{0}}{\pgfpointxy{0}{#4}}}
    {\pgfpointpolarxy{#1}{#3 and #4}}{\pgfpointpolarxy{#2}{#3 and #4}}%
  \fi%
}

\def\tikz@@@arcfinal#1#2#3{%
  #1%
  \pgf@process{#2}%
  \advance\tikz@lastx by-\pgf@x%
  \advance\tikz@lasty by-\pgf@y%
  \pgf@process{#3}%
  \advance\tikz@lastx by\pgf@x%
  \advance\tikz@lasty by\pgf@y%
  \tikz@lastxsaved=\tikz@lastx%
  \tikz@lastysaved=\tikz@lasty%
  \tikz@scan@next@command%
}


% Syntax for coordinates:
% coordinate[options] (coordinate name) at (point)
% where ``at (point)'' is optional
\def\tikz@coordinate ordinate{%
  \pgfutil@ifnextchar[{\tikz@@coordinate@opt}{\tikz@@coordinate@opt[]}}
\def\tikz@@coordinate@opt[#1]
\def\tikz@@coordinate[#1](#2){%
  \pgfutil@ifnextchar a{\tikz@@coordinate@at[#1](#2)}
  {\tikz@fig ode[shape=coordinate,#1](#2){}}}
\def\tikz@@coordinate@at[#1](#2)at#3(#4){%
  \tikz@fig ode[shape=coordinate,#1](#2)at(#4){}}
  


% Syntax for nodes:
% node[options] (node name) {label text}
%
% all of [options], (node name) and {label text} are optional. There
% can be multiple options before the label text as in
% node[draw] (a) [rotate=10] {text}
%
% A label text always ``ends'' the node.
\def\tikz@fig ode{%
  \edef\tikz@save@line@width{\the\pgflinewidth}%
  \begingroup%
  \let\tikz@fig@name=\pgfutil@empty%
    \begingroup%
      \tikz@is@matrixfalse%
      \let\nodepart=\tikz@nodepart%
      \let\tikz@options=\pgfutil@empty%
      \let\tikz@after@node=\pgfutil@empty%
      \let\tikz@afternodepathoptions=\pgfutil@empty%
      \let\tikz@transform=\pgfutil@empty%
      \let\tikz@mode=\pgfutil@empty%
      \def\tikz@node@at{\pgfqpoint{\the\tikz@lastx}{\the\tikz@lasty}}%
      \iftikz@node@is@a@label%
      \else%
        \let\tikz@time=\pgfutil@empty%
      \fi%
      \tikzset{every node/.try}%
      \tikz@@scan@fig}%
\def\tikz@@scan@fig{%
  \pgfutil@ifnextchar a{\tikz@fig@scan@at}
  {\pgfutil@ifnextchar({\tikz@fig@scan@name}
    {\pgfutil@ifnextchar[{\tikz@fig@scan@options}%
      {\pgfutil@ifnextchar\bgroup{\tikz@fig@main}%
      {\PackageError{tikz}{A node must have a (possibly empty) label text}{}%
       \tikz@fig@main{}}}}}}%}}
\def\tikz@fig@scan@at at{%
  \tikz@scan@one@point\tikz@@fig@scan@at}
\def\tikz@@fig@scan@at#1{%
  \def\tikz@node@at{#1}\tikz@@scan@fig}%
\def\tikz@fig@scan@name(#1){\edef\tikz@fig@name{#1}\tikz@@scan@fig}%
\def\tikz@fig@scan@options[#1]{\tikzset{#1}\def\test{#1}\tikz@@scan@fig}%
\def\tikz@fig@main{\afterassignment\tikz@@fig@main\let\next=}
\def\tikz@@fig@main{%
    \pgfutil@ifundefined{pgf@sh@s@\tikz@shape}%
    {\PackageError{tikz}%
      {Unknown shape ``\tikz@shape.'' Using ``rectangle'' instead}{}%
      \def\tikz@shape{rectangle}}%
    {}%
    \tikzset{every \tikz@shape\space node/.try}%
    \iftikz@is@matrix%
      \let\tikz@next=\tikz@do@matrix%
    \else%
      \let\tikz@next=\tikz@do@fig%
    \fi%
    \tikz@next%  
}
\def\tikz@do@fig{%  
    \setbox\pgfnodeparttextbox=\hbox%
      \bgroup%
        \tikzset{every text node part/.try}%
        \ifx\tikz@textopacity\pgfutil@empty%
        \else%
          \pgfsetfillopacity{\tikz@textopacity}%
          \pgfsetstrokeopacity{\tikz@textopacity}%
        \fi%
        \pgfinterruptpicture%
          \tikz@textfont%  
          \ifx\tikz@text@width\pgfutil@empty%
          \else%
            \begingroup%
              \pgfutil@minipage[t]{\tikz@text@width}%
                \tikz@text@action%
          \fi%
          \tikz@atbegin@node%
          \bgroup%
            \aftergroup\unskip%
            \ifx\tikz@textcolor\pgfutil@empty%
            \else%
              \pgfutil@colorlet{.}{\tikz@textcolor}%
            \fi%
            \pgfsetcolor{.}%
            \setbox\tikz@figbox=\box\voidb@x%
            \tikz@uninstallcommands%
            \aftergroup\tikz@fig@collectresetcolor%
            \ignorespaces%
}
\def\tikz@fig@collectresetcolor{%
  \pgfutil@ifnextchar\reset@color%
  {\reset@color\afterassignment\tikz@fig@collectresetcolor\let\tikz@temp=}%
  {\tikz@fig@boxdone}%
}
\def\tikz@fig@boxdone{%
            \tikz@atend@node%
          \ifx\tikz@text@width\pgfutil@empty%
          \else%
              \pgfutil@endminipage%
            \endgroup%
          \fi%
        \endpgfinterruptpicture%
      \egroup%
    \pgfutil@ifnextchar c{\tikz@fig@mustbenamed\tikz@fig@continue}%
    {\pgfutil@ifnextchar[{\tikz@fig@mustbenamed\tikz@fig@continue}%
      {\pgfutil@ifnextchar t{\tikz@fig@mustbenamed\tikz@fig@continue}
        {\pgfutil@ifnextchar e{\tikz@fig@mustbenamed\tikz@fig@continue}
          {\ifx\tikz@after@node\pgfutil@empty\expandafter\tikz@fig@continue\else\expandafter\tikz@fig@mustbenamed\expandafter\tikz@fig@continue\fi}}}}}%}

\def\tikz@do@matrix{%
    \tikzset{every matrix/.try}%
    \tikz@node@transformations%
    \tikz@fig@mustbenamed%
    \setbox\tikz@figbox=\hbox\bgroup%
      \setbox\pgfutil@tempboxa=\copy\tikz@figbox%
      \unhbox\pgfutil@tempboxa%
      \hbox\bgroup\bgroup%
          \pgfinterruptpath%
            \pgfscope%
              \tikz@options%
              \setbox\tikz@figbox=\box\voidb@x%
              \let\tikzmatrixname=\tikz@fig@name%
              \edef\tikz@m@anchor{\ifx\tikz@matrix@anchor\pgfutil@empty\tikz@anchor\else\tikz@matrix@anchor\fi}%
              \expandafter\pgfutil@in@\expandafter{\expandafter.\expandafter}\expandafter{\tikz@m@anchor}%
              \ifpgfutil@in@%
                \expandafter\tikz@matrix@split\tikz@m@anchor\relax%
              \else%
                \def\tikz@matrix@shift{\pgfpointorigin}%  
              \fi%
              \let\tikz@transform=\relax%
              \pgfmatrix%
              {\tikz@shape}%
              {\tikz@m@anchor}%
              {\tikz@fig@name}%
              {%
                \pgfutil@tempdima=\pgflinewidth%
                {\begingroup\tikz@finish}%
                \global\pgflinewidth=\pgfutil@tempdima%
              }%
              {\tikz@matrix@shift}%
              {%
                \tikz@matrix@make@active@ampersand%
                \def\pgfmatrixbegincode{%
                  \pgfsys@beginscope%
                  \tikz@common@matrix@code%
                  \tikz@atbegin@cell%
                }%
                \def\tikz@common@matrix@code{%
                  \let\tikz@options=\pgfutil@empty%
                  \let\tikz@mode=\pgfutil@empty%
                  \tikzset{every cell/.try={\the\pgfmatrixcurrentrow}{\the\pgfmatrixcurrentcolumn}}%
                  \tikzset{column \the\pgfmatrixcurrentcolumn/.try}%
                  \ifodd\pgfmatrixcurrentcolumn%
                    \tikzset{every odd column/.try}%
                  \else%
                    \tikzset{every even column/.try}%
                  \fi%
                  \tikzset{row \the\pgfmatrixcurrentrow/.try}%
                  \ifodd\pgfmatrixcurrentrow%
                    \tikzset{every odd row/.try}%
                  \else%
                    \tikzset{every even row/.try}%
                  \fi%
                  \tikzset{row \the\pgfmatrixcurrentrow\space column \the\pgfmatrixcurrentcolumn/.try}%
                  \tikz@options%
                }%
                \def\pgfmatrixendcode{%
                  \tikz@atend@cell%
                  \pgfsys@endscope%
                }%
                \def\pgfmatrixemptycode{%
                  \pgfsys@beginscope%
                  \tikz@common@matrix@code%
                  \tikz@at@emptycell%
                  \pgfsys@endscope%
                }%
                \aftergroup\tikz@do@matrix@cont}%
              \bgroup%
}
\def\tikz@do@matrix@cont{%            
            \endpgfscope
          \endpgfinterruptpath%
      \egroup\egroup%
    \egroup%
    %
    \tikz@node@finish%
}

{%
  \catcode`\&=13
  \gdef\tikz@matrix@make@active@ampersand{%
    \ifx\tikz@ampersand@replacement\pgfutil@empty%
      \catcode`\&=13%
      \let&=\pgfmatrixnextcell%
    \else%
      \expandafter\let\tikz@ampersand@replacement=\pgfmatrixnextcell%
    \fi%
  }%
}%


\def\tikz@matrix@split#1.#2\relax{%
  \def\tikz@m@anchor{text}%
  \def\tikz@matrix@shift{\pgfpointanchor{#1}{#2}}%
}
  
\def\tikz@fig@continue{%
    \ifx\tikz@text@width\pgfutil@empty%
    \else%
      \pgfmathsetlength{\pgf@x}{\tikz@text@width}%
      \wd\pgfnodeparttextbox=\pgf@x%
    \fi%
    \ifx\tikz@text@height\pgfutil@empty%
    \else%
      \pgfmathsetlength{\pgf@x}{\tikz@text@height}%
      \ht\pgfnodeparttextbox=\pgf@x%
    \fi%
    \ifx\tikz@text@depth\pgfutil@empty%
    \else%
      \pgfmathsetlength{\pgf@x}{\tikz@text@depth}%
      \dp\pgfnodeparttextbox=\pgf@x%
    \fi%
    %
    % Node transformation
    %
    \tikz@node@transformations
    %
    \setbox\tikz@figbox=\hbox{%
      \setbox\pgfutil@tempboxa=\copy\tikz@figbox%
      \unhbox\pgfutil@tempboxa%
      \hbox{{%
          \pgfinterruptpath%
            \pgfscope%
              \tikz@options%
              \setbox\tikz@figbox=\box\voidb@x%
              \pgfmultipartnode{\tikz@shape}{\tikz@anchor}{\tikz@fig@name}{%
                \pgfutil@tempdima=\pgflinewidth%
                {\begingroup\tikz@finish}%
                \global\pgflinewidth=\pgfutil@tempdima%
              }%
            \endpgfscope
          \endpgfinterruptpath%
      }}%
    }%
    %
    \tikz@node@finish%
}


\def\tikz@fig@mustbenamed{%
  \ifx\tikz@fig@name\pgfutil@empty%
    % Assign a dummy name
    \global\advance\tikz@fig@count by1\relax
    \edef\tikz@fig@name{tikz@f@\the\tikz@fig@count}%
  \fi%
}

\def\tikz@node@transformations{
  % 
  % Possibly, we are ``online''
  % 
  \ifx\tikz@time\pgfutil@empty%
    \pgftransformshift{\tikz@node@at}%
    \iftikz@fullytransformed%
    \else%
      \pgftransformresetnontranslations%
    \fi%
  \else%
    \tikz@do@auto@anchor%
    \tikz@timer%
  \fi%
  % Invoke local transformations
  \tikz@transform%
}

\def\tikz@node@finish{%  
    \global\let\tikz@last@fig@name=\tikz@fig@name%
    \global\let\tikz@after@node@smuggle=\tikz@after@node%
    \global\let\tikz@afternodepathoptions@smuggle=\tikz@afternodepathoptions%
    % shift box outside group
    \global\setbox\tikz@tempbox=\copy\tikz@figbox%
  \endgroup\endgroup%
  \setbox\tikz@figbox=\box\tikz@tempbox%
  \pgflinewidth=\tikz@save@line@width%
  \let\tikz@to@last@fig@name=\tikz@last@fig@name%
  \let\tikz@to@use@whom=\tikz@to@use@last@fig@name%
  \let\tikzlastnode=\tikz@last@fig@name%
  \ifx\tikz@after@node@smuggle\pgfutil@empty%
  \else%
    \tikz@scan@next@command{\pgfextra{\tikz@afternodepathoptions@smuggle}\tikz@after@node@smuggle}\pgf@stop%
  \fi%
  \tikz@scan@next@command%
}
\let\tikz@fig@continue@orig=\tikz@fig@continue



% Syntax for parts of  nodes:
% node ... {... \nodepart{name} ... \nodepart{name} ...}

\def\tikz@nodepart#1{%
  \tikz@atend@node%
  \unskip%
  \gdef\tikz@nodepart@name{#1}%
  \global\let\tikz@fig@continue=\tikz@nodepart@continue%
  \pgfutil@ifnextchar x{\egroup\relax}{\egroup\relax}% gobble spaces
}
\def\tikz@nodepart@continue{%
  \global\let\tikz@fig@continue=\tikz@fig@continue@orig%
  % Now start new box:
   \expandafter\setbox\csname pgfnodepart\tikz@nodepart@name box\endcsname=\hbox%
      \bgroup%
        \tikzset{every \tikz@nodepart@name\space node part/.try}%
        \pgfinterruptpicture%
          \tikz@textfont%  
          \ifx\tikz@text@width\pgfutil@empty%
          \else%
            \begingroup%
              \pgfutil@minipage[t]{\tikz@text@width}%
                \tikz@text@action%
          \fi%
          \bgroup%
            \aftergroup\unskip%
            \ifx\tikz@textcolor\pgfutil@empty%
            \else%
              \pgfutil@colorlet{.}{\tikz@textcolor}%
            \fi%
            \pgfsetcolor{.}%
            \setbox\tikz@figbox=\box\voidb@x%
            \tikz@uninstallcommands%
            \tikz@atbegin@node%
            \aftergroup\tikz@fig@collectresetcolor%
            \ignorespaces%
}


% Auto placement

\def\tikz@auto@pre{%
  \begingroup
    \pgfresetnontranslationattimefalse
    \pgfslopedattimetrue%
    \pgfallowupsidedownattimetrue%
    \tikz@timer%
    \pgf@x=\pgf@pt@aa pt% 
    \pgf@y=\pgf@pt@ab pt%
    \pgfpointnormalised{}%
}

\def\tikz@auto@post{%
    \global\let\tikz@anchor@smuggle=\tikz@anchor%
  \endgroup%
  \let\tikz@anchor=\tikz@anchor@smuggle%
}

\def\tikz@auto@anchor{%
    \ifdim\pgf@x>0.05pt%
      \ifdim\pgf@y>0.05pt%
        \def\tikz@anchor{south east}%
      \else\ifdim\pgf@y<-0.05pt%
        \def\tikz@anchor{south west}%
      \else
        \def\tikz@anchor{south}%
      \fi\fi%
    \else\ifdim\pgf@x<-0.05pt%
      \ifdim\pgf@y>0.05pt%
        \def\tikz@anchor{north east}%
      \else\ifdim\pgf@y<-0.05pt%
        \def\tikz@anchor{north west}%
      \else
        \def\tikz@anchor{north}%
      \fi\fi%
    \else%
      \ifdim\pgf@y>0pt%
        \def\tikz@anchor{east}%
      \else%
        \def\tikz@anchor{west}%
      \fi%
    \fi\fi%
}

\def\tikz@auto@anchor@prime{%
    \ifdim\pgf@x>0.05pt%
      \ifdim\pgf@y>0.05pt%
        \def\tikz@anchor{north west}%
      \else\ifdim\pgf@y<-0.05pt%
        \def\tikz@anchor{north east}%
      \else
        \def\tikz@anchor{north}%
      \fi\fi%
    \else\ifdim\pgf@x<-0.05pt%
      \ifdim\pgf@y>0.05pt%
        \def\tikz@anchor{south west}%
      \else\ifdim\pgf@y<-0.05pt%
        \def\tikz@anchor{south east}%
      \else
        \def\tikz@anchor{south}%
      \fi\fi%
    \else%
      \ifdim\pgf@y>0pt%
        \def\tikz@anchor{west}%
      \else%
        \def\tikz@anchor{east}%
      \fi%
    \fi\fi%
}




% Syntax for trees:
% node {...} child [options] {...} child [options] {...} ...
% node {...} child [options] foreach \var in {list} [options] {...} ...

\def\tikz@children{%
  % Start collecting the children:
  \let\tikz@children@list=\pgfutil@empty%
  \tikznumberofchildren=0\relax%
  \tikz@collect@children c}

\def\tikz@collect@children{\pgfutil@ifnextchar c{\tikz@collect@children@cchar}{\tikz@children@collected}}
\def\tikz@collect@children@cchar c{\pgfutil@ifnextchar h{\tikz@collect@child}{\tikz@children@collected c}}
\def\tikz@collect@child hild{\pgfutil@ifnextchar[{\tikz@collect@childA}{\tikz@collect@childA[]}}%}
\def\tikz@collect@childA[#1]{\pgfutil@ifnextchar f{\tikz@collect@children@foreach[#1]}{\tikz@collect@childB[#1]}}
\def\tikz@collect@childB[#1]{%
  \advance\tikznumberofchildren by1\relax
  \expandafter\def\expandafter\tikz@children@list\expandafter{\tikz@children@list \tikz@childnode[#1]}%
  \pgfutil@ifnextchar\bgroup{\tikz@collect@child@code}{\tikz@collect@child@code{}}}
\def\tikz@collect@child@code#1{%
  \expandafter\def\expandafter\tikz@children@list\expandafter{\tikz@children@list{#1}}%
  \tikz@collect@children%
}
\def\tikz@collect@children@foreach[#1]foreach#2in#3{%
  \pgfutil@ifnextchar\bgroup{\tikz@collect@children@foreachA{#1}{#2}{#3}}{\tikz@collect@children@foreachA{#1}{#2}{#3}{}}}
\def\tikz@collect@children@foreachA#1#2#3#4{%
  \expandafter\def\expandafter\tikz@children@list\expandafter
    {\tikz@children@list\tikz@childrennodes[#1]{#2}{#3}{#4}}%
  \c@pgf@counta=\tikznumberofchildren%
  \foreach#2in{#3}%
  {%
    \global\advance\c@pgf@counta by1\relax%
  }%
  \tikznumberofchildren=\c@pgf@counta%
  \tikz@collect@children%
}
\long\def\tikz@children@collected{%
  \begingroup%
    \advance\tikztreelevel by 1\relax%
    \let\tikz@options=\pgfutil@empty%
    \let\tikz@transform=\pgfutil@empty%
    \tikzset{level/.try=\the\tikztreelevel,level \the\tikztreelevel/.try}%
    \tikz@transform%            
    \let\tikzparentnode=\tikz@last@fig@name%
    % Transform to center of node
    \pgftransformshift{\pgfpointanchor{\tikzparentnode}{\tikz@growth@anchor}}%
    \tikznumberofcurrentchild=0\relax%
    \tikz@children@list%
    \global\setbox\tikz@tempbox=\copy\tikz@figbox%
  \endgroup%
  \setbox\tikz@figbox=\box\tikz@tempbox%  
  \tikz@scan@next@command%
}


% Syntax for children:
%
% child [all children options] foreach \var in {values} [child options] {...}
\def\tikz@childrennodes[#1]#2#3#4{%
  \c@pgf@counta=\tikznumberofcurrentchild\relax%
  \setbox\tikz@tempbox=\box\tikz@figbox%
  \foreach#2in{#3}{%
    \tikznumberofcurrentchild=\c@pgf@counta\relax%
    \setbox\tikz@figbox=\box\tikz@tempbox%
    \tikz@childnode[#1]{#4}%
    % we must now make the current child number and the figbox survive
    % the group
    \global\c@pgf@counta=\tikznumberofcurrentchild\relax%
    \global\setbox\tikz@tempbox=\box\tikz@figbox%
  }%
  \tikznumberofcurrentchild=\c@pgf@counta\relax%
  \setbox\tikz@figbox=\box\tikz@tempbox%
}


% Syntax for child:
%
% child
%
% child[options]
%
% child[options] {node (name) {child node text} ...
%   edge from parent[options] node {label text} node {label text}}

\def\tikz@childnode[#1]#2{%
  \advance\tikznumberofcurrentchild by1\relax%
  \setbox\tikz@figbox=\hbox\bgroup%
    \unhbox\tikz@figbox%
    \hbox\bgroup\bgroup%
        \pgfinterruptpath%
          \pgfscope%
            \let\tikz@transform=\pgfutil@empty%
            \tikzset{every child/.try,#1}%
            \tikz@options%
            \tikz@transform%            
            \tikz@grow%
            % Typeset node:
            \edef\tikz@parent@node@name{[name=\tikzparentnode-\the\tikznumberofcurrentchild,style=every child node]}%
            \def\tikz@child@node@text{[shape=coordinate]{}}
            \tikz@parse@child@node#2\pgf@stop%
            \expandafter\expandafter\expandafter\node
            \expandafter\tikz@parent@node@name
              \tikz@child@node@text
              \pgfextra{\global\let\tikz@childnode@name=\tikz@last@fig@name};%
            \let\tikzchildnode=\tikz@childnode@name%
            {%
              \def\tikz@edge@to@parent@needed{edge from parent}
              \ifx\tikz@child@node@rest\pgfutil@empty%
                \path edge from parent;%
              \else%
                \path (0,0) \tikz@child@node@rest \tikz@edge@to@parent@needed;%
              \fi%
            }%
        \endpgfscope%
      \endpgfinterruptpath%
    \egroup\egroup%
  \egroup%
}

\def\tikz@parse@child@node{%
  \pgfutil@ifnextchar n{\tikz@parse@child@node@n}%
  {\pgfutil@ifnextchar c{\tikz@parse@child@node@c}%
    {\tikz@parse@child@node@rest}}}
\def\tikz@parse@child@node@rest#1\pgf@stop{\def\tikz@child@node@rest{#1}}
\def\tikz@parse@child@node@c c{\pgfutil@ifnextchar o{\tikz@parse@child@node@co}{\tikz@parse@child@node@rest c}}
\def\tikz@parse@child@node@co o{\pgfutil@ifnextchar o{\tikz@parse@child@node@coordinate}{\tikz@parse@child@node@rest co}}
\def\tikz@parse@child@node@coordinate ordinate{%
  \pgfutil@ifnextchar ({\tikz@@parse@child@node@coordinate}{%
    \def\tikz@child@node@text{[shape=coordinate]{}}%
    \tikz@parse@child@node@rest}}%}
\def\tikz@@parse@child@node@coordinate(#1){%
  \pgfutil@ifnextchar a{\tikz@p@c@n@c@at(#1)}{%
    \def\tikz@child@node@text{[shape=coordinate,name=#1]{}}%
    \tikz@parse@child@node@rest}}
\def\tikz@p@c@n@c@at(#1)at#2(#3){%
  \def\tikz@child@node@text{[shape=coordinate,name=#1]at(#3){}}%
  \tikz@parse@child@node@rest}%
\def\tikz@parse@child@node@n node{%
  \let\tikz@child@node@text=\pgfutil@empty%
  \tikz@p@c@s}%
\def\tikz@p@c@s}
\def\tikz@p@c@s@at at#1(#2){%
  \expandafter\def\expandafter\tikz@child@node@text\expandafter{\tikz@child@node@text at(#2)}
  \tikz@p@c@s}
\def\tikz@p@c@s@paran(#1){%
  \expandafter\def\expandafter\tikz@child@node@text\expandafter{\tikz@child@node@text(#1)}
  \tikz@p@c@s}
\def\tikz@p@c@s@bra[#1]{%
  \expandafter\def\expandafter\tikz@child@node@text\expandafter{\tikz@child@node@text[#1]}
  \tikz@p@c@s}
\def\tikz@p@c@s@group#1{%
  \expandafter\def\expandafter\tikz@child@node@text\expandafter{\tikz@child@node@text{#1}}
  \tikz@parse@child@node@rest}


%
% Timers
% 

\def\tikz@timer@line{%
  \pgftransformlineattime{\tikz@time}{\tikz@timer@start}{\tikz@timer@end}%
}

\def\tikz@timer@vhline{%
  \ifdim\tikz@time pt<0.5pt% first half
    \pgf@process{\tikz@timer@start}%
    \pgf@xa=\pgf@x%
    \pgf@ya=\pgf@y%
    \pgf@process{\tikz@timer@end}%
    \pgf@xb=\tikz@time pt%
    \pgf@xb=2\pgf@xb%    
    \edef\tikz@marshal{\noexpand\pgftransformlineattime{\pgf@sys@tonumber{\pgf@xb}}{\noexpand\tikz@timer@start}{%
        \noexpand\pgfqpoint{\the\pgf@xa}{\the\pgf@y}}}%
    \tikz@marshal%
  \else% second half
    \pgf@process{\tikz@timer@start}%
    \pgf@xa=\pgf@x%
    \pgf@ya=\pgf@y%
    \pgf@process{\tikz@timer@end}%
    \pgf@xb=\tikz@time pt%
    \pgf@xb=2\pgf@xb%
    \advance\pgf@xb by-1pt%
    \edef\tikz@marshal{\noexpand\pgftransformlineattime{\pgf@sys@tonumber{\pgf@xb}}%
      {\noexpand\pgfqpoint{\the\pgf@xa}{\the\pgf@y}}{\noexpand\tikz@timer@end}}%
    \tikz@marshal%
  \fi%
}

\def\tikz@timer@hvline{%
  \ifdim\tikz@time pt<0.5pt% first half
    \pgf@process{\tikz@timer@start}%
    \pgf@xa=\pgf@x%
    \pgf@ya=\pgf@y%
    \pgf@process{\tikz@timer@end}%
    \pgf@xb=\tikz@time pt%
    \pgf@xb=2\pgf@xb%    
    \edef\tikz@marshal{\noexpand\pgftransformlineattime{\pgf@sys@tonumber{\pgf@xb}}{\noexpand\tikz@timer@start}{%
        \noexpand\pgfqpoint{\the\pgf@x}{\the\pgf@ya}}}%
    \tikz@marshal%
  \else% second half
    \pgf@process{\tikz@timer@start}%
    \pgf@xa=\pgf@x%
    \pgf@ya=\pgf@y%
    \pgf@process{\tikz@timer@end}%
    \pgf@xb=\tikz@time pt%
    \pgf@xb=2\pgf@xb%
    \advance\pgf@xb by-1pt%
    \edef\tikz@marshal{\noexpand\pgftransformlineattime{\pgf@sys@tonumber{\pgf@xb}}%
      {\noexpand\pgfqpoint{\the\pgf@x}{\the\pgf@ya}}{\noexpand\tikz@timer@end}}%
    \tikz@marshal%
  \fi%
}

\def\tikz@timer@curve{%
  \pgftransformcurveattime{\tikz@time}{\tikz@timer@start}{\tikz@timer@cont@one}{\tikz@timer@cont@two}{\tikz@timer@end}%
}



%
% Coordinate systems
% 

\def\tikzdeclarecoordinatesystem#1#2{%
  \expandafter\def\csname tikz@parse@cs@#1\endcsname(##1){%
    \pgf@process{%
      #2%
      % Smuggle outside:
      \iftikz@shapeborder%
        \global\let\tikz@smuggle@a=\tikz@shapebordertrue%
      \else%
        \global\let\tikz@smuggle@a=\tikz@shapeborderfalse%
      \fi%
      \global\let\tikz@smubble@b=\tikz@shapeborder@name%
    }%
    \tikz@smuggle@a%
    \let\tikz@shapeborder@name=\tikz@smubble@b%
    \edef\tikz@return@coordinate{\noexpand\pgfqpoint{\the\pgf@x}{\the\pgf@y}}}%
}
\def\tikzaliascoordinatesystem#1#2{%
  \edef\pgf@marshal{\noexpand\let\expandafter\noexpand\csname
    tikz@parse@cs@#1\endcsname=\expandafter\noexpand\csname
    tikz@parse@cs@#2\endcsname}%
  \pgf@marshal%
}


% Default coodinate systems:

\tikzdeclarecoordinatesystem{canvas}
{%
  \tikzset{cs/.cd,x=0pt,y=0pt,#1}%
  \pgfpoint{\tikz@cs@x}{\tikz@cs@y}%
}

\tikzdeclarecoordinatesystem{canvas polar}
{%
  \tikzset{cs/.cd,angle=0,radius=0cm,#1}%
  \pgfpointpolar{\tikz@cs@angle}{\tikz@cs@xradius/\tikz@cs@yradius}%
}

\tikzdeclarecoordinatesystem{xyz}
{%
  \tikzset{cs/.cd,x=0,y=0,z=0,#1}%
  \pgfpointxyz{\tikz@cs@x}{\tikz@cs@y}{\tikz@cs@z}%
}

\tikzdeclarecoordinatesystem{xyz polar}
{%
  \tikzset{cs/.cd,angle=0,radius=0,#1}%
  \pgfpointpolarxy{\tikz@cs@angle}{\tikz@cs@xradius and \tikz@cs@yradius}%
}
\tikzaliascoordinatesystem{xy polar}{xyz polar}


\tikzdeclarecoordinatesystem{node}
{%
  \tikzset{cs/.cd,name=,anchor=none,angle=none,#1}%
  \ifx\tikz@cs@anchor\tikz@nonetext%
    \ifx\tikz@cs@angle\tikz@nonetext%
      \expandafter\ifx\csname pgf@sh@ns@\tikz@cs@node\endcsname\tikz@coordinate@text%
      \else
        \tikz@shapebordertrue%
        \edef\tikz@shapeborder@name{\tikz@cs@node}%
      \fi%
      \pgfpointanchor{\tikz@cs@node}{center}%
    \else%
      \pgfpointanchor{\tikz@cs@node}{\tikz@cs@angle}%
    \fi%
  \else%
    \pgfpointanchor{\tikz@cs@node}{\tikz@cs@anchor}%
  \fi%
}

\tikzdeclarecoordinatesystem{intersection}
{%
  \tikzset{cs/.cd,#1}%
  \expandafter\tikz@@@scan@@absolute\expandafter\tikz@parse@intersection@a\tikz@cs@line@a@begin%
  \expandafter\tikz@@@scan@@absolute\expandafter\tikz@parse@intersection@b\tikz@cs@line@a@end%
  \expandafter\tikz@@@scan@@absolute\expandafter\tikz@parse@intersection@c\tikz@cs@line@b@begin%
  \expandafter\tikz@@@scan@@absolute\expandafter\tikz@parse@intersection@d\tikz@cs@line@b@end%
  \edef\pgf@marshal{%
    {\noexpand\pgfpointintersectionoflines%
      {\noexpand\pgfqpoint{\the\pgf@xa}{\the\pgf@ya}}%
      {\noexpand\pgfqpoint{\the\pgf@xb}{\the\pgf@yb}}%
      {\noexpand\pgfqpoint{\the\pgf@xc}{\the\pgf@yc}}%
      {\noexpand\pgfqpoint{\the\pgf@x}{\the\pgf@y}}}}%
  \pgf@marshal%
}

\tikzdeclarecoordinatesystem{perpendicular}
{%
  \tikzset{cs/.cd,#1}%
  \expandafter\tikz@@@scan@@absolute\expandafter\tikz@parse@intersection@a\tikz@cs@hori@line%
  \expandafter\tikz@@@scan@@absolute\expandafter\tikz@parse@intersection@b\tikz@cs@vert@line%
  \pgfqpoint{\the\pgf@xb}{\the\pgf@ya}
}

\tikzdeclarecoordinatesystem{barycentric}
{%
  {%
    \pgf@xa=0pt% point
    \pgf@ya=0pt%
    \pgf@xb=0pt% sum
    \tikz@bary@dolist#1,=,%
    \pgfmathparse{1/\the\pgf@xb}%
    \global\pgf@x=\pgfmathresult\pgf@xa%
    \global\pgf@y=\pgfmathresult\pgf@ya%
  }%
}

\def\tikz@bary@dolist#1=#2,{%
  \def\tikz@temp{#1}%
  \ifx\tikz@temp\pgfutil@empty%
  \else
    \pgf@process{\pgfpointanchor{#1}{center}}%
    \pgfmathparse{#2}%
    \advance\pgf@xa by\pgfmathresult\pgf@x%
    \advance\pgf@ya by\pgfmathresult\pgf@y%
    \advance\pgf@xb by\pgfmathresult pt%
    \expandafter\tikz@bary@dolist%
  \fi%
}

\tikzset{cs/x/.store in=\tikz@cs@x}
\tikzset{cs/y/.store in=\tikz@cs@y}
\tikzset{cs/z/.store in=\tikz@cs@z}
\tikzset{cs/angle/.store in=\tikz@cs@angle}
\tikzset{cs/x radius/.store in=\tikz@cs@xradius}
\tikzset{cs/y radius/.store in=\tikz@cs@yradius}
\tikzset{cs/radius/.style={/tikz/cs/x radius=#1,/tikz/cs/y radius=#1}}
\tikzset{cs/name/.store in=\tikz@cs@node}
\tikzset{cs/anchor/.store in=\tikz@cs@anchor}

\tikzset{cs/first line/.code args={(#1)--(#2)}{\def\tikz@cs@line@a@begin{(#1)}\def\tikz@cs@line@a@end{(#2)}}}
\tikzset{cs/second line/.code args={(#1)--(#2)}{\def\tikz@cs@line@b@begin{(#1)}\def\tikz@cs@line@b@end{(#2)}}}

\tikzset{cs/horizontal line through/.store in=\tikz@cs@hori@line}
\tikzset{cs/vertical line through/.store in=\tikz@cs@vert@line}




%
% Coordinate management
%


% Last position visited
\def\tikz@last@position{\pgfqpoint{\tikz@lastx}{\tikz@lasty}}
\def\tikz@last@position@saved{\pgfqpoint{\tikz@lastxsaved}{\tikz@lastysaved}}

% Make given point the last position visited
\def\tikz@make@last@position#1{%
  \pgf@process{#1}%
  \tikz@lastx=\pgf@x\relax%
  \tikz@lasty=\pgf@y\relax%
  \iftikz@updatecurrent%
    \tikz@lastxsaved=\pgf@x\relax%
    \tikz@lastysaved=\pgf@y\relax%
  \fi%
  \tikz@updatecurrenttrue%
}

\newif\iftikz@updatecurrent
\tikz@updatecurrenttrue



% Scanner: Scans a point or a relative point. 
% It then calls the first parameter with the argument set to an
% appropriate pgf command representing that point.

\def\tikz@scan@one@point#1{%
  \let\tikz@to@use@whom=\tikz@to@use@last@coordinate%
  \tikz@shapeborderfalse%
  \pgfutil@ifnextchar+{\tikz@scan@relative#1}{\tikz@scan@absolute#1}}
\def\tikz@scan@absolute#1{%
  \pgfutil@ifnextchar({\tikz@scan@@absolute#1}%)
  {%
    \advance\tikz@expandcount by -1%
    \ifnum\tikz@expandcount<0\relax%
      \let\@next=\tikz@@scangiveup%
    \else%
      \let\@next=\tikz@@scanexpand%
    \fi%
    \@next{#1}%
  }%
}
\def\tikz@@scanexpand#1{\expandafter\tikz@scan@one@point\expandafter#1}
\def\tikz@@scangiveup#1{\PackageError{tikz}{Cannot parse this coordinate}{}#1{\pgfpointorigin}}
\def\tikz@scan@@absolute#1(#2){%
  \edef\tikz@temp{(#2)}%
  \expandafter\tikz@@scan@@absolute\expandafter#1\tikz@temp%
}
\def\tikz@@scan@@absolute#1({%
  \pgfutil@ifnextchar[% uhoh... options!
  {\def\tikz@scan@point@recall{#1}\tikz@scan@options}%
  {\tikz@@@scan@@absolute#1(}%
}

\def\tikz@scan@options[#1]#2{%
  \def\tikz@scan@point@options{#1}%
  \tikz@@@scan@@absolute\tikz@scan@handle@options(#2%
}

\def\tikz@scan@handle@options#1{%
  {%
    % Ok, compute point with options set and zero transformation
    % matrix:
    \pgftransformreset%
    \let\tikz@transform=\pgfutil@empty%
    \expandafter\tikzset\expandafter{\tikz@scan@point@options}%
    \tikz@transform%
    \pgf@process{\pgfpointtransformed{#1}}%
    \xdef\tikz@marshal{\expandafter\noexpand\tikz@scan@point@recall{\noexpand\pgfqpoint{\the\pgf@x}{\the\pgf@y}}}%
  }%
  \tikz@marshal%  
}

\def\tikz@@@scan@@absolute#1(#2){%
  \pgfutil@in@{intersection of}{#2}%
  \ifpgfutil@in@%
    \let\@next\tikz@parse@intersection%
  \else%
    \pgfutil@in@|{#2}%
    \ifpgfutil@in@
      \pgfutil@in@{-|}{#2}%
      \ifpgfutil@in@
        \let\@next\tikz@parse@hv%
      \else%
        \let\@next\tikz@parse@vh%
      \fi%
    \else%
      \pgfutil@in@{cs:}{#2}%
      \ifpgfutil@in@%
        \let\@next\tikz@parse@coordinatesystem%
      \else%
        \pgfutil@in@:{#2}%
        \ifpgfutil@in@
          \let\@next\tikz@parse@polar%
        \else%
          \pgfutil@in@,{#2}%
          \ifpgfutil@in@%      
            \let\@next\tikz@parse@regular%
          \else%
            \let\@next\tikz@parse@node%
          \fi%
        \fi%
      \fi%
    \fi%
  \fi%
  \@next#1(#2)%
}

\def\tikz@parse@coordinatesystem#1(#2 cs:#3){%
  \let\tikz@return@coordinate=\pgfpointorigin%
  \pgfutil@ifundefined{tikz@parse@cs@#2}
  {\PackageError{tikz}{Unknown coordinate system '#2'}{}}
  {\csname tikz@parse@cs@#2\endcsname(#3)}%
  \expandafter#1\expandafter{\tikz@return@coordinate}%
}


\newif\iftikz@isdimension
\def\tikz@checkunit#1{%
  \pgfmathparse{#1}%
  \let\iftikz@isdimension=\ifpgfmathunitsdeclared%
}
\def\tikz@@checkunit{\pgfutil@ifnextchar\tikz@unique{\tikz@checkunit@number}{\tikz@checkunit@dimension}}
\def\tikz@checkunit@number\tikz@unique{\tikz@isdimensionfalse}
\def\tikz@checkunit@dimension#1\tikz@unique{\tikz@isdimensiontrue}

\def\tikz@parse@polar#1(#2:#3){%
  \pgfutil@ifundefined{tikz@polar@dir@#2}
  {\tikz@@parse@polar#1(#2:#3)}
  {\tikz@@parse@polar#1(\csname tikz@polar@dir@#2\endcsname:#3)}%
}
\def\tikz@@parse@polar#1(#2:#3){%
  \pgfutil@in@{ and }{#3}%
  \ifpgfutil@in@%
    \edef\tikz@args{(#2:#3)}%
  \else%
    \edef\tikz@args{(#2:#3 and #3)}%
  \fi%
  \expandafter\tikz@@@parse@polar\expandafter#1\tikz@args%
}
\def\tikz@@@parse@polar#1(#2:#3 and #4){%
  \tikz@checkunit{#3}%
  \iftikz@isdimension%
    \def\tikz@next{#1{\pgfpointpolar{#2}{#3 and #4}}}%
  \else%
    \def\tikz@next{#1{\pgfpointpolarxy{#2}{#3 and #4}}}%
  \fi%
  \tikz@next%
}
\def\tikz@polar@dir@up{90}
\def\tikz@polar@dir@down{-90}
\def\tikz@polar@dir@left{180}
\def\tikz@polar@dir@right{0}
\def\tikz@polar@dir@north{90}
\def\tikz@polar@dir@south{-90}
\def\tikz@polar@dir@east{0}
\def\tikz@polar@dir@west{180}
\expandafter\def\csname tikz@polar@dir@north east\endcsname{45}
\expandafter\def\csname tikz@polar@dir@north west\endcsname{135}
\expandafter\def\csname tikz@polar@dir@south east\endcsname{-45}
\expandafter\def\csname tikz@polar@dir@south west\endcsname{-135}

\def\tikz@parse@regular#1(#2,#3){%
  \pgfutil@in@,{#3}%
  \ifpgfutil@in@%  
    \tikz@parse@splitxyz{#1}{#2}#3,%
  \else%
    \tikz@checkunit{#2}%
    \iftikz@isdimension%
      \def\@next{#1{\pgfpoint{#2}{#3}}}%
    \else%
      \def\@next{#1{\pgfpointxy{#2}{#3}}}%
    \fi%
  \fi%
  \@next%
}

\def\tikz@parse@splitxyz#1#2#3,#4,{%
  \def\@next{#1{\pgfpointxyz{#2}{#3}{#4}}}%
}

\def\tikz@coordinate@text{coordinate}

\def\tikz@parse@node#1(#2){%
  \pgfutil@in@.{#2}% Ok, flag this
  \ifpgfutil@in@
    \tikz@calc@anchor#2\tikz@stop%
  \else%
    \tikz@calc@anchor#2.center\tikz@stop% to be on the save side, in
                                % case iftikz@shapeborder is ignored...
    \expandafter\ifx\csname pgf@sh@ns@#2\endcsname\tikz@coordinate@text%
    \else
      \tikz@shapebordertrue%
      \def\tikz@shapeborder@name{#2}%
    \fi%
  \fi%
  \edef\tikz@marshal{\noexpand#1{\noexpand\pgfqpoint{\the\pgf@x}{\the\pgf@y}}}%
  \tikz@marshal%
}

\def\tikz@calc@anchor#1.#2\tikz@stop{%
  \pgfpointanchor{#1}{#2}%
}


\def\tikz@parse@hv#1(#2){%
  \pgfutil@in@{ -| }{#2}%
  \ifpgfutil@in@%
    \let\tikz@next=\tikz@parse@hvboth%
  \else%
    \pgfutil@in@{ -|}{#2}%
    \ifpgfutil@in@%
      \let\tikz@next=\tikz@parse@hvleft%
    \else%
      \pgfutil@in@{-| }{#2}%
      \ifpgfutil@in@%
        \let\tikz@next=\tikz@parse@hvright%
      \else%
        \let\tikz@next=\tikz@parse@hvdone%
      \fi%
    \fi%
  \fi%
  \tikz@next#1(#2)}
\def\tikz@parse@hvboth#1(#2 -| #3){\tikz@parse@vhdone#1(#3|-#2)}
\def\tikz@parse@hvleft#1(#2 -|#3){\tikz@parse@vhdone#1(#3|-#2)}
\def\tikz@parse@hvright#1(#2-| #3){\tikz@parse@vhdone#1(#3|-#2)}
\def\tikz@parse@hvdone#1(#2-|#3){\tikz@parse@vhdone#1(#3|-#2)}

\def\tikz@parse@vh#1(#2){%
  \pgfutil@in@{ |- }{#2}%
  \ifpgfutil@in@%
    \let\tikz@next=\tikz@parse@vhboth%
  \else%
    \pgfutil@in@{ |-}{#2}%
    \ifpgfutil@in@%
      \let\tikz@next=\tikz@parse@vhleft%
    \else%
      \pgfutil@in@{|- }{#2}%
      \ifpgfutil@in@%
        \let\tikz@next=\tikz@parse@vhright%
      \else%
        \let\tikz@next=\tikz@parse@vhdone%
      \fi%
    \fi%
  \fi%
  \tikz@next#1(#2)}
\def\tikz@parse@vhboth#1(#2 |- #3){\tikz@parse@vhdone#1(#2|-#3)}
\def\tikz@parse@vhleft#1(#2 |-#3){\tikz@parse@vhdone#1(#2|-#3)}
\def\tikz@parse@vhright#1(#2|- #3){\tikz@parse@vhdone#1(#2|-#3)}
\def\tikz@parse@vhdone#1(#2|-#3){%
  {%
    \tikz@@@scan@@absolute\tikz@parse@vh@mid(#2)%
    \tikz@@@scan@@absolute\tikz@parse@vh@end(#3)%
    \xdef\tikz@marshal{\noexpand#1{\noexpand\pgfqpoint{\the\pgf@xa}{\the\pgf@ya}}}%
  }%
  \tikz@shapeborderfalse%
  \tikz@marshal%
}
\def\tikz@parse@vh@mid#1{\pgf@process{#1}\pgf@xa=\pgf@x}
\def\tikz@parse@vh@end#1{\pgf@process{#1}\pgf@ya=\pgf@y}

\def\tikz@parse@intersection#1(intersection of #2--#3 and #4--#5){%
  {%
    \tikz@@@scan@@absolute\tikz@parse@intersection@a(#2)%
    \tikz@@@scan@@absolute\tikz@parse@intersection@b(#3)%
    \tikz@@@scan@@absolute\tikz@parse@intersection@c(#4)%
    \tikz@@@scan@@absolute\tikz@parse@intersection@d(#5)%
    \xdef\tikz@marshal{\noexpand#1{\noexpand\pgfpointintersectionoflines%
        {\noexpand\pgfqpoint{\the\pgf@xa}{\the\pgf@ya}}%
        {\noexpand\pgfqpoint{\the\pgf@xb}{\the\pgf@yb}}%
        {\noexpand\pgfqpoint{\the\pgf@xc}{\the\pgf@yc}}%
        {\noexpand\pgfqpoint{\the\pgf@x}{\the\pgf@y}}}}%
  }%
  \tikz@shapeborderfalse%
  \tikz@marshal%  
}

\def\tikz@parse@intersection@a#1{\pgf@process{#1}\pgf@xa=\pgf@x\pgf@ya=\pgf@y}
\def\tikz@parse@intersection@b#1{\pgf@process{#1}\pgf@xb=\pgf@x\pgf@yb=\pgf@y}
\def\tikz@parse@intersection@c#1{\pgf@process{#1}\pgf@xc=\pgf@x\pgf@yc=\pgf@y}
\def\tikz@parse@intersection@d#1{\pgf@process{#1}}

\def\tikz@scan@relative#1+{%
  \pgfutil@ifnextchar+{\tikz@scan@plusplus#1}{\tikz@scan@oneplus#1}}

\def\tikz@scan@plusplus#1+{%
  \def\tikz@doafter{#1}%
  \tikz@scan@absolute\tikz@add%
}
\def\tikz@add#1{%
  \tikz@doafter{\pgfpointadd{#1}{\tikz@last@position@saved}}%
}
\def\tikz@scan@oneplus#1{%
  \def\tikz@doafter{#1}%
  \tikz@updatecurrentfalse%
  \tikz@scan@absolute\tikz@add%
} 



% Loading further libraries

% Include a library file.
%
% #1 = List of names of library file.
%  
% Description:
%
% This command includes a list of TikZ library files. For each file X in the
% list, the file pgflibrarytikzX.code.tex is included, provided this has
% not been done earlier. 
%
% For the convenience of Context users, both round and square brackets
% are possible for the argument.
%
% Example:
%
% \usetikzlibrary{arrows}
% \usetikzlibrary[patterns,topaths]

\def\usetikzlibrary{\pgfutil@ifnextchar[{\use@tikzlibrary}{\use@@tikzlibrary}}%}
\def\use@tikzlibrary[#1]{\use@@tikzlibrary{#1}}
\def\use@@tikzlibrary#1{%
  \edef\pgf@list{#1}%
  \pgfutil@for\pgf@temp:=\pgf@list\do{%
    \expandafter\ifx\csname tikz@library@\pgf@temp @loaded\endcsname\relax%
      \expandafter\global\expandafter\let\csname tikz@library@\pgf@temp @loaded\endcsname=\pgfutil@empty%
      \expandafter\edef\csname tikz@library@#1@atcode\endcsname{\the\catcode`\@}
      \expandafter\edef\csname tikz@library@#1@barcode\endcsname{\the\catcode`\|}
      \catcode`\@=11
      \catcode`\|=12
      \input pgflibrarytikz\pgf@temp.code.tex
      \catcode`\@=\csname tikz@library@#1@atcode\endcsname
      \catcode`\|=\csname tikz@library@#1@barcode\endcsname
    \fi%
  }%
}


% Always-present libraries:

\usetikzlibrary{topaths}




\endinput


\endinput
\end{verbatim}

The files in the |generic/pgf| directory do the actual work.



\subsubsection{Using the Plain \TeX\ Format}

When using the plain \TeX\ format, you say |% Copyright 2006 by Till Tantau
%
% This file may be distributed and/or modified
%
% 1. under the LaTeX Project Public License and/or
% 2. under the GNU Public License.
%
% See the file doc/generic/pgf/licenses/LICENSE for more details.


\edef\pgfatcode{\the\catcode`\@}
\catcode`\@=11


\input pgfrcs.tex
\ProvidesPackageRCS $Header: /cvsroot/pgf/pgf/plain/pgf/basiclayer/pgf.tex,v 1.9 2008/01/13 10:35:47 vibrovski Exp $

\input pgfcore.tex

\usepgfmodule{shapes,plot}

%\input pgfbasesnakes.tex
%\input pgfbasedecorations.tex
%\input pgfbasematrix.tex

\catcode`\@=\pgfatcode

\endinput
| or
|% Copyright 2006 by Till Tantau
%
% This file may be distributed and/or modified
%
% 1. under the LaTeX Project Public License and/or
% 2. under the GNU Public License.
%
% See the file doc/generic/pgf/licenses/LICENSE for more details.

% This file is tikz.tex

\edef\tikzatcode{\the\catcode`\@}
\catcode`\@=11

\input xkeyval.tex
\input pgf.tex
\input pgffor.tex
\input tikz.code.tex

\catcode`\@=\tikzatcode

\endinput
|. Instead of  |\begin{pgfpicture}| and
  |\end{pgfpicture}| you use  |\pgfpicture| and |\endpicture|. 

Unlike for the \LaTeX\ format, \pgfname\ is not as good as discerning
the appropriate configuration for the plain \TeX\ format. In
particular, it can only automatically determine the correct output
format if you use |pdftex| or |tex| plus |dvips|. For all other output
formats you need to set the macro |\pgfsysdriver| to the correct
value. See the description of using output formats later on. 

\pgfname\ was originally written for use with \LaTeX\ and this shows
in a number of places. Nevertheless, the plain \TeX\ support is
reasonably good.

Like the \LaTeX\ style files, the plain \TeX\ files like |tikz.tex|
also just include the correct |tikz.code.tex| file.



\subsubsection{Using the Con\TeX t Format}

Currently, there is no special support for the Con\TeX t
format. Rather, you have to use \pgfname\ and \tikzname\ as if you
were using the plain \TeX\ format when using Con\TeX t. This may
change in the future.





\subsection{Support Output Formats}
\label{section-drivers}

An output format is a format in which \TeX\ outputs the text it has
typeset. Producing the output is (conceptually) a two-stage process:
\begin{enumerate}
\item
  \TeX\ typesets your text and graphics. The result of this
  typesetting is mainly a long list of letter--coordinate pairs, plus 
  (possibly) some ``special'' commands. This long list of coordinates
  is written to something called a |.dvi|-file.
\item
  Some other program reads this |.dvi|-file and translates the
  letter--coordinate pairs into, say, PostScript commands for placing
  the given letter at the given coordinate.
\end{enumerate}

The classical example of this process is the combination of |latex|
and |dvips|. The |latex| program (which is just the |tex| program
called with the \LaTeX-macros preinstalled) produces a |.dvi|-file as
its output. The |dvips| program takes this output and produces a
|.ps|-file (a PostScript) file. Possibly, this file is further
converted using, say, |ps2pdf|. Another example of programs using this
process is the combination of, 
say, |tex| and |dvipdfm|. The |dvipdfm| program takes a |.dvi|-file as
input and translates the letter--coordinate pairs therein into
\pdf-commands, resulting in a |.pdf| file directly. Finally, the
|tex4ht| is also a program that takes a |.dvi|-file and produces an
output, this time it is a |.html| file. The programs |pdftex| and
|pdflatex| are special: They directly produce a |.pdf|-file without
the intermediate |.dvi|-stage. However, from the programmer's point of
view they behave exactly as if there where an intermediate stage.

Normally, \TeX\ only produces letter--coordinate pairs as its
``output.'' This obviously makes is difficult tho draw, say, a
curve. For this, ``special'' commands can be used. Unfortunately,
these special commands are not the same for the different programs
that process the |.dvi|-file. Indeed, every program that takes a
|.dvi|-file as input has a totally different syntax for the special
commands.

One of the main jobs of \pgfname\ is to ``abstract way'' the
difference in the syntax of the different programs. However, this
means that support for each program has to be ``programmed,'' which is
a time-consuming and complicated process. 


\subsubsection{Selecting the Backend Driver}

When \TeX\ typesets your document, it does not know which program
you are going to use to transform the |.dvi|-file. If your |.dvi|-file
does not contain any special commands, this would be fine; but these
days almost all |.dvi|-files contain lots of special commands. It is
thus necessary to tell \TeX\ which program you are going to use later
on.

Unfortunately, there is no ``standard'' way of telling this to
\TeX. For the \LaTeX\ format a sophisticated mechanism exists inside
the |graphics| package and \pgfname\ plugs into this mechanism. For
other formats and when this plugging does not work as expected, it is
necessary to tell \pgfname\ directly which program you are going to
use. This is done my redefining the macro |\pgfsysdriver| to an
appropriate value \emph{before} you load |pgf|. If you are going to
use the |dvips| program, you set this macro to the value
|pgfsys-dvips.def|; if you use |pdftex| or |pdflatex|, you set it to
|pgfsys-pdftex.def|; and so on. In the following, details of the
support of the different programs are discussed.


\subsubsection{Producing PDF Output}

\pgfname\ supports to programs that produce \pdf\ output (\pdf\ means
``portable document format'' and was invented by the Adobe company):
|dvipdfm| and |pdftex|. The |pdflatex| program is the same as the
|pdftex| program: it uses a different input format, but the output is
exactly the same.

\begin{filedescription}{pgfsys-pdftex.def}
  This is the driver file for use with pdf\TeX, that is, with the
  |pdftex| or |pdflatex| command. It includes
  |pgfsys-common-pdf.def|.

  This driver has the ``complete'' functionality. This means,
  everything \pgfname\ ``can do at all'' is implemented in this
  driver. 
\end{filedescription}

\begin{filedescription}{pgfsys-dvipdfm.def}
  This is a driver file for use with (|la|)|tex| followed by |dvipdfm|. It
  includes |pgfsys-common-pdf.def|.

  This driver supports most of \pgfname's features, but there are some
  restrictions:
  \begin{enumerate}
  \item
    In \LaTeX\ mode it uses |graphicx| for the graphics
    inclusion and does not support masking.
  \item
    In plain \TeX\ mode it does not support image inclusion.
  \end{enumerate}
\end{filedescription}

It is also possible to produce a |.pdf|-file by first producing a
PostScript file (see below) and then using a PostScript-to-\pdf\
conversion program like |ps2pdf| or the Acrobat Distiller.


\subsubsection{Producing PostScript Output}

\begin{filedescription}{pgfsys-dvips.def}
  This is a driver file for use with (|la|)|tex| followed by
  |dvips|. It includes |pgfsys-common-postscript.def|.

  This driver also supports most of \pgfname's features, except for
  the following restrictions:
  \begin{enumerate}
  \item
    In \LaTeX\ mode it uses |graphicx| for the graphics
    inclusion and does not support masking.
  \item
    In plain \TeX\ mode it does not support image inclusion.
  \item
    Shading is fully implemented, but the results will not be 
    as good as with a driver producing |.pdf| as output. 
  \end{enumerate}
\end{filedescription}



\subsubsection{Producing HTML / SVG Output}

The |tex4ht| program converts |.dvi|-files to |.html|-files. While the
\textsc{html}-format cannot be used to draw graphics, the
\textsc{svg}-format can. Using the following driver, you can ask
\pgfname\ to produce an \textsc{svg}-picture for each \pgfname\
graphic in your text.

\begin{filedescription}{pgfsys-tex4ht.def}
  This is a driver file for use with the |tex4ht| program. It includes
  |pgfsys-common-svg.def|.

  When using this driver you should be aware of the following
  restrictions: 
  \begin{enumerate}
  \item
    In \LaTeX\ mode it uses |graphicx| for the graphics
    inclusion.    
  \item
    In plain \TeX\ mode it does not support image inclusion.
  \item
    Text inside |pgfpicture|s is not supported very well. The reason
    is that the \textsc{svg} specification currently does not support
    text very well and it is also not possible to correctly ``escape
    back'' to \textsc{html}. All these problems will hopefully
    disappear in the future, but currently only two kinds of text work
    reasonably well: First, plain text without math mode, special
    characters or anything else special. Second, \emph{very} simple
    mathematical text that contains subscripts or superscripts. Even
    then, variables are not correctly set in italics and, in general,
    text simple does not look very nice.
  \item
    If you use text that contains anything special, even something as
    simple as |$\alpha$|, this may corrupt the graphic since |text4ht|
    does not always produce valid \textsc{xml} code. So, once more,
    \emph{stick to very simple node text inside graphics.} Sorry.
  \item
    Unlike for other output formats, the bounding box of a picture
    ``really crops'' the picture.
  \end{enumerate}

  The driver basically works as follows: When a |{pgfpicture}| is
  started, appropriate |\special| commands are used to directed the
  output of |tex4ht| to a new file called |\jobname-xxx.svg|, where
  |xxx| is a number that is increased for each graphic. Then, till the
  end of the picture, each (system layer) graphic command creates a
  specials that insert appropriate \textsc{svg} literal text into the
  output file. The exact details are a bit complicated since the
  imaging model and the processing model of PostScript/\pdf\ and
  \textsc{svg} are not quite the same; but they are ``close enough''
  for \pgfname's purposes.
\end{filedescription}





\part{Ti\emph{k}Z ist \emph{kein} Zeichenprogramm}
\label{part-tikz}

\vskip3cm
\begin{codeexample}[graphic=white]
\begin{tikzpicture}
  \draw[fill=yellow] (0,0) -- (60:.75cm) arc (60:180:.75cm);
  \draw(120:0.4cm) node {$\alpha$};

  \draw[fill=green!30] (0,0) -- (right:.75cm) arc (0:60:.75cm);
  \draw(30:0.5cm) node {$\beta$};

  \begin{scope}[shift={(60:2cm)}]
    \draw[fill=green!30] (0,0) -- (180:.75cm) arc (180:240:.75cm);
    \draw (30:-0.5cm) node {$\gamma$};

    \draw[fill=yellow] (0,0) -- (240:.75cm) arc (240:360:.75cm);
    \draw (-60:0.4cm) node {$\delta$};
  \end{scope}

  \begin{scope}[thick]
    \draw  (60:-1cm) node[fill=white] {$E$} -- (60:3cm) node[fill=white] {$F$};
    \draw[red]                   (-2,0) node[left] {$A$} -- (3,0) node[right]{$B$};
    \draw[blue,shift={(60:2cm)}] (-3,0) node[left] {$C$} -- (2,0) node[right]{$D$};
  
    \draw[shift={(60:1cm)},xshift=4cm]
    node [right,text width=6cm,rounded corners,fill=red!20,inner sep=1ex]
    {
      When we assume that $\color{red}AB$ and $\color{blue}CD$ are
      parallel, i.\,e., ${\color{red}AB} \mathbin{\|} \color{blue}CD$,
      then $\alpha = \delta$ and $\beta = \gamma$.
    };
  \end{scope}
\end{tikzpicture}
\end{codeexample}



% Copyright 2003 by Till Tantau <tantau@cs.tu-berlin.de>.
%
% This program can be redistributed and/or modified under the terms
% of the LaTeX Project Public License Distributed from CTAN
% archives in directory macros/latex/base/lppl.txt.


\section{Design Principles}

This section describes the design principles behind the \tikzname\
frontend, where \tikzname\ means ``\tikzname\ ist \emph{kein}
Zeichenprogramm.'' To use \tikzname, as a \LaTeX\ user say
|\usepackage{tikz}| somewhere in the preamble, as a plain \TeX\ user
say |\input tikz.tex|. \tikzname's job is to make your life easier by
providing an easy-to-learn and easy-to-use syntax for describing
graphics. 

The commands and syntax of \tikzname\ were influenced by several
sources. The basic command names and the notion of  path operations is
taken from \textsc{metafont}, the option mechanism comes from
\textsc{pstricks}, the notion of styles is reminiscent of
\textsc{svg}. To make it all work together, some compromises were
necessary. I also added some ideas of my own, like meta-arrows and
coordinate transformations. 

The following basic design principles underlie \tikzname:
\begin{enumerate}
\item Special syntax for specifying points.
\item Special syntax for path specifications.
\item Actions on paths.
\item Key-value syntax for graphic parameters.
\item Special syntax for nodes.
\item Grouping of graphic parameters.
\item Coordinate transformation system.
\end{enumerate}



\subsection{Special Syntax For Specifying Points}

\tikzname\ provides a special syntax for specifying points and
coordinates. In the simplest case, you provide two \TeX\ dimensions,
separated by commas, in round brackets as in |(1cm,2pt)|.

You can also specify a point in polar coordinates by using a colon
instead of a comma as in |(30:1cm)|, which means ``1cm in a 30
degrees direction.'' 

If you do not provide a unit, as in |(2,1)|, you specify a point in
\pgfname's $xy$-coordinate system. By default, the unit $x$-vector
goes 1cm to the right and the unit $y$-vector goes 1cm upward.

By specifying three numbers as in |(1,1,1)| you specify a point in
\pgfname's $xyz$-coordinate system.

It is also possible to use an anchor of a previously defined shape
as in |(first node.south)|.

You can add two plus signs before a coordinate as in
|++(1cm,0pt)|. This means ``1cm to the right of the last point
used.'' This allows you to easily specify relative movements. For
example, |(1,0) ++(1,0) ++(0,1)| specifies the three coordinates
|(1,0)|, then |(2,0)|, and |(2,1)|.

Finally, instead of two plus signs, you can also add a single
one. This also specifies a point in a relative manner, but it does
not ``change'' the current point used in subsequent relative
commands. For example, |(1,0) +(1,0) +(0,1)| specifies the three
coordinates |(1,0)|, then |(2,0)|, and |(1,1)|.

\subsection{Special Syntax For Path Specifications}

When creating a picture using \tikzname, your main job is the
specification of \emph{paths}. A path is a series of straight or curved
lines, which need not be connected. \tikzname\ makes it easy to
specify paths, partly using the syntax of \textsc{metapost}. For
example, to specify a triangular path you use
\begin{codeexample}[code only]
(5pt,0pt) -- (0pt,0pt) -- (0pt,5pt) -- cycle
\end{codeexample}
and you get \tikz \draw (5pt,0pt) -- (0pt,0pt) -- (0pt,5pt) -- cycle;
when you draw this path.

\subsection{Actions on Paths}

A path is just a series of straight and curved lines, but it is not
yet specified what should happen with it. One can \emph{draw} a
path, \emph{fill} a path, \emph{shade} it, \emph{clip} it, or do any
combination of these. Drawing (also known as \emph{stroking}) can be
thought of as taking a pen of a certain thickness and moving it
along the path, thereby drawing on the canvas. Filling means that
the interior of the path is filled with a uniform color. Obviously,
filling makes sense only for \emph{closed} paths and a path is
automatically closed prior to filling, if necessary.

Given a path as in |\path (0,0) rectangle (2ex,1ex);|, you can draw
it by adding the |draw| option as in
|\path[draw] (0,0) rectangle (2ex,1ex);|, which yields \tikz \path[draw]
(0,0) rectangle (2ex,1ex);. The |\draw| command is just an abbreviation for
|\path[draw]|. To fill a path, use the |fill| option or the |\fill|
command, which is an abbreviation for |\path[fill]|. The
|\filldraw| command is an abbreviation for
|\path[fill,draw]|. Shading is caused by the |shade| option (there
are |\shade| and |\shadedraw| abbreviations) and clipping by the
|clip| option. There is is also a |\clip| command, which does the
same as |\path[clip]|, but not commands like |\drawclip|. Use, say,
|\draw[clip]| or |\path[draw,clip]| instead.

All of these commands can only be used inside |{tikzpicture}|
environments. 

\tikzname\ allows you to use different colors for filling and
stroking.

\subsection{Key-Value Syntax for Graphic Parameters}
Whenever \tikzname\ draws or fills a path, a large number of graphic
parameters influenced the rendering. Examples include the colors
used, the dashing pattern, the clipping area, the line width, and
many others. In \tikzname, all these options are specified as lists
of so called key-value pairs, as in |color=red|, that are
passed as optional parameters to the path drawing and filling
commands. This usage is similar to \textsc{pstricks}. For
example, the following will draw a thick, red triangle;
\begin{codeexample}[]
\tikz \draw[line width=2pt,color=red] (1,0) -- (0,0) -- (1,0) -- cycle; 
\end{codeexample}

\subsection{Special Syntax for Specifying Nodes}
\tikzname\ introduces a special syntax for adding text or, more
generally, nodes to a graphic. When you specify a path, add nodes as
in the following example:
\begin{codeexample}[]
\tikz \draw (1,1) node {text} -- (2,2);
\end{codeexample}
Nodes are inserted at the current position of
the path, but only \emph{after} the path has been rendered. When
special options are given, as in
|\draw (1,1) node[circle,draw] {text};|, the text is not just put 
at the current position. Rather, it is surrounded by a circle and
this circle is ``drawn.'' 

You can add a name to a node for later reference either by using the
option   |name=|\meta{node name} or by stating the node name in
parentheses outside the text as in |node[circle](name){text}|.

Predefined shapes include |rectangle|, |circle|, and |ellipse|, but
it is possible (though a bit challenging) to define new shapes.

\subsection{Grouping of Graphic Parameters}

Graphic parameters should often apply to several path drawing or
filling commands. For example, we may wish to draw numerous lines all
with the same line width of 1pt. For this, we put these commands
in a |{scope}| environment that takes the desired graphic options
as an optional parameter. Naturally, the specified graphic
parameters apply only to the drawing and filling commands inside the
environment. Furthermore, nested |{scope}| environments or
individual drawing commands can override the graphic parameters of
outer |{scope}| environments. In the following example, three red
lines, two green lines, and one blue line are drawn:

\begin{codeexample}[]
\begin{tikzpicture}
  \begin{scope}[color=red]
    \draw (0mm,10mm) -- (10mm,10mm);
    \draw (0mm, 8mm) -- (10mm, 8mm);
    \draw (0mm, 6mm) -- (10mm, 6mm);
  \end{scope}
  \begin{scope}[color=green]
    \draw             (0mm, 4mm) -- (10mm, 4mm);
    \draw             (0mm, 2mm) -- (10mm, 2mm);
    \draw[color=blue] (0mm, 0mm) -- (10mm, 0mm);
  \end{scope}
\end{tikzpicture}
\end{codeexample}

The |{tikzpicture}| environment itself also behaves like a
|{scope}| environment, that is, you can specify graphic parameters
using an optional argument. These optional apply to all commands in
the picture.


\subsection{Coordinate Transformation System}

\tikzname\ relies entirely on \pgfname's \emph{coordinate} transformation
system to perform transformations. \pgfname\ also supports
\emph{canvas} transformations, a more low-level transformation system,
but this system is not accessible from \tikzname. There are two reasons
for this: First, the canvas transformation must be used with great
care and often results in ``bad'' graphics with changing line width
and text in wrong sizes. Second, \pgfname\ looses track of where nodes
and shapes are positioned when canvas transformations are used.

For more details on the difference between coordinate transformations
and canvas transformations see
Section~\ref{section-design-transformations}. 

% Copyright 2003 by Till Tantau <tantau@cs.tu-berlin.de>.
%
% This program can be redistributed and/or modified under the terms
% of the LaTeX Project Public License Distributed from CTAN
% archives in directory macros/latex/base/lppl.txt.


\section[Hierarchical Structures: Package, Environments, Scopes, and Styles]
{Hierarchical Structures:\\
  Package, Environments, Scopes, and Styles}

The present section explains how your files should be structured when
you use \tikzname. On the top level, you need to include the |tikz|
package. In the main text, each graphic needs to be put in a
|{tikzpicture}| environment. Inside these environments, you can use
|{scope}| environments to create internal groups. Inside the scopes
you use |\path| commands to actually draw something. On all levels
(except for the package level), graphic options can be given that
apply to everything within the environment.



\subsection{Loading the Package}

\begin{package}{tikz}
  This package does not have any options.
  
  This will automatically load the \pgfname\ package and several other
  stuff that \tikzname\ needs (like the |xkeyval| package).

  \pgfname\ needs to know what \TeX\ driver you are intending to use. In
  most cases \pgfname\ is clever enough to determine the correct driver
  for you; this is true in particular if you \LaTeX. Currently, the only
  situation where \pgfname\ cannot know the driver ``by itself'' is when
  you use plain \TeX\ or Con\TeX t together with |dvipdfm|. In this case,
  you have to write |\def\pgfsysdriver{pgfsys-dvipdfm.def}|
  \emph{before} you input |tikz.tex|. 
\end{package}



\subsection{Creating a Picture}

\subsubsection{Creating a Picture Using an Environment}

The ``outermost'' scope of \tikzname\ is the |{tikzpicture}| 
environment. You may give drawing commands only inside this
environment, giving them outside (as is possible in many other
packages) will result in chaos.

In \tikzname, the way graphics are rendered is strongly influenced by
graphic options. For example, there is an option for setting the color used
for drawing, another for setting the color used for filling, and also
more obscure ones like the option  for setting the prefix used in the
filenames of temporary files written while plotting functions using an
external program. The graphic options are nearly always specified in a
so-called key-value style. (The ``nearly always'' refers to the name
of nodes, which can also be specified differently.) All graphic
options are local to the |{tikzpicture}| to which they apply.

\begin{environment}{{tikzpicture}\opt{\oarg{options}}}
  All \tikzname\ commands should be given inside this
  environment, except for the |\tikzstyle| command. Unlike other
  packages, it is not possible to use, say, |\pgfpathmoveto| outside
  this environment and doing so will result in chaos. For \tikzname,
  commands like |\path| are only defined inside this environment, so
  there is little chance that you will do something wrong here. 

  When this environment is encountered, the \meta{options} are
  parsed. All options given here will apply to the whole
  picture. 

  Next, the contents of the environment is processed and the graphic
  commands therein are put into a box. Non-graphic text is suppressed
  as well as possible, but non-\pgfname\ commands inside a
  |{tikzpicture}| environment should not produce any ``output'' since
  this may totally scramble the positioning system of the backend
  drivers. The suppressing of normal text, by the way, is done by
  temporarily switching the font to |\nullfont|. You can, however,
  ``escape back'' to normal \TeX\ typesetting. This happens, for
  example, when you specify a node.

  At the end of the environment, \pgfname\ tries to make a good guess
  at a good guess at the bounding box of the graphic and
  then resizes the box such that the box has this size. To ``make its
  guess,'' everytime \pgfname\ encounters a coordinate, it updates the
  bound box's size such that it encompasses all these
  coordinates. This will usually give a good 
  approximation at the bounding box, but will not always be
  accurate. First, the line thickness is not taken into
  account. Second, controls points of a curve often lie far
  ``outside'' the curve and make the bounding box too large. In this
  case, you should use the |[use as bounding box]| option.

  The following option influences the baseline of the resulting
  picture:
  \begin{itemize}
    \itemoption{baseline}\opt{|=|\meta{dimension}}
    Normally, the lower end of the picture is put on the baseline of
    the surrounding text. For example, when you give the code
    |\tikz\draw(0,0)circle(.5ex);|, \pgfname\ will find out that the
    lower end of the picture is at $-.5\mathrm{ex}$ and that the upper
    end is at $.5\mathrm{ex}$. Then, the lower end will be put on the
    baseline, resulting in the following: \tikz\draw(0,0)circle(.5ex);.

    Using this option, you can specify that the picture should be
    raised or lowered such that the height \meta{dimension} is on the
    baseline. For example, |tikz[baseline=0pt]\draw(0,0)circle(.5ex);|
    yields \tikz[baseline=0pt]\draw(0,0)circle(.5ex); since, now, the
    baseline is on the height of the $x$-axis. If you omit the
    \meta{dimensions}, |0pt| is assumed as default.

    This options is often useful for ``inlined'' graphics as in
\begin{codeexample}[]
$A \mathbin{\tikz[baseline] \draw[->>] (0pt,.5ex) -- (3ex,.5ex);} B$
\end{codeexample}

    \itemoption{execute at begin picture}|=|\meta{code}
    This option can be used to install some code that will be executed
    at the beginning of the picture. This option must be
    given in the argument of the |{tikzpicture}| environment itself
    since this option will not have an effect otherwise. After all,
    the picture has already ``started'' later on.

    This option is mainly used in styles like the |every picture|
    style to execute certain code at the start  of a picture.

    \itemoption{execute at end picture}|=|\meta{code}
    This option installs some code that will be executed
    at the end of the picture. Using this option multiple times will
    cause the code to accumulate. This option must also be given in
    the optional argument of the |{tikzpicture}| environment.

\begin{codeexample}[]
\begin{tikzpicture}[execute at end picture=%
  {
    \begin{pgfonlayer}{background}
      \path[fill=yellow,rounded corners]
        (current bounding box.south west) rectangle
        (current bounding box.north east);
    \end{pgfonlayer}
  }]
  \node at (0,0) {X};
  \node at (2,1) {Y};
\end{tikzpicture}
\end{codeexample}
  \end{itemize}
  
  All options ``end'' at the end of the picture. To set an option
  ``globally'' you can use the following style:
  \begin{itemize}
    \itemstyle{every picture}
    This style is installed at the beginning of each picture.
\begin{codeexample}[code only]
\tikzstyle{every picture}=[semithick]
\end{codeexample}
  \end{itemize}
\end{environment}

In plain \TeX, you should use instead the following commands:

\begin{plainenvironment}{{tikzpicture}\opt{\oarg{options}}}
\end{plainenvironment}


\subsubsection{Creating a Picture Using a Command}

The following two commands are used for ``small'' graphics.

\begin{command}{\tikz\opt{\oarg{options}}\marg{commands}}
  This command places the \meta{commands} inside a
  |{tikzpicture}| environment and adds a semicolon at the end. This is
  just a convenience.

  The \meta{commands} may not contain a paragraph (an empty
  line). This is a precaution to ensure that users really use this
  command only for small graphics.

  \example |\tikz{\draw (0,0) rectangle (2ex,1ex)}| yields
  \tikz{\draw (0,0) rectangle (2ex,1ex);} 
\end{command}


\begin{command}{\tikz\opt{\oarg{options}}\meta{text}|;|}
  If the \meta{text} does not start with an opening brace, the end of
  the \meta{text} is the next semicolon that is encountered.

  \example |\tikz \draw (0,0) rectangle (2ex,1ex);| yields
  \tikz \draw (0,0) rectangle (2ex,1ex);
\end{command}



\subsubsection{Adding a Background}

By default, pictures do not have any background, that is, they are
``transparent'' on all parts on which you do not draw
anything. You may instead wish to have a colored background behind
your picture or a black frame around it or lines above and below it or
some other kind of decoration.

Since backgrounds are often not needed at all, the definition of
styles for adding backgrounds has been put in the library package
|pgflibrarytikzbackgrounds|. This package is documented in
Section~\ref{section-tikz-backgrounds}. 


\subsection{Using Scopes to Structure a Picture}

Inside a |{tikzpicture}| environment you can create scopes
using the |{scope}| environment. This environment is available only
inside the |{tikzpicture}| environment, so once more, there is little
chance of doing anything wrong.

\begin{environment}{{scope}\opt{\oarg{options}}}
  All \meta{options} are local to the \meta{environment
  contents}. Furthermore, the clipping path is also local to the
  environment, that is, any clipping done inside the environment
  ``ends'' at its end.

\begin{codeexample}[]
\begin{tikzpicture}
  \begin{scope}[red]
    \draw (0mm,0mm) -- (10mm,0mm);
    \draw (0mm,1mm) -- (10mm,1mm);
  \end{scope}
  \draw (0mm,2mm) -- (10mm,2mm);
  \begin{scope}[green]
    \draw (0mm,3mm) -- (10mm,3mm);
    \draw (0mm,4mm) -- (10mm,4mm);
    \draw[blue] (0mm,5mm) -- (10mm,5mm);
  \end{scope}
\end{tikzpicture}
\end{codeexample}
  
  The following style influences scopes:
  \begin{itemize}
    \itemstyle{every scope}
    This style is installed at the beginning of every scope. I do not
    know really know what this might be good for, but who knows?
  \end{itemize}

  The following options are useful for scopes:
  \begin{itemize}
    \itemoption{execute at begin scope}|=|\meta{code}
    This option install some code that will be executed
    at the beginning of the scope. This option must be
    given in the argument of the |{scope}| environment.

    The effect applies only to the current scope, not to subscopes.

    \itemoption{execute at end scope}|=|\meta{code}
    This option installs some code that will be executed
    at the end of the  current scope. Using this option multiple times
    will  cause the code to accumulate. This option must also be given
    in the optional argument of the |{scope}| environment. 

    Again, the effect applies only to the current scope, not to subscopes.
  \end{itemize}
\end{environment}


In plain \TeX, you use the following commands instead:

\begin{plainenvironment}{{scope}\opt{\oarg{options}}}
\end{plainenvironment}



\subsection{Using Scopes Inside Paths}

The |\path| command, which is described in much more detail in later
sections, also takes graphic options. These options are local to the
path. Furthermore, it is possible to create local scopes within a
path simply by using curly braces as in
\begin{codeexample}[]
\tikz \draw (0,0) -- (1,1)
           {[rounded corners] -- (2,0) -- (3,1)}
           -- (3,0) -- (2,1);
\end{codeexample}

Note that many options apply only to the path as a whole and cannot be
scoped in this way. For example, it is not possible to scope the
|color| of the path. See the explanations in the section on paths for
more details.

Finally, certain elements that you specify in the argument to the
|\path| command also take local options. For example, a node
specification takes options. In this case, the options apply only to
the node, not to the surrounding path.



\subsection{Using Styles to Manage How Pictures Look}

There is a way of organizing sets of graphic options ``orthogonally''
to the normal scoping mechanism. For example, you might wish all your
``help lines'' to be drawn in a certain way like, say, gray and thin
(do \emph{not} dash them, that distracts). For this, you can use
\emph{styles}.

A style is simply a set of graphic options that is predefined at some
point. Once a style has been defined, it can be used anywhere using
the |style| option:

\begin{itemize}
  \itemoption{style}|=|\meta{style name}
  invokes all options that are currently set in the \meta{style
    name}. An example of a style is the predefined |help lines| style,
  which you should use for lines in the background like grid lines or
  construction lines. You can easily define new styles and modify
  existing ones.
\begin{codeexample}[]
\begin{tikzpicture}
  \draw                   (0,0) grid +(2,2);
  \draw[style=help lines] (2,0) grid +(2,2);
\end{tikzpicture}
\end{codeexample}
\end{itemize}


\begin{command}{\tikzstyle\meta{style name}\opt{|+|}|=[|\meta{options}|]|}
  This command defines the style \meta{style name}. Whenever it is
  used using the |style=|\meta{style name} command, the \meta{options}
  will be invoked. It is permissible that a style invokes another
  style using the |style=| command inside the \meta{options}, which
  allows you to build hierarchies of styles. Naturally, you should
  \emph{not} create cyclic dependencies.

  If the style already has a predefined meaning, it will
  unceremoniously be redefined without a warning.
\begin{codeexample}[]
\tikzstyle{help lines}=[blue!50,very thin]
\begin{tikzpicture}
  \draw                   (0,0) grid +(2,2);
  \draw[style=help lines] (2,0) grid +(2,2);
\end{tikzpicture}
\end{codeexample}

  If the optional |+| is given, the options are \emph{added} to the
  existing definition:
\begin{codeexample}[]
\tikzstyle{help lines}+=[dashed]% aaarghhh!!!
\begin{tikzpicture}
  \draw                   (0,0) grid +(2,2);
  \draw[style=help lines] (2,0) grid +(2,2);
\end{tikzpicture}
\end{codeexample}
\end{command}




% Copyright 2003 by Till Tantau <tantau@cs.tu-berlin.de>.
%
% This program can be redistributed and/or modified under the terms
% of the LaTeX Project Public License Distributed from CTAN
% archives in directory macros/latex/base/lppl.txt.


\section{Specifying Coordinates}


\subsection{Coordinates and Coordinate Options}

A \emph{coordinate} is a position in a picture. \tikzname\ uses a
special syntax for specifying coordinates. Coordinates are always put
in round brackets. The general syntax is
\declare{|(|\opt{|[|\meta{options}|]|}\meta{coordinate  specification}|)|}. 

It is possible to give options that apply only to a single
coordinate, although this makes sense for transformation options
only. To give transformation options for a single coordinate, give
these options at the beginning in brackets:
\begin{codeexample}[]
\begin{tikzpicture}
  \draw[style=help lines] (0,0) grid (3,2);
  \draw      (0,0) -- (1,1);
  \draw[red] (0,0) -- ([xshift=3pt] 1,1);
  \draw      (1,0) -- +(30:2cm);
  \draw[red] (1,0) -- +([shift=(135:5pt)] 30:2cm);
\end{tikzpicture}
\end{codeexample}

\subsection{Simple Coordinates}

The simplest way is to specify as a comma-separated pair of \TeX\
dimensions as in |(1cm,2pt)|. As can be seen, different units can be
mixed. The coordinate specified in this way means ``1cm to the right
and 2pt up from the origin of the picture.'' You can also write things
like |(1cm+2pt,2pt)| since the |calc| package is used, internally.


\subsection{Polar Coordinates}

You can also specify coordinates in polar coordinates. In this case,
you specify an angle and a distance, separated by a colon as in
|(30:1cm)|. The angle must always be given in degrees and should be
between $-360$ and $720$. 

\begin{codeexample}[]
\tikz \draw    (0cm,0cm) -- (30:1cm) -- (60:1cm) -- (90:1cm)
            -- (120:1cm) -- (150:1cm) -- (180:1cm);
\end{codeexample}

Instead of an angle given as a number you can also use certain
words. For example, |up| is the same as |90|, so that you can write
|\tikz \draw (0,0) -- (2ex,0pt) -- +(up:1ex);|
and get \tikz \draw (0,0) -- (2ex,0pt) -- +(up:1ex);. Apart from |up|
you can use |down|, |left|, |right|, |north|, |south|, |west|, |east|,
|north east|, |north west|, |south east|, |south west|, all of which
have their natural meaning.



\subsection{xy-, and xyz-Coordinates}
Next, you can specify coordinates in \pgfname's $xy$-coordinate system. In
this case, you provide two unit-free numbers, separated by a comman as
in |(2,-3)|. This means ``add twice the current \pgfname\ $x$-vector and
subtract three times the $y$-vector.'' By default, the $x$-vector
points 1cm to the right, the $y$-vector points 1cm upwards, but this
can be changed arbitrarily using the |x| and~|y| graphic options.

Finally, you can specify coordinate in the $xyz$-coordinate
system. The only difference to the $xy$-coordinates is that you
specify three numbers separated by commas as in |(1,2,3)|. This is
interpreted as ``once the $x$-vector plus twice the $y$-vector plus
three times the $z$-vector.'' The default $z$-vector points to
$\bigl(-\frac{1}{\sqrt2}
\textrm{cm},-\frac{1}{\sqrt2}\textrm{cm}\bigr)$. Consider the
following example: 

\begin{codeexample}[]
\begin{tikzpicture}[->]
  \draw (0,0,0) -- (1,0,0);
  \draw (0,0,0) -- (0,1,0);
  \draw (0,0,0) -- (0,0,1);
\end{tikzpicture}
\end{codeexample}


\subsection{Node Coordinates}
\label{section-node-coordinates}

In \pgfname\ and in \tikzname\ it is quite easy to define a node that you
wish to reference at a later point. Once you have defined a node,
there are different ways of referencing points of the node.


\subsubsection{Named Anchor Coordinates}

An \emph{anchor coordiante} is a point in a node that you have
previously defined using the curly braces syntax. The syntax is
|(|\meta{node name}|.|\meta{anchor}|)|, where \meta{node name} is
the name that was previously used to name the node using the
|name=|meta{node name} option or the special node name syntax. Here is
an example: 

\begin{codeexample}[]
\begin{tikzpicture}
  \node (shape)   at (0,2)  [draw] {|class Shape|};
  \node (rect)    at (-2,0) [draw] {|class Rectangle|};
  \node (circle)  at (2,0)  [draw] {|class Circle|};
  \node (ellipse) at (6,0)  [draw] {|class Ellipse|};

  \draw (circle.north) |- (0,1);
  \draw (ellipse.north) |- (0,1);
  \draw[-open triangle 90] (rect.north) |- (0,1) -| (shape.south);
\end{tikzpicture}
\end{codeexample}

Section~\ref{section-the-shapes} explain which anchors are available
for the basic shapes. 




\subsubsection{Angle Anchor Coordinates}

In addition to the named anchors, it is possible to use the syntax
\meta{node name}|.|\meta{angle} to name a point of the node's
border. This point is the coordinate where a ray shot from the center
in the given angle hits the border. Here is an example:

\begin{codeexample}[]
\begin{tikzpicture}
  \node (start) [draw,shape=ellipse] {start};
  \foreach \angle in {-90, -80, ..., 90}
    \draw (start.\angle) .. controls +(\angle:1cm) and +(-1,0) .. (2.5,0);
  \end{tikzpicture}
\end{codeexample}


\subsubsection{Anchor-Free Node Coordinates}

It is also possible to just ``leave out'' the anchor and have \tikzname\
calculate an appropriate border position for you. Here is an example:

\begin{codeexample}[]
\begin{tikzpicture}[fill=blue!20]
  \draw[style=help lines] (-1,-2) grid (6,3);
  \path (0,0)  node(a) [ellipse,rotate=10,draw,fill]    {An ellipse}
        (3,-1) node(b) [circle,draw,fill]               {A circle}
        (2,2)  node(c) [rectangle,rotate=20,draw,fill]  {A rectangle}
        (5,2)  node(d) [rectangle,rotate=-30,draw,fill] {Another rectangle};
  \draw[thick] (a) -- (b) -- (c) -- (d);
  \draw[thick,red,->] (a) |- +(1,3) -| (c) |- (b);       
  \draw[thick,blue,<->] (b) .. controls +(right:2cm) and +(down:1cm) .. (d);       
\end{tikzpicture}
\end{codeexample}

\tikzname\ will be reasonably clever at determining the border points that
you ``mean,'' but, naturally, this may fail in some siutations. If
\tikzname\ fails to determine an appropriate border point, the center will
be used, instead.

Automatic computation of anchors works only with the lineto command
|--|, the vertical/horizontal versions \verb!|-! and \verb!-|!, and
with the curveto command |..|. For other path commands such as
|parabola| or |plot|, the center will be used. If this is not desired,
you should give a named anchor or an angle anchor.

Note that if you use an automatic coordinate for both the start and
the end of a lineto, as in |--(b)--|, then \emph{two} border
coordinates are computed with a moveto between them. This is usually
exactly what you want.

If you use relative coordinates together with automatic anchor
coordinates, the relative coordinates are always computed relative to
the node's center, not relative to the border point. Here is an
example:

\begin{codeexample}[]
\tikz \draw (0,0) node(x) [draw] {Text}
            rectangle (1,1)
            (x) -- +(1,1);
\end{codeexample}

Similarly, the control points in the following examples both control
points are $(1,1)$:

\begin{codeexample}[]
\tikz \draw (0,0) node(x) [draw] {X}
            (2,0) node(y) {Y}
            (x) .. controls +(1,1) and +(-1,1) .. (y);
\end{codeexample}

           
\subsubsection{Intersection Coordinates}

Often you wish to specify a point that is on the
intersection of a vertical line going through a point $p$ and a
horizontal line going through some other point $q$.

There are two ways of specifying this intersection. The first is 
\declare{|(|\meta{p}\verb! |- !\meta{q}|)|}, the second is
\declare{|(|\meta{q}\verb! -| !\meta{p}|)|}.

For example, \verb!(2,1 |- 3,4)! and  \verb!(3,4 -| 2,1)! both yield
the same as \verb!(2,4)! (provided the $xy$-coordinate system has not
been modified). 

The most useful application of the syntax is to draw a line up to some
point on a vertical or horizontal line. Here is an example:

\begin{codeexample}[]
\begin{tikzpicture}
  \path (30:1cm) node(p1) {$p_1$}   (75:1cm) node(p2) {$p_2$};

  \draw (-0.2,0) -- (1.2,0) node(xline)[right] {$q_1$};
  \draw (2,-0.2) -- (2,1.2) node(yline)[above] {$q_2$};

  \draw[->] (p1) -- (p1 |- xline);
  \draw[->] (p2) -- (p2 |- xline);
  \draw[->] (p1) -- (p1 -| yline);
  \draw[->] (p2) -- (p2 -| yline);
\end{tikzpicture}
\end{codeexample}

\subsection{Relative and Incremental Coordinates}

You can prefix coordinates by |++| to make them ``relative.'' A
coordinate such as |++(1cm,0pt)| means ``1cm to the right of the
previous position.'' Relative coordinates are often useful in
``local'' contexts:

\begin{codeexample}[]
\begin{tikzpicture}
  \draw (0,0)     -- ++(1,0) -- ++(0,1) -- ++(-1,0) -- cycle;
  \draw (2,0)     -- ++(1,0) -- ++(0,1) -- ++(-1,0) -- cycle;
  \draw (1.5,1.5) -- ++(1,0) -- ++(0,1) -- ++(-1,0) -- cycle;
\end{tikzpicture}
\end{codeexample}

Intead of |++| you can also use a single |+|. This also specifies a
relative coordinate, but it does not ``update'' the current point for
subsequent usages of relative coordinates. Thus, you can use this
notation to specify numerous points, all relative to the same
``initial'' point:

\begin{codeexample}[]
\begin{tikzpicture}
  \draw (0,0)     -- +(1,0) -- +(1,1) -- +(0,1) -- cycle;
  \draw (2,0)     -- +(1,0) -- +(1,1) -- +(0,1) -- cycle;
  \draw (1.5,1.5) -- +(1,0) -- +(1,1) -- +(0,1) -- cycle;
\end{tikzpicture}
\end{codeexample}

There is one special situation, where relative coordinates are
interpreted differently. If you use a relative coordinate as a control
point of a B�zier curve, the following rule applies: First, a relative
first control point is taken relative to the beginning of the
curve. Second, a relative second control point is taken relative to
the end of the curve. Third, a relative end point of a curve is taken
relative to the start of the curve.

This special behaviour makes it easy to specify that a curve should
``leave or arrives from a certain direction'' at the start or end. In
the following example, the curve ``leaves'' at $30^\circ$ and
``arrives'' at $60^\circ$: 

\begin{codeexample}[]
\begin{tikzpicture}
  \draw (1,0) .. controls +(30:1cm) and +(60:1cm) .. (3,-1);
  \draw[gray,->] (1,0) -- +(30:1cm);
  \draw[gray,<-] (3,-1) -- +(60:1cm);
\end{tikzpicture}
\end{codeexample}

% Copyright 2005 by Till Tantau <tantau@cs.tu-berlin.de>.
%
% This program can be redistributed and/or modified under the terms
% of the LaTeX Project Public License Distributed from CTAN
% archives in directory macros/latex/base/lppl.txt.


\section{Syntax for Path Specifications}

A \emph{path} is a series of straight and curved line segments. It is
specified following a |\path| command and the specification must
follow a special syntax, which is described in the subsections of the
present section.


\begin{command}{\path\meta{specification}|;|}
  This command is available only inside a |{tikzpicture}| environment.

  The \meta{specification} is a long stream of \emph{path
  operations}. Most of these path operations tell \tikzname\ how the path
  is build. For example, when you write |--(0,0)|, you use a
  \emph{line-to operation} and it means ``continue the path from
  wherever you are to the origin.''

  At any point where \tikzname\ expects a path operation, you can also
  give some graphic options, which is a list of options in brackets,
  such as |[rounded corners]|. These options can have different
  effects:
  \begin{enumerate}
  \item
    Some options take ``immediate'' effect and apply to all subsequent
    path operations on the path. For example, the |rounded corners|
    option will round all following corners, but not the corners
    ``before'' and if the |sharp corners| is given later on the path
    (in a new set of brackets), the rounding effect will end.

\begin{codeexample}[]
\tikz \draw (0,0) -- (1,1)
           [rounded corners] -- (2,0) -- (3,1)
           [sharp corners] -- (3,0) -- (2,1);
\end{codeexample}
    Another example are the transformation options, which also apply
    only to subsequent coordinates.
  \item
    The options that have immediate effect can be ``scoped'' by
    putting part of a path in curly braces. For example, the above
    example could also be written as follows:

\begin{codeexample}[]
\tikz \draw (0,0) -- (1,1)
           {[rounded corners] -- (2,0) -- (3,1)}
           -- (3,0) -- (2,1);
\end{codeexample}
  \item
    Some options only apply to the path as a whole. For example, the
    |color=| option for determining the color used for, say, drawing
    the path always applies to all parts of the path. If several
    different colors are given for different parts of the path, only
    the last one (on the outermost scope) ``wins'':
 
\begin{codeexample}[]
\tikz \draw (0,0) -- (1,1)
           [color=red] -- (2,0) -- (3,1)
           [color=blue] -- (3,0) -- (2,1);
\end{codeexample}

    Most options are of this type. In the above example, we would have
    had to ``split up'' the path into several |\path| commands:
\begin{codeexample}[]
\tikz{\draw (0,0) -- (1,1);
      \draw [color=red] (2,0) -- (3,1);
      \draw [color=blue] (3,0) -- (2,1);}
\end{codeexample}
  \end{enumerate}

  By default, the |\path| command does ``nothing'' with the
  path, it just ``throws it away.'' Thus, if you write
  |\path(0,0)--(1,1);|, nothing is drawn 
  in your picture. The only effect is that the area occupied by the
  picture is (possibly) enlarged so that the path fits inside the
  area. To actually ``do'' something with the path, an option like
  |draw| or |fill| must be given somewhere on the path. Commands like
  |\draw| do this implicitly.
  
  Finally, it is also possible to give \emph{node specifications} on a
  path. Such specifications can come at different locations, but they
  are always allowed when a normal path operation could follow. A node
  specification starts with |node|. Basically, the effect is to
  typeset the node's text as normal \TeX\ text and to place
  it at the ``current location'' on the path. The details are explained
  in Section~\ref{section-nodes}.

  Note, however, that the nodes are \emph{not} part of the path in any
  way. Rather, after everything has been done with the path what is
  specified by the path options (like filling and drawing the path due
  to a |fill| and a |draw| option somewhere in the
  \meta{specification}), the nodes are added in a post-processing
  step.   
  
  The following style influences scopes:
  \begin{itemize}
    \itemstyle{every path}
    This style is installed at the beginning of every path. This can
    be useful for (temporarily) adding, say, the |draw| option to
    everything in a scope.
\begin{codeexample}[]
\begin{tikzpicture}[fill=examplefill] % only sets the color
  \tikzstyle{every path}=[draw]           % all paths are drawn
  \fill  (0,0) rectangle +(1,1);
  \shade (2,0) rectangle +(1,1);
\end{tikzpicture}
\end{codeexample}
  \end{itemize}
\end{command}




\subsection{The Move-To Operation}

The perhaps simplest operation is the move-to operation, which is
specified by just giving a coordinate where a path operation is
expected.

\begin{pathoperation}{}{\meta{coordinate}}
  The move-to operation normally starts a path at a certain
  point. This does not cause a line segment to be created, but it  
  specifies the starting point of the next segment. If a path is
  already under construction, that is, if several segments have
  already been created, a move-to operation will start a new part of the
  path that is not connected to any of the previous segments.

\begin{codeexample}[]
\begin{tikzpicture}
  \draw (0,0) --(2,0) (0,1) --(2,1);
\end{tikzpicture}
\end{codeexample}

  In the specification |(0,0) --(2,0) (0,1) --(2,1)| two move-to
  operations are specified: |(0,0)| and |(0,1)|. The other two
  operations, namely |--(2,0)| and |--(2,1)| are line-to operations,
  described next.
\end{pathoperation}


\subsection{The Line-To Operation}

\begin{pathoperation}{--}{\meta{coordinate}}
  The line-to operation extends the current path from the current
  point in a straight line to the given coordinate. The ``current
  point'' is the endpoint of the previous drawing operation or the point
  specified by a prior move-to operation.

  You use two minus signs followed by a coordinate in round
  brackets. You can add spaces before and after the~|--|.

  When a line-to operation is used and some path segment has just been
  constructed, for example by another line-to operation, the two line
  segments become joined. This means that if they are drawn, the point
  where they meet is ``joined'' smoothly. To appreciate the difference,
  consider the following two examples: In the left example, the path
  consists of two path segments that are not joined, but that happen to
  share a point, while in the right example a smooth join is shown.

\begin{codeexample}[]
\begin{tikzpicture}[line width=10pt]
  \draw (0,0) --(1,1)  (1,1) --(2,0);
  \draw (3,0) -- (4,1) -- (5,0);
  \useasboundingbox (0,1.5); % make bounding box higher
\end{tikzpicture}
\end{codeexample}

\end{pathoperation}


\subsection{Horizontal and Vertical Line-To Operations}

Sometimes you want to connect two points via straight lines that are
only horizontal and vertical. For this, you can use two path
construction operations.

{\catcode`\|=12
\begin{pathoperation}{-|}{\meta{coordinate}}
  This operation means ``first horizontal, then vertical.''

  \begin{codeexample}[]
\begin{tikzpicture}
  \draw (0,0) node(a) [draw] {A}  (1,1) node(b) [draw] {B};
  \draw (a.north) |- (b.west);
  \draw[color=red] (a.east) -| (2,1.5) -| (b.north);
\end{tikzpicture}
\end{codeexample}
\end{pathoperation}
\begin{pathoperation}{|-}{\meta{coordinate}}
  This operations means  ``first vertical, then horizontal.''
\end{pathoperation}
}

\subsection{The Curve-To Operation}

The curve-to operation allows you to extend a path using a B�zier
curve.

\begin{pathoperation}{..}{\declare{|controls|}\meta{c}\opt{|and|\meta{d}}\declare{|..|\meta{y}}}
  This operation extends the current path from the current
  point, let us call it $x$, via a curve to a the current point~$y$.
  The curve is a cubic B�zier curve. For such a curve, 
  apart from $y$, you also specify two control points $c$ and $d$. The
  idea is that the curve starts at $x$, ``heading'' in the direction
  of~$c$. Mathematically spoken, the tangent of the curve at $x$ goes
  through $c$. Similarly, the curve ends at $y$, ``coming from'' the
  other control point,~$d$. The larger the distance between $x$ and~$c$
  and between $d$ and~$y$, the larger the curve will be.

  If the ``|and|\meta{d}'' part is not given, $d$ is assumed to be
  equal to $c$.

\begin{codeexample}[]
\begin{tikzpicture}
  \draw[line width=10pt] (0,0) .. controls (1,1) .. (4,0)
                               .. controls (5,0) and (5,1) .. (4,1);
  \draw[color=gray] (0,0) -- (1,1) -- (4,0) -- (5,0) -- (5,1) -- (4,1);
\end{tikzpicture}
\end{codeexample}

  As with the line-to operation, it makes a difference whether two curves
  are joined because they resulted from consecutive curve-to or line-to
  operations, or whether they just happen to have the same ending:

\begin{codeexample}[]
\begin{tikzpicture}[line width=10pt]
  \draw (0,0) -- (1,1) (1,1) .. controls (1,0) and (2,0) .. (2,0);
  \draw (3,0) -- (4,1) .. controls (4,0) and (5,0) .. (5,0);
  \useasboundingbox (0,1.5); % make bounding box higher
\end{tikzpicture}
\end{codeexample}
\end{pathoperation}


\subsection{The Cycle Operation}

\begin{pathoperation}{--cycle}{}
  This operation adds a straight line from the current
  point to the last point specified by a move-to operation. Note that
  this need not be the beginning of the path. Furthermore, a smooth join
  is created between the first segment created after the last move-to
  operation and the straight line appended by the cycle operation.

  Consider the following example. In the left example, two triangles are
  created using three straight lines, but they are not joined at the
  ends. In the second example cycle operations are used.

\begin{codeexample}[]
\begin{tikzpicture}[line width=10pt]
  \draw (0,0) -- (1,1) -- (1,0) -- (0,0) (2,0) -- (3,1) -- (3,0) -- (2,0);
  \draw (5,0) -- (6,1) -- (6,0) -- cycle (7,0) -- (8,1) -- (8,0) -- cycle;
  \useasboundingbox (0,1.5); % make bounding box higher
\end{tikzpicture}
\end{codeexample}
\end{pathoperation}


\subsection{Rounding Corners}

All of the path construction operations mentioned up to now are
influenced by the following option:
\begin{itemize}
  \itemoption{rounded corners}\opt{|=|\meta{inset}}
  When this option is in force, all corners (places where a line is
  continued either via line-to or a curve-to operation) are replaced by
  little arcs so that the corner becomes smooth. 

\begin{codeexample}[]
\tikz \draw [rounded corners] (0,0) -- (1,1)
           -- (2,0) .. controls (3,1) .. (4,0);
\end{codeexample}

  The \meta{inset} describes how big the corner is. Note that the
  \meta{inset} is \emph{not} scaled along if you use a scaling option
  like |scale=2|. 

\begin{codeexample}[]
\begin{tikzpicture}
  \draw[color=gray,very thin] (10pt,15pt) circle (10pt);
  \draw[rounded corners=10pt] (0,0) -- (0pt,25pt) -- (40pt,25pt);
\end{tikzpicture}
\end{codeexample}

  You can switch the rounded corners on and off ``in the middle of
  path'' and different corners in the same path can have different
  corner radii:

\begin{codeexample}[]
\begin{tikzpicture}
  \draw (0,0) [rounded corners=10pt] -- (1,1) -- (2,1)
                     [sharp corners] -- (2,0)
               [rounded corners=5pt] -- cycle;
\end{tikzpicture}
\end{codeexample}

Here is a rectangle with rounded corners:
\begin{codeexample}[]
\tikz \draw[rounded corners=1ex] (0,0) rectangle (20pt,2ex);
\end{codeexample}

  You should be aware, that there are several pitfalls when using this
  option. First, the rounded corner will only be an arc (part of a
  circle) if the angle is $90^\circ$. In other cases, the rounded
  corner will still be round, but ``not as nice.''

  Second, if there are very short line segments in a path, the
  ``rounding'' may cause inadverted effects. In such case it may be
  necessary to temporarily switch off the rounding using
  |sharp corners|. 

  \itemoption{sharp corners}
  This options switches off any rounding on subsequent corners of the
  path.   
\end{itemize}



\subsection{The Rectangle Operation}

A rectangle can obviously be created using four straight lines and a
cycle operation. However, since rectangles are needed so often, a
special syntax is available for them.

\begin{pathoperation}{rectangle}{\meta{corner}}
  When this operation is used, one corner will be the current point,
  another corner is given by \meta{corner}, which becomes the new
  current point.

\begin{codeexample}[]
\begin{tikzpicture}
  \draw (0,0) rectangle (1,1);
  \draw (.5,1) rectangle (2,0.5) (3,0) rectangle (3.5,1.5) -- (2,0);
\end{tikzpicture}
\end{codeexample}
\end{pathoperation}


\subsection{The Circle and Ellipse Operations}

A circle can be approximated well using four B�zier curves. However,
it is difficult to do so correctly. For this reason, a special syntax
is available for adding such an approximation of a circle to the
current path.

\begin{pathoperation}{circle}{|(|\meta{radius}|)|}
  The center of the circle is given by the current point. The new
  current point of the path will remain to be the center of the
  circle.  
\end{pathoperation}

\begin{pathoperation}{ellipse}{|(|\meta{half width}| and |\meta{half height}|)|}
  Note that you can add spaces after |ellipse|, but you have to place
  spaces around |and|.

\begin{codeexample}[]
\begin{tikzpicture}
  \draw (1,0) circle (.5cm);
  \draw (3,0) ellipse (1cm and .5cm) -- ++(3,0) circle (.5cm)
    -- ++(2,-.5) circle (.25cm);
\end{tikzpicture}
\end{codeexample}
\end{pathoperation}


\subsection{The Arc Operation}

The \emph{arc operation} allows you to add an arc to the current
path.
\begin{pathoperation}{arc}{|(|\meta{start angle}|:|\meta{end
    angle}|:|\meta{radius}\opt{|/|\meta{half height}}|)|}
  The arc operation adds a part of a circle of the given radius
  between the given angles. The arc will start at the current point
  and will end at the end of the arc.

  \begin{codeexample}[]
\begin{tikzpicture}
  \draw (0,0) arc (180:90:1cm) -- (2,.5) arc (90:0:1cm);
  \draw (4,0) -- +(30:1cm) arc (30:60:1cm) -- cycle;
  \draw (8,0) arc (0:270:1cm/.5cm) -- cycle;
\end{tikzpicture}
\end{codeexample}

\begin{codeexample}[]
\begin{tikzpicture}
  \draw (-1,0) -- +(3.5,0);
  \draw (1,0) ++(210:2cm) -- +(30:4cm);
  \draw (1,0) +(0:1cm) arc (0:30:1cm);      
  \draw (1,0) +(180:1cm) arc (180:210:1cm);
  \path (1,0) ++(15:.75cm) node{$\alpha$};
  \path (1,0) ++(15:-.75cm) node{$\beta$};
\end{tikzpicture}
\end{codeexample}
\end{pathoperation}


\subsection{The Grid Operation}

You can add a grid to the current path using the |grid| path
operation. 

\begin{pathoperation}{grid}{\opt{\oarg{options}}\meta{corner}}
  This operations adss a grid filling a rectangle whose two corners
  are given by \meta{corner} and by the previous coordinate. Thus, the
  typical way in which a grid is drawn is |\draw (1,1) grid (3,3);|,
  which yields a grid filling the rectangle whose corners are at
  $(1,1)$ and $(3,3)$. All coordinate transformations apply to the grid.

\begin{codeexample}[]
\tikz[rotate=30] \draw[step=1mm] (0,0) grid (2,2);
\end{codeexample}

  The stepping of the grid is governed by the following options:

\begin{itemize}
  \itemoption{step}|=|\meta{dimension} sets the stepping in both the
  $x$ and $y$-direction.
  \itemoption{xstep}|=|\meta{dimension} sets the stepping in the
  $x$-direction. 
  \itemoption{ystep}|=|\meta{dimension} sets the stepping in the
  $y$-direction. 
\end{itemize}

  It is important to note that the grid is always ``phased'' such that
  it contains the point $(0,0)$ if that point happens to be inside the
  rectangle. Thus, the grid does \emph{not} always have an intersection
  at the corner points; this occurs only if the corner points are
  multiples of the stepping. Note that due to rounding errors, the
  ``last'' lines of a grid may be omitted. In this case, you have to
  add an epsilon to the corner points.

  The following style is useful for drawing grids:
\begin{itemize}
  \itemstyle{help lines}
  This style makes lines ``subdued'' by using thin gray lines for
  them. However, this style is not installed automatically and you
  have to say for example:
\begin{codeexample}[]
\tikz \draw[style=help lines] (0,0) grid (3,3);
\end{codeexample}
\end{itemize}
\end{pathoperation}



\subsection{The Parabola Operation}

The |parabola| path operation continues the current path with a
parabola. A parabola is a (shifted and scaled) curve defined by the
equation $f(x) = x^2$ and looks like this: \tikz \draw (-1ex,1.5ex)
parabola[parabola height=-1.5ex] +(2ex,0ex);.

\begin{pathoperation}{parabola}{\opt{\oarg{options}|bend|\meta{bend
        coordinate}}\meta{coordinate}}
  This operation adds a parabola through the current point and the
  given \meta{coordinate}. If the |bend| is given, it specifies where
  the bend should go; the \meta{options} can also be used to specify
  where the bend is. By default, the bend is at the old current point. 

\begin{codeexample}[]
\begin{tikzpicture}
  \draw               (0,0) rectangle                (1,1.5)
                      (0,0) parabola                 (1,1.5);
  \draw[xshift=1.5cm] (0,0) rectangle                (1,1.5)
                      (0,0) parabola[bend at end]    (1,1.5);
  \draw[xshift=3cm]   (0,0) rectangle                (1,1.5)
                      (0,0) parabola bend (.75,1.75) (1,1.5);
\end{tikzpicture}
\end{codeexample}

  The following options influence parabolas:
\begin{itemize}
  \itemoption{bend}|=|\meta{coordinate}
  Has the same effect as saying |bend|\meta{coordinate} outside the
  \meta{options}. The option specifies that the bend of the parabola
  should be at the given \meta{coordinate}. You have to take care
  yourself that the bend position is a ``valid'' position; which means
  that if there is no parabola of the form $f(x) = a x^2 + b x + c$
  that goes through the old current point, the given bend, and the new
  current point, the result will not be a parabola.

  There is one special property of the \meta{coordinate}: When a
  relative coordinate is given like |+(0,0)|, the position relative
  to which this coordinate is ``flexible.'' More precisely, this
  position lies somewhere on a line from the old current point to the
  new current point. The exact position depends on the next
  option.

  \itemoption{bend pos}|=|\meta{fraction}
  Specifies where the ``previous'' point is relative to which the bend
  is calculated. The previous point will be at the \meta{fraction}th
  part of the line from the old current point to the new current
  point.

  The idea is the following: If you say |bend pos=0| and
  |bend +(0,0)|, the bend will be at the old current point. If you say
  |bend pos=1| and |bend +(0,0)|, the bend will be at the new current
  point. If you say |bend pos=0.5| and |bend +(0,2cm)| the bend will
  be 2cm above the middle of the line between the start and end
  point. This is most useful in situations such as the following:
\begin{codeexample}[]
\begin{tikzpicture}
  \draw[help lines] (0,0) grid (3,2);
  \draw (-1,0) parabola[bend pos=0.5] bend +(0,2) +(3,0);
\end{tikzpicture}
\end{codeexample}

  In the above example, the |bend +(0,2)| essentially means ``a
  parabola that is 2cm high'' and |+(3,0)| means ``and 3cm wide.''
  Since this situation arises often, there is a special shortcut
  option:
  \itemoption{parabola height}|=|\meta{dimension} This option has the
  same effect as if you had written the following instead:
  |[bend pos=0.5,bend={+(0pt,|\meta{dimension}|)}]|. 
\begin{codeexample}[]
\begin{tikzpicture}
  \draw[help lines] (0,0) grid (3,2);
  \draw (-1,0) parabola[parabola height=2cm] +(3,0);
\end{tikzpicture}
\end{codeexample}
\end{itemize}

The following styles are useful shortcuts:
\begin{itemize}
  \itemstyle{bend at start} This places the bend at the start of a
  parabola. It is a shortcut for the following options:
  |bend pos=0,bend={+(0,0)}|. 
  \itemstyle{bend at end} This places the bend at the end of a
  parabola.
\end{itemize}
\end{pathoperation}


\subsection{The Sine and Cosine Operation}

The |sin| and |cos| operations are similar to the |parabola|
operation. They, too, can be used to draw (parts of) a sine or cosine
curve.

\begin{pathoperation}{sin}{\meta{coordinate}}
  The effect of |sin| is to draw a scaled and shifted version of a sine
  curve in the interval $[0,\pi/2]$. The scaling and shifting is done in
  such a way that the start of the sine curve in the interval is at the
  old current point and that the end of the curve in the interval is at
  \meta{coordinate}. Here is an example that should clarify this:

\begin{codeexample}[]
\tikz \draw (0,0) rectangle (1,1)     (0,0) sin (1,1)
            (2,0) rectangle +(1.57,1) (2,0) sin +(1.57,1);
\end{codeexample}
\end{pathoperation}

\begin{pathoperation}{cos}{\meta{coordinate}}
  This operation works similarly, only a cosine in the interval
  $[0,\pi/2]$ is drawn. By correctly alternating |sin| and |cos|
  operations, you can create a complete sine or cosine curve:

\begin{codeexample}[]
\begin{tikzpicture}[xscale=1.57]
  \draw (0,0) sin (1,1) cos (2,0) sin (3,-1) cos (4,0) sin (5,1);
  \draw[color=red] (0,1.5) cos (1,0) sin (2,-1.5) cos (3,0) sin (4,1.5) cos (5,0);
\end{tikzpicture}
\end{codeexample}
\end{pathoperation}

Note that there is no way to (conveniently) draw an interval on a sine
or cosine curve whose end points are not multiples of $\pi/2$.



\subsection{The Plot Operation}

The |plot| operation can be used to append a line or curve to the path
that goes through a large number of coordinates. These coordinates are
either given in a simple list of coordinates or they are read from
some file.

The syntax of the |plot| comes in different versions.

\begin{pathoperation}{--plot}{\meta{further arguments}}
  This operation plots the curve through the coordinates specified in
  the \meta{further arguments}. The current (sub)path is simply
  continued, that is, a line-to operation to the first point of the
  curve is implicitly added. The details of the \meta{further
    arguments}  will be explained in a moment.
\end{pathoperation}

\begin{pathoperation}{plot}{\meta{further arguments}}
  This operation plots the curve through the coordinates specified in
  the \meta{further arguments} by first ``moving'' to the first
  coordinate of the curve.
\end{pathoperation}

The \meta{further arguments} are used in three different ways to
specifying the coordinates of the points to be plotted:

\begin{enumerate}
\item
  \opt{|--|}|plot|\oarg{local options}\declare{|coordinates{|\meta{coordinate
    1}\meta{coordinate 2}\dots\meta{coordinate $n$}|}|}
\item
  \opt{|--|}|plot|\oarg{local options}\declare{|file{|\meta{filename}|}|}
\item
  \opt{|--|}|plot|\oarg{local options}\declare{|function{|\meta{gnuplot formula}|}|}
\end{enumerate}

These different ways are explained in the following.


\subsubsection{Plotting Points Given Inline}

In the first two cases, the points are given directly in the \TeX-file
as in the following example:

\begin{codeexample}[]
\tikz \draw plot coordinates {(0,0) (1,1) (2,0) (3,1) (2,1) (10:2cm)};
\end{codeexample}

Here is an example showing the difference between |plot| and |--plot|:

\begin{codeexample}[]
\begin{tikzpicture}
  \draw (0,0) -- (1,1) plot coordinates {(2,0)  (4,0)};
  \draw[color=red,xshift=5cm]
        (0,0) -- (1,1) -- plot coordinates {(2,0)  (4,0)};
\end{tikzpicture}
\end{codeexample}


\subsubsection{Plotting Points Read From an External File}

The second way of specifying points is to put them in an external
file named \meta{filename}. Currently, the only file format that
\tikzname\ allows is the following: Each line of the \meta{filename}
should contain one line starting with two numbers, separated by a
space. Anything following the two numbers on the line is
ignored. Also, lines starting with a |%| or a |#| are ignored as well
as empty lines. (This is exactly the format that \textsc{gnuplot}
produces when you say |set terminal table|.) If necessary, more
formats will be supported in the future, but it is usually easy to
produce a file containing data in this form.

\begin{codeexample}[]
\tikz \draw plot[mark=x,smooth] file {plots/pgfmanual-sine.table};
\end{codeexample}

The file |plots/pgfmanual-sine.table| reads:
\begin{codeexample}[code only]
#Curve 0, 20 points
#x y type
0.00000 0.00000  i
0.52632 0.50235  i
1.05263 0.86873  i
1.57895 0.99997  i
...
9.47368 -0.04889  i
10.00000 -0.54402  i
\end{codeexample}
It was produced from the following source, using |gnuplot|:
\begin{codeexample}[code only]
set terminal table
set output "../plots/pgfmanual-sine.table"
set format "%.5f"
set samples 20
plot [x=0:10] sin(x)
\end{codeexample}

The \meta{local options} of the |plot| operation are local to each
plot and do not affect other plots ``on the same path.'' For example,
|plot[yshift=1cm]| will locally shift the plot 1cm upward. Remember,
however, that most options can only be applied to paths as a
whole. For example, |plot[red]| does not have the effect of making the
plot red. After all, you are trying to ``locally'' make part of the
path red, which is not possible.

\subsubsection{Plotting a Function}
\label{section-tikz-gnuplot}

Often, you will want to plot points that are given via a function like
$f(x) = x \sin x$. Unfortunately, \TeX\ does not really have enough
computational power to generate the points on such a function
efficiently (it is a text processing program, after all). However,
if you allow it, \TeX\ can try to call external programs that can
easily produce the necessary points. Currently, \tikzname\ knows how to
call \textsc{gnuplot}.

When \tikzname\ encounters your operation
|plot[id=|\meta{id}|] function{x*sin(x)}| for 
the first time, it will create a file called
\meta{prefix}\meta{id}|.gnuplot|, where \meta{prefix} is |\jobname.| by
default, that is, the name of you main |.tex| file. If no \meta{id} is
given, it will be empty, which is alright, but it is better when each
plot has a unique \meta{id} for reasons explained in a moment. Next,
\tikzname\ writes some initialization code into this file followed by
|plot x*sin(x)|. The initialization code sets up things 
such that the |plot| operation will write the coordinates into another
file called \meta{prefix}\meta{id}|.table|. Finally, this table file
is read as if you had said |plot file{|\meta{prefix}\meta{id}|.table}|. 

For the plotting mechanism to work, two conditions must be met:
\begin{enumerate}
\item
  You must have allowed \TeX\ to call external programs. This is often
  switched off by default since this is a security risk (you might,
  without knowing, run a \TeX\ file that calls all sorts of ``bad''
  commands). To enable this ``calling external programs'' a command
  line option must be given to the \TeX\ program. Usually, it is
  called something like |shell-escape| or |enable-write18|. For
  example, for my |pdflatex| the option |--shell-escape| can be
  given.
\item
  You must have installed the |gnuplot| program and \TeX\ must find it
  when compiling your file.
\end{enumerate}

Unfortunately, these conditions will not always be met. Especially if
you pass some source to a coauthor and the coauthor does not have
\textsc{gnuplot} installed, he or she will have trouble compiling your
files.

For this reason, \tikzname\ behaves differently when you compile your
graphic for the second time: If upon reaching
|plot[id=|\meta{id}|] function{...}| the file \meta{prefix}\meta{id}|.table|
already exists \emph{and} if the \meta{prefix}\meta{id}|.gnuplot| file
contains what \tikzname\ thinks that it ``should'' contain, the |.table|
file is immediately read without trying to call a |gnuplot|
program. This approach has the following advantages: 
\begin{enumerate}
\item
  If you pass a bundle of your |.tex| file and all |.gnuplot| and
  |.table| files to someone else, that person can \TeX\ the |.tex|
  file without having to have |gnuplot| installed.
\item
  If the |\write18| feature is switched off for security reasons (a
  good idea), then, upon the first compilation of the |.tex| file, the
  |.gnuplot| will still be generated, but not the |.table|
  file. You can then simply call |gnuplot| ``by hand'' for each
  |.gnuplot| file, which will produce all necessary |.table| files.
\item
  If you change the function that you wish to plot or its
  domain, \tikzname\ will automatically try to regenerate the |.table|
  file.
\item
  If, out of laziness, you do not provide an |id|, the same |.gnuplot|
  will be used for different plots, but this is not a problem since
  the |.table| will automatically be regenerated for each plot
  on-the-fly. \emph{Note: If you intend to share your files with
  someone else, always use an id, so that the file can by typeset
  without having \textsc{gnuplot} installed.} Also, having unique ids
  for each plot will improve compilation speed since no external
  programs need to be called, unless it is really necessary.
\end{enumerate}

When you use |plot function{|\meta{gnuplot formula}|}|, the \meta{gnuplot
  formula} must be given in the |gnuplot| syntax, whose details are
beyond the scope of this manual. Here is the ultra-condensed
essence: Use |x| as the variable and use the C-syntax for normal
plots, use |t| as the variable for parametric plots. Here are some examples:

\begin{codeexample}[]
\begin{tikzpicture}[domain=0:4]
  \draw[very thin,color=gray] (-0.1,-1.1) grid (3.9,3.9);
  
  \draw[->] (-0.2,0) -- (4.2,0) node[right] {$x$};
  \draw[->] (0,-1.2) -- (0,4.2) node[above] {$f(x)$};
  
  \draw[color=red]    plot[id=x]   function{x}           node[right] {$f(x) =x$};
  \draw[color=blue]   plot[id=sin] function{sin(x)}      node[right] {$f(x) = \sin x$};
  \draw[color=orange] plot[id=exp] function{0.05*exp(x)} node[right] {$f(x) = \frac{1}{20} \mathrm e^x$};
\end{tikzpicture}
\end{codeexample}


The following options influence the plot:

\begin{itemize}
  \itemoption{samples}|=|\meta{number}
  sets the number of samples used in the plot. The default is 25.
  \itemoption{domain}|=|\meta{start}|:|\meta{end}
  sets the domain between which the samples are taken. The default is
  |-5:5|. 
  \itemoption{parametric}\opt{|=|\meta{true or false}}
  sets whether the plot is a parametric plot. If true, then |t| must
  be used instead of |x| as the parameter and two comma-separated
  functions must be given in the \meta{gnuplot formula}. An example is
  the following:
\begin{codeexample}[]
\tikz \draw[scale=0.5,domain=-3.141:3.141,smooth]
  plot[parametric,id=parametric-example] function{t*sin(t),t*cos(t)};
\end{codeexample}
  
  \itemoption{id}|=|\meta{id}
  sets the identifier of the current plot. This should be a unique
  identifier for each plot (though things will also work if it is not,
  but not as well, see the explanations above). The \meta{id} will be
  part of a filename, so it should not contain anything fancy like |*|
  or |$|.%$
  \itemoption{prefix}|=|\meta{prefix}
  is put before each plot file name. The default is |\jobname.|, but
  if you have many plots, it might be better to use, say |plots/| and
  have all plots placed in a directory. You have to create the
  directory yourself.
  \itemoption{raw gnuplot}
  causes the \meta{gnuplot formula} to be passed on to
  \textsc{gnuplot} without setting up the samples or the |plot|
  operation. Thus, you could write
\begin{codeexample}[code only]
plot[raw gnuplot,id=raw-example] function{set samples 25; plot sin(x)}
\end{codeexample}
  This can be 
  useful for complicated things that need to be passed to
  \textsc{gnuplot}. However, for really complicated situations you
  should create a special external generating \textsc{gnuplot} file
  and use the |file|-syntax to include the table ``by hand.''
\end{itemize}

The following styles influence the plot:
\begin{itemize}
  \itemstyle{every plot}
  This style is installed in each plot, that is, as if you always said
\begin{codeexample}[code only]
  plot[style=every plot,...]
\end{codeexample}
 This is most useful for globally setting a prefix for all plots by saying:
\begin{codeexample}[code only]
\tikzstyle{every plot}=[prefix=plots/]
\end{codeexample}
\end{itemize}



\subsubsection{Placing Marks on the Plot}

As we saw already, it is possible to add \emph{marks} to a plot using
the |mark| option. When this option is used, a copy of the plot
mark is placed on each point of the plot. Note that the marks are
placed \emph{after} the whole path has been drawn/filled/shaded. In
this respect, they are handled like text nodes. 

In detail, the following options govern how marks are drawn:
\begin{itemize}
  \itemoption{mark}|=|\meta{mark mnemonic}
  Sets the mark to a mnemonic that has previously been defined using
  the |\pgfdeclareplotmark|. By default, |*|, |+|, and |x| are available,
  which draw a filled circle, a plus, and a cross as marks. Many more
  marks become available when the library |pgflibraryplotmarks| is
  loaded. Section~\ref{section-plot-marks} lists the available plot
  marks.

  One plot mark is special: the |ball| plot mark is available only
  it \tikzname. The |ball color| determines the balls's color. Do not use
  this option with large number of marks since it will take very long
  to render in PostScript.
  
  \begin{tabular}{lc}
    Option & Effect \\\hline \vrule height14pt width0pt
    \plotmarkentrytikz{ball}
  \end{tabular}
  
  \itemoption{mark size}|=|\meta{dimension}
  Sets the size of the plot marks. For circular plot marks,
  \meta{dimension} is the radius, for other plot marks
  \meta{dimension} should be about half the width and height.

  This option is not really necessary, since you achieve the same
  effect by specifying |scale=|\meta{factor} as a local option, where
  \meta{factor} is the quotient of the desired size and the default
  size. However, using |mark size| is a bit faster and more natural. 

  \itemoption{mark options}|=|\meta{options}
  These options are applied to marks when they are drawn. For example,
  you can scale (or otherwise transform) the plot mark or set its
  color. 
\begin{codeexample}[]
\tikz \fill[fill=blue!20]
  plot[mark=triangle*,mark options={color=blue,rotate=180}]
    file{plots/pgfmanual-sine.table} |- (0,0);
\end{codeexample}
\end{itemize}



\subsubsection{Smooth Plots, Sharp Plots, and Comb Plots}

There are different things the |plot| operation can do with the points
it reads from a file or from the inlined list of points. By default,
it will connect these points by straight lines. However, you can also
use options to change the behavior of |plot|.

\begin{itemize}
  \itemoption{sharp plot}
  This is the default and causes the points to be connected by
  straight lines. This option is included only so that you can
  ``switch back'' if you ``globally'' install, say, |smooth|.
  
  \itemoption{smooth}
  This option causes the points on the path to be connected using a
  smooth curve:

\begin{codeexample}[]
\tikz\draw plot[smooth] file{plots/pgfmanual-sine.table};
\end{codeexample}

  Note that the smoothing algorithm is not very intelligent. You will
  get the best results if the bending angles are small, that is, less
  than about $30^\circ$ and, even more importantly, if the distances
  between points are about the same all over the plotting path.

  \itemoption{tension}|=|\meta{value}
  This option influences how ``tight'' the smoothing is. A lower value
  will result in sharper corners, a higher value in more ``round''
  curves. A value of $1$ results in a circle if four points at
  quarter-positions on a circle are given. The default is $0.55$. The
  ``correct'' value depends on the details of plot.
  
\begin{codeexample}[]
\begin{tikzpicture}[smooth cycle]
  \draw                 plot[tension=0.2]
    coordinates{(0,0) (1,1) (2,0) (1,-1)};
  \draw[yshift=-2.25cm] plot[tension=0.5]
    coordinates{(0,0) (1,1) (2,0) (1,-1)};
  \draw[yshift=-4.5cm]  plot[tension=1]
    coordinates{(0,0) (1,1) (2,0) (1,-1)};
\end{tikzpicture}
\end{codeexample}
  
  \itemoption{smooth cycle}
  This option causes the points on the path to be connected using a
  closed smooth curve. 

\begin{codeexample}[]
\tikz[scale=0.5]
  \draw plot[smooth cycle] coordinates{(0,0) (1,0) (2,1) (1,2)}
        plot               coordinates{(0,0) (1,0) (2,1) (1,2)} -- cycle;
\end{codeexample}

  \itemoption{ycomb}
  This option causes the |plot| operation to interpret the plotting
  points differently. Instead of connecting them, for each point of
  the plot a straight line is added to the path from the $x$-axis to the point,
  resulting in a sort of ``comb'' or ``bar diagram.''

\begin{codeexample}[]
\tikz\draw[ultra thick] plot[ycomb,thin,mark=*] file{plots/pgfmanual-sine.table};
\end{codeexample}

\begin{codeexample}[]
\begin{tikzpicture}[ycomb]
  \draw[color=red,line width=6pt]
    plot coordinates{(0,1) (.5,1.2) (1,.6) (1.5,.7) (2,.9)};
  \draw[color=red!50,line width=4pt,xshift=3pt]
    plot coordinates{(0,1.2) (.5,1.3) (1,.5) (1.5,.2) (2,.5)};
\end{tikzpicture}
\end{codeexample}

  \itemoption{xcomb}
  This option works like |ycomb| except that the bars are horizontal. 

\begin{codeexample}[]
\tikz \draw plot[xcomb,mark=x] coordinates{(1,0) (0.8,0.2) (0.6,0.4) (0.2,1)};
\end{codeexample}

  \itemoption{polar comb}
  This option causes a line from the origin to the point to be added
  to the path for each plot point.

\begin{codeexample}[]
\tikz \draw plot[polar comb,
     mark=pentagon*,mark options={fill=white,draw=red},mark size=4pt]
   coordinates {(0:1cm) (30:1.5cm) (160:.5cm) (250:2cm) (-60:.8cm)};
\end{codeexample}


  \itemoption{only marks}
  This option causes only marks to be shown; no path segments are
  added to the actual path. This can be useful for quickly adding some
  marks to a path.

\begin{codeexample}[]
\tikz \draw (0,0) sin (1,1) cos (2,0)
  plot[only marks,mark=x] coordinates{(0,0) (1,1) (2,0) (3,-1)};
\end{codeexample}
\end{itemize}



  

\subsection{The Scoping Operation}

When \tikzname\ encounters and opening or a closing brace (|{| or~|}|) at
some point where a path operation should come, it will open or close a
scope. All options that can be applied ``locally'' will be scoped
inside the scope. For example, if you apply a transformation like
|[xshift=1cm]| inside the scoped area, the shifting only applies to
the scope. On the other hand, an option like |color=red| does not have
any effect inside a scope since it can only be applied to the path as
a whole. 


\subsection{The Node Operation}

You can add nodes to a path using the |node| operation. Since this
operation is quite complex and since the nodes are not really part of
the path itself, there is a separate section dealing with nodes, see
Section~\ref{section-nodes}. 

% Copyright 2003 by Till Tantau <tantau@cs.tu-berlin.de>.
%
% This program can be redistributed and/or modified under the terms
% of the LaTeX Project Public License Distributed from CTAN
% archives in directory macros/latex/base/lppl.txt.


\section{Actions on Paths}

Once a path has been constructed, different things can be done with
it. It can be drawn (or stroked) with a ``pen,'' it can be filled with
a color or shading, it can be used for clipping subsequent drawing, it
can be used to specify the extend of the picture---or all or any
combination of these actions at the same time.

To decide what is to be done with a path, two methods can be
used. First, you can use a special-purpose command like |\draw| to
indicate that the path should be drawn. However, commands like |\draw|
and |\fill| are just abbreviations for special cases of the more
general method: Here, the |\path| command is used to specify the
path. Then, options encountered on the path indicate what should be
done with the path.

For example, |\path (0,0) circle (1cm);| means ``This is a path
consisting of a circle around the origin. Do not do anything with it
(throw it away).'' However, if the option |draw| is encountered
anywhere on the path, the circle will be drawn. ``Anywhere'' is any
point on the path where an option can be given, which is everywhere
where a path command like |circle (1cm)| or |rectangle (1,1)| or even
just |(0,0)| would also be allowed. Thus, the following commands all
draw the same circle:
\begin{codeexample}[code only]
\path [draw] (0,0) circle (1cm);
\path (0,0) [draw] circle (1cm);
\path (0,0) circle (1cm) [draw];
\end{codeexample}
Finally, |\draw (0,0) circle (1cm);| also draws a path, because
|\draw| is an abbreviation for |\path [draw]| and thus the command
expands to the first line of the above example.

Similarly, |\fill| is an abbreviation for |\path[fill]| and
|\filldraw| is an abbreviation for the command
|\path[fill,draw]|. Since options accumulate, the following commands
all have the same effect: 
\begin{codeexample}[code only]
\path [draw,fill]   (0,0) circle (1cm);
\path [draw] [fill] (0,0) circle (1cm);
\path [fill] (0,0) circle (1cm) [draw];
\draw [fill] (0,0) circle (1cm);
\fill (0,0) [draw] circle (1cm);
\filldraw (0,0) circle (1cm);
\end{codeexample}

In the following subsection the different actions are explained that
can be used with a path. The following commands are abbreviations for
certain sets of actions, but for many useful combinations there are no
abbreviations:

\begin{command}{\draw}
  Inside |{tikzpicture}| this is an abbreviation for |\path[draw]|.
\end{command}

\begin{command}{\fill}
  Inside |{tikzpicture}| this is an abbreviation for |\path[fill]|.
\end{command}

\begin{command}{\filldraw}
  Inside |{tikzpicture}| this is an abbreviation for |\path[fill,draw]|.
\end{command}

\begin{command}{\shade}
  Inside |{tikzpicture}| this is an abbreviation for |\path[shade]|.
\end{command}

\begin{command}{\shadedraw}
  Inside |{tikzpicture}| this is an abbreviation for |\path[shade,draw]|.
\end{command}

\begin{command}{\clip}
  Inside |{tikzpicture}| this is an abbreviation for |\path[clip]|.
\end{command}

\begin{command}{\useasboundingbox}
  Inside |{tikzpicture}| this is an abbreviation for |\path[use as bounding box]|.
\end{command}

\begin{command}{\node}
  Inside |{tikzpicture}| this is an abbreviation for |\path node|. Note
  that, for once, |node| is not an option but a path operation.
\end{command}

\begin{command}{\coordinate}
  Inside |{tikzpicture}| this is an abbreviation for |\path coordinate|.
\end{command}

\subsection{Specifying Colors}

The most unspecific option for setting colors is the following:

\begin{itemize}
  \itemoption{color}|=|\meta{color name}
  This option sets the color that is used for fill, drawing, and text
  inside the current scope. Any special settings for filling colors or
  drawing colors are immediately ``overruled'' by this option.

  The \meta{color name} is the name of a previously defined color. For
  \LaTeX\ users, this is just a normla ``\LaTeX-color'' and the
  |xcolor| extensions are allows. Here is an example:

\begin{codeexample}[]
\tikz \fill[color=red!20] (0,0) circle (1ex);
\end{codeexample}

  It is possible to ``leave out'' the |color=| part if you load the
  |xkeyval| package. Thus, if this package is loaded, you can also write
\begin{codeexample}[]
\tikz \fill[red!20] (0,0) circle (1ex);
\end{codeexample}
  What happens is the every option that \tikzname\ does not know, like
  |red!20| gets a ``second chance'' as a color name.

  For plain \TeX\ users, it is not so easy to specify colors since
  plain \TeX\ has no ``standarderized'' color naming
  mechanism. Because of this, \pgfname\ emulates the |xcolor| package,
  though the emulation is \emph{extremely basic} (more precisely, what
  I could hack together in two hours or so). The emulation allows you
  to do the following:
  \begin{itemize}
  \item Specify a new color using |\definecolor|. Only the two color
    models |gray| and |rgb| are supported.
    \example |\definecolor{orange}{rgb}{1,0.5,0}|
  \item Use |\colorlet| to define a new color based on an old
    one. Here, the |!| mechanism is supported, though only ``once''
    (use multiple |\colorlet| for more fancy colors).
    \example |\colorlet{lightgray}{black!25}|
  \item Use |\color|\marg{color name} to set the color in the current
    \TeX\ group. |\aftergroup|-hackery is used to restore the color
    after the group.
  \end{itemize}
\end{itemize}

As pointed out above, the |color=| option applies to ``everything''
(except to shadings), which is not always what you want. Because of
this, there are several more specialised color options. For example,
the |draw=| option sets the color used for drawing, but does not
modify the color used for filling. These color options are documented
where the path action they influence is desribed.


\subsection{Drawing a Path}

You can draw a path using the following option:
\begin{itemize}
  \itemoption{draw}\opt{|=|\meta{color}}
  Causes the path to be drawn. ``Drawing'' (also known as
  ``stroking'') can be thought of as picking up a pen and moving it
  along the path, thereby leaving ``ink'' on the canvas.

  There are numerous parameters that influence how a line is drawn,
  like the thickness or the dash pattern. These options are explained
  below.

  If the optional \meta{color} argument is given, drawing is done
  using the given \meta{color}. This color can be different from the
  current filling color, which allows you to draw and fill a path with
  different colors. If no \meta{color} argument is given, the last
  usage of the |color=| option is used.

  Although this option is normally used on paths to indicate that the
  path should be drawn, it also makes sense to use the option with a
  |{scope}| or |{tikzpicture}| environment. However, this will
  \emph{not} cause all path to drawn. Instead, this just sets the
  \meta{color} to be used for drawing paths inside the environment.

\begin{codeexample}[]
\begin{tikzpicture}
  \path[draw=red] (0,0) -- (1,1) -- (2,1) circle (10pt);
\end{tikzpicture}
\end{codeexample}
\end{itemize}

The following subsections list the different options that influence
how a path is drawn. All of these options only have an effect if the
|draw| options is given (directly or indirectly).

\subsubsection{Graphic Parameters: Line Width, Line Cap, and Line Join}

\label{section-cap-joins}

\begin{itemize}
  \itemoption{line width}|=|\meta{dimension}
  Specifies the line width. Note the space. Default: |0.4pt|.

\begin{codeexample}[]
  \tikz \draw[line width=5pt] (0,0) -- (1cm,1.5ex);
\end{codeexample}
\end{itemize}

There are a number of predefined styles that provide more ``natural''
ways of setting the line width. You can also redefine these
styles. Remember that you can leave out the |style=| when setting a
style.

\begin{itemize}
  \itemstyle{ultra thin}
  Sets the line width to 0.1pt.
\begin{codeexample}[]
  \tikz \draw[ultra thin] (0,0) -- (1cm,1.5ex);
\end{codeexample}

  \itemstyle{very thin}
  Sets the line width to 0.2pt.
\begin{codeexample}[]
  \tikz \draw[very thin] (0,0) -- (1cm,1.5ex);
\end{codeexample}

  \itemstyle{thin}
  Sets the line width to 0.4pt.
\begin{codeexample}[]
  \tikz \draw[thin] (0,0) -- (1cm,1.5ex);
\end{codeexample}

  \itemstyle{semithick}
  Sets the line width to 0.6pt.
\begin{codeexample}[]
  \tikz \draw[semithick] (0,0) -- (1cm,1.5ex);
\end{codeexample}

  \itemstyle{thick}
  Sets the line width to 0.8pt.
\begin{codeexample}[]
  \tikz \draw[thick] (0,0) -- (1cm,1.5ex);
\end{codeexample}

  \itemstyle{very thick}
  Sets the line width to 1.2pt.
\begin{codeexample}[]
  \tikz \draw[very thick] (0,0) -- (1cm,1.5ex);
\end{codeexample}

  \itemstyle{ultra thick}
  Sets the line width to 1.6pt.
\begin{codeexample}[]
  \tikz \draw[ultra thick] (0,0) -- (1cm,1.5ex);
\end{codeexample}
\end{itemize}

\begin{itemize}
  \itemoption{cap}|=|\meta{type}
  Specifies how lines ``end.'' Permissble \meta{type} are |round|,
  |rect|, and |butt| (default). They have the following effects:

\begin{codeexample}[]
\begin{tikzpicture}
  \begin{scope}[line width=10pt]
    \draw[cap=rect] (0,0 ) -- (1,0);
    \draw[cap=butt] (0,.5) -- (1,.5);
    \draw[cap=round] (0,1 ) -- (1,1);
  \end{scope}
  \draw[white,line width=1pt]
    (0,0 ) -- (1,0) (0,.5) -- (1,.5) (0,1 ) -- (1,1);
\end{tikzpicture}
\end{codeexample}

  \itemoption{join}|=|\meta{type}
  Specifies how lines ``join.'' Permissble \meta{type} are |round|,
  |bevel|, and |miter| (default). They have the following effects:

\begin{codeexample}[]
\begin{tikzpicture}[line width=10pt]
  \draw[join=round] (0,0) -- ++(.5,1) -- ++(.5,-1);
  \draw[join=bevel] (1.25,0) -- ++(.5,1) -- ++(.5,-1); 
  \draw[join=miter] (2.5,0) -- ++(.5,1) -- ++(.5,-1); 
\end{tikzpicture}
\end{codeexample}

  \itemoption{miter limit}|=|\meta{factor}
  When you use the miter join and there is a very sharp corner (a
  small angle), the miter join may protrude very far over the actual
  joining point. In this case, if it were to protrude by 
  more than \meta{factor} times the line width, the miter join is
  replaced by a bevel join. Deault value is |10|.

\begin{codeexample}[]
\begin{tikzpicture}[line width=5pt]
  \draw                 (0,0) -- ++(5,.5) -- ++(-5,.5);
  \draw[miter limit=25] (6,0) -- ++(5,.5) -- ++(-5,.5);
\end{tikzpicture}
\end{codeexample}
\end{itemize}

\subsubsection{Graphic Parameters: Dash Pattern}

\begin{itemize}
  \itemoption{dash pattern}|=|\meta{dash pattern}
  Sets the dashing pattern. The syntax is the same as in
  \textsc{metafont}. For example |on 2pt off 3pt on 4pt off 4pt| means ``draw
  2pt, then leave out 3pt, then draw 4pt once more, then leave out 4pt
  again, repeat''. 

\begin{codeexample}[]
\begin{tikzpicture}[dash pattern=on 2pt off 3pt on 4pt off 4pt]
  \draw (0pt,0pt) -- (3.5cm,0pt);
\end{tikzpicture}
\end{codeexample}

  \itemoption{dash phase}|=|\meta{dash phase}
  Shifts the start of the dash pattern by \meta{phase}.

\begin{codeexample}[]
\begin{tikzpicture}[dash pattern=on 20pt off 10pt]
  \draw[dash phase=0pt] (0pt,3pt) -- (3.5cm,3pt);
  \draw[dash phase=10pt] (0pt,0pt) -- (3.5cm,0pt);
\end{tikzpicture}
\end{codeexample}
\end{itemize}

As for the line thickness, some predefined styles allow you to set the
dashing conveniently.

\begin{itemize}
\itemstyle{solid}
  Shorthand for setting a solid line as ``dash pattern.'' This is the default.

\begin{codeexample}[]
\tikz \draw[solid] (0pt,0pt) -- (50pt,0pt);
\end{codeexample}

  \itemstyle{dotted}
  Shorthand for setting a dotted dash pattern.

\begin{codeexample}[]
\tikz \draw[dotted] (0pt,0pt) -- (50pt,0pt);
\end{codeexample}

  \itemstyle{densely dotted}
  Shorthand for setting a densely dotted dash pattern.

\begin{codeexample}[]
\tikz \draw[densely dotted] (0pt,0pt) -- (50pt,0pt);
\end{codeexample}

  \itemstyle{loosely dotted}
  Shorthand for setting a loosely dotted dash pattern.

\begin{codeexample}[]
\tikz \draw[loosely dotted] (0pt,0pt) -- (50pt,0pt);
\end{codeexample}

  \itemstyle{dashed}
  Shorthand for setting a dashed dash pattern.

\begin{codeexample}[]
\tikz \draw[dashed] (0pt,0pt) -- (50pt,0pt);
\end{codeexample}

  \itemstyle{densely dashed}
  Shorthand for setting a densely dashed dash pattern.

\begin{codeexample}[]
\tikz \draw[densely dashed] (0pt,0pt) -- (50pt,0pt);
\end{codeexample}

  \itemstyle{loosely dashed}
  Shorthand for setting a loosely dashed dash pattern.

\begin{codeexample}[]
\tikz \draw[loosely dashed] (0pt,0pt) -- (50pt,0pt);
\end{codeexample}
\end{itemize}




\subsubsection{Graphic Parameters: Arrow Tips}

When you draw a line, you can add arrows at the ends. Currently, it is
only possible to add one arrow at the start and one at the end. Thus,
even if the path consists of several segments, only the first and last
segments get arrows. In general, it is a good idea to add arrows only
to paths that consist of a single, unbroken line. The behaviour for
paths that consist of several segments is not specified and may change
in the future.

\begin{itemize}
\itemoption{arrows}\opt{|=|\meta{start arrow kind}|-|\meta{end arrow kind}}
  This option sets the start and end arrows (an empty value as in |->|
  indicates that no arrow should be drawn at the start).

  \emph{Note: Since the arrow option is so often used, you can leave
    out the text |arrows=|.} What happens is that every option that
  contains a |-| is interpreted as an arrow specification.

\begin{codeexample}[]
\begin{tikzpicture}
  \draw[->]        (0,0)   -- (1,0);
  \draw[o-stealth] (0,0.1) -- (1,0.1);
\end{tikzpicture}
\end{codeexample}

  The permissible values are all defined arrows, though
  you can also define new arrow kinds as explained in
  Section~\ref{section-arrows}. This is often necessary to obtain
  ``double'' arrows and arrows that have a fixed size. Since
  |pgflibraryarrows| is loaded by default, all arrows described in
  Section~\ref{section-library-arrows} are available.

  One arrow kind is special: |>| (and all arrow kinds containing the
  arrow king such as |<<| or \verb!>|!). This arrow type is not  
  fixed. Rather, you can redefine it using the |>=| option, see
  below. 

  \example You can also combine arrow types as in
\begin{codeexample}[]
\begin{tikzpicture}[thick]
  \draw[to reversed-to]   (0,0) .. controls +(.5,0) and +(-.5,-.5) .. +(1.5,1);
  \draw[[-latex reversed] (1,0) .. controls +(.5,0) and +(-.5,-.5) .. +(1.5,1);
  \draw[latex-)]          (2,0) .. controls +(.5,0) and +(-.5,-.5) .. +(1.5,1);
\end{tikzpicture}
\end{codeexample}

  \itemoption{>}|=|\meta{end arrow kind}
  This option can be used to redefine the ``standard'' arrow |>|. The
  idea is that different people have different ideas what arrow kind
  should normally be used. I prefer the arrow of \TeX's |\to| command
  (which is used in things like $f\colon A \to B$). Other people will
  prefer \LaTeX's standard arrow, which looks like this: \tikz
  \draw[-latex] (0,0) -- (10pt,1ex);. Since the arrow kind |>| is
  certainly the most ``natural'' one to use, it is kept free of any
  predefined meaning. Instead, you can change it by saying |>=to| to
  set the ``standard'' arrow kind to \TeX's arrow, whereas |>=latex|
  will set it to \LaTeX's arrow and |>=stealth| will use a
  \textsc{pstricks}-like arrow.

  Apart from redefining the arrow kind |>| (and |<| for the start),
  this option also redefines the following arrow kinds: |>| and |<| as
  the swapped version of \meta{end arrow kind}, |<<| and |>>| as
  doubled versions, |>>| and |<<| as swapped doubled versions, %>>
  and \verb!|<! and \verb!>|! as arrows ending with a vertical bar.

\begin{codeexample}[]
\begin{tikzpicture}
  \begin{scope}[>=latex]
    \draw[->]    (0pt,6ex) -- (1cm,6ex);
    \draw[>->>]  (0pt,5ex) -- (1cm,5ex);
    \draw[|<->|] (0pt,4ex) -- (1cm,4ex);
  \end{scope}
  \begin{scope}[>=diamond]
    \draw[->]    (0pt,2ex) -- (1cm,2ex);
    \draw[>->>]  (0pt,1ex) -- (1cm,1ex);
    \draw[|<->|] (0pt,0ex) -- (1cm,0ex);
  \end{scope} 
\end{tikzpicture}
\end{codeexample}

  \itemoption{shorten >}|=|\meta{dimension}
  This option will shorten the end of lines by the given
  \meta{dimension}. If you specify an arrow, lines are already
  shortened a bit such that the arrow touches the specified endpoint
  and does not ``protrude over'' this point. Here is an example:

\begin{codeexample}[]
\begin{tikzpicture}[line width=20pt]
  \clip (0,0) rectangle (3.5,2);
  \draw[red]     (0,1) -- (3,1);
  \draw[gray,->] (0,1) -- (3,1);
\end{tikzpicture}
\end{codeexample}

  The |shorten >| option allows you to shorten the end on the line
  \emph{additionally} by the given distance. This option can also be
  useful if you have not specified an arrow at all.

\begin{codeexample}[]
\begin{tikzpicture}[line width=20pt]
  \clip (0,0) rectangle (3.5,2);
  \draw[red]                    (0,1) -- (3,1);
  \draw[-to,shorten >=10pt,gray] (0,1) -- (3,1);
\end{tikzpicture}
\end{codeexample}

  \itemoption{shorten <}|=|\meta{dimension} works like |shorten >|.
\end{itemize}



\subsubsection{Graphic Parameters: Double Lines and Bordered Lines}

\begin{itemize}
  \itemoption{double}\opt{|=|\meta{core color}}
  This option causes ``two'' lines to be drawn instead of a single
  one. However, this is not what really happens. In reality, the path
  is drawn twice. First, with the normal drawing color, secondly with
  the \meta{core color}, which is normally |white|. Upon the second
  drawing, the line width is reduced. The net effect is that it
  appears as if two lines had been drawn and this works well even with
  complicated, curved paths:

\begin{codeexample}[]
\tikz \draw[double]
  plot[smooth cycle] coordinates{(0,0) (1,1) (1,0) (0,1)};
\end{codeexample}

  You can also use the doubling option to create an effect in which a
  line seems to have a certain ``border'':

\begin{codeexample}[]
\begin{tikzpicture}
  \draw (0,0) -- (1,1);
  \draw[draw=white,double=red,very thick] (0,1) -- (1,0);
\end{tikzpicture}
\end{codeexample}

  \itemoption{double distance}|=|\meta{dimension}
  Sets the distance the ``two'' are spaced apart (default is
  0.6pt). In reality, this is the thickness of the line that is used
  to draw the path for the second time. The thickness of the
  \emph{first} time the path is drawn is twice the normal line width
  plus the given \meta{dimension}. As a side-effect, this option
  ``selects'' the |double| option.

\begin{codeexample}[]
\begin{tikzpicture}
  \draw[very thick,double]              (0,0) arc (180:90:1cm);
  \draw[very thick,double distance=2pt] (1,0) arc (180:90:1cm);
  \draw[thin,double distance=2pt]       (2,0) arc (180:90:1cm);
\end{tikzpicture}
\end{codeexample}
\end{itemize}


  




\subsection{Filling a Path}
\label{section-rules}
To fill a path, you use the following option:
\begin{itemize}
  \itemoption{fill}\opt{|=|\meta{color}}
  This option causes the path to be filled. All unclosed parts of the
  path are first closed, if necessary. Then, the area enclosed by the
  path is filled with the current filling color, which is either the
  last color set using the general |color=| option or the optional
  color \meta{color}. For self-intersection paths and for paths
  consisting of several closed areas, the ``enclosed area'' is
  somewhat complicated to define and two different definitions exist,
  namely the nonzero winding number rule and the even odd rule, see
  the explanation of these options, below.

\begin{codeexample}[]
\begin{tikzpicture}
  \fill (0,0) -- (1,1) -- (2,1);
  \fill (4,0) circle (.5cm)  (4.5,0) circle (.5cm);
  \fill[even odd rule] (6,0) circle (.5cm)  (6.5,0) circle (.5cm);
  \fill (8,0) -- (9,1) -- (10,0) circle (.5cm);
\end{tikzpicture}
\end{codeexample}

  If the |fill| option is used together with the |draw| option (either
  because both are given as options or because a |\filldraw| command
  is used), the command is draw \emph{firstly}, then the path is filled
  \emph{secondly}. This is especially useful if different colors are
  selected for drawing and for filling. Even if the same color is
  used, there is a difference between this command and a plain 
  |fill|: A ``filldrawn'' area will be slightly larger than a filled
  area because of the thickness of the ``pen.''

\begin{codeexample}[]
\begin{tikzpicture}[fill=yellow,line width=5pt]
  \filldraw (0,0) -- (1,1) -- (2,1);
  \filldraw (4,0) circle (.5cm)  (4.5,0) circle (.5cm);
  \filldraw[even odd rule] (6,0) circle (.5cm)  (6.5,0) circle (.5cm);
  \filldraw (8,0) -- (9,1) -- (10,0) circle (.5cm);
\end{tikzpicture}
\end{codeexample}
\end{itemize}

The following two options can be used to decide on which filling rule
should be used:
\begin{itemize}
  \itemoption{nonzero rule}
  If this rule is used (which is the default), the following method is
  used to determine whether a given point is ``inside'' the path: From
  the point, shoot a ray in some direction towards infinity (the
  direction is chosen such that no strange borderline cases
  occur). Then the ray may hit the path. Whenever it hits the path, we
  increase or decrease a counter, which is initially zero. If the ray
  hits the path as the path goes ``from left to right'' (relative to
  the ray), the counter is increased, otherwise it is decreased. Then,
  at the end, we check whether the counter is nonzero (hence the
  name). If so, the point is deemed to lie ``inside,'' otherwise it is
  ``outside.'' Sounds complicated? It is.

\begin{codeexample}[]
\begin{tikzpicture}
  \filldraw[fill=yellow]
  % Clockwise rectangle
  (0,0) -- (0,1) -- (1,1) -- (1,0) -- cycle
  % Counter-clockwise rectangle
  (0.25,0.25) -- (0.75,0.25) -- (0.75,0.75) -- (0.25,0.75) -- cycle;

  \draw[->] (0,1) (.4,1);
  \draw[->] (0.75,0.75) (0.3,.75);

  \draw[->] (0.5,0.5) -- +(0,1) node[above] {crossings: $-1+1 = 0$};

  \begin{scope}[yshift=-3cm]
    \filldraw[fill=yellow]
    % Clockwise rectangle
    (0,0) -- (0,1) -- (1,1) -- (1,0) -- cycle
    % Clockwise rectangle
    (0.25,0.25) -- (0.25,0.75) -- (0.75,0.75) -- (0.75,0.25) -- cycle;

    \draw[->] (0,1) (.4,1);
    \draw[->] (0.25,0.75) (0.4,.75);
      
    \draw[->] (0.5,0.5) -- +(0,1) node[above] {crossings: $1+1 = 2$};
  \end{scope}
\end{tikzpicture}
\end{codeexample}

\itemoption{even odd rule}
  This option causes a different method to be used for determining the
  inside and outside of paths. Will it is less flexible, it turns out
  to be more intuitive.

  With this method, we also shoot rays from the point for which we
  wish to determine wheter it is inside or outside the filling
  area. However, this time we only count how often we ``hit'' the path
  and declare the point to be ``inside'' if the number of hits is odd.

  Usin the even-odd rule, it is easy to ``drill holes'' into a path.
  
\begin{codeexample}[]
\begin{tikzpicture}
  \filldraw[fill=yellow,even odd rule]
    (0,0) rectangle (1,1) (0.5,0.5) circle (0.4cm);
  \draw[->] (0.5,0.5) -- +(0,1) [above] node{crossings: $1+1 = 2$};
\end{tikzpicture}
\end{codeexample}
\end{itemize}




\subsection{Shading a Path}

You can shade a path using the |shade| option. A shading is like a
filling, only the shading changes its color smoothly from one color to
another.

\begin{itemize}
  \itemoption{shade}
  Causes the path to be shaded using the currently selected shading
  (more on this later). If this option is used together with the
  |draw| option, then the path is first shaded, then drawn.

  It is not an error to use this option together with the |fill|
  option, but it makes no sense.

\begin{codeexample}[]
\tikz \shade (0,0) circle (1ex);
\end{codeexample}

\begin{codeexample}[]
\tikz \shadedraw (0,0) circle (1ex);
\end{codeexample}
\end{itemize}

For some shadings it is not really clear how they can ``fill'' the
path. For example, the |ball| shading normally looks like this: \tikz
\shade[shading=ball] (0,0) circle (0.75ex);. How is this supposed to
fill a rectangle? Or a triangle?

To solve this problem, the predefined shadings like |ball| or |axis|
fill a large rectangle completely in a sensible way. Then, when the
shading is used to ``fill'' a path, what actually happens is that the
path is temporarily used for clipping and then the rectangular shading
is drawn, scaled and shifted such that all parts of the path are
filled.


\subsubsection{Choosing a Shading Type}

As can be seen, the default shading is a smooth transition from gray
to white and from above to bottom. However, other shadings are also
possible, for example a shading that will sweep a color from the
center to the corners outward. To choose the shading, you can use the
|shading=| option which will also automatically invoke the |shade|
option. Note that this does \emph{not} change the shading color, only
the way the colors sweep. For changing the colors, other options are
needed, which are explained below.

\begin{itemize}
  \itemoption{shading}|=|\meta{name}
  This selects a shading named \meta{name}. The following shadings are
  predefined:
  \begin{itemize}
  \item \declare{|axis|}
    This is the default shading in which the color changes gradually
    between two three horizontal lines. The top line is at the top
    (uppermost) point of the path, the middle is in the middle, the
    bottom line is at the bottom of the path.

\begin{codeexample}[]
\tikz \shadedraw [shading=axis] (0,0) rectangle (1,1);
\end{codeexample}

    The default top color is gray, the default bottom color is white,
    the default middle is the ``middle'' of these two.
  \item \declare{|radial|}
    This shading fills the path with a gradual sweep from a certain
    color in the middle to another color at the border. If the path is
    a circle, the outer color will be reached exactly at the
    border. If the shading is not a circle, the outer color will
    continue a bit towards the corners. The default inner color is
    gray, the deafult outer color is white.

\begin{codeexample}[]
\tikz \shadedraw [shading=radial] (0,0) rectangle (1,1);
\end{codeexample}
  \item \declare{|ball|}
    This shading fills the path with a shading that ``looks like a
    ball.'' The default ``color'' of the ball is blue (for no
    particular reason).

\begin{codeexample}[]
\tikz \shadedraw [shading=ball] (0,0) rectangle (1,1);
\end{codeexample}

\begin{codeexample}[]
\tikz \shadedraw [shading=ball] (0,0) circle (.5cm);
\end{codeexample}
  \end{itemize}
  \itemoption{shading angle}|=|\meta{degrees}
  This option rotates the shading (not the path!) by the given
  angle. For example, we can turn a top-to-bottom axis shading into a
  left-to-right shading by rotating it by $90^\circ$.

\begin{codeexample}[]
\tikz \shadedraw [shading=axis,shading angle=90] (0,0) rectangle (1,1);
\end{codeexample}
\end{itemize}


You can also define new shading types yourself. However, for this, you
need to use the basic layer directly, which is, well, more basic and
harder to use. Details on how to create a shading appropriate for
filling paths are given in Section~\ref{section-shading-a-path}.



\subsubsection{Choosing a Shading Color}

The following options can be used to change the colors used for
shadings. When one of these options is given, the |shade| option is
automatically selected and also the ``right'' shading.

\begin{itemize}
  \itemoption{top color}|=|\meta{color}
  This option prescribes the color to be used at the top in an |axis|
  shading. When this option is given, several things happen:
  \begin{enumerate}
  \item
    The |shade| option is selected.
  \item
    The |shading=axis| option is selected.
  \item
    The middle color of the axis shading is set to the average of the
    given top color \meta{color} and of whatever color is currently
    selected for the bottom.
  \item
    The rotation angle of the shading is set to 0.
  \end{enumerate}

\begin{codeexample}[]
\tikz \draw[top color=red] (0,0) rectangle (2,1);
\end{codeexample}
  
  \itemoption{bottom color}|=|\meta{color}
  This option works like |top color|, only for the bottom color.
  
  \itemoption{middle color}|=|\meta{color}
  This option specifies the color for the middle of an axis
  shading. It also sets the |shade| and |shading=axis| options, but it
  does not change the rotation angle.

  \emph{Note:} Since both |top color| and |bottom color| change the
  middle color, this option should be given \emph{last} if all of
  these options need to be given:

\begin{codeexample}[]
\tikz \draw[top color=white,bottom color=black,middle color=red]
  (0,0) rectangle (2,1);
\end{codeexample}  

  \itemoption{left color}|=|\meta{color}
  This option does exactly the same as |top color|, except that the
  shading angle is set to $90^\circ$.

  \itemoption{right color}|=|\meta{color}
  Works like |left color|.

  \itemoption{inner color}|=|\meta{color}
  This option sets the color used at the center of a |radial|
  shading. When this option is used, the |shade| and |shading=radial|
  options are set.
  
\begin{codeexample}[]
\tikz \draw[inner color=red] (0,0) rectangle (2,1);
\end{codeexample}

  \itemoption{outer color}|=|\meta{color}
  This option sets the color used at the border and outside of a
  |radial| shading.
  
\begin{codeexample}[]
\tikz \draw[outer color=red,inner color=white]
  (0,0) rectangle (2,1);
\end{codeexample}

  \itemoption{ball color}|=|\meta{color}
  This option sets the color used for the ball shading. It sets the
  |shade| and |shading=ball| options. Note that the ball will never
  ``completely'' have the color \meta{color}. At its ``highlite'' spot
  a certain amount of white is mixed in, at the border a certain
  amount of black. Because of this, it also makes sense to say
  |ball color=white| or |ball color=black|

\begin{codeexample}[]
\begin{tikzpicture}
  \shade[ball color=white] (0,0) circle (2ex);
  \shade[ball color=red] (1,0) circle (2ex);
  \shade[ball color=black] (2,0) circle (2ex);
\end{tikzpicture}
\end{codeexample}
\end{itemize}




\subsection{Establishing a Bounding Box}

\pgfname\ is quite good at keeping track of the size of your picture
and reserving just the right amount of space for it in the main
document. However, in some cases you may want to say things like
``do not count this for the picture size'' or ``the picture is
actually a little large.'' For this, you can use the option
|use as bouding box| or the command |\useasboundingbox|, which is just
a shorthand for |\path[use as bouding box]|.

\begin{itemize}
  \itemoption{use as bounding box}
  Normally, when this option is given on a path, the bounding box of
  the present path is used to determine the size of the picture and
  the size of all \emph{subsequent} path commands are
  ignored. However, if there were previous path commands that have
  already established a larger bounding box, it will not be made
  smaller by this command.

  In a sense, |use as bounding box| has the same effect as clipping
  all subsequent drawing against the current path---without actually
  doing the clipping, only making \pgfname\ treat everything as if it
  were clipped.

  The first application of this command is to have a |{tikzpicture}|
  overlap with the main text:

\begin{codeexample}[]
Left of picture\begin{tikzpicture}
  \draw[use as bounding box] (2,0) rectangle (3,1);
  \draw (1,.25) -- (4,.75);
\end{tikzpicture}right of picture.
\end{codeexample}

  In a second application, this command can be used to get better
  control over the white space around the picture:
  
\begin{codeexample}[]
Left of picture
\begin{tikzpicture}
  \useasboundingbox (0,0) rectangle (3,1);
  \fill (.75,.25) circle (.5cm);
\end{tikzpicture}
right of picture.
\end{codeexample}

  Note: If this option is used on a path inside a \TeX\ group (scope),
  the effect ``lasts'' only till the end of the scope.
\end{itemize}



\subsection{Using a Path For Clipping}

To use a path for clipping, use the |clip| option. 

\begin{itemize}
  \itemoption{clip}
  This option causes all subsequent drawings to be clipped against the
  current path and the size of subsequent paths will not be important
  for the picture size.  If you clip against a self-intersecting path,
  the even-odd rule or  the nonzero winding number rule is used to
  determine whether a point is inside or outside the clipping region.

  The clipping path is a normal graphic state
  parameter, so it will be reset at the end of the current
  scope. Multiple clippings accumulate, that is, clipping is always
  done against the intersection of all clipping areas that have been
  specified inside the current scopes. The only way of enlarging the
  clipping area is to end a |{scope}|.

\begin{codeexample}[]
\begin{tikzpicture}
  \draw[clip] (0,0) circle (1cm);
  \fill[red] (1,0) circle (1cm);
\end{tikzpicture}
\end{codeexample}

  It  is usually a \emph{very} good idea to apply the |clip| option only
  to the first path command in a scope. 

  If you ``only wish to clip'' and do not wish to draw anything, you can
  use the |clip| option together with the |\path| command or, which
  might be clearer, with the |\useasboundingbox| command. The effect is the
  same.

\begin{codeexample}[]
\begin{tikzpicture}
  \useasboundingbox[clip] (0,0) circle (1cm);
  \fill[red] (1,0) circle (1cm);
\end{tikzpicture}
\end{codeexample}

  To keep clipping local, use |{scope}| environments as in the
  following example:

\begin{codeexample}[]
\begin{tikzpicture}
  \draw (0,0) -- ( 0:1cm);
  \draw (0,0) -- (10:1cm);
  \draw (0,0) -- (20:1cm);
  \draw (0,0) -- (30:1cm);
  \begin{scope}[fill=red]
    \fill[clip] (0.2,0.2) rectangle (0.5,0.5);
    
    \draw (0,0) -- (40:1cm);
    \draw (0,0) -- (50:1cm);
    \draw (0,0) -- (60:1cm);
  \end{scope}
  \draw (0,0) -- (70:1cm);
  \draw (0,0) -- (80:1cm);
  \draw (0,0) -- (90:1cm);
\end{tikzpicture}
\end{codeexample}

  There is a slightly annoying catch: You cannot specify certain graphic
  options for the command used for clipping. For example, in the above
  code we could not have moved the |fill=red| to the |\fill|
  command. The reasons for this have to do with the internals of the
  \pdf\ specification. You do not want to know the details,
  believe me. It is best simply not to specify any options for these
  commands. 
\end{itemize}

% Copyright 2003 by Till Tantau <tantau@cs.tu-berlin.de>.
%
% This program can be redistributed and/or modified under the terms
% of the LaTeX Project Public License Distributed from CTAN
% archives in directory macros/latex/base/lppl.txt.


\section{Nodes}

\label{section-nodes}

\subsection{Nodes and Their Shapes}

\tikzname\ offers an easy way of adding so-called \emph{nodes} to your
pictures. In the simplest case, a node is just some text that is
placed at some coordinate. However, a node can also have a border
drawn around it or have a more complex background. Indeed, some nodes
do not have a text at all, but solely consist of the background. You
can name nodes so that you can reference their coordinates later in
the picture. However, \emph{nodes cannot be referenced across
  different pictures}.

There are no special \TeX\ commands for adding a node to a picture; rather,
there is path operation called |node| for this. Nodes are created
whenever \tikzname\ encounters |node| or |coordinate| at any point on a
path where it would expect a normal path command (like |-- (1,1)| or
|sin (1,1)|). 

The node operation is typically followed by some options, which apply
only to the node. Then, you can optionally \emph{name} the node by
providing a name in round braces. Lastly, for the |node| operation you
must provide some label text for the node in curly braces, while for
the |coordiante| operation you may not. The node is placed at the
current position of the path \emph{after the path has been
  drawn}. Thus, all nodes are drawn ``on top'' of the path and
retained until the path is complete. If there are several nodes on a
path, they are drawn on top of the path in the order they are
encountered. 

\begin{codeexample}[]
\tikz \fill[fill=yellow]
     (0,0) node {first node}
  -- (1,1) node {second node}
  -- (0,2) node {third node};
\end{codeexample}

There are two possible syntax for specifying nodes:
\begin{itemize}
\item
  \declare{|node|\opt{|[|\meta{options}|]|}\opt{|(|\meta{name}|)|}%
    \opt{|at(|\meta{coordinate}|)|}\opt{\marg{text}}}
  As with normal paths, you can give multiple options in multiple
  brackets. Also, you can change the ordering of \meta{name} and
  \meta{options}. For example, the following is a legal node
  specification: |node[red](A)[draw]|.

  The effect of |at| is to place the node at the coordinate given
  after |at| and not, as would normally be the case, at the last
  position. The |at| syntax is not available when a node is given
  inside a path operation (it would not make any sense, there).
\item \declare{|coordinate|\opt{|[|\meta{options}|]|}|(|\meta{name}|)|\opt{|at(|\meta{coordinate}|)|}} 
  This has the same effect as

  |node[shape=coordinate][|\meta{options}|](|\meta{name}|)at(|\meta{coordinate}|){}|,
  
  where the |at| part might be missing. Note that you \emph{must} give
  the options first.
\end{itemize}

A node specification must be given on a path at some point where
\tikzname\ expects a path command. Node specification can also be given 
\emph{inside} certain path commands; this is explained later.

The |(|\meta{name}|)| is a name for later reference and it is
optional. You also add the option |name=|\meta{name} to the
\meta{option} list; it has the same effect.

\begin{itemize}
  \itemoption{name}=\meta{node name}
  assigns a name to the node for later reference. Since this is a
  ``high-level'' name (drivers never know of it), you can use spaces,
  number, letters, or whatever you like when naming a node. Thus, you
  can name a node just |1| or perhaps |start of chart| or even
  |y_1|. Your node name should \emph{not} contain an punctuation like
  a dot, a comma or a colon since these are used to detect what kind
  of coordinate you mean when you reference a node. 
\end{itemize}

The \meta{options} is an optional list of options that \emph{apply
  only to the node} and have no effect outside. The other way round,
most ``outside'' options also apply to the node, but not all. For
example, the ``outside'' rotation does not apply to nodes (unless some
special options are used). Also, the outside path action, like |draw| or
|fill|, never applies to the node and must be given in the node
(unless some special other options are used).

As mentioned before, we can add a border and even a background to a
node:  

\begin{codeexample}[]
\tikz \fill[fill=yellow]
      (0,0) node {first node}
   -- (1,1) node[draw] {second node}
   -- (0,2) node[fill=red!20,draw,double,rounded corners] {third node};
\end{codeexample}

The ``border'' is actually just a special case of a much more general
mechanism. Each node has a certain \emph{shape} which, by default, is
a rectangle. However, we can also ask \tikzname\ to use circle shape
instead or an ellipse shape (you have to inlude |pgflibraryshapes| for
this shape): 

\begin{codeexample}[]
\tikz \fill[fill=yellow]
      (0,0) node{first node}
   -- (1,1) node[ellipse,draw] {second node}
   -- (0,2) node[circle,fill=red!20] {third node};
\end{codeexample}

In the future, there might be much more complicated shapes available
such as, say, a shape for a resistor or a shape for a state of a
finite automaton or a shape for a \textsc{uml} class. Unfortunately,
creating new shapes is a bit tricky and makes it necessary to use the
basic layer directly. Life is hard.

To select the shape of a node, the following option is used:
\begin{itemize}
  \itemoption{shape}|=|\meta{shape name}
  select the shape either of the current node or, when this option is
  not given inside a node but somewhere outside, the shape of all
  nodes in the current scope.

  Since this option is used often, you can leave out the
  |shape=|. In detail, when \tikzname\ encounters an option like |circle|
  that it does not know, it will, after everything else has failed,
  check whether this option is the name of some shape. If so, that
  shape is selected as if you had said |shape=|\meta{shape name}.

  By default, the following shapes are available: |rectangle|,
  |circle|, |coordinate|, and, when the package |pgflibraryshapes| is
  loaded, also |circle|. Details of these shapes, like their anchors
  and size options, are discussed in Section~\ref{section-the-shapes}.
\end{itemize}

  
The following styles influences how nodes are rendered:
\begin{itemize}
  \itemstyle{every node}
  This style is installed at the beginning of every node. 
\begin{codeexample}[]
\begin{tikzpicture}
  \tikzstyle{every node}=[draw] 
  \draw (0,0) node {A} -- (1,1) node {B};
\end{tikzpicture}
\end{codeexample}

  \itemstyle{every \meta{shape} node}
  These styles are installed at the beginning of a node of a given
  \meta{shape}. For example, |every rectangle node| is used for
  rectangle nodes, and so on.
\begin{codeexample}[]
\begin{tikzpicture}
  \tikzstyle{every rectangle node}=[draw] 
  \tikzstyle{every circle node}=   [draw,double] 
  \draw (0,0) node[rectangle] {A} -- (1,1) node[circle] {B};
\end{tikzpicture}
\end{codeexample}
  \end{itemize}


\subsection{Options for the Text in  Nodes}

The simplest option for the text in nodes is its color. Normally, this
color is just the last color installed using |color=|, possibly
inherited from another scope. However, it is possible to specificly
set the color used for text using the following option''

\begin{itemize}
  \itemoption{text}|=|\meta{color}
  Sets the color to be used for text inside nodes. A |color=| option
  will immediately override this option.
\begin{codeexample}[]
\begin{tikzpicture}
  \draw[red]       (0,0) -- +(1,1) node[above]     {red};
  \draw[text=red]  (1,0) -- +(1,1) node[above]     {red};
  \draw            (2,0) -- +(1,1) node[above,red] {red};
\end{tikzpicture}
\end{codeexample}
\end{itemize}

Normally, when a node is typset, all the text you give in the braces
(without the options and the node name, of course) is but in one long
line (in an |\hbox|, to be precise) and the node will become as wide
as necessary.

You can change this behaviour using the following options. They allow
you to limit the width of a node (naturally, at the expence of its
height).

\begin{itemize}
  \itemoption{text width}|=|\meta{dimension}
  This option will put the text of a node in a box of the given width
  (more precisely, in a |{minipage}| of this width; for plain \TeX\ a
  little ``minipage emulation'' is used).

  If the node text is not as wide as \meta{dimension}, it will
  nevertheless be put in a box of this width. If it is larger, line
  breaking will be done.

  By default, when this option is given, a ragged right border will be
  used. This is sensible since, typically, these boxes are narrow and
  justifying the text looks ugly.
\begin{codeexample}[]
\tikz \draw (0,0) node[fill=yellow,text width=3cm]
  {This is a demonstration text for showing how line breaking works.};  
\end{codeexample}
  \itemoption{text justified}
  causes the text to be justified instead of (right)ragged. Use this
  only with pretty broad nodes.
{%
\hbadness=10000
\begin{codeexample}[]
\tikz \draw (0,0) node[fill=yellow,text width=3cm,text justified]
  {This is a demonstration text for showing how line breaking works.};  
\end{codeexample}
}
  In the above example, \TeX\ complains (rightfully) about three very
  badly typeset lines.
  \itemoption{text ragged}
  causes the text to be typeset with a ragged right. This uses the
  original plain \TeX\ definition of a ragged right border, in which
  \TeX\ will try to balance the right border as well as possible. This
  is the default.
\begin{codeexample}[]
\tikz \draw (0,0) node[fill=yellow,text width=3cm,text ragged]
  {This is a demonstration text for showing how line breaking works.};  
\end{codeexample}
  \itemoption{text badly ragged}
  causes the right border to be ragged in the \LaTeX-style, in which
  no balancing occurs. This looks ugly, but it may be useful for very
  narrow boxes and when you wish to avoid hyphenations.
\begin{codeexample}[]
\tikz \draw (0,0) node[fill=yellow,text width=3cm,text badly ragged]
  {This is a demonstration text for showing how line breaking works.};  
\end{codeexample}
  \itemoption{text centered}
  centers the text, but tries to balance the lines.
\begin{codeexample}[]
\tikz \draw (0,0) node[fill=yellow,text width=3cm,text centered]
  {This is a demonstration text for showing how line breaking works.};  
\end{codeexample}
  \itemoption{text badly centered}
  centers the text, without balancing the lines.
\begin{codeexample}[]
\tikz \draw (0,0) node[fill=yellow,text width=3cm,text badly centered]
  {This is a demonstration text for showing how line breaking works.};  
\end{codeexample}
\end{itemize}



\subsection{Placing Nodes Using Anchors}

When you place a node at some coordinate, the node is centered on this
coordinate by default. This is often undesirable and it would be
better to have the node to the right or above the actual coordinate.

\pgfname\ uses a so-called anchoring mechanism to give you a very fine
control over the placement. The idea is simple: Imaging a node of
rectangular shape of a certain size. \pgfname\ defines numerous anchor
positions in the shape. For example to upper right corner is called,
well, not upper right anchor, but the |north east| anchor of the
shape. The center of the shape has an anchor called |center| on top of
it, and so on. Here are some examples (a complete list is given in
Section~\ref{section-the-shapes}).

\medskip\noindent
\begin{tikzpicture}
  \path node[minimum height=2cm,minimum width=5cm,fill=blue!25](x) {Big node};
  \fill (x.north)      circle (2pt) node[above] {|north|}
        (x.north east) circle (2pt) node[above] {|north east|}
        (x.north west) circle (2pt) node[above] {|north west|}
        (x.west) circle (2pt)       node[left]  {|west|}
        (x.east) circle (2pt)       node[right] {|east|}
        (x.base) circle (2pt)       node[below] {|base|};
\end{tikzpicture}

Now, when you place a node at a certain coordinate, you can ask \tikzname\
to place the node is shifted around in such a way that a certain
anchor is at the coordinate. In the following example, we ask \tikzname\
to shift the first node such that its  |north east| anchor is at
coordinate |(0,0)| and that the |west| anchor of the second node is at
coordinate |(1,1)|.

\begin{codeexample}[]
\tikz \draw           (0,0) node[anchor=north east] {first node}
           rectangle (1,1) node[anchor=west] {second node};
\end{codeexample}

Since the default anchor is |center|, the default behaviour is to
shift the node in such a way that it is centered on the current
position.

\begin{itemize}
  \itemoption{anchor}|=|\meta{anchor name}
  causes the node to be shifted such that it's anchor \meta{anchor
  name} lies on the current coordinate.

  The only anchor that is present in all shapes is |center|. However,
  most shapes will at least define anchors in all ``compass
  directions.'' Furthermore, the standard shapes also define a |base|
  anchor, as well as |base west| and |base east|, for placing things on
  the baseline of the text.
  
  The standard shapes also define a |mid| anchor (and |mid west| and
  |mid east|). This anchor is half the height of the character ``x''
  above the base line. This anchor is useful for vertically centering
  multiple nodes that have different heights and depth. Here is an
  example:
\begin{codeexample}[]
\begin{tikzpicture}[scale=3,transform shape]
  % First, center alignment -> whobbles
  \draw[anchor=center] (0,1)  node{x} -- (0.5,1)  node{y} -- (1,1)  node{t};
  % Second, base alignment -> no whobble, but too high
  \draw[anchor=base]   (0,.5) node{x} -- (0.5,.5) node{y} -- (1,.5) node{t};
  % Third, mid alignment
  \draw[anchor=mid]    (0,0)  node{x} -- (0.5,0)  node{y} -- (1,0)  node{t};
\end{tikzpicture}
\end{codeexample}
\end{itemize}

Unfortunately, while perfectly logical, it is often rather
counter-intuitive that in order to place a node \emph{above} a given
point, you need to specify the |south| anchor. For this reason, there
are some useful options that allow you to select the standard anchors
more intuitively:
\begin{itemize}
  \itemoption{above}\opt{|=|\meta{offset}}
  does the same as |anchor=south|. If the \meta{offset} is specified,
  the node is additionally shifted upwards by the given
  \meta{offset}. 
\begin{codeexample}[]
\tikz \fill (0,0) circle (2pt) node[above] {above};
\end{codeexample}
\begin{codeexample}[]
\tikz \fill (0,0) circle (2pt) node[above=2pt] {above};
\end{codeexample}
  \itemoption{above left}\opt{|=|\meta{offset}}
  does the same as |anchor=south east|. If the \meta{offset} is
  specified, the node is additionally shifted upwards and right by
  \meta{offset}. 
\begin{codeexample}[]
\tikz \fill (0,0) circle (2pt) node[above left] {above left};
\end{codeexample}
\begin{codeexample}[]
\tikz \fill (0,0) circle (2pt) node[above left=2pt] {above left};
\end{codeexample}
  \itemoption{above right}\opt{|=|\meta{offset}}
  does the same as |anchor=south west|.
\begin{codeexample}[]
\tikz \fill (0,0) circle (2pt) node[above right] {above right};
\end{codeexample}
  \itemoption{left}\opt{|=|\meta{offset}}
  does the same as |anchor=east|.
\begin{codeexample}[]
\tikz \fill (0,0) circle (2pt) node[left] {left};
\end{codeexample}
  \itemoption{right}\opt{|=|\meta{offset}}
  does the same as |anchor=west|.
  \itemoption{below}\opt{|=|\meta{offset}}
  does the same as |anchor=north|.
  \itemoption{below left}\opt{|=|\meta{offset}}
  does the same as |anchor=north east|.
  \itemoption{below right}\opt{|=|\meta{offset}}
  does the same as |anchor=north west|.
\end{itemize}


\subsection{Transformations}

It is possible to transform nodes, but, by default, transformations do
not apply to nodes. The reason is that you usually do \emph{not} want
your text to be scaled or rotated even if the main graphic is
transformed. Scaling text is evil, rotating slightly less so.

However, sometimes you \emph{do} wish to transform a node, for
example, it certainly sometimes makes sense to rotate a node by
90 degrees. There are two ways in which you can achieve this:

\begin{enumerate}
\item
  You can use the following option:
  \begin{itemize}
    \itemoption{transform shape}
    causes the current ``external'' transformation matrix to be
    applied to the shape. For example, if you said
    |\tikz[scale=3]| and then say |node[transform shape] {X}|, you
    will get a ``huge'' X in your graphic.
  \end{itemize}
\item
  You can give transformation command \emph{inside} the option list of
  the node. \emph{These} transformations allways apply to the node.
\begin{codeexample}[]
\begin{tikzpicture}
  \tikzstyle{every node}=[draw]    
  \draw[style=help lines] (0,0) grid (3,2);
  \draw            (1,0) node{A}
                   (2,0) node[rotate=90,scale=1.5] {B};
  \draw[rotate=30] (1,0) node{A}
                   (2,0) node[rotate=90,scale=1.5] {B};
  \draw[rotate=60] (1,0) node[transform shape] {A}
                   (2,0) node[transform shape,rotate=90,scale=1.5] {B};
\end{tikzpicture}
\end{codeexample}
\end{enumerate}



\subsection{Placing Nodes on a Line or Curve}

Until now, we always placed node on a coordinate that is mentioned in
the path. Often, however, we wish to place nodes on ``the middle'' of
a line and we do not wish to compute these coordinates ``by hand.''
To facilitate such placements, \tikzname\ allows you to specify that a
certain node should be somewhere ``on'' a line. There are two ways of
specifying this: Either explicitly by using the |pos| option or
implicitly by placing the node ``inside'' a path command. These two
ways are described in the following.



\subsubsection{Explicit Use of the Position Option}

\begin{itemize}
  \itemoption{pos}|=|\meta{fraction}
  When this option is given, the node is not anchored on the last
  coordinate. Rather, it is anchored on some point on the line from
  the previous coordinate to the current point. The \meta{fraction}
  dictates how ``far'' on the line the point should be. A
  \meta{fraction} or 0 is the previous coordinate, 1 is the current
  one, everything else is in between. In particular, 0.5 is the
  middle.  

  Now, what is ``the previous line''? This depends on the previous
  path construction command.

  In the simplest case, the previous path command was a ``lineto''
  command, that is, a  |--|\meta{coordinate} command:
\begin{codeexample}[]
\tikz \draw (0,0) -- (3,1)
    node[pos=0]{0} node[pos=0.5]{1/2} node[pos=0.9]{9/10};
\end{codeexample}

  The next case is the curveto command (the |..| command). In this
  case, the ``middle'' of the curve, that is, the position |0.5| is
  not necessarily the point at the exact half distance on the
  line. Rather, it is some point at ``time'' 0.5 of a point travelling
  from the start of the curve, where it is at time 0, to the end of
  the curve, which it reaches at time 0.5. The ``speed'' of the curve
  depends on the length of the support vectors (the vectors that
  connect the start and end points to the control points). The exact
  math is a bit complicated, you may wish to consult a good book on
  computer graphics and Bezi�r curves if you are intrigued.
\begin{codeexample}[]
  \tikz \draw (0,0) .. controls +(right:3.5cm) and +(right:3.5cm) .. (0,3)
    \foreach \p in {0,0.125,...,1} {node[pos=\p]{\p}};
\end{codeexample}

  Another interesting case are the horizontal/vertical lineto commands
  \verb!|-! and \verb!-|!. For them, the position (or time) |0.5| is
  exactly the corner point.

\begin{codeexample}[]
\tikz \draw (0,0) |- (3,1)
  node[pos=0]{0} node[pos=0.5]{1/2} node[pos=0.9]{9/10};
\end{codeexample}

\begin{codeexample}[]
\tikz \draw (0,0) -| (3,1)
  node[pos=0]{0} node[pos=0.5]{1/2} node[pos=0.9]{9/10};
\end{codeexample}

  For all other path construction commands, \emph{the position
  placement does not work}, currently. This will hopefulle change in
  the future (especially for the arc operation).  
  \itemoption{sloped}
  This option causes the node to be rotated such that a horizontal
  line becomes a tangent to the curve. The rotation will always be
  done in such a way that text is never ``upside down.'' If you really
  need upside down text, use |[rotate=180]|.
\begin{codeexample}[]
\tikz \draw (0,0) .. controls +(up:2cm) and +(left:2cm) .. (1,3)
    \foreach \p in {0,0.25,...,1} {node[sloped,above,pos=\p]{\p}};
\end{codeexample}
\begin{codeexample}[]
\begin{tikzpicture}[->]
  \draw (0,0)   -- (2,0.5) node[midway,sloped,above] {$x$};
  \draw (2,-.5) -- (0,0)   node[midway,sloped,below] {$y$};
\end{tikzpicture}
\end{codeexample}
\end{itemize}


There exist styles for specifying positions a bit less ``technically'':
\begin{itemize}
  \itemstyle{midway}
  is set to |pos=0.5|.
\begin{codeexample}[]
\tikz \draw (0,0) .. controls +(up:2cm) and +(left:3cm) .. (1,5)
       node[at end]          {|at end|}
       node[very near end]   {|very near end|}
       node[near end]        {|near end|}
       node[midway]          {|midway|}
       node[near start]      {|near start|}
       node[very near start] {|very near start|}
       node[at start]        {|at start|};
\end{codeexample}
  \itemstyle{near start}
  is set to |pos=0.25|.
  \itemstyle{near end}
  is set to |pos=0.75|.
  \itemstyle{very near start}
  is set to |pos=0.125|.
  \itemstyle{very near end}
  is set to |pos=0.875|.
  \itemstyle{at start}
  is set to |pos=0|.
  \itemstyle{at end}
  is set to |pos=1|.
\end{itemize}


\subsubsection{Implict Use of the Position Option}

When you wish to place a node on the line |(0,0) -- (1,1)|,
it is natural to specify the node not following the |(1,1)|, but
``somewhere in the middle.'' This is, indeed, possible and you can
write |(0,0) -- node{a} (1,1)| to place a node midway between |(0,0)| and
|(1,1)|.

What happens is the following: The syntax of the lineto path
command is actually |--|
\opt{|node|\meta{node specification}}\meta{coordinate}. (It is even
possible to give multiple nodes in this way.) When the optional
|node| is encountered, that is, 
when the |--| is directly followed by |node|, then the
specification(s) are read and ``stored away.'' Then, after the
\meta{coordinate} has finally been reached, they are inserted again,
but with the |pos| option set.

There are two things to note about this: When a node specification is
``stored,'' its catcodes become fixed. This means that you cannot use
overly complicated verbatim text in them. If you really need, say, a
verbatim text, you will have to put it in a normal node following the
coordinate and add the |pos| option.

Second, which |pos| is chosen for the node? The position is inherited
from the surrounding scope. However, this holds only for nodes
specified in this implicit way. Thus, if you add the option
|[near end]| a scope, this does not mean that \emph{all} nodes given
in this scope will be put on near the end of lines. Only the nodes
for which an implicit |pos| is added will be placed near the end. In
essence, this is what you want. Here are some examples that should
make this clearer:

\begin{codeexample}[]
\begin{tikzpicture}[near end]
  \draw (0cm,4em) -- (3cm,4em) node{A};    
  \draw (0cm,3em) --           node{B}          (3cm,3em);
  \draw (0cm,2em) --           node[midway] {C} (3cm,2em);
  \draw (0cm,1em) -- (3cm,1em) node[midway] {D} ;
\end{tikzpicture}
\end{codeexample}

Like the lineto command, the curveto command |..| also allows you to
specify nodes ``inside'' the command. After both the first |..| and
also after the second |..| you can place node specifications. Like for
the |--| command, these will be collected and then reinserted after
the command with the |pos| option set.


\subsection{Connecting Nodes}

Once you have defined a node and given it a name, you can use this
name to reference it. This can be done in two ways, see also
Section~\ref{section-node-coordinates}. Suppose you have said
|\path(0,0) node(x) {Hello World!};| in order to define a node named |x|. 
\begin{enumerate}
\item
  Once the node |x| has been defined, you can use
  |(x.|\meta{anchor}|)| whereever you would normally use a normal
  coordinate. This will yield the position at which the given
  \meta{anchor} is in the picture. Note that transformations do not
  apply to this coordinate, that is, |(x.north)| will be the northern
  anchor of |x| even if you have said |scale=3| or |xshift=4cm|. This
  is usually what you would expect.
\item
  You can also just use |(x)| as a coordinate. In most cases, this
  gives the same coordinate as |(x.center)|. Indeed, if the |shape| of
  |x| is |coordinate|, then |(x)| and |(x.center)| have exactly the
  same effect.

  However, for most other shapes, some path construction commands like
  |--| try to be ``clever'' when this they are asked to draw a line
  from such a coordinate or to such a coordinate. When you say
  |(x)--(1,1)|, the |--| path command will not draw a line from the center
  of |x|, but \emph{from the border} of |x| in the direction going
  towards |(1,1)|. Likewise, |(1,1)--(x)| will also have the line
  end on the border in the direction coming from |(1,1)|.

  In addition to |--|, the curveto path command |..| and the path
  commands \verb!-|! and \verb!|-! will also handle nodes without
  anchors correctly. Here is an example, see also
  Section~\ref{section-node-coordinates}:
\begin{codeexample}[]
\begin{tikzpicture}
  \path (0,0) node             (x) {Hello World!}
        (3,1) node[circle,draw](y) {$\int_1^2 x \mathrm d x$};

  \draw[->,blue]   (x) -- (y);
  \draw[->,red]    (x) -| node[near start,below] {label} (y);
  \draw[->,orange] (x) .. controls +(up:1cm) and +(left:1cm) .. node[above,sloped] {label} (y);
\end{tikzpicture}
\end{codeexample}
\end{enumerate}





\subsection{Predefined Shapes}
\label{section-the-shapes}

\pgfname\ and \tikzname\ define three shapes, by default:
\begin{itemize}
\item
  |rectangle|,
\item
  |circle|, and
\item
  |coordinate|.
\end{itemize}
By loading library packages, you can define more shapes. Currently,
the package |pgflibraryshapes| defines
\begin{itemize}
\item
  |ellipse|.
\end{itemize}

The exact behaviour of these shapes differs, shapes defined for more
special purposes (like a, say, transistor shape) will have even more
custom behaviours. However, there are some options that apply to most
shapes:
\begin{itemize}
  \itemoption{inner sep}|=|\meta{dimension}
  An additional (invisible) separation space of \meta{dimension} will
  be added inside the shape, between the text and the shape's
  background path. The effect is as if you had added appropriate
  horizontal and vertical skips at the beginning and end of the text
  to make it a bit ``larger.'' 

\begin{codeexample}[]
\begin{tikzpicture}
  \draw (0,0)     node[inner sep=0pt,draw]        {tight}
        (0cm,2em) node[inner sep=5pt,draw]        {loose}
        (0cm,4em) node[inner sep=2pt,fill=yellow] {default};
\end{tikzpicture}
\end{codeexample}
  \itemoption{inner xsep}|=|\meta{dimension}
  Specifies the inner separation in the $x$-direction, only.
  \itemoption{inner ysep}|=|\meta{dimension}
  Specifies the inner separation int the $y$-direction, only.
  
  \itemoption{outer sep}|=|\meta{dimension}
  This option adds an additional (invisible) separation space of
  \meta{dimension} outside the background path. The main effect of
  this option is that all anchors will move a little ``to the
  outside.'' 

  The default for this option is half the line width. When the default
  is used and when the background path is draw, the anchors will lie
  exactly on the ``outside border'' of the path (not on the path
  itself). When the shape is filled, but not drawn, this may not be
  desirable. In this case, the |outer sep| should be set to zero
  point. 
\begin{codeexample}[]
\begin{tikzpicture}
  \draw[line width=5pt]
    (0,0) node[outer sep=0pt,fill=yellow]          (f) {filled}
    (2,0) node[inner sep=.5\pgflinewidth+2pt,draw] (d) {drawn};

  \draw[->] (1,-1) -- (f);
  \draw[->] (1,-1) -- (d);  
\end{tikzpicture}
\end{codeexample}
  \itemoption{outer xsep}|=|\meta{dimension}
  Specifies the outer separation in the $x$-direction, only.
  \itemoption{outer ysep}|=|\meta{dimension}
  Specifies the outer separation int the $y$-direction, only.

  \itemoption{minimum height}|=|\meta{dimension}
  This option ensures that the height of the shape (including the
  inner, but ignoring the outer separation) will be at least
  \meta{dimension}. Thus, if the text plus the inner separation is not
  at least as large as \meta{dimension}, the shape will be enlarged 
  appropriately. However, if the text is already larger than
  \meta{dimension}, the shape will not be shrunk.
\begin{codeexample}[]
\begin{tikzpicture}
  \draw (0,0) node[minimum height=1cm,draw] {1cm}
        (2,0) node[minimum height=0cm,draw] {0cm};
\end{tikzpicture}
\end{codeexample}

  \itemoption{minimum width}|=|\meta{dimension}
  same as |minimum height|, only for the width.
\begin{codeexample}[]
\begin{tikzpicture}
  \draw (0,0) node[minimum height=2cm,minimum width=3cm,draw] {$3 \times 2$};
\end{tikzpicture}
\end{codeexample}
  \itemoption{minimum size}|=|\meta{dimension}
  sets both the minimum height and width at the same time.
\begin{codeexample}[]
\begin{tikzpicture}
  \draw (0,0)  node[minimum size=2cm,draw] {square};
  \draw (0,-2) node[minimum size=2cm,draw,circle] {circle};
\end{tikzpicture}
\end{codeexample}
\end{itemize}

\label{section-tikz-coordinate-shape}
The |coordinate| shape is handled in a special way by \tikzname. When
a node |x| whose shape is |coordinate| is used as a coordinate |(x)|,
this has the same effect as if you had said |(x.center)|. None  of the
special ``line shortening rules'' apply in this case. This can be
useful since, normally, the line shortening causes paths to be
segmented and they cannot be used for filling. Here is an example that
demonstrates the difference: 
\begin{codeexample}[]
\begin{tikzpicture}
  \tikzstyle{every node}=[draw]
  \path[yshift=1.5cm,shape=rectangle]
    (0,0) node(a1){} (1,0) node(a2){}
    (1,1) node(a3){} (0,1) node(a4){};
  \filldraw[fill=yellow] (a1) -- (a2) -- (a3) -- (a4);
  
  \path[shape=coordinate]
    (0,0) coordinate(b1) (1,0) coordinate(b2)
    (1,1) coordinate(b3) (0,1) coordinate(b4);
  \filldraw[fill=yellow] (b1) -- (b2) -- (b3) -- (b4);
\end{tikzpicture}
\end{codeexample}

% Copyright 2005 by Till Tantau <tantau@cs.tu-berlin.de>.
%
% This program can be redistributed and/or modified under the terms
% of the LaTeX Project Public License Distributed from CTAN
% archives in directory macros/latex/base/lppl.txt.


\section{Watching Trees Grow}

\label{section-trees}


\subsection{Introduction to the  Child Operation}

\emph{Trees} are a common way of visualizing hierarchical
structures. A simple tree looks like this:
\begin{codeexample}[]
\begin{tikzpicture}
  \node {root}
    child {node {left}}
    child {node {right}
      child {node {child}}
      child {node {child}}
    };
\end{tikzpicture}
\end{codeexample}

Admittedly, in reality trees are more likely to grow \emph{upward} and
not downward as above. This is easy enough to specify in \tikzname:

\begin{codeexample}[]
\begin{tikzpicture}
  \node {root} [grow'=up]
    child {node {left}}
    child {node {right}
      child {node {child}}
      child {node {child}}
    };
\end{tikzpicture}
\end{codeexample}

(You can tell whether the author of a paper is a mathematician or a
computer scientist by looking at the direction their trees grow. A
computer scientist's trees will grow downward while a mathematician's
tree will upward. Naturally, the correct way is the mathematician's
way.)

In \tikzname, trees are specified by adding \emph{child nodes} to a
node on a path. The syntax for the child operation is the following:

\begin{pathoperation}{child}{\opt{\oarg{options}}\opt{\marg{child path}}}
  This operation should directly follow a completed |node| operation
  or another |child| operation, although it is permissible that the
  first |child| operation is preceded by options (we will come to
  that).

  The exact effects of this operation are described in the rest of
  this present section.
\end{pathoperation}





\subsection{Where Children and Their Options Are Specified}

When a |node| operation like |node {X}| is followed by |child|,
\tikzname\ starts counting the number of child nodes that follow the
original |node {X}|. For this, it scans the input and stores away each
|child| and its arguments until it reaches a path operation that is
not a |child|. Note that this will fix the character codes or any
text inside the child arguments, which means, in essence, that you
cannot use verbatim text inside the nodes inside a |child|. Sorry. 

Once the children have been collected and counted, \tikzname\ starts
generating the nodes of the children.

Each |child| may have its own \meta{options}, which apply to ``the
whole child,'' including all of its grandchildren. Here is an
example:

\begin{codeexample}[]
\begin{tikzpicture}[thick,sibling distance=10mm on level 2]
  \coordinate
    child[red]   {child child}
    child[green] {child child[blue]};
\end{tikzpicture}
\end{codeexample}

The options of the root node have no effect on the children since
the options of a node are always ``local'' to that node. Because of
this, the edges in the following tree are black, not red.
  
\begin{codeexample}[]
\begin{tikzpicture}[thick]
  \node [red] {root}
    child
    child;
\end{tikzpicture}
\end{codeexample}
  This raises the problem of how to set options for \emph{all}
  children. Naturally, you could always set options for the whole path
  as in |\path [red] node {root} child child;| but this is bothersome
  in some situations. Instead, it is easier to give the options
  \emph{before the first child} as follows:
\begin{codeexample}[]
\begin{tikzpicture}[thick]
  \node [red] {root}
    [green] % option applies to all children
    child
    child;
\end{tikzpicture}
\end{codeexample}

To sum up: Options for the whole tree are given before the root
node. Options for the root node are given directly to the |node|
operation of the root. Options for all children can be given between
the root node and the first child. Options applying to a specific
child are given as options to the child.


\subsection{The Shapes and Content Child Nodes}

For each |child| of a root node, a node is generated and placed
somewhere (the placement rules will be discussed later). The shape of
the child node depends on the \meta{child path} of the child.

In the easiest case, the \meta{child path} is completely missing
(including the curly braces). An example would be
|\node {x} child child;| where both children miss their \meta{child
  path}. In this case the shape is simply a |coordinate|. Thus, the
child node has no extend and no text. 
\begin{codeexample}[]
\tikz \node {root} child child;
\end{codeexample}

Next, the \meta{child path} may \emph{start} with a |node| or a
|coordinate| specification. An example is
|\node {x} child {node {y}};| where the \meta{child path} consists
of the node specification  |node {y}|. In this case, this first node
on the path becomes the child node. As for any normal node, you can
give this child node a name, shift it around, or use options to
influence how it is rendered.
\begin{codeexample}[]
\begin{tikzpicture}
  \node[rectangle,draw] {root}
    child {node[circle,draw] (left node) {left}}
    child {node[ellipse,draw] (right node) {right}};
  \draw[dashed,->] (left node) -- (right node);
\end{tikzpicture}
\end{codeexample}

A third case occurs when the \meta{child path} exists, but does not
start with a |node| or |coordinate| as in
|child {child};|, where the \meta{child path} start with |child|
itself. In this case, a node of shape |coordinate| is automatically
added at the beginning of the path. 

\subsection{The Placement of Child Nodes}

\subsection{The Edge From the Parent Node}





%%% Local Variables: 
%%% mode: latex
%%% TeX-master: "pgfmanual"
%%% End: 

% Copyright 2003 by Till Tantau <tantau@cs.tu-berlin.de>.
%
% This program can be redistributed and/or modified under the terms
% of the LaTeX Project Public License Distributed from CTAN
% archives in directory macros/latex/base/lppl.txt.


\section{Transformations}

\pgfname\ has a powerful transformation mechanism that is similar to
the transformation capabilities of \textsc{metafont}. The present
section explains how you can access it in \tikzname.


\subsection{The Different Coordinate Systems}

It is a long process from  a coordinate like, say, $(1,2)$ or
$(1\mathrm{cm},5\,mathrm{pt})$, to the position a point is finally
placed on the display or paper. In order to find out where the point
should go, it is constantly ``transformed,'' which means that it is
mostly shifted around and possibly rotated, slanted, scaled, and
otherwise mutilated. 

In detail, (at least) the following transformations are applied to a
coordinate like $(1,2)$ before a point on the screen is chosen:
\begin{enumerate}
\item
  \pgfname\ interprets a coordinate like $(1,2)$  in its
  $xy$-coordinate system as ``add the current $x$-vector once and the
  current $y$-vector twice to obtain the new point.''
\item
  \pgfname\ applies its coordinate transformation matrix to the
  resulting coordinate. This yields the final position of the point 
  inside the picture.
\item
  The backend driver (like |dvips| or |pdftex|) adds transformation
  commands such the coordinate is shifted to the correct position in
  \TeX's page coordinate system.
\item
  \textsc{pdf} (or PostScript) apply the canvas transformation
  matrix to the point, which can once more change the position on the
  page. 
\item
  The viewer application or the printer applies the device
  transformation matrix to transform the coordinate to its final pixel
  coordinate on the screen or paper.  
\end{enumerate}

In reality, the process is even more involved, but the above should
give the idea: A point is constantly transformed by changes of the
coordinate system.

In \tikzname, you only have access to the first two coordinate systems:
The $xy$-coordinate system and the coordinate transformation matrix
(these will be explained later). \pgfname\ also allows you to change
the canvas transformation matrix, but you have to use commands of
the core layer directly to do so and you ``better know what you are
doing'' when you do this. The moment you start modifying the
canvas matrix, \pgfname\ immediately looses track of all
coordinates and shapes, anchors, and bounding box computations will no
longer work.


\subsection{The Xy- and Xyz-Coordinate Systems}

The first and easiest coordinate systems are \pgfname's $xy$- and
$xyz$-coordinate systems. The idea is very simple: Whenever you
specify a coordinate like |(2,3)| this means $2v_x + 3v_y$, where
$v_x$ is the current \emph{$x$-vector} and $v_y$ is the current
\emph{$y$-vector}. Similarly, the coordinate |(1,2,3)| means $v_x +
2v_y + 3v_z$.

Unlike other packages, \pgfname\ does not insist that $v_x$ actually
has a $y$-component of $0$, that is, that it is a horizontal
vector. Instead, the $x$-vector can point anywhere you
want. Naturally, \emph{normally} you will want the $x$-vector to point
horizontally.

One undesirable effect of this flexibility is that it is not possible
to provide mixed coordinates as in $(1,2\mathrm{pt})$. Life is hard.

To change the $x$-, $y$-, and $z$-vectors, you can use the following
options:

\begin{itemize}
\itemoption{x}|=|\meta{dimension}
  Sets the $x$-vector of \pgfname's $xyz$-coordinate system to point
  \meta{dimension} to the right, that is, to
  $(\meta{dimension},0pt)$. The default is 1cm.

\begin{codeexample}[]
\begin{tikzpicture}
  \draw                  (0,0)   -- +(1,0);
  \draw[x=2cm,color=red] (0,0.1) -- +(1,0);
\end{tikzpicture}
\end{codeexample}    

\begin{codeexample}[]
\tikz \draw[x=1.5cm] (0,0) grid (2,2);
\end{codeexample}    

The last example shows that the size of steppings in grids, just like
all other dimensions, are not affected by the $x$-vector. After all,
the $x$-vector is only used to determine the coordinate of the upper
right corner of the grid.
\itemoption{x}|=|\meta{coordinate}
  Sets the $x$-vector of \pgfname's $xyz$-coordinate system to the
  specified \meta{coordinate}. If \meta{coordinate} contains a comma,
  it must be put in braces. 

\begin{codeexample}[]
\begin{tikzpicture}
  \draw                            (0,0) -- (1,0);
  \draw[x={(2cm,0.5cm)},color=red] (0,0) -- (1,0);
\end{tikzpicture}
\end{codeexample}

  You can use this, for example, to exchange the meaning of the $x$- and
  $y$-coordinate.

\begin{codeexample}[]
\begin{tikzpicture}[smooth]
  \draw plot coordinates{(1,0) (2,0.5) (3,0) (3,1)};
  \draw[x={(0cm,1cm)},y={(1cm,0cm)},color=red]
        plot coordinates{(1,0) (2,0.5) (3,0) (3,1)};
\end{tikzpicture}
\end{codeexample}

\itemoption{y}|=|\meta{value}
  Works like the |x=| option, only if \meta{value} is a dimension, the
  resulting vector points to $(0,\meta{value})$.
\itemoption{z}|=|\meta{value}
  Works like the |z=| option, but now a dimension is means the point
  $(\meta{value},\meta{value})$.

\begin{codeexample}[]
\begin{tikzpicture}[z=-1cm,->,thick]
  \draw[color=red] (0,0,0) -- (1,0,0);
  \draw[color=blue] (0,0,0) -- (0,1,0);
  \draw[color=orange] (0,0,0) -- (0,0,1);
\end{tikzpicture}
\end{codeexample}
\end{itemize}



\subsection{Coordinate Transformations}

\pgfname\ and \tikzname\ allow you to specify \emph{coordinate
  transformations}. Whenever you specify a coordinate as in |(1,0)| or
|(1cm,1pt)| or |(30:2cm)|, this coordinate is first
``reduced'' to a position of the form ``$x$ points to the right and
  $y$ points upwards.'' For example, |(1in,5pt)| is reduced to
``$72\frac{72}{100}$ points to the right and 5 points upwards'' and
|(90:100pt)| means ``0pt to the right and 100 points upwards.''

The next step is to apply the current \emph{coordinate transformation
  matrix} to the coordinate. For example, the coordinate
transformation matrix might currently be set such that it adds a
certain constant to the $x$ value. Also, it might be setup such that
it, say, exchanges the $x$ and $y$ value. In general, any
``standard'' transformation like translation, rotation, slanting, or
scaling or any combination thereof is possible. (Internally, \pgfname\
keeps track of a coordinate transformation matrix very much like the
concatenation matrix used by \textsc{pdf} or PostScript.)

\begin{codeexample}[]
\begin{tikzpicture}
  \draw[style=help lines] (0,0) grid (3,2);
  \draw (0,0) rectangle (1,0.5);
  \begin{scope}[xshift=1cm]
    \draw             [red]    (0,0) rectangle (1,0.5);
    \draw[yshift=1cm] [blue]   (0,0) rectangle (1,0.5);
    \draw[rotate=30]  [orange] (0,0) rectangle (1,0.5);
  \end{scope}
\end{tikzpicture}
\end{codeexample}

The most important aspect of the coordinate transformation matrix is
\emph{that it applies to coordinates only!} In particular, the
coordinate transformation has no effect on things like the line width
or the dash pattern or the shading angle. In certain cases, it is not
immediately clear whether the coordinate transformation matrix
\emph{should} apply to a certain dimension. For example, should the
coordinate transformation matrix apply to grids? (It does.) And what
about the size of arced corners? (It does not.) The general rule is
``If there is no `coordinate' involved, even `indirectly,' the matrix
is not applied.'' However, sometimes, you simply have to try or look
it up in the documentation whether the matrix will be applied.

Setting the matrix cannot be done directly. Rather, all you can do is
to ``add'' another transformation to the current matrix. However, all
transformations are local to the current \TeX-group. All
transformations are added using graphic options, which are described
below.

Transformations apply immediately when they are encountered ``in the
middle of a path'' and they apply only to the coordinates on the path
following the transformation option. 

\begin{codeexample}[]
\tikz \draw (0,0) rectangle (1,0.5) [xshift=2cm] (0,0) rectangle (1,0.5);
\end{codeexample}

A final word of warning: You should refrain from using ``aggressive''
transformations like a scaling of a factor of 10000. The reason is
that all transformations are done using \TeX, which has a fairly low
accuracy. Furthermore, in certain situations it is necessary that
\tikzname\ \emph{inverts} the current transformation matrix and this will
fail if the transformation matrix is badly conditioned or even
singular (if you do not know what singular matrices are, you are blessed).   

\begin{itemize}
  \itemoption{shift}|={|\meta{coordinate}|}|
  adds the  \meta{coordinate} to all coordinates.
\begin{codeexample}[]
\begin{tikzpicture}
  \draw[style=help lines] (0,0) grid (3,2);
  \draw                       (0,0) -- (1,1) -- (1,0);
  \draw[shift={(1,1)},blue]   (0,0) -- (1,1) -- (1,0);
  \draw[shift={(30:1cm)},red] (0,0) -- (1,1) -- (1,0);
\end{tikzpicture}
\end{codeexample}

  \itemoption{xshift}|=|\meta{dimension}
  adds \meta{dimension} to the $x$ value of all coordinates.  
\begin{codeexample}[]
\begin{tikzpicture}
  \draw[style=help lines] (0,0) grid (3,2);
  \draw                   (0,0) -- (1,1) -- (1,0);
  \draw[xshift=2cm,blue]  (0,0) -- (1,1) -- (1,0);
  \draw[xshift=-10pt,red] (0,0) -- (1,1) -- (1,0);
\end{tikzpicture}
\end{codeexample}

  \itemoption{yshift}|=|\meta{dimension}
  adds \meta{dimension} to the $y$ value of all coordinates.
  
  \itemoption{scale}|=|\meta{factor}
  multiplies all coordinates by the given \meta{factor}. The
  \meta{factor} should not be excessively large in absolute terms or
  very near to zero.
\begin{codeexample}[]
\begin{tikzpicture}
  \draw[style=help lines] (0,0) grid (3,2);
  \draw               (0,0) -- (1,1) -- (1,0);
  \draw[scale=2,blue] (0,0) -- (1,1) -- (1,0);
  \draw[scale=-1,red] (0,0) -- (1,1) -- (1,0);
\end{tikzpicture}
\end{codeexample}

  \itemoption{xscale}|=|\meta{factor}
  multiplies only the $x$-value of all coordinates by the given
  \meta{factor}. 
\begin{codeexample}[]
\begin{tikzpicture}
  \draw[style=help lines] (0,0) grid (3,2);
  \draw                (0,0) -- (1,1) -- (1,0);
  \draw[xscale=2,blue] (0,0) -- (1,1) -- (1,0);
  \draw[xscale=-1,red] (0,0) -- (1,1) -- (1,0);
\end{tikzpicture}
\end{codeexample}

  \itemoption{yscale}|=|\meta{factor}
  multiplies only the $y$-value of all coordinates by \meta{factor}.
 
  \itemoption{xslant}|=|\meta{factor}
  slants the coordinate horizontally by the given \meta{factor}:
\begin{codeexample}[]
\begin{tikzpicture}
  \draw[style=help lines] (0,0) grid (3,2);
  \draw                (0,0) -- (1,1) -- (1,0);
  \draw[xslant=2,blue] (0,0) -- (1,1) -- (1,0);
  \draw[xslant=-1,red] (0,0) -- (1,1) -- (1,0);
\end{tikzpicture}
\end{codeexample}

  \itemoption{yslant}|=|\meta{factor}
  slants the coordinate vertically by the given \meta{factor}:
\begin{codeexample}[]
\begin{tikzpicture}
  \draw[style=help lines] (0,0) grid (3,2);
  \draw                (0,0) -- (1,1) -- (1,0);
  \draw[yslant=2,blue] (0,0) -- (1,1) -- (1,0);
  \draw[yslant=-1,red] (0,0) -- (1,1) -- (1,0);
\end{tikzpicture}
\end{codeexample}

  \itemoption{rotate}|=|\meta{degree}
  rotates the coordinate system by \meta{degree}:
\begin{codeexample}[]
\begin{tikzpicture}
  \draw[style=help lines] (0,0) grid (3,2);
  \draw                 (0,0) -- (1,1) -- (1,0);
  \draw[rotate=40,blue] (0,0) -- (1,1) -- (1,0);
  \draw[rotate=-20,red] (0,0) -- (1,1) -- (1,0);
\end{tikzpicture}
\end{codeexample}

  \itemoption{rotate around}|={|\meta{degree}|:|\meta{coordinate}|}|
  rotates the coordinate system by \meta{degree} around the point
  \meta{coordinate}.
\begin{codeexample}[]
\begin{tikzpicture}
  \draw[style=help lines] (0,0) grid (3,2);
  \draw                                (0,0) -- (1,1) -- (1,0);
  \draw[rotate around={40:(1,1)},blue] (0,0) -- (1,1) -- (1,0);
  \draw[rotate around={-20:(1,1)},red] (0,0) -- (1,1) -- (1,0);
\end{tikzpicture}
\end{codeexample}

  \itemoption{cm}|={|\meta{$a$}|,|\meta{$b$}|,|\meta{$c$}|,|\meta{$d$}|,|\meta{coordinate}|}|
  applies the following transformation to all coordinates: Let $(x,y)$
  be the coordinate to be transformed and let \meta{coordinate}
  specify the point $(t_x,t_y)$. Then the new coordinate is given by
  $\left(\begin{smallmatrix} a & b \\ c & d\end{smallmatrix}\right)
  \left(\begin{smallmatrix} x \\ y \end{smallmatrix}\right) +
  \left(\begin{smallmatrix} t_x \\ t_y
  \end{smallmatrix}\right)$. Usually, you do not use this option
  directly. 
\begin{codeexample}[]
\begin{tikzpicture}
  \draw[style=help lines] (0,0) grid (3,2);
  \draw                             (0,0) -- (1,1) -- (1,0);
  \draw[cm={1,1,0,1,(0,0)},blue]    (0,0) -- (1,1) -- (1,0);
  \draw[cm={0,1,1,0,(1cm,1cm)},red] (0,0) -- (1,1) -- (1,0);
\end{tikzpicture}
\end{codeexample}

  \itemoption{reset cm}
  completely resets the coordinate transformation matrix to the
  identity matrix. This will destroy not only the transformations
  applied in the current scope, but also all transformations inherited
  from surrounding scopes. Do not use this option.
\end{itemize}






\part{Libraries and Utilities}
\label{part-libraries}

In this part the library and utility packages are documented. The
library packages provide additional predefined graphic objects like
new arrow heads, or new plot marks. These are not loaded by default
since many users will not need them.

The utility packages are not directly involved in creating graphics,
but you may find them useful nonetheless. All of them either directly
depend on \pgfname\ or they are designed to work well together with
\pgfname\ even though they can be used in a stand-alone way.
\vskip2cm
\medskip
\noindent
\begin{codeexample}[graphic=white]
\begin{tikzpicture}[scale=2]
  \shade[top color=blue,bottom color=gray!50] (0,0) parabola (1.5,2.25) |- (0,0);
  \draw (1.05cm,2pt) node[above] {$\displaystyle\int_0^{3/2} \!\!x^2\mathrm{d}x$};
  
  \draw[style=help lines] (0,0) grid (3.9,3.9)
       [step=0.25cm]      (1,2) grid +(1,1);

  \draw[->] (-0.2,0) -- (4,0) node[right] {$x$};
  \draw[->] (0,-0.2) -- (0,4) node[above] {$f(x)$};

  \foreach \x/\xtext in {1/1, 1.5/1\frac{1}{2}, 2/2, 3/3}
    \draw[shift={(\x,0)}] (0pt,2pt) -- (0pt,-2pt) node[below] {$\xtext$};

  \foreach \y/\ytext in {1/1, 2/2, 2.25/2\frac{1}{4}, 3/3}
    \draw[shift={(0,\y)}] (2pt,0pt) -- (-2pt,0pt) node[left] {$\ytext$};
    
  \draw (-.5,.25) parabola bend (0,0) (2,4) node[below right] {$x^2$};
\end{tikzpicture}
\end{codeexample}

% Copyright 2003 by Till Tantau <tantau@cs.tu-berlin.de>.
%
% This program can be redistributed and/or modified under the terms
% of the LaTeX Project Public License Distributed from CTAN
% archives in directory macros/latex/base/lppl.txt.


\section{Libraries}

\subsection{Arrow Tip Library}
\label{section-library-arrows}

\begin{package}{pgflibraryarrows}
  The package defines additional arrow tips, which are described
  below. See page~\pageref{standard-arrows} for the arrows tips that
  are defined by default. Note that neither the standard packages nor
  this package defines an arrow name containing |>| or |<|. These are
  left for the user to defined as he or she sees fit.
\end{package}

\subsubsection{Arrow Tips with Differing Names for the Left and Right Ends}

\begin{tabular}{ll}
  \sarrow{[}{]} \\
  \sarrow{]}{[} \\
  \sarrow{(}{)} \\
  \sarrow{)}{(}
\end{tabular}

\subsubsection{Variants of Other Arrow Tips}

The same name is used for both the start and end arrows. Thus, for
example, to  install the first of the following arrow tips for both
the start and the end, you would say |\pgfsetarrows{latex'-latex'}|: 

\smallskip
\begin{tabular}{ll}
  \symarrow{latex'} \\
  \symarrow{latex' reversed}  \\
  \symarrow{stealth'} \\
  \symarrow{stealth' reversed}
\end{tabular}

\subsubsection{General Purpose Arrow Tips}

\begin{tabular}{ll}
  \symarrow{o} \\
  \symarrow{*} \\
  \symarrow{diamond} \\
  \symarrow{open diamond}   \\
  \symarrow{angle 90} \\
  \symarrow{angle 90 reversed}   \\
  \symarrow{triangle 90} \\
  \symarrow{triangle 90 reversed}   \\
  \symarrow{open triangle 90} \\
  \symarrow{open triangle 90 reversed}   \\
  \symarrow{left to} \\
  \symarrow{left to reversed} \\
  \symarrow{right to} \\
  \symarrow{right to reversed} \\
  \symarrow{hooks} \\
  \symarrow{hooks reversed} \\
  \symarrow{left hook} \\
  \symarrow{left hook reversed} \\
  \symarrow{right hook} \\
  \symarrow{right hook reversed}
\end{tabular}

\subsubsection{Line Caps}

\begin{tabular}{ll}
  \carrow{round cap} \\
  \carrow{butt cap} \\
  \carrow{triangle 90 cap} \\
  \carrow{triangle 90 cap reversed} \\
  \carrow{fast cap} \\
  \carrow{fast cap reversed} \\
\end{tabular}



\subsection{Plot Handler Library}
\label{section-library-plothandlers}

\begin{package}{pgflibraryplothandlers}
  This library packages defines additional plot handlers, see
  Section~\ref{section-plot-handlers} for an introduction to plot
  handlers. The additional handlers are described in the following. 
\end{package}


\subsubsection{Curve Plot Handlers}
  
\begin{command}{\pgfplothandlercurveto}
  This handler will issue a |\pgfpathcurveto| command for each point of
  the plot, \emph{except} possibly for the first. As for the line-to
  handler, what happens with the first point can be specified using
  |\pgfsetmovetofirstplotpoint| or |\pgfsetlinetofirstplotpoint|.

  Obviously, the |\pgfpathcurveto| command needs, in addition to the
  points on the path, some control points. These are generated
  automatically using a somewhat ``dumb'' algorithm: Suppose you have
  three points $x$, $y$, and $z$ on the curve such that $y$ is between
  $x$ and $z$:
\begin{codeexample}[]
\begin{tikzpicture}    
  \draw[gray] (0,0) node {x} (1,1) node {y} (2,.5) node {z};
  \pgfplothandlercurveto
  \pgfplotstreamstart
  \pgfplotstreampoint{\pgfpoint{0cm}{0cm}}
  \pgfplotstreampoint{\pgfpoint{1cm}{1cm}}
  \pgfplotstreampoint{\pgfpoint{2cm}{.5cm}}
  \pgfplotstreamend
  \pgfusepath{stroke}
\end{tikzpicture}
\end{codeexample}

  In order to determine the control points of the curve at the point
  $y$, the handler computes the vector $z-x$ and scales it by the
  tension factor (see below). Let us call the resulting vector
  $s$. Then $y+s$ and $y-s$ will be the control points around $y$. The
  first control point at the beginning of the curve will be the
  beginning itself, once more; likewise the last control point is the
  end itself.
\end{command}

\begin{command}{\pgfsetplottension\marg{value}}
  Sets the factor used by the curve plot handlers to determine the
  distance of the control points from the points they control. The
  default is $0.15$. The higher the curvature of the curve points, the
  higher this value should be.

\begin{codeexample}[]
\begin{tikzpicture}    
  \draw[gray] (0,0) node {x} (1,1) node {y} (2,.5) node {z};
  \pgfsetplottension{0.75}
  \pgfplothandlercurveto
  \pgfplotstreamstart
  \pgfplotstreampoint{\pgfpoint{0cm}{0cm}}
  \pgfplotstreampoint{\pgfpoint{1cm}{1cm}}
  \pgfplotstreampoint{\pgfpoint{2cm}{0.5cm}}
  \pgfplotstreamend
  \pgfusepath{stroke}
\end{tikzpicture}
\end{codeexample}
\end{command}


\begin{command}{\pgfplothandlerclosedcurve}
  This handler works like the curve-to plot handler, only it will
  add a new part to the current path that is a closed curve through
  the plot points.
\begin{codeexample}[]
\begin{tikzpicture}    
  \draw[gray] (0,0) node {x} (1,1) node {y} (2,.5) node {z};
  \pgfplothandlerclosedcurve
  \pgfplotstreamstart
  \pgfplotstreampoint{\pgfpoint{0cm}{0cm}}
  \pgfplotstreampoint{\pgfpoint{1cm}{1cm}}
  \pgfplotstreampoint{\pgfpoint{2cm}{0.5cm}}
  \pgfplotstreamend
  \pgfusepath{stroke}
\end{tikzpicture}
\end{codeexample}
\end{command}


\subsubsection{Comb Plot Handlers}

There are three ``comb'' plot handlers. There name stems from the fact
that the plots they produce look like ``combs'' (more or less).

\begin{command}{\pgfplothandlerxcomb}
  This handler converts each point in the plot stream into a line from
  the $y$-axis to the point's coordinate, resulting in a ``vertical
  comb.''

  This handler is useful for creating ``bar diagrams.''
  
\begin{codeexample}[]
\begin{tikzpicture}    
  \draw[gray] (0,0) node {x} (1,1) node {y} (2,.5) node {z};
  \pgfplothandlerxcomb
  \pgfplotstreamstart
  \pgfplotstreampoint{\pgfpoint{0cm}{0cm}}
  \pgfplotstreampoint{\pgfpoint{1cm}{1cm}}
  \pgfplotstreampoint{\pgfpoint{2cm}{0.5cm}}
  \pgfplotstreamend
  \pgfusepath{stroke}
\end{tikzpicture}
\end{codeexample}
\end{command}


\begin{command}{\pgfplothandlerycomb}
  This handler converts each point in the plot stream into a line from
  the $x$-axis to the point's coordinate, resulting in a ``vertical
  comb.''
  
\begin{codeexample}[]
\begin{tikzpicture}    
  \draw[gray] (0,0) node {x} (1,1) node {y} (2,.5) node {z};
  \pgfplothandlerycomb
  \pgfplotstreamstart
  \pgfplotstreampoint{\pgfpoint{0cm}{0cm}}
  \pgfplotstreampoint{\pgfpoint{1cm}{1cm}}
  \pgfplotstreampoint{\pgfpoint{2cm}{0.5cm}}
  \pgfplotstreamend
  \pgfusepath{stroke}
\end{tikzpicture}
\end{codeexample}
\end{command}

\begin{command}{\pgfplothandlerpolarcomb}
  This handler converts each point in the plot stream into a line from
  the origin to the point's coordinate.
  
\begin{codeexample}[]
\begin{tikzpicture}    
  \draw[gray] (0,0) node {x} (1,1) node {y} (2,.5) node {z};
  \pgfplothandlerpolarcomb
  \pgfplotstreamstart
  \pgfplotstreampoint{\pgfpoint{0cm}{0cm}}
  \pgfplotstreampoint{\pgfpoint{1cm}{1cm}}
  \pgfplotstreampoint{\pgfpoint{2cm}{0.5cm}}
  \pgfplotstreamend
  \pgfusepath{stroke}
\end{tikzpicture}
\end{codeexample}
\end{command}

\subsubsection{Mark Plot Handler}

\label{section-plot-marks}

\begin{command}{\pgfplothandlermark\marg{mark code}}
  This command will execute the \meta{mark code} for each point of the
  plot, but each time the coordinate transformation matrix will be
  setup such that the origin is at the position of the point to be
  plotted. This way, if the \meta{mark code} draws a little circle
  around the origin, little circles will be drawn at each point of the
  plot.
  
\begin{codeexample}[]
\begin{tikzpicture}    
  \draw[gray] (0,0) node {x} (1,1) node {y} (2,.5) node {z};
  \pgfplothandlermark{\pgfpathcircle{\pgfpointorigin}{4pt}\pgfusepath{stroke}}
  \pgfplotstreamstart
  \pgfplotstreampoint{\pgfpoint{0cm}{0cm}}
  \pgfplotstreampoint{\pgfpoint{1cm}{1cm}}
  \pgfplotstreampoint{\pgfpoint{2cm}{0.5cm}}
  \pgfplotstreamend
  \pgfusepath{stroke}
\end{tikzpicture}
\end{codeexample}

  Typically, the \meta{code} will be |\pgfuseplotmark{|\meta{plot mark
      name}|}|, where \meta{plot mark name} is the name of a
  predefined plot mark.
\end{command}

\begin{command}{\pgfuseplotmark\marg{plot mark name}}
  Draws the given \meta{plot mark name} at the origin. The \meta{plot
    mark name} must have been previously declared using
  |\pgfdeclareplotmark|. 

\begin{codeexample}[]
\begin{tikzpicture}    
  \draw[gray] (0,0) node {x} (1,1) node {y} (2,.5) node {z};
  \pgfplothandlermark{\pgfuseplotmark{pentagon}}
  \pgfplotstreamstart
  \pgfplotstreampoint{\pgfpoint{0cm}{0cm}}
  \pgfplotstreampoint{\pgfpoint{1cm}{1cm}}
  \pgfplotstreampoint{\pgfpoint{2cm}{0.5cm}}
  \pgfplotstreamend
  \pgfusepath{stroke}
\end{tikzpicture}
\end{codeexample}
\end{command}

\begin{command}{\pgfdeclareplotmark\marg{plot mark name}\marg{code}}
  Declares a plot mark for later used with the |\pgfuseplotmark|
  command.

\begin{codeexample}[]
\pgfdeclareplotmark{my plot mark}
  {\pgfpathcircle{\pgfpoint{0cm}{1ex}}{1ex}\pgfusepathqstroke}  
\begin{tikzpicture}    
  \draw[gray] (0,0) node {x} (1,1) node {y} (2,.5) node {z};
  \pgfplothandlermark{\pgfuseplotmark{my plot mark}}
  \pgfplotstreamstart
  \pgfplotstreampoint{\pgfpoint{0cm}{0cm}}
  \pgfplotstreampoint{\pgfpoint{1cm}{1cm}}
  \pgfplotstreampoint{\pgfpoint{2cm}{0.5cm}}
  \pgfplotstreamend
  \pgfusepath{stroke}
\end{tikzpicture}
\end{codeexample}
\end{command}


\begin{command}{\pgfsetplotmarksize\marg{dimension}}
  This command sets the \TeX\ dimension |\pgfplotmarksize| to
  \meta{dimension}. This dimension is a ``recommendation'' for plot
  mark code at which size the plot mark should be drawn; plot mark
  code may choose to ignore this \meta{dimension} altogether. For
  circles, \meta{dimension} should  be the radius, for other shapes it
  should be about half the width/height.

  The predefined plot marks all take this dimension into account.

\begin{codeexample}[]
\begin{tikzpicture}    
  \draw[gray] (0,0) node {x} (1,1) node {y} (2,.5) node {z};
  \pgfsetplotmarksize{1ex}
  \pgfplothandlermark{\pgfuseplotmark{*}}
  \pgfplotstreamstart
  \pgfplotstreampoint{\pgfpoint{0cm}{0cm}}
  \pgfplotstreampoint{\pgfpoint{1cm}{1cm}}
  \pgfplotstreampoint{\pgfpoint{2cm}{0.5cm}}
  \pgfplotstreamend
  \pgfusepath{stroke}
\end{tikzpicture}
\end{codeexample}
\end{command}

\begin{textoken}{\pgfplotmarksize}
  A \TeX\ dimension that is a ``recommendation'' for the size of plot
  marks.
\end{textoken}

The following plot marks are predefined (the filling color has been
set to yellow):

\medskip
\begin{tabular}{lc}
  \plotmarkentry{*}
  \plotmarkentry{x}
  \plotmarkentry{+}
\end{tabular}


\subsection{Plot Mark Library}

\begin{package}{pgflibraryplotmarks}
  When this package is loaded, the following plot marks are defined in
  addition to |*|, |x|, and |+| (the filling color has been set to yellow):

  \catcode`\|=12
  \medskip
  \begin{tabular}{lc}
    \plotmarkentry{-}
    \index{*vbar@\protect\texttt{\protect\myvbar} plot mark}%
    \index{Plot marks!*vbar@\protect\texttt{\protect\myvbar}}
    \texttt{\char`\\pgfuseplotmark\char`\{\declare{|}\char`\}} &
    \tikz\draw[color=black!25] plot[mark=|,fill=yellow,draw=black]
    coordinates {(0,0) (.5,0.2) (1,0) (1.5,0.2)};\\
    \plotmarkentry{o}
    \plotmarkentry{asterisk}
    \plotmarkentry{star}
    \plotmarkentry{oplus}
    \plotmarkentry{oplus*}
    \plotmarkentry{otimes}
    \plotmarkentry{otimes*}
    \plotmarkentry{square}
    \plotmarkentry{square*}
    \plotmarkentry{triangle}
    \plotmarkentry{triangle*}
    \plotmarkentry{diamond}
    \plotmarkentry{diamond*}
    \plotmarkentry{pentagon}
    \plotmarkentry{pentagon*}
  \end{tabular}
\end{package}

\subsection{Shape Library}

\begin{shape}{ellipse}
  This shape is an ellipse tightly fitting the text box, if not inner
  separation is given. The following figure shows the anchors this
  shape defines; the anchors |10| and |130| are example of border anchors.
\begin{codeexample}[]
\Huge
\begin{tikzpicture}
  \node[name=s,shape=ellipse,style=shape example] {Ellipse\vrule width 1pt height 2cm};
  \foreach \anchor/\placement in
    {north west/above left, north/above, north east/above right, 
     west/left, center/above, east/right, 
     mid west/right, mid/above, mid east/left, 
     base west/left, base/below, base east/right, 
     south west/below left, south/below, south east/below right, 
     text/left, 10/right, 130/above}
     \draw[shift=(s.\anchor)] plot[mark=x] coordinates{(0,0)}
       node[\placement] {\scriptsize\texttt{(s.\anchor)}};
\end{tikzpicture}
\end{codeexample}
\end{shape}

%%% Local Variables: 
%%% mode: latex
%%% TeX-master: "pgfmanual"
%%% End: 

% Copyright 2003 by Till Tantau <tantau@cs.tu-berlin.de>.
%
% This program can be redistributed and/or modified under the terms
% of the LaTeX Project Public License Distributed from CTAN
% archives in directory macros/latex/base/lppl.txt.

\section{Repeating Things: The Foreach Statement}

In this section the package |pgffor.sty| is described. It can be used
independently of \pgfname, but it works particularly well together with
\pgfname\ and \tikzname.

When you say |\usepackage{pgffor}|, two commands are defined:
|\foreach| and |\breakforeach|. Their behaviour is described in the
following:

\begin{command}{\foreach| |\meta{variables}| in |\marg{list}
    \meta{commands}}
  The syntax of this command is a bit complicated, so let us go
  through it step-by-step.

  In the easiest case, \meta{variables} is a single \TeX-command like
  |\x| or |\point|. (If you want to have some fun, you can also use
  active characters. If you do not know what active characters are,
  you are blessed.)

  Still in the easiest case, \meta{list} is a comma-separated list of
  values. Anything can be used as a value, but numbers are most
  likely.

  Finally, in the easiest case, \meta{commands} is some \TeX-text in
  curly braces.

  With all these assumptions, the |\foreach| statement will execute
  the \meta{commands} repeatedly, once for every element of the
  \meta{list}. Each time the \meta{commands} are executed, the
  \meta{variable} will be set to the current value of the list item.

\begin{codeexample}[]
\foreach \x in {1,2,3,0} {[\x]}
\end{codeexample}

  \medskip
  \textbf{Syntax for the commands.}
  Let use move on to a more complicated setting. The first
  complication occurs when the \meta{commands} are not some text in
  curly braces. If the |\foreach| statement does not encounter an
  opening brace, it will instead scan everything up to the next
  semicolon and use this as \meta{commands}. This is most useful in
  situations like the following:

\begin{codeexample}[]
\tikz
  \foreach \x in {0,1,2,3}
    \draw (\x,0) circle (0.2cm);
\end{codeexample}

  However, the ``reading till the next semicolon'' is not the whole
  truth. There is another rule: If a |\foreach| statement is directly
  followed by another |\foreach| statement, this second foreach
  statement is collected as \meta{commands}. This allows you to write
  the following:

\begin{codeexample}[]
\begin{tikzpicture}
  \foreach \x in {0,1,2,3}
    \foreach \y in {0,1,2,3}
      {
        \draw (\x,\y) circle (0.2cm);
        \fill (\x,\y) circle (0.1cm);
      }
\end{tikzpicture}
\end{codeexample}

  \medskip
  \textbf{The dots notation.}
  The second complication concerns the \meta{list}. If this
  \meta{list} contains the list item ``|...|'', this list item is replaced
  by the ``missing values.'' More precisely, the following happens:

  Normally, when a list item |...| is encountered, there should
  already have been \emph{two} list items before it, which where
  numbers. Examples of \emph{numbers} are |1|, |-10|, or
  |-0.24|. Let us call these numbers $x$ and $y$ and let $d := y-x$ be
  their difference. Next, there should also be one number following
  the three dots, let us call this number~$z$.

  In this situation, the part of the list reading
  ``$x$|,|$y$|,...,|$z$'' is replaced by ``$x$, $x+d$, $x+2d$, $x+3d$,
  \dots, $x+md$,'' where the last dots are semantic dots, not
  syntactic dots. The value $m$ is the largest number such that $x +
  md \le z$ if $d$ is positive or such that $x+md \ge z$ if $d$ is
  negative. 

  Perhaps it is best to explain this by some examples:  The following
  \meta{list} have the same effects:

  |\foreach \x in {1,2,...,6} {\x, }| yields \foreach \x in {1,2,...,6} {\x, }

  |\foreach \x in {1,2,3,...,6} {\x, }| yields \foreach \x in {1,2,3,...,6} {\x, }

  |\foreach \x in {1,3,...,11} {\x, }| yields \foreach \x in {1,3,...,11} {\x, }

  |\foreach \x in {1,3,...,10} {\x, }| yields \foreach \x in {1,3,...,10} {\x, }

  |\foreach \x in {0,0.1,...,0.5} {\x, }| yields \foreach \x in {0,0.1,...,0.5} {\x, }

  |\foreach \x in {a,b,9,8,...,1,2,2.125,...,2.5} {\x, }| yields \foreach \x in {a,b,9,8,...,1,2,2.125,...,2.5} {\x, }

  As can be seen, for fractional steps that are not multiples of
  $2^{-n}$ for some small $n$, rounding errors can occur pretty
  easily. Thus, in the second last case, |0.5| should probably be
  replaced by |0.501| for robustness.
  
  There is yet another special case for the |...| statement: If the
  |...| is used right after the first item in the list, that is, if
  there is an $x$, but no $y$, the difference $d$ obviously cannot be
  computed and is set to $1$ if the number $z$ following the dots is
  larger than $x$ and is set to $-1$ if $z$ is smaller:

  |\foreach \x in {1,...,6} {\x, }| yields \foreach \x in {1,...,6} {\x, }

  |\foreach \x in {9,...,3.5} {\x, }| yields \foreach \x in {9,...,3.5} {\x, }

  \medskip
  \textbf{Special handling of pairs.}
  Different list items are separated by commas. However, this causes a
  problem when the list items contain commas themselves as pairs like
  |(0,1)| do. In this case, you should put the items containing commas
  in braces as in |{(0,1)}|. However, since pairs are such a natural
  and useful case, they get a special treatment by the |\foreach|
  statement. When a list item starts with a |(| everything up to the
  next |)| is made part of the item. Thus, we can write things like
  the following:

\begin{codeexample}[]
\tikz
  \foreach \position in {(0,0), (1,1), (2,0), (3,1)}
    \draw \position rectangle +(.25,.5);
\end{codeexample}
  
  \medskip
  \textbf{Using the foreach-statement inside paths.}
  \tikzname\ allows you to use a |\foreach| statement inside a path
  construction. In such a case, the \meta{commands} must be path
  construction commands. Here are two examples:

\begin{codeexample}[]
\tikz
  \draw (0,0)
    \foreach \x in {1,...,3}
      { -- (\x,1) -- (\x,0) }
    ;
\end{codeexample}

\begin{codeexample}[]
\tikz \draw \foreach \p in {1,...,3} {(\p,1)--(\p,3) (1,\p)--(3,\p)};
\end{codeexample}
    
  \medskip
  \textbf{Multiple variables.}
  You will often wish to iterate over two variables at the same
  time. Since you can nest |\foreach| loops, this is normally
  straight-forward. However, you sometimes wish variables to
  iterate ``simultaneously.'' For example, we might be given a list of
  edges that connect two coordinates and might wish to iterate over
  these edges. While doing so, we would like the source and target of
  the edges to be set to two different variables.

  To achieve this, you can use the following syntax: The
  \meta{variables} may not only be a single \TeX-variable. Instead, it
  can also be a list of variables separated by slashes (|/|). In this
  case the list items can also be lists of values separated by
  slashes.

  Assuming that the \meta{variables} and the list items are lists of
  values, each time the \meta{commands} are executed, each of the
  variables in \meta{variables} is set to one part of the list making
  up the current list item. Here is an example to clarify this:

  \example |\foreach \x / \y in {1/2,a/b} {``\x\ and \y''}| yields
  \foreach \x / \y in {1/2,a/b} {``\x\ and \y''}.

  If some entry in the \meta{list} does not have ``enough'' slashes,
  the last entry will be repeated. Here is an example:
\begin{codeexample}[]
\begin{tikzpicture}
  \foreach \x/\xtext in {0,...,3,2.72 / e}
    \draw (\x,0) node{$\xtext$};
\end{tikzpicture}
\end{codeexample}
  
  Here are more useful examples:
\begin{codeexample}[]
\begin{tikzpicture}
  % Define some coordinates:
  \tikzstyle{every node}=[draw,fill]
  \path[shape=circle,fill=examplefill]
    (0,0)    node(a) {a}
    (2,0.55) node(b) {b}
    (1,1.5)  node(c) {c}
    (2,1.75) node(d) {d};

  % Draw some connections:
  \foreach \source/\target in {a/b, b/c, c/a, c/d}
    \draw (\source) .. controls +(.75cm,0pt) and +(-.75cm,0pt)..(\target);  
\end{tikzpicture}
\end{codeexample}

\begin{codeexample}[]
\begin{tikzpicture}
  % Let's draw circles at interesting points:
  \foreach \x / \y / \diameter in {0 / 0 / 2mm, 1 / 1 / 3mm, 2 / 0 / 1mm}
    \draw (\x,\y) circle (\diameter);
    
  % Same effect
  \foreach \center/\diameter in {{(0,0)/2mm}, {(1,1)/3mm}, {(2,0)/1mm}}
    \draw[yshift=2.5cm] \center circle (\diameter);
\end{tikzpicture}
\end{codeexample}

\begin{codeexample}[]
\begin{tikzpicture}[cap=round,line width=3pt]
  \filldraw [fill=examplefill] (0,0) circle (2cm);

  \foreach \angle / \label in
    {0/3, 30/2, 60/1, 90/12, 120/11, 150/10, 180/9,
     210/8, 240/7, 270/6, 300/5, 330/4}
  {
    \draw[line width=1pt] (\angle:1.8cm) -- (\angle:2cm);
    \draw (\angle:1.4cm) node{\textsf{\label}};
  }
  
  \foreach \angle in {0,90,180,270}
    \draw[line width=2pt] (\angle:1.6cm) -- (\angle:2cm);

  \draw (0,0) -- (120:0.8cm); % hour
  \draw (0,0) -- (90:1cm);    % minute
\end{tikzpicture}%
\end{codeexample}

\begin{codeexample}[]
\tikz[shading=ball]
  \foreach \x / \cola in {0/red,1/green,2/blue,3/yellow}
    \foreach \y / \colb in {0/red,1/green,2/blue,3/yellow}
      \shade[ball color=\cola!50!\colb] (\x,\y) circle (0.4cm);
\end{codeexample}
\end{command}


\begin{command}{\breakforeach}
  If this command is given inside a |\foreach| command, no further
  executions of the \meta{commands} will occur. However, the current
  execution of the \meta{commands} is continued normally, so it is
  probably best to use this command only at the end of a |\foreach|
  command. 

\begin{codeexample}[]
\begin{tikzpicture}
  \foreach \x in {1,...,4}
    \foreach \y in {1,...,4}
    {
      \fill[red!50] (\x,\y) ellipse (3pt and 6pt);

      \ifnum \x<\y
        \breakforeach
      \fi
    }      
\end{tikzpicture}
\end{codeexample}
  
\end{command}



% Copyright 2003 by Till Tantau <tantau@cs.tu-berlin.de>.
%
% This program can be redistributed and/or modified under the terms
% of the LaTeX Project Public License Distributed from CTAN
% archives in directory macros/latex/base/lppl.txt.


\section{Page Management}

This section describes the |pgfpages| packages. Although this package
is not concerned with creating pictures, its implementation relies so
heavily on \pgfname\ that it is documented here. Currently, |pgfpages|
only works with \LaTeX, but if you are adventurous, feel free to hack
the code so that it also works with plain \TeX.

The aim of |pgfpages| is to provide a flexible way of putting multiple
pages on a single page \emph{inside \TeX}. Thus, |pgfpages| is quite
different from useful tools like |psnup| or |pdfnup| insofar as it
creates its output in a single pass. Furthermore, it works uniformly
with both |latex| and |pdflatex|, making it easy to put multiple pages
on a single page without any fuss.

A word of warning: \emph{using |pgfpages| will destroy
  hyperlinks}. Actually, the hyperlinks are not destroyed, only they
will appear at totally wrong positions on the final output. This is
due to a fundamental flaw in the \pdf\ specification: In \pdf\ the
bounding rectangle of a hyperlink is given in ``absolute
page coordinates'' and translations or rotations do not affect
them. Thus, the transformations applied by |pgfpages| to put the pages
where you want them are (cannot, even) be applied to the coordinates
of hyperlinks. It is unlikely that this will change in the foreseeable
future.


\subsection{Basic Usage}

The internals of |pgfpages| are complex since the package can do all
sorts of interesting tricks. For this reason, so-called \emph{layouts}
are predefined that setup all option in appropriate ways.

You use a layout as follows:
\begin{codeexample}[code only]
\documentclass{article}

\usepackage{pgfpages}
\pgfpagesuselayout{2 on 1}[a4paper,landscape,border shrink=5mm]

\begin{document}
This text is shown on the left.
\clearpage
This text is shown on the right.
\end{document}
\end{codeexample}

The layout |2 on 1| puts two pages on a single page. The option
|a4paper| tells |pgfpages| that the \emph{resulting} page (called the
\emph{physical} page in the following) should be |a4paper| and it
should be landscape (which is quite logical since putting two portrait
pages next to each other gives a landscape page). Normally, the
\emph{logical} pages, that is, the pages that \TeX\ ``thinks'' that it
is typesetting, will have the same sizes, but this need not be the
case. |pgfpages| will automatically scale down the logical pages such
that two logical pages fit next to each other inside a DIN A4 page.

The |border shrink| tells |pgfpages| that it should add an additional
5mm to the shrinking such that a 5mm-wide border is shown around the
resulting logical pages.

As a second example, let us put two pages produced by the
\textsc{beamer} class on a single page:

\begin{codeexample}[code only]
\documentclass{beamer}

\usepackage{pgfpages}
\pgfpagesuselayout{2 on 1}[a4paper,border shrink=5mm]

\begin{document}
\begin{frame}
  This text is shown at the top.
\end{frame}
\begin{frame}
  This text is shown at the bottom.
\end{frame}
\end{document}
\end{codeexample}

Note that we do not use the |landscape| option since \textsc{beamer}'s
logical pages are already in landscape mode and putting two landscape
pages on top of each other results in a portrait page. However, if you
had used the |4 on 1| layout, you would have had to add |landscape|
once more, using the |8 on 1| you must not, using |16 on 1| you need
it yet again. And, no, there is no |32 on 1| layout.

Another word of caution: \emph{using |pgfpages| will produce wrong
  page numbers in the |.aux| file}. The reason is that \TeX\
instantiates the page numbers when writing an |.aux| file only when
the physical page is shipped out. Fortunately, this problem is easy to
fix: First, typeset our file normally without using the
|\pgfpagesuselayout| command (just put the comment marker |%| before it)
Then, rerun \TeX\ with the |\pgfpagesuselayout| command included and add
the command |\nofiles|. This command ensures that the |.aux| file is
not modified, which is exactly what you want. So, to typeset the above
example, you should actually first \TeX\ the following file:

\begin{codeexample}[code only]
\documentclass{article}

\usepackage{pgfpages}
%%\pgfpagesuselayout{2 on 1}[a4paper,landscape,border shrink=5mm]
%%\nofiles

\begin{document}
This text is shown on the left.
\clearpage
This text is shown on the right.
\end{document}
\end{codeexample}
and then typeset
\begin{codeexample}[code only]
\documentclass{article}

\usepackage{pgfpages}
\pgfpagesuselayout{2 on 1}[a4paper,landscape,border shrink=5mm]
\nofiles

\begin{document}
This text is shown on the left.
\clearpage
This text is shown on the right.
\end{document}
\end{codeexample}

The final basic example is the |resize to| layout (it works a bit like
a hypothetical |1 on 1| layout). This layout resizes the logical page
such that is fits the specified physical size. Since this does not
change the page numbering, you need not worry about the |.aux| files
with this layout. For example, adding the following lines will ensure
that the physical output will fit on DIN A4 paper:
\begin{codeexample}[code only]
\usepackage{pgfpages}
\pgfpagesuselayout{resize to}[a4paper]
\end{codeexample}

This can be very useful when you have to handle lots of papers that
are typeset for, say, letter paper and you have an A4 printer or the
other way round. For example, the following article will be fit for
printing on letter paper:
\begin{codeexample}[code only]
\documentclass[a4paper]{article}
%% a4 is currently the logical size and also the physical size

\usepackage{pgfpages}
\pgfpagesuselayout{resize to}[letterpaper]
%% a4 is still the logical size, but letter is the physical one

\begin{document}
  \title{My Great Article}
...
\end{document}
\end{codeexample}



\subsection{The Predefined Layouts}

This section explains the predefined layouts in more detail. You
select a layout using the following command:
\begin{command}{\pgfpagesuselayout\marg{layout}\oarg{options}}
  Installs the specified \meta{layout} with the given \meta{options}
  set. The predefined layouts and their permissible options are
  explained below.

  If this function is called multiple times, only the last call
  ``wins.'' You can thereby overwrite any previous settings. In
  particular, layouts \emph{do not} accumulate.

  \example |\pgfpagesuselayout{resize to}[a4paper]|
\end{command}

\begin{pgflayout}{resize to}
  This layout is used to resize every logical page to a specified
  physical size. To determine the target size, the following options
  may be given:
  \begin{itemize}
  \item
    \declare{|physical paper height=|\meta{size}} sets the
    height of the physical pape size to \meta{size}.
  \item
    \declare{|physical paper width=|\meta{size}} sets the
    width of the physical pape size to \meta{size}.
  \item
    \declare{|a0paper|} sets the physical page size to DIN A0 paper.
  \item
    \declare{|a1paper|} sets the physical page size to DIN A1 paper.
  \item
    \declare{|a2paper|} sets the physical page size to DIN A2 paper.
  \item
    \declare{|a3paper|} sets the physical page size to DIN A3 paper.
  \item
    \declare{|a4paper|} sets the physical page size to DIN A4 paper.
  \item
    \declare{|a5paper|} sets the physical page size to DIN A5 paper.
  \item
    \declare{|a6paper|} sets the physical page size to DIN A6 paper.
  \item
    \declare{|letterpaper|} sets the physical page size to the
    American letter paper size.
  \item
    \declare{|legalpaper|} sets the physical page size to the
    American legal paper size.
  \item
    \declare{|executivepaper|} sets the physical page size to the
    American executive paper size.
  \item
    \declare{|landscape|} swaps the height and the width of the
    physical paper.
  \item
    \declare{|border shrink=|\meta{size}} additionally reduces the
    size of the logical page on the physical page by \meta{size}.
  \end{itemize}
\end{pgflayout}

\begin{pgflayout}{2 on 1}
  Puts two logical pages alongside each other on each physical page if
  the logical height is larger than the logical width (logical pages
  are in portrait mode). Otherwise, two
  logical pages are put on top of each other (logical pages are in
  landscape mode). When using this layout, it is advisable to use the
  |\nofiles| command, but this is not done automatically.

  The same \meta{options} as for the |resize to| layout an be used,
  plus the following option:
  \begin{itemize}
  \item
    \declare{|odd numbered pages right|}
    places the first page on the right.
  \end{itemize}
\end{pgflayout}


\begin{pgflayout}{4 on 1}
  Puts four logical pages on a single physical page.
  The same \meta{options} as for the |resize to| layout an be used.
\end{pgflayout}

\begin{pgflayout}{8 on 1}
  Puts eight logical pages on a single physical page. As for |2 on 1|,
  the orientation depends on whether the logical pages are in
  landscape mode or in portrait mode.
\end{pgflayout}

\begin{pgflayout}{16 on 1}
  This is for the \textsc{ceo}.
\end{pgflayout}

\begin{pgflayout}{rounded corners}
  \label{layout-rounded-corners}
  This layout adds ``rounded corners'' to every page, which,
  supposedly, looks nicer during presentations with projectors
  (personally, I doubt this). This is done by (possibly) resizing the
  page to the physical page size. Then four black rectangles are
  drawn in each corner. Next, a clipping region is set up that
  contains all of the logical page except for little rounded
  corners. Finally, the logical page is draw, clipped against the
  clipping region. 

  Note that every logical page should fill its background for this to
  work.

  In addition to the \meta{options} that can be given to |resize to|
  the following options may be given.
  \begin{itemize}
    \item \declare{|corner width=|\meta{size}} specifies the size of
    the corner.
  \end{itemize}

  \begin{codeexample}[code only]
\documentclass{beamer}
\usepackage{pgfpages}
\pgfpagesuselayout{rounded corners}[corner width=5pt]
\begin{document}
...
\end{document}
\end{codeexample}
\end{pgflayout}

\begin{pgflayout}{two screens with lagging second}
  This layout puts two logical pages alongside each other. The second
  page always shows what the main
  page showed on the previous physical page. Thus, the second page
  ``lags behind'' the main page. This can be useful when you have to
  projectors attached to your computer and can show different parts of
  a physical page on different projectors.

  The following \meta{options} may be given:
  \begin{itemize}
  \item \declare{|second right|} puts the second page right of the
    main page. This will make the physical pages twice as wide
    as the logical pages, but it will retain the height.
  \item \declare{|second left|} puts the second page left,
    otherwise it behave the same as |second right|.
  \item \declare{|second bottom|} puts the second page below the main
    page. This make the physical pages twice as high as the logical
    ones.
  \item \declare{|second top|} works like |second bottom|.      
  \end{itemize}
\end{pgflayout}

\begin{pgflayout}{two screens with optional second}
  This layout works similarly to
  |two screens with lagging second|. The difference is that the
  contents of the second screen only changes when one of the commands
  |\pgfshipoutlogicalpage{2}|\marg{box} or
  |\pgfcurrentpagewillbelogicalpage{2}| is called. The first puts the
  given \meta{box} on the second page. The second specifies that the
  current page should be put there, once it is finished.

  The same options as for |two screens with lagging second| may be
  given. 
\end{pgflayout}



You can define your own predefined layouts using the following
command:

\begin{command}{\pgfpagesdeclarelayout\marg{layout}\marg{before
      actions}\marg{after actions}}
  This command predefines a \meta{layout} that can later be installed
  using the |\pgfpagesuselayout| command.

  When |\pgfpagesuselayout|\marg{layout}\oarg{options} is called, the
  following happens: First, the \meta{before actions} are
  executed. They can be used, for example, to setup default values for
  keys. Next, |\setkeys{pgfpagesuselayoutoption}|\marg{options} is
  executed. Finally, the \meta{after actions} are executed.

  Here is an example:
\begin{codeexample}[code only]
\pgfpagesdeclarelayout{resize to}
{
  \def\pgfpageoptionborder{0pt}
}
{
  \pgfpagesphysicalpageoptions
  {%
    logical pages=1,%
    physical height=\pgfpageoptionheight,%
    physical width=\pgfpageoptionwidth%
  }
  \pgfpageslogicalpageoptions{1}
  {%
    resized width=\pgfphysicalwidth,%
    resized height=\pgfphysicalheight,%
    border shrink=\pgfpageoptionborder,%
    center=\pgfpoint{.5\pgfphysicalwidth}{.5\pgfphysicalheight}%
  }%
}
\end{codeexample}
\end{command}




\subsection{Defining a Layout}

If none of the predefined layouts meets your problem or if you wish to
modify them, you can create layouts from scratch. This section
explains how this is done.

Basically, |pgfpages| hooks into \TeX's |\shipout| function. This
function is called whenever \TeX\ has completed typesetting a page and
wishes to send this page to the |.dvi| or |.pdf| file. The |pgfpages|
package redefines this command. Instead of sending the page to the output
file, |pgfpages| stores it in an internal box and then acts as if the
page had been output. When \TeX\ tries to output the next page using
|\shipout|, this call is once more intercepted and the page is stored
in another box. These boxes are called \emph{logical pages}.

At some point, enough logical pages have been accumulated such that a
\emph{physical page} can be output. When this happens, |pgfpages|
possibly scales, rotates, and translates the logical pages (and
possibly even does further modifications) and then puts them at
certain positions of the \emph{physical} page. Once this page is fully
assembled, the ``real'' or ``original'' |\shipout| is called to
send the physical page to the output file.

In reality, things are slightly more complicated. First, once a
physical page has been shipped out, the logical pages are usually
voided, but this need not be the case. Instead, it is possible that
certain logical page just retain their contents after the physical
page has been shipped out and these pages need not be filled once more
before a physical shipout can occur. However, the contents of these
logical pages can still be changed using special commands. It is also
possible that after a shipout certain logical pages are filled with
the contents of \emph{other} logical pages.

A \emph{layout} defines for each logical page where it will go on the
physical page and which further modifications should be done. The
following two commands are used to define the layout:

\begin{command}{\pgfpagesphysicalpageoptions\marg{options}}
  This command sets the characteristic of the ``physical'' page. For
  example, it is used to specify how many logical pages there are and
  how many logical pages must be accumulated before a physical page is
  shipped out. How each individual logical page is typeset is
  specified using the command |\pgfpageslogicalpageoptions|, described
  later.

  \example A layout for putting two portrait pages on a single
  landscape page:
\begin{codeexample}[code only]
\pgfpagesphysicalpageoptions
{%
  logical pages=2,%
  physical height=\paperwidth,%
  physical width=\paperheight,%
}

\pgfpageslogicalpageoptions{1}
{%
  resized width=.5\pgfphysicalwidth,%
  resized height=\pgfphysicalheight,%
  center=\pgfpoint{.25\pgfphysicalwidth}{.5\pgfphysicalheight}%
}%
\pgfpageslogicalpageoptions{2}
{%
  resized width=.5\pgfphysicalwidth,%
  resized height=\pgfphysicalheight,%
  center=\pgfpoint{.75\pgfphysicalwidth}{.5\pgfphysicalheight}%
}%
\end{codeexample}

  The following \meta{options} may be set:
  \begin{itemize}
    \item \declare{|logical pages=|\meta{logical pages}} specified how many
    logical pages there are, in total. These are numbered 1 to
    \meta{logical pages}.
    \item \declare{|first logical shipout=|\meta{first}}. See the the
      next option. By default, \meta{first} is 1.
    \item \declare{|last logical shipout=|\meta{last}}. Together
    with the previous option, these two options define an interval of
    pages inside the range 1 to \meta{logical pages}. Only this range
    is used to store the pages that are shipped out by \TeX. This
    means that after a physical shipout has just occured (or at the
    beginning), the first time \TeX\ wishes to perform a shipout, the
    page to be shipped out is stored in logical page \meta{first}. The
    next time \TeX\ performs a shipout, the page is stored in logical
    page $\meta{first} +1$ and so on, until the logical page
    \meta{last} is also filled. Once this happens, a physical shipout
    occurs and the process starts once more.

    Note that logical pages that lie outside the interval between
    \meta{first} and \meta{last} are filled only indirectly or when
    special commands are used.

    By default, \meta{last} equals \meta{logical pages}.
  \item \declare{|current logical shipout=|\meta{current}} changes
    an internal counter such that \TeX's next logical shipout will be
    stored in logical page \meta{current}.

    This option can be used to ``warp'' the logical page filling
    mechanism to a certain page. You can both skip logical pages and
    overwrite already filled logical pages. After the logical page
    \meta{current} has been filled, the internal counter is
    incremented normally as if the logical page \meta{current} had
    been ``reached'' normally. If you specify a \meta{current} larger
    to \meta{last}, a physical shipout will occur after the logical
    page \meta{current} has been filled.
  \item
    \declare{|physical height=|\meta{height}}
    specifies the height of the physical pages. This height is
    typically different from the normal  |\paperheight|, which is used
    by \TeX\ for its typesetting and page breaking purposes.
  \item
    \declare{|physical width=|\meta{width}}
    specifies the physical width.
  \end{itemize}
\end{command}


\begin{command}{\pgfpageslogicalpageoptions\marg{logical page number}\marg{options}}
  This command is used to specify where the logical page number
  \meta{logical page number} will be placed on the physical page. In
  addition, this command can be used to install additional ``code'' to
  be executed when this page is put on the physical page.

  The number \meta{logical page number} should be between 1 and
  \meta{logical pages}, which has previously been installed using the
  |\pgfpagesphysicalpageoptions| command.

  The following \meta{options} may be given:
  \begin{itemize}
  \item
    \declare{|center=|\meta{pgf point}}
    specifies the center of the logical page inside the physical page
    as a \pgfname-point. The origin of the coordinate system of the
    physical page is at the \emph{lower} left corner.

\begin{codeexample}[code only]
\pgfpageslogicalpageoptions{1}
{% center logical page on middle of left side
  center=\pgfpoint{.25\pgfphysicalwidth}{.5\pgfphysicalheight}%
  resized width=.5\pgfphysicalwidth,%
  resized height=\pgfphysicalheight,%
}
\end{codeexample}

  \item
    \declare{|resized width=|\meta{size}}
    specifies the width that the logical page should have \emph{at
    most} on the physical page. To achieve this width, the pages is
    scaled down appropriately \emph{or more}. The ``or more'' part
    can happen if the |resize height| option is also used. In this
    case, the scaling is chosen such that both the specified height
    and width are met. The aspect ratio of a logical page is not
    modified.
  \item
    \declare{|resized height=|\meta{height}}
    specifies the maximum height of the logical page.
  \item
    \declare{|original width=|\meta{width}}
    specifies the width the \TeX\ ``thinks'' that the logical page
    has. This width is |\paperwidth| at the point of invocation, by
    default. Note that setting this width to something different from
    |\paperwidth| does \emph{not} change the |\pagewidth| during
    \TeX's typesetting. You have to do that yourself.

    You need this option only for special logical pages that have
    a height or width different from the normal one and for which you
    will (later on) set these sizes yourself.
  \item
    \declare{|original height=|\meta{height}}
    works like |original width|.
  \item
    \declare{|scale=|\meta{factor}}
    scales the page by at least the given \meta{factor}. A
    \meta{factor} of |0.5| will half the size of the page, a factor or
    |2| will double the size. ``At least'' means that if options like
    |resize height| are given and if the scaling required to meet that
    option is less than \meta{factor}, that other scaling is used
    instead. 
  \item
    \declare{|xscale=|\meta{factor}}
    scales the logical page along the $x$-axis by the given
    \meta{factor}. This scaling is done independently of any other
    scaling. Mostly, this option is useful for a factor of |-1|, which
    flips the page along the $y$-axis. The aspect ratio is not kept.
  \item
    \declare{|yscale=|\meta{factor}}
    works like |xscale|, only for the $y$-axis.
  \item
    \declare{|rotation=|\meta{degree}}
    rotates the page by \meta{degree} around its center. Use a degree
    of |90| or |-90| to go from portrait to landscape and back. The
    rotation need not be a multiple of |90|.
  \item
    \declare{|copy from=|\meta{logical page number}}.
    Normally, after a physical shipout has occured, all logical pages
    are voided in a loop. However, if this option is given, the
    current logical page is filled with the contents of the old
    logical page number \meta{logical page number}.

    \example Have logical page 2 retain its contents:
\begin{codeexample}[code only]
\pgfpageslogicalpageoptions{2}{copy from=2}
\end{codeexample}

    \example Let logical page 2 show what logical page 1 showed on the
    just-shipped-out physical page:
\begin{codeexample}[code only]
\pgfpageslogicalpageoptions{2}{copy from=1}
\end{codeexample}
  \item
    \declare{|border shrink|=\meta{size}}
    specifies an addition reduction of the size to which the page is
    page is scaled down.
  \item
    \declare{|border code|=\meta{code}}.
    When this option is given, the \meta{code} is executed before the
    page box is inserted with a path preinstalled that is a rectangle
    around the current logical page. Thus, setting \meta{code} to
    |\pgfstroke| draws a rectangle around the logical page. Setting
    \meta{code} to |\pgfsetlinewidth{3pt}\pgfstroke| results in a
    thick (ugly) frame. Adding dashes and filling can result in
    arbitrarily funky and distracting borders.

    You can also call |\pgfdiscardpath| and add your own path
    construction code (for example to paint a rectangle with rounded
    corners). The coordinate system is  setup in such a way that a
    rectangle starting at the origin and having the height and width
    of \TeX-box 0 will result in a rectangle filling exactly the
    logical page currently being put on the physical page. The logical
    page is inserted \emph{after} these commands have been executed.

    \example Add a rectangle around the page:
\begin{codeexample}[code only]
\pgfpageslogicalpageoptions{1}{border code=\pgfstroke}
\end{codeexample}
  \item
    \declare{|corner width|=\meta{size}}
    adds black ``rounded corners'' to the page. See the description of
    the predefined layout |rounded corners| on
    page~\pageref{layout-rounded-corners}. 
  \end{itemize}
\end{command}




\subsection{Creating Logical Pages}

Logical pages are created whenever a \TeX\ thinks that a page is full
and performs a |\shipout| command. This will cause |pgfpages| to store
the box that was supposed to be shipped out internally until enough
logical pages have been collected such that a physical shipout can
occur.

Normally, whenever a logical shipout occurs that current page is
stored in logical page number \meta{current logical page}. This
counter is then incremented, until it is larger than \meta{last
  logical shipout}. You can, however, directly change the value of
\meta{current logical page} by calling |\pgfpagesphysicalpageoptions|.

Another way to set the contents of a logical page is to use the
following command:

\begin{command}{\pgfpagesshipoutlogicalpage\marg{number}\meta{box}}
  This command sets to logical page \meta{number} to \meta{box}. The
  \meta{box} should be the code of a \TeX\ box command. This command
  does not influence the counter \meta{current logical page} and does
  not cause a physical shipout.

\begin{codeexample}[code only]
\pgfpagesshipoutlogicalpage{0}\vbox{Hi!}
\end{codeexample}

  This command can be used to set the contents of logical pages that
  are normally not filled.
\end{command}

The final way of setting a logical page is using the following
command: 

\begin{command}{\pgfpagescurrentpagewillbelogicalpage\marg{number}}
  When the current \TeX\ page has been typeset, it will be become the given
  logical page \meta{number}. This command ``interrupts'' the normal
  order of logical pages, that is, it behaves like the previous
  command and does not update the  \meta{current logical page}
  counter. 

\begin{codeexample}[code only]
\pgfpagesuselayout{two screens with optional second}
...
Text for main page.
\clearpage

\pgfpagescurrentpagewillbelogicalpage{2}
Text that goes to second page
\clearpage

Text for main page.
\end{codeexample}
\end{command}

%%% Local Variables: 
%%% mode: latex
%%% TeX-master: "pgfmanual"
%%% End: 

% Copyright 2003 by Till Tantau <tantau@cs.tu-berlin.de>.
%
% This program can be redistributed and/or modified under the terms
% of the LaTeX Project Public License Distributed from CTAN
% archives in directory macros/latex/base/lppl.txt.

\section{Extended Color Support}

This section documents the package \texttt{xxcolor}, which is
currently distributed as part of \pgfname. This package extends the
\texttt{xcolor} package, written by Uwe Kern, which in turn extends
the \texttt{color} package. I hope that the commands in
\texttt{xxcolor} will some day migrate to \texttt{xcolor}, such that
this package becomes superfluous.

The main aim of the \texttt{xxcolor} package is to provide an
environment inside which all colors are ``washed out'' or ``dimmed.''
This is useful in numerous situations and must typically be achieved
in a roundabout manner if such an environment is not available.

\begin{environment}{{colormixin}\marg{mix-in specification}}
  The mix-in specification is applied to all colors inside
  the environment. At the beginning of the environment, the mix-in is
  applied to the current color, i.\,e., the color that was in effect
  before the environment started. A mix-in specification is a number
  between 0 and 100 followed by an exclamation mark and a color
  name. When a |\color| command is 
  encountered inside a mix-in environment, the number states what
  percentage of the desired color should be used. The rest is
  ``filled up'' with the color given in the mix-in
  specification. Thus, a mix-in specification like |90!blue|
  will mix in 10\% of blue into everything, whereas |25!white| will
  make everything nearly white.

\begin{codeexample}[width=4cm]
\begin{minipage}{3.5cm}\raggedright
\color{red}Red text,%
\begin{colormixin}{25!white}
  washed-out red text,
  \color{blue} washed-out blue text,
  \begin{colormixin}{25!black}
    dark washed-out blue text,
    \color{green} dark washed-out green text,%
  \end{colormixin}
  back to washed-out blue text,%
\end{colormixin}
and back to red.
\end{minipage}%
\end{codeexample}
\end{environment}

Note that the environment only changes colors that have been installed
using the standard \LaTeX\ |\color| command. In particular,
the colors in images are not changed. There is, however, some support
offered by the commands |\pgfuseimage| and
|\pgfuseshading|. If the first command is invoked 
inside a |colormixin| environment with the parameter, say,
|50!black| on an image with the name |foo|, the command
will first check whether there is also a defined image with the name
|foo.!50!black|. If so, this image is used instead. This allows
you to provide a different image for this case. If you nest
|colormixin| environments, the different mix-ins are all appended. For
example, inside the inner environment of 
the above example, |\pgfuseimage{foo}| would first check whether
there exists an image named |foo.!50!white!25!black|.

\begin{command}{\colorcurrentmixin}
  Expands to the current accumulated mix-in. Each nesting of a
  |colormixin| adds a mix-in to this list.
\begin{codeexample}[]
\begin{minipage}{\linewidth-6pt}\raggedright
\begin{colormixin}{75!white}
  \colorcurrentmixin\ should be ``!75!white''\par
  \begin{colormixin}{75!black}
    \colorcurrentmixin\ should be ``!75!black!75!white''\par
    \begin{colormixin}{50!white}
      \colorcurrentmixin\ should be ``!50!white!75!black!75!white''\par
    \end{colormixin}
  \end{colormixin}
\end{colormixin}
\end{minipage}
\end{codeexample}
\end{command}








\part{The Basic Layer}

\vskip1cm
\begin{codeexample}[graphic=white]
\begin{tikzpicture}
  \draw[gray,very thin] (-1.9,-1.9) grid (2.9,3.9)
          [step=0.25cm] (-1,-1) grid (1,1);
  \draw[blue] (1,-2.1) -- (1,4.1); % asymptote
                
  \draw[->] (-2,0) -- (3,0) node[right] {$x(t)$};
  \draw[->] (0,-2) -- (0,4) node[above] {$y(t)$};

  \foreach \pos in {-1,2}
    \draw[shift={(\pos,0)}] (0pt,2pt) -- (0pt,-2pt) node[below] {$\pos$};

  \foreach \pos in {-1,1,2,3}
    \draw[shift={(0,\pos)}] (2pt,0pt) -- (-2pt,0pt) node[left] {$\pos$};

  \fill (0,0) circle (0.064cm);
  \draw[thick,parametric,domain=0.4:1.5,samples=200]
    % The plot is reparameterised such that there are more samples
    % near the center.
    plot[id=asymptotic-example] function{(t*t*t)*sin(1/(t*t*t)),(t*t*t)*cos(1/(t*t*t))}
    node[right] {$\bigl(x(t),y(t)\bigr) = (t\sin \frac{1}{t}, t\cos \frac{1}{t})$};

  \fill[red] (0.63662,0) circle (2pt)
    node [below right,fill=white,yshift=-4pt] {$(\frac{2}{\pi},0)$};
\end{tikzpicture}
\end{codeexample}


% Copyright 2003 by Till Tantau <tantau@cs.tu-berlin.de>.
%
% This program can be redistributed and/or modified under the terms
% of the LaTeX Project Public License Distributed from CTAN
% archives in directory macros/latex/base/lppl.txt.


\section{Design Principles}

This section describes the basic layer of \pgfname. This layer is
build on top of the system layer. Whereas the system layer just
provides the absolute minimum for drawing graphics, the basic
layer provides numerous commands that make it possible to create
sophisticated graphics easily and also quickly.

The basic layer does not provide a convenient syntax for describing
graphics, which is left to frontends like \tikzname. For this reason, the
basic layer is typically used only by ``other programs.'' For example,
the \textsc{beamer} package uses the basic layer extensively, but does
not need a convenient input syntax. Rather, speed and flexibility are
needed when \textsc{beamer} creates graphics.

The following basic design principles underlie the basic layer:
\begin{enumerate}
\item Structuring into a core and several optional packages.
\item Consistently named \TeX\ macros for all graphics commands.
\item Path-centered description of graphics.
\item Coordinate transformation system.
\end{enumerate}



\subsection{Core and Optional Packages}

The basic layer consists of a \emph{core package}, called |pgfcore|,
which provides the most basic commands, and several optional package
like |pgfbaseshade| that offer more special-purpose commands.

You can include the core by saying |\usepackage{pgfcore}| or, as a
plain \TeX\ user, |\input pgfcore.tex|.

The following optional packages are provided by the basic layer:
\begin{itemize}
\item
  |pgfbaseplot| provides commands for plotting functions.
\item
  |pgfbaseshapes| provides commands for drawing shapes and
  nodes.   
\item
  |pgfbaseimage| provides commands for including external
  images. The |graphicx| package does a much better job at this than
  the |pgfbaseimage| package does, so you should normally use
  |\includegraphics| and not |\pgfimage|. However, in some situations
  (like when masking is needed or when plain \TeX\ is used) this
  package is needed.
\end{itemize}

If you say |\usepackage{pgf}| or |\input pgf.tex|, all of the optional
packages are loaded (as well as the core and the system layer).


\subsection{Communicating with the Basic Layer via Macros}

In order to ``communicate'' with the basic layer you use long
sequences of commands that start with |\pgf|. You are only allowed to
give these commands inside a |{pgfpicture}| environment. (Note that
|{tikzpicture}| opens a |{pgfpicture}| internally, so you can freely
mix \pgfname\ commands and \tikzname\ commands inside a
|{tikzpicture}|.) It is possible to ``do other things'' between the
commands. For example, you might use one command to move to a certain
point, then have a complicated computation of the next point, and then
move there. 

\begin{codeexample}[]
\newdimen\myypos
\begin{pgfpicture}
  \pgfpathmoveto{\pgfpoint{0cm}{\myypos}}
  \pgfpathlineto{\pgfpoint{1cm}{\myypos}}
  \advance \myypos by 1cm
  \pgfpathlineto{\pgfpoint{1cm}{\myypos}}
  \pgfpathclose
  \pgfusepath{stroke}
\end{pgfpicture}
\end{codeexample}

The following naming conventions are used in the basic layer:

\begin{enumerate}
\item
  All commands and environments start with |pgf|.
\item
  All commands that specify a point (a coordinate) start with |\pgfpoint|.
\item
  All commands that extend the current path start with |\pgfpath|.
\item
  All commands that set/change a graphics parameter start with |\pgfset|.
\item
  All commands that use a previously declared object (like a path,
  image or shading) start with |\pgfuse|.
\item
  All commands having to do with coordinate transformations start with
  |\pgftransform|. 
\item
  All commands having to do with arrow tips start with |\pgfarrows|.
\item
  All commands for ``quickly'' extending or drawing a path start with
  |\pgfpathq| or |\pgfusepathq|.
\end{enumerate}


\subsection{Path-Centered Approach}

In \pgfname\ the most important entity is the \emph{path}. All
graphics are composed of numerous paths that can be stroked,
filled, shaded, or clipped against. Paths can be closed or open, they
can self-intersect and consist of unconnected parts.

Paths are first \emph{constructed} and then \emph{used}. In order to
construct a path, you can use commands starting with |\pgfpath|. Each
time such a command is called, the current path is extended in some
way.

Once a path has been completely constructed, you can use it using the
command |\pgfusepath|. Depending on the parameters given to this
command, the path will be stroked (drawn) or filled or subsequent
drawings will be clipped against this path.




\subsection{Coordinate Versus Canvas Transformations}

\label{section-design-transformations}

\pgfname\ provides two transformation systems: \pgfname's own
\emph{coordinate} transformation matrix and \pdf's or PostScript's
\emph{canvas} transformation matrix. These two systems are quite
different. Whereas a scaling by a factor of, say, $2$ of the canvas
causes \emph{everything} to be scaled by this factor (including
the thickness of lines and text), a scaling of two in the coordinate 
system causes only the \emph{coordinates} to be scaled, but not the
line width nor text.

By default, all transformations only apply to the coordinate
transformation system. However, using the command |\pgflowlevel|
it is possible to apply a transformation to the canvas.

Coordinate transformations are often preferable over canvas
transformations. Text and lines that are transformed using canvas 
transformations suffer from differing sizes and lines whose thickness 
differs depending on whether the line is horizontal or vertical. To
appreciate the difference, consider the following two ``circles'' both
of which have been scaled in the $x$-direction by a factor of $3$ and
by a factor of $0.5$ in the $y$-direction. The left circle uses a
canvas transformation, the right uses \pgfname's coordinate
transformation (some viewers will render the left graphic incorrectly
since they do no apply the low-level transformation the way they
should):  

\begin{tikzpicture}[line width=5pt]
  \useasboundingbox (-1.5,-1) rectangle (6,1);
  
  \begin{scope}
    \pgflowlevel{\pgftransformxscale{3}}
    \pgflowlevel{\pgftransformyscale{.5}}

    \draw (0,0) circle (0.5cm);
    \draw (.55cm,0pt) node[right] {canvas};
  \end{scope}
  \begin{scope}[xshift=9cm,xscale=3,yscale=.5]
    \draw (0,0) circle (0.5cm);
    \draw (.55cm,0pt) node[right] {coordinate};
  \end{scope}
\end{tikzpicture}


% Copyright 2003 by Till Tantau <tantau@cs.tu-berlin.de>.
%
% This program can be redistributed and/or modified under the terms
% of the LaTeX Project Public License Distributed from CTAN
% archives in directory macros/latex/base/lppl.txt.


\section[Hierarchical Structures: Package, Environments, Scopes, and Text]
{Hierarchical Structures:\\
  Package, Environments, Scopes, and Text}


\subsection{Overview}

\pgfname\ uses two kinds of hierarchical structuring: First, the
package itself is structured hierarchically, consisting of different
packages that build on top of each other. Second, \pgfname\ allows you
to structure your graphics hierarchically using environments and scopes.

\subsubsection{The  Hierarchical Structure of the Package}

The \pgfname\ system consists of several layers:

\begin{description}
\item[System layer.]
  The lowest layer is called the \emph{system layer}, though it might
  also be called ``driver layer'' or perhaps ``backend layer.'' Its
  job is to provide an abstraction of the details of which driver
  is used to transform the |.dvi| file. The system layer is
  implemented by the package |pgfsys|, which will load appropriate
  driver files as needed.

  The system layer is documented in Part~\ref{part-system}.
\item[Basic layer.]
  The basic layer is loaded by the package |pgf|. Some
  applications do not need all of the functionality of the basic
  layer, so it is possible to load only the |pgfcore| and some other
  packages starting with |pgfbase|.

  The basic layer is documented in the present part.
\item[Frontend layer.]
  The frontend layer is not loaded by a single packages. Rather,
  different packages, like \tikzname\ or \textsc{pgfpict2e}, are
  different frontends to the basic layer.

  The \tikzname\ frontend is documented in Part~\ref{part-tikz}.
\end{description}

Each layer will automatically the necessary files of the layers below
it.

In addition to the packages of these layers, there are also some
library packages. These packages provide additional definitions of
things like new arrow tips or new plot handlers.

The library packages are documented in Part~\ref{part-libraries}.




\subsubsection{The Hierarchical Structure of Graphics}

Graphics in \pgfname\ are typically structured
hierarchically. Hierarchical structuring can be used to identify
groups of graphical elements that are to be treated ``in the same
way.'' For example, you might group together a number of paths, all of
which are to be drawn in red. Then, when you decide later on that you
like them to be drawn in, say, blue, all you have to do is to change
the color once.

The general mechanism underlying hierarchical structuring is known as
\emph{scoping} in computer science. The idea is that all changes to
the general ``state'' of the graphic that are done inside a scope are
local to that scope. So, if you change the color inside a scope, this
does not affect the color used outside the scope. Likewise, when you
change the line width in a scope, the line width outside is not
changed, and so on.

There are different ways of starting and ending scopes of graphic
parameters. Unfortunately, these scopes are sometimes ``in conflict''
with each other and it is sometimes not immediately clear which scopes
apply. In essence, the following scoping mechanisms are available:

\begin{enumerate}
\item
  The ``outermost'' scope supported by \pgfname\ is the |{pgfpicture}|
  environment. All changes to the graphic state done inside a
  |{pgfpicture}| are local to that picture.

  In general, it is \emph{not} possible to set graphic parameters
  globally outside any |{pgfpicture}| environments. Thus, you can
  \emph{not} say |\pgfsetlinewidth{1pt}| at the beginning of your
  document to have a default line width of one point. Rather, you have
  to (re)set all graphic parameters inside each |{pgfpicture}|. (If
  this is too bothersome, try defining some macro that does the job
  for you.)
\item
  Inside a |{pgfpicture}| you can use a |{pgfscope}| environment to
  keep changes of the graphic state local to that environment.

  The effect of commands that change the graphic state are local to
  the current |{pgfscope}| but not always to the current \TeX\
  group. Thus, if you open a \TeX\ group (some text in curly braces)
  inside a |{pgfscope}|, and if you change, for example, the dash
  pattern, the effect of this changed dash pattern will persist till
  the end of the |{pgfscope}|.

  Unfortunately, this is not always the case. \emph{Some} graphic
  parameters only persist till the end of the current \TeX\ group. For
  example, when you use |\pgfsetarrows| to set the arrow tip kind
  inside a \TeX\ group, the effect lasts only till the end of the
  current \TeX\ group.
\item
  Some graphic parameters are not scoped by |{pgfscope}| but
  ``already'' by \TeX\ groups. For example, the effect of coordinate
  transformation commands is always local to the current \TeX\
  group.

  Since every |{pgfscope}| automatically creates a \TeX\ group, all
  graphic parameters that are local to the current \TeX\ group are
  also local to the current |{pgfscope}|.
\item
  Some graphic parameters can only be scoped using \TeX\ groups, since
  in some situations it is not possible to introduce a
  |{pgfscope}|. For example, a path always has to be completely
  constructed and used in the same |{pgfscope}|. However, we might
  wish to have different coordinate transformations apply to different
  points on the path. In this case, we can use \TeX\ groups to keep
  the effect local, but we could not use |{pgfscope}|.  
\item
  The |\pgftext| command can be used to create a scope in which \TeX\
  ``escapes back'' to normal \TeX\ mode. The text passed to the
  |\pgftext| is ``heavily guarded'' against having any effect on the
  scope in which it is used. For example, it is possibly to use
  another  |{pgfpicture}| environment inside the argument of
  |\pgftext|. 
\end{enumerate}


Most of the complications can be avoided if you stick to the following
rules:

\begin{itemize}
\item
  Give graphic commands only inside |{pgfpicture}| environments.
\item
  Use |{pgfscope}| to structure graphics.
\item
  Do not use \TeX\ groups inside graphics, \emph{except} for keeping
  the effect of coordinate transformations local.
\end{itemize}



\subsection{The Hierarchical Structure of the Package}

Before we come to the structuring commands provided by \pgfname\ to
structure your graphics, let us first have a look at the structure of
the package itself.

\subsubsection{The Main Package}

To use \pgfname, include the following package:

\begin{package}{pgf}
  This package loads the complete ``basic layer'' of \pgfname. That
  is, it will load all of the commands described in the current part
  of this manual, but it will not load frontends like \tikzname.

  In detail, this package will load the following packages, each of
  which can also be loaded individually:
  \begin{itemize}
  \item
    |pgfsys|, which is the lowest layer of \pgfname\ and which is
    always needed. This file will read |pgf.cfg| to discern which
    driver is to be used. See Section~\ref{section-pgfsys} for
    details. 
  \item
    |pgfcore|, which is the central core of \pgfname\ and which is
    always needed unless you intend to write a new basic layer from
    scratch.
  \item
    |pgfbaseimage|, which provides commands for declaring and
    using images, An example is |\pgfuseimage|.
  \item
    |pgfbaseshapes|, which provides commands for declaring and using
    shapes. An example is |\pgfshape|.
  \item
    |pgfbaseplot|, which provides commands for plotting functions.    
  \end{itemize}

  Including any of the last three packages will automatically load the
  first two.
\end{package}

In \LaTeX, the package takes two options:
\begin{packageoption}{draft}
  When this option is set, all images will be replaced by empty
  rectangles. This can speedup compilation.
\end{packageoption}
 
\begin{packageoption}{strict}
  This option will suppress loading of a large number of compatibility
  commands. 
\end{packageoption}


\subsubsection{The Core Package}

\begin{package}{pgfcore}
  This package defines all of the basic layer's commands, except for
  the commands defined in the additional packages like
  |pgfbaseplot|. Typically commands defined by the core include
  |\pgfusepath| or   |\pgfpoint|. The core is internally structured
  into several subpackages, but the subpackages cannot be loaded
  individually since they are all ``interrelated.''
\end{package}


\subsubsection{The Optional Basic Layer Packages}

The |pgf| package automatically loads the following packages, but you
can also load them individually (all of them automatically include the
core):

\begin{itemize}
  \item |pgfbaseshapes|
  This package provides commands for drawing nodes and shapes. These
  commands are explained in Section~\ref{section-shapes}.

  \item |pgfbaseplot|
  This package provides commands for plotting function. The
  commands are explained in Section~\ref{section-plots}.

  \item |pgfbaseimage|
  This package provides commands for including (external) images. The 
  commands are explained in Section~\ref{section-images}.
\end{itemize}





\subsection{The Hierarchical Structure of the Graphics}

\subsubsection{The Main Environment}


Most, but not all, commands of the \pgfname\ package must be given
within a |{pgfpicture}| environment. The only commands that (must) be
given outside are commands having to do with including images (like
|\pgfuseimage|) and with inserting complete shadings (like
|\pgfuseshading|). However, just to keep life entertaining, the
|\pgfshadepath| command must be given \emph{inside} a |{pgfpicture}|
environment.

\begin{environment}{{pgfpicture}}
  This environment will insert a \TeX\ box containing the graphic drawn by
  the \meta{environment contents} at the current position. Note that
  \tikzname\ redefines this environment so that it takes an optional
  options argument.

  \medskip
  \textbf{The size of the bounding box.}
  The size of the box is determined in the following
  manner: While \pgfname\ parses the \meta{environment contents}, it
  keeps track of a bounding box for the graphic. Essentially, this
  bounding box is the smallest box that contains all coordinates
  mentioned in the graphics. Some coordinates may be ``mentioned'' by
  \pgfname\ itself; for example, when you add circle to the current
  path, the support points of the curve making up the circle are also
  ``mentioned'' despite the fact that you will not ``see'' them in
  your code.

  Once the \meta{environment contents} has been parsed completely, a
  \TeX\ box is created whose size is the size of the computed bounding
  box and this box is inserted at the current position.

\begin{codeexample}[]
Hello \begin{pgfpicture}
  \pgfpathrectangle{\pgfpointorigin}{\pgfpoint{2ex}{1ex}}
  \pgfusepath{stroke}
\end{pgfpicture} World!
\end{codeexample}

  Sometimes, you may need more fine-grained control over the size of
  the bounding box. For example, the computed bounding box may be too
  large or you intensionally wish the box to be ``too small.'' In
  these cases, you can use the command
  |\pgfusepath{use as bounding box}|, as described in
  Section~\ref{section-using-bb}.


  \medskip
  \textbf{The baseline of the bounding box.}
  When the box containing the graphic is inserted into the normal
  text, the baseline of the graphic is normally at the bottom of the
  graphic. For this reason, the following two sets of code lines have
  the same effect, despite the fact that the second graphic uses
  ``higher'' coordinates than the first:
 
\begin{codeexample}[]
Rectangles \begin{pgfpicture}
  \pgfpathrectangle{\pgfpointorigin}{\pgfpoint{2ex}{1ex}}
  \pgfusepath{stroke}
\end{pgfpicture} and \begin{pgfpicture}
  \pgfpathrectangle{\pgfpoint{0ex}{1ex}}{\pgfpoint{2ex}{1ex}}
  \pgfusepath{stroke}
\end{pgfpicture}.
\end{codeexample}

  You can change the baseline using the |\pgfsetbaseline| command, see
  below. 

\begin{codeexample}[]
Rectangles \begin{pgfpicture}
  \pgfpathrectangle{\pgfpointorigin}{\pgfpoint{2ex}{1ex}}
  \pgfusepath{stroke}
  \pgfsetbaseline{0pt}
\end{pgfpicture} and \begin{pgfpicture}
  \pgfpathrectangle{\pgfpoint{0ex}{1ex}}{\pgfpoint{2ex}{1ex}}
  \pgfusepath{stroke}
  \pgfsetbaseline{0pt}
\end{pgfpicture}.
\end{codeexample}

  \medskip
  \textbf{Including text and images in a picture.}
  You cannot directly include text and images in a picture. Thus, you
  should \emph{not} simply write some text in a |{pgfpicture}| or use
  a command like |\includegraphics| or even |\pgfimage|. In all these
  cases, you need to place the text inside a |\pgftext| command. This
  will ``escape back'' to normal \TeX\ mode, see
  Section~\ref{section-text-command} for details.
\end{environment}

\begin{plainenvironment}{{pgfpicture}}
  The plain \TeX\ version of the environment. Note that in this
  version, also, a \TeX\ group is created around the environment.
\end{plainenvironment}

\makeatletter
\begin{command}{\pgfsetbaseline\marg{dimension}}
  This command specifies a $y$-coordinate of the picture that should
  be used as the baseline of the whole picture. When a \pgfname\
  picture has been typeset completely, \pgfname\ must decide at which
  height the baseline of the picture should lie. Normally, the
  baseline is set to the $y$-coordinate of the bottom of the picture,
  but it is often desirable to use another height.

\begin{codeexample}[]
Text \tikz{\pgfpathcircle{\pgfpointorigin}{1ex}\pgfusepath{stroke}},
     \tikz{\pgfsetbaseline{0pt}
          \pgfpathcircle{\pgfpointorigin}{1ex}\pgfusepath{stroke}},
     \tikz{\pgfsetbaseline{.5ex}
          \pgfpathcircle{\pgfpointorigin}{1ex}\pgfusepath{stroke}},
     \tikz{\pgfsetbaseline{-1ex}
          \pgfpathcircle{\pgfpointorigin}{1ex}\pgfusepath{stroke}}.
\end{codeexample}
\end{command}

\subsubsection{Graphic Scope Environments}

Inside a |{pgfpicture}| environment you can substructure your picture
using the following environment:

\begin{environment}{{pgfscope}}
  All changes to the graphic state done inside this environment are
  local to the environment. The graphic state includes the following:
  \begin{itemize}
  \item
    The line width.
  \item
    The stroke and fill colors.
  \item
    The dash pattern.
  \item
    The line join and cap.
  \item
    The miter limit.
  \item
    The canvas transformation matrix.
  \item
    The clipping path.
  \end{itemize}
  Other parameters may also influence how graphics are rendered, but they
  are \emph{not} part of the graphic state. For example, the arrow tip
  kind is not part of the graphic state and the effect of commands
  setting the arrow tip kind are local to the current \TeX\ group, not
  to the current |{pgfscope}|. However, since |{pgfscope}| starts and
  ends a \TeX\ group automatically, a |{pgfscope}| can be used to
  limit the effect of, say, commands that set the arrow tip kind.

\begin{codeexample}[]
\begin{pgfpicture}
  \begin{pgfscope}
    {
      \pgfsetlinewidth{2pt}
      \pgfpathrectangle{\pgfpointorigin}{\pgfpoint{2ex}{2ex}}
      \pgfusepath{stroke}
    }
    \pgfpathrectangle{\pgfpoint{3ex}{0ex}}{\pgfpoint{2ex}{2ex}}
    \pgfusepath{stroke}
  \end{pgfscope}
  \pgfpathrectangle{\pgfpoint{6ex}{0ex}}{\pgfpoint{2ex}{2ex}}
  \pgfusepath{stroke}
\end{pgfpicture}
\end{codeexample}
  
\begin{codeexample}[]
\begin{pgfpicture}
  \begin{pgfscope}
    {
      \pgfsetarrows{-to}
      \pgfpathmoveto{\pgfpointorigin}\pgfpathlineto{\pgfpoint{2ex}{2ex}}
      \pgfusepath{stroke}
    }
    \pgfpathmoveto{\pgfpoint{3ex}{0ex}}\pgfpathlineto{\pgfpoint{5ex}{2ex}}
    \pgfusepath{stroke}
  \end{pgfscope}
  \pgfpathmoveto{\pgfpoint{6ex}{0ex}}\pgfpathlineto{\pgfpoint{7ex}{2ex}}
  \pgfusepath{stroke}
\end{pgfpicture}
\end{codeexample}

  At the start of the scope, the current path must be empty, that is,
  you cannot open a scope while constructing a path.

  It is usually a good idea \emph{not} to introduce \TeX\ groups
  inside a |{pgfscope}| environment.
\end{environment}

\begin{plainenvironment}{{pgfscope}}
  Plain \TeX\ version of the |{pgfscope}| environment.
\end{plainenvironment}


The following scopes also encapsulate certain properties of the
graphic state. However, they are typically not used directly by the
user.

\begin{environment}{{pgfinterruptpath}}
  This environment can be used to temporarily interrupt the
  construction of the current path. The effect will be that the path
  currently under construction will be ``stored away'' and restored at
  the end of the environment. Inside the environment you can construct
  a new path and do something with it.

  An example application of this environment is the arrow tip
  caching. Suppose you ask \pgfname\ to use a specific arrow tip
  kind. When the arrow tip needs to be rendered for the first time,
  \pgfname\ will ``cache'' the path that makes up the arrow tip. To do
  so, it interrupts the current path construction and then protocols
  the path of the arrow tip. The |{pgfinterruptpath}| environment is
  used to ensure that this does not interfere with the path to which
  the arrow tips should be attached.

  This command does \emph{not} install a |{pgfscope}|. In particular,
  it does not call any |\pgfsys@| commands at all, which would,
  indeed, be dangerous in the middle of a path construction.
\end{environment}

\begin{environment}{{pgfinterruptpicture}}
  This environment can be used to temporarily interrupt a
  |{pgfpicture}|. However, the environment is intended only to be used
  at the beginning and end of a box that is (later) inserted into a
  |{pgfpicture}| using |\pgfqbox|. You cannot use this environment
  directly inside a |{pgfpicture}|.

\begin{codeexample}[]
\begin{pgfpicture}
  \pgfpathmoveto{\pgfpoint{0cm}{0cm}} % In the middle of path, now
  \newbox\mybox
  \setbox\mybox=\hbox{
    \begin{pgfinterruptpicture}
      Sub-\begin{pgfpicture} % a subpicture
        \pgfpathmoveto{\pgfpoint{1cm}{0cm}}
        \pgfpathlineto{\pgfpoint{1cm}{1cm}}
        \pgfusepath{stroke}
      \end{pgfpicture}-picture.
    \end{pgfinterruptpicture}
  }
  \ht\mybox=0pt
  \wd\mybox=0pt
  \dp\mybox=0pt
  \pgfqbox{\box\mybox}%
  \pgfpathlineto{\pgfpoint{0cm}{1cm}}
  \pgfusepath{stroke}
\end{pgfpicture}\hskip3.9cm
\end{codeexample}
\end{environment}


\subsubsection{Inserting Text and Images}

\label{section-text-command}

Often, you may wish to add normal \TeX\ text at a certain point inside
a |{pgfpicture}|. You cannot do so ``directly,'' that is, you cannot
simply write this text inside the |{pgfpicture}| environment. Rather,
you must pass the text as an argument to the |\pgftext| command.

You must \emph{also} use the |\pgftext| command to insert an image or
a shading into a |{pgfpicture}|.

\begin{command}{\pgftext\opt{\oarg{options}}\marg{text}}
  This command will typeset \meta{text} in normal \TeX\ mode and
  insert the resulting box into the |{pgfpicture}|. The bounding box
  of the graphic will be updated so that all of the text box is
  inside. Be default, the text box is centered at the origin, but this
  can be changed either by giving appropriate \meta{options} or by
  applying an appropriate coordinate transformation beforehand.

  The \meta{text} may contain verbatim text. (In other words, the
  \meta{text} ``argument'' is not a normal argument, but is put in a
  box and some |\aftergroup| hackery is used to find the end of the
  box.)

  \pgfname's current (high-level) coordinate transformation is
  synchronized with the canvas transformation matrix temporarily
  when the text box is inserted. The effect is that if there is
  currently a high-level rotation of, say, 30 degrees, the \meta{text}
  will also be rotated by thirty degrees. If you do not want this
  effect, you have to (possibly temporarily) reset the high-level
  transformation matrix.

  The following \meta{options} may be given as conveniences:
  \begin{itemize}
    \itemoption{left}
    causes the text box to be placed such that its left border is on the origin.
\begin{codeexample}[]
\tikz{\draw[help lines] (-1,-.5) grid (1,.5);
     \pgftext[left] {lovely}}
\end{codeexample}
    \itemoption{right}
    causes the text box to be placed such that its right border is on the origin.
\begin{codeexample}[]
\tikz{\draw[help lines] (-1,-.5) grid (1,.5);
     \pgftext[right] {lovely}}
\end{codeexample}
    \itemoption{top}
    causes the text box to be placed such that its top is on the
    origin. This option can be used together with the |left| or
    |right| option.
\begin{codeexample}[]
\tikz{\draw[help lines] (-1,-.5) grid (1,.5);
     \pgftext[top] {lovely}}
\end{codeexample}
\begin{codeexample}[]
\tikz{\draw[help lines] (-1,-.5) grid (1,.5);
     \pgftext[top,right] {lovely}}
\end{codeexample}
    \itemoption{bottom}
    causes the text box to be placed such that its bottom is on the
    origin.
\begin{codeexample}[]
\tikz{\draw[help lines] (-1,-.5) grid (1,.5);
     \pgftext[bottom] {lovely}}
\end{codeexample}
\begin{codeexample}[]
\tikz{\draw[help lines] (-1,-.5) grid (1,.5);
     \pgftext[bottom,right] {lovely}}
\end{codeexample}
    \itemoption{base}
    causes the text box to be placed such that its baseline is on the
    origin.
\begin{codeexample}[]
\tikz{\draw[help lines] (-1,-.5) grid (1,.5);
     \pgftext[base] {lovely}}
\end{codeexample}
\begin{codeexample}[]
\tikz{\draw[help lines] (-1,-.5) grid (1,.5);
     \pgftext[base,right] {lovely}}
\end{codeexample}
    \itemoption{at}|=|\meta{point}
      Translates the origin (that is, the point where the text is
      shown) to \meta{point}. 
\begin{codeexample}[]
\tikz{\draw[help lines] (-1,-.5) grid (1,.5);
     \pgftext[base,at={\pgfpoint{1cm}{0cm}}] {lovely}}
\end{codeexample}
    \itemoption{x}|=|\meta{dimension}
      Translates the origin by \meta{dimension} along the $x$-axis.
\begin{codeexample}[]
\tikz{\draw[help lines] (-1,-.5) grid (1,.5);
     \pgftext[base,x=1cm,y=-0.5cm] {lovely}}
\end{codeexample}
  \itemoption{y}|=|\meta{dimension}
    works like the |x| option.
  \itemoption{rotate}|=|\meta{degree}
    Rotates the coordinate system by \meta{degree}. This will also
    rotate the text box.
\begin{codeexample}[]
\tikz{\draw[help lines] (-1,-.5) grid (1,.5);
     \pgftext[base,x=1cm,y=-0.5cm,rotate=30] {lovely}}
\end{codeexample}
  \end{itemize}
  
\end{command}



% Copyright 2003 by Till Tantau <tantau@cs.tu-berlin.de>.
%
% This program can be redistributed and/or modified under the terms
% of the LaTeX Project Public License Distributed from CTAN
% archives in directory macros/latex/base/lppl.txt.


\section{Specifying Coordinates}

\label{section-points}

\subsection{Overview}

Most \pgfname\ commands expect you to provide the coordinates of a
\emph{point} or \emph{coordinate} inside your picture. Points are
always ``local'' to your picture, that is, they never refer to an
absolute position on the page, but to a position inside the current
|{pgfpicture}| environment.

To specify a coordinate, you can use commands that start with
|\pgfpoint|. The simplest of these commands is just |\pgfpoint| and it
takes two arguments: An $x$-dimension and a $y$-dimension. These
dimensions are given as \TeX-dimensions.

\subsection{Basic Coordinate Commands}

The following commands are the most basic  for specifying a
coordinate.

\begin{command}{\pgfpoint\marg{x coordinate}\marg{y coordinate}}
  Yields a point location. The coordinates are given as \TeX\
  dimensions.

\begin{codeexample}[]
\begin{tikzpicture}
  \draw[help lines] (0,0) grid (3,2);

  \pgfpathcircle{\pgfpoint{1cm}{1cm}} {2pt}
  \pgfpathcircle{\pgfpoint{2cm}{5pt}} {2pt}
  \pgfpathcircle{\pgfpoint{0pt}{.5in}}{2pt}
  \pgfusepath{fill}
\end{tikzpicture}   
\end{codeexample}
\end{command}

\begin{command}{\pgfpointorigin}
  Yields the origin. Same as |\pgfpoint{0pt}{0pt}|.
\end{command}

\begin{command}{\pgfpointpolar\marg{degree}\marg{radius}}
  Yields a point location given in polar coordinates. You can specify
  the angle only in degrees, radians are not supported.
\begin{codeexample}[]
\begin{tikzpicture}
  \draw[help lines] (0,0) grid (3,2);

  \foreach \angle in {0,10,...,90}
    {\pgfpathcircle{\pgfpointpolar{\angle}{1cm}}{2pt}}
  \pgfusepath{fill}
\end{tikzpicture}   
\end{codeexample}
\end{command}



\subsection{Coordinates in the xy- and xyz-Coordinate Systems}

Coordinates can also be specified as multiples of an $x$-vector and a
$y$-vector. Normally, the $x$-vector points one centimeter in the
$x$-direction and the $y$-vector points one centimeter in the
$y$-direction, but using the commands |\pgfsetxvec| and
|\pgfsetyvec| they can be changed. Note that the $x$- and
$y$-vector do not necessarily point ``horizontally'' and
``vertically.''

It is also possible to specify a point as a multiple of three vectors,
the $x$-, $y$-, and $z$-vector. This is useful for creating simple
three dimensional graphics.

\begin{command}{\pgfpointxy\marg{$s_x$}\marg{$s_y$}}
  Yields a point that is situated at $s_x$ times the
  $x$-vector plus $s_y$ times the $y$-vector.
\begin{codeexample}[]
\begin{tikzpicture}
  \draw[help lines] (0,0) grid (3,2);
  \pgfpathmoveto{\pgfpointxy{1}{0}}
  \pgfpathlineto{\pgfpointxy{2}{2}}
  \pgfusepath{stroke}
\end{tikzpicture}   
\end{codeexample}
\end{command}

\begin{command}{\pgfpointxyz\marg{$s_x$}\marg{$s_y$}\marg{$s_z$}}
  Yields a point that is situated at $s_x$ times the
  $x$-vector plus $s_y$ times the $y$-vector plus  $s_z$ times the
  $z$-vector.
\begin{codeexample}[]
\begin{pgfpicture}
  \pgfsetarrowsend{to}
  
  \pgfpathmoveto{\pgfpointorigin}
  \pgfpathlineto{\pgfpointxyz{0}{0}{1}}
  \pgfusepath{stroke}
  \pgfpathmoveto{\pgfpointorigin}
  \pgfpathlineto{\pgfpointxyz{0}{1}{0}}
  \pgfusepath{stroke}
  \pgfpathmoveto{\pgfpointorigin}
  \pgfpathlineto{\pgfpointxyz{1}{0}{0}}
  \pgfusepath{stroke}
\end{pgfpicture}
\end{codeexample}
\end{command}


\begin{command}{\pgfsetxvec\marg{point}}
  Sets that current $x$-vector for usage in the $xyz$-coordinate
  system. 
  \example
\begin{codeexample}[]
\begin{tikzpicture}
  \draw[help lines] (0,0) grid (3,2);
  
  \pgfpathmoveto{\pgfpointxy{1}{0}}
  \pgfpathlineto{\pgfpointxy{2}{2}}
  \pgfusepath{stroke}

  \color{red}
  \pgfsetxvec{\pgfpoint{0.75cm}{0cm}}
  \pgfpathmoveto{\pgfpointxy{1}{0}}
  \pgfpathlineto{\pgfpointxy{2}{2}}
  \pgfusepath{stroke}
\end{tikzpicture}   
\end{codeexample}
\end{command}

\begin{command}{\pgfsetyvec\marg{point}}
  Works like |\pgfsetyvec|.
\end{command}

\begin{command}{\pgfsetzvec\marg{point}}
  Works like |\pgfsetzvec|.
\end{command}




\subsection{Building Coordinates From Other Coordinates}

Many commands allow you to construct a coordinate in terms of other
coordinates.


\subsubsection{Basic Manipulations of Coordiantes}

\begin{command}{\pgfpointadd\marg{$v_1$}\marg{$v_2$}}
  Returns the sum vector $\meta{$v_1$} + \meta{$v_2$}$.
\begin{codeexample}[]
\begin{tikzpicture}
  \draw[help lines] (0,0) grid (3,2);
  \pgfpathcircle{\pgfpointadd{\pgfpoint{1cm}{0cm}}{\pgfpoint{1cm}{1cm}}}{2pt}
  \pgfusepath{fill} 
\end{tikzpicture}
\end{codeexample}
\end{command}

\begin{command}{\pgfpointscale\marg{factor}\marg{coordinate}}
  Returns the vector $\meta{factor}\meta{coordinate}$.
\begin{codeexample}[]
\begin{tikzpicture}
  \draw[help lines] (0,0) grid (3,2);
  \pgfpathcircle{\pgfpointscale{1.5}{\pgfpoint{1cm}{0cm}}}{2pt}
  \pgfusepath{fill} 
\end{tikzpicture}
\end{codeexample}
\end{command}

\begin{command}{\pgfpointdiff\marg{start}\marg{end}}
  Returns the difference vector $\meta{end} - \meta{start}$.
\begin{codeexample}[]
\begin{tikzpicture}
  \draw[help lines] (0,0) grid (3,2);
  \pgfpathcircle{\pgfpointdiff{\pgfpoint{1cm}{0cm}}{\pgfpoint{1cm}{1cm}}}{2pt}
  \pgfusepath{fill} 
\end{tikzpicture}
\end{codeexample}
\end{command}


\begin{command}{\pgfpointnormalised\marg{point}}
  This command returns a normalised version of \meta{point}, that is,
  a vector of length 1pt pointing in the direction of \meta{point}. If
  \meta{point} is the $0$-vector or extremely short, a vector of
  length 1pt pointing upwards is returned.

  This command is \emph{not} implemented by calculating the length of
  the vector, but rather by calculating the angle of the vector and
  then using (something equivalent to) the |\pgfpointpolar|
  command. On the one hand this ensures that the point will really
  have length 1pt, on the other hand it is not guaranteed that the
  vector will \emph{precisely} point in the direction of \meta{point}
  due to the fact that the polar tables are accurate only up to one
  degree. Normally, this is not a problem.
\begin{codeexample}[]
\begin{tikzpicture}
  \draw[help lines] (0,0) grid (3,2);
  \pgfpathcircle{\pgfpoint{2cm}{1cm}}{2pt}
  \pgfpathcircle{\pgfpointscale{20}
    {\pgfpointnormalised{\pgfpoint{2cm}{1cm}}}}{2pt}
  \pgfusepath{fill} 
\end{tikzpicture}
\end{codeexample}  
\end{command}


\subsubsection{Points Travelling along Lines and Curves}

\label{section-pointsattime}

The commands in this section allow you to specify points on a line or
a curve. Imaging a point ``travelling'' along a curve from some point
$p$ to another point $q$. At time $t=0$ the point is at $p$ and at
time $t=1$ it is at $q$ and at time, say, $t=1/2$ it is ``somewhere in
the middle.'' The exact location at time $t=1/2$ will not necessarily
be the ``halfway point,'' that is, the point whose distance on the
curve from $p$ and $q$ is equal. Rather, the exact location will
depend on the ``speed'' at which the point is travelling, which in
turn depends on the lengths of the support vectors in a complicated
manner. If you are interested in the details, please see a good book
on Bezi�r curves.



\begin{command}{\pgfpointlineattime\marg{time $t$}\marg{point $p$}\marg{point $q$}}
  Yields a point that is the $t$th fraction between $p$
  and~$q$, that is, $p + t(q-p)$. For $t=1/2$ this is the middle of
  $p$ and $q$.

\begin{codeexample}[]
\begin{tikzpicture}
  \draw[help lines] (0,0) grid (3,2);
  \pgfpathmoveto{\pgfpointorigin}
  \pgfpathlineto{\pgfpoint{3cm}{2cm}}
  \pgfusepath{stroke}
  \foreach \t in {0,0.25,0.5,0.75,1}
    {\pgftext[at=
      \pgfpointlineattime{\t}{\pgfpointorigin}{\pgfpoint{3cm}{2cm}}]{\t}}
\end{tikzpicture}    
\end{codeexample}
\end{command}

\begin{command}{\pgfpointlineatdistance\marg{distance}\marg{start point}\marg{end point}}
  Yields a point that is located \meta{distance} many units removed
  from the start point in the direction of the end point. In other
  words, this is the point that results if we travel \meta{distance}
  steps from \meta{start point} towards \meta{end point}.
  \example
\begin{codeexample}[]
\begin{tikzpicture}
  \draw[help lines] (0,0) grid (3,2);
  \pgfpathmoveto{\pgfpointorigin}
  \pgfpathlineto{\pgfpoint{3cm}{2cm}}
  \pgfusepath{stroke}
  \foreach \d in {0pt,20pt,40pt,70pt}
    {\pgftext[at=
      \pgfpointlineatdistance{\d}{\pgfpointorigin}{\pgfpoint{3cm}{2cm}}]{\d}}
\end{tikzpicture}    
\end{codeexample}
\end{command}

\begin{command}{\pgfpointcurveattime\marg{time $t$}\marg{point $p$}\marg{point $s_1$}\marg{point $s_2$}\marg{point $q$}}
  Yields a point that is on the Bezi�r curve from $p$ to $q$ with the
  support points $s_1$ and $s_2$. The time $t$ is used to determine
  the location, where $t=0$ yields $p$ and $t=1$ yields $q$.

\begin{codeexample}[]
\begin{tikzpicture}
  \draw[help lines] (0,0) grid (3,2);
  \pgfpathmoveto{\pgfpointorigin}
  \pgfpathcurveto
    {\pgfpoint{0cm}{2cm}}{\pgfpoint{0cm}{2cm}}{\pgfpoint{3cm}{2cm}}
  \pgfusepath{stroke}
  \foreach \t in {0,0.25,0.5,0.75,1}
    {\pgftext[at=\pgfpointcurveattime{\t}{\pgfpointorigin}
                                         {\pgfpoint{0cm}{2cm}}
                                         {\pgfpoint{0cm}{2cm}}
                                         {\pgfpoint{3cm}{2cm}}]{\t}}
\end{tikzpicture}    
\end{codeexample}
\end{command}

\subsubsection{Points on Borders of Objects}

The following commands are useful to specify a point that lies on the
border of special shapes. They are used, for example, by the shape
mechanism to determine border points of shapes.

\begin{command}{\pgfpointborderrectangle\marg{direction point}\marg{corner}}
  This command returns a point that lies on the intersection of a line
  starting at the origin and going towards the point \meta{direction
    point} and a rectangle whose center is in the origin and whose
  upper right corner is at \meta{corner}.

  The \meta{direction point} should have length ``about 1pt,'' but it
  will be normalised automatically. Nevertheless, the ``nearer'' the
  length is to 1pt, the less rounding errors there will be.

\begin{codeexample}[]
\begin{tikzpicture}
  \draw[help lines] (0,0) grid (2,1.5);
  \pgfpathrectanglecorners{\pgfpoint{-1cm}{-1.25cm}}{\pgfpoint{1cm}{1.25cm}}
  \pgfusepath{stroke}

  \pgfpathcircle{\pgfpoint{5pt}{5pt}}{2pt}
  \pgfpathcircle{\pgfpoint{-10pt}{5pt}}{2pt}
  \pgfusepath{fill}
  \color{red}
  \pgfpathcircle{\pgfpointborderrectangle
    {\pgfpoint{5pt}{5pt}}{\pgfpoint{1cm}{1.25cm}}}{2pt}
  \pgfpathcircle{\pgfpointborderrectangle
    {\pgfpoint{-10pt}{5pt}}{\pgfpoint{1cm}{1.25cm}}}{2pt}
  \pgfusepath{fill}
\end{tikzpicture}    
\end{codeexample}
\end{command}


\begin{command}{\pgfpointborderellipse\marg{direction point}\marg{corner}}
  This command works like the corresponding command for rectangles,
  only this time the \meta{corner} is the corner of the bounding
  rectangle of an ellipse.

\begin{codeexample}[]
\begin{tikzpicture}
  \draw[help lines] (0,0) grid (2,1.5);
  \pgfpathellipse{\pgfpointorigin}{\pgfpoint{1cm}{0cm}}{\pgfpoint{0cm}{1.25cm}}
  \pgfusepath{stroke}

  \pgfpathcircle{\pgfpoint{5pt}{5pt}}{2pt}
  \pgfpathcircle{\pgfpoint{-10pt}{5pt}}{2pt}
  \pgfusepath{fill}
  \color{red}
  \pgfpathcircle{\pgfpointborderellipse
    {\pgfpoint{5pt}{5pt}}{\pgfpoint{1cm}{1.25cm}}}{2pt}
  \pgfpathcircle{\pgfpointborderellipse
    {\pgfpoint{-10pt}{5pt}}{\pgfpoint{1cm}{1.25cm}}}{2pt}
  \pgfusepath{fill}
\end{tikzpicture}    
\end{codeexample}
\end{command}


\subsubsection{Points on the Intersction of Lines}


\begin{command}{\pgfpointintersectionoflines\marg{$p$}\marg{$q$}\marg{$s$}\marg{$t$}}
  This command returns the intersection of a line going through $p$
  and $q$ and a line going through $s$ and $t$. If the lines do not
  intersection, an arithmetic overflow will occur.

\begin{codeexample}[]
\begin{tikzpicture}
  \draw[help lines] (0,0) grid (2,2);
  \draw (.5,0) -- (2,2);
  \draw (1,2) -- (2,0);
  \pgfpathcircle{%
    \pgfpointintersectionoflines
      {\pgfpointxy{.5}{0}}{\pgfpointxy{2}{2}}
      {\pgfpointxy{1}{2}}{\pgfpointxy{2}{0}}}
    {2pt}
  \pgfusepath{stroke}
\end{tikzpicture}    
\end{codeexample}
\end{command}

\subsection{Extracting Coordinates}

There are two commands that can be used to ``extract'' the $x$- or
$y$-coordinate of a coordiante. The extraction process will
``un-big-point-correct'' the coordinates (if you do not know what this
means, things will work fine).

\begin{command}{\pgfextractx\marg{dimension}\marg{point}}
  Sets the \TeX-\meta{dimension} to the $x$-coordinate of the point.

\begin{codeexample}[code only]
\newdimen\mydim
\pgfextractx{\mydim}{\pgfpoint{2cm}{4pt}}
%% \mydim is now 2cm
\end{codeexample}
\end{command}

\begin{command}{\pgfextracty\marg{dimension}\marg{point}}
  Like |\pgfextractx|, except for the $y$-coordinate.
\end{command}




\subsection{Internals of How Point Commands Work}

As a normal user of \pgfname, you do not need to read this section. It
is relevant only if you need to understand how the point commands work
internally. 

When a command like |\pgfpoint{1cm}{2pt}| is called, all that happens
is that the two \TeX-dimension variables |\pgf@x| and |\pgf@y| are set
to |1cm| and |2pt|, respectively. A command like |\pgfpathmoveto| that
takes a coordinate as parameter will just execute this parameter and
then use the values of |\pgf@x| and |\pgf@y| as the coordinates to
which it will move the pen on the current path.

Commands like |\pgfpointnormalised| modify other variables besides
|\pgf@x| and |\pgf@y| during the computation of the final values of
|\pgf@x| and |\pgf@y|, it is a good idea to enclose a call of a
command like |\pgfpoint| in a \TeX-scope and then make the changes of
|\pgf@x| and |\pgf@y| global as in the following example:
\begin{codeexample}[code only]
...
{ % open scope
  \pgfpointnormalised{\pgfpoint{1cm}{1cm}}
  \global\pgf@x=\pgf@x % make the change of \pgf@x persist past the scope
  \global\pgf@y=\pgf@y % make the change of \pgf@y persist past the scope
}
% \pgf@x and \pgf@y are now set correctly, all other variables are
% unchanged 
\end{codeexample}

\makeatletter
Since this situation arises very often, the macro |\pgf@process| can
be used to perform the above code:
\begin{command}{\pgf@process\marg{code}}
  Executes the \meta{code} in a scope and then makes |\pgf@x| and
  |\pgf@y| global.
\end{command}

Note that this macro is used often internally. For this reason, it is
not a good idea to keep anything important in the variables |\pgf@x|
and |\pgf@y| since they will be overwritten and changed
frequently. Instead, intermediate values can ge stored in the
\TeX-dimensions |\pgf@xa|, |\pgf@xb|, |\pgf@xc| and their
|y|-counterparts |\pgf@ya|, |\pgf@yb|, |pgf@yc|. For examples, here is
the code of the command |\pgfpointadd|:
\begin{codeexample}[code only]
\def\pgfpointadd#1#2{%
  \pgf@process{#1}%
  \pgf@xa=\pgf@x%
  \pgf@ya=\pgf@y%
  \pgf@process{#2}%
  \advance\pgf@x by\pgf@xa%
  \advance\pgf@y by\pgf@ya}
\end{codeexample}



% Copyright 2003 by Till Tantau <tantau@cs.tu-berlin.de>.
%
% This program can be redistributed and/or modified under the terms
% of the LaTeX Project Public License Distributed from CTAN
% archives in directory macros/latex/base/lppl.txt.


\section{Constructing Paths}

\subsection{Overview}

The ``basic entity of drawing'' in \pgfname\ is the \emph{path}. A
path consists of several parts, each of which is either a closed or
open curve. An open curve has a starting point and an end point and
inbetween it consists of several \emph{segmets}, each of which is
either a straight line or a Bezi�r curve. Here is an example of a
path (in red) consisting of two parts, one open, one closed:

\begin{codeexample}[]
\begin{tikzpicture}[scale=2]
  \draw[thick,red]
       (0,0) coordinate (a)
    -- coordinate (ab) (1,.5) coordinate (b)
    .. coordinate (bc) controls +(up:1cm) and +(left:1cm) .. (3,1)  coordinate (c)
       (0,1) -- (2,1) -- coordinate (x) (1,2) -- cycle;

  \draw (a)  node[below] {start part 1}
        (ab) node[below right] {straight segment}
        (b)  node[right] {end first segment}
        (c)  node[right] {end part 1}
        (x)  node[above right]  {part 2 (closed)};        
\end{tikzpicture}
\end{codeexample}

A path, by itself, has no ``effect,'' that is, it does not leave any
marks on the page. It is just a set of points on the plane. However,
you can \emph{use} a path in different ways. The most natural actions
are \emph{stroking} (also known as \emph{drawing}) and
\emph{filling}. Stroking can be imagined as picking up a pen of a
certain diameter and ``moving it along the path.'' Filling means that
everything ``inside'' the path is filled with a uniform
color. Naturally, the open parts of a path must first be closed before
a path can be filled.

In \pgfname, there are numerous commands for constructing paths, all
of which start with |\pgfpath|. There are also commands for
\emph{using} paths, though most operations can be performed by calling
|\pgfusepath| with an appropriate parameter.

As a side-effect, the path construction commands keep track of two
bounding boxes. One is the bounding box for the current path, the
other is a bounding box for all paths in the current picture. See
Section~\ref{section-bb} for more details.

Each path construction command extends the current path in some
way. The ``current path'' is a global entity that persists accross
\TeX\ groups. Thus, between calls to the path construction commands
you can perform arbitrary computations and even open and closed \TeX\
groups. The current path only gets ``flushed'' when the |\pgfusepath|
command is called (or when the soft-path subsystem is used directly,
see Section~\ref{section-soft-paths}).

\subsection{The Move-To Path Operation}

The most basic operation is the move-to operation. It must be given at
the beginning of every path. This operation can also be used to start
a new part of a path.

\begin{command}{\pgfpathmoveto\marg{coordinate}}
  This command expects a \pgfname-coordinate like |\pgfpointorigin| as
  its parameter. When the current path is empty, this operation will
  start the path at the given \meta{coordinate}. If a path has already
  been partly constructed, this command will end the current part of
  the path and start a new one.
\begin{codeexample}[]
\begin{pgfpicture}
  \pgfpathmoveto{\pgfpointorigin}
  \pgfpathlineto{\pgfpoint{1cm}{1cm}}
  \pgfpathlineto{\pgfpoint{2cm}{1cm}}
  \pgfpathlineto{\pgfpoint{3cm}{0.5cm}}
  \pgfpathlineto{\pgfpoint{3cm}{0cm}}
  \pgfsetfillcolor{yellow}
  \pgfusepath{fill,stroke}
\end{pgfpicture}
\end{codeexample}
\begin{codeexample}[]
\begin{pgfpicture}
  \pgfpathmoveto{\pgfpointorigin}
  \pgfpathlineto{\pgfpoint{1cm}{1cm}}
  \pgfpathlineto{\pgfpoint{2cm}{1cm}}
  \pgfpathmoveto{\pgfpoint{2cm}{1cm}} % New part
  \pgfpathlineto{\pgfpoint{3cm}{0.5cm}}
  \pgfpathlineto{\pgfpoint{3cm}{0cm}}
  \pgfsetfillcolor{yellow}
  \pgfusepath{fill,stroke}
\end{pgfpicture}
\end{codeexample}
  The command will apply the current coordinate transformation matrix
  to \meta{coordinate} before using it.

  The command will update the bounding box of the current path and
  picture, if necessary. 
\end{command}


\subsection{The Line-To Path Operation}

\begin{command}{\pgfpathlineto\marg{coordinate}}
  This command extends the current path in a straight line to the
  given \meta{coordinate}. If this command is given at the beginning
  of path without any other path construction command given before (in
  particular without a move-to operation), the \TeX\ file may compile
  without an error message, but a viewer application may display an
  error message when trying to render the picture. 
\begin{codeexample}[]
\begin{pgfpicture}
  \pgfpathmoveto{\pgfpointorigin}
  \pgfpathlineto{\pgfpoint{1cm}{1cm}}
  \pgfpathlineto{\pgfpoint{2cm}{1cm}}
  \pgfsetfillcolor{yellow}
  \pgfusepath{fill,stroke}
\end{pgfpicture}
\end{codeexample}
  The command will apply the current coordinate transformation matrix
  to \meta{coordinate} before using it.

  The command will update the bounding box of the current path and
  picture, if necessary. 
\end{command}


\subsection{The Curve-To Path Operation}

\begin{command}{\pgfpathcurveto\marg{support 1}\marg{support 2}\marg{coordinate}}
  This command extends the current path with a Bezi�r curve from the
  last point of the path to  \meta{coordinate}. The \meta{support 1}
  and \meta{support 2} are the first and second support point of the
  Bezi�r curve. For more information on Bezi�r curve, please consult a
  standard textbook on computer graphics.

  Like the line-to command, this command may not be the first path
  construction command in a path.
\begin{codeexample}[]
\begin{pgfpicture}
  \pgfpathmoveto{\pgfpointorigin}
  \pgfpathcurveto
    {\pgfpoint{1cm}{1cm}}{\pgfpoint{2cm}{1cm}}{\pgfpoint{3cm}{0cm}}
  \pgfsetfillcolor{yellow}
  \pgfusepath{fill,stroke}
\end{pgfpicture}
\end{codeexample}
  The command will apply the current coordinate transformation matrix
  to \meta{coordinate} before using it.

  The command will update the bounding box of the current path and
  picture, if necessary. However, the bounding box is simply made
  large enough such that it encompasses all of the support points and
  the \meta{coordinate}. This will guarantee that the curve is
  completely inside the bounding box, but the bounding box will
  typically be quite a bit too large. It is not clear (to me) how this 
  can be avoided without resorting to ``some serious math'' in order
  to calculate a precise bounding box. 
\end{command}


\subsection{The Close Path Operation}

\begin{command}{\pgfpathclose}
  This command closed the current part of the path by appending a
  straight line to the start point of the current part. Note that there
  \emph{is} a difference between closing a path and using the line-to
  operation to add a straight line to the start of the current
  path. The difference is demonstrated by the upper corners of the triangles
  in the following example: 
\begin{codeexample}[]
\begin{tikzpicture}
  \draw[help lines] (0,0) grid (3,2);
  \pgfsetlinewidth{5pt}
  \pgfpathmoveto{\pgfpoint{1cm}{1cm}}
  \pgfpathlineto{\pgfpoint{0cm}{-1cm}}
  \pgfpathlineto{\pgfpoint{1cm}{-1cm}}
  \pgfpathclose
  \pgfpathmoveto{\pgfpoint{2.5cm}{1cm}}
  \pgfpathlineto{\pgfpoint{1.5cm}{-1cm}}
  \pgfpathlineto{\pgfpoint{2.5cm}{-1cm}}
  \pgfpathlineto{\pgfpoint{2.5cm}{1cm}}
  \pgfusepath{stroke}
\end{tikzpicture}
\end{codeexample}
\end{command}


\subsection{Arc, Ellipse and Circle Path Operations}

The path construction commands that we have discussed up to now are
sufficient to create all paths that can be created ``at all.''
However, it is useful to have special commands to create certain
shapes, like circle, that arise often in practice.

In the following, the commands for adding (parts of) (transformed)
circles to a path.

\begin{command}{\pgfpatharc\marg{start angle}\marg{end
      angle}\marg{radius}}
  This command appends a part of a circle (or an ellipse) to the current
  path. Imaging the curve between \meta{start angle} and \meta{end
    angle} on a circle of radius \meta{radius} (if $\meta{start angle}
  < \meta{end angle}$, the curve goes around the circle
  counterclockwise, otherwise clockwise). This curve is now moved such
  that the point where the curve starts is the last point of the
  path. Note that this command will \emph{not} start a new part of the
  path, which is important for example for filling purposes. 

\begin{codeexample}[]
\begin{tikzpicture}
  \draw[help lines] (0,0) grid (3,2);
  \pgfpathmoveto{\pgfpointorigin}
  \pgfpathlineto{\pgfpoint{0cm}{1cm}}
  \pgfpatharc{180}{90}{.5cm}
  \pgfpathlineto{\pgfpoint{3cm}{1.5cm}}
  \pgfpatharc{90}{-45}{.5cm}
  \pgfusepath{fill}
\end{tikzpicture}
\end{codeexample}

  Saying |\pgfpatharc{0}{360}{1cm}| ``nearly'' gives you a full
  circle. The ``nearly'' refers to the fact that the circle will not
  be closed. You can close it using |\pgfpathclose|.

  The \meta{radius} need not always be a single \TeX\
  dimension. Instead, it can also contain a slash, in which case it
  must consist of two dimensions separated by this slash. In this
  case, the first dimension is the $x$-radius and the second the
  $y$-radius of the ellipse from which the curve is taken:  

\begin{codeexample}[]
\begin{tikzpicture}
  \draw[help lines] (0,0) grid (3,2);
  \pgfpathmoveto{\pgfpointorigin}
  \pgfpatharc{180}{45}{2cm/1cm}
  \pgfusepath{draw}
\end{tikzpicture}
\end{codeexample}

  The axes of the circle or ellipse from which the arc is ``taken''
  always point up and right. However, the current coordinate
  transformation matrix will have an effect on the arc. This can be
  used to, say, rotate an arc:

\begin{codeexample}[]
\begin{tikzpicture}
  \draw[help lines] (0,0) grid (3,2);
  \pgftransformrotate{30}
  \pgfpathmoveto{\pgfpointorigin}
  \pgfpatharc{180}{45}{2cm/1cm}
  \pgfusepath{draw}
\end{tikzpicture}
\end{codeexample}

  The command will update the bounding box of the current path and
  picture, if necessary. Unless rotaion or shearing transformations
  are applied, the bounding box will be tight.
\end{command}

\begin{command}{\pgfpathellipse\marg{center}\marg{first
      axis}\marg{second axis}}
  The effect of this command is to append an ellipse to the current
  path (if the path is not empty, a new part is started). The
  ellipse's center will be \meta{center} and \meta{first axis} and
  \meta{second axis} are the axis \emph{vectors}. The same effect as
  this command can also be achieved using an appropriate sequence of
  move-to, arc, and close operations, but this command is easier and
  faster. 

\begin{codeexample}[]
\begin{tikzpicture}
  \draw[help lines] (0,0) grid (3,2);
  \pgfpathellipse{\pgfpoint{1cm}{0cm}}
                 {\pgfpoint{1.5cm}{0cm}}
                 {\pgfpoint{0cm}{1cm}}
  \pgfusepath{draw}
  \color{red}               
  \pgfpathellipse{\pgfpoint{1cm}{0cm}}
                 {\pgfpoint{1cm}{1cm}}
                 {\pgfpoint{-0.5cm}{0.5cm}}
  \pgfusepath{draw}
\end{tikzpicture}
\end{codeexample}

  The command will apply coordinate transformations to all coordinates
  of the ellipse. However, the coordinate transformations are applied
  only after the ellipse is ``finished conceptually.'' Thus, a
  transformation of 1cm to the right will simply shift the ellipse one
  centimeter to the right; it will not add 1cm to the $x$-coordinates
  of the two axis vectors.

  The command will update the bounding box of the current path and
  picture, if necessary. 
\end{command}

\begin{command}{\pgfpathcirlce\marg{center}\marg{radius}}
  A shorthand for |\pgfpathellipse| applied to \meta{center} and the
  two axis vectors $(\meta{radius},0)$ and $(0,\meta{radius})$. 
\end{command}


\subsection{Rectangle Path Operations}

Another shape that arises frequently is the rectangle. Two commands
can be used to add a rectangle to the current path. Both commands will
start a new part of the path.


\begin{command}{\pgfpathrectangle\marg{corner}\marg{diagonal vector}}
  Adds a rectangle to the path whose one corner is \meta{corner} and
  whose opposite corner is given by $\meta{corner} + \meta{diagonal
    vector}$.

\begin{codeexample}[]
\begin{tikzpicture}
  \draw[help lines] (0,0) grid (3,2);
  \pgfpathrectangle{\pgfpoint{1cm}{0cm}}{\pgfpoint{1.5cm}{1cm}}
  \pgfpathrectangle{\pgfpoint{1.5cm}{0.25cm}}{\pgfpoint{1.5cm}{1cm}}
  \pgfpathrectangle{\pgfpoint{2cm}{0.5cm}}{\pgfpoint{1.5cm}{1cm}}
  \pgfusepath{draw}
\end{tikzpicture}
\end{codeexample}
  The command will apply coordinate transformations and update the
  bounding boxes tightly.
\end{command}


\begin{command}{\pgfpathrectanglecorners\marg{corner}\marg{opposite corner}}
  Adds a rectangle to the path whose two opposing corners are
  \meta{corner} and \meta{opposite corner}.
\begin{codeexample}[]
\begin{tikzpicture}
  \draw[help lines] (0,0) grid (3,2);
  \pgfpathrectanglecorners{\pgfpoint{1cm}{0cm}}{\pgfpoint{1.5cm}{1cm}}
  \pgfusepath{draw}
\end{tikzpicture}
\end{codeexample}
  The command will apply coordinate transformations and update the
  bounding boxes tightly.
\end{command}



\subsection{The Grid Path Operation}

\begin{command}{\pgfpathgrid\oarg{options}\marg{lower left}\marg{upper right}}
  Appends a grid to the current path. That is, a (possibly large)
  numer of parts are added to the path, each part consisting of a
  single horizontal or vertical straight line segment.

  Conceptually, the origin is part of the grid and the grid is clipped 
  to the rectanlge specified by the \meta{lower left} and
  the \meta{upper right} corner. However, no clipping occurs (this
  command just adds parts to the current path). Rather, the points
  where the lines enter and leave the ``clipping area'' are computed
  and used to add simple lines to the current path.

  Allowed \meta{options} are:
  \begin{description}
  \item[\declare{|stepx=|\meta{dimension}}]
    Sets the horizontal steping to \meta{dimension}. Default is 1cm.
  \item[\declare{|stepy=|\meta{dimension}}]
    Sets the vertical steping to \meta{dimension}. Default is 1cm.
  \item[\declare{|step=|\meta{vector}}]
    Sets the horizontal stepping to the $x$-coordinate of
    \meta{vector} and the vertical steping its $y$-coordinate.
  \end{description}
\begin{codeexample}[]
\begin{pgfpicture}
  \pgfsetlinewidth{0.8pt}
  \pgfpathgrid[step={\pgfpoint{1cm}{1cm}}]
    {\pgfpoint{-3mm}{-3mm}}{\pgfpoint{33mm}{23mm}}
  \pgfusepath{stroke}
  \pgfsetlinewidth{0.4pt}
  \pgfpathgrid[stepx=1mm,stepy=1mm]
    {\pgfpoint{-1.5mm}{-1.5mm}}{\pgfpoint{31.5mm}{21.5mm}}
  \pgfusepath{stroke}
\end{pgfpicture}
\end{codeexample}
  The command will apply coordinate transformations and update the
  bounding boxes tightly. As for ellipses, the transformations are
  applied to the ``conceptually finished'' grid. 
\begin{codeexample}[]
\begin{pgfpicture}
  \pgftransformrotate{10}
  \pgfpathgrid[stepx=1mm,stepy=2mm]{\pgfpoint{0mm}{0mm}}{\pgfpoint{30mm}{30mm}}
  \pgfusepath{stroke}
\end{pgfpicture}
\end{codeexample}
\end{command}


\subsection{The Parabola Path Operation}

\begin{command}{\pgfpathparabola\marg{bend vector}\marg{end vector}}
  This command appends two half-parabolas to the  current path. The
  first starts at the current point and ends at the current point plus
  \meta{bend vector}. At his point, it has its bend. The second half
  parabola starts at that bend point and end at point that is given by
  the bend plus \meta{end vector}.

  If you set \meta{end vector} to the null vector, you append only a
  half parabola that goes from the current point to the bend; by
  setting \meta{bend vecotr} to the null vector, you append only a
  half parabola that goes to current point plus \meta{end vector} and
  has its bend at the current point.

  It is not possible to use this command to draw a part of a parabola
  that does not contain the bend.

\begin{codeexample}[]
\begin{pgfpicture}
  % Half-parabola going ``up and right''
  \pgfpathmoveto{\pgfpointorigin}
  \pgfpathparabola{\pgfpointorigin}{\pgfpoint{2cm}{4cm}}
  \color{red}
  \pgfusepath{stroke}

  % Half-parabola going ``down and right''
  \pgfpathmoveto{\pgfpointorigin}
  \pgfpathparabola{\pgfpoint{-2cm}{4cm}}{\pgfpointorigin}
  \color{blue}
  \pgfusepath{stroke}

  % Full parabola
  \pgfpathmoveto{\pgfpoint{-2cm}{2cm}}
  \pgfpathparabola{\pgfpoint{1cm}{-1cm}}{\pgfpoint{2cm}{4cm}}
  \color{orange}
  \pgfusepath{stroke}
\end{pgfpicture}
\end{codeexample}
  The command will apply coordinate transformations and update the
  bounding boxes.
\end{command}


\subsection{Sine and Cosine Path Operations}

Sine a cosine curves often need to be drawn and the following commands
may help with this. However, they only allow you to append sine and
cosine curves in intervals that are multiples of $\pi/2$.

\begin{command}{\pgfpathsine\marg{vector}}
  This command appends a sine curve in the interval $[0,\pi/2]$ to the
  current path. The sine curve is squeezed or strechted such that the
  curve starts at the current point and ends at the current point plus
  \meta{vector}.
\begin{codeexample}[]
\begin{tikzpicture}
  \draw[help lines] (0,0) grid (3,1);
  \pgfpathmoveto{\pgfpoint{1cm}{0cm}}
  \pgfpathsine{\pgfpoint{1cm}{1cm}}
  \pgfusepath{stroke}

  \color{red}
  \pgfpathmoveto{\pgfpoint{1cm}{0cm}}
  \pgfpathsine{\pgfpoint{-2cm}{-2cm}}
  \pgfusepath{stroke}
\end{tikzpicture}
\end{codeexample}
  The command will apply coordinate transformations and update the
  bounding boxes.  
\end{command}

\begin{command}{\pgfpathcosine\marg{vector}}
  This command appends a cosine curve in the interval $[0,\pi/2]$ to the
  current path. The curve is squeezed or strechted such that the
  curve starts at the current point and ends at the current point plus
  \meta{vector}. Using several sine and cosine operations in sequence
  allows you to produce a complete sine or cosine curve
\begin{codeexample}[]
\begin{pgfpicture}
  \pgfpathmoveto{\pgfpoint{0cm}{0cm}}
  \pgfpathsine{\pgfpoint{1cm}{1cm}}
  \pgfpathcosine{\pgfpoint{1cm}{-1cm}}
  \pgfpathsine{\pgfpoint{1cm}{-1cm}}
  \pgfpathcosine{\pgfpoint{1cm}{1cm}}
  \pgfsetfillcolor{yellow}
  \pgfusepath{fill,stroke}
\end{pgfpicture}
\end{codeexample}
  The command will apply coordinate transformations and update the
  bounding boxes.  
\end{command}


\subsection{Plot Path Operations}

There exist several commands for appending
plots to a path. These
commands are available through the package |pgfbaseplot|. They are
documented in Section~\ref{section-plots}.


\subsection{Rounded Corners}

Normally, when you connect two straight line segments or when you
connect two curves that end and start ``at different angles'' you get
``sharp corners'' between the lines or curves. In some cases it would
be desirable that we get ``rounded corners'' instead. Thus, the lines
or curves should be shortened a bit and then connected by arcs.

\pgfname\ offers an easy way to achieve this effect, by calling the
following two commands.

\begin{command}{\pgfsetcornersarced\marg{point}}
  This command causes all subsequent corners to be replaced by little
  arcs. The effect of this command lasts till the end of the current
  \TeX\ scope.

  The \meta{point} dictates how large the corner arc will be. Consider
  a corner made by two lines $l$ and~$r$ and assume that the line $l$
  comes first on the path. The $x$-dimension of the \meta{point}
  decides by how much of the line $l$ will be shortened, the
  $y$-dimension of \meta{point} decides by how much the line $r$ will
  be shortened. Then, the shortened lines are connected by an arc.

\begin{codeexample}[]
\begin{tikzpicture}
  \draw[help lines] (0,0) grid (3,2);

  \pgfsetcornersarced{\pgfpoint{5mm}{5mm}}
  \pgfpathrectanglecorners{\pgfpointorigin}{\pgfpoint{3cm}{2cm}}
  \pgfusepath{stroke}
\end{tikzpicture}
\end{codeexample}

\begin{codeexample}[]
\begin{tikzpicture}
  \draw[help lines] (0,0) grid (3,2);

  \pgfsetcornersarced{\pgfpoint{10mm}{5mm}}
  % 10mm entering,
  % 5mm leaving.
  \pgfpathmoveto{\pgfpointorigin}
  \pgfpathlineto{\pgfpoint{0cm}{2cm}}
  \pgfpathlineto{\pgfpoint{3cm}{2cm}}
  \pgfpathcurveto
    {\pgfpoint{3cm}{0cm}}
    {\pgfpoint{2cm}{0cm}}
    {\pgfpoint{1cm}{0cm}}
  \pgfusepath{stroke}
\end{tikzpicture}
\end{codeexample}

  If the $x$- and $y$-coordinates of \meta{point} are the same and the
  corner is a right angle, you will get a perfect quarter circle
  (well, not quite perfect, but perfect up to six decimals). When the
  angle is not $90^\circ$, you only get a fair approximation.

  More or less ``all'' corners will be rounded, even the corner
  generated by a |\pgfpathclose| command. (The author is a bit proud
  of this feature.)
  
\begin{codeexample}[]
\begin{pgfpicture}
  \pgfsetcornersarced{\pgfpoint{4pt}{4pt}}
  \pgfpathmoveto{\pgfpointpolar{0}{1cm}}
  \pgfpathlineto{\pgfpointpolar{72}{1cm}}
  \pgfpathlineto{\pgfpointpolar{144}{1cm}}
  \pgfpathlineto{\pgfpointpolar{216}{1cm}}
  \pgfpathlineto{\pgfpointpolar{288}{1cm}}
  \pgfpathclose
  \pgfusepath{stroke}
\end{pgfpicture}
\end{codeexample}

  To return to normal (unrounded) corners, use
  |\pgfsetcornersarced{\pgfpointorigin}|.

  Note that the rounding will produce strange and undesirable effects
  if the lines at the corners are too short. In this case, the
  shortening may cause the lines to ``suddenly extend over the other
  end'' which is rarely desirable. 
\end{command}




\subsection{Internal Tracking of Bounding Boxes for Paths and Pictures}

\label{section-bb}

\makeatletter

The path construction commands keep track of two bounding boxes: One
for the current path, which is reset whenever the path is used and
thereby flushed, and a bounding box for the current |{pgfpicture}|. 

The bounding boxes are not accessible by ``normal'' macros. Rather,
two sets of four dimension variables are used for this, all of which
contain the letter~|@|.

\begin{textoken}{\pgf@pathminx}
  The minimum $x$-coordinate ``mentioned'' in the current
  path. Initially, this is set to $16000$pt.
\end{textoken}

\begin{textoken}{\pgf@pathmaxx}
  The maxmimum $x$-coordinate ``mentioned'' in the current
  path. Initially, this is set to $-16000$pt.
\end{textoken}

\begin{textoken}{\pgf@pathminy}
  The minimum $y$-coordinate ``mentioned'' in the current
  path. Initially, this is set to $16000$pt.
\end{textoken}

\begin{textoken}{\pgf@pathmaxy}
  The maxmimum $y$-coordinate ``mentioned'' in the current
  path. Initially, this is set to $-16000$pt.
\end{textoken}

\begin{textoken}{\pgf@picminx}
  The minimum $x$-coordinate ``mentioned'' in the current
  picture. Initially, this is set to $16000$pt.
\end{textoken}

\begin{textoken}{\pgf@picmaxx}
  The maxmimum $x$-coordinate ``mentioned'' in the current
  picture. Initially, this is set to $-16000$pt.
\end{textoken}

\begin{textoken}{\pgf@picminy}
  The minimum $y$-coordinate ``mentioned'' in the current
  picture. Initially, this is set to $16000$pt.
\end{textoken}

\begin{textoken}{\pgf@picmaxy}
  The maxmimum $y$-coordinate ``mentioned'' in the current
  picture. Initially, this is set to $-16000$pt.
\end{textoken}


Each time a path construction command is called, the above variables
are (globally) updated. To facilitate this, you can use the following
command:

\begin{command}{\pgf@protocolsizes\marg{x-dimension}\marg{y-dimension}}
  Updates all of the above dimension in such a way that the point
  specified by the two arguments is inside both bounding boxes. For
  the picture's bounding box this updating occurs only if
  |\ifpgf@relevantforpicturesize| is true, see below.
\end{command}

For the bounding box of the picture it is not always desirable that
every path construction command affects this bounding box. For
example, if you have just used a clip command, we do not want anything
outside the clipping area to affect the boundinb box. For this reason,
there exists a special ``\TeX\ if'' that (locally) decides whether
updating should be applied to the picture's bounding box. Clipping
will set this if to false, as will certain other commands.

\begin{command}{\pgf@relevantforpicturesizefalse}
  Suppresses updating of the picture's bounding box.
\end{command}

\begin{command}{\pgf@relevantforpicturesizetrue}
  Causes updating of the picture's bounding box.
\end{command}


% Copyright 2003 by Till Tantau <tantau@cs.tu-berlin.de>.
%
% This program can be redistributed and/or modified under the terms
% of the LaTeX Project Public License Distributed from CTAN
% archives in directory macros/latex/base/lppl.txt.


\section{Using Paths}

\subsection{Overview}

Once a path has been constructed, it can be \emph{used} in different
ways. For example, you can draw the path or fill it or use it for
clipping.

Numerous graph parameters influence how a path will be rendered. For
example, when you draw a path, the line width is important as well as
the dashing pattern. The options that govern how paths are rendered
can all be set with commands starting with |\pgfset|. \emph{All
  options that influence how a path is rendered always influence the
  complete path.} Thus, it is not possible to draw part of a path
using, say, a red color and drawing another part using a green
color. To achieve such an effect, you must use two paths.

In detail, paths can be used in the following ways:

\begin{enumerate}
\item
  You can \emph{stroke} (also known as \emph{draw}) a path.
\item
  You can \emph{fill} a path with a uniform color.
\item
  You can \emph{clip} subsequent renderings against the path.
\item
  You can \emph{shade} a path.
\item
  You can \emph{use the path as bounding box} for the whole picture.
\end{enumerate}
You can also perform any combination of the above, though it makes no
sense to fill and shade a path at the same time.

To perform (a combination of) the first three actions, you can use the
following command:
\begin{command}{\pgfusepath\marg{actions}}
  Applies the given \meta{actions} to the current path. Afterwards,
  the current path is (globally) empty. The following actions are
  possible:
  \begin{itemize}
  \item \declare{|fill|}
    fills the path. See Section~\ref{section-fill} for further details.
\begin{codeexample}[]
\begin{pgfpicture}
  \pgfpathmoveto{\pgfpointorigin}
  \pgfpathlineto{\pgfpoint{1cm}{1cm}}
  \pgfpathlineto{\pgfpoint{1cm}{0cm}}
  \pgfusepath{fill}
\end{pgfpicture}
\end{codeexample}
  \item \declare{|stroke|}
    strokes the path. See Section~\ref{section-stroke} for further details.
\begin{codeexample}[]
\begin{pgfpicture}
  \pgfpathmoveto{\pgfpointorigin}
  \pgfpathlineto{\pgfpoint{1cm}{1cm}}
  \pgfpathlineto{\pgfpoint{1cm}{0cm}}
  \pgfusepath{stroke}
\end{pgfpicture}
\end{codeexample}
  \item \declare{|clip|}
    clips all subsequent drawings against the path. See
    Section~\ref{section-clip} for further details.
\begin{codeexample}[]
\begin{pgfpicture}
  \pgfpathmoveto{\pgfpointorigin}
  \pgfpathlineto{\pgfpoint{1cm}{1cm}}
  \pgfpathlineto{\pgfpoint{1cm}{0cm}}
  \pgfusepath{stroke,clip}
  \pgfpathcircle{\pgfpoint{1cm}{1cm}}{0.5cm}
  \pgfusepath{fill}
\end{pgfpicture}
\end{codeexample}
  \item \declare{|discard|}
    discards the path, that is, it is not used at all. Giving this
    option (alone) has the same effect as giving an empty options
    list.
  \end{itemize}
  When more than one of the first three actions are given, they are
  applied in the above ordering, regardless of their ordering in
  \meta{actions}. Thus, |{stroke,fill}| and |{fill,stroke}| have the
  same effect. 
\end{command}

To shade a path, use the |\pgfshadepath| command, which is explained
in Section~\ref{section-shadings}.



\subsection{Stroking a Path}
\label{section-stroke}

When you use |\pgfusepath{stroke}| to stroke a path, several graphic
parameters influence how the path is drawn. The commands for setting
these parameters are explained in the following.

Note that all graphic parameters apply to the path as a whole, never
only to a part of it.

All graphic parameters are local to the current |{pgfscope}|, but they
persists past \TeX\ groups, \emph{except} for the interior rule
(even-odd or nonzero) and the arrow tip kinds. The latter graphic
parameters only persist till the end of the current \TeX\ group, but 
this may change in the future, so do not count on this.

\subsubsection{Graphic Parameter: Line Width}

\begin{command}{\pgfsetlinewidth\marg{line width}}
  This command sets the line width for subsequent strokes (in the
  current |pgfscope|). The line width is given as a normal \TeX\
  dimension like |0.4pt| or |1mm|.

\begin{codeexample}[]
\begin{pgfpicture}
  \pgfsetlinewidth{1mm}
  \pgfpathmoveto{\pgfpoint{0mm}{0mm}}
  \pgfpathlineto{\pgfpoint{2cm}{0mm}}
  \pgfusepath{stroke}
  \pgfsetlinewidth{2\pgflinewidth} % double in size
  \pgfpathmoveto{\pgfpoint{0mm}{5mm}}
  \pgfpathlineto{\pgfpoint{2cm}{5mm}}
  \pgfusepath{stroke}
\end{pgfpicture}
\end{codeexample}
\end{command}

\begin{textoken}{\pgflinewidth}
  You can access the current line width via the \TeX\ dimension
  |\pgflinewidth|. It will be set to the correct line width, that is,
  even when a \TeX\ group closed, the value will be correct since it
  is set globally, but when a |{pgfscope}| closes, the value is set to
  the correct value it had before the scope.
\end{textoken}


\subsubsection{Graphic Parameter: Caps and Joins}

\begin{command}{\pgfsetbuttcap}
  Sets the line cap to a butt cap. See Section~\ref{section-cap-joins}
  for an explanation of what this is.
\end{command}
\begin{command}{\pgfsetroundcap}
  Sets the line cap to a round cap. See again
  Section~\ref{section-cap-joins}.
\end{command}
\begin{command}{\pgfsetrectcap}
  Sets the line cap to a square cap. See again
  Section~\ref{section-cap-joins}. 
\end{command}
\begin{command}{\pgfsetroundjoin}
  Sets the line join to a round join. See again
  Section~\ref{section-cap-joins}. 
\end{command}
\begin{command}{\pgfsetbeveljoin}
  Sets the line join to a bevel join. See again
  Section~\ref{section-cap-joins}. 
\end{command}
\begin{command}{\pgfsetmiterjoin}
  Sets the line join to a miter join. See again
  Section~\ref{section-cap-joins}. 
\end{command}
\begin{command}{\pgfsetmiterlimit\marg{miter limit factor}}
  Sets the miter limit to  \meta{miter limit factor}. See again 
  Section~\ref{section-cap-joins}. 
\end{command}

\subsubsection{Graphic Parameter: Dashing}

\begin{command}{\pgfsetdash\marg{list of even length of dimensions}\marg{phase}}
  Sets the dashing of a line. The first entry in the list specifies
  the length of the first solid part of the list. The second entry
  specifies the length of the following gap. Then comes the length of
  the second solid part, following by the length of the second gap,
  and so on. The \meta{phase} specifies where the first solid part
  starts relative to the beginning of the line.

\begin{codeexample}[]
\begin{pgfpicture}
  \pgfsetdash{{0.5cm}{0.5cm}{0.1cm}{0.2cm}}{0cm}
  \pgfpathmoveto{\pgfpoint{0mm}{0mm}}
  \pgfpathlineto{\pgfpoint{2cm}{0mm}}
  \pgfusepath{stroke}
  \pgfsetdash{{0.5cm}{0.5cm}{0.1cm}{0.2cm}}{0.1cm}
  \pgfpathmoveto{\pgfpoint{0mm}{1mm}}
  \pgfpathlineto{\pgfpoint{2cm}{1mm}}
  \pgfusepath{stroke}
  \pgfsetdash{{0.5cm}{0.5cm}{0.1cm}{0.2cm}}{0.2cm}
  \pgfpathmoveto{\pgfpoint{0mm}{2mm}}
  \pgfpathlineto{\pgfpoint{2cm}{2mm}}
  \pgfusepath{stroke}
\end{pgfpicture}
\end{codeexample}

  Use |\pgfsetdash{}{0pt}| to get a solid dashing.
\end{command}

\subsubsection{Graphic Parameter: Stroke Color}

\begin{command}{\pgfsetstrokecolor\marg{color}}
  Sets the color used for stroking lines to \meta{color}, where
  \meta{color} is a \LaTeX\ color like |red| or |black!20!red|. Unlike
  the |\color| command, the effect of this command lasts till the end
  of the current |{pgfscope}| and not till the end of the current
  \TeX\ group.

  The color used for stroking may be different from the color used for
  filling. However, a |\color| command will always ``immediately
  override'' any special settings for the stroke and fill colors.

  In plain \TeX, this command will also work, but the problem of
  \emph{defining} a color arises. After all, plain \TeX\ does not
  provide \LaTeX\ colors. For this reason, \pgfname\ implements a
  minimalistic ``emulation'' of the |\definecolor|, |\colorlet|, and
  |\color| commands. Only gray-scale and rgb colors are supported. For
  most cases this turns out to be enough.

\begin{codeexample}[]
\begin{pgfpicture}
  \pgfsetlinewidth{1pt}
  \color{red}
  \pgfpathcircle{\pgfpoint{0cm}{0cm}}{3mm} \pgfusepath{fill,stroke}
  \pgfsetstrokecolor{black}
  \pgfpathcircle{\pgfpoint{1cm}{0cm}}{3mm} \pgfusepath{fill,stroke}
  \color{red}
  \pgfpathcircle{\pgfpoint{2cm}{0cm}}{3mm} \pgfusepath{fill,stroke}
\end{pgfpicture}
\end{codeexample}
\end{command}

\begin{command}{\pgfsetcolor\marg{color}}
  Sets both the stroke and fill color. The difference to the normal
  |\color| command is that the effect lasts till the end of the
  current |{pgfscope}|, not only till the end of the current \TeX\
  group. 
\end{command}


\subsubsection{Graphic Parameter: Arrows}

After a path has been drawn, \pgfname\ can add arrow tips at the
ends. Currently, it will only add arrows correctly at the end of paths
that consist of a single open part. For other paths, like closed paths
or path consisting of multiple parts, the result is not defined.

\begin{command}{\pgfsetarrowsstart\marg{arrow kind}}
  Sets the arrow tip kind used at the start of a (possibly curved)
  path. When this option is used, the line will often be slightly
  shortened to ensure that the tip of the arrow will exactly ``touch''
  the ``real'' start of the line.

  To ``clear'' the start arrow, say |\pgfsetarrowsstart{}|.
\begin{codeexample}[]
\begin{pgfpicture}
  \pgfsetarrowsstart{latex}
  \pgfpathmoveto{\pgfpointorigin}
  \pgfpathlineto{\pgfpoint{1cm}{0cm}}
  \pgfusepath{stroke}
  \pgfsetarrowsstart{to}
  \pgfpathmoveto{\pgfpoint{0cm}{2mm}}
  \pgfpathlineto{\pgfpoint{1cm}{2mm}}
  \pgfusepath{stroke}
\end{pgfpicture}
\end{codeexample}

  The effect of this command persists only till the end of the current
  \TeX\ scope.

  The different possible arrow kinds are explained in
  Section~\ref{section-arrows}.  
\end{command}

\begin{command}{\pgfsetarrowsend\marg{arrow kind}}
  Sets the arrow tip kind used at the end of a path.
\begin{codeexample}[]
\begin{pgfpicture}
  \pgfsetarrowsstart{latex}
  \pgfsetarrowsend{to}
  \pgfpathmoveto{\pgfpointorigin}
  \pgfpathlineto{\pgfpoint{1cm}{0cm}}
  \pgfusepath{stroke}
\end{pgfpicture}
\end{codeexample}
\end{command}

\begin{command}{\pgfsetarrows{\texttt{\char`\{}}\meta{start kind}|-|\meta{end kind}{\texttt{\char`\}}}}
  Sets the start arrow kind to \meta{start kind} and the end kind to
  \meta{end kind}.
\begin{codeexample}[]
\begin{pgfpicture}
  \pgfsetarrows{latex-to}
  \pgfpathmoveto{\pgfpointorigin}
  \pgfpathlineto{\pgfpoint{1cm}{0cm}}
  \pgfusepath{stroke}
\end{pgfpicture}
\end{codeexample}
\end{command}

\begin{command}{\pgfsetshortenstart\marg{dimension}}
  This command will shortened the start of every stroked path by the
  given dimension. This shortening is done in addition to automatic
  shortening done by a start arrow, but it can be used even if no
  start arrow is given.

  This command is useful if you wish arrows or lines to ``stop shortly
  before'' a given point.
\begin{codeexample}[]
\begin{pgfpicture}
  \pgfpathcircle{\pgfpointorigin}{5mm}
  \pgfusepath{stroke}
  \pgfsetarrows{latex-}
  \pgfsetshortenstart{4pt}
  \pgfpathmoveto{\pgfpoint{5mm}{0cm}} % would be on the circle
  \pgfpathlineto{\pgfpoint{2cm}{0cm}}
  \pgfusepath{stroke}
\end{pgfpicture}
\end{codeexample}
\end{command}
  
\begin{command}{\pgfsetshortenend\marg{dimension}}
  Works like |\pgfsetshortenstart|.
\end{command}



\subsection{Filling a Path}
\label{section-fill}

Filling a path means coloring every interior point of the path with
the current fill color. It is not always obvious whether a point is
``inside'' a  path when the path is self-intersecting and/or consists
or multiple parts. In this case either the nonzero winding number rule
or the even-odd crossing number rule is used to decide, which points
lie ``inside.'' These rules are explained in
Section~\ref{section-rules}. 

\subsubsection{Graphic Parameter: Interior Rule}

You can set which rule is used using the following commands:

\begin{command}{\pgfseteorule}
  Dictates that the even-odd rule is used in subsequent fillings in
  the current \emph{\TeX\ scope}. Thus, for once, the effect of this
  command does not persist past the current \TeX\ scope.

\begin{codeexample}[]
\begin{pgfpicture}
  \pgfseteorule
  \pgfpathcircle{\pgfpoint{0mm}{0cm}}{7mm}
  \pgfpathcircle{\pgfpoint{5mm}{0cm}}{7mm}
  \pgfusepath{fill}
\end{pgfpicture}
\end{codeexample}
\end{command}

\begin{command}{\pgfsetnonzerorule}
  Dictates that the nonzero winding number rule is used in subsequent
  fillings in the current \TeX\ scope. This is the default.

\begin{codeexample}[]
\begin{pgfpicture}
  \pgfsetnonzerorule
  \pgfpathcircle{\pgfpoint{0mm}{0cm}}{7mm}
  \pgfpathcircle{\pgfpoint{5mm}{0cm}}{7mm}
  \pgfusepath{fill}
\end{pgfpicture}
\end{codeexample}
\end{command}

\subsubsection{Graphic Parameter: Filling Color}

\begin{command}{\pgfsetfillcolor\marg{color}}
  Sets the color used for filling paths to \meta{color}. Like the
  stroke color, the effect lasts only till the next use of |\color|. 
\end{command}


\subsection{Clipping a Path}
\label{section-clip}

When you add the |clip| option, the current path is used for
clipping subsequent drawings. The same rule as for filling is used to
decide whether a point is inside or outside the path, that is, either
the even-odd rule or the nonzero rule.

Clipping never enlarges the clipping area. Thus, when you clip against
a certain path and then clip again against another path, you clip
against the intersection of both.

The only way to enlarge the clipping path is to end the |{pgfscope}|
in which the clipping was done. At the end of a |{pgfscope}| the
clipping path that was in force at the beginning of the scope is
reinstalled. 

\subsection{Using a Path as a Bounding Box}
\label{section-using-bb}

When you add the |use as bounding box| option, the bounding box of the
picture will be enlarged such that the path in encompassed, but any
\emph{subsequent} paths of the current \TeX\ scope will not have any
effect on the size of the bounding box. Typically, you use this
command at the very beginning of a |{pgfpicture}| environment.

\begin{codeexample}[]
Left
\begin{pgfpicture}
  \pgfpathrectangle{\pgfpointorigin}{\pgfpoint{2ex}{1ex}}
  \pgfusepath{use as bounding box} % draws nothing

  \pgfpathcircle{\pgfpointorigin}{2ex}
  \pgfusepath{stroke}
\end{pgfpicture}
right.
\end{codeexample}


% Copyright 2003 by Till Tantau <tantau@cs.tu-berlin.de>.
%
% This program can be redistributed and/or modified under the terms
% of the LaTeX Project Public License Distributed from CTAN
% archives in directory macros/latex/base/lppl.txt.


\section{Arrow Tips}
\label{section-arrows}


\subsection{Overview}

\subsubsection{When Does PGF Draw Arrow Tips?}

\pgfname\ offers an interface for placing \emph{arrow tips} at the end
of lines. The interface works as follows:

\begin{enumerate}
\item
  You assign a name to a certain kind of arrow tips. For example, the
  arrow tip |latex| is the arrow tip used by the standard \LaTeX\
  picture environment; the arrow tip |to| looks like the tip of the
  arrow in \TeX's |\to| command; and so on.

  This is done once at the beginning of the document.
\item
  Inside some picture, at some point you specify that in the current
  scope from now on you would like tips of, say, kind |to| to be added
  at the end and/or beginning of all paths.

  When an arrow kind has been installed and when \pgfname\ is about to
  stroke a path, the following things happen:
  \begin{enumerate}
  \item
    The beginning and/or end of the path is shortened appropriately.
  \item
    The path is stroked.
  \item
    The arrow tip is drawn at the beginning and/or end of the path,
    appropriately rotated and appropriately resized.
  \end{enumerate}
\end{enumerate}

In the above description, there are a number of ``appropriatelies.''
The exact details are not quite trivial and described later on.

\subsubsection{Meta-Arrow Tips}

In \pgfname, arrows are ``meta-arrows'' in the same way that fonts in
\TeX\ are ``meta-fonts.'' When a meta-arrow is resized, it is not
simply scaled, but a possibly complicated transformation is applied to
the size.

A meta-font is not one particular font at a specific size with a
specific stroke width (and with a large number of other parameters
being fixed). Rather, it is a ``blueprint'' (actually, more like a
program) for generating such a font at a particular size and
width. This allows the designer of a meta-font to make sure that, say,
the font is somewhat thicker and wider at very small sizes. To
appreciate the difference: Compare the following texts: ``Berlin'' and
``\tikz{\node [scale=2,inner sep=0pt,outer sep=0pt]{\tiny
    Berlin};}''. The first is a ``normal'' text, the second is the tiny
version scaled by a factor of two. Obviously, the first look
better. Now, compare  ``\tikz{\node [scale=.5,inner sep=0pt,outer
  sep=0pt]{Berlin};}'' and ``{\tiny Berlin}''. This time, the normal
text was scaled down, while the second text is a ``normal'' tiny
text. The second text is easier to read. 

\pgfname's meta-arrows work in a similar fashion: The shape of an
arrow tip can vary according to the line width of the arrow tip is
used. Thus, an arrow tip drawn at a line width of 5pt will typically
\emph{not} be five times as large as an arrow tip of line width
1pt. Instead, the size of the arrow will get bigger only slowly as the
line width increases.

To appreciate the difference, here are the |latex| and |to| arrows, as
drawn by \pgfname\ at four different sizes:

\medskip
\begin{tikzpicture}
  \draw[-latex,line width=0.1pt] (0pt,0ex) -- +(3,0)  node[thin,right] {line width is 0.1pt};
  \draw[-latex,line width=0.4pt] (0pt,-2em) -- +(3,0) node[thin,right] {line width is 0.4pt};
  \draw[-latex,line width=1.2pt] (0pt,-4em) -- +(3,0) node[thin,right] {line width is 1.2pt};
  \draw[-latex,line width=5pt]   (0pt,-6em) -- +(3,0) node[thin,right] {line width is 5pt};

  \draw[-to,line width=0.1pt] (6cm,0ex) -- +(3,0)  node[thin,right] {line width is 0.1pt};
  \draw[-to,line width=0.4pt] (6cm,-2em) -- +(3,0) node[thin,right] {line width is 0.4pt};
  \draw[-to,line width=1.2pt] (6cm,-4em) -- +(3,0) node[thin,right] {line width is 1.2pt};
  \draw[-to,line width=5pt]   (6cm,-6em) -- +(3,0) node[thin,right] {line width is 5pt};
\end{tikzpicture}

\medskip
Here, by comparison, is the same arrow when it is simply ``resized''
(as done by most programs):

\pgfarrowsdeclare{bad latex}{bad latex}
{
  \pgfarrowsleftextend{-1\pgflinewidth}
  \pgfarrowsrightextend{9\pgflinewidth}
}
{
  \pgfpathmoveto{\pgfpoint{9\pgflinewidth}{0pt}}
  \pgfpathcurveto
  {\pgfpoint{6.3333\pgflinewidth}{.5\pgflinewidth}}
  {\pgfpoint{2\pgflinewidth}{2\pgflinewidth}}
  {\pgfpoint{-1\pgflinewidth}{3.75\pgflinewidth}}
  \pgfpathlineto{\pgfpoint{-1\pgflinewidth}{-3.75\pgflinewidth}}
  \pgfpathcurveto
  {\pgfpoint{2\pgflinewidth}{-2\pgflinewidth}}
  {\pgfpoint{6.3333\pgflinewidth}{-.5\pgflinewidth}}
  {\pgfpoint{9\pgflinewidth}{0pt}}
  \pgfusepathqfill
}

\pgfarrowsdeclare{bad to}{bad to}
{
  \pgfarrowsleftextend{-2\pgflinewidth}
  \pgfarrowsrightextend{\pgflinewidth}
}
{
  \pgfsetlinewidth{0.8\pgflinewidth}
  \pgfsetdash{}{0pt}
  \pgfsetroundcap
  \pgfsetroundjoin
  \pgfpathmoveto{\pgfpoint{-3\pgflinewidth}{4\pgflinewidth}}
  \pgfpathcurveto
  {\pgfpoint{-2.75\pgflinewidth}{2.5\pgflinewidth}}
  {\pgfpoint{0pt}{0.25\pgflinewidth}}
  {\pgfpoint{0.75\pgflinewidth}{0pt}}
  \pgfpathcurveto
  {\pgfpoint{0pt}{-0.25\pgflinewidth}}
  {\pgfpoint{-2.75\pgflinewidth}{-2.5\pgflinewidth}}
  {\pgfpoint{-3\pgflinewidth}{-4\pgflinewidth}}
  \pgfusepathqstroke
}

\medskip
\begin{tikzpicture}
  \draw[-bad latex,line width=0.1pt] (0pt,0ex) -- +(3,0)  node[thin,right] {line width is 0.1pt};
  \draw[-bad latex,line width=0.4pt] (0pt,-2em) -- +(3,0) node[thin,right] {line width is 0.4pt};
  \draw[-bad latex,line width=1.2pt] (0pt,-4em) -- +(3,0) node[thin,right] {line width is 1.2pt};
  \draw[-bad latex,line width=5pt]   (0pt,-6em) -- +(3,0) node[thin,right] {line width is 5pt};

  \draw[-bad to,line width=0.1pt] (6cm,0ex) -- +(3,0)  node[thin,right] {line width is 0.1pt};
  \draw[-bad to,line width=0.4pt] (6cm,-2em) -- +(3,0) node[thin,right] {line width is 0.4pt};
  \draw[-bad to,line width=1.2pt] (6cm,-4em) -- +(3,0) node[thin,right] {line width is 1.2pt};
  \draw[-bad to,line width=5pt]   (6cm,-6em) -- +(3,0) node[thin,right] {line width is 5pt};
\end{tikzpicture}

\bigskip
As can be seen, simple scaling produces arrow tips that are way too
large at larger sizes and way too small at smaller sizes.



\subsection{Declaring an Arrow Tip Kind}

To declare an arrow kind ``from scratch,'' the following command is
used:

\begin{command}{\pgfarrowsdeclare\marg{start name}\marg{end
      name}\marg{extend code}\marg{arrow tip code}}
  This command declares a new arrow kind. An arrow kind has two names,
  which will typically be the same. When the arrow tip needs to be
  drawn, the \meta{arrow tip code} will be invoked, but the canvas
  trasformation is setup beforehand to a rotation such that when an
  arrow tip pointing right is specified, the arrow tip that is
  actually drawn points in the direction of the line.

  \medskip
  \textbf{Naming the arrow kind.}
  The \meta{start name} is the name
  used for the arrow when it is at the start of a path, the \meta{end
    name} is the name used at the end of a path. For example, the
  arrow kind that looks like a paranthesis has the \meta{start
    name} |(| and the \meta{end name} |)| so that you can say
  |\pgfsetarrows{(-)}| to specify that you want paranthesis arrows and
  both ends.

  The \meta{end name} and \meta{start name} can be quite arbitrary and
  may contain spaces.

  \medskip
  \textbf{Basics of the arrow tip code.}
  Let us next have a look at the \meta{arrow tip code}. This code will
  be used to draw the arrow when \pgfname\ thinks this is
  necessary. The code should draw an arrow that ``points right,''
  which means that is should draw an arrow at the end of a line coming
  from the left and ending at the origin.

  As an example, suppose we wanted to declare an arrow tip consisting
  of two arcs, that is, we want the arrow tip to look more or less
  like the red part of the following picture:
\begin{codeexample}[]
\begin{tikzpicture}[line width=3pt]
  \draw (-2,0) -- (0,0);
  \draw[red,join=round,cap=round]
        (-10pt,10pt) arc (180:270:10pt) arc (90:180:10pt);
\end{tikzpicture}
\end{codeexample}

  We could use the following as \meta{arrow tip code} for this:
\begin{codeexample}[code only]
\pgfarrowsdeclare{arcs}{arcs}{...}
{
  \pgfsetdash{}{0pt} % do not dash
  \pgfsetroundjoin   % fix join
  \pgfsetroundcap    % fix cap
  \pgfpathmoveto{\pgfpoint{-10pt}{10pt}}
  \pgfpatharc{180}{270}{10pt}
  \pgfpatharc{90}{180}{10pt}
  \pgfusepathqstroke
}
\end{codeexample}

  Indeed, when the |...| is set appropriately (in a moment), we can
  write the following:
\pgfarrowsdeclare{arcs}{arcs}{\pgfarrowsleftextend{0pt}\pgfarrowsrightextend{0pt}}
{
  \pgfsetdash{}{0pt} % do not dash
  \pgfsetroundjoin   % fix join
  \pgfsetroundcap    % fix cap
  \pgfpathmoveto{\pgfpoint{-10pt}{10pt}}
  \pgfpatharc{180}{270}{10pt}
  \pgfpatharc{90}{180}{10pt}
  \pgfusepathqstroke
}
\begin{codeexample}[]
\begin{tikzpicture}
  \draw[-arcs,line width=3pt] (-2,0)  -- (0,0);
  \draw[arcs-arcs,line width=1pt] (-2,-1.5) -- (0,-1);
\end{tikzpicture}
\end{codeexample}

  As can be seen in the second example, the arrow tip is automatically
  rotated as needed when the arrow is drawn. This is achieved by a
  canvas rotation.

  \medskip
  \textbf{Special considertations about the arrow tip code.}
  There are several things you need to be aware of when designing
  arrow tip code:
  \begin{itemize}
  \item
    Inside the code, you may not use the |\pgfusepath|
    command. The reason is that this command internally calls arrow
    construction commands, which something you obviously do not want
    to happen.

    Instead of |\pgfusepath|, use the quick versions. Typically, you
    will use |\pgfusepathqstroke|, |\pgfusepathqfill|, or
    |\pgfusepathqfillstroke|.
  \item
    The code will be executed only once, namely the first time the
    arrow tip needs to be drawn. The resulting low-level driver
    commands are protocolled and stored away. In all subsequent 
    uses of the arrow tip, the protocolled code is directly inserted.
  \item
    However, the code will be executed anew for each line width. Thus,
    an arrow of line width 2pt may result in a different protocol than
    the same arrow for a line width of 0.4pt.
  \item
    If you stroke the path that you construct, you should first set
    the dashing to solid and setup fixed joins and caps, as
    needed. This will ensure that the arrow tip will always look the
    same.
  \item
    When the arrow tip code is executed, it is automatically put
    inside a low-level scope, so nothing will ``leak out'' from the
    scope.
  \item
    The high-level coordinate transformation matrix will be set to the
    identity matrix when the code is executed for the first time.
  \end{itemize}

  \medskip
  \textbf{Designing meta-arrows.}
  The \meta{arrow tip code} should adjust the size of the arrow in
  accordance with the line width. For a small line width, the arrow
  tip should be small, for a large line width, it should be
  larger. However, the size of the arrow typically \emph{should not}
  grow linearly with the line width. On the other hand, the size of
  the arrow head typically \emph{should} grow ``a bit'' with the line
  width. 

  For these reasons, \pgfname\ will not simply executed your arrow
  code within a scaled scope, where the scaling depends on the line
  width. Instead, you \meta{arrow tip code} is reexecuted again for
  each different line width.

  In our example, we could use the following code for the new arrow
  tip kind |arc'| (note the prime):
\begin{codeexample}[code only]
\newdimen\arrowsize    
\pgfarrowsdeclare{arcs'}{arcs'}{...}
{
  \arrowsize=0.2pt
  \advance\arrowsize by .5\pgflinewidth
  \pgfsetdash{}{0pt} % do not dash
  \pgfsetroundjoin   % fix join
  \pgfsetroundcap    % fix cap
  \pgfpathmoveto{\pgfpoint{-4\arrowsize}{4\arrowsize}}
  \pgfpatharc{180}{270}{4\arrowsize}
  \pgfpatharc{90}{180}{4\arrowsize}
  \pgfusepathqstroke
}
\end{codeexample}
\newdimen\arrowsize    
\pgfarrowsdeclare{arcs'}{arcs'}{\pgfarrowsleftextend{0pt}\pgfarrowsrightextend{0pt}}
{
  \arrowsize=0.2pt
  \advance\arrowsize by .5\pgflinewidth
  \pgfsetdash{}{0pt} % do not dash
  \pgfsetroundjoin   % fix join
  \pgfsetroundcap    % fix cap
  \pgfpathmoveto{\pgfpoint{-4\arrowsize}{4\arrowsize}}
  \pgfpatharc{180}{270}{4\arrowsize}
  \pgfusepathqstroke
  \pgfpathmoveto{\pgfpointorigin}
  \pgfpatharc{90}{180}{4\arrowsize}
  \pgfusepathqstroke
}
\begin{codeexample}[]
\begin{tikzpicture}
  \draw[-arcs',line width=3pt] (-2,0)  -- (0,0);
  \draw[arcs'-arcs',line width=1pt] (-2,-1.5) -- (0,-1);
\end{tikzpicture}
\end{codeexample}
  
  However, sometimes, it can also be useful to have arrows that do not
  resize at all when the line width changes. This can be achieved by
  givin absolute size coordinates in the code, as done for |arc|. On
  the other hand, you can also have the arrow resize linearly with the
  line width by specifying all coordinates as multiples of
  |\pgflinewidth|.

  \textbf{The left and right extend.}
  Let us have another look at the exact left and right ``ends'' of our
  arrow tip. Let us draw the arrow tip |arc'| at a very large size:

\begin{codeexample}[]
\begin{tikzpicture}
  \draw[help lines] (-2,-1) grid (1,1);
  \draw[line width=10pt,-arcs'] (-2,0) -- (0,0);
  \draw[line width=2pt,white] (-2,0) -- (0,0);
\end{tikzpicture}
\end{codeexample}

  As one can see, the arrow tip does not ``touch'' the origin as it
  should, but protrudes a little over the origin. One remedy to this
  undesirable effect is to change the code of the arrow tip such that
  everything is shifted half an |\arrowsize| to the left. While this
  will cause the arrow tip to touch the origin, the line itself will
  then interfere with the arrow: The arrow tip will be partly
  ``hidden'' by the line itself.

  \pgfname\ uses a different approach to solving the problem: The
  \meta{extend code} argument can be used to ``tell'' \pgfname\ how
  much the arrow protrudes over the origin. The argument is also used
  to tell \pgfname\ where the ``left'' end of the arrow is. However,
  this number is important only when the arrow is being reversed.

  Once \pgfname\ knows the right extend of an arrow kind, it can
  \emph{shorten} lines by this amount when drawing arrows.

  Here is a picture that shows what the visualizes the extends. The
  arrow tip itself is shown in red once more:

  \medskip
  \begin{tikzpicture}
    \draw[line width=1cm,-arcs',red] (-6,0) -- (0,0);
    \draw[line width=1cm,black]      (-6,0) -- (0,0);
    \draw[help lines] (-6,0) -- (2,0)     (0,-3) -- (0,3) coordinate (a);
    \draw[help lines,xshift=0.5cm]        (0,-3) -- (0,3) coordinate (b);
    \draw[help lines,xshift=-2.5cm-0.8pt] (0,-3) -- (0,3) coordinate (c);

    \coordinate (xline 1) at (0,1.5);
    \coordinate (xline 2) at (0,2.8);
    
    \draw[|->|] (xline 1 -| a) -- node[above=2pt] {right extend} (xline 1 -| b);    
    \draw[|<-|] (xline 2 -| c) -- node[above=2pt] {left extend}  (xline 2 -| a);    
   \end{tikzpicture}
  

  The \meta{extend code} is normal \TeX\ code that is executed
  whenever \pgfname\ wants to know how far the arrow tip will protrude
  to the right and left. The code should call the following two
  commands: \declare{|\pgfarrowsrightextend|} and
  \declare{|\pgfarrowsleftextend|}. Both arguments take one argument
  that contain the size. Here is the final code for the |arc''| arrow
  tip: 
\begin{codeexample}[]
\pgfarrowsdeclare{arcs''}{arcs''}
{
  \arrowsize=0.2pt
  \advance\arrowsize by .5\pgflinewidth
  \pgfarrowsleftextend{-4\arrowsize-.5\pgflinewidth}
  \pgfarrowsrightextend{.5\pgflinewidth}
}
{
  \arrowsize=0.2pt
  \advance\arrowsize by .5\pgflinewidth
  \pgfsetdash{}{0pt} % do not dash
  \pgfsetroundjoin   % fix join
  \pgfsetroundcap    % fix cap
  \pgfpathmoveto{\pgfpoint{-4\arrowsize}{4\arrowsize}}
  \pgfpatharc{180}{270}{4\arrowsize}
  \pgfusepathqstroke
  \pgfpathmoveto{\pgfpointorigin}
  \pgfpatharc{90}{180}{4\arrowsize}
  \pgfusepathqstroke
}
\begin{tikzpicture}
  \draw[help lines] (-2,-1) grid (1,1);
  \draw[line width=10pt,-arcs''] (-2,0) -- (0,0);
  \draw[line width=2pt,white] (-2,0) -- (0,0);
\end{tikzpicture}
\end{codeexample}
\end{command}

\pgfarrowsdeclare{arcs''}{arcs''}
{
  \arrowsize=0.2pt
  \advance\arrowsize by .5\pgflinewidth
  \pgfarrowsleftextend{-4\arrowsize-.5\pgflinewidth}
  \pgfarrowsrightextend{.5\pgflinewidth}
}
{
  \arrowsize=0.2pt
  \advance\arrowsize by .5\pgflinewidth
  \pgfsetdash{}{0pt} % do not dash
  \pgfsetroundjoin   % fix join
  \pgfsetroundcap    % fix cap
  \pgfpathmoveto{\pgfpoint{-4\arrowsize}{4\arrowsize}}
  \pgfpatharc{180}{270}{4\arrowsize}
  \pgfusepathqstroke
  \pgfpathmoveto{\pgfpointorigin}
  \pgfpatharc{90}{180}{4\arrowsize}
  \pgfusepathqstroke
}


\subsection{Declaring a Derived Arrow Tip Kind}

It is possible to declare arrow kinds in terms of existing ones. For
these command to work correctly, the left and right extends must be
set correctly.

\begin{command}{\pgfarrowsdeclarealias\marg{start name}\marg{end
      name}\marg{old start name}\marg{old end name}}
  This command can be used to create an alias (another name) for an
  existing arrow kind.

\begin{codeexample}[]
\pgfarrowsdeclarealias{<}{>}{arcs''}{arcs''}%
\begin{pgfpicture}
  \pgfsetarrows{<->}
  \pgfsetlinewidth{1ex}
  \pgfpathmoveto{\pgfpointorigin}
  \pgfpathlineto{\pgfpoint{4cm}{2cm}}
  \pgfusepath{stroke}
\end{pgfpicture}
\end{codeexample}
\end{command}


\begin{command}{\pgfarrowsdeclarereversed\marg{start name}\marg{end
      name}\marg{old start name}\marg{old end name}}
  This command creates a new arrow kind that is the ``reverse'' of an
  existing arrow kind. The (automatically cerated) code of the new
  arrow kind will contain a flip of the canvas and the meanings of the
  left and right extend will be reversed. 

\begin{codeexample}[]
\pgfarrowsdeclarereversed{arcs reversed}{arcs reversed}{arcs''}{arcs''}%
\begin{pgfpicture}
  \pgfsetarrows{arcs reversed-arcs reversed}
  \pgfsetlinewidth{1ex}
  \pgfpathmoveto{\pgfpointorigin}
  \pgfpathlineto{\pgfpoint{4cm}{2cm}}
  \pgfusepath{stroke}
\end{pgfpicture}
\end{codeexample}
\end{command}



\begin{command}{\pgfarrowsdeclarecombine\opt{|*|}\opt{\oarg{offset}}\marg{start
      name}\marg{end name}\marg{first start name}\marg{first end
      name}\penalty0\marg{second start name}\marg{second end name}}
  This command creates a new arrow kind that combines two existing
  arrow kinds. The first arrow kind is the ``innermost'' arrow kind,
  the second arrow kind is the ``outermost.''

  The code for the combined arrow kind will install a canvas
  translation before the innermost arrow kind in drawn. This
  translation is calculated such that the right tip of the innermost
  arrow touches the right  end of the outermost arrow. The optional
  \meta{offset} can be used to increase (or decrease) the distance
  between the inner and outermost arrow.

\begin{codeexample}[]
\pgfarrowsdeclarecombine[\pgflinewidth]
  {combined}{combined}{arcs''}{arcs''}{latex}{latex}%
\begin{pgfpicture}
  \pgfsetarrows{combined-combined}
  \pgfsetlinewidth{1ex}
  \pgfpathmoveto{\pgfpointorigin}
  \pgfpathlineto{\pgfpoint{4cm}{2cm}}
  \pgfusepath{stroke}
\end{pgfpicture}
\end{codeexample}

  In the star variant, the end of the line is not in the outermost
  arrow, but inside the innermost arrow.

\begin{codeexample}[]
\pgfarrowsdeclarecombine*[\pgflinewidth]
  {combined'}{combined'}{arcs''}{arcs''}{latex}{latex}%
\begin{pgfpicture}
  \pgfsetarrows{combined'-combined'}
  \pgfsetlinewidth{1ex}
  \pgfpathmoveto{\pgfpointorigin}
  \pgfpathlineto{\pgfpoint{4cm}{2cm}}
  \pgfusepath{stroke}
\end{pgfpicture}
\end{codeexample}
\end{command}


\begin{command}{\pgfarrowsdeclaredouble\opt{\oarg{offset}}\marg{start
      name}\marg{end name}\marg{old start name}\marg{old end
      name}}
  This command is a shortcut for combining an arrow kind with itself.

\begin{codeexample}[]
\pgfarrowsdeclaredouble{<<}{>>}{arcs''}{arcs''}%
\begin{pgfpicture}
  \pgfsetarrows{<<->>}
  \pgfsetlinewidth{1ex}
  \pgfpathmoveto{\pgfpointorigin}
  \pgfpathlineto{\pgfpoint{4cm}{2cm}}
  \pgfusepath{stroke}
\end{pgfpicture}
\end{codeexample} 
\end{command}


\begin{command}{\pgfarrowsdeclaretriple\opt{\oarg{offset}}\marg{start
      name}\marg{end name}\marg{old start name}\marg{old end
      name}}
  This command is a shortcut for combining an arrow kind with itself
  and then again.

\begin{codeexample}[]
\pgfarrowsdeclaretriple{<<<}{>>>}{arcs''}{arcs''}%
\begin{pgfpicture}
  \pgfsetarrows{<<<->>>}
  \pgfsetlinewidth{1ex}
  \pgfpathmoveto{\pgfpointorigin}
  \pgfpathlineto{\pgfpoint{4cm}{2cm}}
  \pgfusepath{stroke}
\end{pgfpicture}
\end{codeexample} 
\end{command}





\subsection{Using an Arrow Tip Kind}

The following commands install the arrow kind that will be used when
stroking is done.

\begin{command}{\pgfsetarrowsstart\marg{start arrow kind}}
  Installs the given \meta{start arrow kind} for all subsequent
  strokes in the in the current \TeX-group. If \meta{start arrow kind}
  is empty, no arrow tips will be drawn at the start of paths.
\begin{codeexample}[]
\begin{pgfpicture}
  \pgfsetarrowsstart{latex}
  \pgfsetlinewidth{1ex}
  \pgfpathmoveto{\pgfpointorigin}
  \pgfpathlineto{\pgfpoint{4cm}{2cm}}
  \pgfusepath{stroke}
\end{pgfpicture}
\end{codeexample} 
\end{command}

\begin{command}{\pgfsetarrowsend\marg{start arrow kind}}
  Like |\pgfsetarrowsstart|, only for the end of the arrow.
\begin{codeexample}[]
\begin{pgfpicture}
  \pgfsetarrowsend{latex}
  \pgfsetlinewidth{1ex}
  \pgfpathmoveto{\pgfpointorigin}
  \pgfpathlineto{\pgfpoint{4cm}{2cm}}
  \pgfusepath{stroke}
\end{pgfpicture}
\end{codeexample} 
\end{command}

\emph{Warning:} If the compatibility mode is active (which is the
default), there also exist old commands called |\pgfsetstartarrow| and 
|\pgfsetendarrow|, which are incompatible with the meta-arrow
management.


\begin{command}{\pgfsetarrows\texttt{\char`\{}\meta{start kind}|-|\meta{end kind}\texttt{\char`\}}}
  Calls |\pgfsetarrowsstart| for \meta{start kind} and
  |\pgfsetarrowsend| for \meta{end kind}.
\begin{codeexample}[]
\begin{pgfpicture}
  \pgfsetarrows{latex-to}
  \pgfsetlinewidth{1ex}
  \pgfpathmoveto{\pgfpointorigin}
  \pgfpathlineto{\pgfpoint{4cm}{2cm}}
  \pgfusepath{stroke}
\end{pgfpicture}
\end{codeexample} 
\end{command}


\subsection{Predefined Arrow Tip Kinds}

\label{standard-arrows}

The following arrow tip kinds are always defined:

{
\bigskip
\catcode`\|=12
\begin{tabular}{ll}
  \sarrow{stealth}{stealth} \\
  \sarrow{stealth reversed}{stealth reversed}  \\
  \sarrow{to}{to} \\
  \sarrow{to reversed}{to reversed}  \\
  \sarrow{latex}{latex} \\
  \sarrow{latex reversed}{latex reversed}  \\
  \index{*vbar@\protect\texttt{\protect\myvbar} arrow tip}%
  \index{Arrow tips!*vbar@\protect\texttt{\protect\myvbar}}
  \texttt{|-|}& yields thick  
  \begin{tikzpicture}[arrows={|-|},thick]
    \useasboundingbox (0pt,-0.5ex) rectangle (1cm,2ex);
    \draw (0,0) -- (1,0);
  \end{tikzpicture} and thin
  \begin{tikzpicture}[arrows={|-|},thin]
    \useasboundingbox (0pt,-0.5ex) rectangle (1cm,2ex);
    \draw (0,0) -- (1,0);
  \end{tikzpicture}
\end{tabular}
}

For further arrow tips, see page~\pageref{section-library-arrows}.

% Copyright 2003 by Till Tantau <tantau@cs.tu-berlin.de>.
%
% This program can be redistributed and/or modified under the terms
% of the LaTeX Project Public License Distributed from CTAN
% archives in directory macros/latex/base/lppl.txt.


\section{Nodes and Shapes}

\label{section-shapes}

This section describes the |pgfbaseshapes| package.

\begin{package}{pgfbaseshapes}
  This package defines commands both for creating nodes and for
  creating shapes. The package is loaded automatically by |pgf|, but
  you can load it manually if you have  only included |pgfcore|.  
\end{package}


\subsection{Overview}

\pgfname\ comes with a sophisticated set of commands for creating
\emph{nodes} and \emph{shapes}. A \emph{node} is a graphical object
that consists (typically) of a text label and some additional stroked
or filled paths. Each node has a certain \emph{shape}, which may be
something simple like a |rectangle| or a |circle|, but it may also be
something complicated like a |uml class diagram| (this shape is
currently not implemented, though). Different nodes that have the same
shape may look quite different, however, since shapes (need not)
specify whether the shape path is stroked or filled.

\subsubsection{Creating and Referencing Nodes}

You create a node by calling the macro |\pgfnode|. This macro takes
several parameters and draws the requested shape at a certain
position. In addition, it will ``remember'' the node's position within
the current |{pgfpicture}|. You can then, later on, refer to the
node's position. Coordinate transformations are ``fully supported,''
which means that if you used coordinate transformations to shift or
rotate the shape of a node, the node's position will still be correctly
determined by \pgfname. This is \emph{not} the case if you use canvas
transformations, instead.

\subsubsection{Anchors}

An important property of a node or a shape in general are its
\emph{anchors}. Anchors are ``important'' positions in a shape. For
example, the |center| anchor lies at the center of a shape, the
|north| anchor is usually ``at the top, in the middle'' of a shape,
the |text| anchor is the lower left corner of the shape's label, and
so on.

Anchors are important both when you create a node and when you
reference it. When you create a node, you specify the node's
``position'' by asking \pgfname\ to place the shape in such a way that
a certain anchor lies at a certain point. For example, you might ask
that the node is placed such that the |north| anchor is at the
origin. This will effectively cause the node to be placed below the
origin.

When you reference a node, you always reference an anchor of the
node. For example, when you request the ``|north| anchor of the node
just placed'' you will get the origin. However, you can also request
the ``|south| anchor of this node,'' which will give you a point
somewhere below the origin. When a coordinate transformation was in
force at the time of creation of a node, all anchors are also
transformed accordingly.

\subsubsection{Layers of a Shape}

The simplest shape, the |coordinate|, has just one anchor, namely the
|center|, and a label (which is usually empty). More complicated
shapes like the |rectangle| shape also have a \emph{background
  path}. This is a \pgfname-path that is defined by the shape. The
shape does not prescribe what should happen with the path: When a node
is created this path may be stroked (resulting in a frame around the
label), filled (resulting in a background color for the text), or just
discarded.

Although most shapes consist just of a background path plus some label
text, when a shape is drawn, up to seven different layers are drawn:

\begin{enumerate}
\item
  The ``behind the background layer.'' Unlike the background path,
  which be used in different ways by different nodes, the graphic
  commands given for this layer will always stroke or
  always fill the path they construct. They might also insert some
  text that is ``behind everything.''
\item
  The background path layer. How this path is used depends on how the
  arguments of the |\pgfnode| command.
\item
  The ``before the background path layer.'' This layer works like the
  first one, only the commands of this layer are executed after the
  background path has been used (in whatever way the creator of the
  node chose).
\item
  The label layer. This layer inserts the node's text box.
\item
  The ``behind the foreground layer.'' This layer, like the
  first layer, once more contains graphic commands that are ``simply
  executed.''
\item
  The foreground path layer. This path is treated in the same way as the
  background path, only it is drawn only after the label text has been
  drawn.
\item
  The ``before the foreground layer.''
\end{enumerate}

Which of these layers are actually used depends on the shape.

\subsection{Creating Nodes}

You create a node using the following command:

\begin{command}{\pgfnode\marg{shape}\marg{anchor}\marg{label
      text}\marg{name}\marg{path usage command}} 
  This command creates a new node. The \meta{shape} of the node must
  have been declared previously using |\pgfdeclareshape|.

  The shape is shifted such that the \meta{anchor} is at the
  origin. In order to place the shape somewhere else, use the
  coordinate transformation prior to calling this command.

  The \meta{name} is a name for later reference. If no name is given,
  nothing will be ``saved'' for the node, it will just be drawn.

  The \meta{path usage command} is executed for the background and the
  foreground path (if the shape defines them).

\begin{codeexample}[]
\begin{tikzpicture}
  \draw[help lines] (0,0) grid (4,3);
  {
    \pgftransformshift{\pgfpoint{1cm}{1cm}}
    \pgfnode{rectangle}{north}{Hello World}{hellonode}{\pgfusepath{stroke}}
  }
  {
    \color{red!20}
    \pgftransformrotate{10}
    \pgftransformshift{\pgfpoint{3cm}{1cm}}
    \pgfnode{rectangle}{center}
      {\color{black}Hello World}{hellonode}{\pgfusepath{fill}}
  }
\end{tikzpicture}
\end{codeexample}

  As can be seen, all coordinate transformations are also applied to
  the text of the shape. Sometimes, it is desirable that the
  transformations are applied to the point where the shape will be
  anchored, but you do not wish the shape itself to the
  transformed. In this case, you should call
  |\pgftransformresetnontranslations| prior to calling the |\pgfnode|
  command. 

\begin{codeexample}[]
\begin{tikzpicture}
  \draw[help lines] (0,0) grid (4,3);
  {
    \color{red!20}
    \pgftransformrotate{10}
    \pgftransformshift{\pgfpoint{3cm}{1cm}}
    \pgftransformresetnontranslations
    \pgfnode{rectangle}{center}
      {\color{black}Hello World}{hellonode}{\pgfusepath{fill}}
  }
\end{tikzpicture}
\end{codeexample}
\end{command}

There are a number of values that have an influence on the size of a
node. These parameters can be changed using the following commands:

\begin{command}{\pgfsetshapeminwidth\marg{dimension}}
  This command sets the macro \declare{|\pgfshapeminwidth|} to
  \meta{dimension}. This dimension is the \emph{recommended} minimum
  width of a shape. Thus, when a shape is drawn and when the shape's
  width would be smaller than \meta{dimension}, the shape's width is
  enlarged by adding some empty space.

  Note that this value is just a recommendation. A shape may choose to
  ignore the value of |\pgfshapeminwidth|.
  
\begin{codeexample}[]
\begin{tikzpicture}
  \draw[help lines] (-2,0) grid (2,1);

  \pgfsetshapeminwidth{3cm}
  \pgfnode{rectangle}{center}{Hello World}{}{\pgfusepath{stroke}}
\end{tikzpicture}
\end{codeexample}
\end{command}

\begin{command}{\pgfsetshapeminheight\marg{dimension}}
  Works like |\pgfsetshapeminwidth|.
\end{command}


\begin{command}{\pgfsetshapeinnerxsep\marg{dimension}}
  This command sets the macro \declare{|\pgfshapeinnerxsep|} to
  \meta{dimension}. This dimension is the \emph{recommended} horizontal
  inner separation between the label text and the background path. As
  before, this value is just a recommendation and a shape may choose
  to ignore the value of |\pgfshapeinnerxsep|.
  
\begin{codeexample}[]
\begin{tikzpicture}
  \draw[help lines] (-2,0) grid (2,1);

  \pgfsetshapeinnerxsep{1cm}
  \pgfnode{rectangle}{center}{Hello World}{}{\pgfusepath{stroke}}
\end{tikzpicture}
\end{codeexample}
\end{command}

\begin{command}{\pgfsetshapeinnerysep\marg{dimension}}
  Works like |\pgfsetshapeinnerysep|.
\end{command}



\begin{command}{\pgfsetshapeouterxsep\marg{dimension}}
  This command sets the macro \declare{|\pgfshapeouterxsep|} to
  \meta{dimension}. This dimension is the recommended horizontal
  outer separation between the background path and the ``outer
  anchors.'' For example, if \meta{dimension} is |1cm| then the
  |east| anchor will be 1cm to the right of the right border of the
  background path. 

  As before, this value is just a recommendation.
  
\begin{codeexample}[]
\begin{tikzpicture}
  \draw[help lines] (-2,0) grid (2,1);

  \pgfsetshapeouterxsep{.5cm}
  \pgfnode{rectangle}{center}{Hello World}{x}{\pgfusepath{stroke}}

  \pgfpathcircle{\pgfpointanchor{x}{north}}{2pt}
  \pgfpathcircle{\pgfpointanchor{x}{south}}{2pt}
  \pgfpathcircle{\pgfpointanchor{x}{east}}{2pt}
  \pgfpathcircle{\pgfpointanchor{x}{west}}{2pt}
  \pgfpathcircle{\pgfpointanchor{x}{north east}}{2pt}
  \pgfusepath{fill}
\end{tikzpicture}
\end{codeexample}
\end{command}

\begin{command}{\pgfsetshapeouterysep\marg{dimension}}
  Works like |\pgfsetshapeouterysep|.
\end{command}


\subsection{Using Anchors}

Each shape defines a set of anchors. We saw already that the anchors
are used when the shape is drawn: the shape is placed in such a way
that the given anchor is at the origin (which in turn is typically
translated somewhere else).

One has to look up the set of anchors of each shape, there is no
``default'' set of anchors, except for the |center| anchor, which
should always be present.

Once a node has been defined, you can refer to its anchors using the
following commands:

\begin{command}{\pgfpointanchor\marg{node}\marg{anchor}}
  This command is another ``point command'' like the commands
  described in Section~\ref{section-points}. It returns the coordinate
  of the given \meta{anchor} in the given \meta{node}. The command can
  be used in commands like |\pgfpathmoveto|.

\begin{codeexample}[]
\begin{pgfpicture}
  \pgftransformrotate{30}
  \pgfnode{rectangle}{center}{Hello World!}{x}{\pgfusepath{stroke}}

  \pgfpathcircle{\pgfpointanchor{x}{north}}{2pt}
  \pgfpathcircle{\pgfpointanchor{x}{south}}{2pt}
  \pgfpathcircle{\pgfpointanchor{x}{east}}{2pt}
  \pgfpathcircle{\pgfpointanchor{x}{west}}{2pt}
  \pgfpathcircle{\pgfpointanchor{x}{north east}}{2pt}
  \pgfusepath{fill}
\end{pgfpicture}
\end{codeexample}

  In the above example, you may have noticed something curious: The
  rotation transformation is still in force when the anchors are
  invoked, but it does not seem to have an effect. You might expect
  that the rotation should apply to the already rotated points once
  more.

  However, |\pgfpointanchor| returns a point that takes the current
  transformation matrix into account: \emph{The inverse transformation
    to the current coordinate transformation is applied to an anchor
    point before returning it.}

  This behavior may seem a bit strange, but you will find it very
  natural in most cases. If you really want to apply a transformation
  to an anchor point (for example, to ``shift it away'' a little bit),
  you have to invoke |\pgfpointanchor| without any transformations in
  force. Here is an example:

\makeatletter
\begin{codeexample}[]
\begin{pgfpicture}
  \pgftransformrotate{30}
  \pgfnode{rectangle}{center}{Hello World!}{x}{\pgfusepath{stroke}}

  {
    \pgftransformreset
    \pgfpointanchor{x}{east}
    \xdef\mycoordinate{\noexpand\pgfpoint{\the\pgf@x}{\the\pgf@y}}
  }
    
  \pgfpathcircle{\mycoordinate}{2pt}
  \pgfusepath{fill}
\end{pgfpicture}
\end{codeexample}
\end{command}

\begin{command}{\pgfpointshapeborder\marg{node}\marg{point}}
  This command returns the point on the border of the shape that lies
  on a straight line from the center of the node to \meta{point}. For
  complex shapes it is not guaranteed that this point will actually
  lie on the border, it may be on the border of a ``simplified''
  version of the shape.

\begin{codeexample}[]
\begin{pgfpicture}
  \begin{pgfscope}
    \pgftransformrotate{30}
    \pgfnode{rectangle}{center}{Hello World!}{x}{\pgfusepath{stroke}}
  \end{pgfscope}
  \pgfpathcircle{\pgfpointshapeborder{x}{\pgfpoint{2cm}{1cm}}}{2pt}
  \pgfpathcircle{\pgfpoint{2cm}{1cm}}{2pt}
  \pgfpathcircle{\pgfpointshapeborder{x}{\pgfpoint{-1cm}{1cm}}}{2pt}
  \pgfpathcircle{\pgfpoint{-1cm}{1cm}}{2pt}
  \pgfusepath{fill}
\end{pgfpicture}
\end{codeexample}
\end{command}



\subsection{Declaring New Shapes}

Defining a shape is, unfortunately, a not-quite-trivial process. The
reason is that shapes need to be both very flexible (their size will
vary greatly according to circumstances) and they need to be
constructed reasonably ``fast.'' \pgfname\ must be able to handle
pictures with several hundreds of nodes and documents with thousands
of nodes in total. It would not do if \pgfname\ had to compute and
store, say, dozens of anchor positions for every node.

\subsubsection{What Must Be Defined For a Shape?}

In order to define a new shape, you must provide:
\begin{itemize}
\item
  a \emph{shape name},
\item
  code for computing the  \emph{saved anchors} and \emph{saved
    dimensions}, 
\item
  code for computing \emph{anchor} positions in terms of the saved anchors,
\item
  optionally code for the \emph{background path} and \emph{foreground path},
\item
  optionally code for \emph{things to be drawn before or behind} the
  background and foreground paths.
\end{itemize}


\subsubsection{Normal Anchors Versus Saved Anchors}

Anchors  are special places in shape. For example, the |north east|
anchor, which is a normal anchor, lies at the upper right corner of
the  |rectangle| shape, as does |\northeast|, which is a saved
anchor. The difference is the following: \emph{saved anchors are 
  computed and stored for each node, anchors are only computed as
  needed.} The user only has access to the normal anchors, but a
normal anchor can just ``copy'' or ``pass through'' the location of a
saved anchor. 

The idea behind all this is that a shape can declare a very large
number of normal anchors, but when a node of this shape is created,
these anchors are not actually computed. However, this causes a
problem: When we wish to reference an anchor of a node at some later
time, we must still able to compute the position of the anchor. For 
this, we may need a lot of information: What was the transformation
matrix that was in force when the node was created? What was the size
of the text box? What were the values of the different separation
dimensions? And so on. 

To solve this problem, \pgfname\ will always compute the locations of
all \emph{saved anchors} and store these positions. Then, when an
normal anchor position is requested later on, the anchor position can
be given just from knowing where the locations of the saved anchors.

As an example, consider the |rectangle| shape. For this shape two
anchors are saved: The |\northeast| corner and the |\southwest|
corner. A normal anchor like |north west| can now easily be expressed
in terms of these coordinates: Take the $x$-position of the
|\southwest| point  and the $y$-position of the |\northeast| point. 
The |rectangle| shape currently defines 13 normal anchors, but needs
only two saved anchors. Adding new anchors like a  |south south east|
anchor would not increase the memory and computation requirements of
pictures. 

All anchors (both saved and normal) are specified in a local
\emph{shape coordinate space}. This is also true for the background
and foreground paths. The |\pgfnode| macro will automatically apply
appropriate transformations to the coordinates so that the shape is
shifted to the right anchor or otherwise transformed. 


\subsubsection{Command for Declaring New Shapes}

The following command declares a new shape:
\begin{command}{\pgfdeclareshape\marg{shape name}\marg{shape
      specification}}
  This command declares a new shape named \meta{shape name}. The shape
  name can later be used in commands like |\pgfnode|.

  The \meta{shape specification} is some \TeX\ code containing calls
  to special commands that are only defined inside the \meta{shape
    specification} (similarly to commands like |\draw| that are only
  available inside the |{tikzpicture}| environment).

  \example Here is the code of the |coordinate| shape:
\begin{codeexample}[code only]
\pgfdeclareshape{coordinate}
{
  \savedanchor\centerpoint{%
    \pgf@x=.5\wd\pgfshapebox%
    \pgf@y=.5\ht\pgfshapebox%
    \advance\pgf@y by -.5\dp\pgfshapebox%
  }
  \anchor{center}{\centerpoint}
  \anchorborder{\centerpoint}
}
\end{codeexample}

  The special commands are explained next. In the examples given for
  the special commands a new shape will be constructed, which we might
  call |simple rectangle|. It should behave like the normal rectangle
  shape, only without bothering about the fine details like inner and
  outer separations. The skeleton for the shape is the following.
\begin{codeexample}[code only]
\pgfdeclareshape{simple rectangle}{
  ...
}
\end{codeexample}

  \begin{command}{\savedanchor\marg{command}\marg{code}}
    This command declares a saved anchor. The argument \meta{command}
    should be a \TeX\ macro name like |\centerpoint|.

    The \meta{code} will be executed each time |\pgfnode| is called to
    create a node of the shape \meta{shape name}. When the \meta{code}
    is executed, the box |\pgfshapebox| will contain the text label of
    the node. Possibly, this box is void. The \meta{code} can now
    use the width, height, and depth of the box to compute the
    location of the saved anchor. In addition, the \meta{code} can
    take into account the values of dimensions like
    |\pgfshapeminwidth| or |\pgfshapeinnerxsep|. Furthermore, the
    \meta{code} can take into consideration the values of any further
    shape-specific variables that are set at the moment when
    |\pgfnode| is called.

    The net effect of the \meta{code} should be to set the two \TeX\
    dimensions |\pgf@x| and |\pgf@y|. One way to achieve this is to
    say |\pgfpoint{|\meta{x value}|}{|\meta{y value}|}| at the end of
    the \meta{code}, but you can also just set these variables.
    The values that |\pgf@x| and |\pgf@y| have after the code has been
    executed, let us call them $x$ and $y$, will be recorded and
    stored together with the node that is created by the command
    |\pgfnode|.

    The macro \meta{command} is defined to be
    |\pgfpoint{|$x$|}{|$y$|}|. However, the \meta{command} is only
    locally defined while anchor positions are being computed. Thus,
    it is possible to use very simple names for \meta{command}, like
    |\center| or |\a|, without causing a name-clash. (To be precise,
    very simple \meta{command} names will clash with existing names,
    but only locally inside the computation of anchor positions; and
    we do not need the normal |\center| command during these
    computations.)

    For our |simple rectangle| shape, we will need only one saved
    anchor: The upper right corner. The lower left corner could either
    be the origin or the ``mirrored'' upper right corner, depending on
    whether we want the text label to have its lower left corner at
    the origin or whether the text label should be centered on the
    origin. Either will be fine, for the final shape this will make no
    difference since the shape will be shifted anyway. So, let us
    assume that the text label is centered on the origin (this will be
    specified later on using the |text| anchor). We get 
    the following code for the upper right corner:
\begin{codeexample}[code only]
\shapepoint{\upperrightcorner}{
  \pgf@y=.5\ht\pgfshapebox % height of the box, ignoring the depth
  \pgf@x=.5\wd\pgfshapebox % width of the box
}
\end{codeexample}

    If we wanted to take, say, the |\pgfshapeminwidth| into account,
    we could use the following code:
    
\begin{codeexample}[code only]
\shapepoint{\upperrightcorner}{
  \pgf@y=.\ht\pgfshapebox % height of the box
  \pgf@x=.\wd\pgfshapebox % width of the box
  \setlength{\pgf@xa}{\pgfshapeminwidth}
  \ifdim\pgf@x<.5\pgf@xa
    \pgf@x=.5\pgf@xa
  \fi
}
\end{codeexample}
    Note that we could not have written |.5\pgfshapeminwidth| since
    the minium width is stored in a ``plain text macro,'' not as a
    real dimension. So if |\pgfshapeminwidth| depth were 
    2cm, writing |.5\pgfshapeminwidth| would yield the same as |.52cm|.

    In the ``real'' |rectangle| shape the code is somewhat more
    complex, but you get the basic idea.
  \end{command}  
  \begin{command}{\saveddimen\marg{command}\marg{code}}
    This command is similar to |\savedanchor|, only instead of setting
    \meta{command} to |\pgfpoint{|$x$|}{|$y$|}|, the \meta{command} is
    set just to (the value of) $x$.

    In the |simple rectangle| shape we might use a saved dimension to
    store the depth of the shape box.
  
\begin{codeexample}[code only]
\shapedimen{\depth}{
  \pgf@x=\dp\pgfshapebox 
}
\end{codeexample}
  \end{command}  
  \begin{command}{\anchor\marg{name}\marg{code}}
    This command declares an anchor named \meta{name}. Unlike for saved
    anchors, the \meta{code} will not be executed each time a node is
    declared. Rather, the \meta{code} is only executed when the anchor
    is specifically requested; either for anchoring the node during
    its creation or as a  position in the shape referenced later on.

    The \meta{name} is a quite arbitrary string that is not ``passed
    down'' to the system level. Thus, names like |south| or |1| or
    |::| would all be fine.

    A saved anchor is not automatically also a normal anchor. If you
    wish to give the users access to a saved anchor you must declare a
    normal anchor that just returns the position of the saved anchor.

    When the \meta{code} is executed, all saved anchor macros will be
    defined. Thus, you can reference them in your \meta{code}. The
    effect of the \meta{code} should be to set the values of |\pgf@x|
    and |\pgf@y| to the coordinates of the anchor.

    Let us consider some example for the |simple rectangle|
    shape. First, we would like to make the upper right corner
    publicly available, for example as |north east|:
    
\begin{codeexample}[code only]
\anchor{north east}{\upperrightcorner}
\end{codeexample}

    The |\upperrightcorner| macro will set |\pgf@x| and |\pgf@y| to
    the coordinates of the upper right corner. Thus, |\pgf@x| and
    |\pgf@y| will have exactly the right values at the end of the
    anchor's code.

    Next, let us define a |north west| anchor. For this anchor, we can
    negate the |\pgf@x| variable:
   
\begin{codeexample}[code only]
\anchor{north west}{
  \upperrightcorner
  \pgf@x=-\pgf@x
}
\end{codeexample}

    Finally, it is a good idea to always define a |center| anchor,
    which will be the default location for a shape.

\begin{codeexample}[code only]
\anchor{center}{\pgfpointorigin}
\end{codeexample}

    You might wonder whether we should not take into consideration
    that the node is not placed at the origin, but has been shifted
    somewhere. However, the anchor positions are always specified in
    the shape's ``private'' coordinate system. The ``outer''
    transformation that has been applied to the shape upon its
    creation is applied automatically to the coordinates returned by
    the anchor's \meta{code}.

    There is one anchor that is special: The |text| anchor. This
    anchor is used upon creation of a node to determine the lower left
    corner of the text label (within the private coordinate system of
    the shape). By default, the |text| anchor is at the origin, but
    you may change this. For example, we would say
\begin{codeexample}[code only]
\anchor{text}{\pgfpoint{-.5\wd\pgfshapebox}{-.5\ht\pgfshapebox}}
\end{codeexample}
    to center the text label on the origin in the shape coordinate space. 
  \end{command}  
  \begin{command}{\anchorborder\marg{code}}
    A \emph{border anchor} is an anchor point on the border of the
    shape. What exactly is considered as the ``border'' of the shape
    depends on the shape.

    When the user request a point on the border of the shape using the
    |\pgfpointshapeborder| command, the \meta{code} will be executed
    to discern this point. When the execution of  the \meta{code}
    starts, the dimensions |\pgf@x| and |\pgf@y| will have been set to
    a location $p$ in the shape's coordinate system. It is now the job of
    the \meta{code} to setup |\pgf@x| and |\pgf@y| such that they
    specify the point on the shape's border that lies on a straight
    line from the shape's center to the point $p$. Usually, this is a
    somewhat complicated computation, involving many case distinctions
    and some basic math.

    For our |simple rectangle| we must compute a point on the border
    of a rectangle whose one corner is the origin (ignoring the depth
    for simplicity) and whose other corner is |\upperrightcorner|. The
    following code might be used:
\begin{codeexample}[code only]
\anchorborder{%
  % Call a function that computes a border point. Since this
  % function will modify dimensions like \pgf@x, we must move it to
  % other dimensions.
  \@tempdima=\pgf@x
  \@tempdimb=\pgf@y
  \pgfpointborderrectangle{\pgfpoint{\@tempdima}{\@tempdimb}}{\upperrightcorner}
}
\end{codeexample}
  \end{command}  
  \begin{command}{\backgroundpath\marg{code}}
    This command specifies the path that ``makes up'' the background
    of the shape. Note that the shape cannot prescribe what is going
    to happen with the path: It might be drawn, shaded, filled, or
    even thrown away. If you want to specify that something should
    ``always'' happen when this shape is drawn (for example, if the
    shape is a stop-sign, we \emph{always} want it to be filled with a
    red color), you can use commands like |\beforebackgroundpath|,
    explained below.

    When the \meta{code} is executed, all saved anchors will be in
    effect. The \meta{code} should contain path construction
    commands.

    For our |simple rectangle|, the following code might be used:
\begin{codeexample}[code only]
\backgroundpath{
  \pgfpathrectanglecorners
    {\upperrightcorner}
    {\pgfpointscale{-1}{\upperrightcorner}}
}  
\end{codeexample}
    As the name suggests, the background path is used ``behind'' the
    text label. Thus, this path is used first, then the text label is
    drawn, possibly obscuring part of the path.
  \end{command}  
  \begin{command}{\foregroundpath\marg{code}}
    This command works like |\backgroundpath|, only it is invoked
    after the text label has been drawn. This means that this path can
    possibly obscure (part of) the text label.
  \end{command}  
  \begin{command}{\behindbackgroundpath\marg{code}}
    Unlike the previous two commands, \meta{code} should not only
    construct a path, it should also use this path in whatever way is
    appropriate. For example, the \meta{code} might fill some area
    with a uniform color.

    Whatever the \meta{code} does, it does it first. This means that
    any drawing done by \meta{code} will be even behind the background
    path.

    Note that the \meta{code} is protected with a |{pgfscope}|.
  \end{command}  
  \begin{command}{\beforebackgroundpath\marg{code}}
    This command works like |\behindbackgroundpath|, only the
    \meta{code} is executed after the background path has been used,
    but before the text label is drawn.
  \end{command}  
  \begin{command}{\behindforegroundpath\marg{code}}
    The \meta{code} is executed after the text label has been drawn,
    but before the foreground path is used.
  \end{command}  
  \begin{command}{\beforeforegroundpath\marg{code}}
    This \meta{code} is executed at the very end.
  \end{command}  
  \begin{command}{\inheritsavedanchors|[from=|\marg{another shape name}|]|}
    This command allows you to inherit the code for saved anchors from
    \meta{another shape name}. The idea is that if you wish to create
    a new shape that is just a small modification of a another shape,
    you can recycle the code used for \meta{another shape name}.

    The effect of this command is the same as if you had called
    |\savedanchor| and |\saveddimen| for each saved anchor or saved
    dimension declared in \meta{another shape name}. Thus, it is not
    possible to ``selectively'' inherit only some saved anchors, you
    always have to inherit all saved anchors from another
    shape. However, you can inherit the saved anchors of more than one
    shape by calling this command several times.
  \end{command}  
  \begin{command}{\inheritbehindbackgroundpath|[from=|\marg{another shape name}|]|}
    This command can be used to inherit the code used for the
    drawings behind the background path from \meta{another shape name}. 
  \end{command}  
  \begin{command}{\inheritbackgroundpath|[from=|\marg{another shape name}|]|}
    Inherits the background path code from \meta{another shape name}.
  \end{command}  
  \begin{command}{\inheritbeforebackgroundpath|[from=|\marg{another shape name}|]|}
    Inherits the before background path code from \meta{another shape name}.
  \end{command}  
  \begin{command}{\inheritbehindforegroundpath|[from=|\marg{another shape name}|]|}
    Inherits the behind foreground path code from \meta{another shape name}.
  \end{command}  
  \begin{command}{\inheritforegroundpath|[from=|\marg{another shape name}|]|}
    Inherits the foreground path code from \meta{another shape name}.
  \end{command}  
  \begin{command}{\inheritbeforeforegroundpath|[from=|\marg{another shape name}|]|}
    Inherits the before foreground path code from \meta{another shape name}.
  \end{command}  
  \begin{command}{\inheritanchor|[from=|\marg{another shape name}|]|\marg{name}}
    Inherits the code of one specific anchor named \meta{name} from
    \meta{another shape name}. Thus, unlike saved anchors, which must
    be inherited collectively, normal anchors can and must be
    inherited individually.
  \end{command}  
  \begin{command}{\inheritanchorborder|[from=|\marg{another shape name}|]|}
    Inherits the border anchor code from \meta{another shape name}.
  \end{command}

  The following example shows how a shape can be defined that relies
  heavily on inheritance:
\makeatletter
\begin{codeexample}[]
\pgfdeclareshape{document}{
  \inheritsavedanchors[from=rectangle] % this is nearly a rectangle
  \inheritanchorborder[from=rectangle]
  \inheritanchor[from=rectangle]{center}
  \inheritanchor[from=rectangle]{north}
  \inheritanchor[from=rectangle]{south}
  \inheritanchor[from=rectangle]{west}
  \inheritanchor[from=rectangle]{east}
  % ... and possibly more
  \backgroundpath{% this is new
    % store lower right in xa/ya and upper right in xb/yb
    \southwest \pgf@xa=\pgf@x \pgf@ya=\pgf@y
    \northeast \pgf@xb=\pgf@x \pgf@yb=\pgf@y
    % compute corner of ``flipped page''
    \pgf@xc=\pgf@xb \advance\pgf@xc by-5pt % this should be a parameter
    \pgf@yc=\pgf@yb \advance\pgf@yc by-5pt
    % construct main path
    \pgfpathmoveto{\pgfpoint{\pgf@xa}{\pgf@ya}}
    \pgfpathlineto{\pgfpoint{\pgf@xa}{\pgf@yb}}
    \pgfpathlineto{\pgfpoint{\pgf@xc}{\pgf@yb}}
    \pgfpathlineto{\pgfpoint{\pgf@xb}{\pgf@yc}}
    \pgfpathlineto{\pgfpoint{\pgf@xb}{\pgf@ya}}
    \pgfpathclose
    % add little corner
    \pgfpathmoveto{\pgfpoint{\pgf@xc}{\pgf@yb}}
    \pgfpathlineto{\pgfpoint{\pgf@xc}{\pgf@yc}}
    \pgfpathlineto{\pgfpoint{\pgf@xb}{\pgf@yc}}
    \pgfpathlineto{\pgfpoint{\pgf@xc}{\pgf@yc}}
 }
}\hskip-1.2cm
\begin{tikzpicture}
  \node[shade,draw,shape=document,inner sep=2ex] (x) {Remark};
  \node[fill=examplefill,draw,ellipse,double]
    at ([shift=(-80:3cm)]x) (y) {Use Case};

  \draw[dashed] (x) -- (y);  
\end{tikzpicture}
\end{codeexample}
  
\end{command}




\subsection{Predefined Shapes}

\begin{shape}{rectangle}
  This shape is a rectangle tightly fitting the text box. Use inner or
  outer separation to increase the distance between the text box and
  the border and the anchors. The following figure shows the anchors
  defined by this shape; the anchors |10| and |130| are example of border
  anchors. 
\begin{codeexample}[]
\Huge
\begin{tikzpicture}
  \node[name=s,shape=rectangle,style=shape example] {Rectangle\vrule width 1pt height 2cm};
  \foreach \anchor/\placement in
    {north west/above left, north/above, north east/above right, 
     west/left, center/above, east/right, 
     mid west/right, mid/above, mid east/left, 
     base west/left, base/below, base east/right, 
     south west/below left, south/below, south east/below right, 
     text/left, 10/right, 130/above}
    \draw[shift=(s.\anchor)] plot[mark=x] coordinates {(0,0)}
      node[\placement] {\scriptsize\texttt{(s.\anchor)}};
\end{tikzpicture}
\end{codeexample}
\end{shape}



\begin{shape}{coordinate}
  The |coordinate| shape is a special shape that is mainly intended to
  be used to store locations using the node mechanism. This shape does
  not have any background path and options like |draw| have no effect on
  it. If you specify some text, this text will be typeset, but only ``a
  bit unwillingly'' since this shape is not really intended for drawing
  text.

  \tikzname\ handles this shape in a special way, see
  Section~\ref{section-tikz-coordinate-shape}. 
\end{shape}


\begin{shape}{circle}
  This shape is a circle tightly fitting the text box.
\begin{codeexample}[]
\Huge
\begin{tikzpicture}
  \node[name=s,shape=circle,style=shape example] {Circle\vrule width 1pt height 2cm};
  \foreach \anchor/\placement in
    {north west/above left, north/above, north east/above right, 
     west/left, center/above, east/right, 
     mid west/right, mid/above, mid east/left, 
     base west/left, base/below, base east/right, 
     south west/below left, south/below, south east/below right, 
     text/left, 10/right, 130/above}
     \draw[shift=(s.\anchor)] plot[mark=x] coordinates {(0,0)}
       node[\placement] {\scriptsize\texttt{(s.\anchor)}};
\end{tikzpicture}
\end{codeexample}
\end{shape}



%%% Local Variables: 
%%% mode: latex
%%% TeX-master: "pgfmanual"
%%% End: 

% Copyright 2003 by Till Tantau <tantau@cs.tu-berlin.de>.
%
% This program can be redistributed and/or modified under the terms
% of the LaTeX Project Public License Distributed from CTAN
% archives in directory macros/latex/base/lppl.txt.


\section{Coordinate and Canvas Transformations}

\subsection{Overview}

\pgfname\ offers two different ways of scaling, shifting, and rotating
(these operations are generally known as \emph{transformations}) or
your graphic: You can apply \emph{coordinate transformations} to all
coordinates and you can apply \emph{canvas transformations} to the
canvas on which you draw. (The names ``coordinate'' and ``canvas''
transformations are not standard, they are specially introduced for
the purposes of this manual.)

The difference is the following:

\begin{itemize}
\item
  As the name ``coordinate transformation'' suggests, coordinate
  transformations apply only to coordinates. For example, if you
  specify a coordinate like |\pgfpoint{1cm}{2cm}| and you wish to
  ``use'' this coordinate---for example as an argument toa
  |\pgfpathmoveto| command---then the coordinate transformation matrix
  is applied to the coordinate, resulting in a new coordinate. For
  example, if the current coordinate transformation is ``scale by a
  factor of two,'' the coordinate |\pgfpoint{1cm}{2cm}| actually
  designates the point $(2\mathrm{cm},4\mathrm{cm})$.

  Note that coordinate transformations apply \emph{only} to
  coordinates. They do not apply to, say, line width or shadings or
  text.
\item
  The effect of a ``canvas transformation'' like ``scale by a factor
  of two'' can be imagined as follows: You first draw your picture on
  a ``rubber canvas'' normally. Then, once you are done, the whole
  canvas is transformed, in this case stretched by a factor of
  two. In the resulting image \emph{everything} will be larger: Text,
  lines, coordinates, and shadings.
\end{itemize}

In many cases, it is preferable that you use coordinate
transformations and not canvas transformations. When canvas
transformations are used, \pgfname\ looses track of the coordinates of
nodes and shapes. Also, canvas transformations often cause undesirable
effects like changing text size. For these reasons, \pgfname\ makes it
easy to setup the coordinate transformation, but a bit harder to
change the canvas transformation.


\subsection{Coordinate Transformations}

\subsubsection{How PGF Keeps Track of the Coordinate Transformation
  Matrix}

\pgfname\ has an internal coordinate transformation matrix. This
matrix is applied to coordinates ``in certain situations.'' This means
that the matrix is not always applied to every coordinate ``no matter
what.'' Rather, \pgfname\ tries to be reasonably smart at when and how
this matrix should be applied. The most prominent examples are the
path construction commands, which apply the coordinate transformation
matrix to their inputs.

The coordinate transformation matrix consists of four numbers $a$,
$b$, $c$, and $d$, and two dimensions $s$ and $t$. When the coordinate
transformation matrix is applied to a coordinate $(x,y)$ the new
coordinate $(ax+by+s,cx+dy+t)$ results. For more details on how
transformation matrices work in general, please see, for example, the
\textsc{pdf} or PostScript reference or a textbook on computer
graphics.

The coordinate transformation matrix is equal to the identity matrix
at the beginning. More precisely, $a=1$, $b=0$, $c=0$, $d=1$,
$s=0\mathrm{pt}$, and $t=0\mathrm{pt}$.

The different coordinate transformation commands will modify the
matrix by concatenating it with another transformation matrix. This
way the effect of applying several transformation commands will
\emph{accumulate}.

The coordinate transformation matrix is local to the current \TeX\
group (unlike the canvas transformation matrix, which is local to the
current |{pgfscope}|). Thus, the effect of adding a coordinate
transformation to the coordinate transformation matrix will last only
till the end of the current \TeX\ group.




\subsubsection{Commands for Relative Coordinate Transformations}

The following commands add a basic coordinate transformation to the
current coordinate transformation matrix. For all commands, the
transformation is applied \emph{in addition} to any previous
coordinate transformations.

\begin{command}{\pgftransformshift\marg{point}}
  Shifts coordinates by \meta{point}.
\begin{codeexample}[]
\begin{tikzpicture}
  \draw[help lines] (0,0) grid (3,2);
  \draw      (0,0) -- (2,1) -- (1,0);
  \pgftransformshift{\pgfpoint{1cm}{1cm}}
  \draw[red] (0,0) -- (2,1) -- (1,0);
\end{tikzpicture}
\end{codeexample}
\end{command}

\begin{command}{\pgftransformxshift\marg{dimensions}}
  Shifts coordinates by \meta{dimension} along the $x$-axis.
\begin{codeexample}[]
\begin{tikzpicture}
  \draw[help lines] (0,0) grid (3,2);
  \draw      (0,0) -- (2,1) -- (1,0);
  \pgftransformxshift{.5cm}
  \draw[red] (0,0) -- (2,1) -- (1,0);
\end{tikzpicture}
\end{codeexample}
\end{command}

\begin{command}{\pgftransformyshift\marg{dimensions}}
  Like |\pgftransformxshift|, only for the $y$-axis.
\end{command}

\begin{command}{\pgftransformscale\marg{factor}}
  Scales coordinates by \meta{factor}.
\begin{codeexample}[]
\begin{tikzpicture}
  \draw[help lines] (0,0) grid (3,2);
  \draw      (0,0) -- (2,1) -- (1,0);
  \pgftransformscale{.75}
  \draw[red] (0,0) -- (2,1) -- (1,0);
\end{tikzpicture}
\end{codeexample}
\end{command}

\begin{command}{\pgftransformxscale\marg{factor}}
  Scales coordinates by \meta{factor} in the $x$-direction.
\begin{codeexample}[]
\begin{tikzpicture}
  \draw[help lines] (0,0) grid (3,2);
  \draw      (0,0) -- (2,1) -- (1,0);
  \pgftransformxscale{.75}
  \draw[red] (0,0) -- (2,1) -- (1,0);
\end{tikzpicture}
\end{codeexample}
\end{command}


\begin{command}{\pgftransformyscale\marg{factor}}
  Like |\pgftransformxscale|, only for the $y$-axis.
\end{command}


\begin{command}{\pgftransformxslant\marg{factor}}
  Slants coordinates by \meta{factor} in the $x$-direction. Here, a
  factor of |1| means $45^\circ$.
\begin{codeexample}[]
\begin{tikzpicture}
  \draw[help lines] (0,0) grid (3,2);
  \draw      (0,0) -- (2,1) -- (1,0);
  \pgftransformxslant{.5}
  \draw[red] (0,0) -- (2,1) -- (1,0);
\end{tikzpicture}
\end{codeexample}
\end{command}


\begin{command}{\pgftransformyslant\marg{factor}}
  Slants coordinates by \meta{factor} in the $y$-direction.
\begin{codeexample}[]
\begin{tikzpicture}
  \draw[help lines] (0,0) grid (3,2);
  \draw      (0,0) -- (2,1) -- (1,0);
  \pgftransformyslant{-1}
  \draw[red] (0,0) -- (2,1) -- (1,0);
\end{tikzpicture}
\end{codeexample}
\end{command}

  

\begin{command}{\pgftransformrotate\marg{degrees}}
  Rotates coordinates counterclockwise by \meta{degrees}.
\begin{codeexample}[]
\begin{tikzpicture}
  \draw[help lines] (0,0) grid (3,2);
  \draw      (0,0) -- (2,1) -- (1,0);
  \pgftransformrotate{30}
  \draw[red] (0,0) -- (2,1) -- (1,0);
\end{tikzpicture}
\end{codeexample}
\end{command}

  
\begin{command}{\pgftransformcm\marg{a}\marg{b}\marg{c}\marg{d}\marg{point}}
  Applies the transformation matrix given by $a$, $b$, $c$, and $d$
  and the shift \meta{point} to coordinates (in addition to any
  previous transformations already in force).
\begin{codeexample}[]
\begin{tikzpicture}
  \draw[help lines] (0,0) grid (3,2);
  \draw      (0,0) -- (2,1) -- (1,0);
  \pgftransformcm{1}{1}{0}{1}{\pgfpoint{.25cm}{.25cm}}
  \draw[red] (0,0) -- (2,1) -- (1,0);
\end{tikzpicture}
\end{codeexample}
\end{command}

  
\begin{command}{\pgftransformarrow\marg{start}\marg{end}}
  Shift coordinates to the end of the line going from \meta{start} 
  to \meta{end} with the correct rotation. 
\begin{codeexample}[]
\begin{tikzpicture}
  \draw[help lines] (0,0) grid (3,2);
  \draw      (0,0) -- (3,1);
  \pgftransformarrow{\pgfpointorigin}{\pgfpoint{3cm}{1cm}}
  \pgftext{tip}
\end{tikzpicture}
\end{codeexample}
\end{command}

  
\begin{command}{\pgftransformlineattime\marg{time}\marg{start}\marg{end}}
  Shifts coordinates by a specific point on a line at a specific
  time. The point by which the coordinate is shifted is calculated by
  calling |\pgfpointlineattime|, see
  Section~\ref{section-pointsattime}.

  In addition to shifting the coordinate, a rotation \emph{may} also
  be applied. Whether this is the case depends on whether the \TeX\ if
  |\ifpgfslopedattime| is set to true or not.
\begin{codeexample}[]
\begin{tikzpicture}
  \draw[help lines] (0,0) grid (3,2);
  \draw      (0,0) -- (2,1);
  \pgftransformlineattime{.25}{\pgfpointorigin}{\pgfpoint{2cm}{1cm}}
  \pgftext{Hi!}
\end{tikzpicture}
\end{codeexample}
\begin{codeexample}[]
\begin{tikzpicture}
  \draw[help lines] (0,0) grid (3,2);
  \draw      (0,0) -- (2,1);
  \pgfslopedattimetrue
  \pgftransformlineattime{.25}{\pgfpointorigin}{\pgfpoint{2cm}{1cm}}
  \pgftext{Hi!}
\end{tikzpicture}
\end{codeexample}

  There is another \TeX\ if that influences this command. If you set
  |\ifpgfresetnontranslationattime| to true, then, between
  shifting the coordinate and (possibly) rotating/sloping the
  coordinate, the command |\pgftransformresetnontranslations| is
  called. See the description of this command for details.
\begin{codeexample}[]
\begin{tikzpicture}
  \draw[help lines] (0,0) grid (3,2);
  \pgftransformscale{1.5}
  \draw      (0,0) -- (2,1);
  \pgfslopedattimetrue
  \pgfresetnontranslationattimefalse
  \pgftransformlineattime{.25}{\pgfpointorigin}{\pgfpoint{2cm}{1cm}}
  \pgftext{Hi!}
\end{tikzpicture}
\end{codeexample}
\begin{codeexample}[]
\begin{tikzpicture}
  \draw[help lines] (0,0) grid (3,2);
  \pgftransformscale{1.5}
  \draw      (0,0) -- (2,1);
  \pgfslopedattimetrue
  \pgfresetnontranslationattimetrue
  \pgftransformlineattime{.25}{\pgfpointorigin}{\pgfpoint{2cm}{1cm}}
  \pgftext{Hi!}
\end{tikzpicture}
\end{codeexample}
\end{command}


\begin{command}{\pgftransformcurveattime\marg{time}\marg{start}\marg{first
      support}\marg{second support}\marg{end}}
  Shifts coordinates by a specific point on a curve at a specific
  time, see  Section~\ref{section-pointsattime} once more.

  As for the line-at-time transformation command, |\ifpgfslopedattime|
  decides whether an additional rotation should be applied.
\begin{codeexample}[]
\begin{tikzpicture}
  \draw[help lines] (0,0) grid (3,2);
  \draw      (0,0) .. controls (0,2) and (1,2) .. (2,1);
  \pgftransformcurveattime{.25}{\pgfpointorigin}
    {\pgfpoint{0cm}{2cm}}{\pgfpoint{1cm}{2cm}}{\pgfpoint{2cm}{1cm}}
  \pgftext{Hi!}
\end{tikzpicture}
\end{codeexample}
\begin{codeexample}[]
\begin{tikzpicture}
  \draw[help lines] (0,0) grid (3,2);
  \draw      (0,0) .. controls (0,2) and (1,2) .. (2,1);
  \pgfslopedattimetrue
  \pgftransformcurveattime{.25}{\pgfpointorigin}
    {\pgfpoint{0cm}{2cm}}{\pgfpoint{1cm}{2cm}}{\pgfpoint{2cm}{1cm}}
  \pgftext{Hi!}
\end{tikzpicture}
\end{codeexample}
  The value of |\ifpgfresetnontranslationsattime| is also taken into account.
\end{command}


{
  \let\ifpgfslopedattime=\relax
  \begin{textoken}{\ifpgfslopedattime}
    Decides whether the ``at time'' transformation commands also
    rotate coordinates or not.
  \end{textoken}
}
{
  \let\ifpgfresetnontranslationsattime=\relax
  \begin{textoken}{\ifpgfresetnontranslationsattime}
    Decides whether the ``at time'' transformation commands should
    reset the non-translations between shifting and rotating.
  \end{textoken}
}


\subsubsection{Commands for Absolute Coordinate Transformations}

The coordinate transformation commands introduced above are always
applied in addition to any previous transformations. In contrast, the
commands presented in the following can be used to change the
transformation matrix ``absolutely.'' Note that this is, in general,
dangerous at will often produce unexpected effects. You should use
these commands only if you really know what you are doing.

\begin{command}{\pgftransformreset}
  Resets the coordinate transformation matrix to the identity
  matrix. Thus, once this command is given no transformations are
  applied till the end of the scope.
\begin{codeexample}[]
\begin{tikzpicture}
  \draw[help lines] (0,0) grid (3,2);
  \pgftransformrotate{30}
  \draw      (0,0) -- (2,1) -- (1,0);
  \pgftransformreset
  \draw[red] (0,0) -- (2,1) -- (1,0);
\end{tikzpicture}
\end{codeexample}
\end{command}


\begin{command}{\pgftransformresetnontranslations}
  This command sets the $a$, $b$, $c$, and $d$ part of the coordinate
  transformation matrix to $a=1$, $b=0$, $c=0$, and $d=1$. However,
  the current shifting of the matrix is not modified.

  The effect of this command is that any rotation/scaling/slanting is
  undone in the current \TeX\ group, but the origin is not ``moved
  back.''

  This command is mostly useful directly before a |\pgftext| command
  to ensure that the text is not scaled or rotated.
\begin{codeexample}[]
\begin{tikzpicture}
  \draw[help lines] (0,0) grid (3,2);
  \pgftransformscale{2}
  \pgftransformrotate{30}
  \pgftransformxshift{1cm}
  {\color{red}\pgftext{rotated}}
  \pgftransformresetnontranslations
  \pgftext{shifted only}
\end{tikzpicture}
\end{codeexample}
\end{command}


\begin{command}{\pgftransforminvert}
  Replaces the coordinate transformation matrix by a coordinate
  transformation matrix that ``exactly undoes the original
  transformation.'' For example, if the original transformation was
  ``scale by 2 and then shift right by 1cm'' the new one is ``shift
  left by 1cm and then scale by $1/2$.''

  This command will produce an error if the determinant of
  the matrix is too small, that is, if the matrix is near-singular.
\begin{codeexample}[]
\begin{tikzpicture}
  \draw[help lines] (0,0) grid (3,2);
  \pgftransformrotate{30}
  \draw      (0,0) -- (2,1) -- (1,0);
  \pgftransforminvert
  \draw[red] (0,0) -- (2,1) -- (1,0);
\end{tikzpicture}
\end{codeexample}
\end{command}

\subsubsection{Saving and Restoring the Coordinate Transformation
  Matrix}

There are two commands for saving and later on restoring coordinate
transformation matrices.

\begin{command}{\pgfgettransform\marg{macro}}
  This command will (locally) define \meta{macro} to a representation
  of the current coordinate transformation matrix. This matrix can
  later on be reinstalled using |\pgfsettransform|.
\end{command}


\begin{command}{\pgfsettransform\marg{macro}}
  Reinstalls a coordinate transformation matrix that was previously
  saved using |\pgfgettransform|.
\end{command}



\subsection{Canvas Transformations}

The canvas transformation matrix is not managed by \pgfname, but by
the output format like \pdf\ or PostScript. All the \pgfname\ does is
to call appropriate low-level |\pgfsys@| commands to change the canvas
transformation matrix.

Unlike coordinate transformations, canvas transformations apply to
``everything,'' including images, text, shadings, line thickness, and
so on. The idea is that a canvas transformation really stretches and
deforms the canvas after the graphic is finished.

Unlike coordinate transformations, canvas transformations are local to
the current |{pgfscope}|, not to the current \TeX\ group. This is due
to the fact that they are managed by the backend driver, not by \TeX\
or \pgfname.

Unlike the coordinate transformation matrix, it is not possible to
``reset'' the canvas transformation matrix. The only way to change it
is to concatenate it with another canvas transformation matrix or to
end the current |{pgfscope}|.

Unlike coordinate transformations, \pgfname\ does not ``keep track''
of canvas transformations. In particular, it will not be able to
correctly save the coordinates of shapes or nodes when a canvas
transformation is used.

\pgfname\ does not offer a whole set of special commands for modifying
the canvas transformation matrix. Instead, different commands allow
you to concatenate the canvas transformation matrix with a coordinate
transformation matrix (and there are numerous commands for specifying
a coordinate transformation, see the previous section).

\begin{command}{\pgflowlevelsynccm}
  This command concatenates the canvas transformation matrix with the
  current coordinate transformation matrix. Afterward, the coordinate
  transformation matrix is reset.

  The effect of this command is to ``synchronize'' the coordinate
  transformation matrix and the canvas transformation matrix. All
  transformations the were previously applied by the coordinate
  transformations matrix are now applied by the canvas transformation
  matrix.

\begin{codeexample}[]
\begin{tikzpicture}
  \draw[help lines] (0,0) grid (3,2);
  \pgfsetlinewidth{1pt}
  \pgftransformscale{5}
  \draw      (0,0) -- (0.4,.2);
  \pgftransformxshift{0.2cm}
  \pgflowlevelsynccm
  \draw[red] (0,0) -- (0.4,.2);
\end{tikzpicture}
\end{codeexample}
\end{command}


\begin{command}{\pgflowlevel\marg{transformation code}}
  This command concatenates the canvas transformation matrix with the
  coordinate transformation specified by \meta{transformation code}.

\begin{codeexample}[]
\begin{tikzpicture}
  \draw[help lines] (0,0) grid (3,2);
  \pgfsetlinewidth{1pt}
  \pgflowlevel{\pgftransformscale{5}}
  \draw      (0,0) -- (0.4,.2);
\end{tikzpicture}
\end{codeexample}
\end{command}


\begin{command}{\pgflowlevelobj\marg{transformation code}\marg{code}}
  This command creates a local |{pgfscope}|. Inside this scope,
  |\pgflowlevel| is first called with the argument
  \meta{transformation code}, then the \meta{code} is inserted. 

\begin{codeexample}[]
\begin{tikzpicture}
  \draw[help lines] (0,0) grid (3,2);
  \pgfsetlinewidth{1pt}
  \pgflowlevelobj{\pgftransformscale{5}}    {\draw (0,0) -- (0.4,.2);}
  \pgflowlevelobj{\pgftransformxshift{-1cm}}{\draw (0,0) -- (0.4,.2);}
\end{tikzpicture}
\end{codeexample}
\end{command}


\begin{environment}{{pgflowlevelscope}\marg{transformation code}}
  This environment first surrounds the \meta{environment contents} by
  a |{pgfscope}|. Then it calls |\pgflowlevel| with the argument
  \meta{transformation code}.

\begin{codeexample}[]
\begin{tikzpicture}
  \draw[help lines] (0,0) grid (3,2);
  \pgfsetlinewidth{1pt}
  \begin{pgflowlevelscope}{\pgftransformscale{5}}
    \draw (0,0) -- (0.4,.2);
  \end{pgflowlevelscope}
  \begin{pgflowlevelscope}{\pgftransformxshift{-1cm}}
    \draw (0,0) -- (0.4,.2);
  \end{pgflowlevelscope}
\end{tikzpicture}
\end{codeexample}
\end{environment}




%%% Local Variables: 
%%% mode: latex
%%% TeX-master: "pgfmanual"
%%% End: 

% Copyright 2003 by Till Tantau <tantau@cs.tu-berlin.de>.
%
% This program can be redistributed and/or modified under the terms
% of the LaTeX Project Public License Distributed from CTAN
% archives in directory macros/latex/base/lppl.txt.


\section{Declaring and Using Images}
\label{section-images}


This section describes the |pgfbaseimage| package.

\begin{package}{pgfbaseimage}
  This package offers an abstraction of the image inclusion
  process. It is loaded automatically by |pgf|, but you can load it
  manually if you have  only included |pgfcore|.  
\end{package}

\subsection{Overview}

To be quite frank, \LaTeX's |\includegraphics| is designed better than
|pgfbaseimage|. For this reason, \emph{I recommend that you use the
  standard image inclusion mechanism of your format}. Thus, \LaTeX\
users are encouraged to use |\includegraphics| to include images.

However, there are reasons why you might need to use the image
inclusion facilities of \pgfname:
\begin{itemize}
\item
  There is no standard image inclusion mechanism in your format. For
  example, plain \TeX\ does not have one, so \pgfname's inclusion
  mechanism is ``better than nothing.''

  However, this applies only to the |pdftex| backend. For all other
  backends, \pgfname\ currently maps its commands back to the |graphicx|
  package. Thus, in plain \TeX, this does not really help. It might be
  a good idea to fix this in the future such that \pgfname\ becomes
  independent of \LaTeX, thereby providing a uniform image abstraction
  for all formats. 
\item
  You wish to use masking. This is a feature that is only supported by
  \pgfname, though I hope that someone will implement this also for
  the graphics package in \LaTeX\ in the future.
\end{itemize}

Whatever your choice, you can still use the usual image inclusion
facilities of the |graphics| package.

The general approach taken by \pgfname\ to including an image is the
following: First, |\pgfdeclareimage| declares the
image. This must be done prior to the first use of the image. Once you
have declared an image, you can insert it into the text using
|\pgfuseimage|. The advantage of this two-phase approach is that, at
least for \textsc{pdf}, the image data will only be included once in the
file. This can drastically reduce the file size if you use an image
repeatedly, for example in an overlay. However, there is also a
command called |\pgfimage| that declares and then immediately uses the
image.

To speedup the compilation, you may wish to use the following class
option:
\begin{packageoption}{draft}
  In draft mode boxes showing the image name replace the
  images. It is checked whether the image files exist, but they are
  not read. If either height or width is not given, 1cm is used
  instead. 
\end{packageoption}

\subsection{Declaring an Image}

\begin{command}{\pgfdeclareimage\oarg{options}\marg{image
      name}\marg{filename}}
  Declares an image, but does not paint anything. To draw the image,
  use |\pgfuseimage{|\meta{image name}|}|. The \meta{filename} may not
  have an extension.  For \textsc{pdf}, the extensions |.pdf|, |.jpg|,
  and |.png| will automatically tried. For PostScript, the extensions
  |.eps|, |.epsi|, and |.ps| will be tried. 

  The following options are possible:
  \begin{itemize}
  \item
    \declare{|height=|\meta{dimension}} sets the height of the
    image. If the width is not specified simultaneously, the aspect
    ratio of the image is kept.
  \item
    \declare{|width=|\meta{dimension}} sets the width of the
    image. If the height is not specified simultaneously, the aspect
    ratio of the image is kept.
  \item
    \declare{|page=|\meta{page number}} selects a given page number
    from a multipage document. Specifying this option will have the
    following effect: first, \pgfname\ tries to find a file named
    \begin{quote}
      \meta{filename}|.page|\meta{page number}|.|\meta{extension}
    \end{quote}
    If such a file is found, it will be used instead of the originally
    specified filename. If not, \pgfname\ inserts the image stored in
    \meta{filename}|.|\meta{extension} and if a recent version of
    |pdflatex| is used, only the selected page is inserted. For older
    versions of |pdflatex| and for |dvips| the complete document is
    inserted and a warning is printed.    
  \item
    \declare{|interpolate=|\meta{true or false}} selects whether the
    image should ``smoothed'' when zoomed. False by default.
  \item
    \declare{|mask=|\meta{mask name}} selects a transparency mask. The
    mask must previously be declared using |\pgfdeclaremask| (see
    below). This option only has an effect for |pdf|. Not all viewers
    support masking. 
  \end{itemize}

\begin{codeexample}[code only]
\pgfdeclareimage[interpolate=true,height=1cm]{image1}{pgf-tu-logo}
\pgfdeclareimage[interpolate=true,width=1cm,height=1cm]{image2}{pgf-tu-logo}
\pgfdeclareimage[interpolate=true,height=1cm]{image3}{pgf-tu-logo}
\end{codeexample}
\end{command}


\begin{command}{\pgfaliasimage\marg{new image name}\marg{existing image name}}
  The \marg{existing image name} is ``cloned'' and the \marg{new image
    name} can now be used whenever original image is used. This
  command is useful for creating aliases for alternate extensions
  and for accessing the last image inserted using |\pgfimage|.

  \example |\pgfaliasimage{image.!30!white}{image.!25!white}|
\end{command}


\subsection{Using an Image}

\begin{command}{\pgfuseimage\marg{image name}}
  Inserts a previously declared image into the \emph{normal text}. If
  you wish to use it in a |{pgfpicture}| environment, you must put a
  |\pgftext| around it.

  If the macro |\pgfalternateextension| expands to some nonempty
  \meta{alternate extension}, \pgfname\ will first try to use the image
  names \meta{image name}|.|\meta{alternate extension}. If this
  image is not defined, \pgfname\ will next check whether \meta{alternate
    extension} contains a |!| character. If so, everything up to this
  exclamation mark and including it is deleted from \meta{alternate
    extension} and the \pgfname\ again tries to use the image \meta{image
    name}|.|\meta{alternate extension}. This is repeated until
  \meta{alternate extension} no longer contains a~|!|. Then the
  original image is used.

  The |xxcolor| package sets the alternate extension to the current
  color mixin. 

\begin{codeexample}[]
\pgfdeclareimage[interpolate=true,width=1cm,height=1cm]{image1}{pgf-tu-logo}
\pgfdeclareimage[interpolate=true,width=1cm]{image2}{pgf-tu-logo}
\pgfdeclareimage[interpolate=true,height=1cm]{image3}{pgf-tu-logo}
\begin{pgfpicture}
  \pgftext[at=\pgfpoint{1cm}{5cm},left,base]{\pgfuseimage{image1}}
  \pgftext[at=\pgfpoint{1cm}{3cm},left,base]{\pgfuseimage{image2}}
  \pgftext[at=\pgfpoint{1cm}{1cm},left,base]{\pgfuseimage{image3}}

  \pgfpathrectangle{\pgfpoint{1cm}{5cm}}{\pgfpoint{1cm}{1cm}}
  \pgfpathrectangle{\pgfpoint{1cm}{3cm}}{\pgfpoint{1cm}{1cm}}
  \pgfpathrectangle{\pgfpoint{1cm}{1cm}}{\pgfpoint{1cm}{1cm}}
  \pgfusepath{stroke}
\end{pgfpicture}
\end{codeexample}

  The following example demonstrates the effect of using
  |\pgfuseimage| inside a color mixin environment.

\begin{codeexample}[]
\pgfdeclareimage[interpolate=true,width=1cm,height=1cm]
  {image1.!25!white}{pgf-tu-logo.25}
\pgfdeclareimage[interpolate=true,width=1cm]
  {image2.25!white}{pgf-tu-logo.25}
\pgfdeclareimage[interpolate=true,height=1cm]
  {image3.white}{pgf-tu-logo.25}
\begin{colormixin}{25!white}
\begin{pgfpicture}
  \pgftext[at=\pgfpoint{1cm}{5cm},left,base]{\pgfuseimage{image1}}
  \pgftext[at=\pgfpoint{1cm}{3cm},left,base]{\pgfuseimage{image2}}
  \pgftext[at=\pgfpoint{1cm}{1cm},left,base]{\pgfuseimage{image3}}

  \pgfpathrectangle{\pgfpoint{1cm}{5cm}}{\pgfpoint{1cm}{1cm}}
  \pgfpathrectangle{\pgfpoint{1cm}{3cm}}{\pgfpoint{1cm}{1cm}}
  \pgfpathrectangle{\pgfpoint{1cm}{1cm}}{\pgfpoint{1cm}{1cm}}
  \pgfusepath{stroke}
\end{pgfpicture}
\end{colormixin}
\end{codeexample}
\end{command}

\begin{command}{\pgfalternateextension}
  You should redefine this command to install a different alternate
  extension.

  \example |\def\pgfalternateextension{!25!white}|
\end{command}


\begin{command}{\pgfimage\oarg{options}\marg{filename}}
  Declares the image under the name |pgflastimage| and
  immediately uses it. You can ``save'' the image for later usage by
  invoking |\pgfaliasimage| on |pgflastimage|.
  
\begin{codeexample}[]
\begin{colormixin}{25!white}
\begin{pgfpicture}
  \pgftext[at=\pgfpoint{1cm}{5cm},left,base]
    {\pgfimage[interpolate=true,width=1cm,height=1cm]{pgf-tu-logo}}
  \pgftext[at=\pgfpoint{1cm}{3cm},left,base]
    {\pgfimage[interpolate=true,width=1cm]{pgf-tu-logo}}
  \pgftext[at=\pgfpoint{1cm}{1cm},left,base]
    {\pgfimage[interpolate=true,height=1cm]{pgf-tu-logo}}

  \pgfpathrectangle{\pgfpoint{1cm}{5cm}}{\pgfpoint{1cm}{1cm}}
  \pgfpathrectangle{\pgfpoint{1cm}{3cm}}{\pgfpoint{1cm}{1cm}}
  \pgfpathrectangle{\pgfpoint{1cm}{1cm}}{\pgfpoint{1cm}{1cm}}
  \pgfusepath{stroke}
\end{pgfpicture}
\end{colormixin}
\end{codeexample}
\end{command}



\subsection{Masking an Image}


\begin{command}{\pgfdeclaremask\oarg{options}\marg{mask  name}\marg{filename}}
  Declares a transparency mask named \meta{mask name} (called a
  \emph{soft mask} in the \textsc{pdf} specification). This mask is
  read from the file \meta{filename}. This file should contain a
  grayscale image that is as large as the actual image. A white
  pixel in the mask will correspond to ``transparent,'' a black pixel
  to ``solid,'' and gray values correspond to intermediate values. The
  mask must have a single ``color channel.'' This means that the
  mask must be a ``real'' grayscale image, not an \textsc{rgb}-image
  in which all \textsc{rgb}-triples happen to have the same
  components.

  You can only mask images the are in a ``pixel format.'' These are
  |.jpg| and |.png|.  You cannot mask |.pdf| images in this way. Also,
  again, the mask file and the image file must have the same size.

  The following options may be given:
  \begin{itemize}
  \item |matte=|\marg{color components} sets the so-called
    \emph{matte} of the actual image (strangely, this has to be
    specified together with the mask, not with the image itself). The
    matte is the color that has been used to preblend the image. For
    example, if the image has been preblended with a red background,
    then \meta{color components} should be set to |{1 0 0}|. The
    default is |{1 1 1}|, which is white in the rgb model.

    The matte is specified in terms of the parent's image color
    space. Thus, if the parent is a grayscale image, the matte has to
    be set to |{1}|.
  \end{itemize}
  \example
\begin{codeexample}[]
%% Draw a large colorful background
\pgfdeclarehorizontalshading{colorful}{5cm}{color(0cm)=(red);
color(2cm)=(green); color(4cm)=(blue); color(6cm)=(red);
color(8cm)=(green); color(10cm)=(blue); color(12cm)=(red);
color(14cm)=(green)}
\hbox{\pgfuseshading{colorful}\hskip-14cm\hskip1cm
\pgfimage[height=4cm]{pgf-apple}\hskip1cm
\pgfimage[height=4cm]{pgf-apple.mask}\hskip1cm
\pgfdeclaremask{mymask}{pgf-apple.mask}
\pgfimage[mask=mymask,height=4cm,interpolate=true]{pgf-apple}}
\end{codeexample}
\end{command}

%%% Local Variables: 
%%% mode: latex
%%% TeX-master: "pgfmanual"
%%% End: 

% Copyright 2003 by Till Tantau <tantau@cs.tu-berlin.de>.
%
% This program can be redistributed and/or modified under the terms
% of the LaTeX Project Public License Distributed from CTAN
% archives in directory macros/latex/base/lppl.txt.


\section{Declaring and Using Shadings}

\label{section-shadings}

\subsection{Overview}

A shading is an area in which the color changes smoothly between different
colors. Note also that |ghostview| may do a poor job at displaying
shadings when doing anti-aliasing. 

Similarly to an image, a shading must first be declared before it can
be used. Also similarly to an image, a shading is put into a
\TeX-box. Hence, in order to include a shading in a |{pgfpicture}|,
you have to use |\pgftext| around it.

There are three kinds of shadings: horizontal, vertical, and radial
shadings. However, you can rotate and clip shadings like any other
graphics object, which allows you to create more complicated
shadings. Horizontal shadings could be created by rotating a vertical
shading by 90 degrees, but explicit commands for creating both
horizontal and vertical shadings are included for convenience.

Once you have declared a shading, you can insert it into text using
the command |\pgfuseshading|. This command cannot be used directly in
a |{pgfpicture}|, you have to put a |\pgftext| around it. The second
command for using shadings, |\pgfshadepath|, on the other hand, can
only be used  inside |{pgfpicture}| environments. It will ``fill'' the
current path with the shading.

A horizontal shading is a horizontal bar of a certain height whose
color changes smoothly. You must at least specify the colors at the
left and at the right end of the bar, but you can also add color
specifications for points inbetween. For example, suppose you
which to create a bar that is red at the left end, green in the
middle, and blue at the end. Suppose you would like the bar to be 4cm
long. This could be specified as follows:
\begin{codeexample}[code only]
rgb(0cm)=(1,0,0); rgb(2cm)=(0,1,0); rgb(4cm)=(0,0,1)
\end{codeexample}
This line means that at 0cm (the left end) of the bar, the color
should be red, which has red-green-blue (rgb) components (1,0,0). At
2cm, the bar should be green, and at 4cm it should be blue.
Instead of |rgb|, you can currently also specify |gray| as
color model, in which case only one value is needed, or |color|,
in which case you must provide the name of a color in round
brackets. In a color specification the individual specifications must
be separated using a semicolon, which may be followed by a whitespace
(like a space or a newline). Individual specifications must be given
in increasing order. 

\subsection{Declaring Shadings}

\begin{command}{\pgfdeclarehorizontalshading\oarg{color list}\marg{shading
      name}\marg{shading height}\marg{color specification}}
  Declares a horizontal shading named \meta{shading name} of the specified
  \meta{height} with the specified colors. The length of the bar is
  automatically deduced from the maximum specification.

\begin{codeexample}[]
\pgfdeclarehorizontalshading{myshading}
  {1cm}{rgb(0cm)=(1,0,0); color(2cm)=(green); color(4cm)=(blue)}
\pgfuseshading{myshading}
\end{codeexample}

  The effect of the \meta{color list}, which is a
  comma-separated list of colors, is the following: Normally, when
  this list is empty, once a shading is declared it becomes
  ``frozen.'' This means that even if you change a color that was used
  in the declaration of the shading later on, the shading will not
  change. By specifying a \meta{color list} you can specify
  that the shading should be recalculated whenever one of the colors
  listed in the list changes (this includes effects like color
  mixins). Thus, when you specify a \meta{color list},
  whenever the shading is used, \pgfname\ first converts the colors in the
  list to \textsc{rgb} triples using the current values of the
  colors and taking any mixins and blendings into account. If the
  resulting \textsc{rgb} triples have not yet been   used, a new
  shading is internally created and used. Note that if the 
  option \meta{color list} is used, then no shading is created until
  the first use of |\pgfuseshading|. In particular, the colors
  mentioned in the shading need not be defined when the declaration is
  given.

  When a shading is recalculated because of a change in the
  colors mentioned in \meta{color list}, the complete shading
  is recalculated. Thus even colors not mentioned in the list will be
  used with their current values, not with the values they had upon
  declaration.
  
\begin{codeexample}[]
\pgfdeclarehorizontalshading[mycolor]{myshading}
  {1cm}{rgb(0cm)=(1,0,0); color(2cm)=(mycolor)}
\colorlet{mycolor}{green}
\pgfuseshading{myshading}
\colorlet{mycolor}{blue}
\pgfuseshading{myshading}
\end{codeexample}
\end{command}


\begin{command}{\pgfdeclareverticalshading\oarg{color list}\marg{shading
      name}\marg{shading width}\marg{color specification}}
   Declares a vertical shading named \meta{shading name} of the
   specified \meta{width}. The height of the bar is automatically
   deduced from the maximum specification. The effect of \opt{color
     list} is the same as for horizontal shadings.

\begin{codeexample}[]
\pgfdeclareverticalshading{myshading2}
  {4cm}{rgb(0cm)=(1,0,0); rgb(1.5cm)=(0,1,0); rgb(2cm)=(0,0,1)}
\pgfuseshading{myshading2}
\end{codeexample}
\end{command}


\begin{command}{\pgfdeclareradialshading\oarg{color list}\marg{shading
      name}\marg{center point}\marg{color specification}}
  Declares an radial shading. A radial shading is a circle whose inner
  color changes as specified by the color specification. Assuming that
  the center of the shading is at the origin, the color of the center
  will be the color specified for 0cm and the color of the border of
  the circle will be the color for the maximum specification. The
  radius of the circle will be the maximum specification. If the
  center coordinate is not at the origin, the whole shading inside the
  circle (whose size remains exactly the same) will be distorted such
  that the given center now has the color specified for 0cm. The
  effect of \opt{color list} is the same as for horizontal shadings.

\begin{codeexample}[]  
\pgfdeclareradialshading{sphere}{\pgfpoint{0.5cm}{0.5cm}}%
  {rgb(0cm)=(0.9,0,0);
   rgb(0.7cm)=(0.7,0,0);
   rgb(1cm)=(0.5,0,0);
   rgb(1.05cm)=(1,1,1)}
\pgfuseshading{sphere}
\end{codeexample}
\end{command}


\begin{command}{\pgfaliasshading\marg{new shading name}\marg{existing shading name}}
  The \meta{existing shading name} is ``cloned'' and the shading
  \meta{new shading name} can now be used whenever original shading
  is used. This command is mainly useful for creating aliases for
  environments that use alternate extensions.
  \example \verb/\pgfaliasshading{shading!30}{shading!25}/
\end{command}


\subsection{Using Shadings}
\label{section-shading-a-path}

\begin{command}{\pgfuseshading\marg{shading name}}
  Inserts a previously declared shading into the text. If you wish to
  use it in a |pgfpicture| environment, you should put a |\pgfbox|
  around it. Like |\pgfuseimage|, alternate extensions are tried
  before the actual shading is used.

\begin{codeexample}[]
\begin{pgfpicture}
  \pgfdeclareverticalshading{shading}
    {20pt}{color(0pt)=(red); color(20pt)=(blue)}
  \pgftext[at=\pgfpoint{1cm}{0cm}]  {\pgfuseshading{shading}}
  \pgftext[at=\pgfpoint{2cm}{0.5cm}]{\pgfuseshading{shading}}
\end{pgfpicture}
\end{codeexample}
\end{command}

\begin{command}{\pgfshadepath\marg{shading name}\marg{angle}}
  This command must be used inside a |{pgfpicture}| environment. The
  effect is a bit complex, so let us go over it step by step.

  First, \pgfname\ will setup a local scope.

  Second, it uses the current path to clip everything inside this
  scope. However, the current path is once more available after the
  scope, so it can be used, for example, to stroke it.

  Now, the \meta{shading name} should be a shading whose width and
  height are 100\,bp, that is, 100 big points. \pgfname\ has a look at
  the bounding box of the current path. This bounding box is computed
  automatically when a path is computed; however, it can sometimes be
  (quite a bit) too large, especially when complicated curves are
  involved. 

  Inside the scope, the lowlevel transformation matrix is modified.
  The center of the shading is translated (moved) such that it lies on
  the center of the bounding box of the path. The lowlevel coordinate
  system is also scaled such that the shading ``covers'' the shading (the 
  details are a bit more complex, see below). Finally, the coordinate
  system is rotated by \meta{angle}.

  After everything has been set up, the shading is inserted. Due to
  the transformations and clippings, the effect will be that  the
  shading seems to ``fill'' the path.

  If both the path and the shadings were always be rectangles and if
  rotation were never involved, it would be easy to scale shadings
  such they always cover the path. However, when a vertical shading is
  rotated, must must obviously ``magnify'' the shading so that it
  still covers the path. Things get worse when the path is not a
  rectangle itself.

  For these reasons, things work slightly differently ``in reality.''
  The shading is scaled (more precisely, the coordinate system is
  scaled, but never mind the difference) and translated such that the
  the point $(50\mathrm{bp},50\mathrm{bp})$, which is the middle of
  the shading, is at the middle of the path and such that the the
  point $(25\mathrm{bp},25\mathrm{bp})$ is at the lower left corner of
  the path and that  $(75\mathrm{bp},75\mathrm{bp})$  is at upper
  right corner.

  In other words, only the center quarter of the shading will actually
  ``survive the clipping'' if the path is a rectangle. If it is not a
  rectangle, but, say, a circle, even less is seen of the
  shading. Here is an example that demonstrates this effect:

\begin{codeexample}[]
\pgfdeclareverticalshading{shading}{100bp}    
 {color(0bp)=(red); color(25bp)=(green);  color(75bp)=(blue);  color(100bp)=(black)}
\pgfuseshading{shading}
\hskip 1cm
\begin{pgfpicture}
  \pgfpathrectangle{\pgfpointorigin}{\pgfpoint{2cm}{1cm}}
  \pgfshadepath{shading}{0}
  \pgfusepath{stroke}
  \pgfpathrectangle{\pgfpoint{3cm}{0cm}}{\pgfpoint{1cm}{2cm}}
  \pgfshadepath{shading}{0}
  \pgfusepath{stroke}
  \pgfpathrectangle{\pgfpoint{5cm}{0cm}}{\pgfpoint{2cm}{2cm}}
  \pgfshadepath{shading}{45}
  \pgfusepath{stroke}
  \pgfpathcircle{\pgfpoint{9cm}{1cm}}{1cm}
  \pgfshadepath{shading}{45}
  \pgfusepath{stroke}
\end{pgfpicture}
\end{codeexample}

  As can be seen above in the last case, the ``hidden'' part of the
  shading actually \emph{can} become visible if the shading is
  rotated. The reason is that it is scaled as if no rotation took
  place, then the rotation is done.

  The following graphics show which part of the shading are actually
  shown: 

\begin{codeexample}[]
\begin{tikzpicture}
  \draw (50bp,50bp) node {\pgfuseshading{shading}};
  \draw[white,thick] (25bp,25bp) rectangle (75bp,75bp);
  \draw (50bp,0bp) node[below] {first two applications};

  \begin{scope}[xshift=5cm]
    \draw (50bp,50bp) node{\pgfuseshading{shading}};
    \draw[rotate around={45:(50bp,50bp)},white,thick] (25bp,25bp) rectangle (75bp,75bp);
    \draw (50bp,0bp) node[below] {third application};
  \end{scope}

  \begin{scope}[xshift=10cm]
    \draw (50bp,50bp) node{\pgfuseshading{shading}};
    \draw[white,thick] (50bp,50bp) circle (25bp);
    \draw (50bp,0bp) node[below] {fourth application};
  \end{scope}
\end{tikzpicture}
\end{codeexample}
  
  An advantage of this approach is that when you rotate a radial
  shading, no distortion is introduced:

\begin{codeexample}[]
\pgfdeclareradialshading{ballshading}{\pgfpoint{-10bp}{10bp}}
 {color(0bp)=(red!15!white); color(9bp)=(red!75!white);
 color(18bp)=(red!70!black); color(25bp)=(red!50!black); color(50bp)=(black)}
\pgfuseshading{ballshading}
\hskip 1cm
\begin{pgfpicture}
  \pgfpathrectangle{\pgfpointorigin}{\pgfpoint{1cm}{1cm}}
  \pgfshadepath{ballshading}{0}
  \pgfusepath{}
  \pgfpathcircle{\pgfpoint{3cm}{0cm}}{1cm}
  \pgfshadepath{ballshading}{0}
  \pgfusepath{}
  \pgfpathcircle{\pgfpoint{6cm}{0cm}}{1cm}
  \pgfshadepath{ballshading}{45}
  \pgfusepath{}
\end{pgfpicture}
\end{codeexample}

  If you specify a rotation of $90^\circ$
  and if the path is not a square, but an elongated rectangle,  the
  ``desired'' effect results: The shading will exactly vary between
  the colors at the 25bp and 75bp boundaries. Here is an example:
  
\begin{codeexample}[]
\begin{pgfpicture}
  \pgfpathrectangle{\pgfpointorigin}{\pgfpoint{2cm}{1cm}}
  \pgfshadepath{shading}{0}
  \pgfusepath{stroke}
  \pgfpathrectangle{\pgfpoint{3cm}{0cm}}{\pgfpoint{2cm}{1cm}}
  \pgfshadepath{shading}{90}
  \pgfusepath{stroke}
  \pgfpathrectangle{\pgfpoint{6cm}{0cm}}{\pgfpoint{2cm}{1cm}}
  \pgfshadepath{shading}{45}
  \pgfusepath{stroke}
\end{pgfpicture}
\end{codeexample}


  As a final example, let us define a ``rainbow spectrum'' shading for
  use with \tikzname.
\begin{codeexample}[]
\pgfdeclareverticalshading{rainbow}{100bp}
 {color(0bp)=(red); color(25bp)=(red); color(35bp)=(yellow);
  color(45bp)=(green); color(55bp)=(cyan); color(65bp)=(blue);
  color(75bp)=(violet); color(100bp)=(violet)}
\begin{tikzpicture}[shading=rainbow]
  \shade (0,0) rectangle node[white] {\textsc{pride}} (2,1);
  \shade[shading angle=90] (3,0) rectangle +(1,2);
\end{tikzpicture}
\end{codeexample}

  Note that rainbow shadings are \emph{way} to colorful in almost all
  applications. 
\end{command}

% Copyright 2003 by Till Tantau <tantau@cs.tu-berlin.de>.
%
% This program can be redistributed and/or modified under the terms
% of the LaTeX Project Public License Distributed from CTAN
% archives in directory macros/latex/base/lppl.txt.


\section{Creating Plots}

\label{section-plots}

This section describes the |pgfbaseplot| package.

\begin{package}{pgfbaseplot}
  This package provides a set of commands that are intended to make it
  reasonably easy to plot functions using \pgfname. It is loaded
  automatically by |pgf|, but you can load it manually if you have
  only included |pgfcore|.  
\end{package}

\subsection{Overview}

\subsubsection{When Should One Use PGF for Generating Plots? }

There exist many powerful programs that produce plots, examples are
\textsc{gnuplot} or \textsc{mathematica}. These programs can produce
two different kinds of output: First, they can output a complete plot
picture in a certain format (like \pdf) that includes all low-level
commands necessary for drawing the complete plot (including axes and
labels). Second, they can usually also produce ``just plain data'' in
the form of a long list of coordinates. Most of the powerful programs
consider it a to be ``a bit boring'' to just output tabled data and
very much prefer to produce fancy pictures. Nevertheless, when coaxed,
they can also provide the plain data.

The plotting mechanism described in the following deals only with
plotting data given in the form of a list of coordinates. Thus, this
section is about using \pgfname\ to turn lists of coordinates into
plots.

\emph{Note that is often not necessary to use \pgfname\ for this.}
Programs like \textsc{gnuplot} can produce very sophisticated plots
and it is usually much easier to simply include these plots as a
finished \textsc{pdf} or PostScript graphics.

However, there are a number of reasons why you may wish to invest time
and energy into mastering the \pgfname\ commands for creating plots:

\begin{itemize}
\item
  Virtually all plots produced by ``external programs'' use different
  fonts from the one used in your document.
\item
  Even worse, formulas will look totally different, if they can be
  rendered at all.
\item
  Line width will usually be too large or too small.
\item
  Scaling effects upon inclusion can create a mismatch between sizes
  in the plot and sizes in the text.
\item
  The automatic grid generated by most programs is mostly
  distracting. 
\item
  The automatic ticks generated by most programs are cryptic
  numerics. (Try adding a tick reading ``$\pi$'' at the right point.)
\item
  Most programs make it very easy to create ``chart junk'' in a most
  convenient fashion.  All show, no content.
\item
  Arrows and plot marks will almost never match the arrows used in the
  rest of the document.
\end{itemize}

The above list is not exhaustive, unfortunately.


\subsubsection{How PGF Handles Plots}

\pgfname\ (conceptually) uses a two-stage process for generating
plots. First, a \emph{plot stream} must be produced. This stream
consists (more or less) of a large number of coordinates. Second a 
\emph{plot handler} is applied to the stream. A plot handler ``does
something'' with the stream. The standard handler will issue
line-to operations to the coordinates in the stream. However, a
handler might also try to issue appropriate curve-to operations in
order to smooth the curve. A handler may even do something else
entirely, like writing each coordinate to another stream, thereby
duplicating the original stream.

Both for the creation of streams and the handling of streams different
sets of commands exist. The commands for creating streams start with
|\pgfplotstream|, the commands for setting the handler start with
|\pgfplothandler|.



\subsection{Generating Plot Streams}

\subsubsection{Basic Building Blocks of Plot Streams}
A \emph{plot stream} is a (long) sequence of the following three
commands:
\begin{enumerate}
\item
  |\pgfplotstreamstart|,
\item
  |\pgfplotstreampoint|, and
\item
  |\pgfplotstreamend|.
\end{enumerate}
Between calls of these commands arbitrary other code may be
called. Obviously, the stream should start with the first command and
end with the last command. Here is an example of a plot stream:
\begin{codeexample}[code only]
\pgfplotstreamstart
\pgfplotstreampoint{\pgfpoint{1cm}{1cm}}
\newdimen\mydim
\mydim=2cm
\pgfplotstreampoint{\pgfpoint{\mydim}{2cm}}
\advance \mydim by 3cm
\pgfplotstreampoint{\pgfpoint{\mydim}{2cm}}
\pgfplotstreamend
\end{codeexample}

\begin{command}{\pgfplotstreamstart}
  This command signals that a plot stream starts. The effect of this
  command is to call the internal command |\pgf@plotstreamstart|,
  which is set by the current plot handler to do whatever needs to be
  done at the beginning of the plot.
\end{command}

\begin{command}{\pgfplotstreampoint\marg{point}}
  This command adds a \meta{point} to the current plot stream. The
  effect of this command is to call the internal command |\pgf@plotstreampoint|,
  which is also set by the current plot handler. This command should
  now ``handle'' the point in some sensible way. For example, a
  line-to command might be issued for the point.
\end{command}

\begin{command}{\pgfplotstreamend}
  This command signals that a plot stream ends. It calls
  |\pgf@plotstreamend|, which should now do any necessary ``cleanup.''
\end{command}

Note that plot streams are not buffered, that is, the different points
are handled immediately. However, using the recording handler, it is
possible to record a stream.

\subsubsection{Commands That Generate Plot Streams}

Plot streams can be created ``by hand'' as in the earlier
example. However, most of the time the coordinates will be produced
internally by some command. For example, the |\pgfplotxyfile| reads a
file and converts it into a plot stream.

\begin{command}{\pgfplotxyfile\marg{filename}}
  This command will try to open the file \meta{filename}. If this
  succeeds, it will convert the file contents into a plot stream as
  follows: A |\pgfplotstreamstart| is issued. Then, each nonempty line
  of the file should start with two numbers separated by a space, such
  as |0.1 1| or |100 -.3|. Anything following the numbers is ignored.

  Each pair \meta{x} and \meta{y} of numbers is converted into one
  plot stream point in the xy-coordinate system. Thus, a line like
\begin{codeexample}[code only]
2 -5 some text
\end{codeexample}
  is turned into 
\begin{codeexample}[code only]
\pgfplotstreampoint{\pgfpointxy{2}{-5}}
\end{codeexample}

  The two characters |%| and |#| are also allowed in a file and they
  are both treated as comment characters. Thus, a line starting with
  either of them is empty and, hence, ignored.

  When the file has been read completely, |\pgfplotstreamend| is
  called. 
\end{command}


\begin{command}{\pgfplotxyzfile\marg{filename}}
  This command works like |\pgfplotxyfile|, only \emph{three} numbers
  are expected on each non-empty line. They are converted into points
  in the xyz-coordinate system. Consider, the following file:
\begin{codeexample}[code only]
% Some comments
# more comments
2 -5  1 first entry
2 -.2 2 second entry
2 -5  2 third entry
\end{codeexample}
  It is turned into the following stream:
\begin{codeexample}[code only]
\pgfplotstreamstart
\pgfplotstreampoint{\pgfpointxyz{2}{-5}{1}}
\pgfplotstreampoint{\pgfpointxyz{2}{-.2}{2}}
\pgfplotstreampoint{\pgfpointxyz{2}{-5}{2}}
\pgfplotstreamend
\end{codeexample}
\end{command}


Currently, there is no command that can decide automatically whether
the xy-coordinate system should be used or whether the xyz-system
should be used. However, it would not be terribly difficult to write a
``smart file reader'' that parses coordinate files a bit more
intelligently. 


\begin{command}{\pgfplotgnuplot\oarg{prefix}\marg{function}}
  This command will ``try'' to call the \textsc{gnuplot} program to
  generate the coordinates of the \meta{function}. In detail, the
  following happens:

  This command works with two files: \meta{prefix}|.gnuplot| and
  \meta{prefix}|.table|.  If the optional argument \meta{prefix} is
  not given, it is set to |\jobname|.

  Let us start with the situation where none of these files
  exists. Then \pgfname\ will first generate the file
  \meta{prefix}|.gnuplot|. In this file it writes
\begin{codeexample}[code only]
set terminal table; set output "#1.table"; set format "%.5f"
\end{codeexample}
  where |#1| is replaced by \meta{prefix}. Then, in a second line, it
  writes the text \meta{function}.

  Next, \pgfname\ will try to invoke the program |gnuplot| with the
  argument \meta{prefix}|.gnuplot|. This call may or may not succeed,
  depending on whether the |\write18| mechanism (also known as
  shell escape) is switched on and whether the |gnuplot| program is
  available.

  Assuming that the call succeeded, the next step is to invoke
  |\pgfplotxyfile| on the file \meta{prefix}|.table|; which is exactly
  the file that has just been created by |gnuplot|.
  
\begin{codeexample}[]
\begin{tikzpicture}
  \draw[help lines] (0,-1) grid (4,1);
  \pgfplothandlerlineto
  \pgfplotgnuplot[plots/pgfplotgnuplot-example]{plot [x=0:3.5] x*sin(x)}
  \pgfusepath{stroke}
\end{tikzpicture}
\end{codeexample}

  The more difficult situation arises when the |.gnuplot| file exists,
  which will be the case on the second run of \TeX\ on the \TeX\
  file. In this case \pgfname\ will read this file and check whether
  it contains exactly what \pgfname\ ``would have written'' into
  this file. If this is not the case, the file contents is overwritten
  with what ``should be there'' and, as above, |gnuplot| is invoked to
  generate a new |.table| file. However, if the file contents is ``as
  expected,'' the external |gnuplot| program is \emph{not}
  called. Instead, the \meta{prefix}|.table| file is immediately
  read.

  As explained in Section~\ref{section-tikz-gnuplot}, the net effect
  of the above mechanism is that |gnuplot| is called as little as
  possible and that when you pass along the |.gnuplot| and |.table|
  files with your |.tex| file to someone else, that person can
  \TeX\ the |.tex| file without having |gnuplot| installed and without
  having the |\write18| mechanism switched on.
\end{command}



\subsection{Plot Handlers}

\label{section-plot-handlers}

A \emph{plot handler}  prescribes what ``should be done'' with a
plot stream. You must set the plot handler before the stream starts.
The following commands install the most basic plot handlers; more plot
handlers are defined in the file |pgflibraryplothandlers|, which is
documented in Section~\ref{section-library-plothandlers}.

All plot handlers work by setting redefining the following three
macros: |\pgf@plotstreamstart|, |\pgf@plotstreampoint|, and
|\pgf@plotstreamend|.

\begin{command}{\pgfplothandlerlineto}
  This handler will issue a |\pgfpathlineto| command for each point of
  the plot, \emph{except} possibly for the first. What happens with
  the first point can be specified using the two commands described
  below.

\begin{codeexample}[]
\begin{pgfpicture}
  \pgfpathmoveto{\pgfpointorigin}
  \pgfplothandlerlineto
  \pgfplotstreamstart
  \pgfplotstreampoint{\pgfpoint{1cm}{0cm}}
  \pgfplotstreampoint{\pgfpoint{2cm}{1cm}}
  \pgfplotstreampoint{\pgfpoint{3cm}{2cm}}
  \pgfplotstreampoint{\pgfpoint{1cm}{2cm}}
  \pgfplotstreamend
  \pgfusepath{stroke}
\end{pgfpicture}
\end{codeexample}
\end{command}

\begin{command}{\pgfsetmovetofirstplotpoint}
  Specifies that the line-to plot handler (and also some other plot 
  handlers) should issue a move-to command for the
  first point of the plot instead of a line-to. This will start a new
  part of the current path, which is not always, but often,
  desirable. This is the default.
\end{command}

\begin{command}{\pgfsetlinetofirstplotpoint}
  Specifies that  plot handlers should issue a line-to command for the
  first point of the plot.

\begin{codeexample}[]
\begin{pgfpicture}
  \pgfpathmoveto{\pgfpointorigin}
  \pgfsetlinetofirstplotpoint
  \pgfplothandlerlineto
  \pgfplotstreamstart
  \pgfplotstreampoint{\pgfpoint{1cm}{0cm}}
  \pgfplotstreampoint{\pgfpoint{2cm}{1cm}}
  \pgfplotstreampoint{\pgfpoint{3cm}{2cm}}
  \pgfplotstreampoint{\pgfpoint{1cm}{2cm}}
  \pgfplotstreamend
  \pgfusepath{stroke}
\end{pgfpicture}
\end{codeexample}
\end{command}

\begin{command}{\pgfplothandlerdiscard}
  This handler will simply throw away the stream.
\end{command}

\begin{command}{\pgfplothandlerrecord\marg{macro}}
  When this handler is installed, each time a plot stream command is
  called, this command will be appended to \meta{macros}. Thus, at
  the end of the stream, \meta{macro} will contain all the
  commands that were issued on the stream. You can then install
  another handler and invoke \meta{macro} to ``replay'' the stream
  (possibly many times).
 
\begin{codeexample}[]
\begin{pgfpicture}
  \pgfplothandlerrecord{\mystream}
  \pgfplotstreamstart
  \pgfplotstreampoint{\pgfpoint{1cm}{0cm}}
  \pgfplotstreampoint{\pgfpoint{2cm}{1cm}}
  \pgfplotstreampoint{\pgfpoint{3cm}{1cm}}
  \pgfplotstreampoint{\pgfpoint{1cm}{2cm}}
  \pgfplotstreamend
  \pgfplothandlerlineto
  \mystream
  \pgfplothandlerclosedcurve
  \mystream
  \pgfusepath{stroke}
\end{pgfpicture}
\end{codeexample} 
\end{command}

%%% Local Variables: 
%%% mode: latex
%%% TeX-master: "pgfmanual"
%%% End: 

% Copyright 2003 by Till Tantau <tantau@cs.tu-berlin.de>.
%
% This program can be redistributed and/or modified under the terms
% of the LaTeX Project Public License Distributed from CTAN
% archives in directory macros/latex/base/lppl.txt.


\section{Layered Graphics}

\label{section-layers}

\begin{package}{pgfbaselayers}
  This package provides a commands and environments for composing a
  picture from multiple layers. The package is loaded automatically by
  |pgf|, but you can load it manually if you have only included
  |pgfcore|.   
\end{package}



\subsection{Overview}

\pgfname\ provides a layering mechanism for composing graphics from
multiple layers. (This mechanism is not be confused with the
conceptual ``software layers'' the \pgfname\ system is composed of.)
Layers are often used in graphic programs. The idea is that you can
draw on the different layers in any order. So you might start drawing
something on the ``background'' layer, then something on the
``foreground'' layer, then something on the ``middle'' layer, and then
something on the background layer once more, and so on. At the end, no
matter in which ordering you drew on the different layers, the layers
are ``stacked on top of each other'' in a fixed ordering to produce
the final picture. Thus, anything drawn on the middle layer would come
on top of everything of the background layer.

Normally, you do not need to use different layers since you will have
little trouble ``ordering'' your graphic commands in such a way that
layers are superfluous. However, in certain situations you only
``know'' what you should draw behind something else after the
``something else'' has been drawn.

For example, suppose you wish to draw a yellow background behind your
picture. The background should be as large as the bounding box of the
picture, plus a little border. If you know the size of the bounding box
of the picture at its beginning, this is easy to accomplish. However,
in general this is not the case and you need to create a
``background'' layer in addition to the standard ``main'' layer. Then,
at the end of the picture, when the bounding box has been established,
you can add a rectangle of the appropriate size to the picture.



\subsection{Declaring Layers}

In \pgfname\ layers are referenced using names. The standard layer,
which is a bit special in certain ways, is called |main|. If nothing
else is specified, all graphic commands are added to the |main|
layer. You can declare a new layer using the following command:

\begin{command}{\pgfdeclarelayer\marg{name}}
  This command declares a layer named \meta{name} for later
  use. Mainly, this will setup some internal bookkeeping.
\end{command}

The next step toward using a layer is to tell \pgfname\ which layers
will be part of the actual picture and which will be their
ordering. Thus, it is possible to have more layers declared than are
actually used.

\begin{command}{\pgfsetlayers\marg{layer list}}
  This command, which should be used \emph{outside} a |{pgfpicture}|
  environment, tells \pgfname\ which layers will be used in
  pictures. They are stacked on top of each other in the order
  given. The layer |main| should always be part of the list. Here is
  an example:
\begin{codeexample}[code only]
\pgfdeclarelayer{background}
\pgfdeclarelayer{foreground}  
\pgfsetlayers{background,main,foreground}
\end{codeexample}
\end{command}


\subsection{Using Layers}

Once the layers of your picture have been declared, you can start to
``fill'' them. As said before, all graphics commands are normally
added to the |main| layer. Using the |{pgfonlayer}| environment, you
can tell \pgfname\ that certain commands should, instead, be added to
the given layer.

\begin{environment}{{pgfonlayer}\marg{layer name}}
  The whole \meta{environment contents} is added to the layer with the
  name \meta{layer name}. This environment can be used anywhere inside
  a picture. Thus, even if it is used inside a |{pgfscope}| or a \TeX\
  group, the contents will still be added to the ``whole'' picture.
  Using this environment multiple times inside the same picture will
  cause the \meta{environment contents} to accumulate.

  \emph{Note:} You can \emph{not} add anything to the |main| layer
  using this environment. The only way to add anything to the main
  layer is to give graphic commands outside all |{pgfonlayer}|
  environments. 

\begin{codeexample}[]
\pgfdeclarelayer{background layer}
\pgfdeclarelayer{foreground layer}
\pgfsetlayers{background layer,main,foreground layer}
\begin{tikzpicture}
  % On main layer:
  \fill[blue] (0,0) circle (1cm);
  
  \begin{pgfonlayer}{background layer}
    \fill[yellow] (-1,-1) rectangle (1,1);
  \end{pgfonlayer}
  
  \begin{pgfonlayer}{foreground layer}
    \node[white] {foreground};
  \end{pgfonlayer}
  
  \begin{pgfonlayer}{background layer}
    \fill[black] (-.8,-.8) rectangle (.8,.8);
  \end{pgfonlayer}

  % On main layer again:
  \fill[blue!50] (-.5,-1) rectangle (.5,1);
\end{tikzpicture}
\end{codeexample}
\end{environment}

\begin{plainenvironment}{{pgfonlayer}\marg{layer name}}
  This is the plain \TeX\ version of the environment.
\end{plainenvironment}





%%% Local Variables: 
%%% mode: latex
%%% TeX-master: "pgfmanual"
%%% End: 

% Copyright 2003 by Till Tantau <tantau@cs.tu-berlin.de>.
%
% This program can be redistributed and/or modified under the terms
% of the LaTeX Project Public License Distributed from CTAN
% archives in directory macros/latex/base/lppl.txt.


\section{Quick Commands}

This section explains the ``quick'' commands of \pgfname. These
commands are executed more quickly than the normal commands of
\pgfname, but offer less functionality. You should use these commands
only if you either have a very large amount of commands that need to
be processed or if you expect your commands to be executed very often.



\subsection{Quick Path Construction Commands}

The difference between the quick and the normal path commands is that
the quick path commands
\begin{itemize}
\item
  do not keep track of the bounding boxes,
\item
  do not allow you to arc corners,
\item
  do not apply coordinate transformations.
\end{itemize}

However, the do use the soft-path subsystem (see
Section~\ref{section-soft-paths} for details), which allows you to mix
quick and normal path commands arbitrarily.

All quick path construction commands start with |\pgfpathq|.

\begin{command}{\pgfpathqmoveto\marg{x dimension}\marg{y dimension}}
  Either starts a path or starts a new part of a path at the coordinate
  $(\meta{x dimension},\meta{y dimension})$. The coordinate is
  \emph{not} transformed by the current coordinate transformation
  matrix. However, any lowlevel transformations apply.

\begin{codeexample}[]
\begin{tikzpicture}
  \draw[help lines] (0,0) grid (3,2);
  \pgftransformxshift{1cm}
  \pgfpathqmoveto{0pt}{0pt} % no effect
  \pgfpathqlineto{1cm}{1cm} % no effect
  \pgfpathlineto{\pgfpoint{2cm}{0cm}}
  \pgfusepath{stroke}
\end{tikzpicture}
\end{codeexample}
\end{command}

\begin{command}{\pgfpathqlineto\marg{x dimension}\marg{y dimension}}
  The quick version of the line-to operation.
\end{command}

\begin{command}{\pgfpathqcurveto\marg{$s^1_x$}\marg{$s^1_y$}\marg{$s^2_x$}\marg{$s^2_y$}\marg{$t_x$}\marg{$t_y$}}
  The quick version of the curve-to operation. The first support point
  is $(s^1_x,s^1_y)$, the second support point is  $(s^2_x,s^2_y)$,
  and the target is $(t_x,t_y)$.
 
\begin{codeexample}[]
\begin{tikzpicture}
  \draw[help lines] (0,0) grid (3,2);
  \pgfpathqmoveto{0pt}{0pt}
  \pgfpathqcurveto{1cm}{1cm}{2cm}{1cm}{3cm}{0cm}
  \pgfusepath{stroke}
\end{tikzpicture}
\end{codeexample}
\end{command}

\begin{command}{\pgfpathqcircle\marg{radius}}
  Adds a raduis around the origin of the given \meta{raduis}. This
  command is orders of magnitude faster than
  |\pgfcircle{\pgfpointorigin}{|meta{radius}|}|. 
 
\begin{codeexample}[]
\begin{tikzpicture}
  \draw[help lines] (0,0) grid (1,1);
  \pgfpathqcircle{10pt}
  \pgfsetfillcolor{yellow}
  \pgfusepath{stroke,fill}
\end{tikzpicture}
\end{codeexample}
\end{command}



\subsection{Quick Path Usage Commands}

The quick path usage commands perform similar tasks as |\pgfusepath|,
but the they
\begin{itemize}
\item
  do not add arrows,
\item
  do not modify the path in any way, in particular,
\item
  ends are not shortened,
\item
  corners are nto replaced by arcs.
\end{itemize}

Note that you \emph{have to} use the quick versions in the code of
arrow definitions since, inside these definition, you obviously do not
want arrows to be drawn.

\begin{command}{\pgfusepathqstroke}
  Strokes the path without further ado. No arrows are drawn, no
  corners are arced.

\begin{codeexample}[]
\begin{pgfpicture}
  \pgfpathqcircle{5pt}
  \pgfusepathqstroke
\end{pgfpicture}
\end{codeexample}
\end{command}

\begin{command}{\pgfusepathqfill}
  Fills the path without further ado.
\end{command}

\begin{command}{\pgfusepathqfillstroke}
  Fills and then strokes the path without further ado.
\end{command}

\begin{command}{\pgfusepathqclip}
  Clips all subsequent drawings against the current path. The path is
  not processed.
\end{command}


\subsection{Quick Text Box Commands}

\begin{command}{\pgfqbox\marg{box}}
  This command inserts a \TeX\ box into a |{pgfpicture}| by
  ``escaping'' to \TeX, inserting the \meta{box} at the origin, and
  then returning to the typesetting the picture.

  The \meta{box} \emph{must} have a height, width, and depth of zero
  points. Otherwise, the output may become corrupted.
\end{command}





\part{The System Layer}
\label{part-system}

This part describes the low-level interface of \pgfname, called the
\emph{system layer}. This interface provides a complete abstraction of
the internals of the underlying drivers. 

Unless you intend to port \pgfname\ to another driver or unless you intend
to write your own optimized frontend, you need not read this part.

In the following it is assumed that you are familiar with the basic
workings of the |graphics| package and that you know what
\TeX-drivers are and how they work.

\vskip1cm
\begin{codeexample}[graphic=white]
\begin{tikzpicture}[shorten >=1pt,->]
  \tikzstyle{vertex}=[circle,fill=black!25,minimum size=17pt,inner sep=0pt]
  
  \foreach \name/\x in {s/1, 2/2, 3/3, 4/4, 15/11, 16/12, 17/13, 18/14, 19/15, t/16}
    \node[vertex] (G-\name) at (\x,0) {$\name$};

  \foreach \name/\angle/\text in {P-1/234/5, P-2/162/6, P-3/90/7, P-4/18/8, P-5/-54/9}
    \node[vertex,xshift=6cm,yshift=.5cm] (\name) at (\angle:1cm) {$\text$};
  
  \foreach \name/\angle/\text in {Q-1/234/10, Q-2/162/11, Q-3/90/12, Q-4/18/13, Q-5/-54/14}
    \node[vertex,xshift=9cm,yshift=.5cm] (\name) at (\angle:1cm) {$\text$};

  \foreach \from/\to in {s/2,2/3,3/4,3/4,15/16,16/17,17/18,18/19,19/t}
    \draw (G-\from) -- (G-\to);  

  \foreach \from/\to in {1/2,2/3,3/4,4/5,5/1,1/3,2/4,3/5,4/1,5/2}
    { \draw (P-\from) -- (P-\to); \draw (Q-\from) -- (Q-\to); }

  \draw (G-3) .. controls +(-30:2cm) and +(-150:1cm) .. (Q-1);
  \draw (Q-5) -- (G-15);
\end{tikzpicture}
\end{codeexample}

% Copyright 2003 by Till Tantau <tantau@cs.tu-berlin.de>.
%
% This program can be redistributed and/or modified under the terms
% of the LaTeX Project Public License Distributed from CTAN
% archives in directory macros/latex/base/lppl.txt.

\section{Design of the System Layer}

\makeatletter


\subsection{Driver Files}
\label{section-pgfsys}

The \pgfname\ system layer mainly consists of a large number of
commands starting with |\pgfsys@|. These commands will be called
\emph{system commands} in the following. The higher layers
``interface'' with the system layer by calling these commands. The
higher layers should never use |\special| commands directly or even
check whether |\pdfoutput| is defined. Instead, all drawing requests
should be ``channeled'' through the system commands. 

The system layer is loaded and setup by the following package:

\begin{package}{pgfsys}
  This file provides ``default implementations'' of all system
  commands, but most simply produce a warning that they are not
  implemented. The actual implementations of the system commands for a
  particular driver like, say, |pdftex| reside in files called
  |pgfsys-xxxx.sty|, where |xxxx| is the driver name. These will be
  called \emph{driver files} in the following.

  When |pgfsys.sty| is loaded, it will try to determine which driver
  is used by loading |pgf.cfg|. This file should setup the macro
  |\pgfsysdriver| appropriately. The, |pgfsys.sty| will input the
  appropriate |pgfsys-|\meta{drivername}|.sty|. 
\end{package}

\begin{command}{\pgfsysdriver}
  This macro should expand to the name of the driver to be used by
  |pgfsys|. The default from |pgf.cfg| is |pgfsys-\Gin@driver|. This
  is very likely to be correct if you are using \LaTeX. For plain
  \TeX, the macro will be set to |pgfsys-pdftex.def| if |pdftex| is
  used and to |pgfsys-dvips.def| otherwise.
\end{command}

\begin{filedescription}{pgf.cfg}
  This file should setup the command |\pgfsysdriver| correctly. If
  |\pgfsysdriver| is already set to some value, the driver normally
  should not change it. Otherwise, it should make a ``good guess'' at
  which driver will be appropriate.
\end{filedescription}



\subsection{System Commands Shared Between Different Drivers}

Some definitions of system layer commands can be ``shared'' between
different drivers. For example, the literal text needed to stroke a
path in pdf is |S|, independently of the driver. For this reason,
the drivers for |pdftex| and for |dvipdfm|, both of which produce
|.pdf| in the end, both include the file |pgfsys-common-pdf.def|,
which defines all common commands. Similarly, all PostScript based
drivers can used |pgfsys-common-postscript.def| for the ``standard''
postscript commands.


\subsection{Supported Drivers}

With the current version of \pgfname, the following drivers are
implemented:

\begin{filedescription}{pgfsys-pdftex.def}
  This is a driver file for use with pdf\TeX, that is, with the
  |pdftex| or |pdflatex| command. It includes
  |pgfsys-common-pdf.def|. This driver has the most functionality. 
\end{filedescription}

\begin{filedescription}{pgfsys-dvipdfm.def}
  This is a driver file for use with (|la|)|tex| followed by |dvipdfm|. It
  includes |pgfsys-common-pdf.def|. This driver uses |graphicx| for the
  graphics inclusion and does not support masking. It does not
  support image inclusion in plain \TeX\ mode.
\end{filedescription}

\begin{filedescription}{pgfsys-dvips.def}
  This is a driver file for use with (|la|)|tex| followed by
  |dvips|. It includes |pgfsys-common-postscript.def|. This driver
  uses |graphicx| for the graphics inclusion and does not support
  masking. Shading is implemented, but the results will not be
  as good as with a driver producing |.pdf| as output. It does not
  support image inclusion in plain \TeX\ mode.
\end{filedescription}


\subsection{Common Definition Files}

Some drivers share many |\pgfsys@| commands. For the reason, files
defining these ``common'' commands are available. These files are
\emph{not} usable alone.

\begin{filedescription}{pgfsys-common-postscript}
  This file defines some |\pgfsys@| commands so that they produce
  appropriate PostScript code.
\end{filedescription}

\begin{filedescription}{pgfsys-common-pdf}
  This file defines some |\pgfsys@| commands so that they produce
  appropriate \textsc{pdf} code.
\end{filedescription}


%%% Local Variables: 
%%% mode: latex
%%% TeX-master: "pgfmanual"
%%% End: 

% Copyright 2003 by Till Tantau <tantau@cs.tu-berlin.de>.
%
% This program can be redistributed and/or modified under the terms
% of the LaTeX Project Public License Distributed from CTAN
% archives in directory macros/latex/base/lppl.txt.

\section{Commands of the System Layer}

\makeatletter

\subsection{Beginning and Ending a Stream of System Commands}

A ``user'' of the \pgfname\ system layer (like the basic layer or a
frontend) will interface with the system layer by calling a stream of
commands starting with |\pgfsys@|. From the system layer's point of
view, these commands form a long stream. Between calls to the system
layer, control goes back to the user.

The driver files implement system layer commands by inserting
|\special| commands the implement the desired operation. For example,
|\pgfsys@stroke| will be mapped to |\special{pdf: S}| by the driver
file for |pdftex|.

For many drivers, when such a stream of specials starts, it is
necessary to install an appropriate transformation and perhaps perform
some more burocratic tasks. For this reason, every stream will start
with a |\pgfsys@beginpicture| and will end with a corresponding ending
command.

\begin{command}{\pgfsys@beginpicture}
  Called at the beginning of a |{pgfpicture}|. By default,  this just
  opens a scope.
  
  This command has a default implementation and need not be
  implemented by a driver file.
\end{command}

\begin{command}{\pgfsys@endpicture}
  Called at the end of a pgfpicture.  By
  default, this discards the path and closes the scope.
  
  This command has a default implementation and need not be
  implemented by a driver file.
\end{command}

\begin{command}{\pgfsys@beginpurepicture}
  This version of the |\pgfsys@beginpicture| picture command can be
  used for pictures that are guaranteed not to contain any escaped
  hboxes (see below). In this case, a driver might provide a more
  compact version of the command. 
  
  This command has a default implementation and need not be
  implemented by a driver file.
\end{command}

\begin{command}{\pgfsys@endpurepicture}
  Called at the end of a ``pure'' |{pgfpicture}|.
  
  This command has a default implementation and need not be
  implemented by a driver file.
\end{command}

Inside a stream it is sometimes necessary to ``escape'' back into
normal typesetting mode; for example to insert some normal text, but
with all of the current transformations and clippings being in
force. For this escaping, the following commands are used:

\begin{command}{\pgfsys@beginhbox}
  Called before a TeX hbox is typeset inside a pgfpicture. By default,
  this just opens a scope.
  
  This command has a default implementation and need not be
  implemented by a driver file.
\end{command}


\begin{command}{\pgfsys@endhbox}
  Called after a TeX hbox has been typeset inside a |{pgfpicture}|. By 
  default, this discards the path and closes the scope.
  
  This command has a default implementation and need not be
  implemented by a driver file.
\end{command}



\subsection{Path Construction System Commands}

\begin{command}{\pgfsys@moveto\marg{x}\marg{y}}
  This command is used to start a path at a specific point
  $(x,y)$ or to move the current point of the current path to  $(x,y)$
  without drawing anything upon stroking (the current path is
  ``interrupted'').

  Both \meta{x} and \meta{y} are given as \TeX\ dimensions. It is the
  driver's job to transform these to the coordinate system of the
  backend. Typically, this means converting the \TeX\ dimension into a
  dimensionless multiple of $\frac{1}{72}\mathrm{in}$. The function
  |\pgf@sys@bp| helps with this conversion.

  \example Draw a line from $(10\mathrm{pt},10\mathrm{pt})$ to the
  origin of the picture. 
\begin{codeexample}[code only]
\pgfsys@moveto{10pt}{10pt}
\pgfsys@lineto{0pt}{0pt}
\pgfsys@stroke
\end{codeexample}

  This command is protocolled, see Section~\ref{section-protocols}.
\end{command}


\begin{command}{\pgfsys@lineto\marg{x}\marg{y}}
  Continue the current path to $(x,y)$ with
  a straight line.

  This command is protocolled, see Section~\ref{section-protocols}.
\end{command}


\begin{command}{\pgfsys@curveto\marg{$x_1$}\marg{$y_1$}\marg{$x_2$}\marg{$y_2$}\marg{$x_3$}\marg{$y_3$}}
  Continue the current path to $(x_3,y_3)$
  with a Bezi�r curve that has the two control points  $(x_1,y_1)$ and  $(x_2,y_2)$.

  \example Draw a good approximation of a quarter circle:
\begin{codeexample}[code only]
\pgfsys@moveto{10pt}{0pt}
\pgfsys@curveto{10pt}{5.55pt}{5.55pt}{10pt}{0pt}{10pt}
\pgfsys@stroke
\end{codeexample}

  This command is protocolled, see Section~\ref{section-protocols}.
\end{command}


\begin{command}{\pgfsys@rect\marg{x}\marg{y}\marg{width}\marg{height}}
  Append a rectangle to the current path whose lower left corner is
  at $(x,y)$ and whose width and height in
  big points are  given by \meta{width} and \meta{height}.

  This command can be ``mapped back'' to |\pgfsys@moveto| and
  |\pgfsys@lineto| commands, but it is included since \pdf\ has a
  special, quick version of this command. 

  This command is protocolled, see Section~\ref{section-protocols}.
\end{command}


\begin{command}{\pgfsys@closepath}
  Close the current path. This results in joining the current point of
  the path with the point specified by the last |\pgfsys@moveto|
  operation. Typically, this is preferable over using |\pgfsys@lineto|
  to the last point specified by a |\pgfsys@moveto|, since the line
  starting at this point and the line ending at this point will be
  smoothly joined by |\pgfsys@closepath|.

  \example Consider
\begin{codeexample}[code only]
\pgfsys@moveto{0pt}{0pt}
\pgfsys@lineto{10bp}{10bp}
\pgfsys@lineto{0bp}{10bp}
\pgfsys@closepath
\pgfsys@stroke
\end{codeexample}
  and
\begin{codeexample}[code only]
\pgfsys@moveto{0bp}{0bp}
\pgfsys@lineto{10bp}{10bp}
\pgfsys@lineto{0bp}{10bp}
\pgfsys@lineto{0bp}{0bp}
\pgfsys@stroke
\end{codeexample}
  
  The difference between the above will be that in the second triangle
  the corner at the origin will be wrong; it will just be the overlay
  of two lines going in different directions, not a sharp pointed
  corner.

  This command is protocolled, see Section~\ref{section-protocols}.
\end{command}




\subsection{Coordinate System Transformation System Commands}

\begin{command}{\pgfsys@transformcm\marg{a}\marg{b}\marg{c}\marg{d}\marg{e}\marg{f}}
  Perform a concatenation of the canvas transformation matrix with the
  matrix given by the values \meta{a} to \meta{f}, see the \pdf\ or
  PostScript manual for details. The values \meta{a} to \meta{d} are
  dimensionless factors, \meta{e} and \meta{f} are \TeX\ dimensions 

  \example |\pgfsys@transformcm{1}{0}{0}{1}{1cm}{1cm}|.

  This command is protocolled, see Section~\ref{section-protocols}.  
\end{command}


\begin{command}{\pgfsys@transformshift\marg{x displacement}\marg{y displacement}}
  This command will change the origin of the canvas to $(x,y)$.

  This command has a default implementation and need not be
  implemented by a driver file.

  This command is protocolled, see Section~\ref{section-protocols}.
\end{command}

\begin{command}{\pgfsys@transformxyscale\marg{x scale}\marg{y scale}}
  This command will scale the canvas (and  everything that is drawn)
  by a factor of \meta{x scale} in the $x$-direction and \meta{y
    scale} in the  $y$-direction. Note that this applies to
  everything, including  lines. So a scaled line will have a different
  width and may even have a different width when going along the
  $x$-axis and when going along the $y$-axis, if the scaling is
  different in these directions. Usually, you do not want this.

  This command has a default implementation and need not be
  implemented by a driver file.

  This command is protocolled, see Section~\ref{section-protocols}.
\end{command}


\subsection{Stroking, Filling, and Clipping System Commands}

\begin{command}{\pgfsys@stroke}
  Stroke the current path (as if it were drawn with a pen). A number
  of graphic state parameters influence this, which can be
  set using appropriate system commands described later.

  \begin{description}
  \item[linewidth]
    The ``thickness'' of the line. A width of 0 is the thinnest width
    renderable on the device. On a high-resolution printer this may
    become invisible and should be avoided. A good choice is 0.4pt,
    which is the default.

  \item[stroke color]
    This special color is used for stroking. If it is not set, the
    current color is used.
 
  \item[cap]
    The cap describes how the endings of lines a drawn. A round cap
    adds a little half circle to these endings. A butt cap ends the
    lines exactly at the end (or start) point without anything
    added. A rectangular cap ends the lines like the butt cap, but the
    lines protrude over the endpoint by the line thickness. (See also
    the \pdf\ manual). If the path has been closed, no cap
    is drawn.
 
  \item[join]
    This describes how a bend (a join) in a path is rendered. A round
    join draws bends using small arcs. A bevel join just draws the two
    lines and then fills the join minimally so that it becomes
    convex. A miter join extends the lines so that they form a single
    sharp corner, but only up to a certain miter limit. (See the \pdf\
    manual once more).
 
  \item[dash]
    The line may be dashed according to a dashing pattern.
 
  \item[clipping area]
    If a clipping area is established, only those parts of the path
    that are inside the clipping area will be drawn.
  \end{description}
  
  In addition to stroking a path, the path may also be used for
  clipping after it has been stroked. This will happen if the
  |\pgfsys@clipnext| is used prior to this command, see there for
  details.

  This command is protocolled, see Section~\ref{section-protocols}.
\end{command}


\begin{command}{\pgfsys@closestroke}
  This command should have the same effect as first closing the path
  and then stroking it.

  This command has a default implementation and need not be
  implemented by a driver file.

  This command is protocolled, see Section~\ref{section-protocols}.
\end{command}


\begin{command}{\pgfsys@fill}
  This command fills the area surrounded by the current path. If the
  path has not yet been closed, it is closed prior to filling. The
  path itself is not stroked. For self-intersecting paths or paths
  consisting of multiple parts, the nonzero winding number rule is
  used to determine whether a point is inside or outside the
  path, except if |\ifpgfsys@eorule| holds -- in which case the
  even-odd rule should be used. (See the \pdf\ or PostScript manual
  for details.)  
 
  The following graphic state parameters influence the filling:
 
  \begin{description}
  \item[eo rule]
    If |\ifpgfsys@eorule| is set, the even-odd rule is used, otherwise
    the non-zero winding number rule.
 
  \item[fill]
    If the fill color is not especially set, the current color is
    used. 
 
  \item[clipping area]
    If a clipping area is established, only those parts of the filling
    area that are inside the clipping area will be drawn.
  \end{description}

  In addition to filling the path, the path will also be used for
  clipping if |\pgfsys@clipnext| is used prior to this command.

  This command is protocolled, see Section~\ref{section-protocols}.
\end{command}

\begin{command}{\pgfsys@fillstroke}
  First, the path is filled, then the path is stroked. If the fill and
  stroke colors are the same (or if they are not specified and the
  current color is used), this yields almost the same as a
  |\pgfsys@fill|. However, due to the line thickness of the stroked
  path, the fillstroked area will be slightly larger.

  In addition to stroking and filling the path, the path will also be
  used for clipping if |\pgfsys@clipnext| is used prior to this command.

  This command is protocolled, see Section~\ref{section-protocols}.
\end{command}


\begin{command}{\pgfsys@discardpath}
 Normally, this command should `throw away' the current path.
 However, after |\pgfsys@clipnext| has been called, the current path
 should subsequently be used for clipping. See |\pgfsys@clipnext| for 
 details. 

  This command is protocolled, see Section~\ref{section-protocols}.
\end{command}


\begin{command}{\pgfsys@clipnext}
  This command should be issued after a path has been constructed, but
  before it has been stroked and/or filled or discarded. When the
  command is used, the next stroking/filling/discarding command will
  first be executed normally. Then, afterwards, the just-used path
  will be used for subsequent clipping. If there has already been a
  clipping region, this region is intersected with the new clipping
  path (the clipping cannot get bigger). The nonzero winding number
  rule is used to determine whether a point is inside or outside the
  clipping area or the even-odd rule, depending on whether
  |\ifpgfsys@eorule| holds.
\end{command}




\subsection{Graphic State Option System Commands}

\begin{command}{\pgfsys@setlinewidth\marg{width}}
  Sets the width of lines, when stroked, to \meta{width}, which must
  be a \TeX\ dimension.

  This command is protocolled, see Section~\ref{section-protocols}.
\end{command}

\begin{command}{\pgfsys@buttcap}
  Sets the cap to a butt cap. See |\pgfsys@stroke|.

  This command is protocolled, see Section~\ref{section-protocols}.
\end{command}

\begin{command}{\pgfsys@roundcap}
  Sets the cap to a round cap. See |\pgfsys@stroke|.

  This command is protocolled, see Section~\ref{section-protocols}.
\end{command}

\begin{command}{\pgfsys@rectcap}
  Sets the cap to a rectangular cap. See |\pgfsys@stroke|.

  This command is protocolled, see Section~\ref{section-protocols}.
\end{command}

\begin{command}{\pgfsys@miterjoin}
  Sets the join to a miter join. See |\pgfsys@stroke|.

  This command is protocolled, see Section~\ref{section-protocols}.
\end{command}

\begin{command}{\pgfsys@setmiterlimit\marg{dimension}}
  Sets the miter limit of lines to \meta{dimension} big points. See
  the \pdf or PostScript for details on what the miter limit is.

  This command is protocolled, see Section~\ref{section-protocols}.
\end{command}

\begin{command}{\pgfsys@roundjoin}
  Sets the join to a round join. See |\pgfsys@stroke|.

  This command is protocolled, see Section~\ref{section-protocols}.
\end{command}

\begin{command}{\pgfsys@beveljoin}
  Sets the join to a bevel join. See |\pgfsys@stroke|.

  This command is protocolled, see Section~\ref{section-protocols}.
\end{command}

\begin{command}{\pgfsys@setdash\marg{pattern}\marg{phase}}
  Sets the dashing patter. \meta{pattern} should be a list of \TeX\
  dimensions lengths separated by commas. \meta{phase} should be a
  single dimension.

  \example |\pgfsys@setdash{3pt,3pt}{0pt}|
 
  The list of values in \meta{pattern} is used to determine the
  lengths of the `on' phases of the dashing and of the `off'
  phases. For example, if \meta{pattern} is `3bp,4bp', then the dashing
  pattern is ``3bp on followed by 4bp  off, followed by 3bp on,
  followed by 4bp off, and so on.'' A pattern of ``.5 4 3 1.5'' means
  ``.5bp on, 4bp off, 3bp on, 1.5bp off, .5bp on, \dots'' If the
  number of entries is odd, the last one is used twice, so ``3'' means
  ``3bp on, 3bp off, 3bp on, 3bp off, \dots'' An empty list 
  means  ``always on.''
 
  The second argument determines the ``phase'' of the pattern. For
  example, for a pattern of ``3bp,4bp'' and a phase of ``1bp'', the pattern
  would start: ``2bp on, 4bp off, 3bp on, 4bp off, 3bp on, 4bp off,
  \dots''

  This command is protocolled, see Section~\ref{section-protocols}.
\end{command}

{\let\ifpgfsys@eorule=\relax
\begin{command}{\ifpgfsys@eorule}
  Determines whether the even odd rule is used for filling ang
  clipping or not.
\end{command}
}




\subsection{Color System Commands}

The \pgfname\ system layer provides a number of system commands for
setting colors. These command coexist with commands from the |color|
and |xcolor| package, which perform similar functions. However, the
|color| package does not support having two different colors for
stroking and filling, which is a useful feature that is supported by
\pgfname. For this reason, the \pgfname\ system layer offers commands for
setting these colors separatedly. Also, plain \TeX\ profits from the
fact that \pgfname\ can set colors.

For \pdf, implementing these color commands is easy since \pdf\ 
supports different stroking and filling colors directly. For
PostScript, a more complicated approach is needed in which the colors
need to be stored in special PostScript variables that are set
wheneven a stroking or a filling operation is done.

\begin{command}{\pgfsys@color@rgb@\marg{red}\marg{green}\marg{blue}}
  Sets the color used for stroking and filling operations to the given 
  red/green/blue tuple (numbers between 0 and 1).

  This command is protocolled, see Section~\ref{section-protocols}.
\end{command}

\begin{command}{\pgfsys@color@rgb@stroke\marg{red}\marg{green}\marg{blue}}
  Sets the color used for stroking operations to the given
  red/green/blue tuple (numbers between 0 and 1). 

  \example Make stroked text dark red: |\pgfsys@color@rgb@stroke{0.5}{0}{0}|

  The special stroking color is only used if the stroking color has
  been set since the last |\color| or |\pgfsys@color@xxx|
  command. Thus, each |\color| command will reset both the stroking
  and filling colors by calling |\pgfsys@color@reset|. 

  This command is protocolled, see Section~\ref{section-protocols}.
\end{command}

\begin{command}{\pgfsys@color@rgb@fill\marg{red}\marg{green}\marg{blue}}
  Sets the color used for filling operations to the given
  red/green/blue tuple (numbers between 0 and 1). This color may be
  different from the stroking color.

  This command is protocolled, see Section~\ref{section-protocols}.
\end{command}

\begin{command}{\pgfsys@color@cmyk\marg{cyan}\marg{magenta}\marg{yellow}\marg{black}}
  Sets the color used for stroking and filling operations to the given
  cymk tuple (numbers between 0 and 1). 

  This command is protocolled, see Section~\ref{section-protocols}.
\end{command}

\begin{command}{\pgfsys@color@cmyk@stroke\marg{cyan}\marg{magenta}\marg{yellow}\marg{black}}
  Sets the color used for stroking operations to the given cymk tuple
  (numbers between 0 and 1). 

  This command is protocolled, see Section~\ref{section-protocols}.
\end{command}

\begin{command}{\pgfsys@color@cmyk@fill\marg{cyan}\marg{magenta}\marg{yellow}\marg{black}}
  Sets the color used for filling operations to the given cymk tuple
  (numbers between 0 and 1). 

  This command is protocolled, see Section~\ref{section-protocols}.
\end{command}

\begin{command}{\pgfsys@color@cmy\marg{cyan}\marg{magenta}\marg{yellow}}
  Sets the color used for stroking and filling operations to the given
  cym tuple (numbers between 0 and 1). 

  This command is protocolled, see Section~\ref{section-protocols}.
\end{command}

\begin{command}{\pgfsys@color@cmy@stroke\marg{cyan}\marg{magenta}\marg{yellow}}
  Sets the color used for stroking operations to the given cym tuple
  (numbers between 0 and 1). 

  This command is protocolled, see Section~\ref{section-protocols}.
\end{command}

\begin{command}{\pgfsys@color@cmy@fill\marg{cyan}\marg{magenta}\marg{yellow}}
  Sets the color used for filling operations to the given cym tuple
  (numbers between 0 and 1). 

  This command is protocolled, see Section~\ref{section-protocols}.
\end{command}

\begin{command}{\pgfsys@color@gray\marg{black}}
  Sets the color used for stroking and filling operations to the given
  black value, where 0 means black and 1 means white.

  This command is protocolled, see Section~\ref{section-protocols}.
\end{command}

\begin{command}{\pgfsys@color@gray@stroke\marg{black}}
  Sets the color used for stroking operations to the given black value,
  where 0 means black and 1 means white.

  This command is protocolled, see Section~\ref{section-protocols}.
\end{command}

\begin{command}{\pgfsys@color@gray@fill\marg{black}}
  Sets the color used for filling operations to the given black value,
  where 0 means black and 1 means white.

  This command is protocolled, see Section~\ref{section-protocols}.
\end{command}

\begin{command}{\pgfsys@color@reset}
  This command will be called when the |\color| command is used. It
  should purge any internal settings of stroking and filling
  color. After this call, till the next use of a command like
  |\pgfsys@color@rgb@fill|, the current color installed by the
  |\color| command should be used.

  If the \TeX-if |\pgfsys@color@reset@inorder| is set to true, this
  command may ``assume'' that any call to a color command that sets
  the fill or stroke color came ``before'' the call to this command
  and may try to optimize the output accordingly.

  An example of an incorrect ``out of order'' call would be using
  |\pgfsys@color@reset| at the beginning of a bos that is constructed
  using |\setbox|. Then, when the box is constructed, no special fill
  or stroke color might be in force. However, when the box is later on
  inserted at some point, a special fill color might already have been
  set. In this case, this command is not guaranteed to reset the color
  correctly. Use |\pgfsys@color@reset@outoforder| in such cases.
\end{command}

\begin{command}{\pgfsys@color@reset@inordertrue}
  Sets the optimized ``in order'' version of the color resetting. This
  is the default.
\end{command}

\begin{command}{\pgfsys@color@reset@inorderfalse}
  Switches off the optimized color resetting. 
\end{command}

\begin{command}{\pgfsys@color@unstacked\marg{\LaTeX\ color}}
  This slightly obscure command causes the color stack to be
  tricked. When called, this command should set the current color to
  \meta{\LaTeX\ color} without causing any change in the color stack.

  \example |\pgfsys@color@unstacked{red}|
\end{command}


\subsection{Scoping System Commands}

The scoping commands are used to keep changes of the graphics state
local.

\begin{command}{\pgfsys@beginscope}
  Saves the current graphic state on a graphic state stack. All
  changes to the graphic state parameters mentioned for |\pgfsys@stroke|
  and |\pgfsys@fill| will be local to the current graphic state and will
  the old values will be restored after endscope is used.
 
  \emph{Warning:} \pdf and PostScript differ with respect to the
  question of whether the current path is part of the graphic state or
  not. For this reason, you should never use this command unless the
  path is currently empty. For example, it might be a good idea to use 
  discardpath prior to calling this command. 

  This command is protocolled, see Section~\ref{section-protocols}.
\end{command}

\begin{command}{\pgfsys@endscope}
  Restores the last saved graphic state.

  This command is protocolled, see Section~\ref{section-protocols}.
\end{command}







\subsection{Image System Commands}

The sytem layer provides some commands for image inclusion. However,
this whole subsystem is not really well-designed and the |graphics|
package does a better job at image inclusion than \pgfname\
does. However, this subsystem is currently still needed since only
\pgfname\ supports image masking. Once this feature is added to the
|graphics| package, the whole subsystem will be mapped back to
|graphics| and become obsolet. 

\begin{command}{\pgfsys@imagesuffixlist}
  This macro should expand to a list of suffixes, separated by `:',
  that will be tried when searching for an image.

  \example |\def\pgfsys@imagesuffixlist{eps:epsi:ps}|
\end{command}


\begin{command}{\pgfsys@defineimage}
  Called, when an image should be defined. 
 
  This command does not take any parameters. Instead, certain macros
  will be preinstalled with appropriate values when this command is
  invoked. These are:
 
  \begin{itemize}
  \item\declare{|\pgf@filename|}
    File name of the image to be defined.

  \item\declare{|\pgf@imagewidth|}
    Will be set to the desired (scaled) width of the image.

  \item\declare{|\pgf@imageheight|}
    Will be set to the desired (scaled) height of the image.
 
    If this macro and also the height macro are empty, the image
    should have its `natural' size.
 
    If exactly only of them is specified, the undefined value the
    image is scaled so that the aspect ratio is kept.
 
    If both are set, the image is scaled in both directions
    independently, possibly changing the aspect ratio.
  \end{itemize}
 
  The following macros presumable mostly make sense for drivers that
  can handle \pdf: 

  \begin{itemize}
  \item \declare{|\pgf@imagepage|}
    The desired page number to be extracted from a multi-page
    ``image.''

  \item\declare{|\pgf@imagemask|}
    If set, it will be set to |/SMask x 0 R| where |x| is the pdf 
    object number of a soft mask to be applied to the image.

  \item\declare{|\pgf@imageinterpolate|}
    If set, it will be set to |/Interpolate true| or
    |/Interpolate false|, indicating whether the image should be
    interpolated in \pdf. 
  \end{itemize}
 
  The command should now setup the macro |\pgf@image| such that calling
  this macro will result in typesetting the image. Thus, |\pgf@image| is
  the ``return value'' of the command.

  This command has a default implementation and need not be
  implemented by a driver file.
\end{command}


\begin{command}{\pgfsys@definemask}
  This command declares a mask for usage with images. It works similar
  to |\pgfsys@defineimage|: Certain macros are set when the command is
  called. The result should be to set the macro |\pgf@mask| to a pdf
  object count that can subsequently be used as a soft mask. The
  following macros will be set when this command is invoked:
 
  \begin{itemize}
  \item \declare{|\pgf@filename|}
    File name of the mask to be defined.

  \item \declare{|\pgf@maskmatte|}
    The so-called matte of the mask (see the \pdf\ documentation for
    details). The matte is a color specification consisting of 1, 3 or
    4 numbers between 0 and 1. The number of numbers depends on the
    number of color channels in the image (not in the mask!). It will
    be assumed that the image has been preblended with this color.
  \end{itemize}
\end{command}


\subsection{Shading System Commands}


\begin{command}{\pgfsys@horishading\marg{name}\marg{height}\marg{specification}}
  Declares a horizontal shading for later use. The effect of this
  command should be the definition of a macro called |@pgfshading|meta{name}|!|
  (or |\csname @pdfshading|\meta{name}|!\endcsname|, to be
  precise). When invoked, this new macro should insert a shading at
  the current position. 
 
  \meta{name} is the name of the shading, which is also used in the
  output macro name. \meta{height} is the height of the shading and
  must be given as a TeX dimension like |2cm| or
  |10pt|. \meta{specification} is a shading color 
  specification as specified in Section~\ref{section-shadings}. The
  shading specification implicitly fixes the width of the shading. 
 
  When |@pgfshading|meta{name}|!| is invoked, it should insert a box
  of height \meta{height} and the width implicit in the shading
  declaration. 
\end{command}


\begin{command}{\pgfsys@vertshading\marg{name}\marg{width}\marg{specification}}
  Like the horizontal version, only for vertical shadings. This time,
  the height of the shading is implicit in \meta{specification} and
  the width is given as \meta{width}.
\end{command}

\begin{command}{\pgfsys@radialshading\marg{name}\marg{starting point}\marg{specification}}
  Declares a radial shading. Like the previous macros, this command
  should setup the macro |@pgfshading|meta{name}|!|, which upon invocation
  should insert a radial shading whose size is implicit in \meta{specification}. 

  The parameter \meta{starting point} is a pgf point specification if
  the starting point of the shading.
\end{command}


\subsection{Reusable Objects System Commands}

\begin{command}{\pgfsys@invoke\marg{literals}}
  This command gets protocolled literals and should insert them into
  the |.pdf| or |.dvi| file using an appropriate |\special|.
\end{command}

\begin{command}{\pgfsys@defobject\marg{name}\marg{lower
      left}\marg{upper right}\marg{code}}
  Declares an object for later use. The idea is that the object can be
  precached in some way and then be rendered more quickly when used
  several times. For example, an arrow head might be defined and
  prerendered in this way.
 
  The parameter \meta{name} is the name for later use. \meta{lower
  left} and \meta{upper right} are \pgfname\ points specifying a bounding
  box for the object. \meta{code} is the code for the object. The code
  should not be too fancy.

  This command has a default implementation and need not be
  implemented by a driver file.
\end{command}

\begin{command}{\pgfsys@useobject\marg{name}\marg{extra code}}
  Renders a previously declared object. The first parameter is the
  name of the the object. The second parameter is extra code that
  should be executed right \emph{before} the object is
  rendered. Typically, this will be some transformation code.

  This command has a default implementation and need not be
  implemented by a driver file.
\end{command}


\subsection{Invisibility System Commands}

All drawing or stroking or text rendering between calls of the
following commands should be suppressed. A similar effect can be
achieved by clipping against an empty region, but the following
commands do not open a graphics scope and can be opened and closed
``orthogonally'' to other scopings.

\begin{command}{\pgfsys@begininvisible}
  Between this command and the closing endinvisible, all output should
  be suppressed. Nothing should be drawn at all, which includes all
  paths, images and shadings. However, no groups (neither \TeX\ groups
  nor graphic state groups) should be opened by this command.

  This command has a default implementation and need not be
  implemented by a driver file.

  This command is protocolled, see Section~\ref{section-protocols}.
\end{command}
  
\begin{command}{\pgfsys@endinvisible}
  Ends the invisibilty section, unless invisibility blocks have been
  nested. In this case, only the ``last'' one restores visibility.

  This command has a default implementation and need not be
  implemented by a driver file.

  This command is protocolled, see Section~\ref{section-protocols}.
\end{command}




\subsection{Internal Conversion Commands}

The system commands take \TeX\ dimensions as input, but the dimensions
that have to be inserted into \pdf\ and PostScript files need to be
dimensionless values that are interpreted as multiples of
$\frac{1}{72}\mathrm{in}$. For example, the \TeX\ dimension $2bp$
should be inserted as |2| into a \pdf\ file and the \TeX\ dimension
$10pt$ as |9.9626401|. To make this conversion easier, the following
command may be useful:

\begin{command}{\pgf@sys@bp\marg{dimensions}}
  Inserts how many multiples of $\frac{1}{72}\mathrm{in}$ the
  \meta{dimension} is into the current protocol stream (buffered).

  \example |\pgf@sys@bp{\pgf@x}| or |\pgf@sys@bp{1cm}|.
\end{command}

Note that this command is \emph{not} a system command that can/needs
to be overwritten by a driver. 

% Copyright 2003 by Till Tantau <tantau@cs.tu-berlin.de>.
%
% This program can be redistributed and/or modified under the terms
% of the LaTeX Project Public License Distributed from CTAN
% archives in directory macros/latex/base/lppl.txt.

\section{The Soft Path Subsystem}

\label{section-soft-paths}

\makeatletter


This section describes a set of commands for creating \emph{soft
  paths} as opposed to the commands of the previous section, which
created \emph{hard paths}. A soft path is a path that can still be
``changed'' or ``molded.'' Once you (or the \pgfname\ system) is
satisfied with a soft path, it is turned into a hard path, which can
be inserted into the resulting |.pdf| or |.ps| file.

Note that the commands described in this section are ``high-level'' in
the sense that they are not implemented in driver files, but rather
directly by the \pgfname-system layer. For this reason, the commands for
creating soft paths do not start with |\pgfsys@|, but rather with
|\pgfsyssoftpath@|. On the other hand, as a user you will never use
these commands directly, so they are described as part of the
low-level interface. 



\subsection{Path Creating Process}

When the user writes a command like |\draw (0bp,0bp) -- (10bp,0bp);|
quite a lot happens behind the scenes:
\begin{enumerate}
\item
  The frontend command is translated by |tikz| into commands
  of the basic layer. In essence, the command is translated to
  something like
\begin{codeexample}[code only]
\pgfpathmoveto{\pgfpoint{0bp}{0bp}}
\pgfpathlineto{\pgfpoint{10bp}{0bp}}
\pgfusepath{stroke}
\end{codeexample}
\item
  The |\pgfpathxxxx| command do \emph{not} directly call ``hard''
  commands like |\pgfsys@xxxx|. Instead, the command |\pgfpathmoveto|
  invokes a special command called |\pgfsyssoftpath@moveto| and
  |\pgfpathlineto| invokes |\pgfsyssoftpath@lineto|. 

  The |\pgfsyssoftpath@xxxx| commands, which are described below,
  construct a soft path. Each time such a command is used, special
  tokens are added to the end of an internal macro that stores the
  soft path currently being constructed. 
\item
  When the |\pgfusepath| is encountered, the soft path stored in
  the internal macro is ``invoked.'' Only now does a special macro
  iterate over the soft path. For each lineto or moveto
  operation on this path it calls an appropriate |\pgfsys@moveto| or
  |\pgfsys@lineto| in order to, finally, create the desired hard path,
  namely, the string of literals in the |.pdf| or |.ps| file.
\item
  After the path has been invoked, |\pgfsys@stroke| is called to
  insert the literal for stroking the path.
\end{enumerate}

Why such a complicated process? Why not have |\pgfpathlineto| directly
call |\pgfsys@lineto| and be done with it? There are two reasons:
\begin{enumerate}
\item
  The \pdf\ specification requires that a path is not interrupted with
  any non-path-construction commands. Thus, the following code will
  result in a corrupted |.pdf|:
\begin{codeexample}[code only]
\pgfsys@moveto{0}{0}
\pgfsys@setlinewidth{1}
\pgfsys@lineto{10}{0}
\pgfsys@stroke
\end{codeexample}
  Such corrupt code is \emph{tolerated} by most viewers, but not
  always. It is much better to create only (reasonably) legal code.
\item
  A soft path can still be changed, while a hard path is fixed. For
  example, one can still change the starting and end points of a soft
  path or do optimizations it. Such transformations are not possible
  on hard paths.
\end{enumerate}


\subsection{Starting and Ending a Soft Path}

No special action must be taken in order to start the creation of a
soft path. Rather, each time a command like |\pgfsyssoftpath@lineto|
is called, a special token is added to the (global) current soft path
being constructed.

However, you can access and change the current soft path. In this way,
it is possible to store a soft path, to manipulate it, or to invoke
it.

\begin{command}{\pgfsyssoftpath@getcurrentpath\marg{macro name}}
  This command will store the current soft path in \meta{macro name}.
\end{command}

\begin{command}{\pgfsyssoftpath@setcurrentpath\marg{macro name}}
  This command will set the current soft path to be the path stored in
  \meta{macro name}. This macro should store a path that has
  previously been extracted using the |getcurrentpath| command and has
  possibly been modified subsequently.
\end{command}

\begin{command}{\pgfsyssoftpath@invokecurrentpath}
  This command will turnn the current soft path in a ``hard'' path. To
  do so, it iterates over the soft path and calls an appropriate
  |\pgfsys@xxxx| command for each element of the path. Note that the
  current soft path is \emph{not changed} by this command. Thus, in
  order to start a new soft path after the old one has been invoked
  and is no longer needed, you need to set the current soft path to be
  empty. This may seems strange, but it is often useful to immediately
  use the last soft path again.
\end{command}

\begin{command}{\pgfsyssoftpath@flushcurrentpath}
  This command will invoke the current soft path and then set it to be
  empty. 
\end{command}



\subsection{Soft Path Creation Commands}

\begin{command}{\pgfsyssoftpath@moveto\marg{x}\marg{y}}
  This command appends a ``moveto'' segment to the current soft
  path. The coordinates \meta{x} and \meta{y} are given as big points,
  as always in the system level.

  \example One way to draw a line:
\begin{codeexample}[code only]
\pgfsyssoftpath@moveto{0}{0}
\pgfsyssoftpath@lineto{10}{10}
\pgfsyssoftpath@flushcurrentpath
\pgfsys@stroke
\end{codeexample}
\end{command}

\begin{command}{\pgfsyssoftpath@lineto\marg{x}\marg{y}}
  Appends a ``lineto'' segment to the current soft path. 
\end{command}

\begin{command}{\pgfsyssoftpath@curevto\marg{a}\marg{b}\marg{c}\marg{d}\marg{x}\marg{y}}
  Appends a ``curveto'' segment to the current soft path with controls
  $(a,b)$ and $(c,d)$.
\end{command}

\begin{command}{\pgfsyssoftpath@rect\marg{lower left x}\marg{lower left y}\meta{width}\meta{height}}
  Appends a rectangle segment to the current soft path. 
\end{command}

\begin{command}{\pgfsyssoftpath@closepath}
  Appends a ``closepath'' segment to the current soft path. 
\end{command}




\subsection{The Soft Path Data Structure}

A soft path is stored in a standarized way, which makes it possible to
modify it before it becomes ``hard.'' Basically, a soft path is a long
sequence of triples. Each triple starts with a \emph{token} that
identifies what is going on. This token is followed by two numbers in
braces. For example, the following is a soft path that means ``the
path starts at $(0\operatorname{bp}, 0\operatorname{bp})$ and then
continues in a straight line to $(10\operatorname{bp},
0\operatorname{bp})$.''

\begin{codeexample}[code only]
\pgfsyssoftpath@movetotoken{0}{0}\pgfsyssoftpath@linetotoken{10}{0}
\end{codeexample}

A curveto is hard to express in this way since we need six numbers to
express it, not two. For this reasons, a curveto is expressed using
three triples as follows: A
|\pgfsyssoftpath@curevto{1}{2}{3}{4}{5}{6}| results in the following
three triples:
\begin{codeexample}[code only]
\pgfsyssoftpath@curvetosupportatoken{1}{2}
\pgfsyssoftpath@curvetosupportbtoken{3}{4}
\pgfsyssoftpath@curvetotoken{5}{6}
\end{codeexample}

These three triples must always ``remain together.'' Thus, a lonely
|supportbtoken| is forbidden.

In details, the following tokens exist:
\begin{itemize}
\item
  \declare{|\pgfsyssoftpath@movetotoken|} indicates a moveto
  operation. The two following numbers indicate the position to which
  the current point should be moved.
\item
  \declare{|\pgfsyssoftpath@linetotoken|} indicates a lineto
  operation. 
\item
  \declare{|\pgfsyssoftpath@curvetosupportatoken|} indicates the first
  control point of a curveto operation. The triple must be followed
  by a |\pgfsyssoftpath@curvetosupportbtoken|.
\item
  \declare{|\pgfsyssoftpath@curvetosupportbtoken|} indicates the second
  control point of a curveto operation. The triple must be followed
  by a |\pgfsyssoftpath@curvetotoken|.
\item
  \declare{|\pgfsyssoftpath@curvetotoken|} indicates the target
  of a curveto operation.
\item
  \declare{|\pgfsyssoftpath@rectcornertoken|} indicates the corner of
  a rectangle on the soft path. The triple must be followed
  by a |\pgfsyssoftpath@rectsizetoken|.
\item
  \declare{|\pgfsyssoftpath@rectcornertoken|} indicates the size of
  a rectangle on the soft path.
\item
  \declare{|\pgfsyssoftpath@closepath|} indicates that the subpath
  begun with the last moveto operation should be closed. The parameter
  numbers are currently not important, but if set to anything
  different from |{0}{0}|, they should be set to the coordinate of the
  original moveto operation to which the path ``returns'' now.
\end{itemize}





% Copyright 2003 by Till Tantau <tantau@cs.tu-berlin.de>.
%
% This program can be redistributed and/or modified under the terms
% of the LaTeX Project Public License Distributed from CTAN
% archives in directory macros/latex/base/lppl.txt.

\section{The Protocol Subsystem}

\label{section-protocols}

\makeatletter

This section describes commands for \emph{protocolling} literal text
created by \pgfname. The idea is that some literal text, like the string
of commands used to draw an arrow head, will be used over and over
again in a picture. It is then much more efficient to compute the
necessary literal text just once and to quickly insert it ``in a
single sweep.''

When protocolling is ``switched on,'' there is a ``current protocol''
to which literal text gets appended. Once all commands that needed to
be protocoled have been issued, the protocol can be obtained and
stored using |\pgfsysprotocol@getcurrentprotocol|. At any point, the
current protocol can be changed using a corresponding setting
command. Finally, |\pgfsysprotocol@invokecurrentprotocol| is used to
insert the protocoled commands into the |.pdf| or |.dvi| file.

Only those |\pgfsys@| commands can be protocolled that use the
command |\pgfsysprotocol@literal| interally. For example, the
definition of |\pgfsys@moveto| in |pgfsys-common-pdf.def| is
\begin{codeexample}[code only]
\def\pgfsys@moveto#1#2{\pgfsysprotocol@literal{#1 #2 m}}
\end{codeexample}
All ``normal'' system-level commands can be protocolled. However,
commands for creating or invoking shadings, images, or whole pictures
require special |\special|'s and cannot be protocolled.

\begin{command}{\pgfsysprotocol@literalbuffered\marg{literal text}}
  Adds the \meta{literal text} to the current protocol, after it has
  been ``|\edef|ed.'' This command will always protocol.
\end{command}

\begin{command}{\pgfsysprotocol@literal\marg{literal text}}
  First calls |\pgfsysprotocol@literalbuffered| on \meta{literal
    text}. Then, if protocolling is currently switched off, the
  \meta{literal text} is passed on to |\pgfsys@invoke|.
\end{command}

\begin{command}{\pgfsysprotocol@bufferedtrue}
  Turns on protocolling. All subsequent calls of
  |\pgfsysprotocol@literal| will append their argument to the current
  protocol. 
\end{command}

\begin{command}{\pgfsysprotocol@bufferedfalse}
  Turns off protocolling. Subsequent calls of
  |\pgfsysprotocol@literal| directly insert their argument into the
  current |.pdf| or |.ps|.

  Note that if the current protocol is not empty when protocolling is
  switched off, the next call to |\pgfsysprotocol@literal| will first
  flush the current protocol, that is, insert it into the file.
\end{command}

\begin{command}{\pgfsysprotocol@getcurrentprotocol\marg{macro name}}
  Stores the current protocol in \meta{macro name} for later use.
\end{command}

\begin{command}{\pgfsysprotocol@setcurrentprotocol\marg{macro name}}
  Sets the current protocol to \meta{macro name}.
\end{command}

\begin{command}{\pgfsysprotocol@invokecurrentprotocol}
  Inserts the text stored in the current protocol into the |.pdf| or
  |.dvi| file. This does \emph{not} change the current protocol.
\end{command}

\begin{command}{\pgfsysprotocol@flushcurrentprotocol}
  First inserts the current protocol, then sets the current protocol
  to the empty string.
\end{command}


%%% Local Variables: 
%%% mode: latex
%%% TeX-master: "pgfmanual"
%%% End: 




\part{References and Index}

\vskip1cm
\begin{codeexample}[graphic=white]
\begin{tikzpicture}
  \draw[line width=0.3cm,color=red!30,cap=round,join=round] (0,0)--(2,0)--(2,5);
  \draw[help lines] (-2.5,-2.5) grid (5.5,7.5);
  \draw[very thick] (1,-1)--(-1,-1)--(-1,1)--(0,1)--(0,0)--
    (1,0)--(1,-1)--(3,-1)--(3,2)--(2,2)--(2,3)--(3,3)--
    (3,5)--(1,5)--(1,4)--(0,4)--(0,6)--(1,6)--(1,5)
    (3,3)--(4,3)--(4,5)--(3,5)--(3,6)
    (3,-1)--(4,-1);
  \draw[below left] (0,0) node(s){$s$};
  \draw[below left] (2,5) node(t){$t$};
  \fill (0,0) circle (0.06cm) (2,5) circle (0.06cm);
  \draw[->,rounded corners=0.2cm,shorten >=2pt]
    (1.5,0.5)-- ++(0,-1)-- ++(1,0)-- ++(0,2)-- ++(-1,0)-- ++(0,2)-- ++(1,0)--
    ++(0,1)-- ++(-1,0)-- ++(0,-1)-- ++(-2,0)-- ++(0,3)-- ++(2,0)-- ++(0,-1)--
    ++(1,0)-- ++(0,1)-- ++(1,0)-- ++(0,-1)-- ++(1,0)-- ++(0,-3)-- ++(-2,0)--
    ++(1,0)-- ++(0,-3)-- ++(1,0)-- ++(0,-1)-- ++(-6,0)-- ++(0,3)-- ++(2,0)--
    ++(0,-1)-- ++(1,0);
\end{tikzpicture}
\end{codeexample}

\printindex

\end{document}



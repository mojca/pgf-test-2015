% This file has been generated from the lua sources using LuaDoc.
% To regenerate it call "make genluadoc" in
% doc/generic/pgf/version-for-luatex/en.

\paragraph{pgflibrarygraphdrawing-sys.lua}


\begin{luacommand}{{Sys:beginShipout}()}
Begins the shipout of nodes by opening a scope in pgf.



\end{luacommand}\begin{luacommand}{{Sys:endShipout}()}
Ends the shipout by closing the opened scope.



See also:
\begin{itemize}
	\item[] |Sys:beginShipout()|
\end{itemize}

\end{luacommand}\begin{luacommand}{{Sys:escapeTeXNodeName}(\meta{nodename})}
Adds a ``not yet positionedPGFGDINTERNAL'' prefix to a node name. The prefix is required by pgf to place the node. Actually, when deferring the node placement, the prefix is added to avoid references to the node.

Parameters:
\begin{itemize}
	\item[] \meta{nodename} \subitem Name of the node to prefix.
\end{itemize}


Return value:
\begin{itemize} \item[] A newly composed string. \end{itemize}


\end{luacommand}\begin{luacommand}{{Sys:getTeXBox}()}
Retrieves a box from the transfer box register.



See also:
\begin{itemize}
	\item[] |putTeXBox|
\end{itemize}

\end{luacommand}\begin{luacommand}{{Sys:getVerboseMode}()}
Checks the verbosity of the subsystems output.


Return value:
\begin{itemize} \item[] Boolean value specifying the verbosity. \end{itemize}


\end{luacommand}\begin{luacommand}{{Sys:logMessage}(\meta{...})}
Prints objects to the TeX output, formatting them with tostring.

Parameters:
\begin{itemize}
	\item[] \meta{...} \subitem List of parameters.
\end{itemize}



\end{luacommand}\begin{luacommand}{{Sys:putEdge}(\meta{edge},\meta{Edge})}
Assembles and outputs the TeX command to draw an edge.

Parameters:
\begin{itemize}
	\item[] \meta{Edge} \subitem A lua edge object.
\end{itemize}



\end{luacommand}\begin{luacommand}{{Sys:putTeXBox}(\meta{nodename},\meta{texnode},\meta{minX},\meta{minY},\meta{maxX},\meta{maxY},\meta{posX},\meta{posY},\meta{nodeName})}
Saves a box from the transfer box register.

Parameters:
\begin{itemize}
	\item[] \meta{texnode} \subitem The box which contains the \TeX\ node.\item[] \meta{minX} \subitem Maximum y of the bounding box.\item[] \meta{minY} \subitem Minimal y of the bounding box.\item[] \meta{posX} \subitem X coordinate where to put the node in the output.\item[] \meta{posY} \subitem Y coordinate where to put the node in the output.\item[] \meta{nodeName} \subitem The name of the node in the box.
\end{itemize}



\end{luacommand}\begin{luacommand}{{Sys:setBoxNumber}(\meta{bn})}
Init method, sets the box register number. This method is called when the \tikzname\ (pgf) library is loaded.

Parameters:
\begin{itemize}
	\item[] \meta{bn} \subitem Number of the box register used for transfering boxes of the current graph.
\end{itemize}



\end{luacommand}\begin{luacommand}{{Sys:setVerboseMode}(\meta{mode})}
Enables or disables verbose logging for the graph drawing library.

Parameters:
\begin{itemize}
	\item[] \meta{mode} \subitem If true, enable verbose logging. Otherwise it'll be disabled.
\end{itemize}



\end{luacommand}\begin{luacommand}{{Sys:unescapeTeXNodeName}(\meta{nodename})}
Removes the ``not yet positionedPGFGDINTERNAL'' prefix from a node name.

Parameters:
\begin{itemize}
	\item[] \meta{nodename} \subitem Nodename without prefix.
\end{itemize}


Return value:
\begin{itemize} \item[] The substring in question. \end{itemize}


See also:
\begin{itemize}
	\item[] |Sys:escapeTeXNodeName(nodename)|
\end{itemize}

\end{luacommand}

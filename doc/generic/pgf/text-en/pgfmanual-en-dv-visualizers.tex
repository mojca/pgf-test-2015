% Copyright 2010 by Till Tantau
%
% This file may be distributed and/or modified
%
% 1. under the LaTeX Project Public License and/or
% 2. under the GNU Free Documentation License.
%
% See the file doc/generic/pgf/licenses/LICENSE for more details.


\section{Visualizers}
\label{section-dv-visualizers}

\subsection{Overview}

In a data visualization a long stream of data points is
\emph{visualized} using \emph{visualizers}. Recall that it is the job
of the axis systems as described in Section~\ref{section-dv-axes} to
determine \emph{where} data points are visualized. It is the job of
the visualizers to determine \emph{how} they are visualized.

The most basic and common visualizer is the \emph{line visualizer}. It
simply connects subsequent data points by straight lines to indicate
either that the points on these lines interpolate between the real
data points or the straight lines are used to indicate the order in
which the data points appear. A different, more ``conservative''
visualizer is the \emph{scatter visualizer} or \emph{mark visualizer},
which just places a small mark at each data point. Such a visualizer
does not imply any interpolation or ordering between the data points.

Visualizers may, however, also be more complicated. For instance, a
visualizer used for a box plot could visualize a data point as a box
with a median value, standard deviation, outliers, and other
information; a rectangle visualizer might visualize data points as
larger areas; a projection visualizer might visualize the projection
of data points onto different axes; and so.

Creating a new visualizer is not quite trivial since a new \pgfname\
class needs to be implemented. Fortunately, using visualizers is much
simpler: For each kind of visualizer there is a key that allows you to
create such a visualizer. You can then use further keys to configure
the visualizer and to connect it to the data.

In a data visualization multiple visualizers may exist at the same
time. This happens in different situations:
\begin{itemize}
\item A data visualization may contain several independent data sets
  that are to be visualized. There might be a line plot, for which a
  line visualizer is used, and also a scatter plot, for which a
  scatter visualizer would be used.

  In this case, for each data point only one visualizer will do
  anything. To achieve this, each data point has an attribute called
  |visualizer| which tells the visualizer objects whether they should
  ``react'' to the data point or not.
\item A single data point might be visualized several times. For
  instance, a scatter visualizer might draw a mark at the data point's
  position on the page and a projection visualizer might draw,
  additionally, a mark at the projected position.
\end{itemize}


\subsection{Usage}

\subsubsection{Using a Single Visualizer}

The simplest scenario for using visualizers are data visualizations in
which there is only a single data set that is visualized in one
style. In this case, all that needs to be done in order to choose a
visualizer is use one of the options starting with |visualize as ...|
together with the |\datavisualization| command:

\begin{codeexample}[]
% Define a data set:  
\tikz \datavisualization data set {example} = {
data {
  x, y
  0, 0
  0.5, 2
  1, 2
  1.5, 1.5
  2, 0.5
}};
\tikz \datavisualization [school book axes, visualize as line]        data set {example};
\qquad 
\tikz \datavisualization [school book axes, visualize as smooth line] data set {example};
\qquad 
\tikz \datavisualization [school book axes, visualize as scatter]     data set {example};
\end{codeexample}

Methods for styling visualizers are discussed in Section~\ref{section-dv-visualizer-styling}.


\subsubsection{Using Multiple Visualizers}

A data visualization may contain multiple data sets and for each data
set we might wish to use a different visualizer. In this case, we need
some way of telling the data visualization engine to which visualizer
should be used with the different data points.

To solve this problem, you can \emph{name} a visualizer. The
visualizer's name can then both be used to configure the visualizer
and also to indicate that data points ``belong'' to the visualizer.

Naming a visualizer is quite simple: The |visualize as ...| keys
actually take a single parameter, which is the name of the
visualizer. For instance, the following code creates three
visualizers, named |sin|, |cos|, and |tan|:

\begin{codeexample}[code only]
visualize as line=sin,
visualize as line=cos,
visualize as scatter=tan
\end{codeexample}

(When you just say |visualize as line| without providing a name, the
name |line| is chosen as a default, for |visualize as scatter| the
name |scatter| is the default and so.)

In order to indicate which data points should be visualized by which
of these visualizers, the following key is important:

\begin{key}{/data point/visualizer}
  A visualizer will only act on a data point when its name matches the
  value of this key. Initially, it is set to the last visualizer
  created, so if there is only one, there is no need to set or worry
  about this key.
\end{key}

Since the |visualizer| key has the path prefix |/data point|, it can
be set like any other attribute of a data key:

\begin{codeexample}[width=7cm]
\tikz \datavisualization
 [scientific clean axes,
  visualize as line=sin,
  visualize as line=cos,
  visualize as scatter=tan]
data {
  x, y, visualizer
  0, 0, sin
  1, 1, sin
  2, 0, sin
  3, -1, sin
  4, 0, sin
  0, 1, cos
  1, 0, cos
  0, 0, tan
  1, 1, tan
  2, 2, tan
  3, 4, tan
  2, -1, cos
  3, 0, cos
  4, 1, cos
};
\end{codeexample}

As can be seen, the data points with the same |visualizer| attribute
do not need to be consecutive.

The above method of specifying the visualizer works nicely, but in
most cases it would be more natural to keep the |visualizer| attribute
out of the table. This is easy to achieve by using multiple |data|
blocks and by simply adding the options like
|[/data point/visualizer=sin]| to the |data| blocks:

\begin{codeexample}[width=7cm]
\tikz \datavisualization
 [scientific clean axes,
  visualize as line=sin,
  visualize as line=cos]
data [/data point/visualizer=sin] {
  x, y
  0, 0
  1, 1
  2, 0
  3, -1
  4, 0
}
data [/data point/visualizer=cos] {
  x, y
  0, 1
  1, 0
  2, -1
  3, 0
  4, 1
};
\end{codeexample}

It turns out that there is an even simpler way of achieving the above:
You can simply pass the name of the visualizer as an option to the
|data| block command and it will setup the |visualizer| key
correctly. This works because when a visualizer is named using
|visualize as ...| in addition to creating a visualizer object, the
following key is also setup: 
\begin{key}{/pgf/data/\meta{visualizer name}}
  This key is a shorthand for
  |/data point/visualizer=|\meta{visualizer name}.  
\end{key}

\begin{codeexample}[width=7cm]
\tikz \datavisualization
 [scientific clean axes,
  visualize as line=sin,
  visualize as line=cos]
data [sin] {
  x, y
  0, 0
  1, 1
  2, 0
  3, -1
  4, 0
}
data [cos] {
  x, y
  0, 1
  1, 0
  2, -1
  3, 0
  4, 1
};
\end{codeexample}

When you need to visualize several similar things in a single plot
(like ten lines that all get visualized by |visualize as line|), it is
somewhat cumbersome having to write this ten times. In this case you
can shorten your code by making use of the |.list| key handler: When
you add it to a key, the ``value'' passed to the key is parsed as a
list of values. The key is then executed once for each of these
values:

\begin{codeexample}[width=7cm]
\tikz \datavisualization
 [scientific clean axes,
  visualize as line/.list={sin, cos, tan}]
data [sin, format=function] {
  var x : interval[0:3*pi];
  func y = sin(\value x r);
}
data [cos, format=function] {
  var x : interval[0:3*pi];
  func y = cos(\value x r);
}
data [tan, format=function] {
  var x : interval[0:pi/2.2];
  func y = tan(\value x r);
};
\end{codeexample}



\subsubsection{Styling a Visualizer}
\label{section-dv-visualizer-styling}

In order to style a visualizer that has been created using for
instance |visualize as line=|\meta{visualizer name}, you can use the
following key: 

\begin{key}{/tikz/data visualization/\meta{visualizer
      name}=\meta{options}}
  For each visualizer, a key of the same name is created with the path
  prefix |/tikz/data visualization|. This key takes the \meta{options}
  and executes them with the path prefix
\begin{codeexample}[code only]
/tikz/data visualization/visualizer options/  
\end{codeexample}
  These options are then used to configure the appearance of the
  current visualizer. (This is quite similar to the way options are
  passed to an axis in order to configure the axis.)
  Possible options include |style|, but also |label in legend| and
  |label in data|. The latter two options are discussed in
  Section~\ref{section-dv-labels-in}, the first option below.

\begin{codeexample}[width=7cm]
\tikz \datavisualization
 [scientific clean axes,
  visualize as smooth line/.list={sin, cos},
  sin={style=red},
  cos={style=blue}]
data [sin, format=function] {
  var x : interval[0:3*pi];
  func y = sin(\value x r);
}
data [cos, format=function] {
  var x : interval[0:3*pi];
  func y = cos(\value x r);
};
\end{codeexample}
  
  (Note that the key |/pgf/data/|\meta{visualizer name} is also
  created for each visualizer, but the purpose of that key is to
  change the |/data point/visualizer| attribute, whereas the purpose
  of the present key is to configure the visualizer.)
\end{key}

\begin{key}{/tikz/data visualization/visualizer
    options/style=\meta{options}}
  The \meta{options} given to this key should be normal \tikzname\
  options. They will be executed when the visualizer is used.

\begin{codeexample}[width=7cm]
\tikz \datavisualization
 [scientific clean axes,
  visualize as smooth line=sin,
  sin={style={red, densely dotted}},
  visualize as smooth line=cos,
  cos={style={mark=x}},
]
data [sin, format=function] {
  var x : interval[0:3*pi];
  func y = sin(\value x r);
}
data [cos, format=function] {
  var x : interval[0:3*pi];
  func y = cos(\value x r);
};
\end{codeexample}
\end{key}

In addition to the options passed to a visualizer via |style|, the
following also gets executed when a visualizer is used:

\begin{stylekey}{/tikz/data visualization/every visualizer}
  This style is used with every visualizer. Note that it should
  contain normal \tikzname\ keys.

\begin{codeexample}[width=7cm]
\tikz \datavisualization
 [scientific clean axes,
  every visualizer/.style={dashed},
  visualize as smooth line]
data [format=function] {
  var x : interval[0:3*pi];
  func y = sin(\value x r);
};
\end{codeexample}
\end{stylekey}


\subsection{Reference: Basic Visualizers}

\begin{key}{/tikz/data visualizers/visualize as line=\meta{visualizer
      name} (default line)}
  Creates a new visualizer named \meta{visualizer name}. Basically, 
  this visualizer connects all data points for which the
  |/data point/visualizer| attribute equals \meta{visualizer name} by
  a line that is styled by the visualizer's style.

  In more detail, the following happens:
  \begin{enumerate}
  \item A new object is created (of class |plot handler visualizer|)
    that is configured to collect the canvas positions of all data
    points whose |visualizer| attribute equals \meta{visualizer name}.
  \item During the end of the data visualization, \pgfname's plotting
    mechanism (see Section~\ref{section-plots}) is used to plot the
    stream of recorded data points.

    This means that, in principle, all of the plot handlers available
    in \tikzname\ could be used for the visualization (such as the
    |smooth| handler). However, some plot handlers such as, say, the
    |xcomb| are unsuitable as plot handlers since they do not support
    the advanced axis handling done by the data visualization
    engine. Because of this (and also for other reasons), you cannot
    set the plot handler directly, but must use one of the options
    described in a moment. 
  \item Additionally, plot marks can be drawn at the collected data
    points. Here, all of the options available to \tikzname\ for
    drawing plot marks are available.
  \end{enumerate}
  
  The following keys can be used for changing the plot handler (the
  way the ``line'' is rendered).
  
  \begin{key}{/tikz/data visualization/visualizer options/straight line}
    Causes the data points to be connected by straight lines.
\begin{codeexample}[]
\tikz [scale=.55] \datavisualization
 [scientific clean axes, all axes={ticks=few},
  visualize as smooth line=my data,  my data={straight line}]
data [format=function] {
  var t : interval [0:4] samples 5;
  func x = cos(\value t r);
  func y = sin(\value t r);
};
\end{codeexample}
  \end{key}

  \begin{key}{/tikz/data visualization/visualizer options/straight cycle}
    Causes the data points to be connected by a polygon.
\begin{codeexample}[]
\tikz [scale=.55] \datavisualization
 [scientific clean axes, all axes={ticks=few},
  visualize as smooth line=my data,  my data={straight cycle}]
data [format=function] {
  var t : interval [0:4] samples 5;
  func x = cos(\value t r);
  func y = sin(\value t r);
};
\end{codeexample}
 \end{key}
 
 \begin{key}{/tikz/data visualization/visualizer options/polygon}
   This is an alias for |straight cycle|.
 \end{key}
 
 \begin{key}{/tikz/data visualization/visualizer options/smooth line}
   Causes the data points to be connected by a line that is smoothed
   at the joins:
\begin{codeexample}[]
\tikz [scale=.55] \datavisualization
 [scientific clean axes, all axes={ticks=few},
  visualize as smooth line=my data,  my data={smooth line}]
data [format=function] {
  var t : interval [0:4] samples 5;
  func x = cos(\value t r);
  func y = sin(\value t r);
};
\end{codeexample}
 \end{key}
 
 \begin{key}{/tikz/data visualization/visualizer options/smooth cycle}
   Causes the data points to be connected by a circular line that is
   smoothed at the joins:
\begin{codeexample}[]
\tikz [scale=.55] \datavisualization
 [scientific clean axes, all axes={ticks=few},
  visualize as smooth line=my data,  my data={smooth cycle}]
data [format=function] {
  var t : interval [0:4] samples 5;
  func x = cos(\value t r);
  func y = sin(\value t r);
};
\end{codeexample}
 \end{key}
 
 \begin{key}{/tikz/data visualization/visualizer options/gap line}
   This key causes the data points to be connected by lines that ``do
   not quite touch'' the data points. This is implemented by using the
   |\pgfplothandlergaplineto|, see Section~\ref{section-plot-gapped}. 
\begin{codeexample}[]
\tikz [scale=.55] \datavisualization
 [scientific clean axes, all axes={ticks=few},
  visualize as smooth line=my data,  my data={gap line}]
data [format=function] {
  var t : interval [0:4] samples 5;
  func x = cos(\value t r);
  func y = sin(\value t r);
};
\end{codeexample}
 \end{key}
 
 \begin{key}{/tikz/data visualization/visualizer options/gap cycle}
   Like |gapped line|, only with a cycle:
\begin{codeexample}[]
\tikz [scale=.55] \datavisualization
 [scientific clean axes, all axes={ticks=few},
  visualize as smooth line=my data,  my data={gap cycle}]
data [format=function] {
  var t : interval [0:4] samples 5;
  func x = cos(\value t r);
  func y = sin(\value t r);
};
\end{codeexample}
 \end{key}
 
 \begin{key}{/tikz/data visualization/visualizer options/no lines}
   Suppresses the line. This option only makes sense when the |mark|
   option is used.
\begin{codeexample}[]
\tikz [scale=.55] \datavisualization
 [scientific clean axes, all axes={ticks=few},
  visualize as smooth line=my data,  my data={no lines, style={mark=x}}]
data [format=function] {
  var t : interval [0:4] samples 5;
  func x = cos(\value t r);
  func y = sin(\value t r);
};
\end{codeexample}
 \end{key}

 As indicated earlier, marks will be drawn at the data points when the
 |mark| option is used. All options offered by \tikzname\ for
 configuring marks are available such as |mark repeat|: 
\begin{codeexample}[width=7cm]
\tikz \datavisualization
 [scientific clean axes, 
  visualize as line=my data,
  my data={style={mark=x, mark repeat=3}}]
data [format=function] {
  var x : interval [0:pi] samples 10;
  func y = sin(\value x r);
};
\end{codeexample} 
\end{key}


\begin{key}{/tikz/data visualizers/visualize as smooth line=\meta{visualizer
      name} (default line)}
  A shorthand |visualize as line=|\meta{visualizer name}
  followed \meta{visualizer name}|=smooth line|.
\end{key}

\begin{key}{/tikz/data visualizers/visualize as scatter=\meta{visualizer
      name} (default scatter)}
  A shorthand  |visualize as line=|\meta{visualizer name}
  followed  \meta{visualizer name}|=no lines| and setting
  the |style| of the visualizer so that is will use |mark=x| (plus
  some size adjustments) to draw marks at the data points. 
\begin{codeexample}[width=7cm]
\tikz \datavisualization
 [scientific clean axes, 
  visualize as scatter]
data [format=function] {
  var x : interval [0:pi] samples 10;
  func y = sin(\value x r);
};
\end{codeexample} 
\end{key}


\subsection{Advanced: Creating New Visualizers}

To be written...
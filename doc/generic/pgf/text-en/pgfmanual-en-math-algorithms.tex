% Copyright 2007 by Mark Wibrow
%
% This file may be distributed and/or modified
%
% 1. under the LaTeX Project Public License and/or
% 2. under the GNU Free Documentation License.
%
% See the file doc/generic/pgf/licenses/LICENSE for more details.

\section{Reimplementing the Algorithms}

\label{pgfmath-reimplement}

Perhaps you are not satisfied with the Newton-Raphson approximation to
square-roots. Perhaps you have a fantastically more accurate
and efficient way of calculating the sine or cosine of angles. Perhaps
 you would like the library to interface with a package such as |fp| 
 for fixed-point arithmetic (but note that |fp| is \emph{very} slow).
How can you change the calculation of any or all of the implemented
algorithms? With the greatest of ease.

The all the algorithms for the operations and functions (with
one or two exceptions) are implemented in the following manner:

\begin{itemize}
\item |\pgfmathXXX| 

	This macro is the ``public'' interface for the function (here, the
	`mystery' function |XXX|). All arguments passed to this macro are 
	evaulated using |\pgfmathparse| and then passed on to\ldots

\item |\pgfmathXXX@|

	This macro is the ``non-public'' implementation of the functions 
	algorithm (but note that, for speed, the parser calls this macro 
	rather than the ``public'' one). Arguments passed to this macro 
	are expected to be numbers \emph{without units}. This is the macro 
	which should be rewritten with your prize-winning new algorithm.
	
\end{itemize}

The format for |\pgfmathXXX@| is very simple:

\begin{codeexample}[code only]
\def\pgfmathXXX@#1#2...{%
   \begingroup%
      ... code for algorithm XXX ...
      \pgfmath@returnone\pgfmath@x%
   \endgroup%
}
\end{codeexample}


The macro |\pgfmath@returnone<macro>| uses some |\aftergroup| magic to
save result of the algorithm, by defining |\pgfmathresult| as the 
expansion of |<macro>| \emph{without units}. |<macro>| can be a macro
containing a number (with or without units), or a dimension or count
(or possibly a skip) register. By performing the algorithm within a
\TeX{} group, \pgfname{} registers such as |\pgf@x|, |\pgf@y| and 
|\c@pgf@counta|, |\c@pgfcountb|, and so forth can be used at will.
Note that current the implementation uses |\pgfmath@x|, |\pgfmath@y|, 
and |\c@pgfmath@counta|, |\c@pgfmath@countb| throughout, so for 
consistency these should be employed. Whilst they are currently |\let|
to their \pgfname{} equivalents (see |pgfmathutil.code.tex|), this 
could change (as could the \pgfname{} registers), so keeping things
consistent is probably a good idea.
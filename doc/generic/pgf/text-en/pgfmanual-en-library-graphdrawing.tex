% Copyright 2010 by Renée Ahrens, Olof Frahm, Jens Kluttig, Matthias Schulz, Stephan Schuster
%
% This file may be distributed and/or modified
%
% 1. under the LaTeX Project Public License and/or
% 2. under the GNU Free Documentation License.
%
% See the file doc/generic/pgf/licenses/LICENSE for more details.

\section{Algorithmic Graph Drawing}
{\noindent {\emph{by Ren\'{e}e Ahrens, Olof-Joachim Frahm,
Jens Kluttig, Matthias Schulz and Stephan Schuster}}} 
\label{section-library-graphdrawing}

\begin{tikzlibrary}{graphdrawing}
This package provides capabilities for automatic layouting of graphs.
\end{tikzlibrary}

\ifluatex\relax\else{LuaTeX is required for setting this manual section.}\endinput\fi

\subsection{Requirements}
Since automatic graph drawing can be a complex thing, this package does not
rely on the limited computational capabilities of \TeX, but requires \LuaTeX, a
new version of pdf\TeX\ including the small, but powerful scripting language Lua.

This package should work with \LuaTeX\ 0.4 or higher. By now
the latest releases of the major \TeX\ distributions all include \LuaTeX.

\subsection{Overview}
Drawing a huge graph can be quite labor-intensive. This package aims to reduce
this amount of work for cases that do not require a certain layout of the graph.
It provides a set of algorithms trying to create a readable layout for a given
graph that is specified by its nodes and edges, but not by any coordinates.
These algorithms are implemented in Lua, therefore \LuaTeX\ is required
to use this package.  

The following example produces a simple tree.

\begin{codeexample}[]
\tikzpicture[graph drawing=standard tree]
  \graph { a [root] ->{b -> {c,d}, e}};
\endtikzpicture
\end{codeexample}

As you can see, the text nodes aren't quite aligned, so the common fix
is to use the |text depth| and |text height| keys to force the text
nodes to a specific size.

\begin{codeexample}[]
\tikzpicture[graph drawing=standard tree, text depth=.2em, text height=.8em]
  \graph { a [root] ->{b -> {c,d}, e}};
\endtikzpicture
\end{codeexample}

The |graph| library provides an intuitive approach to specify which nodes and edges are present, but its main purpose is not easing the layouting of a graph. The graph drawing library aims to support the task of visualizing a graph by providing a set of algorithms for automated placement of the nodes. 
% TODO: some more details

\subsection{The Algorithms}
Currently two layouting strategies are provided by the library that will be explained in detail now. 
The first paragraph on each algorithm will tell you what you need to now as a normal user. The rest covers details you might be interested in if you were going to implement own layouting strategies.

\begin{gdalgorithm}{standard tree}
The tree algorithm positions the nodes of a tree. Therefore you have to ensure that the given graph is in fact a tree. Otherwise an error is raised. By default the root of the tree is positioned horizontally centered at the top. Child nodes are positioned beside each other according to their level. 

The tree algorithm can be found in the file |pgflibrarygraphdrawing-algorithms-arbitrarytree.lua| and is part of the |pgf.graphdrawing| Lua module.

\paragraph{Usage.} 

To use this algorithm you have to use the \tikzname\ key |graph drawing|. By setting its value to |standard tree| the tree algorithm layouts the graph.
\begin{codeexample}[]
\tikz [graph drawing=standard tree]
  \graph { 1 [root] ->{2 , 3}};
\end{codeexample}
\paragraph{Parameters.} 
The keys affecting the algorithm are |root|, |sibling distance| and |level distance|.

\begin{key}{/tikz/graphs/graph drawing/root}
  Determines the root of a tree and is mandatory for this algorithm.
\end{key}

\begin{key}{/tikz/graphs/graph drawing/level distance=\meta{leveldistance} (default 1)}
  |level distance| is optional and affects the horizontal space between the nodes.
  The following example shows the usage of |level distance| with the values 1 cm and 2 cm.
  \begin{codeexample}[]
    \tikz [graph drawing={standard tree, level distance=1 cm}]
    \graph { 1 [root] ->{2 , 3}};
    \tikz [graph drawing={standard tree, level distance=2 cm}]
    \graph { 1 [root] ->{2 , 3}};
  \end{codeexample}
  Note that it is not possible to change |level distance| within a |tikzpicture| environment.
\end{key}

\begin{key}{/tikz/graphs/graph drawing/sibling distance=\meta{siblingdistance} (default 1)}
  |sibling distance| is optional and affects the vertical space between the nodes.
  The following example shows the usage of |sibling distance| with the values 1 cm and 2 cm.
  \begin{codeexample}[]
    \tikz [graph drawing={standard tree, sibling distance=1 cm}]
    \graph { 1 [root] ->{2 , 3}};
    \tikz [graph drawing={standard tree, sibling distance=2 cm}]
    \graph { 1 [root] ->{2 , 3}};
  \end{codeexample}
  Note that it is not possible to change |sibling distance| within a |tikzpicture| environment.
\end{key}

\paragraph{How does this algorithm work?}
The tree algorithm works recursively. During the recursion one step is performed for each subgraph of the tree. 

The process builds a kind of a box structure of the given graph. This means a leaf of a tree returns itself as a box. Its parent returns itself and its children in a bigger box etc. as shown in the following figure.

\begin{quote}
\begin{tikzpicture}[
    level 0/.style={draw=black!50,very thick},
    level 1/.style={draw=orange!50,very thick},
    level 2/.style={draw=blue!50,very thick},
    level 3/.style={draw=green!50,very thick}]

    \node[level 1] (1) {1}
      child {node[level 2] (3) {3}
        child {node[level 3] (4) {4}
          child{node[level 3] (6) {6}}
          child{node[level 3] (7) {7}}
        }
        child {node[level 2] (5) {5}}
      }
      child {node[level 1] (2) {2}};

    \begin{pgfonlayer}{background}
        \node [level 0, fit=(1) (6) (2)] {};
        \node [level 1, fit=(3) (5) (6) (7)] {};
        \node [level 2,fit=(4) (6) (7)] {};
    \end{pgfonlayer}
\end{tikzpicture}
\end{quote}

In each step the current boxes can be compared by their size, sorted and positioned. In the figure above the boxes of one step are represented in the same color.

In the tree algorithm the boxes of each tree level are first sorted ascendingly by their size and then arranged as follows: The the biggest box is positioned in the middle. Then the following boxes are positioned alternately left and right.

After this arrangement the relative coordinates for the position of each box have to be computed. The \emph{y}-coordinate of a box (except for the root node of the step) are determined by the maximum height of all boxes to guarantee a uniform layout of the tree. Nodes on the same level in the tree are positioned at the same height. The \emph{x}-coordinate of a box depends on the coordinates of its left neighbour box and an additional spacing (by default 10pt), which can be influenced by the |sibling distance| key. The \emph{y}-coordinate of the root node of each step is set to the maximum \emph{y}-value of the other boxes adding the same spacing meantioned above (by default 10pt, influenced by |level distance|). Its \emph{x}-coordinate is determined by the width of the other boxes divided by 2. This means the root node is positioned in the middle above the other boxes.

Because each box knows its root node, it is possible to determine the absolute position of each box or node afterwards. 

At the end of the step the current boxes are added to a result box and returned.
\end{gdalgorithm}

\begin{gdalgorithm}{few intersections}
For computing the layout of an arbitrary graph you can use the |few intersections|-option. As input any graph is feasible. Note that for complex graphs the time to compute a layout can be long.

The algorithm can be found in the file
|pgflibrarygraphdrawing-algorithms-localsearchgraph.lua| and is part
of the |pgf.graphdrawing| Lua module.

\paragraph{Usage.}

\begin{key}{/tikz/graphs/graph drawing/few intersections}
  Just specifying the |few intersections| key is enough to enable this algorithm.
\end{key}

\begin{codeexample}[]
\tikz [graph drawing={few intersections}, scale=2]
  \graph { 1 -> {2, 3} -> 4};
\end{codeexample}

\begin{codeexample}[]
\tikz [graph drawing={few intersections}, scale=2]
  \graph { 6 -> 3 -> 5 -> 1 ->{2 -> {3, 4, 6}}, 5 -> 2};
\end{codeexample}

The algorithm that is going to be described now provides a standard approach for generating a layout for a generic graph.
The resulting layout of this algorithm can occupy lots of space. Therefore you can limit its height and width.

\begin{key}{/tikz/graphs/graph drawing/max height=\meta{height} (default max height of all nodes multiplied two times by number of nodes)}
Limits the height of the drawing.
\begin{codeexample}[]
\tikz [graph drawing={few intersections, max height=50pt}, scale=2]
  \graph { 6 -> 3 -> 5 -> 1 ->{2 -> {3, 4, 6}}, 5 -> 2};
\end{codeexample}
\end{key}

\begin{key}{/tikz/graphs/graph drawing/max width=\meta{width} (default max width of all nodes multiplied two times by number of nodes)}
Limits the width of the drawing.
\begin{codeexample}[]
\tikz [graph drawing={few intersections, max width=50pt, max height=50pt},
              scale=2]
  \graph { 6 -> 3 -> 5 -> 1 ->{2 -> {3, 4, 6}}, 5 -> 2};
\end{codeexample}
Since nothing is for free, the compact layout comes at the cost of an unwanted intersection of edges.
\end{key}

\paragraph{How Does This Algorithm Work?}
The algorithm that is going to be described now provides a standard approach for generating a layout for a generic graph.
It uses the principle of local search to find a layout of the nodes where primarily the number of intersections of paths is minimized.
\par To realize the local search algorithm for the node positioning problem an initial state has to be defined. A state is defined by the arrangement of all nodes on a grid. This grid consists of rows and columns of the same size that is set to twice the number of nodes in the graph by default, but can be changed by the |max height| and |max width| keys. The height of the rows is determined by the heighest node and the width of the columns by widest.
\par In the beginning the nodes are positioned in the middle of the grid in shape of a rectangle. As the next step one node has to change its position on the grid and the resulting arrangement has to be evaluated. The evaluation of a state is done by a cost function which counts the number of path intersections in the current arrangement. Additional the average length of 33\% of the longest paths is added to the number of intersections, so that intersections and the path lengths will be minimized. An arrangement is optimal if there are no intersections and all nodes are positioned together. 
\par The algorithm checks all possible neighbors of the current state if there is a state with lower costs and picks the first neighbor with lower costs as the state for the next step. The algorithm determines if their is no neighbor with lower costs.
\par To implement the local search, the following components are required:
\begin{itemize}
\item a representation of the |graph|-object as an arrangement on a grid
\item a cost function, that represents all aspects which have to be minimized. In this implementation that are primarily the intersections and the length of the longest paths.
\item a neighbor function, which provides all grid arrangements resulting from the movement of one node
\end{itemize}
Important for the final layout is at first the initial arrangement of the nodes that offers an advantageous starting point for the improvement and secondly how to reach a quiet aesthetic and compact representation. The initial arrangement is done by grouping the nodes in the middle of the grid.
\par The aesthetics claim is quiet difficult to realize by an algorithm. An simple approach is to minimize the longest paths of the layout.

The aesthetics claim is quiet difficult to realize by an algorithm. An simple approach is to minimize the longest paths of the layout.
\end{gdalgorithm}

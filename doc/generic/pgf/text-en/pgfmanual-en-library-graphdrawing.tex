% Copyright 2010 by Renée Ahrens, Olof Frahm, Jens Kluttig, Matthias Schulz, Stephan Schuster
%
% This file may be distributed and/or modified
%
% 1. under the LaTeX Project Public License and/or
% 2. under the GNU Free Documentation License.
%
% See the file doc/generic/pgf/licenses/LICENSE for more details.

\section{Automated graph drawing with \tikzname}
{\noindent {\emph{by Ren\'{e}e Ahrens, Olof-Joachim Frahm,
Jens Kluttig, Matthias Schulz and Stephan Schuster}}} 
\label{section-library-graphdrawing}

\begin{tikzlibrary}{graphdrawing}
This package provides capabilities for automatic layouting of graphs.
\end{tikzlibrary}

\subsection{Overview}
Drawing a huge graph can be quite labor-intensive. This package aims to reduce
this amount of work for cases that do not require a certain layout of the graph.
It provides a set of algorithms trying to create a readable layout for a given
graph that is specified by its nodes and edges, but not by any coordinates.
These algorithms are implemented in Lua, therefore \LuaTeX\ is required
to use this package.  

The graph drawing framework employs the |graph| library (refer to
\ref{section-library-graphs}) for defining the nodes and edges by
default. The following example produces a simple tree.

\begin{codeexample}[]
\tikzpicture[graph drawing=standard tree]
  \graph { a [root] ->{b -> {c,d}, e}};
\endtikzpicture
\end{codeexample}

The |graph| library provides a very intuitive approach to specify which nodes and edges are present, but its main purpose is not easing the layouting of a graph. The graph drawing library aims to support the task of visualizing a graph by providing a set of algorithms for automated placement of the nodes. 
% TODO: some more details

\subsection{Requirements}
Since automatic graph drawing can be a quite complex thing this package does not
rely on the limited computational capabilities of \TeX, but requires \LuaTeX, a
new version of pdf\TeX\ including the small, but powerful scripting language Lua.

\LuaTeX\ has not yet reached a stable version at the time these lines are written
(spring 2011). But this package should work with \LuaTeX\ 0.4 or higher. By now
the latest releases of the major \TeX\ distributions all include \LuaTeX.

\subsection{The algorithms}
By now two layouting strategies are provided by the library that will be explained in detail now. 
The first paragraph on each algorithm will tell you what you need to now as a normal user. The rest covers details you might be interested in if you were going to implement own layouting strategies.
\subsubsection{Drawing Trees}
The tree algorithm positions the nodes of a tree. Therefore you have to ensure that the given graph is in fact a tree. Otherwise an error is raised. By default the root of the tree is positioned horizontally centered at the top. Child nodes are positioned beside each other according to their level. 

The tree algorithm can be found in the file |pgflibrarygraphdrawing-algorithms-arbitrarytree.lua| is part of the |pgf.graphdrawing| Lua module.

\paragraph{Usage} 

To use this algorithm you have to use the \tikzname\ key |graph drawing|. By setting its value to |standard tree| the tree algorithm layouts the graph.
\begin{codeexample}[]
\tikzpicture[graph drawing=standard tree]
  \graph { 1 [root] ->{2 , 3}};
\endtikzpicture
\end{codeexample}
\paragraph{Parameters} 
The keys affecting the algorithm are |root| and |tree scale|. The key |root|, which determines the root of the given tree, is required. Otherwise the algorithm cannott work. |tree scale| is optional and affects the horizontal and vertical space between the nodes. By default its value is 1. The following example shows the usage of |tree scale| with the values 1 and 3.
\begin{codeexample}[]
\tikzpicture[graph drawing={standard tree, tree scale=1}]
  \graph { 1 [root] ->{2 , 3}};
\endtikzpicture
\tikzpicture[graph drawing={standard tree, tree scale=3}]
  \graph { 1 [root] ->{2 , 3}};
\endtikzpicture
\end{codeexample}
Note that it is not possible to change |tree scale| within a |tikzpicture| environment.

\paragraph{How does this algorithm work?}
The tree algorithm works recursively. During the recursion one step is performed for each subgraph of the tree. 

The process builds a kind of a box structure of the given graph. This means a leaf of a tree returns itself as a box. Its parent returns itself and its children in a bigger box etc. as shown in the following figure.

\begin{quote}
\begin{tikzpicture}[
    level 0/.style={draw=black!50,very thick},
    level 1/.style={draw=orange!50,very thick},
    level 2/.style={draw=blue!50,very thick},
    level 3/.style={draw=green!50,very thick}]

    \node[level 1] (1) {1}
      child {node[level 2] (3) {3}
        child {node[level 3] (4) {4}
          child{node[level 3] (6) {6}}
          child{node[level 3] (7) {7}}
        }
        child {node[level 2] (5) {5}}
      }
      child {node[level 1] (2) {2}};

    \begin{pgfonlayer}{background}
        \node [level 0, fit=(1) (6) (2)] {};
        \node [level 1, fit=(3) (5) (6) (7)] {};
        \node [level 2,fit=(4) (6) (7)] {};
    \end{pgfonlayer}
\end{tikzpicture}
\end{quote}

In each step the current boxes can be compared by their size, sorted and positioned. In the figure above the boxes of one step are represented in the same color.

In the tree algorithm the boxes of each tree level are first sorted ascendingly by their size and then arranged as follows: The the biggest box is positioned in the middle. Then the following boxes are positioned alternately left and right.

After this arrangement the relative coordinates for the position of each box have to be computed. The \emph{y}-coordinate of a box (except for the root node of the step) are determined by the maximum height of all boxes to guarantee a uniform layout of the tree. Nodes on the same level in the tree are positioned at the same height. The \emph{x}-coordinate of a box depends on the coordinates of its left neighbour box and an additional spacing (by default 10pt), which can be influenced by the |tree scale| key. The \emph{y}-coordinate of the root node of each step is set to the maximum \emph{y}-value of the other boxes adding the same spacing meantioned above (by default 10pt, influenced by |tree scale|). Its \emph{x}-coordinate is determined by the width of the other boxes divided by 2. This means the root node is positioned in the middle above the other boxes.

Because each box knows its root node, it is possible to determine the absolute position of each box or node afterwards. 

At the end of the step the current boxes are added to a result box and returned.

\subsubsection{Drawing arbitrary graphs with few intersections}
The algorithm that is going to be described now provides a standard approach for generating a layout for a generic graph.
The principle is based on the local search algorithm. At the beginning, all nodes are placed in the middle of a grid.
Afterwards the algorithm tries to rearrange the position of the nodes, so that primarily the intersections of edges will be minimized.

\paragraph{Usage}
To use this algorithm, you have to set the |graph drawing| key to |few intersections|. As an Input any graph is feasible. Note that for complex graphs the time to retrieve an layout can be long.

\begin{codeexample}[]
\tikzpicture [graph drawing={few intersections}, scale=2]
  \graph { 6 -> 3 -> 5 -> 1 ->{2 -> {3, 4, 6}}, 5 -> 2};
\endtikzpicture
\end{codeexample}

\paragraph{Parameters}
The resulting layout of this algorithm can occupy lots of space. Therefore you can limit its height and width.
This can be done using the two keys |max height| and |max width|, which are measured in |pt|. In the following example, you see the same graph again with limited height.
\begin{codeexample}[]
\tikzpicture [graph drawing={few intersections, max height=50}, scale=2]
  \graph { 6 -> 3 -> 5 -> 1 ->{2 -> {3, 4, 6}}, 5 -> 2};
\endtikzpicture
\end{codeexample}
As you might have noticed the limited height increases the width compared to the previous example.
If you desire a compact layout or your space is limited the two parameters can be combined.
\begin{codeexample}[]
\tikzpicture [graph drawing={few intersections, max width=50, max height=50},
              scale=2]
  \graph { 6 -> 3 -> 5 -> 1 ->{2 -> {3, 4, 6}}, 5 -> 2};
\endtikzpicture
\end{codeexample}
Since nothing is for free, the compact layout comes at the cost of an unwanted intersection of edges.

\paragraph{How does this algorithm work?}
The algorithm uses the principle of local search to find a layout of the nodes where primarily the number of intersections of paths is minimized.\\ To realize the local search algorithm for the node positioning problem an initial state has to be defined. A state is defined by the arrangement of all nodes on a grid. This grid consists of rows and columns of the same size that is set to twice the number of nodes in the graph by default, but can be changed by the |max height| and |max width| keys. The height of the rows is determined by the heighest node and the width of the columns by widest.
\\ In the beginning the nodes are positioned in the middle of the grid in shape of a rectangle. As the next step one node has to change its position on the grid and the resulting arrangement has to be evaluated. The evaluation of a state is done by a cost function which counts the number of path intersections in the current arrangement. Additional the average length of 33\% of the longest paths is added to the number of intersections, so that intersections and the path lengths will be minimized. An arrangement is optimal if there are no intersections and all nodes are positioned together. 
\\ The algorithm checks all possible neighbors of the current state if there is a state with lower costs and picks the first neighbor with lower costs as the state for the next step. The algorithm determines if their is no neighbor with lower costs.
\\ To implement the local search, the following components are required:
\begin{itemize}
\item a representation of the |graph|-object as an arrangement on a grid
\item a cost function, that represents all aspects which have to be minimized. In this implementation that are primarily the intersections and the length of the longest paths.
\item a neighbor function, which provides all grid arrangements resulting from the movement of one node
\end{itemize}
Important for the final layout is at first the initial arrangement of the nodes that offers an advantageous starting point for the improvement and secondly how to reach a quiet aesthetic and compact representation. The initial arrangement is done by grouping the nodes in the middle of the grid.
\\ The aesthetics claim is quiet difficult to realize by an algorithm. An simple approach is to minimize the longest paths of the layout.

\subsection{Under the hood}
As mentioned before, the graph drawing library makes use of Lua. But
where does the control flow leave \TeX\ and what happens to your
\tikzname\ nodes? The subsequent sections will discuss this process in
deep. The general approach is to intercept the immediate placement of
the nodes and pass them down to Lua, which does all the placement
stuff. After the selected graph drawing algorithm has finished, it writes
the nodes back to \tikzname\ to have the graph drawn. 

This proceeding consists of a front end layer for \tikzname, an
interface to Lua and of course a set of Lua classes to represent the
graph. An algorithms can be developed independently. Only knowledge about the Lua interface is required; specific \TeX\ programming skills not necessary.

\subsubsection{The front end layer}
Let's have a look at a simple example to see what the front end looks
like:

\begin{codeexample}[]
\tikzpicture[graph drawing={standard tree,tree scale=0.5},scale=2]
  \graph{root [as=Hello,root] -> World[fill=blue!20]};
\endtikzpicture
\end{codeexample}

As you may see, the syntax is exactly the same as described in the
chapter about specifying graphs (section
\ref{section-library-graphs}).

You enable this library with the key |graph drawing|, which sets the
algorithm to use and its specific parameters. All other
\tikzname\ keys are accepted as well, like |scale| in the example
above. Each algorithm has its own keys to parametrize it. Please refer to
the appropriate sections for more information.

The keys are given within the |graph drawing| key family for graph options and per node for node specific options. Furthermore you can use any valid \tikzname\ keys as usual. 

There are some things which will not work with the graph drawing
library, like preordering the nodes. Consider for example the
|chain shift| key of the graphs library to place the nodes on a
certain grid: 

\begin{codeexample}[]
\tikzpicture
  \graph[chain shift=(45:1)] {
    a -> b -> c;
    d -> e;
  };
\endtikzpicture
\end{codeexample}

The graph drawing library does not take care of any predefined layout options by now, so the above example will be set differently:

\begin{codeexample}[]
\tikzpicture[graph drawing={few intersections}, scale=2]
  \graph[chain shift=(45:1)] {
    a -> b -> c;
    d -> e;
  };
\endtikzpicture
\end{codeexample}

A graph drawing algorithm will always place the nodes in its own manner. 

% what is happening in the tikz..tex file. Matthias

\subsubsection{The interface to Lua}
The main entry point for the library to Lua is defined in the
appropriate |code| file of the library. It employs three Lua classes
to create graphs, pass down nodes and to communicate the given
options.

An overview of what happens is illustrated by the following call graph:

\begin{tikzpicture}[
    class name/.style={draw,minimum size=20pt, fill=blue!20},
    object node/.style={draw,minimum size=15pt, fill=yellow!20},
    p/.style={->,>=stealth},
    livespan/.style={thick,double},
    scale=0.9]
  % class names above
  \node (tikz) at (0,4) [class name] {\tikzname\ graph};
  \node (tex) at (5,4) [class name] {\TeX\ Interface};
  \node (interface) at (10,4) [class name] {Lua Interface};
  \node (sys) at (15,4) [class name] {Sys};
  % lines from the class names to the bottom of the picture
  \draw[livespan] (tikz) -- (0,-6.5);
  \draw[livespan] (tex) -- (5,-6.5);
  \draw[livespan] (interface) -- (10,-6.5);
  \draw[livespan] (sys) -- (15,-6.5);
  % first command: \graph{  -- generates new graph in lua interface
  \node (tikz-begin-graph) at (0,3) [object node] {|\graph{|}; %}
  \node (tex-begin-graph) at (5,3) [object node] {|\pgfgdbeginscope|};  
  \node (interface-new-graph) at (10,3) [object node] {|newGraph(|...|)|};
  \draw [p] (tikz-begin-graph.east) -- (tex-begin-graph.west);
  \draw [p] (tex-begin-graph.east) -- (interface-new-graph.west);    
  % second command: a -> b   -- generates two nodes in lua
  % and one edge
  \node (tikz-node) at (0,2) [object node] {|a -> b;|};
  \node (tex-node) at (5,2) [object node] {|\pgf@gd@positionnode@callback|};
  \node (interface-add-node-behind) at (10.1,1.9) [object node] {|addNode(|...|)|};
  \draw[p] (tikz-node.east) -- (tex-node.west);
  
  \node (interface-add-node) at (10,2) [object node] {|addNode(|...|)|};
  \draw[p] (tex-node.east) -- (interface-add-node.west);

  \node (tex-add-edge) at (5,1) [object node] {|\pgfgdaddedge|};
  \node (interface-add-edge) at (10,1) [object node] {|addEdge(|...|)|};
  \draw[p] (tikz-node.east) -- (1.5,2) -- (1.5,1) -- (tex-add-edge.west);
  \draw[p] (tex-add-edge.east) -- (interface-add-edge.west);

  % scope ends -- cloes graph, layouts it and draws it
  \node (tikz-end) at (0,0) [object node] {|};|};
  \node (tex-end) at (5,0) [object node] {|\pgfgdendscope|};
  \node (interface-draw-graph) at (10,0) [object node] {|drawGraph()|};
  \node (interface-finish-graph) at (10,-2) [object node] {|finishGraph()|};

  \node (invoke-algorithm) at (12.5,-1) [object node] {invoke algorithm};
  \draw[p] (tikz-end.east) -- (tex-end.west);
  \draw[p] (tex-end.east) -- (interface-draw-graph.west);
  \draw[p] (interface-draw-graph.east) -- (12.5,0) -- (invoke-algorithm.north);
  \draw[p] (tex-end.east) -- (7.5,0) -- (7.5,-2) -- (interface-finish-graph.west);

  % begin shipout
  \node (sys-begin-shipout) at (15,-2) [object node] {|beginShipout()|};
  \draw[p] (interface-finish-graph.east) -- (sys-begin-shipout.west);
  \node (tex-begin-shipout) at (5,-3) [object node] {|\pgfgdbeginshipout|};
  \draw[p] (sys-begin-shipout.187) -- (12,-2.2) -- (12,-3) -- (tex-begin-shipout.east);

  % put tex box
  \node (sys-puttexbox-behind) at (15.1,-4.1) [object node] {|putTeXBox(|...|)|};
  \node (sys-puttexbox) at (15,-4) [object node] {|putTeXBox(|...|)|};
  \node (tex-puttexbox) at (5,-4) [object node] {|\pgfgdinternalshipoutnode|};

  \draw[p] (12.5,-2) -- (12.5,-4) -- (sys-puttexbox.west);
  %(interface-finish-graph.east) -- (12.5,-2) -- (12.5,-4) -- (sys-puttexbox.west);
  \draw[p] (sys-puttexbox.187) -- (12,-4.2) -- (12,-4) -- (tex-puttexbox.east);

  % put edge
  \node (sys-put-edge-behind) at (15.1,-5.1) [object node] {|putEdge(|...|)|};
  \node (sys-put-edge) at (15,-5) [object node] {|putEdge(|...|)|};
  \draw[p] (12.5,-4) -- (12.5,-5) -- (sys-put-edge.west);
  %(interface-finish-graph.east) -- (12.5,-2) -- (12.5,-5) -- (sys-put-edge.west);
  % end shipout
  \node (sys-end-shipout) at (15,-6) [object node] {|endShipout()|};
  \draw[p] (12.5,-5) -- (12.5,-6) -- (sys-end-shipout.west);
  %(interface-finish-graph.east) -- (12.5,-2) -- (12.5,-6) -- (sys-end-shipout.west);
  \node (tex-end-shipout) at (5,-6) [object node] {|\pgfgdendshipout|};
  \draw[p] (sys-end-shipout.187) -- (12,-6.175) -- (12,-6) -- (tex-end-shipout.east);
\end{tikzpicture}


\paragraph{The \TeX\ side}
\label{section-library-graphdrawing-the-tex-side}

In order to layout a graph, we need to keep \tikzname\ from placing the nodes immediately. This is done using the macro
|\pgfpositionnodelater| as described in chapter \ref{section-shapes},
subchapter \ref{section-shapes-deferred-node-positioning}. 

In short terms this works as follows: This macro takes another \meta{macro} as
first argument. If this is |\relax|, the behaviour is to immediately
place the node into the current picture. Any other \meta{macro} that is passed
will be executed. It works like a
callback function -- the node will be put into a box register, the
name of the node and the bounding box coordinates are stored in
separate macros and afterwars \meta{macro} will be called.

As we have to make sure, that the unplaced node will not be referenced by \tikzname\ keys like |right of|, it is temporarly renamed to |not yet positioned@|\meta{nodename}.

To finally
insert the node into the picture, we need to set the mentioned macros
and put the node into the box register. Then we can call
|\pgfpositionnodenow| with the target coordinates of the node.

The code file of the graph drawing library sets the callback function
at the beginning of a graph drawing scope, e.g.\ when a |\graph|
starts. This can also be triggered using |\pgfgdbeginscope| and
|\pgfgdendscope|, which can be used to create a sub scope in an
existing graph drawing scope. Opening a scope yields in creating a new
graph on the Lua graph stack. All subsequent operations (like adding
nodes or edges) apply to the top of the stack. 

%\begin{codeexample}[]
%\tikzpicture[graph drawing={few intersections}, scale=2]
%\graph{
%  a->b;
%  \graph{c->d;}; TODO: triggers an infinite loop.
%  };
%\endtikzpicture
%\end{codeexample}

The callback function gets all option keys in
|/tikz/graphs/graph drawing/|, copies the box register and passes all information down to the Lua interface class.

When the library is loaded, it initialises the Lua subsystem. This takes place by checking if \LuaTeX\ is present and then invoking the Lua loader class. 

The library code file consists mainly the following macros:

\begin{command}{\pgfgdbeginscope}
  The begin scope macro opens a new graph drawing scope. This creates a new graph object on the top of the Lua graph stack. All subsequent operations will work on this graph until |\pgfgdendscope| will be called.

It is not necessary to call it manually, because in a graph drawing environment it is executed by default at the beginning of a |\graph| statement.
\end{command}


\makeatletter
\begin{command}{\pgf@gd@positionnode@callback}
  This macro saves the keys from |/tikz/graphs/graph drawing/| into a temporary macro, sets the box register |\pgf@gd@box| to the |\pgfpositionnodelaterbox| and passes these informations down to Lua. Additionally the node name and the bounding box is passed down, too. This macro is only used internally.
\end{command}
\makeatother

\begin{command}{\pgfgdaddedge\marg{from}\marg{to}\marg{direction}}
  Adds an edge to the Lua graph object. It requires the name of the target node \meta{from}, the destination node \meta{to} with a distinct \meta{direction} like |->|.

  It is called when a |->|, |--|, |<-| or |-!-| is encountered in a graph.
\end{command}

\begin{command}{\pgfgdendscope}
  At the end of a graph drawing scope the selected algorithm runs and layouts the graph. After finishing this task the macro pops the graph from the stack.
\end{command}

\begin{command}{\pgfgdbeginshipout}
  When the layout is completed and the scope ended, this macro places a |\scope| into the output stream. The layouted graph will be placed inside an extra scope.
\end{command}

\begin{command}{\pgfgdinternalshipoutnode\marg{name}\marg{x min}\marg{x max}\marg{y min}\marg{y max}\marg{x pos}\marg{y pos}\marg{box}}
  When the algorithm finished the layout and the scope ended, the nodes have to be passed back to \tikzname. This macro takes the name of the node, the bounding box, the newly computed position and a box register number. It restores the macros set by |\pgfpositionnodelater| as mentioned above, fills the box register |\pgfpositionnodelaterbox| and then calls |\pgfpositionnodenow| with the coordinates of the node. This macro inserts the node into the current picture.
\end{command}

\begin{command}{\pgfgdendshipout}
  Issues a |\endscope| macro to close the scope opened by |\pgfgdbeginshipout|.
\end{command}

\paragraph{Lua interface class}

The class |Interface| is the main entry point in Lua. Every communication from \TeX\ to Lua is done here.
It provides methods to create graphs, add nodes and edges to graphs and finally to invoke the selected algorithm. The |Interface| class manages the stack of graphs.

When the |newGraph()| function is called, it generates a new graph object and pushes it on the graph stack. The methods |addNode()| and |addEdge()| are called for each node and each edge, creating the actual Lua objects and adding them to the current graph.

After adding nodes and edges, when the scope ends, the interface invokes the actual algorithm to layout the graph. This is done in the |drawGraph()| function. The next step is to put the nodes back in the \TeX\ output stream. This is invoked by the |finishGraph()| method.

For a reference about the functions and their usage, please refer to section~\ref{section-library-graphdrawing-lua-documentation-interface}.

\paragraph{Lua system class}

Communication with \TeX\ on a basic layer is done in the |Sys| class. The |beginShipout()| function opens a new scope in \tikzname\ to put all graph drawing nodes into. This prevents other graph objects outside the graph drawing scope from referencing these nodes. The |endShipout()| method closes the scope.

Nodes and edges are put in the output stream by the methods |putTeXBox()| and |putEdge()|. The first calls the |\pgfgdinternalshipoutnode| macro, which is explained in section \ref{section-library-graphdrawing-the-tex-side}. The latter method writes the appropriate |\draw| directly to the output stream. 

For a reference about the functions and their usage, please refer to section~\ref{section-library-graphdrawing-lua-documentation-sys}.

\subsubsection{Lua graph representation}
Most classes in the framework implement the |__tostring()| method,
meaning that you can get a somewhat useful string representation of
the object via the standard |tostring()| function.

The main class which contains references to all other objects is
|Graph|.  New graphs are usually created automatically, so common ways
to get new graph objects are the |copy()| method, which creates a
shallow copy (without coying nodes or edges), and the
|subGraphParent()| method, which creates a deep copy of the graph, edge
and node objects starting at a designated parent node. If you need
more control by supplying your own set of already visited nodes, use
the underlying function |subGraph()|.

A graph allows you to add and remove nodes and edges via |addNode()|,
|addEdge()|, |removeNode()| and |removeEdge()| respectively.  There are also
variants which remove all incident edges on a node removal and
conversely, |deleteNode()| and |deleteEdge()|.

Only nodes can be looked up by name with |findNode()|, a
method implemented using the more generic |findNodeIf()|, which supports
an arbitrary test predicate.

Lastly the |walkDepth()| and |walkBreadth()| methods may be used to get
iterators over all nodes and edges in a depth-first or breadth-first
order (other traversal orders may require a rewrite or extension of the
|walkAux()| method).

Positions are represented using the dedicated class |Position|, the member
variables |x| and |y| contain the coordinates.  Positions can also be
relative to other positions, which can be tested using |isAbsPosition()|.
The conversion to absolute coordinates is done with |getAbsCoordinates()|.
The equality test method implements comparing two positions by using their
absolute positions.

For a detailed description of the mentioned classes and methods refer
to section~\ref{section-library-graphdrawing-lua-documentation-graphrep}.

\paragraph{Common graph operations}
The following tasks are typical for manipulating the graph.
Those snippets will get you started even if you do not have any Lua
experience.

\begin{itemize}
\item Iterate over all nodes.
\begin{codeexample}[code only]
for node in values(graph.nodes) do
   ...
end
\end{codeexample}
\item Get or set width/height of a node, e.g.\ for measuring.
\begin{codeexample}[code only]
local width, height = node.width, node.height
\end{codeexample}
\item Get or set x-y-coordinates of a node.
\begin{codeexample}[code only]
node.pos.x = node.pos.x + 1
node.pos.y = node.pos.y + 1
\end{codeexample}
\item Relate the position of node to the position of another.
\begin{codeexample}[code only]
newNode.pos.x, newNode.pos.y = 1, 1
--sets position of newNode 1 pt in y- and x-direction relative to node
newNode.pos:relateTo(node.pos)
\end{codeexample}
\item Get absolute x-y-coordinate of node, with or without relative coordinates.
\begin{codeexample}[code only]
absX, absY = newNode:getAbsCoordinates()
\end{codeexample}
\item Iterate over all edges and all nodes of the current edge.
\begin{codeexample}[code only]
for edge in values(graph.edges) do
   for node in values(edge:getNodes()) do
      ...
   end
end
\end{codeexample}
\item Get the nodes connected by an edge.
\begin{codeexample}[code only]
local nodeLeft = edge:getNodes()[1]
local nodeRight = edge:getNodes()[2]
\end{codeexample}
\end{itemize}

A full example for a user-defined algorithm is shown in section \ref{section-library-graphdrawing-ownAlgorithm}.

\subsection{Creating your own algorithm}
\label{section-library-graphdrawing-ownAlgorithm}
There are two ways to make a user-definded algorithm
available to the graph drawing library.
You can create your own graph drawing algorithm by naming it like
|drawGraphAlgorithm_xyz| and placing it into the |pgf.graphdrawing|
Lua module, where |xyz| is the string which is supplied to the
\TeX\ interface.  This way the function is looked up before the
framework tries to load a file named
|pgflibrarygraphdrawing-algorithms-xyz.lua| anywhere in the accessible
path, which is the second way to define your algorithm.

You may load a file named according to the above-mentioned scheme that
contains an algorithm on your own using the |Interface:loadAlgorithm()|
function, which
accepts the name string as single argument. This will usually modify
some module entry, so while e.g. defining wrappers around other
algorithms you have to take care of that behavior.
% TODO: I do not understand, what you are trying to say. - Stephan

The algorithm will be called with the graph object as single argument
and should do its work by modifying this object. Any return
values are discarded.

For example, the following code fragment (taken and slightly altered
from the file\\ |pgflibrarygraphdrawing-algorithms-simpleexample.lua|)
implements a rather simple algorithm, placing all nodes on a fixed-size
circle.  It is accessed with the name |simpleexample|, so both the
file- and function name agree on that.

\begin{codeexample}[code only]
pgf.module("pgf.graphdrawing")

--- A very, very simple node placing algorithm for demonstration purposes.
-- All nodes are positioned on a fixed-size circle.
function drawGraphAlgorithm_simpleexample(graph)
   local radius = 20
   local nodeCount = 0

   -- count nodes
   for _ in values(graph.nodes) do
      nodeCount = nodeCount + 1
   end

   local alpha = (2 * math.pi) / nodeCount
   local i = 0
   for node in values(graph.nodes) do
      -- the interesting part...
      node.pos.x = radius * math.cos(i * alpha)
      node.pos.y = radius * math.sin(i * alpha)
      i = i + 1
   end
end
\end{codeexample}

It is important not to use a |local| declaration before the function
header, because it wouldn't be available in the |pgf.graphdrawing|
module anymore.

The algorithm computes a circular layout like in the following.%
\footnote{Actually you don't even need Lua to compute such a layout by
  using the |clockwise|~\ref{key-graphs-clockwise} or
  |counterclockwise|~\ref{key-graphs-counterclockwise} options.}

\begin{codeexample}[]
\tikzpicture [graphs/.cd, graph drawing engine, algorithm=simpleexample]
  \graph { f -> c -> e -> a ->{b -> {c, d, f}, e -> b}};
\endtikzpicture
\end{codeexample}

The invocation above also shows how to use an algorithm which is not
registered as a \tikzname\ key.  In general, you will probably want to
register your algorithm with |\tikzgraphsset| to make your code more
succinct, but also to be able to change algorithm options by manipulating
\tikzname\ keys, which is not possible without registration.

To do so, we have to modify the first line of the example algorithm.

\begin{codeexample}[code only]
   local radius = graph:getOption("radius") or 20
\end{codeexample}

Using the |getOption| method we obtain the value of the
\tikzname\ option or a |nil| value, therefore there has to be a
default value for any option or more elaborate error handling.  Now,
the following code block can be used to register this algorithm and
its single option.

\begin{codeexample}[code only]
\tikzgraphsset{
  simpleexample/.style={
    graph drawing engine,
    algorithm=simpleexample
  },
  graph drawing/register math key=radius
}
\end{codeexample}

\tikzgraphsset{
  simpleexample/.style={
    graph drawing engine,
    algorithm=simpleexample
  },
  graph drawing/register math key=radius
}

Eventually this fragment will have to be entered into the
|tikzlibrarygraphdrawing.code.tex| file if it is to be included in the
\pgfname\ source code.

Once registered, specifying the algorithm gets a bit easier. Note the
increases radius compared to the previous example.

\begin{codeexample}[]
\tikzpicture [graph drawing={simpleexample, radius = 30}]
  \graph { f -> c -> e -> a ->{b -> {c, d, f}, e -> b}};
\endtikzpicture
\end{codeexample}

\subsection{Module system}
The package defines its own Lua module system, which is characterised by a
more dynamic view on importing symbols.  Basically, each module has a
set of imported modules and the lookup for names first happens in the local
scope, then in the current module and subsequently in all imported
modules.  Since no name is statically imported, newly assigned
variables in other modules are still visible when those were
previously imported.

Modules are accessed with the |pgf.module()| call, which enables the
module for the current context, i.e. the current file. If a module
does not exist, it will be created.  Importing modules is done via
|pgf.import()|.  Both functions accept a string argument for the
module name.

Modules are named hierarchically and defined modules are exported into
each parent module.  If the module name contains no period, it is
exported into the global environment.  Nevertheless, importing is only
done on request; importing a module twice doesn't do anything.
It is recommended to dedicate a single module definition file
to create it and import other modules.  For example, the package
contains a single file containing only the following two lines for
creating the |pgf.graphdrawing| module in the first place.

\begin{codeexample}[code only]
pgf.module("pgf.graphdrawing")
pgf.import("pgf")
\end{codeexample}

Symbol lookup first happens in the local namespace, then in the
current module and subsequently in all imported modules and the global
namespace.  Assignment of new variables happens in the current module
(or for variables declared |local| in the local namespace).  If you
need to assign values to the global environment use the special table
|_G| as you'd normally do in Lua.

The |pgf| module is created during the definition of the module system
and mostly contains functions for loading and debugging.  Developers
probably shouldn't touch the |pgf| namespace and instead add new
functionality to modules below this level or in new top-level
modules.

\subsubsection{Module examples}
Let's see what consequences this module system has in praxis.  The
following code fragment starts from a clean state after rendering it
with \LuaTeX\ and then enters the |pgf.graphdrawing| module,
overwriting the global |pgf| binding and then again reverting this
change.

\begin{codeexample}[code only]
  \input tikz

  \usetikzlibrary{graphdrawing}

  \directlua{
    pgf.graphdrawing.Sys:logMessage("1: pgf is " .. tostring(pgf))
    pgf.graphdrawing.Sys:logMessage("1: graphdrawing is " .. tostring(graphdrawing))
    
    pgf.module("pgf.graphdrawing")

    Sys:logMessage("2: pgf is " .. tostring(pgf))
    Sys:logMessage("2: graphdrawing is " .. tostring(graphdrawing))

    pgf = 1

    Sys:logMessage("3: pgf is " .. tostring(pgf))
    Sys:logMessage("3: graphdrawing is " .. tostring(graphdrawing))

    pgf = nil

    Sys:logMessage("4: pgf is " .. tostring(pgf))

    pgf.graphdrawing = nil

    Sys:logMessage("5: pgf is " .. tostring(pgf))

    _G.pgf = nil

    Sys:logMessage("6: pgf is " .. tostring(pgf))
  }
\end{codeexample}

The result will be as follows:

\begin{codeexample}[code only]
1: pgf is <module 'pgf', table: 0x7979600>
1: graphdrawing is nil

2: pgf is <module 'pgf', table: 0x7979600>
2: graphdrawing is <module 'pgf.graphdrawing', table: 0x7973c60>

3: pgf is 1
3: graphdrawing is <module 'pgf.graphdrawing', table: 0x7973c60>

4: pgf is <module 'pgf', table: 0x7979600>
5: pgf is <module 'pgf', table: 0x7979600>
6: pgf is nil
\end{codeexample}

As you can see the |pgf| table is available in the global environment
and also after using the |pgf.graphdrawing| module, although we don't
refer to it with its full name.  Assigning a new value to |pgf|
doesn't overwrite the global object, but introduces a local binding
shadowing the global one. Assigning |nil| then removes the local
binding, therefore in the next line the global variable is available
again.

Note that in all but the first case the binding to |graphdrawing|
stays the same.  Also, using these assignments, you can't accidentally
remove your access to the |pgf| or any imported modules as the last
two assignments show (the |Sys:logMessage| method still works).

\subsection{Lua documentation}
This sections provides a full documentation of all relevant Lua classes
used for graph drawing.
\label{section-library-graphdrawing-lua-documentation}
\subsubsection{Graph representation}
\label{section-library-graphdrawing-lua-documentation-graphrep}
% This file has been generated from the lua sources using LuaDoc.
% To regenerate it call "make genluadoc" in
% doc/generic/pgf/version-for-luatex/en.

\paragraph{pgflibrarygraphdrawing-graph.lua}


\begin{luacommand}{{Graph:\textunderscore{}\textunderscore{}tostring}()}
Returns a string representation of this graph including all nodes and edges.


Return value:
\begin{itemize} \item[] Graph as string. \end{itemize}


\end{luacommand}\begin{luacommand}{{Graph:addEdge}(\meta{edge})}
Adds an edge to the graph.

Parameters:
\begin{itemize}
	\item[] \meta{edge} \subitem The edge to be added.
\end{itemize}



\end{luacommand}\begin{luacommand}{{Graph:addNode}(\meta{node})}
Adds a node to the graph.

Parameters:
\begin{itemize}
	\item[] \meta{node} \subitem The node to be added.
\end{itemize}



\end{luacommand}\begin{luacommand}{{Graph:copy}()}
Creates a shallow copy of a graph. That is, without nodes or edges.


Return value:
\begin{itemize} \item[] A copy of the graph. \end{itemize}


\end{luacommand}\begin{luacommand}{{Graph:createEdge}(\meta{nodeA},\meta{nodeB},\meta{direction},\meta{options})}
Creates and adds a new edge to the graph. The edge contains the given nodes and its direction and options are set to the param direction/option.

Parameters:
\begin{itemize}
	\item[] \meta{nodeA} \subitem The first node of the new edge.\item[] \meta{nodeB} \subitem The second node of the new edge.\item[] \meta{direction} \subitem The direction of the new edge.\item[] \meta{options} \subitem The options of the new edge.
\end{itemize}


Return value:
\begin{itemize} \item[] The newly created edge. \end{itemize}


\end{luacommand}\begin{luacommand}{{Graph:deleteEdge}(\meta{edge})}
Like removeEdge, but also removes the edge from the nodes incident with it.

Parameters:
\begin{itemize}
	\item[] \meta{edge} \subitem The edge to be removed.
\end{itemize}


Return value:
\begin{itemize} \item[] The edge or nil. \end{itemize}


\end{luacommand}\begin{luacommand}{{Graph:deleteNode}(\meta{node})}
Like removeNode, but also removes all edges incident to the removed node and for all nodes incident to the removed edges, remove the edges from them, too.

Parameters:
\begin{itemize}
	\item[] \meta{node} \subitem The node to be deleted with its edges.
\end{itemize}


Return value:
\begin{itemize} \item[] The node or nil if the node wasn't contained in the graph. \end{itemize}


\end{luacommand}\begin{luacommand}{{Graph:findNode}(\meta{name})}
Searches the nodes of the graph by the given name.

Parameters:
\begin{itemize}
	\item[] \meta{name} \subitem Name of the node you're looking for.
\end{itemize}


Return value:
\begin{itemize} \item[] The node with the given name or nil if it wasn't contained in the graph. \end{itemize}


\end{luacommand}\begin{luacommand}{{Graph:findNodeIf}(\meta{test})}
Searches the nodes of the graph by the given test-function and returns the first matching node.

Parameters:
\begin{itemize}
	\item[] \meta{test} \subitem A function (with a parameter of node) returning a boolean value.
\end{itemize}


Return value:
\begin{itemize} \item[] The matching node or nil. \end{itemize}


\end{luacommand}\begin{luacommand}{{Graph:getOption}(\meta{name})}
Returns the value of the option defined by name.

Parameters:
\begin{itemize}
	\item[] \meta{name} \subitem Name of the option.
\end{itemize}


Return value:
\begin{itemize} \item[] The stored value of the option or nil. \end{itemize}


\end{luacommand}\begin{luacommand}{{Graph:mergeOptions}(\meta{options})}
Merges the given options into options of the graph.

Parameters:
\begin{itemize}
	\item[] \meta{options} \subitem The options to be merged.
\end{itemize}



See also:
\begin{itemize}
	\item[] |mergeTable|
\end{itemize}

\end{luacommand}\begin{luacommand}{{Graph:new}(\meta{values})}
Creates a new graph.

Parameters:
\begin{itemize}
	\item[] \meta{values} \subitem Values (e.g. options) to be merged with the default-metatable of a graph
\end{itemize}


Return value:
\begin{itemize} \item[] The new graph. \end{itemize}


\end{luacommand}\begin{luacommand}{{Graph:removeEdge}(\meta{edge})}
Removes an edge from the graph, if possible and returns it.

Parameters:
\begin{itemize}
	\item[] \meta{edge} \subitem The edge to be removed.
\end{itemize}


Return value:
\begin{itemize} \item[] The edge or nil. \end{itemize}


\end{luacommand}\begin{luacommand}{{Graph:removeNode}(\meta{node})}
Removes a node from the graph, if possible and returns it.

Parameters:
\begin{itemize}
	\item[] \meta{node} \subitem The node to remove.
\end{itemize}


Return value:
\begin{itemize} \item[] The node or nil if it wasn't contained in the graph. \end{itemize}


\end{luacommand}\begin{luacommand}{{Graph:setOption}(\meta{name},\meta{value})}
Sets the option name to value.

Parameters:
\begin{itemize}
	\item[] \meta{name} \subitem Name of the option to be set.\item[] \meta{value} \subitem Value for the option defined by name.
\end{itemize}



\end{luacommand}\begin{luacommand}{{Graph:subGraph}(\meta{root},\meta{graph},\meta{visited})}
The function returns a new subgraph. The result graph begins at the node root, excludes all nodes and edges which are marked as visited.

Parameters:
\begin{itemize}
	\item[] \meta{root} \subitem Root node where operation starts.\item[] \meta{graph} \subitem Result graph object or nil.\item[] \meta{visited} \subitem Set of already visited nodes/edges or nil; will be modified.
\end{itemize}



\end{luacommand}\begin{luacommand}{{Graph:subGraphParent}(\meta{root},\meta{parent},\meta{graph})}
Creates a new subgraph with the parent marked visited. Useful if the graph is a tree structure (and parent is the parent of root).

Parameters:
\begin{itemize}
	\item[] \meta{parent} \subitem Parent of the recursion step before.
\end{itemize}



See also:
\begin{itemize}
	\item[] |subGraph|
\end{itemize}

\end{luacommand}\begin{luacommand}{{Graph:walkAux}(\meta{root},\meta{visited},\meta{removeIndex})}
Auxiliary function to walk a graph. Does nothing if no nodes exist.

Parameters:
\begin{itemize}
	\item[] \meta{root} \subitem The first node to be visited.  If nil, chooses some node.\item[] \meta{visited} \subitem Set of already seen things (nodes and edges). |visited[v] == true| indicates that the object v was already seen.\item[] \meta{removeIndex} \subitem Is either nil or a numeric value where the objects are removed from the local queues (nil therefore designates queue behaviour, 1 a stack behaviour).
\end{itemize}



See also:
\begin{itemize}
	\item[] |walkDepth|\item[] |walkBreadth|
\end{itemize}

\end{luacommand}\begin{luacommand}{{Graph:walkBreadth}(\meta{root},\meta{visited})}
The function returns an iterator to walk the graph breadth-first. The iterator then returns all edges and nodes one at a time and once only.  Use a filter function to return only edges or nodes.



See also:
\begin{itemize}
	\item[] |iterator.filter|
\end{itemize}

\end{luacommand}\begin{luacommand}{{Graph:walkDepth}(\meta{root},\meta{visited})}
The function returns an iterator to walk the graph depth-first. The iterator then returns all edges and nodes one at a time and once only.  Use a filter function to return only edges or nodes.



See also:
\begin{itemize}
	\item[] |iterator.filter|
\end{itemize}

\end{luacommand}

% This file has been generated from the lua sources using LuaDoc.
% To regenerate it call "make genluadoc" in
% doc/generic/pgf/version-for-luatex/en.

\paragraph{pgflibrarygraphdrawing-node.lua}


\begin{luacommand}{{Node:\textunderscore{}\textunderscore{}eq}(\meta{object})}
Compares two nodes by name.

Parameters:
\begin{itemize}
	\item[] \meta{object} \subitem The node to be compared to self
\end{itemize}


Return value:
\begin{itemize} \item[] True if self is equal to object. \end{itemize}


\end{luacommand}\begin{luacommand}{{Node:\textunderscore{}\textunderscore{}tostring}()}
Returns a formatted string representation of the node.


Return value:
\begin{itemize} \item[] String representation of the node. \end{itemize}


\end{luacommand}\begin{luacommand}{{Node:addEdge}(\meta{edge})}
Adds new Edge to the Node.

Parameters:
\begin{itemize}
	\item[] \meta{edge} \subitem The edge to be added.
\end{itemize}



\end{luacommand}\begin{luacommand}{{Node:copy}()}
Creates a shallow copy of a node.


Return value:
\begin{itemize} \item[] Copy of the node. \end{itemize}


\end{luacommand}\begin{luacommand}{{Node:degree}()}
Computes the number of neighbour nodes.


Return value:
\begin{itemize} \item[] Number of neighbours. \end{itemize}


\end{luacommand}\begin{luacommand}{{Node:getEdges}()}
Gets all Edges of the node.


Return value:
\begin{itemize} \item[] The edges of the node as a table. \end{itemize}


\end{luacommand}\begin{luacommand}{{Node:getOption}(\meta{name})}
Returns the value of option name or nil.

Parameters:
\begin{itemize}
	\item[] \meta{name} \subitem Name of the option.
\end{itemize}


Return value:
\begin{itemize} \item[] The stored value of the option or nil. \end{itemize}


\end{luacommand}\begin{luacommand}{{Node:getTexHeight}()}
Computes the Heigth of the Node.


Return value:
\begin{itemize} \item[] Height of the Node. \end{itemize}


\end{luacommand}\begin{luacommand}{{Node:getTexWidth}()}
Computes the Width of the Node.


Return value:
\begin{itemize} \item[] Width of the Node. \end{itemize}


\end{luacommand}\begin{luacommand}{{Node:mergeOptions}(\meta{options})}
Merges options.

Parameters:
\begin{itemize}
	\item[] \meta{options} \subitem The options to be merged.
\end{itemize}



See also:
\begin{itemize}
	\item[] |mergeTable|
\end{itemize}

\end{luacommand}\begin{luacommand}{{Node:new}(\meta{values})}
Creates a new node.

Parameters:
\begin{itemize}
	\item[] \meta{values} \subitem Values (e.g. position) to be merged with the default-metatable of a node
\end{itemize}


Return value:
\begin{itemize} \item[] A newly allocated node object. \end{itemize}


\end{luacommand}\begin{luacommand}{{Node:removeEdge}(\meta{edge})}
Removes an edge from the node.

Parameters:
\begin{itemize}
	\item[] \meta{edge} \subitem The edge to remove.
\end{itemize}



\end{luacommand}\begin{luacommand}{{Node:setOption}(\meta{name},\meta{value})}
Sets the option name to value.

Parameters:
\begin{itemize}
	\item[] \meta{name} \subitem Name of the option to be set.\item[] \meta{value} \subitem Value for the option defined by name.
\end{itemize}



\end{luacommand}

% This file has been generated from the lua sources using LuaDoc.
% To regenerate it call "make genluadoc" in
% doc/generic/pgf/version-for-luatex/en.

\paragraph{pgflibrarygraphdrawing-edge.lua}


\begin{luacommand}{{Edge:\textunderscore{}\textunderscore{}tostring}()}
Returns a readable string representation of the edge.


Return value:
\begin{itemize} \item[] String representation of the edge. \end{itemize}


\end{luacommand}\begin{luacommand}{{Edge:addNode}(\meta{node})}
Adds node to the edge.

Parameters:
\begin{itemize}
	\item[] \meta{node} \subitem The node to be added to the edge.
\end{itemize}



\end{luacommand}\begin{luacommand}{{Edge:containsNode}(\meta{node})}
Tests if edge contains a node.


Return value:
\begin{itemize} \item[] True if the edge contains a node. \end{itemize}


\end{luacommand}\begin{luacommand}{{Edge:copy}()}
Copies an edge (preventing accidental use).


Return value:
\begin{itemize} \item[] Shallow copy of the edge. \end{itemize}


\end{luacommand}\begin{luacommand}{{Edge:getDegree}()}
Returns number of nodes on the edge.


Return value:
\begin{itemize} \item[] Number of nodes of the edge. \end{itemize}


\end{luacommand}\begin{luacommand}{{Edge:getNeighbour}(\meta{node})}
Gets first neighbour of the node (disregarding hyperedges).

Parameters:
\begin{itemize}
	\item[] \meta{node} \subitem The node which first neighbour should be returned.
\end{itemize}


Return value:
\begin{itemize} \item[] The first neighbour of the node. \end{itemize}


\end{luacommand}\begin{luacommand}{{Edge:getNeighbours}(\meta{node})}
Returns all neighbours of a node.

Parameters:
\begin{itemize}
	\item[] \meta{node} \subitem The node which neighbours should be returned.
\end{itemize}


Return value:
\begin{itemize} \item[] Array of neighbour nodes. \end{itemize}


\end{luacommand}\begin{luacommand}{{Edge:getNodes}()}
Returns the nodes of an edge.


Return value:
\begin{itemize} \item[] Array of nodes of the edge. \end{itemize}


\end{luacommand}\begin{luacommand}{{Edge:getPath}()}
Returns the path of an edge.


Return value:
\begin{itemize} \item[] The path the edge belongs to. \end{itemize}


\end{luacommand}\begin{luacommand}{{Edge:isHyperedge}()}
Returns a boolean whether the edge is a hyperedge.


Return value:
\begin{itemize} \item[] True if the edge is a hyperedge. \end{itemize}


\end{luacommand}\begin{luacommand}{{Edge:new}(\meta{values})}
Creates an edge between nodes of a graph.

Parameters:
\begin{itemize}
	\item[] \meta{values} \subitem Values (e.g. direction) to be merged with the default-metatable of an edge.
\end{itemize}


Return value:
\begin{itemize} \item[] The new edge. \end{itemize}


\end{luacommand}\begin{luacommand}{{Edge:setPath}(\meta{path})}
Sets the path of an edge.

Parameters:
\begin{itemize}
	\item[] \meta{path} \subitem The path the edge belongs to.
\end{itemize}



\end{luacommand}

% This file has been generated from the lua sources using LuaDoc.
% To regenerate it call "make genluadoc" in
% doc/generic/pgf/version-for-luatex/en.

\paragraph{pgflibrarygraphdrawing-position.lua}


\begin{luacommand}{{Position.calcCoordsTo}(\meta{posFrom},\meta{posTo})}
Returns a vector between two positions.

Parameters:
\begin{itemize}
	\item[] \meta{posFrom} \subitem Position A.\item[] \meta{posTo} \subitem Position B.
\end{itemize}


Return value:
\begin{itemize} \item[] x- and y-coordinates of the vector between posFrom and posTo. \end{itemize}


\end{luacommand}\begin{luacommand}{{Position:\textunderscore{}\textunderscore{}tostring}()}
Returns a readable string representation of the position.


Return value:
\begin{itemize} \item[] string representation of the position. \end{itemize}


\end{luacommand}\begin{luacommand}{{Position:copy}()}
Creates a copy of this position object.


Return value:
\begin{itemize} \item[] Copy of the position. \end{itemize}


\end{luacommand}\begin{luacommand}{{Position:equals}(\meta{pos})}
Returns a boolean value whether the object is equal to the given position.


Return value:
\begin{itemize} \item[] true if the position is equal to the given position pos. \end{itemize}


\end{luacommand}\begin{luacommand}{{Position:getAbsCoordinates}(\meta{x},\meta{y})}
Computes absolute coordinates of a position.

Parameters:
\begin{itemize}
	\item[] \meta{x} \subitem Just used internally for recursion.
	\item[] \meta{y} \subitem Just used internally for recursion.
\end{itemize}


Return value:
\begin{itemize} \item[] Absolute position. \end{itemize}


\end{luacommand}\begin{luacommand}{{Position:isAbsPosition}()}
Determines if the position is absolute.


Return value:
\begin{itemize} \item[] True if the position is absolute, else false. \end{itemize}


\end{luacommand}\begin{luacommand}{{Position:new}(\meta{values})}
Represents a relative postion.

Parameters:
\begin{itemize}
	\item[] \meta{values} \subitem Values (e.g. x- and y-coordinate) to be merged with the default-metatable of a position.
\end{itemize}


Return value:
\begin{itemize} \item[] A new position object. \end{itemize}


\end{luacommand}\begin{luacommand}{{Position:relateTo}(\meta{pos},\meta{keepAbsPosition})}
Relates a position to the given position.

Parameters:
\begin{itemize}
	\item[] \meta{pos} \subitem The relative position.\item[] \meta{keepAbsPosition} \subitem If true, the coordinates of the position are computed in the relation to the given position pos.
\end{itemize}



\end{luacommand}

% This file has been generated from the lua sources using LuaDoc.
% To regenerate it call "make genluadoc" in
% doc/generic/pgf/version-for-luatex/en.

\paragraph{pgflibrarygraphdrawing-box.lua}


\begin{luacommand}{{Box:addBox}(\meta{box})}
Adds new internal Box.

Parameters:
\begin{itemize}
	\item[] \meta{box} \subitem The box to be added.
\end{itemize}



\end{luacommand}\begin{luacommand}{{Box:getPaths}()}
Provides all Paths this box contains.


Return value:
\begin{itemize} \item[] Recursive iteration over all paths. \end{itemize}


\end{luacommand}\begin{luacommand}{{Box:getPosAt}(\meta{place},\meta{absolute})}
Calculates the coordinates of the box according to the param place.

Parameters:
\begin{itemize}
	\item[] \meta{place} \subitem Determines of which position of the box the coordinates should be returned (e.g. the center of the box requieres the param Box.CENTER). Possible values are: Box.UPPERLEFT, Box.UPPERRIGHT, Box.CENTER, Box.LOWERRIGHT, Box.LOWERLEFT.\item[] \meta{absolute} \subitem If true the absolute coordinates of the box will be returned, otherwise its relative coordinates.
\end{itemize}


Return value:
\begin{itemize} \item[] X- and y-coordinates of the box. \end{itemize}


\end{luacommand}\begin{luacommand}{{Box:new}(\meta{values})}
Creates a new box.

Parameters:
\begin{itemize}
	\item[] \meta{values} \subitem Values (e.g. height) to be merged with the default-metatable of a box.
\end{itemize}


Return value:
\begin{itemize} \item[] The new box. \end{itemize}


\end{luacommand}\begin{luacommand}{{Box:recalculateSize}()}
Checks internal Boxes and resets width and height.



\end{luacommand}\begin{luacommand}{{Box:removeBox}(\meta{box})}
Removes internal Box.

Parameters:
\begin{itemize}
	\item[] \meta{box} \subitem The box to remove.
\end{itemize}



\end{luacommand}

%% This file has been generated from the lua sources using LuaDoc.
% To regenerate it call "make genluadoc" in
% doc/generic/pgf/version-for-luatex/en.

\paragraph{pgflibrarygraphdrawing-path.lua}


\begin{luacommand}{{Path.toString}(\meta{path})}
Returns a readable string representation of the path.


Return value:
\begin{itemize} \item[] String representation of the path \end{itemize}


\end{luacommand}\begin{luacommand}{{Path:\textunderscore{}intersects}(\meta{a1},\meta{a2},\meta{b1},\meta{b2},\meta{allowedIntersections})}
Checks if the lines a1a2 and b1b2 intersect.

Parameters:
\begin{itemize}
	\item[] \meta{a1} \subitem Start of the first line.\item[] \meta{a2} \subitem End of the first line.\item[] \meta{b1} \subitem Start of the second line.\item[] \meta{b2} \subitem End of the second line.\item[] \meta{allowedIntersections} \subitem A boolean table with the keys a1, a2, b1 and b2. If two or three of those values are true, the corresponding start and/or end points are allowed to match without being seen as intersection. If all four keys are true any matching of start and end points is allowed as long as the two lines are not coincedent. If three of the keys are true or start and end of a line are allowed to match, nill will be returned. If this optional parameter is not given, any matching points will be seen as intersections.
\end{itemize}


Return value:
\begin{itemize} \item[] true, if lines intersect, false otherwise. If allowedIntersections contained an invalid value, nil will be returned. \end{itemize}


\end{luacommand}\begin{luacommand}{{Path:addPoint}(\meta{point},\meta{keepAbsPosition})}
Appends new point at the end of path.

Parameters:
\begin{itemize}
	\item[] \meta{point} \subitem Point to be added to the path\item[] \meta{keepAbsPosition} \subitem true if the coordinates of the point are absolute
\end{itemize}



\end{luacommand}\begin{luacommand}{{Path:createPath}(\meta{posStart},\meta{posEnd},\meta{keepAbsPosition})}
Adds a new segment to the path.

Parameters:
\begin{itemize}
	\item[] \meta{posStart} \subitem Startposition of the new segment\item[] \meta{posEnd} \subitem Endposition of the new segment
\end{itemize}



\end{luacommand}\begin{luacommand}{{Path:getLastPoint}()}
Returns last point in path.


Return value:
\begin{itemize} \item[] last point \end{itemize}


\end{luacommand}\begin{luacommand}{{Path:getLenght}()}
Returns the lenght


Return value:
\begin{itemize} \item[] lenght of the whole path \end{itemize}


\end{luacommand}\begin{luacommand}{{Path:getPoints}()}
Copies the internal points of a path.


Return value:
\begin{itemize} \item[] array of points \end{itemize}


\end{luacommand}\begin{luacommand}{{Path:intersects}(\meta{path})}
Tests if the path is intersected by path.

Parameters:
\begin{itemize}
	\item[] \meta{path} \subitem other path
\end{itemize}



\end{luacommand}\begin{luacommand}{{Path:move}(\meta{x},\meta{y})}
Adds new point with x,y relative to last point.

Parameters:
\begin{itemize}
	\item[] \meta{x} \subitem x-coordinate of the new point\item[] \meta{y} \subitem y-coordinate of the new point
\end{itemize}



\end{luacommand}\begin{luacommand}{{Path:new}(\meta{values})}
Creates a new path.

Parameters:
\begin{itemize}
	\item[] \meta{values} \subitem Values to be merged with the default-metatable of a path
\end{itemize}


Return value:
\begin{itemize} \item[] A new path. \end{itemize}


\end{luacommand}


\subsubsection{Base layer}
% This file has been generated from the lua sources using LuaDoc.
% To regenerate it call "make genluadoc" in
% doc/generic/pgf/version-for-luatex/en.

\paragraph{pgflibrarygraphdrawing-interface.lua}


\begin{luacommand}{{Interface:addEdge}(\meta{from},\meta{to},\meta{direction},\meta{options})}
Adds an edge from one node to another by name.  That is, both parameters are node names and have to exist before an edge can be created between them.

Parameters:
\begin{itemize}
	\item[] \meta{options} \subitem A key=value string, which is currently only passed back to the \TeX layer during shipout (deprecated, use \tikzname\ keys instead).
\end{itemize}



See also:
\begin{itemize}
	\item[] |addNode|
\end{itemize}

\end{luacommand}\begin{luacommand}{{Interface:addNode}(\meta{name},\meta{xMin},\meta{yMin},\meta{xMax},\meta{yMax},\meta{options})}
Adds a new node to the graph.  The options string is parsed and assigned.

Parameters:
\begin{itemize}
	\item[] \meta{name} \subitem Name of the node.\item[] \meta{xMin} \subitem Minimum x point of the bouding box.\item[] \meta{yMin} \subitem Minimum y point of the bouding box.\item[] \meta{xMax} \subitem Maximum x point of the bouding box.\item[] \meta{yMax} \subitem Maximum y point of the bouding box.\item[] \meta{options} \subitem Options to pass to the node (deprecated, use \tikzname\ keys instead).
\end{itemize}



\end{luacommand}\begin{luacommand}{{Interface:drawEdge}(\meta{object})}
Helper function to put visible edges back to the TeX layer.

Parameters:
\begin{itemize}
	\item[] \meta{object} \subitem Lua edge object to draw.
\end{itemize}



\end{luacommand}\begin{luacommand}{{Interface:drawGraph}()}
Draws/layouts the current graph using the specified algorithm.  The algorithm is derived from the options attribute and is loaded on demand from the corresponding file, e.g. for algorithm ``simple'' it is ``pgflibrarygraphdrawing-algorithms-simple.lua'' which has to define a function named ``graph_drawing_algorithm\_simple'' in the pgf.graphdrawing module.  It is then called with the graph as single parameter.



\end{luacommand}\begin{luacommand}{{Interface:drawNode}(\meta{object})}
Helper function to actually put the node back to the TeX layer.

Parameters:
\begin{itemize}
	\item[] \meta{object} \subitem The lua node object to draw.
\end{itemize}



\end{luacommand}\begin{luacommand}{{Interface:finishGraph}()}
Pops the top graph from the graph stack (which is the current graph) and actually draws the nodes and edges on the canvas.



\end{luacommand}\begin{luacommand}{{Interface:getOption}(\meta{name})}
Returns the value of the graph option name.

Parameters:
\begin{itemize}
	\item[] \meta{name} \subitem Name of the option.
\end{itemize}


Return value:
\begin{itemize} \item[] The stored value or nil. \end{itemize}


\end{luacommand}\begin{luacommand}{{Interface:loadAlgorithm}(\meta{name})}
Loads the file with the ``pgflibrarygraphdrawing-algorithms-xyz.lua'' naming scheme.

Parameters:
\begin{itemize}
	\item[] \meta{name} \subitem Name of  the algorithm, like ``xyz''.
\end{itemize}


Return value:
\begin{itemize} \item[] The algorithm function or nil. \end{itemize}


\end{luacommand}\begin{luacommand}{{Interface:newGraph}(\meta{options})}
Creates a new graph and pushes it on top of the graph stack.  The options string is parsed and assigned.

Parameters:
\begin{itemize}
	\item[] \meta{options} \subitem A list of options for this graph (deprecated, use \tikzname\ keys instead).
\end{itemize}



See also:
\begin{itemize}
	\item[] |finishGraph|
\end{itemize}

\end{luacommand}\begin{luacommand}{{Interface:setOption}(\meta{name},\meta{value})}
Sets a graph option name to value.

Parameters:
\begin{itemize}
	\item[] \meta{name} \subitem The name of the option to set.\item[] \meta{value} \subitem New value for the option.
\end{itemize}



\end{luacommand}

\label{section-library-graphdrawing-lua-documentation-interface}
% This file has been generated from the lua sources using LuaDoc.
% To regenerate it call "make genluadoc" in
% doc/generic/pgf/version-for-luatex/en.

\paragraph{pgflibrarygraphdrawing-sys.lua}


\begin{luacommand}{{Sys:beginShipout}()}
Begins the shipout of nodes by opening a scope in pgf.



\end{luacommand}\begin{luacommand}{{Sys:endShipout}()}
Ends the shipout by closing the opened scope.



See also:
\begin{itemize}
	\item[] |Sys:beginShipout()|
\end{itemize}

\end{luacommand}\begin{luacommand}{{Sys:escapeTeXNodeName}(\meta{nodename})}
Adds a ``not yet positionedPGFGDINTERNAL'' prefix to a node name. The prefix is required by pgf to place the node. Actually, when deferring the node placement, the prefix is added to avoid references to the node.

Parameters:
\begin{itemize}
	\item[] \meta{nodename} \subitem Name of the node to prefix.
\end{itemize}


Return value:
\begin{itemize} \item[] A newly composed string. \end{itemize}


\end{luacommand}\begin{luacommand}{{Sys:getTeXBox}()}
Retrieves a box from the transfer box register.



See also:
\begin{itemize}
	\item[] |putTeXBox|
\end{itemize}

\end{luacommand}\begin{luacommand}{{Sys:getVerboseMode}()}
Checks the verbosity of the subsystems output.


Return value:
\begin{itemize} \item[] Boolean value specifying the verbosity. \end{itemize}


\end{luacommand}\begin{luacommand}{{Sys:logMessage}(\meta{...})}
Prints objects to the TeX output, formatting them with tostring.

Parameters:
\begin{itemize}
	\item[] \meta{...} \subitem List of parameters.
\end{itemize}



\end{luacommand}\begin{luacommand}{{Sys:putEdge}(\meta{edge},\meta{Edge})}
Assembles and outputs the TeX command to draw an edge.

Parameters:
\begin{itemize}
	\item[] \meta{Edge} \subitem A lua edge object.
\end{itemize}



\end{luacommand}\begin{luacommand}{{Sys:putTeXBox}(\meta{nodename},\meta{texnode},\meta{minX},\meta{minY},\meta{maxX},\meta{maxY},\meta{posX},\meta{posY},\meta{nodeName})}
Saves a box from the transfer box register.

Parameters:
\begin{itemize}
	\item[] \meta{texnode} \subitem The box which contains the \TeX\ node.\item[] \meta{minX} \subitem Maximum y of the bounding box.\item[] \meta{minY} \subitem Minimal y of the bounding box.\item[] \meta{posX} \subitem X coordinate where to put the node in the output.\item[] \meta{posY} \subitem Y coordinate where to put the node in the output.\item[] \meta{nodeName} \subitem The name of the node in the box.
\end{itemize}



\end{luacommand}\begin{luacommand}{{Sys:setBoxNumber}(\meta{bn})}
Init method, sets the box register number. This method is called when the \tikzname\ (pgf) library is loaded.

Parameters:
\begin{itemize}
	\item[] \meta{bn} \subitem Number of the box register used for transfering boxes of the current graph.
\end{itemize}



\end{luacommand}\begin{luacommand}{{Sys:setVerboseMode}(\meta{mode})}
Enables or disables verbose logging for the graph drawing library.

Parameters:
\begin{itemize}
	\item[] \meta{mode} \subitem If true, enable verbose logging. Otherwise it'll be disabled.
\end{itemize}



\end{luacommand}\begin{luacommand}{{Sys:unescapeTeXNodeName}(\meta{nodename})}
Removes the ``not yet positionedPGFGDINTERNAL'' prefix from a node name.

Parameters:
\begin{itemize}
	\item[] \meta{nodename} \subitem Nodename without prefix.
\end{itemize}


Return value:
\begin{itemize} \item[] The substring in question. \end{itemize}


See also:
\begin{itemize}
	\item[] |Sys:escapeTeXNodeName(nodename)|
\end{itemize}

\end{luacommand}

\label{section-library-graphdrawing-lua-documentation-sys}
% This file has been generated from the lua sources using LuaDoc.
% To regenerate it call "make genluadoc" in
% doc/generic/pgf/version-for-luatex/en.

\paragraph{pgflibrarygraphdrawing-texboxregister.lua}


\begin{luacommand}{{TeXBoxRegister:getBox}(\meta{boxReference})}
Gets a box by its reference.

Parameters:
\begin{itemize}
	\item[] \meta{boxReference} \subitem Reference id of the box to get.
\end{itemize}



See also:
\begin{itemize}
	\item[] |TeXBoxRegister:insertBox(texbox)|
\end{itemize}

\end{luacommand}\begin{luacommand}{{TeXBoxRegister:insertBox}(\meta{texbox})}
Adds the content of a \TeX\ box to the box register class. Contents of the box will be stored. 



\end{luacommand}


\subsubsection{Helper classes}
% This file has been generated from the lua sources using LuaDoc.
% To regenerate it call "make genluadoc" in
% doc/generic/pgf/version-for-luatex/en.

\paragraph{pgflibrarygraphdrawing-helper.lua}


\begin{luacommand}{{copyTable}(\meta{table},\meta{result})}
Copies a table, preserving its metatable.

Parameters:
\begin{itemize}
	\item[] \meta{table} \subitem The table from which values are copied.\item[] \meta{result} \subitem The table to which values are copied or nil.
\end{itemize}


Return value:
\begin{itemize} \item[] A new table containing all the keys and values. \end{itemize}


\end{luacommand}\begin{luacommand}{{countKeys}(\meta{table})}
Counts keys in an dictionary, where value is nil.

Parameters:
\begin{itemize}
	\item[] \meta{table} \subitem Dictionary.
\end{itemize}


Return value:
\begin{itemize} \item[] Number of keys. \end{itemize}


\end{luacommand}\begin{luacommand}{{filter}(\meta{iterator},\meta{test})}
Returns all results from iterator for which test returns a true value.



\end{luacommand}\begin{luacommand}{{findTable}(\meta{table},\meta{object})}
Finds an object in a table.


Return value:
\begin{itemize} \item[] The first index for a value which is equal to the object or nil. \end{itemize}


\end{luacommand}\begin{luacommand}{{keys}(\meta{map})}
Returns all keys in arbitrary order.



See also:
\begin{itemize}
	\item[] |pairs|
\end{itemize}

\end{luacommand}\begin{luacommand}{{mergeTable}(\meta{values},\meta{defaults})}
Merges two tables. Every nil value in values is replaced by its default value in defaults.  The metatable from defaults is likewise preserved. As luatex supplies its own version of table.merge, we can't use that same name.

Parameters:
\begin{itemize}
	\item[] \meta{values} \subitem New values or nil.  Same as return value if non-nil.
\end{itemize}



\end{luacommand}\begin{luacommand}{{parseBraces}(\meta{string},\meta{default})}
Parses a braced list of {key}{value} pairs and returns a table mapping keys to values.



\end{luacommand}\begin{luacommand}{{values}(\meta{table})}
Returns all values in numerical order.



See also:
\begin{itemize}
	\item[] |ipairs|
\end{itemize}

\end{luacommand}


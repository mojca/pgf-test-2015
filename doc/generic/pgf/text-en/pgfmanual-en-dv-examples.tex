% Copyright 2006 by Till Tantau
%
% This file may be distributed and/or modified
%
% 1. under the LaTeX Project Public License and/or
% 2. under the GNU Free Documentation License.
%
% See the file doc/generic/pgf/licenses/LICENSE for more details.


My first tikz plot:

\begin{tikzpicture}

  \datavisualization[
      candle stick plot,
      index/source=dax,
    ]
    data sets[format=space separated columns,
         column 1=dax/low,
         column 2=dax/high,
         column 3=dax/entry,
         column 4=dax/exit]
    {
      \dataset
      {% today
        2000 2300 2100 2200 
        2000 2350 2200 2500
        2200 2300 2250 2260 
        1800 2260 2260 1900 
        2000 2300 2100 2200
      }
    
      \dataset[format=comma separated columns]
      {
        % yesterday
        2000, 2350, 2200, 2250 
        2200, 2300, 2250, 2260 
      }

      \dataset[format=function]
      {
        var x=[2000:2260] samples 5;
        var y=[0:100] samples 5;
        func dax/low=\value x;
        func dax/high=\value x+100;
        func dax/entry=\value x+\value y;
        func dax/exit=\value x+2*\value y;
      }
    };
\end{tikzpicture}

\endinput



A candle-stick plot:

\pgfdeclaredataformat{dax line}{}{}{dax: #1 #2 #3 #4}{
  \pgfkeyssetvalue{/data point/dax/low}{#1}
  \pgfkeyssetvalue{/data point/dax/high}{#2}
  \pgfkeyssetvalue{/data point/dax/entry}{#3}
  \pgfkeyssetvalue{/data point/dax/exit}{#4}
  \pgfdatapoint
}{}{}

\def\daxline#1 #2 #3 #4 {
  \pgfkeyssetvalue{/data point/dax/low}{#1}
  \pgfkeyssetvalue{/data point/dax/high}{#2}
  \pgfkeyssetvalue{/data point/dax/entry}{#3}
  \pgfkeyssetvalue{/data point/dax/exit}{#4}
  \pgfdatapoint
}



\begin{tikzpicture}
  \pgfnewdatavisualization

  \pgfoonew \obj=new count up(day)
  \obj.default connects()
  
  \pgfoonew \obj=new line transformer(day,\pgfqpoint{3mm}{0mm})
  \obj.default connects()
  \pgfoonew \obj=new line transformer(y,\pgfqpoint{1mm}{1cm})
  \obj.default connects()

  \pgfoonew \obj=new attribute mapper(dax,0,100,y,0,1,)
  \obj.default connects()
  \pgfoonew \obj=new candle stick visualizer(dax)
  \obj.default connects()

  \pgfoonew \dayinterval=new interval(,)
  \dayinterval.default connects()
  \dayinterval.set min adjust relative(-.1)
  \dayinterval.set max adjust relative(.1)
  \pgfoonew \daybounder=new attribute bounder(day,\dayinterval)
  \daybounder.default connects()

  \pgfoonew \daxinterval=new interval(,)
  \daxinterval.default connects()
  \daxinterval.set min adjust relative(-.1)
  \daxinterval.set max adjust relative(.1)
  \pgfoonew \daxbounder=new attribute bounder(dax,\daxinterval)
  \daxbounder.default connects()
  
  \pgfoonew \dayaxis=new straight axis(day,\dayinterval)
  \dayaxis.default connects()
  \pgfoonew \daxaxis=new straight axis(dax,\daxinterval)
  \daxaxis.default connects()

  \pgfdatavisualizationrender
    [format=space separated columns,
     column 1=dax/low,
     column 2=dax/high,
     column 3=dax/entry,
     column 4=dax/exit]
   \data
     {
      % today
      2000 2300 2100 2200 
      2000 2350 2200 2500
      2200 2300 2250 2260 
      1800 2260 2260 1900 
      2000 2300 2100 2200
    }
    
    \data[format=comma separated columns]
    {
      % yesterday
      2000, 2350, 2200, 2250 
      2200, 2300, 2250, 2260 
    }
    %\data[source=data,format=comma separated columns]
  \relax

  \pgfkeys{/data point/dax/low/.initial=,
    /data point/dax/high/.initial=,
    /data point/dax/entry/.initial=,
    /data point/dax/exit/.initial=}
    
  \pgfdatavisualizationrender{
    \data[format=key value pairs]{
      dax/low=1800, dax/high=2200, %
      dax/entry=1800, dax/exit=1850
    }
  }
\end{tikzpicture}


A simple plot

\def\dataline#1 #2 #3 #4 #5 #6 #7 #8 {%
  \pgfkeyssetvalue{/data point/a}{#1}
  \pgfkeyssetvalue{/data point/b}{#2}
  \pgfkeyssetvalue{/data point/c}{#3}
  \pgfkeyssetvalue{/data point/d}{#4}
  \pgfkeyssetvalue{/data point/e}{#5}
  \pgfkeyssetvalue{/data point/f}{#6}
  \pgfkeyssetvalue{/data point/g}{#7}
  \pgfkeyssetvalue{/data point/h}{#8}
  \pgfdatapoint
}

\begin{tikzpicture}
  \pgfnewdatavisualization

  %\pgfoonew new attribute mapper(a,0,1,x,0,1,)
  %\pgfoonew new attribute mapper(b,0,1,y,0,1,)
  \pgfoonew \obj=new count up(y)
  \obj.default connects()
  
  \pgfoonew \obj=new sum(a)
  \obj.default connects()
  
  \pgfkeyssetvalue{/data point/x}{\pgfkeysvalueof{/data point/a/sum}}
  
  \pgfoonew \transformer=new line transformer(x,\pgfpointxy{.1}{.05})
  \transformer.default connects()
  
  \pgfoonew \transformer=new line transformer(y,\pgfpointxy{-0.1}{1})
  \transformer.default connects()

  \pgfoonew \visualizer=new plot handler visualizer(\pgfplothandlerlineto)
  \visualizer.default connects()
  \visualizer.set use path(\color{red}\pgfusepath{stroke})
  \pgfoonew \visualizer=new plot mark visualizer(*)
  \visualizer.default connects()
  
  \pgfoonew \interval=new interval(0,)
  \interval.default connects()
  \interval.set min adjust relative(-.1)
  \interval.set max adjust relative(.1)
  \pgfoonew \abounder=new attribute bounder(x,\interval)
  \abounder.default connects()
  
  \pgfoonew \yinterval=new interval(,)
  \yinterval.set min adjust relative(-.1)
  \yinterval.set max adjust relative(.1)
  \yinterval.default connects()
  
  \pgfoonew \bbounder=new attribute bounder(y,\yinterval)
  \bbounder.default connects()
  
  \pgfoonew \pinterval=new interval(,)
  \pinterval.default connects()
  \pgfoonew \pbounder=new attribute bounder(angle,\pinterval)
  \pbounder.default connects()

  \pgfoonew \rinterval=new interval(,)
  \rinterval.default connects()
  \pgfoonew \rbounder=new attribute bounder(b,\rinterval)
  \rbounder.default connects()
  
  \pgfoonew \axis=new straight axis(x,\interval)
  \axis.default connects()
  \pgfoonew \bxis=new straight axis(y,\yinterval)
  \bxis.default connects()

  \pgfdatavisualizationrender{
    \data[format=TeX code]
  {
    \dataline 10.0	8.04	10.0	9.14	10.0	7.46	8.0	6.58 
    \dataline 8.0	6.95	8.0	8.14	8.0	6.77	8.0	5.76 
    \dataline 13.0	7.58	13.0	8.74	13.0	12.74	8.0	7.71 
    \dataline 9.0	8.81	9.0	8.77	9.0	7.11	8.0	8.84 
    \dataline 11.0	8.33	11.0	9.26	11.0	7.81	8.0	8.47 
    \dataline 14.0	9.96	14.0	8.10	14.0	8.84	8.0	7.04 
    {\dataline 6.0	7.24	6.0	6.13	6.0	6.08	8.0	5.25 }
    \dataline 4.0	4.26	4.0	3.10	4.0	5.39	19.0	12.50 
    \dataline 12.0	10.84	12.0	9.13	12.0	8.15	8.0	5.56 
    {\pgfkeyssetvalue{/data point/name}{yesterday} \dataline 7.0	4.82	7.0	7.26	7.0	6.42	8.0	7.91 }
    {\pgfkeyssetvalue{/data point/name}{today} \dataline 5.0	5.68	5.0	4.74	5.0	5.73	8.0    6.89 }
  }}

  \draw (yesterday) -- (today);
  \pgfkeyssetvalue{/data point/tick lower length}{4pt}
  \pgfkeyssetvalue{/data point/tick higher length}{4pt}

  \begin{scope}[->,thick]
     {
       \bbounder.goto min()
       \pgfkeyssetvalue{/data point/y}{0}
       \axis.visualize()
       % Grid lines
       \begin{scope}[-,thin,black!50]
         \bbounder.goto min()
         \pgfkeysgetvalue{/data point/y}{\mymin}
         \bbounder.goto max()
         \pgfkeysgetvalue{/data point/y}{\mymax}
         \foreach \yval in {\mymin,...,\mymax}
         {
           \pgfkeyssetvalue{/data point/y}{\yval}
           \axis.visualize()
           \begin{scope}[thick,black]
             \axis.visualize as tick()
           \end{scope}
         }
       \end{scope}
     }
    {
      \abounder.goto min()
      \bxis.visualize()
      % Grid lines
      \begin{scope}[-,thin,black!50]
        \abounder.goto min()
        \pgfkeysgetvalue{/data point/x}{\mymin}
        \abounder.goto max()
        \pgfkeysgetvalue{/data point/x}{\mymax}
        \foreach \xval in {0,10,...,\mymax}
        {
          \pgfkeyssetvalue{/data point/x}{\xval}
          \bxis.visualize()
            \begin{scope}[thick,black]
              \bxis.visualize as tick()
            \end{scope}

            \pgfkeyssetvalue{/data point/tick lower length}{0pt}
            \pgfkeyssetvalue{/data point/tick higher length}{3pt}
          \pgfmathparse{\xval+9}
          \let\temp=\pgfmathresult
          \foreach \xvala in {\xval,...,\temp}
          {
            \pgfkeyssetvalue{/data point/x}{\xvala}
            \bxis.visualize as tick()
          }
        }
      \end{scope}
    }
  \end{scope}
  

\end{tikzpicture}

Anscombe's Quartet
I	II	III	IV
x	y	x	y	x	y	x	y
10.0	8.04	10.0	9.14	10.0	7.46	8.0	6.58
8.0	6.95	8.0	8.14	8.0	6.77	8.0	5.76
13.0	7.58	13.0	8.74	13.0	12.74	8.0	7.71
9.0	8.81	9.0	8.77	9.0	7.11	8.0	8.84
11.0	8.33	11.0	9.26	11.0	7.81	8.0	8.47
14.0	9.96	14.0	8.10	14.0	8.84	8.0	7.04
6.0	7.24	6.0	6.13	6.0	6.08	8.0	5.25
4.0	4.26	4.0	3.10	4.0	5.39	19.0	12.50
12.0	10.84	12.0	9.13	12.0	8.15	8.0	5.56
7.0	4.82	7.0	7.26	7.0	6.42	8.0	7.91
5.0	5.68	5.0	4.74	5.0	5.73	8.0	6.89



%%% Local Variables: 
%%% mode: latex
%%% TeX-master: "pgfmanual"
%%% End: 

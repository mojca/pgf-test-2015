% Copyright 2012 by Till Tantau
%
% This file may be distributed and/or modified
%
% 1. under the LaTeX Project Public License and/or
% 2. under the GNU Free Documentation License.
%
% See the file doc/generic/pgf/licenses/LICENSE for more details.

\section{Graph Drawing Algorithms: Layered Layouts}

{\emph{by  Till Tantau and Jannis Pohlmann}}

File status: Large parts still need to be written

\begin{tikzlibrary}{graphdrawing.layered}
  This library provides keys for drawing graphs using the Sugiyama
  method, which is especially useful for drawing hierachical graphs.
  You should load the |graphdrawing| library first.
\end{tikzlibrary}



\subsection{Overview}

A ``layered'' layout of a graph tries to arrange the nodes in
consecutive horizontal layers (naturally, by rotating the graph, this
can be changed in to vertical layers) such that edges tend to be only
between nodes on adjacent layers. Trees, for instance, can always be
laid out in this way. This method of laying out a graph is especially
useful for hierarchical graphs.

The method implemented in this library is often called the
\emph{Sugiyama method}, which is a rather advanced method of
assigning nodes to layers and positions on these layers. The same
method is also used in the popular GraphViz program, indeed, the
implementation in \tikzname\ is based on the same pseudo-code from the
same paper as the implementation used in GraphViz and both programs
will often generate the same layout (but not always, as explained
below). The current implementation is due to Jannis Pohlmann, who
implemented it as part of his Diploma thesis. Please consult this
thesis for a detailed explanation of the Sugiyama method and its
history:

\begin{itemize}
\item
  Jannis Pohlmann,
  \newblock \emph{Configurable Graph Drawing Algorithms
    for the \tikzname\ Graphics Description Language,}
  \newblock Diploma Thesis,
  \newblock Institute of Theoretical Computer Science, Univerist\"at
  zu L\"ubeck, 2011.\\[.5em]
  \newblock Online at 
  \url{http://www.tcs.uni-luebeck.de/downloads/papers/2011/2011-configurable-graph-drawing-algorithms-jannis-pohlmann.pdf}
  \\[.5em]
  (Note that since the publication of this thesis some option names
  have been changed. Most noticeably, the option name
  |layered drawing| was changed to |layered layout|, which is somewhat
  more consistent with other names used in the graph drawing
  libraries.) 
\end{itemize}

The Sugiyama methods lays out a graph in five steps:
\begin{enumerate}
\item Cycle removal.
\item Layer assignment (sometimes called node ranking).
\item Crossing minimization (also referred to as node ordering).
\item Node positioning (or coordinate assignment).
\item Edge routing.
\end{enumerate}
It turns out that behind each of these steps there lurks an
NP-complete problem, which means, in practice, that each step is
impossible to perform optimally for larger graphs. For this reason,
heuristics and approximation algorithms are used to find a ``good''
way of performing the steps.

A distinctive feature of Pohlmann's implementation of the Sugiyama
method for \tikzname\ is that the algorithms used for each of the
steps can easily be exchanged, just specify a different option. For
the user, this means that by specifying a different 
option and thereby using a different heuristic for one of the steps, a
better layout can often be found. For the researcher, this means that
one can very easily test new approaches and new heuristics without
having to implement all of the other steps anew. 



\subsection{Standard Layered Layout}

In order to compute a layered layout of a graph, use the following option:

\begin{gdalgorithm}{layered layout}{Sugiyama Modular Layered}
  The |layered layout| is the key used to select the modular Sugiyama
  layout algorithm. As explained in the overview of this section, this
  algorithm consists of five consecutive steps, each of which can be
  configured independently of the other ones (how this is done is
  explained later in this section). Naturally, the ``best'' heuristics
  are selected by default, so there is typically no need to change the
  settings, but what is the ``best'' method for one graph need not be
  the best one for another graph.
  
\begin{codeexample}[]
\tikz \graph [layered layout, sibling distance=7mm]
{
  a -> {
    b,
    c -> { d, e, f }
  } ->
  h ->
  a
};    
\end{codeexample}

  As can be seen in the above example, the algorithm will not only
  position the nodes of a graph, but will also perform an edge
  routing. This will look visually quite pleasing if you add the
  |rounded corners| option:

\begin{codeexample}[]
\tikz [rounded corners] \graph [layered layout, sibling distance=7mm]
{
  a -> {
    b,
    c -> { d, e, f }
  } ->
  h -> 
  a
};    
\end{codeexample}


\end{gdalgorithm}



%%% Local Variables: 
%%% mode: latex
%%% TeX-master: "pgfmanual-pdftex-version"
%%% End: 

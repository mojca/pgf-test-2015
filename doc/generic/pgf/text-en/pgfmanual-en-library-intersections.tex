% Copyright 2008 by Mark Wibrow
%
% This file may be distributed and/or modified
%
% 1. under the LaTeX Project Public License and/or
% 2. under the GNU Free Documentation License.
%
% See the file doc/generic/pgf/licenses/LICENSE for more details.


\section{Intersections Library}

{\bf\emph{This library is experimental and likely to change,
move, or disappear, without warning.}}

\begin{pgflibrary}{intersections}
  This library enables the calculation of intersections of
  two arbitrary paths.
\end{pgflibrary}

\subsection{Intesecting Two Paths in PGF}

\begin{command}{\pgfintersectionofpaths\marg{path 1}\marg{path 2}}
  This command finds the intersection points on the paths 
  \meta{path 1} and \meta{path 2}. The number of intersection points
  (``solutions'') that are found will be stored, and each point 
  can be accessed separately. The code for \meta{path 1} and 
  \meta{path 2} is executed within a \TeX{} group and so can contain
  transformations (which will be in addition to any existing
  transformations). The code should not use the path in any way, 
  unless the path is saved first and restored afterwards.
  \pgfname{} will regard solutions as ``a bit
  special'', in that the points returned  will be ``absolute'' and 
  unaffected by any further transformations.

\begin{codeexample}[]
\begin{pgfpicture}
\pgfintersectionofpaths
{
  \pgfpathellipse{\pgfpointxy{0}{0}}{\pgfpointxy{1}{0}}{\pgfpointxy{0}{2}}
  \pgfgetpath\temppath
  \pgfusepath{stroke}
  \pgfsetpath\temppath
}
{
  \pgftransformrotate{-30}
  \pgfpathrectangle{\pgfpointorigin}{\pgfpointxy{2}{2}}
  \pgfgetpath\temppath
  \pgfusepath{stroke}
  \pgfsetpath\temppath
}
\foreach \s in {1,...,\pgfintersectionsolutions}
  {\pgfpathcircle{\pgfpointintersectionsolution{\s}}{2pt}}
\pgfusepath{stroke}
\end{pgfpicture}
\end{codeexample}

\end{command}

\begin{command}{\pgfintersectionsolutions}
  After using the |\pgfintersectionofpaths| command, this \TeX-counter
  will hold the number of solutions found.
\end{command}

\begin{command}{\pgfpointintersectionsolution\marg{number}}
  After using the |\pgfintersectionofpaths| command, this command
  will return the point for solution \meta{number}. Unfortunately
  there can be no guarantee of a ``helpful'' ordering of solutions.
\end{command}

\subsection{Intersecting Two Paths in \tikzname}

  To intersect two paths in \tikzname, they must first be
  ``named''. A ``named path'' is a path that has been named using 
  the following key:
    
\begin{key}{/tikz/path name=\meta{name}}

  The effect of this key is that, after the path has been constructed,
  just before it is used, it is associated \meta{name}. This 
  association is global, so lasts after the final semi-colon. 
  
\begin{codeexample}[code only]
\path [name=straight line] (0,0) -- (3,2);
\end{codeexample}
\end{key}

  The solution(s) for the intersection can be accessed using the
  |intersection| coodinate system, using the |/tikz/cs/solution| key
  to access different solutions. This library adds two more keys
  to specify the named paths:

\begin{key}{/tikz/cs/first path name=\meta{name}}
  Specify the first path name.
\end{key}

\begin{key}{/tikz/cs/second path name=\meta{name}}
  Specify the second path name.
\end{key}
  
  These keys can then be used as follows:
  
\begin{codeexample}[]
\begin{tikzpicture}
  \draw [help lines] grid (3,2);
  \draw [path name=circle]  (1.5,1) circle (0.75cm);
  \draw [path name=curve] (0,0) .. controls (0,2) and (3,0) .. (3,2);
  \fill [red, opacity=0.5]
    (intersection cs:first path name=circle, second path name=curve)
     circle (2pt);
\end{tikzpicture}
\end{codeexample}

  One drawback with the |intersection| coodinate system is that
  evey time it is used, the solutions are recalculated. It would be
  easier if the solutions could be calculated ``all at once'', and
  then accessed later. This is possible by using the following key:
  
\begin{key}{/tikz/intersection of named paths=\meta{name 1}| |and| |\meta{name 2}}
  This key will calculate all the intersections of the named paths
  \meta{name 1} and \meta{name 2}. The following commands will then
  be available until the end of the path or scope:

\begin{command}{\solutions}
  This will expand to the number of solutions found.
\end{command}

\begin{command}{\solution\marg{number}}
  This will expand to the coordinates of the solution specified or the 
  origin, if solution \meta{number} was found. For safety this should
  be used with the |shift| key, but in most circumstances it can be
  used in a coordinate in the usual way.
  Unfortunately there is no easy way to guarantee the order of the
  solutions.
\end{command}

  

\begin{codeexample}[]
\begin{tikzpicture}[every node/.style={opacity=1, circle, black, above left}]
  \clip (-2,-2) rectangle (2,2);
  \draw [path name=curve 1] (-2,-1) .. controls (8,-1) and (-8,1) .. (2,1);
  \draw [path name=curve 2] [rotate=90] 
     (-2,-1) .. controls (8,-1) and (-8,1) .. (2,1);
  \fill [red, opacity=0.5, intersection of named paths=curve 1 and curve 2]
    \foreach \s in {1,...,\solutions}
      { (\solution \s) circle (2pt) node {\footnotesize\s} };
\end{tikzpicture}
\end{codeexample}

   One important fact to bear in mind when using the |\solution|
   command is that the points are \emph{not} immune to transformations.

\begin{codeexample}[]
\begin{tikzpicture}[every node/.style={opacity=1, circle, black, above left}]
  \draw [help lines] grid (3,2);
  \draw [path name=circle] (1.5,1) circle (0.75cm);
  \draw [path name=rectangle]  (0.5,0.5) rectangle +(2,1);
  \tikzset{intersection of named paths=circle and rectangle}
  \fill [red, opacity=0.5] \foreach \s in {1,...,\solutions}
      { (\solution \s) circle (2pt) node {\footnotesize\s} };
  \fill [blue, shift={(0.5,0.5)}, opacity=0.5] 
    \foreach \s [count=\i from 4] in {1,...,\solutions}
      { (\solution \s) circle (2pt) node {\footnotesize\i} };   
\end{tikzpicture}%
\end{codeexample}
   
\end{key}



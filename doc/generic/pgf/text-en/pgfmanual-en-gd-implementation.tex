% Copyright 2010-2011 by Renée Ahrens
% Copyright 2010-2011 by Olof Frahm
% Copyright 2010-2011 by Jens Kluttig
% Copyright 2010-2011 by Matthias Schulz
% Copyright 2010-2011 by Stephan Schuster
% Copyright 2011 by Jannis Pohlmann
% Copyright 2011 by Till Tantau
%
% This file may be distributed and/or modified
%
% 1. under the LaTeX Project Public License and/or
% 2. under the GNU Free Documentation License.
%
% See the file doc/generic/pgf/licenses/LICENSE for more details.


\section{Implementing Graph Drawing Algorithms: The \TeX\ Side}

{\noindent {\itshape{Original by Ren\'ee Ahrens, Olof-Joachim Frahm, Jens Kluttig,
  Matthias Schulz, Stephan Schuster\par\noindent Rewritten by Till Tantau}}}

\label{section-base-graphdrawing}
\label{section-gd-internals}

File status: Layout management has changed, this is not reflected by
the contents of this file...


\begin{pgflibrary}{graphdrawing}
  This library is what users will usually load. It loads the above
  library automatically. Nevertheless, the graph drawing engine can be
  used independently of \tikzname.
\end{pgflibrary}


This chapter explains the technical details of how the \TeX\ layer
communicates with the Lua side of the graph drawing engine. A note of
caution:

\begin{quote}
  \itshape You do not need to understand the details of this
  chapter to write new graph drawing algorithms. It is of interest
  only to those wishing to extend the syntactic features of the \TeX\ side
  of the graph drawing engine.
\end{quote}


\subsection{The Call Graph of the Graph Drawing Engine}

Let us start with the broad picture. The general idea behind the whole
graph drawing engine is that whenever \pgfname\ creates a node, we
intercept this node creation and \emph{do not} immediately place the
node. Rather, we pass it down to Lua, which tucks it away in
some internal tables. For edges, we introduce a special command called
|\pgfgdedge| that tells Lua that there is an edge between two
tucked-away nodes. Then, after the graph has been completely
specified, a graph drawing algorithm written in Lua starts to work on
the graph by computing new positions for the nodes. Then, the graph
drawing engine will send back the nodes and edges to \pgfname, which
then finally places them at their final positions.
Note that graph drawing algorithms need no knowledge of how any of
these internals work.

Let us have a look at a simple example to see what happens when a
graph is specified:

\begin{codeexample}[]
\tikz[tree layout]
  \graph {root [as=Hello] -> World[fill=blue!20]};
\end{codeexample}

The key |tree layout| internally calls the key |request scope and layout|,
which in turn calls the macro |\pgfgdbeginscope|, which starts up the
graph drawing engine. Once this macro has been 
called, until the next call of |\pgfgdendscope|, all nodes that are
created actually get passed down to the graph drawing engine. This is
implemented on the lowest layer, namely by directly intercepting
nodes freshly created using |\pgfnode|. In our example, this happens
in two places: For the |root| node and for the |World| node. The
|graph| library and \tikzname\ internally call the |\pgfnode| macro
for these two nodes (after a large number of internal syntax
translations, but the graph drawing engine does not care about them).

Note that the node boxes will have been fully created before they are
passed down to the graph drawing engine -- only their final position
is not yet fixed. It is not possible to modify the size of nodes
inside the graph drawing engine. 

In contrast, the single edge of the graph that is created by the |->|
command is not fully created before it is passed down to the
engine. This would not really make sense since before the final
positions of the nodes are fixed, we cannot even begin to compute the
length of this edge, let alone where it should start or end. For this
reason, on the upper \tikzname\ layer, the normal edge creation that
would be caused by |->| via |new ->| is suppressed. Instead, the
command |\pgfgdedge| is called. Similarly, inside a graph drawing
scope, \tikzname\ will suppress both the |edge| and the
|edge from parent| command and cause |\pgfgdedge| to be called
instead. 

An overview of what happens is illustrated by the following call graph:

\begin{tikzpicture}[
    class name/.style={draw,minimum size=20pt, fill=blue!20},
    object node/.style={draw,minimum size=15pt, fill=yellow!20},
    p/.style={->,>=spaced stealth'},
    livespan/.style={very thick},
    xscale=0.8]
  % class names above
  \node (tikz) at (0,4) [class name] {\tikzname\ layer};
  \node (tex) at (6,4) [class name] {\pgfname\ layer};
  \node (interface) at (13,4) [class name] {Lua};
  % lines from the class names to the bottom of the picture
  \draw[livespan] (tikz) -- (0,-6.5);
  \draw[livespan] (tex) -- (6,-6.5);
  \draw[livespan] (interface) -- (13,-6.5);
  % first command: \graph{  -- generates new graph in lua interface
  \node (tikz-begin-graph) at (0,3) [object node] {|\graph[... layout]{|}; %}
  \node (tex-begin-graph) at (6,3) [object node] {|\pgfgdbeginscope|};  
  \node (interface-new-graph) at (13,3) [object node] {|beginGraphDrawingScope(|...|)|};
  \draw [p] (tikz-begin-graph.east) -- (tex-begin-graph.west);
  \draw [p] (tex-begin-graph.east) -- (interface-new-graph.west);    
  % second command: a -> b   -- generates two nodes in lua
  % and one edge
  \node (tikz-node) at (0,2) [object node] {|a -> b;|};
  \node (tex-node) at (6,2) [object node, double copy shadow] {|\pgfnode|};
  \draw[p] (tikz-node.east) -- (tex-node.west);
  
  \node (interface-add-node) at (13,2) [object node, double copy shadow] {|addPgfNode(|...|)|};
  \draw[p] (tex-node.east) -- (interface-add-node.west);

  \node (tex-add-edge) at (6,1) [object node, double copy shadow] {|\pgfgdedge|};
  \node (interface-add-edge) at (13,1) [object node, double copy shadow] {|addPgfEdge(|...|)|};
  \draw[p] (tikz-node.east) -- (1.5,2) -- (1.5,1) -- (tex-add-edge.west);
  \draw[p] (tex-add-edge.east) -- (interface-add-edge.west);

  % scope ends -- cloes graph, layouts it and draws it
  \node (tikz-end) at (0,0) [object node] {|};|};
  \node (tex-end) at (6,0) [object node] {|\pgfgdendscope|};
  \node (interface-draw-graph) at (13,0) [object node] {|runGraphDrawingAlgorithm()|};
  \node (interface-finish-graph) at (13,-2) [object node] {|endGraphDrawingScope()|};

  \node (invoke-algorithm) at (15.5,-1) [object node] {invoke algorithm};
  \draw[p] (tikz-end.east) -- (tex-end.west);
  \draw[p] (tex-end.east) -- (interface-draw-graph.west);
  \draw[p] (interface-draw-graph.east) -| (invoke-algorithm.20);
  \draw[p] (tex-end.east) -- (9.5,0) -- (9.5,-2) -- (interface-finish-graph.west);

  % begin shipout
  \node (tex-begin-shipout) at (6,-3) [object node] {|\pgfgdbeginshipout|};

  \node (tex-puttexbox) at (6,-4) [object node, double copy shadow] {|\pgfpositionnodenow|};
  \node (tex-putedge) at (6,-5) [object node, double copy shadow] {|\pgfgdedgecallback|};
  
  \node (tex-end-shipout) at (6,-6) [object node] {|\pgfgdendshipout|};
  
  \draw [p] (interface-finish-graph.-170) |- (tex-begin-shipout.east);
  \draw [p] (interface-finish-graph.-170) |- (tex-puttexbox.east);
  \draw [p] (interface-finish-graph.-170) |- (tex-putedge.east);
  \draw [p] (interface-finish-graph.-170) |- (tex-end-shipout.east);
  %(interface-finish-graph.east) -- (12.5,-2) -- (12.5,-4) -- (sys-puttexbox.west);

  % put edge
  %(interface-finish-graph.east) -- (12.5,-2) -- (12.5,-5) -- (sys-put-edge.west);
  % end shipout
  %(interface-finish-graph.east) -- (12.5,-2) -- (12.5,-6) -- (sys-end-shipout.west);
\end{tikzpicture}

In Section~\ref{section-gd-scopes} we detail the different commands
needed to communicate with the graph drawing engine. In particular,
the commands for beginning and ending scopes are explained here. The
next section, Section~\ref{section-gd-parameters} explains how graph
parameters are setup and configured.


\subsection{Graph Drawing Scopes}
\label{section-gd-scopes}

\subsubsection{The Main Graph Drawing Scope}

\begin{command}{\pgfgdbeginscope}
  This macro starts a \TeX\ scope. The following things will happen:
  \begin{enumerate}
  \item A new |scope| object is created and put on an internal stack
    of scopes. All subsequent operations will work on this graph until
    |\pgfgdendscope| is called. In particular, it is permissible to
    have nested graph drawing scopes.
  \item Inside the scope, no nodes are placed immediately. Rather, 
    the macro |\pgfpositionnodelater|, see
    Section~\ref{section-shapes-deferred-node-positioning}, is used to
    send all nodes created inside the scope to some internal collector
    macro. This internal collector macro will pass down the created
    node to Lua.
  \item Some additional \meta{code} is executed, which has been set
    using the following command:
    \begin{command}{\pgfgdaddspecificationhook\marg{code}}
      This command adds the \meta{code} to the code that is executed
      whenever a graph drawing scope starts. For instance, the
      \tikzname\ library |graphdrawing| uses this macro to add some
      \meta{code} that will redirect the |edge| and |edge from parent|
      path commands to |\pgfgdedge|.
    \end{command}
  \item |\pgftransformreset| is called. See
    Section~\ref{section-gd-lua-coordinates} for details on the effect
    of this.
  \item The following \TeX-if is set to true:
    {
      \let\ifpgfgdgraphdrawingscopeactive=\relax
      \begin{textoken}{\ifpgfgdgraphdrawingscopeactive}
        Will be true inside a graph drawing scope.
      \end{textoken}
    }
  \end{enumerate}
  The above has a number of consequences for what can happen inside a
  graph drawing scope:
  \begin{itemize}
  \item Since no nodes are actually created before the end of the
    scope, you cannot reference these nodes. Thus, you cannot write
\begin{codeexample}[code only]
\tikz [spring layout] {
  \node (a) {a};
  \node (b) {b};
  \draw (a) -- (b);
}
\end{codeexample}
    The problem is that we cannot connect |(a)| and |(b)| via a
    straight line since these nodes do not exist at that point (they
    are available only deeply inside the Lua layer).
  \item In order to create edges between nodes inside a graph drawing
    scope, you need to call the |\pgfgdedge| command, described below.
  \end{itemize}
\end{command}


\begin{command}{\pgfgdendscope}
  This macro is used to end a graph drawing scope. It must be given on
  the same \TeX\ grouping level as the corresponding
  |\pgfgdbeginscope|. When the macro is called, it triggers a lot of
  new calls:
  \begin{enumerate}
  \item The special treatment of newly created boxes is ended. Nodes
    are once more created normally.
  \item The effects of the \meta{code} that was inserted via the
    specification hook command also ends (provided it had no global
    effects).
  \item Then, on the Lua layer, the graph drawing algorithm is
    started. The hierarchy of layouts processed as described in
    Section~\ref{section-gd-layout-scopes} below.
  \item When the processing has finished, a so-called \emph{shipout
      phase} is started.
  \item During this phase, all nodes that were intercepted during the
    graph drawing scope get inserted into the output stream at the
    positions that were computed for them. Also, for each edge that
    was requested via |\pgfgdedge|, the macro |\pgfgdedgecallback| is
    called (see below).
  \item Finally, the shipout phase is finished, the scope is popped
    from the scope stack and the \TeX\ scope is ended. 
  \end{enumerate}
\end{command}



\subsubsection{Layout Scopes}

\label{section-gd-layout-scopes}

As described in Section~\ref{section-gd-sublayouts}, the graph drawing
engine does not always apply only a single algorithm. Rather, several
different algorithm may be applied to different parts of the
graph. How this happens, exactly, is goverened by a hierarchy of
layouts, which are setup using the following command:

TODO.


\subsubsection{Algorithm Selection}

TODO

\subsubsection{Handling of Edges}

Inside a graph drawing scope, nodes are
automatically passed down to the graph drawing engine, while for edges
a command has to be called explicitly:

\begin{command}{\pgfgdedge\marg{first node}\marg{second
      node}\marg{edge direction}\marg{edge options}\marg{edge nodes}}
  This command is used to tell the graph drawing engine that there is
  an edge between \meta{first node} and \meta{second node} in your
  graph. The ``kind'' of connection is indicated by \meta{direction},
  which may be one of the following:
  \begin{itemize}
  \item 
    |->| indicates a directed edge (also known as an arc) from
    \meta{first node} to \meta{second node}.
  \item |--| indicates an undirected edge between \meta{first node}
    and \meta{second node},
  \item |<-| indicates a directed edge from \meta{second node} to
    \meta{first node}, but with the ``additional hint'' that this is a
    ``backward'' edge. A graph drawing algorithm may  or may not take
    this hint into account.
  \item |<->| indicates a bi-directed edge between \meta{first node}
    and \meta{second node}. 
  \item |-!-| indicates that the edge from \meta{first node} to
    \meta{second node} is ``missing''. 
  \end{itemize}
  Note that in all cases, the syntatic digraph will contain an arc
  from \meta{first node} to \meta{second node}, regardless of the
  value of \meta{direction}. The \meta{direction} is ``just'' a
  ``semantic annotation''.
  
  The parameters \meta{edge options} and \meta{edge nodes} are a bit
  more tricky. When an edge between two vertices of a graph is created
  via |\pgfgdedge|, nothing is actually done immediately. After all,
  without knowing the final positions of the nodes \meta{first node}
  and \meta{second node}, there is no way of creating the actual
  drawing commands for the edge. Thus, the actual drawing of the edge
  is done only when the graph drawing algorithm is done (namely in the
  macro |\pgfgdedgecallback|, see later). 

  Because of this ``delayed'' drawing of edges, options that influence
  the edge must be retained until the moment when the edge is actually
  drawn. Parameters \meta{edge options} and \meta{edge nodes} store such
  options. 

  Let us start with \meta{edge options}. This parameter should be set
  to a list of key-value pairs like
\begin{codeexample}[code only]
/tikz/.cd, color=red, very thick, this edge must be vertical
\end{codeexample}
  Some of these options may be of interest to the graph drawing
  algorithm (like the last option) while others will 
  only be important during the drawing of edge (like the first
  option). The options that are important for the graph drawing
  algorithm must be passed to the algorithm via setting keys that have
  been declared using the handler |.edge parameter|, see
  Section~\ref{section-gd-parameters}. 

  The tricky part is that options that are of interest to the graph
  drawing algorithm must be executed \emph{before} the algorithm starts,
  but the options as a whole are usually only executed during the
  drawing of the edges, which is \emph{after} the algorithm has finished.
  To overcome this problem, the following happens:

  The options in \meta{edge options} are executed ``tentatively'' inside
  |\pgfgdedge|. However, this execution is done in a ``heavily guarded
  sandbox'' where all effects of the options (like changing the
  color or the line width) do not propagate beyond the sandbox. Only
  the changes of the graph drawing edge parameters leave the
  sandbox. These parameters are then passed down to the graph drawing
  engine.

  Later, when the edge is drawn using |\pgfgdedgecallback|, the
  options \meta{edge options} are available once more and then they
  are executed normally.

  Note that when the options in \meta{edge options} are executed, no
  path is preset. Thus, you typically need to start it with, say,
  |/tikz/.cd|. Also note that the sandbox is not perfect and changing
  global values will have an effect outside the sandbox. Indeed,
  ``putting things in a sandbox'' just means that the options are
  executed inside a \TeX\ scope inside an interrupted path inside a
  \TeX\ box that is thrown away immediately. 
  
  The text in \meta{edge nodes} is some ``auxilliary'' text that is
  simply stored away and later directly to |\pgfgdedgecallback|. This
  is used for instance by \tikzname\ to store its node labels. 
\end{command}


\begin{command}{\pgfgdsetedgecallback\marg{macro}}
  This command allows you to change the \meta{macro} that gets called
  form inside the graph drawing engine at the end of the creation of a
  graph, when the nodes have been positioned. The \meta{macro} will be 
  called once for each edge with the following parameters:
  \begin{quote}
    \meta{macro}\marg{first node}\marg{second
      node}\marg{direction}\marg{edge options}\marg{edge
      nodes}\marg{algorithm-generated options}\marg{bend information} 
  \end{quote}

  The first five parameters are the original values that were passed
  down to the |\pgfgdedge| command.
  
  The \meta{algorithm-generated options} have been ``computed by the
  algorithm''. For instance, an algorithm might have determined, say,
  flow capacities for edges and it might now wish to communicate this
  information back to the upper layers. These options should be
  executed with the path |/graph drawing|.
  
  The parameter \meta{bend information} contains
  algorithmically-computed information concerning how the 
  edge should bend. Currently, this will be a text like
  |(10pt,20pt)--(30pt,40pt)| in \tikzname-syntax, but this may change
  to make things more portable.

  The default \meta{macro} simply draws a line between the nodes. When
  the |graphdrawing| library of the \tikzname\ layer is loaded, a more
  fancy \meta{macro} is used that takes all of the parameters into
  account.
\end{command}



\subsubsection{Layout Keys}

Users of the graph drawing engine typically use a key like
|tree layout| to specify a graph drawing algorithm. They call the
macro |\pgfgdbeginscope| internally. All of these keys can (only) be
used as an option when a \tikzname\ scope is started. Thus, you can
pass them to |\tikz|, to |{tikzpicture}|, to |\scoped|, to |{scope}|,
to |graph|, and to |{graph}|. For instance, the |tree layout| option can
be used in the following ways:
\begin{codeexample}[]
\tikz [tree layout] \graph        {1 -> {b,c}};  
\tikz \graph [tree layout]        {2 -> {b,c}};
\tikz \path graph [tree layout]   {3 -> {b,c}};

\begin{tikzpicture}[tree layout]
  \graph                          {4 -> {b,c}};
\end{tikzpicture}

\begin{tikzpicture}
  \scoped [tree layout] \graph    {5 -> {b,c}};
    
  \begin{scope}[tree layout, xshift=1.5cm]
    \graph                        {6 -> {b,c}};
  \end{scope}
\end{tikzpicture}
\end{codeexample}

You can \emph{not} use layout keys with a single
node or on a path. In particular, to typeset a tree given in the
|child| syntax somewhere inside a |{tikzpicture}|, you must prefix
it with the |\scoped| command:
\begin{codeexample}[]
\begin{tikzpicture}
  \scoped [tree layout]
    \node {root}
    child { node {left child} }
    child { node {right child} };
\end{tikzpicture}
\end{codeexample}
  Naturally, the above could have been written more succinctly as
\begin{codeexample}[]
\tikz [tree layout]
  \node {root}
  child { node {left child} }
  child { node {right child} };
\end{codeexample}
  Or even more succinctly:
\begin{codeexample}[]
\tikz \graph [tree layout] { root -- {left child, right child} };
\end{codeexample}

  In detail, adding a layout key to a scope has the following effects:
  \begin{itemize}
  \item The basic layer is informed, using the
    |execute at begin scope| key, that the current scope will contain
    nodes that should be positioned by a graph drawing engine. Which
    algorithm is used depends on the value of the |algorithm| key.
  \item If the |graphs| library has been loaded, the default
    positioning mechanisms of this library are switched off, leaving
    the positioning to the graph drawing engine. Also, when an edge is
    created by the |graphs| library, this is signalled to the graph
    drawing library. (To be more precise: The keys |new ->| and so on
    are redefined so that they call |\pgfgdedge| instead of creating
    an edge.
  \item The |edge| path command is modified so that it also calls
    |\pgfgdedge| instead of immediately creating any edges.
  \item The |edge from parent| path command is modified so that is
    also calls |\pgfgdedge|.
  \item The keys |append after command| and |prefix after command|
    keys are modified so that they are executed only via
    |late options| when the node has ``reached its final parking
    position''. 
  \end{itemize}


\subsection{Parameters}
\label{section-gd-parameters}

When a graph drawing algorithm starts working, a set of options,
called ``graph drawing parameters'' or just ``parameters'' (in
constrast to ``options'') in the following, can influence
the way the algorithm works. For instance, a graph drawing parameter
might be the average distance between vertices which the algorithm
should take into account. Another example might be the fact the
certain nodes are special nodes and that a certain edge should have
a large label.

These graph drawing parameters are different from ``usual'' \pgfname\
options: An alogrithmic parameter influences the way the algorithm
works, while usual options influence the way the result
looks like. For instance, the fact that a node is red is not a
graph drawing parameter, while the shape of a node might be an graph
drawing parameter. 


TODO: Document

Specifying the set of graph drawing parameters for a given graph or
node or edge works as follows: When the graph drawing engine is
started for a graph (using |\pgfgdbeginscope|), a snapshot is taken of
all graph drawing graph parameters currently setup at this
point. Similarly, when a node is created inside such a scope (using
|\pgfnode|), a snapshot is taken of the set of all graph drawing node
parameters in force at this point and stored together with
the node. Finally, when an edge is created (using |\pgfgdedge|), a
snapshot of the setting of the graph drawing edge parameters is
taken. 

The options set in this way can later be access from the Lua code via
the |options| fields of digraphs, vertices, and arcs. (For arcs, the
situation is a bit more involved, but most of the time, you can access
the option as expected.)



\subsection{Collections of Nodes and Edges}

\label{section-gd-collections}

A \emph{collection} is a pair consisting of a set of nodes and a set of
edges. Since the edges need not be between the nodes of the same
collection, collections are not subgraphs, but something more
``loose'' -- namely collections.

Collections are grouped into ``kinds'', that is, all collections
having the same kind will be grouped together. On the Lua layer, all
collections of the same kind can be accessed directly via a single
array; on the \pgfname\ layer, for each kind a separate key is available
to indicate that a node or an edge belongs to a collection. 

Collections serve two purposes: First, they can be seen as ``hints''
to graph drawing algorithms that certain nodes and/or edges ``belong
together''. For instance, collections of kind |same layer| are used
by the Sugiyama algorithm to group together nodes that should appear
at the same height of the output. Second, since collections are also
passed back to the \pgfname\ layer in a postprocessing step, they can be
used to render complicated concepts such as hyperedges (which are
just collections of nodes, after all) or subgraphs.

TODO: Describe collection keys.

  The following shows the use of the |same layer| kind:
  \begin{codeexample}[]
\tikz [rounded corners] \graph [layered layout] {
    1972 -> 1976 -> 1978 -> 1980 -> 1982 -> 1984 -> 1986 -> 1988 -> 1990 -> future;
        
    { [same layer] 1972, Thompson };
    { [same layer] 1976, Mashey, Bourne },
    { [same layer] 1978, Formshell, csh },
    { [same layer] 1980, esh, vsh },
    { [same layer] 1982, ksh, "System-V" },
    { [same layer] 1984, v9sh, tcsh },
    { [same layer] 1986, "ksh-i" },
    { [same layer] 1988, KornShell ,Perl, rc },
    { [same layer] 1990, tcl, Bash },
    { [same layer] "future", POSIX, "ksh-POSIX" },

    Thompson -> { Mashey, Bourne, csh -> tcsh},
    Bourne -> { ksh, esh, vsh, "System-V", v9sh -> rc, Bash},
    { "ksh-i", KornShell } -> Bash,
    { esh, vsh, Formshell, csh } -> ksh,
    { KornShell, "System-V" } -> POSIX,
    ksh -> "ksh-i" -> KornShell -> "ksh-POSIX",
    Bourne -> Formshell,
    
    { [edge={draw=none}]
      Bash -> tcl,
      KornShell -> Perl
    }
  };
\end{codeexample}



\subsection{Events}
\label{section-gd-events}

\emph{Events} are used to pass information from the parser about the
syntactic structure of a graph to graph drawing algorithms. Consider,
for instance, a graph that is actually a tree in which some node
``misses'' its first child. In this case, the information that the
child is missing is neither part of any node (because the node is
missing, after all) nor is it an option of the whole graph. However,
events are created by the parser the allow an algorithm to reconstruct
the fact that the child is missing. Naturally, graph drawing
algorithms may choose to ignore events and most will.

Most of the creation and handling of events is done automatically. The
only reason you might wish to use the following commands is when you
write a ``parser extension'' together with a new graph drawing
algorithm. For instance, you might come up with new options that, when
used, trigger events.

\begin{command}{\pgfgdevent\marg{kind}\marg{parameter}}
  Creates a new |Event| object on the Lua layer whose |kind| field is
  set to \meta{kind} and the |parameters| field to \meta{parameter}.
  You must be inside a graph drawing scope to use this command.
\end{command}

\begin{command}{\pgfgdbegineventgroup\marg{parameter}}
  Starts an event group. This just means that an |Event| of kind
  |begin| is created with the given \meta{parameter}.
\end{command}

\begin{command}{\pgfgdendeventgroup}
  Ends an event group. This is done by adding an event of kind |end|
  without any parameters to the event string.
\end{command}

\begin{command}{\pgfgdeventgroup\marg{parameters}}
  Starts an event group just like |\pgfgdbegineventgroup|, but adds a
  corresponding closing |end| event at the end of the current \TeX\
  group (using |\aftergroup|).
\end{command}




\subsection{Interface to Display Layers}

\includeluadocumentationof{pgf.gd.interface.InterfaceCore}
\includeluadocumentationof{pgf.gd.interface.InterfaceToDisplay}

\subsection{Bindings}

\includeluadocumentationof{pgf.gd.bindings.Binding}
\includeluadocumentationof{pgf.gd.bindings.BindingToPGF}
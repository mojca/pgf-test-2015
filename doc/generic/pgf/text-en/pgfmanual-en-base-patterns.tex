% Copyright 2006 by Till Tantau
%
% This file may be distributed and/or modified
%
% 1. under the LaTeX Project Public License and/or
% 2. under the GNU Free Documentation License.
%
% See the file doc/generic/pgf/licenses/LICENSE for more details.


\section{Patterns}

\label{section-patterns}

\subsection{Overview}

There are many ways of filling a path. First, you can fill it using a
solid color and this is also the fastest method. Second, you can also
fill it using a shading, which means that the color changes smoothly
between two (or more) different colors. Third, you can fill it using a
tiling pattern and it is explained in the following how this is done.

A tiling pattern can be imagined as a rectangular tile (hence the
name) on which a small picture is painted. There is not a single tile,
but (conceptually) an infinite number of tiles, all showing the same
picture, and these tiles are arranged horizontally and vertically to
fill the plane. When you use a tiling pattern to fill a path, what
happens is that the path clips out a ``window'' through which we see
part of this infinite plane.

Patterns come in two versions: \emph{inherently colored patterns} and
\emph{form-only patterns}. (These are often called ``color patterns''
and ``uncolored patterns,'' but these names are misleading since
uncolored patterns do have a color and the color changes. As I said,
the name is misleading\dots) An inherently colored pattern is just a
colored tile like, say, a red star with a black outline. A form-only
pattern can be imagined as a tile that is a kind of rubber stamp. When
this pattern is used, the stamp is used to print copies of the stamp
picture onto the plane, but we can use a different stamp color each
time we use a form-only pattern.

\pgfname\ provides a special support for patterns. You can declare a
pattern and then use it very much like a fill color. \pgfname\
directly maps patterns to the pattern facilities of the underlying
graphic languages (PostScript, \textsc{pdf}, and \textsc{svg}). This
means that filling a path using a pattern will be nearly as fast as if
you used a uniform color.

There are a number of pitfalls and restrictions when using
patterns. First, once a pattern has been declared, you cannot change
it anymore. In particular, it is not possible to enlarge it or change
the line width. Such flexibility would require that the repeating of
the pattern were not done by the graphic language, but on the
\pgfname\ level. This would make patterns orders of magnitude slower
to produce and to render. However, \pgfname{} does provide a
more-or-less successful emulation of ``mutable'' patterns, although
internally, a new (fixed) instance of a pattern is declared when
the parameters of a pattern change.

Second, the phase of patterns is not well-defined, that is, it is not
clear where the origin of the ``first'' tile is. To be more precise,
PostScript and \textsc{pdf} on the one hand and \textsc{svg} on the
other hand define the origin differently. PostScript and \textsc{pdf}
define a fixed origin that is independent of where the path lies. This
has the highly desirable effect that if you use the same pattern to
fill multiple paths, the outcome is the same as if you had filled a single path consisting of the union of all these paths. By
comparison, \textsc{svg} uses the upper-left (?) corner of the path to
be filled as the origin. However, the \textsc{svg} specification is a
bit vague on this question.


\subsection{Declaring a Pattern}

Before a pattern can be used, it must be declared. The following
command is used for this:

\begin{command}{\pgfdeclarepatternformonly%
	\opt{\oarg{variables}}%
	\marg{name}%
	\marg{bottom left}%
	\marg{top right}%
	\marg{tile size}%
	\marg{code}}

	This command declares a new form-only pattern. The \meta{name} is a
  name for later reference. The two parameters \meta{lower left} and
  \meta{upper right} must describe a bounding box that is large enough
  to encompass the complete tile.

  The size of a tile is given by \meta{tile size}, that is, a tile is
  a rectangle whose lower left   corner is the origin and whose upper
  right corner is given by \meta{tile size}. This might make you
  wonder why the second and third parameters are needed. First, the
  bounding box might be smaller than the tile size if the tile is
  larger than the picture on the tile. Second, the bounding box might
  be bigger, in which case the picture will ``bleed'' over the tile.

  The \meta{code} should be \pgfname\ code than can be protocolled. It
  should not contain any color code.


\begin{codeexample}[]
\pgfdeclarepatternformonly{stars}
{\pgfpointorigin}{\pgfpoint{1cm}{1cm}}
{\pgfpoint{1cm}{1cm}}
{
  \pgftransformshift{\pgfpoint{.5cm}{.5cm}}
  \pgfpathmoveto{\pgfpointpolar{0}{4mm}}
  \pgfpathlineto{\pgfpointpolar{144}{4mm}}
  \pgfpathlineto{\pgfpointpolar{288}{4mm}}
  \pgfpathlineto{\pgfpointpolar{72}{4mm}}
  \pgfpathlineto{\pgfpointpolar{216}{4mm}}
  \pgfpathclose%
  \pgfusepath{fill}
}
\begin{tikzpicture}
  \filldraw[pattern=stars] (0,0)   rectangle (1.5,2);
  \filldraw[pattern=stars,pattern color=red]
                           (1.5,0) rectangle (3,2);
\end{tikzpicture}
\end{codeexample}

	The optional argument \meta{variables} consists of a comma
	separated	list of macros,	registers or keys, representing the
	parameters of the pattern that may vary. If a variable is a key,
	then the full path name must be used (specifically, it must start
	with |/|).
	As an example, a list might look like the following:
	|\mymacro,\mydimen,/pgf/my key|. Note that macros and keys should
	be ``simple''. They should only store values in themselves.
	
	The effect of \meta{variables}, is the following:
  Normally, when this argument is empty, once a pattern has been
  declared, it becomes ``frozen''. This means that it is not possible
  to enlarge the pattern or change the line width later on.
  By specifying \meta{variables}, no pattern is actually created.
  Instead, the arguments are stored away
  (so the macros,	registers or keys do not have to be defined in advance).

  When the fill pattern is set, \pgfname{} checks if the pattern has
  already been created with the \meta{variables} set to their current
  values (\pgfname{} is usually ``smart enough'' to distinguish between
  macros, registers and keys). If so, this already-declared-pattern
  is used as the fill pattern.
  If not, a new instance of the pattern (which will have a
  unique internal name) is declared using the current values of
  \meta{variables}. These values are then saved and the fill pattern
  set accordingly.
	
	The following shows an example of a pattern which varies
	according to the values of the macro |\size|, the key |/tikz/radius|,
	and the \TeX{} dimension |\thickness|.

\begin{codeexample}[]
\pgfdeclarepatternformonly[/tikz/radius,\thickness,\size]{rings}
{\pgfpoint{-0.5*\size}{-0.5*\size}}
{\pgfpoint{0.5*\size}{0.5*\size}}
{\pgfpoint{\size}{\size}}
{
  \pgfsetlinewidth{\thickness}
  \pgfpathcircle\pgfpointorigin{\pgfkeysvalueof{/tikz/radius}}
  \pgfusepath{stroke}
}
\newdimen\thickness
\tikzset{
  radius/.initial=4pt,
  size/.store in=\size, size=20pt,
  thickness/.code={\thickness=#1},
  thickness=0.75pt
}
\begin{tikzpicture}[rings/.style={pattern=rings}]
  \filldraw [rings, radius=2pt, size=6pt]      (0,0)   rectangle +(1.5,2);
  \filldraw [rings, radius=2pt, size=8pt]      (2,0)   rectangle +(1.5,2);
  \filldraw [rings, radius=6pt, thickness=2pt] (0,2.5) rectangle +(1.5,2);
  \filldraw [rings, radius=8pt, thickness=4pt] (2,2.5) rectangle +(1.5,2);
\end{tikzpicture}
\end{codeexample}
\end{command}


\begin{command}{\pgfdeclarepatterninherentlycolored
		\opt{\oarg{variables}}
    \marg{name}
    \marg{lower left}
    \marg{upper right}
    \marg{tile size}
    \marg{code}}
  This command works like |\pgfdeclarepatternuncolored|, only the
  pattern will have an inherent color. To set the color, you should
  use \pgfname's color commands, not the |\color| command, since this
  fill is not protocolled.

\begin{codeexample}[]
\pgfdeclarepatterninherentlycolored{green stars}
{\pgfpointorigin}{\pgfpoint{1cm}{1cm}}
{\pgfpoint{1cm}{1cm}}
{
  \pgfsetfillcolor{green!50!black}
  \pgftransformshift{\pgfpoint{.5cm}{.5cm}}
  \pgfpathmoveto{\pgfpointpolar{0}{4mm}}
  \pgfpathlineto{\pgfpointpolar{144}{4mm}}
  \pgfpathlineto{\pgfpointpolar{288}{4mm}}
  \pgfpathlineto{\pgfpointpolar{72}{4mm}}
  \pgfpathlineto{\pgfpointpolar{216}{4mm}}
  \pgfpathclose%
  \pgfusepath{stroke,fill}
}
\begin{tikzpicture}
  \filldraw[pattern=green stars] (0,0) rectangle (3,2);
\end{tikzpicture}
\end{codeexample}
\end{command}




\subsection{Setting a Pattern}

Once a pattern has been declared, it can be used.

\begin{command}{\pgfsetfillpattern\marg{name}\marg{color}}
  This command specifies that paths that are filled should be filled
  with the ``color'' by the pattern \meta{name}. For an inherently
  colored pattern, the \meta{color} parameter is ignored. For
  form-only patterns, the \meta{color} parameter specifies the color
  to be used for the pattern.
\begin{codeexample}[]
\begin{tikzpicture}
  \pgfsetfillpattern{stars}{red}
  \filldraw (0,0) rectangle (1.5,2);

  \pgfsetfillpattern{green stars}{red}
  \filldraw (1.5,0) rectangle (3,2);
\end{tikzpicture}
\end{codeexample}
\end{command}



%%% Local Variables:
%%% mode: latex
%%% TeX-master: "pgfmanual"
%%% End:

% Copyright 2006 by Till Tantau
%
% This file may be distributed and/or modified
%
% 1. under the LaTeX Project Public License and/or
% 2. under the GNU Free Documentation License.
%
% See the file doc/generic/pgf/licenses/LICENSE for more details.

\section{Arrows}

\tikzset{to/.tip=Computer Modern Rightarrow}

\subsection{Overview}

\tikzname\ allows you to add arrow tips to the end of
lines as in \tikz [baseline] \draw [-Computer Modern Rightarrow]
(0,.5ex) -- (3ex,.5ex); or in \tikz [baseline] \draw [-{Latex}]
(0,.5ex) -- (3ex,.5ex);. It is possible to change which arrow tips are
used ``on-the-fly,'' you can have several arrow tips in a row, and you
can change the appearance of each of them individually using a special
syntax. The following example is a perhaps slightly ``excessive''
demonstration of what you can do:
\begin{codeexample}[]
\tikz {
  \node [circle,draw] (A)              {A};
  \node [circle,draw] (B) [right=of A] {B};

  \draw [draw = blue, thick,
         arrows={
           Computer Modern Rightarrow [sep] 
         - Latex[blue!50,length=8pt,bend] Stealth[length=8pt,open,bend,sep]}]
    (A) edge [bend left=45] (B)
    (B) edge [in=-110, out=-70,looseness=8] (B);
}
\end{codeexample}

There are a number of predefined generic arrow tip kinds whose
appearance you can modify in many ways using various options. It is
also possible to define completely new arrow tip kinds, see
Section~\ref{section-arrows}, but doing this is somewhat harder than
configuring an existing kind (it is like the difference between using
a font at different sizes or faces like italics, compared to
designing a new font yourself).

In the present section, we go over the various ways in which you can
configure which particular arrow tips are \emph{used.} The glorious
details of how new arrow tips can be defined are explained in
Section~\ref{section-arrows}.


\subsection{Where and When Arrow Tips Are Placed}

In order to add arrow tips to the lines you draw, the following
conditions must be met:

\begin{enumerate}
\item You have specified that arrow tips should be added to 
  lines, using the |arrows| key or its short form.
\item You use the |draw| key (directly or indirectly) with the current
  path. 
\item The path actually has two ``end points'' (it is not
  ``closed'').
\end{enumerate}

Let us start with an introduction to the basics of the |arrows| key: 

\begin{key}{/tikz/arrows=\meta{start arrow specification}|-|\meta{end
      arrow specification}} 
  This option sets the arrow tip(s) to be used at the start and end of
  lines. An empty value as in |->| for the start indicates that no
  arrow tip should be drawn at the start.% 
  \indexoption{arrows}

  \emph{Note: Since the arrow option is so often used, you can leave
    out the text |arrows=|.} What happens is that every (otherwise
  unknown) option that contains a |-| is interpreted as an arrow specification.

\begin{codeexample}[]
\begin{tikzpicture}
  \draw[->]        (0,0)   -- (1,0);
  \draw[>-Stealth] (0,0.3) -- (1,0.3);
\end{tikzpicture}
\end{codeexample}

  In the above example, the first start specification is empty and the
  second is |>|. The end specifications are |>| for the first line and
  |Stealth| for the second line. Note that it makes a difference
  whether |>| is used in a start specification or in an end
  specification: In an end specification it creates, as one would
  expect, a pointed tip  at the end of the line. In the start
  specification, however, it creates a ``reversed'' version if this
  arrow -- which happens to be what one would expect here.

  The above specifications are very simple and only select a single
  arrow tip without any special configuration options, resulting in
  the ``natural'' versions of these arrow tips. It is also possible to
  ``configure'' arrow tips in many different ways, as explained in
  detail in Section~\ref{section-arrow-config} below by adding options
  in square brackets following the arrow tip kind:

\begin{codeexample}[]
\begin{tikzpicture}
  \draw[-{Stealth[red]}] (0,0)   -- (1,0);
\end{tikzpicture}
\end{codeexample}

  Note that in the example I had to surround the end specification by
  braces. This is necessary so that \tikzname\ does not mistake the
  closing square bracket of the |Stealth| arrow tip's options for the
  end of the options of the |\draw| command. In general, you often
  need to add braces when specifying arrow tips except for simple case
  like |->| or |<<->|, which are pretty frequent, though. When in
  doubt, say |arrows={|\meta{start spec}|-|\meta{end spec}|}|, that
  will always work.

  It is also possible to specify multiple (different) arrow tips in a
  row inside a specification, see Section~\ref{section-arrow-spec}
  below for details.
\end{key}

As was pointed out earlier, to add arrow tips to a path, the path must
have ``end points'' and not be ``closed'' -- otherwise adding arrow
tips makes little sense, after all. However, a path can actually
consist of several subpath, which may be open or not and may even
consist of only a single point (a single move-to). In this case, it is
not immediately obvious, where arrow heads should be placed. Here are
the rules that \tikzname\ uses:

\begin{enumerate}
\item If the path is empty (as in |\path ;|), no arrow tips are drawn.
\item If at least one of the subpaths of a path is closed (|cycle| is
  used somewhere or something like |circle| or |rectangle|), no arrow
  tips are drawn anywhere -- even if there are open subpaths.
\item Otherwise, if all subpaths are closed and there is at least one
  subpath, we consider only the last subpath. Arrow tips are added to
  this last subpath.
\item If this last subpath is degenerate (only a ``move-to'' as in
  |\path (0,0);| or a ``move-to'' followed by a ``line-to'' to the
  same position as in |\path (1,2) -- (1,2)|), arrow tips are drawn
  pointing upwards.
\end{enumerate}

\begin{codeexample}[]
% No path, no arrow tips:
\tikz [<->] \draw; 
\end{codeexample}
\begin{codeexample}[]
% Degenerate path, draw arrow tips (but no path, it is degenerate...)
\tikz [<->] \draw (0,0); 
\end{codeexample}
\begin{codeexample}[]
% Normal case:
\tikz [<->] \draw (0,0) -- (1,0); 
\end{codeexample}
\begin{codeexample}[]
% Two subpaths, only second gets tips
\tikz [<->] \draw (0,0) -- (1,0) (2,0) -- (3,0);
\end{codeexample}
\begin{codeexample}[]
% Two subpaths, second degenerate, but still gets tips
\tikz [<->] \draw (0,0) -- (1,0) (2,0);
\end{codeexample}
\begin{codeexample}[]
% Two subpaths, but one is closed: No tips, even though last subpath is open
\tikz [<->] \draw (0,0) circle[radius=2pt] (2,0) -- (3,0);
\end{codeexample}

\subsection{Arrow Keys: Configuring The Appearance of a Single Arrow Tip}
\label{section-arrow-config}

For standard arrow tip kinds, like |Stealth| or |Latex| or |Bar|, 
you can easily change their size, aspect ratio, color, and other
parameters. This is similar to selecting a font face from a font
family: \emph{``This text''} is not just typeset in the font 
``Computer Modern,'' but rather in ``Computer Modern, italic face,
11pt size, medium weight, black color, no underline, \dots''
Similarly, an arrow tip is not just a ``Stealth'' arrow tip, but
rather a ``Stealth arrow tip at its natural size, flexing, but not
bending along the path, miter line caps, draw and fill colors
identical to the path draw color, \dots''

Just as most programs make it easy to ``configure'' which font should
be used at a certain point in a text, \tikzname\ tries to make it easy
which configuration of an arrow tip should be used. You use
\emph{arrow keys}, where a certain parameter like the |length| of an
arrow is set to a given value using the standard key--value
syntax. You can provide several arrow keys following an arrow tip kind
in  an arrow tip specification as in
|Stealth[length=4pt,width=2pt]|.

While selecting a font may be easy, \emph{designing} a new font is a
highly creative and difficult process and more often than not, not all
faces of a font are available on any given system. The difficulties
involved in designing a new arrow tip are somewhat to designing a new
letter for a font and, thus, it may also happen that not all
configuration options are actually implemented for a given arrow
tip. Naturally, for the standard arrow tips, all configuration options
are available -- but for special-purpose arrow tips it may well happen
that an arrow tip kind simply ``ignores'' some of the configurations
given by you.


\subsubsection{Size}

The most important configuration parameter of an arrow tip is
undoubtedly its size. The following two keys are the main keys that
are important in this context:

\begin{key}{/pgf/arrow key/length=\meta{dimension}| |\opt{\meta{line width factor}}%
    | |\opt{\meta{double factor}}}
  This parameter is usually the most important parameter that governs
  the size of an arrow tip: The \meta{dimension} that you provide
  dictates the distance from the ``very tip'' of the arrow to its
  ``back end'':
\begin{codeexample}[]
\tikz{
  \draw [-{Stealth[length=5mm]}] (0,0) -- (2,0);
  \draw [|<->|] (1.5,.4) -- node[above=1mm] {5mm} (2,.4);
}
\end{codeexample}

  Following the \meta{dimension}, you may put a space followed by a
  \meta{line width factor}, which must be a plain number. When you
  provide such a number, the size of the arrow tip is not just
  \meta{dimension}, but rather $\meta{dimension} + \meta{line width
    factor}\cdot\mathit{linewidth}$ where $\mathit{linewidth}$ is the
  width of the to-be-drawn path. This makes it very easy to vary the
  size of an arrow tip in accordance with the line width -- usually a
  very good idea since thicker lines will need thicker arrow tips.

  As an example, when you write |length=0pt 3|, the length of the
  arrow will be exactly three times the current line width. The
  default length of a |Latex| arrow is |length=2.8pt 3|, which means
  that for the standard line width of |0.4pt|, the length of a |Latex|
  arrow will be exactly 4pt (2.8pt plus three times |0.4pt|).

  Following the line width factor, you can additionally provide a
  \meta{double factor}, again preceded by a space. This factor is
  important\dots

  For a well-designed arrow tip kind, the specified dimension will
  \emph{exactly} equal the distance from the tip to the back end, even
  if the arrow tip is constructed by stroking a path. In other words,
  the length should take the thickness of lines into account that make
  up the drawing of the arrow tip itself:
\begin{codeexample}[]
\tikz{
  \draw [line width=.5mm, -{Stealth[length=5mm, open]}]
          (0,0) -- (2,0);
  \draw [|<->|] (1.5,.4) -- node[above=1mm] {5mm} (2,.4);
}
\end{codeexample}
  An exception to this rule occurs when the line caps and joins are
  rounded, see Section~\ref{section-arrow-key-caps} for details on
  this. 
\end{key}


\subsubsection{Swapping, Reversing, Halving}

\subsubsection{Coloring}

\subsubsection{Line Caps and Joins}
\label{section-arrow-key-caps}

\subsubsection{Bending and Flexing}


\subsection{Arrow Tip Specifications}
\label{section-arrow-spec}

\subsubsection{Syntax}

\subsubsection{Specifying Paddings}

\subsubsection{Specifying the Line End}

\subsubsection{Defining Shorthands and Styles}

\begin{key}{/tikz/>=\meta{end arrow kind}}
  This option can be used to redefine the ``standard'' arrow tip |>|. The
  idea is that different people have different ideas what arrow tip kind
  should normally be used. I prefer the arrow tip of \TeX's |\to| command
  (which is used in things like $f\colon A \to B$). Other people will
  prefer \LaTeX's standard arrow tip, which looks like this: \tikz
  \draw[-latex] (0,0) -- (10pt,1ex);. Since the arrow tip kind |>| is
  certainly the most ``natural'' one to use, it is kept free of any
  predefined meaning. Instead, you can change it by saying |>=to| to
  set the ``standard'' arrow tip kind to \TeX's arrow tip, whereas |>=latex|
  will set it to \LaTeX's arrow tip and |>=stealth| will use a
  \textsc{pstricks}-like arrow tip.

  Apart from redefining the arrow tip kind |>| (and |<| for the start),
  this option also redefines the following arrow tip kinds: |>| and
  \declareandlabel{<} as the swapped version of \meta{end arrow kind},
  and \verb!|<! and \verb!>|! as arrow tips ending with a vertical
  bar.

\begin{codeexample}[]
\begin{tikzpicture}[scale=2,ultra thick]
  \begin{scope}[>=latex]
    \draw[>->]    (0pt,3ex) -- (1cm,3ex);
    \draw[|<->|] (0pt,2ex) -- (1cm,2ex);
  \end{scope}
  \begin{scope}[>=stealth']
    \draw[>->]    (0pt,1ex) -- (1cm,1ex);
    \draw[|<->|] (0pt,0ex) -- (1cm,0ex);
  \end{scope}
\end{tikzpicture}
\end{codeexample}

\end{key}


% Copyright 2010 by Renée Ahrens, Olof Frahm, Jens Kluttig, Matthias Schulz, Stephan Schuster
% Copyright 2011 by Till Tantau
% Copyright 2011 by Jannis Pohlmann
%
% This file may be distributed and/or modified
%
% 1. under the LaTeX Project Public License and/or
% 2. under the GNU Free Documentation License.
%
% See the file doc/generic/pgf/licenses/LICENSE for more details.


\section{Introduction to Algorithmic Graph Drawing}

{\noindent\emph{by Till Tantau}}
\\

\emph{Algorithmic graph drawing} (or just \emph{graph drawing} in the
following) means that algorithms are used to decide where the nodes of
a graph are positioned on a page so that the graph ``looks nice.'' The
idea is that you, as human (or you, as a machine, if you happen to be
a machine and happen to be reading this document) just specify which
nodes are present in a graph and which edges are
present. Additionally, you may add some ``hints'' like ``this node
should be near the center'' or ``this edge is pretty important.'' You
do \emph{not} specify where, exactly, the nodes and edges should
be. This is something you leave to a \emph{graph drawing
  algorithm}. The algorithm gets your description of the graph as an
input and then decides where the nodes should go on the page.

\begin{codeexample}[]
\tikz \graph [binary tree layout, level distance=5mm] {
  4 -- {
    3 -- 0 -- 1[second],
    10 -- {
      8 -- {
        6 -- {5,7},
        9
  } } }
};
\end{codeexample}

\begin{codeexample}[]
\tikz \graph [spring layout,
  edge quotes mid,
  edges={nodes={font=\scriptsize, fill=white, sloped, inner sep=1pt}}]
{
  1 ->["Das"] 2 ->["ist"] 3 ->["das"] 4 ->["Haus"]
  2 ->["vom" near start] 5 ->["Ni"] 4 ->["ko" near start]
  1 ->["laus", orient=right] 5;
};  
\end{codeexample}

Naturally, graph drawing is a bit of a (black?) art. There is no
``perfect'' way of drawing a graph, rather, depending on the
circumstances there are several different ways of drawing the same
graph and often it will just depend on the aesthetic sense of the
reader which layout he or she would prefer. For this reason, there are
a huge number of graph drawing algorithms ``out there'' and there are
scientific conference devoted to such algorithms, where each
year dozens of new algorithms are proposed.

Unlike the rest of \pgfname\ and \tikzname, which is implemented
purely in \TeX, the graph drawing algorithms are simply too complex to
implement them in \TeX. Instead, the programming language Lua is used
by the graph drawing library -- a programming language that has been
integrated into recent versions of \TeX. This means that (a) as a user
of the graph drawing engine you run \TeX\ on your documents
in the usual way, no external programs are called since Lua is already
integrated into \TeX\ and (b) it is pretty easy to implement new graph
drawing algorithms for \tikzname\ since Lua can be used and no \TeX\
programming knowledge is needed. 

The graph drawing engine of \tikzname\ provides two main features:
\begin{enumerate}
\item ``Users'' of the graph drawing engine can invoke the graph
  drawing algorithms often by just adding a single option to their
  picture. Here is a typical example, where the |layered layout| option
  tells \tikzname\ that the graph should be drawn (``should be layed
  out'') using a so-called ``layered graph drawing algorithm'' (what
  these are will be explained later):
\begin{codeexample}[]
\tikz [>=spaced stealth']
  \graph [layered layout, components go right top aligned, nodes=draw, edges=rounded corners]
  {
    first root -> {1 -> {2, 3,7} -> {4, 5}, 6 }, 4 -- 5;
    second root -> x -> {a -> {/,/}, b, c -> d -> {/,/} };
    third root -> child -> grandchild -> youngster -> third root;    
  };
\end{codeexample}
  Here is another example, where a different layout method is used
  that is more appropriate for trees:
\begin{codeexample}[]
\tikz [grow'=up, binary tree layout, nodes={circle,draw}]
  \node {1}
  child { node {2}
    child { node {3} }
    child { node {4}
      child { node {5} }
      child { node {6} }
    }
  }
  child { node {7}
    child { node {8}
      child[missing]
      child { node {9} }
    }
  };
\end{codeexample}
  An a final example, this time using a ``spring electrical layout''
  (whatever that might be\dots):
\begin{codeexample}[]
\tikz [spring electrical layout]
{
  \foreach \i in {1,...,6}
    \node (node \i) [fill=blue!50, text=white, circle] {\i};
    
  \draw (node 1) edge (node 2)
        (node 2) edge (node 3)
                 edge (node 4)
        (node 3) edge (node 4)
                 edge (node 5)
                 edge (node 6);
}
\end{codeexample}
  In all of the example, the positions of the nodes have only been
  computed \emph{after} all nodes have been created and the edges have
  been specified. For instance, in the last example, without the
  option |spring electrical layout|, all of the nodes would have been
  placed on top of each other.
\item The graph drawing engine is also intended to make is
  (relatively) easy to implement new graph drawing algorithms. These
  algorithms can and must be implemented in the Lua programming
  language (which is \emph{much} easier to program than \TeX\
  itself). The Lua code for a graph drawing algorithm gets an
  object-oriented model of the input graph as an input and must just
  compute the desired new positions of the nodes. The complete
  handling of passing options and configurations back-and-forth
  between the different \tikzname\ and \pgfname\ layers is handled by
  the graph drawing engine. Here is an example of a complete (but,
  admittedly, very simple) graph drawing algorithm that arranges the
  nodes in a circle of radius |30pt| (an advanced version of the
  algorithm that can be ``configured to death'' is available through
  the |circular layout| option): 
\begin{codeexample}[code only]
-- File VerySimpleDemo.lua

local VerySimpleDemo = pgf.gd.new_algorithm_class()

function VerySimpleDemo:run()
  local alpha = (2 * math.pi) / #self.ugraph.vertices
  for i,vertex in ipairs(self.ugraph.vertices) do
    vertex.pos.x = math.cos(i * alpha) * 30  -- radius is 30pt
    vertex.pos.y = math.sin(i * alpha) * 30
  end
end

return VerySimpleDemo      
\end{codeexample}

  As a caveat, the graph drawing engine comes with a library of
  functions and methods that can vastly simplify the writing of new
  graph drawing algorithms. As a typical example, when you implement a 
  graph drawing algorithm for trees, you typically require that your
  input is a tree; but you can bet that users will feed all sorts of
  graphs to your algorithm, including disjoint unions of cliques. The
  graph drawing engine offers you to say that your algorithm
  |needs_a_spanning_tree| and instead of the original graph your
  algorithm will be provided with a spanning tree of the graph on
  which it can work. There are numerous further automatic pre- and
  postprocessing steps that include orienting, anchoring, and packing
  of components, to name a few.
  
  The bottom line is that the graph drawing engine makes it easy
  to try out new graph drawing algorithms for medium sized graphs (up
  to a few hundred nodes).
\end{enumerate}

The documentation of the graph drawing engine is structured as
follows: The next section explains the graph drawing engine from
``the user's point of view''.  It is followed by details on the
graph drawing algorithms that are currently implemented in \tikzname,
see Sections~\ref{section-first-graphdrawing-library-in-manual}
to~\ref{section-last-graphdrawing-library-in-manual}.

Readers who which to implement their own graph drawing algorithms
should read Section~\ref{section-gd-own-algorithm}, which shows how a
graph drawing algorithm can be implemented and how the supporting Lua
libraries work. At the end, Section~\ref{section-gd-internals} covers
the internals of how the graph drawing engine works.  



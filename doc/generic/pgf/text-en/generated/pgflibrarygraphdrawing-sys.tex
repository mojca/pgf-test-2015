% This file has been generated from the lua sources using LuaDoc.
% To regenerate it call "make genluadoc" in
% doc/generic/pgf/version-for-luatex/en.

\begin{filedescription}{pgflibrarygraphdrawing-sys.lua}


\begin{luacommand}{{Sys:beginShipout}()}
Begins the shipout of nodes by opening a scope in PGF. 



\end{luacommand}
\begin{luacommand}{{Sys:endShipout}()}
Ends the shipout by closing the scope opened in PGF. 



See also:
\begin{itemize}
	\item[] |Sys:beginShipout() |
\end{itemize}

\end{luacommand}
\begin{luacommand}{{Sys:escapeTeXNodeName}(\meta{name})}
Adds a |not yet positionedPGFGDINTERNAL| prefix to the name of a node.  The prefix is required by PGF to place the node. Actually, when deferring the node placement, the prefix is added to avoid references to the node. 

Parameters:
\begin{parameterdescription}
	\item[\meta{name}] Name of a node to be prefixed. 
\end{parameterdescription}


Return value:
\begin{parameterdescription} 
  \item[] Name of the node prefixed with |not yet positionedPGFGDINTERNAL|. 
\end{parameterdescription}


\end{luacommand}
\begin{luacommand}{{Sys:getTeXBox}()}
Retrieves a box from the transfer box register. 



See also:
\begin{itemize}
	\item[] |putTeXBox |
\end{itemize}

\end{luacommand}
\begin{luacommand}{{Sys:getVerbose}()}
Returns whether or not verbose logging is enabled. 


Return value:
\begin{parameterdescription} 
  \item[] |true| if verbose logging is enabled. |false| otherwise. 
\end{parameterdescription}


\end{luacommand}
\begin{luacommand}{{Sys:log}(\meta{...})}
Writes log messages to the \TeX\ output, separating the parameters by spaces. 

Parameters:
\begin{parameterdescription}
	\item[\meta{...}] List of parameters to write to the \TeX\ output. 
\end{parameterdescription}



\end{luacommand}
\begin{luacommand}{{Sys:putEdge}(\meta{edge})}
Assembles and outputs the TeX command to draw an edge. 

Parameters:
\begin{parameterdescription}
	\item[\meta{edge}] Edge to generate the \TeX/\tikzname\ command for. 
\end{parameterdescription}



\end{luacommand}
\begin{luacommand}{{Sys:putTeXBox}(\meta{node},\meta{texnode},\meta{minX},\meta{minY},\meta{maxX},\meta{maxY},\meta{posX},\meta{posY})}
Saves a box from the transfer box register. 

Parameters:
\begin{parameterdescription}
	\item[\meta{node}] Node in the box.\item[\meta{texnode}] The box which contains the \TeX\ node.\item[\meta{minX}] Maximum y coordinate of the bounding box.\item[\meta{minY}] Minimum y coordinate of the bounding box.\item[\meta{posX}] X coordinate where to put the node in the output.\item[\meta{posY}] Y coordinate where to put the node in the output. 
\end{parameterdescription}



\end{luacommand}
\begin{luacommand}{{Sys:setBoxNumber}(\meta{boxregister})}
Initializes the graph drawing system by setting the box register number.  This method is called when the \tikzname\ |graphdrawing| library is loaded. 

Parameters:
\begin{parameterdescription}
	\item[\meta{boxregister}] Number of the box register used for transfering boxes of the current graph. 
\end{parameterdescription}



\end{luacommand}
\begin{luacommand}{{Sys:setVerbose}(\meta{verbose})}
Enables or disables verbose logging for the graph drawing library. 

Parameters:
\begin{parameterdescription}
	\item[\meta{verbose}] Enables verbose logging if set to |true|. 
\end{parameterdescription}



\end{luacommand}
\begin{luacommand}{{Sys:unescapeTeXNodeName}(\meta{name})}
Removes the |not yet positionedPGFGDINTERNAL| prefix from the name of a node. 

Parameters:
\begin{parameterdescription}
	\item[\meta{name}] Name of a node with the internal prefix present. 
\end{parameterdescription}


Return value:
\begin{parameterdescription} 
  \item[] Name of the node with the internal name prefix removed. 
\end{parameterdescription}


See also:
\begin{itemize}
	\item[] |Sys:escapeTeXNodeName(name) |
\end{itemize}

\end{luacommand}

\end{filedescription}
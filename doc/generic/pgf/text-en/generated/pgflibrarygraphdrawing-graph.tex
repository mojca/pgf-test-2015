% This file has been generated from the lua sources using LuaDoc.
% To regenerate it call "make genluadoc" in
% doc/generic/pgf/version-for-luatex/en.

\begin{filedescription}{pgflibrarygraphdrawing-graph.lua}


\begin{luacommand}{{Graph:\textunderscore{}\textunderscore{}tostring}()}
Returns a string representation of this graph including all nodes and edges. 


Return value:
\begin{itemize} \item[] Graph as string.  \end{itemize}


\end{luacommand}\begin{luacommand}{{Graph:addEdge}(\meta{edge})}
Adds an edge to the graph. 

Parameters:
\begin{parameterdescription}
	\item[\meta{edge}] The edge to be added. 
\end{parameterdescription}



\end{luacommand}\begin{luacommand}{{Graph:addNode}(\meta{node})}
Adds a node to the graph. 

Parameters:
\begin{parameterdescription}
	\item[\meta{node}] The node to be added. 
\end{parameterdescription}



\end{luacommand}\begin{luacommand}{{Graph:copy}()}
Creates a shallow copy of a graph.  The nodes and edges of the original graph are not preserved in the copy. 


Return value:
\begin{itemize} \item[] A shallow copy of the graph.  \end{itemize}


\end{luacommand}\begin{luacommand}{{Graph:createEdge}(\meta{nodeA},\meta{nodeB},\meta{direction},\meta{edgenodes},\meta{options},\meta{tikzoptions})}
Creates and adds a new edge to the graph. 

Parameters:
\begin{parameterdescription}
	\item[\meta{nodeA}] The first node of the new edge.\item[\meta{nodeB}] The second node of the new edge.\item[\meta{direction}] The direction of the new edge. Possible values are |Edge.UNDIRECTED|, |Edge.LEFT|, |Edge.RIGHT|, |Edge.BOTH| and |Edge.NONE| (for invisible edges).\item[\meta{edgenodes}] A string of \tikzname\ edge nodes that needs to be passed back to the \TeX layer unmodified.\item[\meta{options}] The options of the new edge.\item[\meta{tikzoptions}] A table of \tikzname\ options to be used by graph drawing algorithms to treat the edge in special ways. 
\end{parameterdescription}


Return value:
\begin{itemize} \item[] The newly created edge.  \end{itemize}


\end{luacommand}\begin{luacommand}{{Graph:deleteEdge}(\meta{edge})}
Like removeEdge, but also removes the edge from its adjacent nodes. 

Parameters:
\begin{parameterdescription}
	\item[\meta{edge}] The edge to be deleted. 
\end{parameterdescription}


Return value:
\begin{itemize} \item[] The removed edge or |nil| if it was not found in the graph.  \end{itemize}


\end{luacommand}\begin{luacommand}{{Graph:deleteNode}(\meta{node})}
Like removeNode, but also deletes all adjacent edges of the removed node.  This function also removes the deleted adjacent edges from all neighbours of the removed node. 

Parameters:
\begin{parameterdescription}
	\item[\meta{node}] The node to be deleted together with its adjacent edges. 
\end{parameterdescription}


Return value:
\begin{itemize} \item[] The removed node or |nil| if the node was not found in the graph.  \end{itemize}


\end{luacommand}\begin{luacommand}{{Graph:findNode}(\meta{name})}
If possible, looks up the node with the given name in the graph. 

Parameters:
\begin{parameterdescription}
	\item[\meta{name}] Name of the node to look up. 
\end{parameterdescription}


Return value:
\begin{itemize} \item[] The node with the given name or |nil| if it was not found in the graph.  \end{itemize}


\end{luacommand}\begin{luacommand}{{Graph:findNodeIf}(\meta{test})}
Looks up the first node for which the function \meta{test} returns |true|. 

Parameters:
\begin{parameterdescription}
	\item[\meta{test}] A function that takes one parameter (a |Node|) and returns |true| or |false|. 
\end{parameterdescription}


Return value:
\begin{itemize} \item[] The first node for which \meta{test} returns |true|.  \end{itemize}


\end{luacommand}\begin{luacommand}{{Graph:getOption}(\meta{name})}
Returns the value of the graph option \meta{name}. 

Parameters:
\begin{parameterdescription}
	\item[\meta{name}] Name of the option. 
\end{parameterdescription}


Return value:
\begin{itemize} \item[] The value of the graph option \meta{name} or |nil|.  \end{itemize}


\end{luacommand}\begin{luacommand}{{Graph:mergeOptions}(\meta{options})}
Merges the given options into the options of the graph. 

Parameters:
\begin{parameterdescription}
	\item[\meta{options}] The options to be merged. 
\end{parameterdescription}



See also:
\begin{itemize}
	\item[] |mergeTable |
\end{itemize}

\end{luacommand}\begin{luacommand}{{Graph:new}(\meta{values})}
Creates a new graph. 

Parameters:
\begin{parameterdescription}
	\item[\meta{values}] Values to override default graph settings. The following parameters can be set:\par |nodes|: The nodes of the graph.\par |edges|: The edges of the graph.\par |pos|: Initial position of the graph.\par |options|: A table of node options passed over from \tikzname. 
\end{parameterdescription}


Return value:
\begin{itemize} \item[] A newly-allocated graph.  \end{itemize}


\end{luacommand}\begin{luacommand}{{Graph:removeEdge}(\meta{edge})}
If possible, removes an edge from the graph and returns it. 

Parameters:
\begin{parameterdescription}
	\item[\meta{edge}] The edge to be removed. 
\end{parameterdescription}


Return value:
\begin{itemize} \item[] The removed edge or |nil| if it was not found in the graph.  \end{itemize}


\end{luacommand}\begin{luacommand}{{Graph:removeNode}(\meta{node})}
If possible, removes a node from the graph and returns it. 

Parameters:
\begin{parameterdescription}
	\item[\meta{node}] The node to remove. 
\end{parameterdescription}


Return value:
\begin{itemize} \item[] The removed node or |nil| if it was not found in the graph.  \end{itemize}


\end{luacommand}\begin{luacommand}{{Graph:setOption}(\meta{name},\meta{value})}
Sets the graph option \meta{name} to \meta{value}. 

Parameters:
\begin{parameterdescription}
	\item[\meta{name}] Name of the option to be changed.\item[\meta{value}] New value for the graph option \meta{name}. 
\end{parameterdescription}



\end{luacommand}\begin{luacommand}{{Graph:subGraph}(\meta{root},\meta{graph},\meta{visited})}
Returns a subgraph.  The resulting subgraph begins at the node root, excludes all nodes and edges that are marked as visited. 

Parameters:
\begin{parameterdescription}
	\item[\meta{root}] Root node where the operation starts.\item[\meta{graph}] Result graph object or |nil| if the original graph should be used as the parent graph.\item[\meta{visited}] Set of already visited nodes/edges or |nil|. This set will be modified so make sure not to use a table that you want to remain untouched. 
\end{parameterdescription}



\end{luacommand}\begin{luacommand}{{Graph:subGraphParent}(\meta{root},\meta{parent},\meta{graph})}
Creates a new subgraph with \meta{parent} marked as visited.  This function can be useful if the graph is a tree structure (and \meta{parent} is the parent node of \meta{root}). 

Parameters:
\begin{parameterdescription}
	\item[\meta{root}] Root node where the operation starts.\item[\meta{parent}] Parent of the recursion step before.\item[\meta{graph}] Result graph object or |nil| if the original graph should be used as the parent graph. 
\end{parameterdescription}



See also:
\begin{itemize}
	\item[] |subGraph |
\end{itemize}

\end{luacommand}\begin{luacommand}{{Graph:walkAux}(\meta{root},\meta{visited},\meta{removeIndex})}
Auxiliary function to walk a graph. Does nothing if no nodes exist. 

Parameters:
\begin{parameterdescription}
	\item[\meta{root}] The first node to be visited.  If nil, chooses some node.\item[\meta{visited}] Set of already visited nodes and edges. |visited[v] == true| indicates that the node or edge |v| has already been visited.\item[\meta{removeIndex}] A numeric value or |nil| that defines the order in which nodes and edges are visited while traversing the graph. |nil| results in queue behavior, |1| in stack behavior. 
\end{parameterdescription}



See also:
\begin{itemize}
	\item[] |walkDepth|\item[] |walkBreadth |
\end{itemize}

\end{luacommand}\begin{luacommand}{{Graph:walkBreadth}(\meta{root},\meta{visited})}
Returns an iterator to walk the graph in a breadth-first traversal.  The iterator returns all edges and nodes one at a time. In case only the nodes are of interest, a filter function like |iter.filter| can be used to ignore edges. 

Parameters:
\begin{parameterdescription}
	\item[\meta{root}] The first node to be visited.  If nil, chooses some node.\item[\meta{visited}] Set of already visited nodes and edges. |visited[v] == true| indicates that the node or edge |v| has already been visited. 
\end{parameterdescription}



See also:
\begin{itemize}
	\item[] |iter.filter |
\end{itemize}

\end{luacommand}\begin{luacommand}{{Graph:walkDepth}(\meta{root},\meta{visited})}
Returns an iterator to walk the graph in a depth-first traversal.  The iterator returns all edges and nodes one at a time. In case only the nodes are of interest, a filter function like |iter.filter| can be used to ignore edges. 

Parameters:
\begin{parameterdescription}
	\item[\meta{root}] The first node to be visited.  If nil, chooses some node.\item[\meta{visited}] Set of already visited nodes and edges. |visited[v] == true| indicates that the node or edge |v| has already been visited. 
\end{parameterdescription}



See also:
\begin{itemize}
	\item[] |iter.filter |
\end{itemize}

\end{luacommand}
\end{filedescription}
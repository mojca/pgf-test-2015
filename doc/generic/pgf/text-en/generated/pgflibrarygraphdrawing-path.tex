% This file has been generated from the lua sources using LuaDoc.
% To regenerate it call "make genluadoc" in
% doc/generic/pgf/version-for-luatex/en.

\begin{filedescription}{pgflibrarygraphdrawing-path.lua}


\begin{luacommand}{{Path:\textunderscore{}\textunderscore{}tostring}()}
Returns a readable string representation of the path.


Return value:
\begin{itemize} \item[] String representation of the path \end{itemize}


\end{luacommand}\begin{luacommand}{{Path:\textunderscore{}intersects}(\meta{a1},\meta{a2},\meta{b1},\meta{b2},\meta{allowedIntersections})}
Checks if the lines a1a2 and b1b2 intersect.

Parameters:
\begin{parameterdescription}
	\item[\meta{a1}] Start of the first line.\item[\meta{a2}] End of the first line.\item[\meta{b1}] Start of the second line.\item[\meta{b2}] End of the second line.\item[\meta{allowedIntersections}] A boolean table with the keys a1, a2, b1 and b2. If two or three of those values are true, the corresponding start and/or end points are allowed to match without being seen as intersection. If all four keys are true any matching of start and end points is allowed as long as the two lines are not coincedent. If three of the keys are true or start and end of a line are allowed to match, nill will be returned. If this optional parameter is not given, any matching points will be seen as intersections.
\end{parameterdescription}


Return value:
\begin{itemize} \item[] true, if lines intersect, false otherwise. If allowedIntersections contained an invalid value, nil will be returned. \end{itemize}


\end{luacommand}\begin{luacommand}{{Path:addPoint}(\meta{point},\meta{keepAbsPosition})}
Appends new point at the end of path.

Parameters:
\begin{parameterdescription}
	\item[\meta{point}] Point to be added to the path\item[\meta{keepAbsPosition}] true if the coordinates of the point are absolute
\end{parameterdescription}



\end{luacommand}\begin{luacommand}{{Path:createPath}(\meta{posStart},\meta{posEnd},\meta{keepAbsPosition})}
Adds a new segment to the path.

Parameters:
\begin{parameterdescription}
	\item[\meta{posStart}] Startposition of the new segment\item[\meta{posEnd}] Endposition of the new segment
\end{parameterdescription}



\end{luacommand}\begin{luacommand}{{Path:getLastPoint}()}
Returns last point in path.


Return value:
\begin{itemize} \item[] last point \end{itemize}


\end{luacommand}\begin{luacommand}{{Path:getLength}()}
Returns the length of the whole path.


Return value:
\begin{itemize} \item[] Length of the whole path. \end{itemize}


\end{luacommand}\begin{luacommand}{{Path:getPoints}()}
Copies the internal points of a path.


Return value:
\begin{itemize} \item[] array of points \end{itemize}


\end{luacommand}\begin{luacommand}{{Path:intersects}(\meta{path})}
Tests if the path is intersected by path.

Parameters:
\begin{parameterdescription}
	\item[\meta{path}] other path
\end{parameterdescription}



\end{luacommand}\begin{luacommand}{{Path:move}(\meta{x},\meta{y})}
Adds new point with x,y relative to last point.

Parameters:
\begin{parameterdescription}
	\item[\meta{x}] x-coordinate of the new point\item[\meta{y}] y-coordinate of the new point
\end{parameterdescription}



\end{luacommand}\begin{luacommand}{{Path:new}(\meta{values})}
Creates a new path.

Parameters:
\begin{parameterdescription}
	\item[\meta{values}] Values to be merged with the default-metatable of a path
\end{parameterdescription}


Return value:
\begin{itemize} \item[] A new path. \end{itemize}


\end{luacommand}
\end{filedescription}
% Copyright 2012 by Till Tantau
%
% This file may be distributed and/or modified
%
% 1. under the LaTeX Project Public License and/or
% 2. under the GNU Free Documentation License.
%
% See the file doc/generic/pgf/licenses/LICENSE for more details.




\section{Writing Graph Drawing Algorithms in C}
\label{section-algorithms-in-c}

\noindent{\emph{by Till Tantau}}

\bigskip

In the present section we have a look at how graph drawing
algorithms written in the C programming language (or in C++) can be
used in the graph drawing framework.

\begin{quote}
  \itshape Warning: Graph drawing algorithms written in C can be
  incredibly fast if you use the facilities of C
  correctly. \emph{However,} C code is much less portable than Lua 
  code in the sense that it has to be compiled for the specific
  platform used by the user and that it has to be linked dynamically
  during a run of the \TeX\ program. All of this in possible (and
  works, as demonstrated by the linking of the \textsc{ogdf}
  framework), but it is \emph{much} harder to get right than writing
  Lua code.

  Bottom line, you really should be using this method
  only if it is really necessary (namely, when Lua code is simply not
  fast enough).
\end{quote}

In the following, we first have a look at the general method of
establishing a connection between the Lua part of the graph drawing
framework and generic C code. In
Section~\ref{section-gd-ogdf-connection} we then explore the special
case of the Open Graph Drawing Framework and how a connection to it
can be established conveniently.


\subsection{Establishing a Connection to Generic C Code}

In previous sections we have already encountered the Lua code of the
``hello world'' of graph drawing: placing all nodes on a circle. We
will now use this algorithm as a running example to show how a graph
drawing algorithm can be implemented in generic C. Later on, we will
have a look at the special case of the Open Graph Drawing Framework
and how this framework can be connected. 


\subsubsection{The Hello World of Graph Drawing in C}

As our first example, as always, the ``hello world'' of graph drawing
simply places nodes on a circle. For this, we implemented a
function |fast_hello_world| in a file |FastSimpleDemo.c|. It starts as
follows:

\begin{codeexample}[code only]
#include <pgf/gd/interface/c/InterfaceFromC.h>
#include <math.h>

static void fast_hello_world (pgfgd_SyntacticDigraph* graph) {
  ...
}
\end{codeexample}

As we can see, we first include a special head file of a rather small
library that does all the hard work of translating between Lua and C
for us (InterfaceFromC). One of the things this library declares is
the type |pgfgd_SyntacticDigraph|. In a moment, we will see that we
can setup a key |fast simple demo layout| such that when this key is
used on the \tikzname\ layer, the function |fast_hello_world| gets
called. When it is called, the |graph| parameter will be a full
representation of the to-be-laid-out graph. We can access the fields
of thie graph and even directly modify some of its fields (in
particular, we can modify the |pos| fields of the vertices). Here is
the complete code of the algorithm:

\begin{codeexample}[code only]
static void fast_hello_world (pgfgd_SyntacticDigraph* graph) {
  double angle  = 6.28318530718 / graph->vertices.length;
  double radius = pgfgd_tonumber(graph->options, "my radius");
  
  int i;
  for (i = 0; i < graph->vertices.length; i++) {
    pgfgd_Vertex* v = graph->vertices.array[i];
    v->pos.x = cos(angle*i) * radius;
    v->pos.y = sin(angle*i) * radius;
  }
}
\end{codeexample}

That is all that is needed, the C library will take care of both
creating the |graph| object as all well as of deleting it and of
copying back the computed values of the |pos| fields of the vertices.

Our next task is to setup the key |fast simple demo layout|. We can
(and must) also do this from C, using the following code:

\begin{codeexample}[code only]
int luaopen_pgf_gd_examples_c_FastSimpleDemo (struct lua_State *state) {
  
  pgfgd_Declaration* d = pgfgd_new_key ("fast simple demo layout");
  pgfgd_key_summary          (d, "The C version of the hello world of graph drawing");
  pgfgd_key_algorithm        (d, fast_hello_world);
  pgfgd_key_add_precondition (d, "connected");
  pgfgd_key_add_precondition (d, "tree");
  pgfgd_declare              (state, d)
  pgfgd_free_key             (d);
  
  return 0;
}
\end{codeexample}

The function |luaopen_pgf_gd_examples_c_FastSimpleDemo| is the
function that will be called by Lua (we will come to that). More
important for us, at the moment, is the declaration of the key: We use
|pgfgd_new_key| to create a declaration record and then fill the
different fields using appropriate funciton calls. In particular, the
call |pgfgd_key_algorithm| allows us to link the key with a particular
C function. The |pgfgd_declare| will then pass the whole declaration
back to Lua, so the effect of the above is essentially the same as if
you had written in Lua:
\begin{codeexample}[code only]
declare {
  key = "fast simple demo layout",
  summary = "The C version of the hello world of graph drawing",
  preconditions = {
    connected = true,
    tree = true,
  },
  algorithm = {
    run =  -- something magic we cannot express in Lua
  }
}
\end{codeexample}

In our algorithm, in addition to the above key, we also use the
|my radius| key, which is a simple length key. This key, too, can be
declared on the C layer:

\begin{codeexample}[code only]
  d = pgfgd_new_key ("my radius");
  pgfgd_key_summary (d, "A radius value for the hello world of graph drawing");
  pgfgd_key_type    (d, "length");
  pgfgd_key_initial (d, "1cm");
  pgfgd_declare     (state, d);
  pgfgd_free_key    (d);
\end{codeexample}

We simply add this code to the startup function above.

Now it is time to compile and link the code. For this, you must, well,
compile it, link it against the library
|pgfgdinterfaceInterfaceFromC|, and build a shared library out of
it. Also, you must place it somewhere where Lua\TeX\ will find
it. Then, all you need to do to use it is to write in Lua 

\begin{codeexample}[code only]
require 'pgf.gd.examples.c.FastSimpleDemo'  
\end{codeexample}
or in \tikzname
\begin{codeexample}[code only]
\usegdlibrary {examples.c.FastSimpleDemo}
\end{codeexample}

This will cause Lua\TeX\ to find the shared library, load it, and then
call the function in that library with the lengthy name (the name is
always |luaopen_| followed by the path and filename with slashes
replaced by underscores. 


\subsubsection{How Graphs are Represented in C}

Currently, please see |graphdrawing/c/pgf/gd/interface/c/InterfaceFromC.h|.

%\label{section-interfacetoc}

%\includeluadocumentationof{pgf.gd.interface.InterfaceToC}



\subsection{Establishing a Connection to the Open Graph Drawing Framework}


% Copyright 2010 by Till Tantau
% Copyright 2011 by Jannis Pohlmann
%
% This file may be distributed and/or modified
%
% 1. under the LaTeX Project Public License and/or
% 2. under the GNU Free Documentation License.
%
% See the file doc/generic/pgf/licenses/LICENSE for more details.

\section{Specifying Graphs}
\label{section-library-graphs}


\subsection{Overview}

\tikzname\ offers a powerful path command for specifying how the nodes
in a graph are connected by edges and arcs: The |graph| path
command, which becomes available when you load the |graphs| library.

\begin{tikzlibrary}{graphs}
  The package must be loaded to use the |graph| path command.
\end{tikzlibrary}

In this section, by \emph{graph} we refer to a set of nodes together
with some edges (sometimes also called arcs, in case they are
directed) such as the following:

\begin{codeexample}[]
\tikz \graph { a -> {b, c} -> d };  
\end{codeexample}

\begin{codeexample}[]
\tikz \graph {
  subgraph I_nm [V={a, b, c}, W={1,...,4}];

  a -> { 1, 2, 3 };
  b -> { 1, 4 };
  c -> { 2 [>green!75!black], 3, 4 [>red]}
};
\end{codeexample}

\begin{codeexample}[]
\tikz
  \graph [nodes={draw, circle}, clockwise, radius=.5cm, empty nodes, n=5] {
    subgraph I_n [name=inner] --[complete bipartite]
    subgraph I_n [name=outer]
  };
\end{codeexample}

\begin{codeexample}[]
\tikz
  \graph [nodes={draw, circle}, clockwise, radius=.75cm, empty nodes, n=8] {
    subgraph C_n [name=inner] <->[shorten <=1pt, shorten >=1pt]
    subgraph C_n [name=outer]
  };
\end{codeexample}

\begin{codeexample}[width=6.6cm]
\tikz [>={To[sep]}, rotate=90, xscale=-1,
       mark/.style={fill=black!50}, mark/.default=]
  \graph [trie, simple, 
          nodes={circle,draw},
          edges={nodes={
              inner sep=1pt, anchor=mid,
              fill=graphicbackground}}, % yellowish background
          put node text on incoming edges]
    {
      root[mark] -> {
        a -> n -> {
          g [mark],
          f -> a -> n -> g [mark]
        },
        f -> a -> n -> g [mark],
        g[mark],
        n -> {
          g[mark],
          f -> a -> n -> g[mark]
        }
      },
      { [edges=red] % highlight one path
        root -> f -> a -> n
      }    
    };
\end{codeexample}

The nodes of a graph are normal \tikzname\ nodes, the edges are
normal lines drawn between nodes. There is nothing in the |graph|
library that you cannot do using the normal |\node| and the |edge|
commands. Rather, its purpose is to offer a concise and powerful way of
\emph{specifying} which nodes are present 
and how they are connected. The |graph| library only offers simple
methods for specifying \emph{where} the nodes should be shown, its
main strength is in specifying which nodes and edges are present in 
principle. The problem of finding ``good positions on the canvas'' for
the nodes of a graph is left to \emph{graph drawing algorithms}, which
are covered in Part~\ref{part-gd} of this manual and which
are not part of the |graphs| library; indeed, these algorithms can be
used also with graphs specified using |node| and |edge|
commands. \ifluatex
As an example, consider the above drawing of a trie, which is drawn
without using the graph drawing libraries. Its layout can be 
somewhat improved by loading the |layered| graph drawing library,
saying |\tikz[layered layout,...|, and then using Lua\TeX, resulting
in the following drawing of the same graph:
\medskip

\tikz [layered layout, >={To[sep]}, rotate=90, xscale=-1,
       mark/.style={fill=black!50}, mark/.default=]
  \graph [trie, simple, sibling distance=8mm,
          nodes={circle,draw},
          edges={nodes={
              inner sep=1pt, anchor=mid, fill=white}},
          put node text on incoming edges]
    {
      root[mark] -> {
        a -> n -> {
          g [mark],
          f -> a -> n -> g [mark]
        },
        f -> a -> n -> g [mark],
        g[mark],
        n -> {
          g[mark],
          f -> a -> n -> g[mark]
        }
      },
      { [edges=red] % highlight one path
        root -> f -> a -> n
      }    
    };
\medskip
\fi

The |graph| library uses a syntax that is quite different from the
normal \tikzname\ syntax for specifying nodes. The reason for this is
that for many medium-sized graphs it can become quite cumbersome to
specify all the nodes using |\node| repeatedly and then using a great
number of |edge| command; possibly with complicated |\foreach|
statements. Instead, the syntax of the |graph| library is loosely
inspired by the \textsc{dot} format, which is quite useful for
specifying medium-sized graphs, with some extensions on top.



\subsection{Concepts}

The present section aims at giving a quick overview of the main
concepts behind the |graph| command. The exact syntax is explained in
more detail in later sections.


\subsubsection{Concept: Node Chains}

The basic way of specifying a graph is to write down a \emph{node
  chain} as in the following example: 

\begin{codeexample}[]
\tikz [every node/.style = draw]
  \graph { foo -> bar -> blub };  
\end{codeexample}

As can be seen, the text |foo -> bar -> my node| creates three nodes,
one with the text |foo|, one with |bar| and one with the text
|blub|. These nodes are connected by arrows, which are caused by
the |->| between the node texts. Such a sequence of node texts and
arrows between them is called a \emph{chain} in the following. 

Inside a graph there can be more than one chain:

\begin{codeexample}[]
\tikz \graph {
  a -> b -> c;
  d -> e -> f;
  g -> f;
};  
\end{codeexample}

Multiple chains are separated by a semicolon or a comma (both have
exactly the same effect). As the example shows, when a node text is
seen for the second time, instead of creating a new node, a connection
is created to the already existing node.

When a node like |f| is created, both the node name and the node text
are identical by default. This is not always desirable and can be
changed by using the |as| key or by providing another text after
a slash:

\begin{codeexample}[]
\tikz \graph {
  x1/$x_1$ -> x2 [as=$x_2$, red] -> x34/{$x_3,x_4$};
  x1 -> [bend left] x34;
};  
\end{codeexample}

When you wish to use a node name that contains special symbols like
commas or dashes, you must surround the node name by quotes. This
allows you to use quite arbitrary text as a ``node name'':
\begin{codeexample}[]
\tikz \graph {
  "$x_1$" -> "$x_2$"[red] -> "$x_3,x_4$";
  "$x_1$" ->[bend left] "$x_3,x_4$";
};  
\end{codeexample}


\subsubsection{Concept: Chain Groups}

Multiple chains that are separated by a semicolon or a comma and that
are surrounded by curly braces form what will be called a \emph{chain
  group} or just a \emph{group}. A group in itself has no special
effect. However, things get interesting when you write down a node or
even a whole group and connect it to another group. In this case, the
``exit points'' of the first node or group get connected to the
``entry points'' of the second node or group:

\begin{codeexample}[]
\tikz \graph {
  a -> {
    b -> c,
    d -> e
  } -> f
};  
\end{codeexample}

Chain groups make it easy to create tree structures:

\begin{codeexample}[width=10cm]
\tikz
  \graph [grow down,
          branch right=2.5cm] {
  root -> {
    child 1,
    child 2 -> {
      grand child 1,
      grand child 2
    },
    child 3 -> {
      grand child 3
    }
  }
};
\end{codeexample}

As can be seen, the placement is not particularly nice by default, use
the algorithms from the graph drawing libraries to get a better
layout. For instance, adding |tree layout| to the above code results in the
following somewhat more pleasing rendering:
\ifluatex
\medskip

\tikz \graph [grow down, branch right=2.5cm, tree layout] {
  root -> {
    child 1,
    child 2 -> {
      grand child 1,
      grand child 2
    },
    child 3 -> {
      grand child 3
    }
  }
};
\else
(You need to use Lua\TeX\ to typeset this graphic.)
\fi

\subsubsection{Concept: Edge Labels and Styles}

When connectors like |->| or |--| are used to connect nodes or whole
chain groups, one or more edges will typically be created. These edges
can be styles easily by providing options in square brackets directly
after these connectors:

\begin{codeexample}[]
\tikz \graph {
  a ->[red] b --[thick] {c, d};
};
\end{codeexample}

Using the quotes syntax, see Section~\ref{section-label-quotes},
you can even add labels to the edges easily by putting the labels in
quotes: 

\begin{codeexample}[]
\tikz \graph {
  a ->[red, "foo"] b --[thick, "bar"] {c, d};
};
\end{codeexample}

For the first edge, the effect is as desired, however 
between |b| and the group |{c,d}| two edges are inserted and the
options |thick| and the label option |"bar"| is applied to both of
them. While this is the correct and consistent behaviour, we typically
might wish to specify different labels for the edge going from |b| to
|c| and the edge going from |b| to |d|. To achieve this effect, we can
no longer specify the label as part of the options of |--|. Rather, we
must pass the desired label to the nodes |c| and |d|, but we must
somehow also indicate that these options actually ``belong'' to the
edge ``leading to'' to nodes. This is achieved by preceding the
options with a greater-than sign:

\begin{codeexample}[]
\tikz \graph {
  a -> b -- {c [> "foo"], d [> "bar"']};
};
\end{codeexample}

Symmetrically, preceding the options by |<| causes the options and
labels to apply to the ``outgoing'' edges of the node:

\begin{codeexample}[]
\tikz \graph {
  a [< red] -> b -- {c [> blue], d [> "bar"']};
};
\end{codeexample}

This syntax allows you to easily create trees with special edge
labels as in the following example of a treap:

\begin{codeexample}[]
\tikz 
  \graph [edge quotes={fill=white,inner sep=1pt},
          grow down, branch right, nodes={circle,draw}] {
    "" -> h [>"9"] -> {
      c [>"4"] -> {
        a [>"2"],
        e [>"0"]
      },
      j [>"7"]
    }
  };
\end{codeexample}



\subsubsection{Concept: Node Sets}

When you write down some node text inside a |graph| command, a new
node is created by default unless this node has already been created
inside the same |graph| command. In particular, if a node has
already been declared outside of the current |graph| command, a new
node of the same name gets created.

This is not always the desired behaviour. Often, you may wish to make
nodes part of a graph than have already been defined prior to the use
of the |graph| command. For this, simply surround a node name by
parentheses. This will cause a reference to be created to an already
existing node:

\begin{codeexample}[]
\tikz {
  \node (a) at (0,0) {A};
  \node (b) at (1,0) {B};
  \node (c) at (2,0) {C};
  
  \graph { (a) -> (b) -> (c) };
}
\end{codeexample}

You can even go a step further: A whole collection of nodes can all be
flagged to belong to a \emph{node set} by adding the option
|set=|\meta{node set name}. Then, inside a |graph| command, you can
collectively refer to these nodes by surrounding the node set name in
parentheses: 

\begin{codeexample}[]
\tikz [new set=my nodes] {
  \node [set=my nodes, circle,    draw] at (1,1)   {A};
  \node [set=my nodes, rectangle, draw] at (1.5,0) {B};
  \node [set=my nodes, diamond,   draw] at (1,-1)  {C};
  \node (d)           [star,      draw] at (3,0)   {D};

  \graph { X -> (my nodes) -> (d) };
}
\end{codeexample}


\subsubsection{Concept: Graph Macros}

Often, a graph will consist -- at least in parts -- of standard
parts. For instance, a graph might contain a cycle of certain size or
a path or a clique. To facilitate specifying such graphs, you can
define a \emph{graph macro}. Once a graph macro has been defined, you
can use the name of the graph to make a copy of the graph part of the
graph currently being specified:

\begin{codeexample}[]
\tikz \graph { subgraph K_n [n=6, clockwise] };
\end{codeexample}

\begin{codeexample}[]
\tikz \graph { subgraph C_n [n=5, clockwise] -> mid };
\end{codeexample}

The library |graphs.standard| defines a number of such graphs,
including the complete clique $K_n$ on $n$ nodes, the complete
bipartite graph $K_{n,m}$ with shores sized $n$ and $m$, the cycle
$C_n$ on $n$ nodes, the path $P_n$ on $n$ nodes, and the independent
set $I_n$ on $n$ nodes.


\subsubsection{Concept: Graph Expressions and Color Classes}

When a graph is being constructed using the |graph| command, it is
constructed recursively by uniting smaller graphs to larger
graphs. During this recursive union process the nodes
of the graph get implicitly \emph{colored} (conceptually) and you can
also explicitly assign colors to individual nodes and even change the
colors as the graph is being specified. All nodes having the same
color form what is called a \emph{color class}.

The power of color class is that special \emph{connector operators}
allow you to add edges between nodes having certain colors. For instance,
saying |clique=red| at the beginning of a group will
cause all nodes that have been flagged as being (conceptually) ``red''
to be connected as a clique. Similarly, saying
|complete bipartite={red}{green}| will cause edges to be added
between all red and all green nodes. More advanced connectors, like
the |butterfly| connector, allow you to add edges between color
classes in a fancy manner.

\begin{codeexample}[]
\tikz [x=8mm, y=6mm, circle]
  \graph [nodes={fill=blue!70}, empty nodes, n=8] {
    subgraph I_n [name=A] --[butterfly={level=4}]
    subgraph I_n [name=B] --[butterfly={level=2}]
    subgraph I_n [name=C] --[butterfly]
    subgraph I_n [name=D] -- 
    subgraph I_n [name=E]  
  };
\end{codeexample}



\subsection{Syntax of the Graph Path Command}

\subsubsection{The Graph Command}

In order to construct a graph, you should use the |graph| path
command, which can be used anywhere on a path at any place where
you could also use a command like, say, |plot| or |--|.

\begin{command}{\graph}
  Inside a |{tikzpicture}| this is an abbreviation for |\path graph|.
\end{command}

\begin{pathoperation}{graph}{\opt{\oarg{options}}\meta{group specification}}
  When this command is encountered on a path, the construction of the
  current path is suspended (similarly to an |edge| command or a
  |node| command). In a local scope, the \meta{options} are first
  executed with the key path |/tikz/graphs| using the following
  command:
  \begin{command}{\tikzgraphsset\marg{options}}
    Executes the \meta{options} with the path prefix |/tikz/graphs|.    
  \end{command}
  Apart from the keys explained in the following, further permissible
  keys will be listed during the course of the rest of this section.

  \begin{stylekey}{/tikz/graphs/every graph}
    This style is executed at the beginning of every |graph| path
    command prior to the \meta{options}.
  \end{stylekey}

  Once the scope has been set up and once the \meta{options} have been
  executed, a parser starts to parse the \meta{group
    specification}. The exact syntax of such a group specification
  in explained in detail in
  Section~\ref{section-library-graphs-group-spec}. Basically, a group
  specification is a list of chain specifications, separated by commas
  or semicolons.

  Depending on the content of the \meta{group specification}, two
  things will happen:
  \begin{enumerate}
  \item A number of new nodes may be created. These will be inserted
    into the picture in the same order as if they had been created
    using multiple |node| path commands at the place where the |graph|
    path command was used. In other words, all nodes created in a
    |graph| path command will be painted on top of any nodes created
    earlier in the path and behind any nodes created later in the
    path. Like normal nodes, the newly created nodes always lie on top
    of the path that is currently being created (which is often
    empty, for instance when the |\graph| command is used).
  \item Edges between the nodes may be added. They are added in the
    same order as if the |edge| command had been used at the position
    where the |graph| command is being used.
  \end{enumerate}

  Let us now have a look at some common keys that may be used inside
  the \meta{options}:
  \begin{key}{/tikz/graphs/nodes=\meta{options}}
    This option causes the \meta{options} to be applied to each newly
    created node inside the \meta{group specification}.
    \begin{codeexample}[]
\tikz \graph [nodes=red] { a -> b -> c };      
    \end{codeexample}
    Multiple uses of this key accumulate.
  \end{key}
  \begin{key}{/tikz/graphs/edges=\meta{options}}
    This option causes the \meta{options} to be applied to each newly
    created edge inside the \meta{group specification}.
    \begin{codeexample}[]
\tikz \graph [edges={red,thick}] { a -> b -> c };      
    \end{codeexample}
    Again, multiple uses of this key accumulate.
  \end{key}
  \begin{key}{/tikz/graphs/edge=\meta{options}}
    This is an alias for |edges|.
  \end{key}
  
  \begin{key}{/tikz/graphs/edge node=\meta{node specification}}
    This key specifies that the \meta{node specification} should be
    added to each newly created edge as an implicitly placed node. 
    \begin{codeexample}[]
\tikz \graph [edge node={node [red, near end] {X}}] { a -> b -> c };      
    \end{codeexample}
    Again, multiple uses of this key accumulate.
    \begin{codeexample}[]
\tikz \graph [edge node={node [near end] {X}},
              edge node={node [near start] {Y}}] { a -> b -> c };      
    \end{codeexample}
  \end{key}
  
  \begin{key}{/tikz/graphs/edge label=\meta{text}}
    This key is an abbreviation for
    |edge node=node[auto]{|\meta{text}|}|. The net effect is that the
    |text| is placed next to the newly created edges.
    \begin{codeexample}[]
\tikz \graph [edge label=x] { a -> b -> {c,d} };      
    \end{codeexample}
  \end{key}
  
  \begin{key}{/tikz/graphs/edge label'=\meta{text}}
    This key is an abbreviation for
    |edge node=node[auto,swap]{|\meta{text}|}|. 
    \begin{codeexample}[]
\tikz \graph [edge label=out, edge label'=in]
  { subgraph C_n [clockwise, n=5] };      
    \end{codeexample}
  \end{key}  
\end{pathoperation}


\subsubsection{Syntax of Group Specifications}
\label{section-library-graphs-group-spec}

A \meta{group specification} inside a |graph| path command has the
following syntax:
\begin{quote}
  |{|\opt{\oarg{options}}\meta{list of chain specifications}|}|
\end{quote}
The \meta{chain specifications} must contain chain specifications,
whose syntax is detailed in the next section, separated by either
commas or semicolons; you can freely mix them.
It is permissible to use empty lines (which are mapped to |\par|
commands internally) to structure the chains visually, they are simply
ignored by the parser. 

In the following example, the group specification consists of three
chain specifications, namely of |a -> b|, then |c| alone, and finally
|d -> e -> f|:
\begin{codeexample}[]
\tikz \graph {
  a -> b,
  c;

  d -> e -> f
};
\end{codeexample}
The above has the same effect as the more compact group specification
|{a->b,c,d->e->f}|.

Commas are used to detect where chain specifications end. However, you
will often wish to use a comma also inside the options of a single
node like in the following example:

\begin{codeexample}[]
\tikz \graph {
  a [red, draw] -> b [blue, draw],
  c [brown, draw, circle]
};
\end{codeexample}

Note that the above example works as expected: The first comma inside
the option list of |a| is \emph{not} interpreted as the end of the
chain specification ``|a [red|''. Rather, commas inside square
brackets are ``protected'' against being interpreted as separators of
group specifications.

The \meta{options} that can be given at the beginning of a group
specification are local to the group. They are executed with the path
prefix |/tikz/graphs|. Note that for the outermost group specification
of a graph it makes no difference whether the options are passed to
the |graph| command or whether they are given at the beginning of this
group. However, for groups nested inside other groups, it does make a
difference:

\begin{codeexample}[]
\tikz \graph {
  a -> { [nodes=red] % the option is local to these nodes:
    b, c
  } ->
  d
};
\end{codeexample}

\medskip
\textbf{Using foreach.}
There is special support for the |\foreach| statement inside groups:
You may use the statement inside a group
specification at any place where a \meta{chain specification} would
normally go. In this case, the |\foreach| statement is executed and
for each iteration the content of the statement's body is treated and
parsed as a new chain specification.

\begin{codeexample}[]
\tikz \graph [math nodes, branch down=5mm] {
  a -> { 
    \foreach \i in {1,2,3} {
      a_\i -> { x_\i, y_\i }
    },
    b
  }
};
\end{codeexample}

\medskip
\textbf{Using macros.}
In some cases you may wish to use macros and \TeX\ code to compute
which nodes and edges are present in a group. You cannot use macros in
the normal way inside a graph specification since the parser does not
expand macros as it scans for the start and end of groups and node
names. Rather, only after commas, semicolons, and hyphens have already
been detected and only after all other parsing decisions have been
made will macros be expanded. At this point, when a macro expands to,
say |a,b|, this will not result in two nodes to be created since the
parsing is already done. For these reasons, a special key is needed to
make it possible to ``compute'' which nodes should be present in a
group.

\begin{key}{/tikz/graph/parse=\meta{text}}
  This key can only be used inside the \meta{options} of a \meta{group
    specification}. Its effect is that the \meta{text} is inserted at
  the beginning of the current group as if you had entered it there.
  Naturally, it makes little sense to just write down some static
  \meta{text} since you could just as well directly place it at the
  beginning of the group. The real power of this command stems from
  the fact that the keys mechanism allows you to say, for instance,
  |parse/.expand once| to insert the text stored in some macro into
  the group.
\begin{codeexample}[]
\def\mychain{ a -> b -> c; }  
\tikz \graph { [parse/.expand once=\mychain] d -> e };
\end{codeexample}
  In the following, more fancy example we use a loop to create a chain
  of dynamic length.
\begin{codeexample}[]
\def\mychain#1{
  \def\mytext{1}
  \foreach \i in {2,...,#1} {
    \xdef\mytext{\mytext -> \i}
  }
}
\tikzgraphsset{my chain/.style={
    /utils/exec=\mychain{#1},
    parse/.expand once=\mytext}
}
\tikz \graph { [my chain=4] };
\end{codeexample}
  Multiple uses of this key accumulate, that is, all the \text{text}s
  given in the different uses is inserted in the order it is given.
\end{key}


\subsubsection{Syntax of Chain Specifications}

A \meta{chain specification} has the following syntax: It consists of
a sequence of \meta{node specifications}, where subsequent node 
specifications are separated by \meta{edge specifications}. Node
specifications, which typically consist of some text, are discussed in
the next section in more detail. They normally represent a single node
that is either newly created or exists already, but they may also
specify a whole set of nodes.

An \meta{edge specification} specifies \emph{which} of the node(s) to
the left of the edge specification should be connected to which
node(s) to the right of it and it also specifies in which direction
the connections go. In the following, we only discuss how the
direction is chosen, the powerful mechanism behind choosing which
nodes should be connect is detailed in 
Section~\ref{section-library-graphs-color-classes}.

The syntax of an edge specification is always one of the following
five possibilities: 

\begin{quote}
  |->| \opt{\oarg{options}}\\
  |--| \opt{\oarg{options}}\\
  |<-| \opt{\oarg{options}}\\
  |<->| \opt{\oarg{options}}\\
  |-!-| \opt{\oarg{options}}
\end{quote}

The first four correspond to a directed edge, an undirected edge, a
``backward'' directed edge, and a bidirected edge, respectively. The
fifth edge specification means that there should be no edge (this
specification can be used together with the |simple| option to remove
edges that have previously been added, see
Section~\ref{section-library-graphs-simple}). 

Suppose the nodes \meta{left nodes} are to the left of the \meta{edge
  specification} and \meta{right nodes} are to the right and suppose
we have written |->| between them. Then the following happens:
\begin{enumerate}
\item The \meta{options} are executed (inside a local scope) with the
  path |/tikz/graphs|.  These options may setup the connector algorithm
  (see below) and may also use keys like |edge| or |edge label| to
  specify how the edge should look like. As a convenience, whenever an
  unknown key is encountered for the path |/tikz/graphs|, the key is
  passed to the |edge| key. This means that you can directly use
  options like |thick| or |red| inside the \meta{options} and they
  will apply to the edge as expected.
\item The chosen connector algorithm, see 
  Section~\ref{section-library-graphs-color-classes}, is used to
  compute from which of the \meta{left nodes} an edge should lead to
  which of the \meta{right nodes}. Suppose that $(l_1,r_1)$, \dots,
  $(l_n,r_n)$ is the list of node pairs that result (so there should
  be an edge between $l_1$ and $r_1$ and another edge between $l_2$
  and $r_2$ and so on).
\item For each pair $(l_i,r_i)$ an edge is created. This is done by
  calling the following key (for the edge specification |->|, other
  keys are executed for the other kinds of specifications):
  \begin{key}{/tikz/graphs/new ->=\marg{left node}\marg{right node}\marg{edge options}\marg{edge nodes}}
    This key will be called for a |->| edge specification with the
    following four parameters: 
    \begin{enumerate}
    \item \meta{left node} is the name of the ``left'' node, that is,
      the name of $l_i$.
    \item \meta{right node} is the name of the right node.
    \item \meta{edge options} are the accumulated options from all
      calls of |/tikz/graph/edges| in groups that surround the edge
      specification.
    \item \meta{edge nodes} is text like |node {A} node {B}| that
      specifies some nodes that should be put as labels on the edge
      using \tikzname's implicit positioning mechanism.
    \end{enumerate}
    By default, the key executes the following code:
    \begin{quote}
      |\path [->,every new ->]|\\
      \hbox{}\quad|(|\meta{left node}|\tikzgraphleftanchor) edge [|%
      \meta{edge options}|]| \meta{edge nodes}||\\
      \hbox{}\quad|(|\meta{right node}|\tikzgraphrightanchor);|
    \end{quote}
    You are welcome to change the code underlying the key.
    \begin{stylekey}{/tikz/every new ->}
      This key gets executed by default for a |new ->|.
    \end{stylekey}
  \end{key}
  \begin{key}{/tikz/graphs/left anchor=\meta{anchor}}
    This anchor is used for the node that is to the left of an edge
    specification. Setting this anchor to the empty string means that
    no special anchor is used (which is the default). The
    \meta{anchor} is stored in the macro |\tikzgraphleftanchor| with a
    leading dot.
    \begin{codeexample}[]
\tikz \graph {
  {a,b,c} -> [complete bipartite] {e,f,g}
};
    \end{codeexample}
    \begin{codeexample}[]
\tikz \graph [left anchor=east, right anchor=west] {
  {a,b,c} -- [complete bipartite] {e,f,g}
};
    \end{codeexample}
  \end{key}
  \begin{key}{/tikz/graphs/right anchor=\meta{anchor}}
    Works like |left anchor|, only for |\tikzgraphrightanchor|.
  \end{key}
  For the other three kinds of edge specifications, the following keys
  will be called:
  \begin{key}{/tikz/graphs/new --=\marg{left node}\marg{right node}\marg{edge options}\marg{edge nodes}}
    This key is called for |--| with the same parameters as above. The
    only difference in the definition is that in the |\path| command
    the |->| gets replaced by |-|.
    \begin{stylekey}{/tikz/every new --}
    \end{stylekey}
  \end{key}
  \begin{key}{/tikz/graphs/new <->=\marg{left node}\marg{right node}\marg{edge options}\marg{edge nodes}}
    Called for |<->| with the same parameters as above. The |->| is
    replaced by |<-|
    \begin{stylekey}{/tikz/every new <->}
    \end{stylekey}
  \end{key}
  \begin{key}{/tikz/graphs/new <-=\marg{left node}\marg{right node}\marg{edge options}\marg{edge nodes}}
    Called for |<-| with the same parameters as above.%
    \footnote{You might
      wonder why this key is needed: It seems more logical at first
      sight to just call |new edge directed| with swapped first
      parameters. However, a positioning
      algorithm might wish to take the fact into account that an edge is
      ``backward'' rather than ``forward''  in order to
      improve the layout. Also, different arrow heads might be used.}
    \begin{stylekey}{/tikz/every new <-}
    \end{stylekey}
  \end{key}
  \begin{key}{/tikz/graphs/new -\protect\exclamationmarktext-=\marg{left node}\marg{right node}\marg{edge options}\marg{edge nodes}}
    Called for |-!-| with the same parameters as above. Does nothing
    by default.
  \end{key}
\end{enumerate}

Here is an example that shows the default rendering of the different
edge specifications:

\begin{codeexample}[]
\tikz \graph [branch down=5mm] {
  a -> b;
  c -- d;
  e <- f;
  g <-> h;
  i -!- j;
};  
\end{codeexample}



\subsubsection{Syntax of Node Specifications}

\label{section-library-graphs-node-spec}

Node specifications are the basic building blocks of a graph
specification. There are three different possible kinds of node
specifications, each of which has a different syntax: 

\begin{description}
\item[Direct Node Specification]
  \ \\
  \opt{|"|}\meta{node name}\opt{|"|}\opt{|/|\opt{|"|}\meta{text}\opt{|"|}} \opt{\oarg{options}}\\
  (note that the quotation marks are optional and only needed when the
  \meta{node name} contains special symbols)
\item[Reference Node Specification]
  \ \\
  |(|\meta{node name or node set name}|)|
\item[Group Node Specification]
  \ \\
  \meta{group specification}
\end{description}

The rule for determining which of the possible kinds is meant is
as follows: If the node specification starts with an opening
parenthesis, a reference node specification is meant; if it starts
with an opening curly brace, a group specification is meant; and in 
all other cases a direct node specification is meant.

\medskip
\textbf{Direct Node Specifications.} If after reading the first symbol
of a node specification is has been detected to be \emph{direct},
\tikzname\ will collect all text up to the next edge
specification and store it as the \meta{node name}; however, square
brackets are used to indicate options and a slash ends the \meta{node
  name} and start a special \meta{text} that is used as a 
rendering text instead of the original \meta{node name}.

Due to the way the parsing works and due to the restrictions on node
names, most special characters are forbidding inside the \meta{node
  name}, including commas, semicolons, hyphens, braces, dots,
parentheses, slashes, dashes, and more (but spaces, single
underscores, and the hat character \emph{are} allowed). To use special
characters in the name of a node, you can optionally surround the
\meta{node name} and/or the \meta{text} by quotation marks. In this
case, you can use all of the special symbols once more. The details
of what happens, exactly, when the \meta{node name} is surrounded by
quotation marks is explained later; surrounding the \meta{text} by
quotation marks has essentially the same effect as surrounding it by
curly braces.

Once the node name has been determined, it is checked whether the same
node name was already used inside the current graph. If this is the
case, then we say that the already existing node is \emph{referenced};
otherwise we say that the node is \emph{fresh}.

\begin{codeexample}[]
\tikz \graph {
  a -> b; % both are fresh
  c -> a; % only c is fresh, a is referenced
};
\end{codeexample}

This behaviour of deciding whether a node is fresh or referenced can,
however, be modified by using the following keys:
\begin{key}{/tikz/graphs/use existing node=\opt{\meta{true or
        false}} (default true)}
  When this key is set to |true|, all nodes will be considered to the
  referenced, no node will be fresh. This option is useful if you have
  already created all the nodes of a graph prior to using the |graph|
  command and you now only wish to connect the nodes.
\end{key}
\begin{key}{/tikz/graphs/fresh nodes=\opt{\meta{true or
        false}} (default true)}
  When this key is set to |true|, all nodes will be considered to be
  fresh. This option is useful when you create for instance a tree
  with many identical nodes.

  When a node name is encountered that was already used previously,
  a new name is chosen is follows: An apostrophe (|'|) is appended
  repeatedly until a node name is found that has not yet been
  used:
\begin{codeexample}[]
\tikz \graph [branch down=5mm] {
  { [fresh nodes]
    a -> {
      b -> {c, c},
      b -> {c, c},
      b -> {c, c},
    }
  },  
  b' -- b''
};
\end{codeexample}
\end{key}

\begin{key}{/tikz/graphs/number nodes=\opt{\meta{start number}} (default 1)}
  When this key is used in a scope, each encountered node name will
  get appended a new number, starting with \meta{start}. Typically,
  this ensures that all node names are different. Between the original
  node name and the appended number, the setting of the following will
  be inserted:
  \begin{key}{/tikz/graphs/number nodes sep=\meta{text} (initially \normalfont space)}
  \end{key}
\begin{codeexample}[]
\tikz \graph [branch down=5mm] {
  { [number nodes]
    a -> {
      b -> {c, c},
      b -> {c, c},
      b -> {c, c},
    }
  },  
  b 2 -- b 5
};
\end{codeexample}
\end{key}

When a fresh node has been detected, a new node is created in the
inside a protecting scope. For this, the current
placement strategy is asked to compute a default position for the
node, see Section~\ref{section-library-graphs-placement} for
details. Then, the command
\begin{quote}
  |\node (|\meta{full node name}|) [|\meta{node options}|] {|\meta{text}|};|
\end{quote}
is called. The different parameters are as follows:
\begin{itemize}
\item
  The \meta{full node name} is normally the \meta{node name} that has
  been determined as described before. However, there are two exceptions:

  First, if the \meta{node name} is empty (which happens when there
  is no \meta{node name} before the slash), then a fresh internal node
  name is created and used as 
  \meta{full node name}. This name is guaranteed to be different from all
  node names used in this or any other graph. Thus, a direct node
  starting with a slash represents an anonymous fresh node. 

  Second, you can use the following key to prefix the \meta{node name}
  inside the \meta{full node name}:

  \begin{key}{/tikz/graphs/name=\meta{text}}
    This key prepends the \meta{text}, followed by a separating symbol
    (a space by default), to all
    \meta{node name}s inside a \meta{full node name}. Repeated calls
    of this key accumulate, leading to ever-longer ``name paths'':
\begin{codeexample}[]
\begin{tikzpicture}
  \graph {
    { [name=first]  1, 2, 3} --
    { [name=second] 1, 2, 3}
  };
  \draw [red] (second 1) circle [radius=3mm];
\end{tikzpicture}
\end{codeexample}
    Note that, indeed, in the above example six nodes are created even
    though the first and second set of nodes have the same \meta{node
      name}. The reason is that the full names of the six nodes are
    all different. Also note that only the \meta{node name} is used as
    the node text, not the full name. This can be changed as described
    later on.

    This key can be used repeatedly, leading to ever longer node names.
  \end{key}

  \begin{key}{/tikz/graphs/name separator=\meta{symbols} (initially \string\space)}
    Changes the symbol that is used to separate the \meta{text} from
    the \meta{node name}. The default is |\space|, resulting in a
    space.
\begin{codeexample}[]
\begin{tikzpicture}
  \graph [name separator=] { % no separator
    { [name=first]  1, 2, 3} --
    { [name=second] 1, 2, 3}
  };
  \draw [red] (second1) circle [radius=3mm];
\end{tikzpicture}
\end{codeexample}
\begin{codeexample}[]
\begin{tikzpicture}
  \graph [name separator=-] {
    { [name=first]  1, 2, 3} --
    { [name=second] 1, 2, 3}
  };
  \draw [red] (second-1) circle [radius=3mm];
\end{tikzpicture}
\end{codeexample}
  \end{key}
\item 
  The \meta{node options} are
  \begin{enumerate}
  \item The options that have accumulated in calls to |nodes| from
    the surrounding scopes.
  \item The local \meta{options}.
  \end{enumerate}
  The options are executed with the path prefix |/tikz/graphs|, but
  any unknown key is executed with the prefix |/tikz|. This means, in
  essence, that some esoteric keys are more difficult to use inside
  the options and that any key with the prefix |/tikz/graphs| will
  take precedence over a key with the prefix |/tikz|.
\item The \meta{text} that is passed to the |\node| command is
  computed as follows: First, you can use the following key to
  directly set the \meta{text}: 
  \begin{key}{/tikz/graphs/as=\meta{text}}
    The \meta{text} is used as the text of the node. This allows you
    to provide a text for the node that differs arbitrarily from the
    name of the node.
    \begin{codeexample}[]
\tikz \graph { a [as=$x$] -- b [as=$y_5$] -> c [red, as={a--b}] };
    \end{codeexample}
    This key always takes precedence over all of the mechanisms
    described below.
  \end{key}
  In case the |as| key is not used, a default text
  is chosen as follows: First, when a direct node specification
  contains a slash (or, for historical reasons, a double underscore),
  the text to the right of the slash (or double underscore) is stored
  in the macro |\tikzgraphnodetext|; if 
  there is no slash, the \meta{node name} is stored in
  |\tikzgraphnodetext|, instead. Then, the current value of the
  following key is used as \meta{text}:
  \begin{key}{/tikz/graphs/typeset=\meta{code}}
    The macro or code stored in this key is used as the
    \meta{text} if the node. Inside the \meta{code}, the following
    macros are available:
    \begin{command}{\tikzgraphnodetext}
      This macro expands to the \meta{text} to the right of the double
      underscore or slash in a direct node specification or, if there
      is no slash, to the \meta{node name}.
    \end{command}
    \begin{command}{\tikzgraphnodename}
      This macro expands to the name of the current node with the
      path. 
    \end{command}
    \begin{command}{\tikzgraphnodepath}
      This macro expands to the current path of the node. These
      paths result from the use of the |name| key as described above.
    \end{command}
    \begin{command}{\tikzgraphnodefullname}
      This macro contains the concatenation of the above two.
    \end{command}
  \end{key}
  By default, the typesetter is just set to |\tikzgraphnodetext|,
  which means that the default text of a node is its name. However,
  it may be useful to change this: For instance, you might wish that
  the text of all graph nodes is, say, surrounded by parentheses:
  \begin{codeexample}[]
\tikz \graph [typeset=(\tikzgraphnodetext)]
  { a -> b -> c };
  \end{codeexample}
  A more advanced macro might take apart the node text and render it
  differently: 
  \begin{codeexample}[]
\def\mytypesetter{\expandafter\myparser\tikzgraphnodetext\relax}
\def\myparser#1 #2 #3\relax{%
  $#1_{#2,\dots,#3}$
}
\tikz \graph [typeset=\mytypesetter, grow down]
  { a 1 n -> b 2 m -> c 4 nm };
  \end{codeexample}
  The following styles install useful predefined typesetting macros:
  \begin{key}{/tikz/graphs/empty nodes}
    Just sets |typeset| to nothing, which causes all nodes to have an
    empty text (unless, of course, the |as| option is used):
    \begin{codeexample}[]
\tikz \graph [empty nodes, nodes={circle, draw}] { a -> {b, c} };  
    \end{codeexample}
  \end{key}
  \begin{key}{/tikz/graphs/math nodes}
    Sets |typeset| to |$\tikzgraphnodetext$|, which causes all nodes
    names to be typeset in math mode:
    \begin{codeexample}[]
\tikz \graph [math nodes, nodes={circle, draw}] { a_1 -> {b^2, c_3^n} };  
    \end{codeexample}
  \end{key}
\end{itemize}

If a node is referenced instead of fresh, then this node becomes the
node that will be connected by the preceding or following edge
specification to other 
nodes. The \meta{options} are executed even for a referenced node, but
they cannot be used to change the appearance of the node (because the
node exists already). Rather, the \meta{options} can only be used to
change the logical coloring of the node, see
Section~\ref{section-library-graphs-color-classes} for details.

\medskip
\textbf{Quoted Node Names.} When the \meta{node name} and/or the
\meta{text} of a node is surrounded by quotation marks, you can use
all sorts of special symbols as part of the text that are
normally forbidden:
\begin{codeexample}[]
\begin{tikzpicture}
  \graph [grow right=2cm] {
    "Hi, World!"       -> "It's \emph{important}!"[red,rotate=-45];
    "name"/actual text -> "It's \emph{important}!";
  };
  \draw (name) circle [radius=3pt];
\end{tikzpicture}
\end{codeexample}

In detail, for the following happens when qutation marks are
encountered at the beginning of a node name or its text:
\begin{itemize}
\item Everything following the quotation mark up to the next single
  quotation mark is collected into a macro \meta{collected}. All sorts
  of special characters, including commas, square brackets, dashes,
  and even backslashes are allowed here. Basically, the only
  restriction is that braces must be balanced.
\item A double quotation mark (|""|) does not count as the ``next
  single quotation mark.'' Rather, it is replaced by a single
  quotation mark. For instance, |"He said, ""Hello world."""| would be
  stored inside \meta{collected} as |He said, "Hello world."|
  However, this rule applies only on the outer-most level of
  braces. Thus, in
\begin{codeexample}[code only]
"He {said, ""Hello world.""}"
\end{codeexample}
  we would get |He {said, ""Hello world.""}| as \meta{collected}.
\item ``The next single quotation mark'' refers to the next
  quotation mark on the current level of braces, so in
  |"hello {"} world"|, the next quotation mark would be the one
  following |world|.
\end{itemize}

Now, once the \meta{collected} text has been gather, it is used as
follows: When used as \meta{text} (what is actually displayed), it is
just used ``as is''. When it is used as \meta{node name}, however, the
following happens: Every ``special character'' in \meta{collected} is
replaced by its Unicode name, surrounded by |@|-signs. For instance,
if \meta{collected} is |Hello, world!|, the \meta{node name} is the
somewhat longer text |Hello@COMMA@ world@EXCLAMATION MARK@|. Admittedly,
referencing such a node from outside the graph is 
cumbersome, but when you use exactly the same \meta{collected} text
once more, the same \meta{node name} will result. The
following characters are considered ``special'':
\begin{quote}
  \texttt{\char`\|}|$&^~_[](){}/.-,+*'`!":;<=>?@#%\{}|%$
\end{quote}
These are exactly the Unicode character with a decimal code number
between 33 and 126 that are neither digits nor letters. 


\medskip
\textbf{Reference Node Specifications.} A reference node specification
is a node specification that starts with an opening parenthesis. In
this case, parentheses must surround a \meta{name} as in |(foo)|,
where |foo| is the \meta{name}. The following will now happen:

\begin{enumerate}
\item It is tested whether \meta{name} is the name of a currently
  active \emph{node set}. This case will be discussed in a moment.
\item Otherwise, the \meta{name} is interpreted and treated as a
  referenced node, but independently of whether the node has already
  been fresh in the current graph or not. In other words, the node
  must have been defined either already inside the graph (in which
  case the parenthesis are more or less superfluous) or it must have
  been defined outside the current picture.

  The way the referenced node is handled is the same way as for a
  direct node that is a referenced node.

  If the node does not already exist, an error message is printed.
\end{enumerate}

Let us now have a look at node sets. Inside a |{tikzpicture}| you can
locally define a \emph{node set} by using the following key:
\begin{key}{/tikz/new set=\meta{set name}}
  This will setup a node set named \meta{set name} within the current
  scope. Inside the scope, you can add nodes to the node set using the
  |set| key. If a node set of the same name already exists in the
  current scope, it will be reset and made empty for the current
  scope.

  Note that this command has the path |/tikz| and is normally used
  \emph{outside} the |graph| command.
\end{key}
\begin{key}{/tikz/set=\meta{set name}}
  This key can be used as an option with a |node| command. The
  \meta{set name} must be the name of a node set that has previously
  been created inside some enclosing scope via the |new set| key. The
  effect is that the current node is added to the node set.
\end{key}

When you use a |graph| command inside a scope where some node set
called \meta{set name} is defined, then inside this |graph| command
you use |(|\meta{set name}|)| to reference \emph{all} of the nodes in
the node set. The effect is the same as if instead of the reference to
the set name you had created a group specification containing a list
of references to all the nodes that are part of the node set.

\begin{codeexample}[]
\begin{tikzpicture}[new set=red, new set=green, shorten >=2pt]
  \foreach \i in {1,2,3} {
    \node [draw, red!80,         set=red]   (r\i) at (\i,1) {$r_\i$};
    \node [draw, green!50!black, set=green] (g\i) at (\i,2) {$g_\i$};
  }
  \graph {
    root [xshift=2cm] ->
    (red)             -> [complete bipartite, right anchor=south]
    (green)
  };
\end{tikzpicture}
\end{codeexample}

There is an interesting caveat with referencing node sets: Suppose
that at the beginning of a graph you just say |(foo);| where |foo| is
a set name. Unless you have specified special options, this will cause
the following to happen: A group is created whose members are all the
nodes of the node set |foo|. These nodes become referenced nodes, but
otherwise nothing happens since, by default, the nodes of a group are
not connected automatically. However, the referenced nodes have now
been referenced inside the graph, you can thus subsequently access
them as if they had been defined inside the graph. Here is an example
showing how you can create nodes outside a |graph| command and then
connect them inside as if they had been declared inside:

\begin{codeexample}[]
\begin{tikzpicture}[new set=import nodes]
  \begin{scope}[nodes={set=import nodes}] % make all nodes part of this set
    \node [red] (a) at (0,1) {$a$};
    \node [red] (b) at (1,1) {$b$};
    \node [red] (d) at (2,1) {$d$};
  \end{scope}

  \graph {
    (import nodes);         % "import" the nodes

    a -> b -> c -> d -> e;  % only c and e are new
  };
\end{tikzpicture}
\end{codeexample}


\medskip
\textbf{Group Node Specifications.}
At a place where a node specification should go, you can also instead
provide a group specification. Since nodes specifications are part of
chain specifications, which in turn are part of group specifications,
this is a recursive definition.

\begin{codeexample}[]
\tikz \graph { a -> {b,c,d} -> {e -> {f,g}, h} };
\end{codeexample}

As can be seen in the above example, when two groups of nodes are
connected via an edge specification, it is not immediately obvious
which connecting edges are added. This is detailed in
Section~\ref{section-library-graphs-color-classes}. 



\subsubsection{Specifying Tries}

In computer science, a \emph{trie} is a special kind of tree, where
for each node and each symbol of an alphabet, there is at most one
child of the node labeled with this symbol.

The |trie| key is useful for drawing tries, but it can also be used in
other situations. What it does, essentially, is to prepend the node
names of all nodes \emph{before} the current node of the current chain
to the node's name. This will often make it easier or more natural to
specify graphs in which several nodes have the same label. 

\begin{key}{/tikz/graphs/trie=\opt{\meta{true or false}} (default true, initially false)}
  If this key is set to |true|, after a node has been created on a
  chain, the |name| key is executed with the node's \meta{node
    name}. Thus, all nodes later on this chain have the ``path'' of
  nodes leading to this node as their name. This means, in particular,
  that
  \begin{enumerate}
  \item two nodes of the same name but in different parts of a chain
    will be different,
  \item while if another chain starts with the same nodes, no new
    nodes get created.
  \end{enumerate}
  In total, this is exactly the behaviour you would expect of a trie:
\begin{codeexample}[]
\tikz \graph [trie] {
  a -> {
    a,
    c -> {a, b},
    b
  }
};
\end{codeexample}
  You can even ``reiterate'' over a path in conjunction with the
  |simple| option. However, in this case, the default placement
  strategies will not work and you will need options like
  |layered layout| from the graph drawing libraries, which need
  Lua\TeX. 
\ifluatex  
\begin{codeexample}[]
\tikz \graph [trie, simple, layered layout] {
  a -> b -> a,
  a -> b -> c,
  a -> {d,a}
};
\end{codeexample}
  In the following example, we setup the |typeset| key so that it
  shows the complete names of the nodes:
\begin{codeexample}[]
\tikz \graph [trie, simple, layered layout,
              typeset=\tikzgraphnodefullname] {
  a -> b -> a,
  a -> b -> c,
  a -> {d,a}
};
\end{codeexample}
\fi
  You can also use the |trie| key locally and later reference nodes
  using their full name:
\begin{codeexample}[]
\tikz \graph {
  { [trie, simple]
    a -> {
      b,
      c -> a
    }
  },
  a b ->[red] a c a
};
\end{codeexample}
\end{key}


\subsection{Quick Graphs}

\label{section-library-graphs-quick}

The graph syntax is powerful, but this power comes at a price: parsing
the graph syntax, which is done by \TeX, can take some time. Normally,
the parsing is fast enough that you will not notice it, but it can be
bothersome when you have graphs with hundreds of nodes as happens
frequently when nodes are generated algorithmically by some other
program. Fortunately, when another program generated a graph
specification, we typically do not need the full power of the graph
syntax. Rather, a small subset of the graph syntax would suffice that
allows to specify nodes and edges. For these reasons, the is a special
``quick'' version of the graph syntax.

Note, however, that using this syntax will usually at most halve the
time needed to parse a graph. Thus, it really mostly makes sense in
conjunction with large, algorithmically generated graphs.

\begin{key}{/tikz/graphs/quick}
  When you provide this key with a graph, the syntax of graph
  specifications gets restricted. You are no longer allowed to use
  certain features of the graph syntax; but all features that are
  still allowed are also allowed in the same way when you do not
  provide the |quick| option. Thus, leaving out the |quick| option
  will never hurt.

  Since the syntax is so severely restricted, it is easier to explain
  which aspects of the graph syntax \emph{will} still work:
  
  \begin{enumerate}
  \item 
    A quick graph consists of a sequence of either nodes, edges sequences, or
    groups. These are separated by commas or semicolons.
  \item  
    Every node is of the form
    
    \begin{quote}
      |"|\meta{node name}|"|\opt{|/"|\meta{node text}|"[|\meta{options}|]|}
    \end{quote}

    The quotation marks are mandatory. The part |/"|\meta{node
      text}|"| may  be missing, in which case the node name is used as
    the node text. The \meta{options} may also be missing. The
    \meta{node name} may not contain any ``funny'' characters (unlike
    in the normal graph command).
  \item
    Every chain is of the form
    
    \begin{quote}
      \meta{node spec} \meta{connector} \meta{node spec}
      \meta{connector} \dots \meta{connector} \meta{node spec}|;|
    \end{quote}

    Here, the \meta{node spec} are node specifications as described
    above, the \meta{connector} is one of the four connectors |->|,
    |<-|, |--|, and |<->| (the connector |-!-| is not allowed since
    the |simple| option is also not allowed). Each connector may be
    followed by options in square brackets. The semicolon may be
    replaced by a comma. 
  \item
    Every group is of the form
    
    \begin{quote}
      |{ [|\meta{options}|]| \meta{chains and groups} |};|
    \end{quote}
    The \meta{options} are compulsory. The semicolon can, again, be 
    replaced by a comma.
  \item
    The |number nodes| option will work as expected.
  \end{enumerate}

  Here is a typical way this syntax might be used:
\begin{codeexample}[]
\tikz \graph [quick] { "a" --["foo"] "b"[x=1] };
\end{codeexample}

\begin{codeexample}[]
\tikz \graph [quick] {
  "a"/"$a$" -- "b"[x=1] --[red] "c"[x=2];
  { [nodes=blue] "a" -- "d"[y=1]; };
};
\end{codeexample}

  Let us now have a look at the most important things that will
  \emph{not} work when the |quick| option is used:

  \begin{itemize}
  \item Connecting a node and a group as in |a->{b,c}|.
  \item Node names without quotation marks as in |a--b|.
  \item Everything described in subsequent subsections, which includes
    subgraphs (graph macros), graph sets, graph color classes,
    anonymous nodes, the |fresh nodes| option, sublayouts, simple
    graphs, edge annotations.
  \item Placement strategies -- you either have to define all node
    positions explicitly using |at=| or |x=| and |y=| or you must use
    a graph drawing algorithm like |layered layout|.
  \end{itemize}  
\end{key}


\subsection{Simple Versus Multi-Graphs}

\label{section-library-graphs-simple}

The |graph| library allows you to construct both simple graphs and
multi-graphs. In a simple graph there can be at most one edge between
any two vertices, while in a multi-graph there can be multiple edges
(hence the name). The two keys |multi| and |simple| allow you to
switch (even locally inside on of the graph's scopes) between which
kind of graph is being constructed. By default, the |graph| command
produces a multi-graph since these are faster to construct.

\begin{key}{/tikz/graphs/multi}
  When this edge is set for a whole graph (which is the default) or
  just for a group (which is useful if the whole graph is simple in
  general, but a part is a multi-graph), then when you specify an edge
  between two nodes several times, several such edges get created:

\begin{codeexample}[]
\tikz \graph [multi] { % "multi" is not really necessary here
  a ->[bend left,  red]  b;
  a ->[bend right, blue] b;
};
\end{codeexample}
  In case |multi| is used for a scope inside a larger scope where the
  |simple| option is specified, then inside the local |multi| scope
  edges are immediately created and they are completely ignored when
  it comes to deciding which kind of edges should be present in the
  surrounding simple graph. From the surrounding scope's point of view
  it is as if the local |multi| graph contained no edges at all.

  This means, in particular, that you can use the |multi| option with
  a single edge to ``enforce'' this edge to be present in a simple
  graph. 
\end{key}

\begin{key}{/tikz/graphs/simple}
  In contrast a multi-graph, in a simple graph, at most one edge gets
  created for every pair of vertices:
\begin{codeexample}[]
\tikz \graph [simple]{
  a ->[bend left,  red]  b;
  a ->[bend right, blue] b;
};
\end{codeexample}
  As can be seen, the second edge ``wins'' over the first edge. The
  general rule is as follows: In a simple graph, whenever an edge
  between two vertices is specified multiple times, only the very last
  specification and its options will actually be executed.
  
  The real power of the |simple| option lies in the fact that you can
  first create a complicated graph and then later redirect and otherwise
  modify edges easily: 

\begin{codeexample}[]
\tikz \graph [simple, grow right=2cm] {
  {a,b,c,d} ->[complete bipartite] {e,f,g,h};

  { [edges={red,thick}] a -> e -> d -> g -> a };
};
\end{codeexample}

  One particularly interesting kind of edge specification for a simple
  graph is |-!-|. Recall that this is used to indicate that ``no
  edge'' should be added between certain nodes. In a multi-graph, this
  key usually has no effect (unless the key |new -!-| has been
  redefined) and is pretty superfluous. In a simple graph, however, it
  counts as an edge kind and you can thus use it to remove an edge
  that been added previously:

\begin{codeexample}[]
\tikz \graph [simple] {
  subgraph K_n [n=8, clockwise];
  % Get rid of the following edges:
  1 -!- 2;
  3 -!- 4;
  6 -!- 8;
  % And make one edge red:
  1 --[red] 3;
};
\end{codeexample}

  Creating a graph such as the above in other fashions is pretty
  awkward.

  For every unordered pair $\{u,v\}$ of vertices at most one edge will
  be created in a simple graph. In particular, when you say |a -> b|
  and later also |a <- b|, then only the edge |a <- b| will be
  created. Similarly, when you say |a -> b| and later |b -> a|, then
  only the edge |b -> a| will be created. 

  The power of the |simple| command comes at a certain cost: As the
  graph is being constructed, a (sparse) array is created that keeps
  track for each edge of the last edge being specified. Then, at the
  end of the scope containing the |simple| command, for every pair of
  vertices the edge is created. This is implemented by two nested
  loops iterating over all possible pairs of vertices -- which may
  take quite a while in a graph of, say, 1000 vertices.
  Internally, the |simple| command is implemented as an operator that
  adds the edges when it is called, but
  this should be unimportant in normal situations.
\end{key}




\subsection{Graph Edges: Labeling and Styling}

When the |graph| library creates an edge between two nodes in a graph,
the appearance (called ``styling'' in \tikzname) can be specified in
different ways. Sometimes you will simply wish to say ``the edges
between these two groups of node should be red,'' but sometimes you
may wish to say ``this particular edge going into this node should be
red.'' In the following, different ways of specifying such styling
requirements are discussed. Note that adding labels to edges is, from
\tikzname's point of view, almost the same as styling edges, since
they are also specified using options.


\subsubsection{Options For All Edges Between Two Groups}

When you write |... ->[options] ...| somewhere inside your graph
specification, this typically cause one or more edges to be created
between the nodes in the chain group before the |->| and the nodes in
the chain group following it. The |options| are applied to all of
them. In particular, if you use the |quotes| library and you write
some text in quotes inside the |options|, this text will be added as a
label to each edge:

\begin{codeexample}[]
\tikz 
  \graph [edge quotes=near start] {
    { a, b } -> [red, "x", complete bipartite] { c, d };
  };
\end{codeexample}

As documented in the |quotes| library in more detail, you can easily
modify the appearance of edge labels created using the quotes syntax
by adding options after the closing quotes:

\begin{codeexample}[]
\tikz \graph {
  a ->["x"] b ->["y"'] c ->["z" red] d;
};
\end{codeexample}

The following options make it easy to setup the styling of nodes
created in this way:

\begin{key}{/tikz/graphs/edge quotes=\opt{\meta{options}}}
  A shorthand for setting the style |every edge quotes| to \meta{options}.  
\begin{codeexample}[]
  \tikz \graph [edge quotes={blue,auto}] {
  a ->["x"] b ->["y"'] c ->["b" red] d;
};
\end{codeexample}
\end{key}

\begin{key}{/tikz/graphs/edge quotes center}
  A shorthand for |edge quotes| to |anchor=center|.
\begin{codeexample}[]
\tikz \graph [edge quotes center] {
  a ->["x"] b ->["y"] c ->["z" red] d;
};
\end{codeexample}
\end{key}

\begin{key}{/tikz/graphs/edge quotes mid}
  A shorthand for |edge quotes| to |anchor=mid|.
\begin{codeexample}[]
\tikz \graph [edge quotes mid] {
  a ->["x"] b ->["y"] c ->["z" red] d;
};
\end{codeexample}
\end{key}



\subsubsection{Changing Options For Certain Edges}


Consider the following tree-like graph:

\begin{codeexample}[]
\tikz \graph { a -> {b,c} };
\end{codeexample}

Suppose we wish to specify that the edge from |a| to |b| should be
red, while the edge from |a| to |c| should be blue. The difficulty
lies in the fact that \emph{both} edges are created by the single |->|
operator and we can only add one of these option |red| or |blue| to
the operator.

There are several ways to solve this problem. First, we can simply
split up the specification and specify the two edges separately:

\begin{codeexample}[]
\tikz \graph {
  a -> [red]  b;
  a -> [blue] c;
};  
\end{codeexample}
While this works quite well, we can no longer use the nice chain group
syntax of the |graphs| library. For the rather simple graph |a->{b,c}|
this is not a big problem, but if you specify a tree with, say, 30
nodes it is really worthwhile being able to specify the tree ``in its
natural form in the \TeX\ code'' rather than having to list all of the
edges explicitly. Also, as can be seen in the above example, the
node placement is changed, which is not always desirable. 

One can sidestep this problem using the |simple| option: This option
allows you to first specify a graph and then, later on, replace edges
by other edges and, thereby, provide new options:

\begin{codeexample}[]
\tikz \graph [simple] {
  a -> {b,c};
  a -> [red]  b;
  a -> [blue] c;
};  
\end{codeexample}

The first line is the original specification of the tree, while the
following two lines replace some edges of the tree (in this case, all
of them) by edges with special options. While this method is slower
and in the above example creates even longer code, it is very useful
if you wish to, say, highlight a path in a larger tree: First specify
the tree normally and, then, ``respecify'' the path or paths with some
other edge options in force. In the following example, we use this to
highlight a whole subtree of a larger tree:

\begin{codeexample}[]
\tikz \graph [simple] {
  % The larger tree, no special options in force
  a -> {
    b -> {c,d},
    e -> {f,g},
    h
  },  
  { [edges=red] % Now highlight a part of the tree
    a -> e -> {f,g}
  }
};  
\end{codeexample}



\subsubsection{Options For Incoming and Outgoing Edges}

When you use the syntax |... ->[options] ...| to specify options, you
specify options for the ``connections between two sets of nodes''. In
many cases, however, it will be more natural to specify options ``for
the edges lead to or coming from a certain node'' and you will want to
specify these options ``at the node''. Returning to the example of the
graph |a->{b,c}| where we want a red edge between |a| and |b| and a
blue edge between |a| and |c|, this could also be phrased as follows:
``Make the edge leading to |b| red and make the edge leading to |c|
blue.''

For this situation, the |graph| library offers a number of special
keys, which are documented in the following. However, most of the time
you will not use these keys directly, but, rather, use a special
syntax explained in Section~\ref{section-syntax-outgoing-incoming}.

\begin{key}{/tikz/graphs/target edge style=\meta{options}}
  This key can (only) be used with a \emph{node} inside a graph 
  specification. When used, the \meta{options} will be added to every
  edge that is created by a connector like |->| in which the node is a
  \emph{target}. Consider the following example:
\begin{codeexample}[]
\tikz \graph {
  { a, b } ->
  { c [target edge style=red], d } ->
  { e, f }
};
\end{codeexample}
  In the example, only when the edge from |a| to |c| is created, |c|
  is the ``target'' of the edge. Thus, only this edge becomes red.
  
  When an edge already has options set directly, the \meta{options}
  are executed after these direct options, thus, they ``overrule''
  them:
\begin{codeexample}[]
\tikz \graph {
  { a, b } -> [blue, thick]
  { c [target edge style=red], d } ->
  { e, f }
};
\end{codeexample}
  
  The \meta{options} set in this way will stay attached to the node,
  so also for edges created later on that lead to the node will have
  these options set:
\begin{codeexample}[]
\tikz \graph {
  { a, b } ->
  { c [target edge style=red], d } ->
  { e, f },
  b -> c
};
\end{codeexample}

  Multiple uses of this key accumulate. However, you may sometimes
  also wish to ``clear'' these options for a key since at some later
  point you no longer wish the \meta{options} to be added when some
  further edges are added. This can be achieved using the following
  key:
  \begin{key}{/tikz/graphs/target edge clear}
    Clears all \meta{options} for edges with the node as a target and
    also edge labels (see below) for this node.    
  \end{key}
\begin{codeexample}[]
\tikz \graph {
  { a, b } ->
  { c [target edge style=red], d },
  b -> c[target edge clear]
};
\end{codeexample}
\end{key}

\begin{key}{/tikz/graphs/target edge node=\meta{node specification}}
  This key works like |target edge style|, only the \meta{node
    specification} will not be added as options to any newly created
  edges with the current node as their target, but rather it will be
  added as a node specification.  
\begin{codeexample}[]
\tikz \graph {
  { a, b } ->
  { c [target edge node=node{X}], d } ->
  { e, f }
};
\end{codeexample}
  As for |target edge style| multiple uses of this key accumulate and
  the key |target edge clear| will (also) clear all target edge nodes
  that have been set for a node earlier on.
\end{key}

\begin{key}{/tikz/graphs/source edge style=\meta{options}}
  Works exactly like |target edge style|, only now the \meta{options}
  are only added when the node is a source of a newly created edge:
\begin{codeexample}[]
\tikz \graph {
  { a, b } ->
  { c [source edge style=red], d } ->
  { e, f }
};
\end{codeexample}
  If both for the source and also for the target of an edge
  \meta{options} have been specified, the options are applied in the
  following order:
  \begin{enumerate}
  \item First come the options from the edge itself.
  \item Then come the options contributed by the source node using
    this key.
  \item Then come the options contributed by the target node using
    |target node style|.    
  \end{enumerate}
\begin{codeexample}[]
\tikz \graph {
  a [source edge style=red] ->[green]
  b [target edge style=blue]  % blue wins
};
\end{codeexample}
\end{key}

\begin{key}{/tikz/graphs/source edge node=\meta{node specification}}
  Works like |source edge style| and |target edge node|.  
\end{key}

\begin{key}{/tikz/graphs/source edge clear=\meta{node specification}}
  Works like |target edge clear|.  
\end{key}



\subsubsection{Special Syntax for Options For Incoming and Outgoing Edges}
\label{section-syntax-outgoing-incoming}

The keys |target node style| and its friends are powerful, but a bit
cumbersome to write down. For this reason, the |graphs| library
introduces a special syntax that is based on what I call the
``first-char syntax'' of keys. Inside the options of a node inside a
graph, the following special rules apply:
\begin{enumerate}
\item Whenever an option starts with |>|, the rest of the options are
  passed to |target edge style|. For instance, when you write
  |a[>red]|, then this has the same effect as if you had written
\begin{codeexample}[code only]
a[target edge style={red}]    
\end{codeexample}
\item Whenever an options starts with |<|, the rest of the options are
  passed to |source edge style|.
\item In both of the above case, in case the options following the |>|
  or |<| sign start with a quote, the created edge label is passed to
  |source edge node| or |target edge node|, respectively.

  This is exactly what you want to happen.
\end{enumerate}
Additionally, the following styles provide shorthands for ``clearing''
the target and source options:
\begin{key}{/tikz/graphs/clear >}
  A more easy-to-remember shorthand for |target edge clear|.  
\end{key}
\begin{key}{/tikz/graphs/clear <}
  A more easy-to-remember shorthand for |source edge clear|.  
\end{key}

These mechanisms make it especially easy to create trees in which the
edges are labeled in some special way:
\begin{codeexample}[]
\tikz 
  \graph [edge quotes={fill=white,inner sep=1pt},
          grow down, branch right] {
    / -> h [>"9"] -> {
      c [>"4" text=red,] -> {
        a [>"2", >thick],
        e [>"0"]
      },
      j [>"7"]
    }
  };
\end{codeexample}


\subsubsection{Placing Node Texts on Incoming Edges}

Normally, the text of a node is shown (only) inside the node. In some
case, for instance when drawing certain kind of trees, the nodes
themselves should not get any text, but rather the edge leading to the
node should be labeled as in the following example:
\begin{codeexample}[]
\tikz \graph [empty nodes]
{
  root -> {
    a [>"a"],
    b [>"b"] -> {
      c [>"c"],
      d [>"d"] 
    }
  }
};
\end{codeexample}
As the example shows, it is a bit cumbersome that we have to label the
nodes and then specify the same text once more using the incoming edge
syntax.

For these cases, it would be better if the text of the node where not
used with the node but, rather, be passed directly to the incoming or
the outgoing edge. The following styles do exactly this:

\begin{key}{/tikz/graphs/put node text on incoming edges=\opt{\meta{options}}}
  When this key is used with a node or a group, the following happens:
  \begin{enumerate}
  \item The command |target edge node={node[|\meta{options}|]{\tikzgraphnodetext}}|
    is executed. This means that all incoming edges of the node get a
    label with the text that would usually be displayed in the
    node. You can use keys like |math nodes| normally.
  \item The command |as={}| is executed. This means that the node
    itself will display nothing.
  \end{enumerate}
  Here is an example that show how this command is used.
\begin{codeexample}[]
\tikz \graph [put node text on incoming edges,
              math nodes, nodes={circle,draw}]
  { a -> b -> {c, d} };
\end{codeexample}
\end{key}

\begin{key}{/tikz/graphs/put node text on outgoing
    edges=\opt{\meta{options}}}
  Works like the previous key, only with |target| replaced by |source|.  
\end{key}



\subsection{Graph Operators, Color Classes, and Graph Expressions}
\label{section-library-graphs-color-classes}

\tikzname's |graph| command employs a powerful mechanism for
adding edges between nodes and sets of nodes. To a graph theorist,
this mechanism may be known as a \emph{graph expression}: A graph is
specified by starting with small graphs and then applying
\emph{operators} to them that form larger graphs and that connect and
recolor colored subsets of the graph's node in different ways. 


\subsubsection{Color Classes}

\label{section-library-graph-coloring}

\tikzname\ keeps track of a \emph{(multi)coloring} of the graph as it
is being constructed. This does not mean that the 
actual color of the nodes on the page will be different, rather, in
the following we refer to ``logical'' colors in the way graph
theoreticians do. These ``logical'' colors are only important while
the graph is being constructed and they are ``thrown away'' at the end
of the construction. The actual (``physical'') colors of the nodes are
set independently of these logical colors.

As a graph is being constructed, each node can be part of one or more
overlapping \emph{color classes}. So, unlike what is sometimes called a
\emph{legal coloring}, the logical colorings that \tikzname\ keeps
track of may assign multiple colors to the same node and two nodes
connected by an edge may well have the same color.

Color classes must be declared prior to use. This is done using the
following key:
\begin{key}{/tikz/graphs/color class=\meta{color class name}}
  This sets up a new color class called \meta{color class name}. Nodes
  and whole groups of nodes can now be colored with \meta{color class
    name}. This is done using the following keys, which become
  available inside the current scope: 
  \begin{key}{/tikz/graphs/\meta{color class name}}
    This key internally uses the |operator| command to setup an
    operator that will cause all nodes of the current group to get the
    ``logical color'' \meta{color class name}. Nodes retain this color
    in all encompassing scopes, unless it is explicitly changed (see
    below) or unset (again, see below).
\begin{codeexample}[]
\tikz \graph [color class=red] {
  [cycle=red]  % causes all "logically" red nodes to be connected in
               % a cycle
  a,
  b [red],
  { [red] c ->[bend right] d },
  e
};
\end{codeexample}
\begin{codeexample}[]
\tikz \graph [color class=red, color class=green,
              math nodes, clockwise, n=5] {
  [complete bipartite={red}{green}]
  { [red]   r_1, r_2 },
  { [green] g_1, g_2, g_3 }
};
\end{codeexample}
  \end{key}
  \begin{key}{/tikz/graphs/not \meta{color class name}}
    Sets up an operator for the current scope so that all nodes in it
    loose the color \meta{color class name}. You can also use
    |!|\meta{color class name} as an alias for this key.
\begin{codeexample}[]
\tikz \graph [color class=red, color class=green,
              math nodes, clockwise, n=5] {
  [complete bipartite={red}{green}]
  { [red]   r_1, r_2 },
  { [green] g_1, g_2, g_3 },
  g_2 [recolor green by=red]
};
\end{codeexample}
  \end{key}
  \begin{key}{/tikz/graphs/recolor \meta{color class name} by=\meta{new color}}
    Causes all keys having color \meta{color class name} to get
    \meta{new color} instead. They loose having color \meta{color
      class name}, but other colors are not affected.
\begin{codeexample}[]
\tikz \graph [color class=red, color class=green,
              math nodes, clockwise, n=5] {
  [complete bipartite={red}{green}]
  { [red]   r_1, r_2 },
  { [green] g_1, g_2, g_3 },
  g_2 [not green]
};
\end{codeexample}
  \end{key}
\end{key}

The following color classes are available by default:
\begin{itemize}
\item 
  Color class |all|. Every node is part of this class by default. This
  is useful to access all nodes of a (sub)graph, since you can simply
  access all nodes of this color class. 
\item
  Color classes |source| and |target|. These classes are used to
  identify nodes that lead ``into'' a group of nodes and nodes from
  which paths should ``leave'' the group. Details on how these colors
  are assigned are explained in
  Section~\ref{section-library-graphs-join}. By saying |not source| or
  |not target| with a node, you can influence how it is connected:
  \begin{codeexample}[]
\tikz \graph { a -> { b, c, d } -> e };    
  \end{codeexample}
  \begin{codeexample}[]
\tikz \graph { a -> { b[not source], c, d[not target] } -> e };    
  \end{codeexample}
\item
  Color classes |source'| and |target'|. These are temporary colors
  that are also explained in
  Section~\ref{section-library-graphs-join}.
\end{itemize}



\subsubsection{Graph Operators on Groups of Nodes}

Recall that the |graph| command constructs graphs recursively from
nested \meta{group specifications}. Each such \meta{group
  specification} describes a subset of the nodes of the final graph. A
\emph{graph operator} is an algorithm that gets the nodes of a group
as input and (typically) adds edges between these nodes in some
sensible way. For instance, the |clique| operator will simply add
edges between all nodes of the group.

\begin{key}{/tikz/graphs/operator=\meta{code}}
  This key has an effect in three places:
  \begin{enumerate}
  \item It can be used in the \meta{options} of a \meta{direct node specification}.
  \item It can be used in the \meta{options} of a \meta{group
      specification}.
  \item It can be used in the \meta{options} of an \meta{edge specification}.
  \end{enumerate}
  The first case is a special case of the second, since it is treated
  like a group specification containing a single node. The last case
  is more complicated and discussed in the next section.
  So, let us focus on the second case.

  Even though the \meta{options} of a group are given at the beginning
  of the \meta{group specification}, the \meta{code} is only executed
  when the group has been parsed completely and all its nodes have
  been identified. If you use the |operator| multiple times in the
  \meta{options}, the effect accumulates, that is, all code passed to
  the different calls of |operator| gets executed in the order it is
  encountered. 

  The \meta{code} can do ``whatever it wants,'' but it will typically
  add edges between certain nodes. You can configure what kind of
  edges (directed, undirected, etc.) are created by using the
  following keys:
  \begin{key}{/tikz/graphs/default edge kind=\meta{value} (initially --)}
    This key stores one of the five edge kinds |--|, |<-|, |->|,
    |<->|, and |-!-|. When an operator wishes to create a new edge, it
    should typically set
\begin{codeexample}[code only]
\tikzgraphsset{new \pfkeysvalueof{/tikz/graphs/default edge kind}=...}      
\end{codeexample}
    While this key can be set explicitly, it may be more convenient to
    use the abbreviating keys listed below. Also, this key is
    automatically set to the current value of \meta{edge
      specification} when a joining operator is called, see the
    discussion of joining operators in
    Section~\ref{section-library-graphs-join}.
  \end{key}
  \begin{key}{/tikz/graphs/--}
    Sets the |default edge kind| to |--|.
\begin{codeexample}[]
\tikz \graph { subgraph K_n [--, n=5, clockwise, radius=6mm] };      
\end{codeexample}
  \end{key}
  \begin{key}{/tikz/graphs/->}
    Sets the |default edge kind| to |->|.
\begin{codeexample}[]
\tikz \graph { subgraph K_n [->, n=5, clockwise, radius=6mm] };      
\end{codeexample}
  \end{key}
  \begin{key}{/tikz/graphs/<-}
    Sets the |default edge kind| to |<-|.
\begin{codeexample}[]
\tikz \graph { subgraph K_n [<-, n=5, clockwise, radius=6mm] };      
\end{codeexample}
  \end{key}
  \begin{key}{/tikz/graphs/<->}
    Sets the |default edge kind| to |<->|.
\begin{codeexample}[]
\tikz \graph { subgraph K_n [<->, n=5, clockwise, radius=6mm] };      
\end{codeexample}
  \end{key}
  \begin{key}{/tikz/graphs/-\protect\exclamationmarktext-}
    Sets the |default edge kind| to |-!-|.
  \end{key}
  
  When the \meta{code} of an operator is executed, the following
  commands can be used to find the nodes that should be connected:
  \begin{command}{\tikzgraphforeachcolorednode\marg{color name}\marg{macro}}
    When this command is called inside \meta{code}, the following will
    happen: \tikzname\ will iterate over all nodes inside the
    just-specified group that have the color \meta{color name}. The
    order in which they are iterated over is the order in which they
    appear inside the group specification (if a node is encountered
    several times inside the specification, only the first occurrence
    counts). Then, for each node the \meta{macro} is executed with the
    node's name as the only argument.

    In the following example we use an operator to connect every
    node colored |all| inside the subgroup to he node |root|. 
    \begin{codeexample}[]
\def\myconnect#1{\tikzset{graphs/new ->={root}{#1}{}{}}}      

\begin{tikzpicture}
  \node (root) at (-1,-1) {root};
  
  \graph {
    x,
    {
      [operator=\tikzgraphforeachcolorednode{all}{\myconnect}]
      a, b, c
    }
  };
\end{tikzpicture}
    \end{codeexample}
  \end{command}
  
  \begin{command}{\tikzgraphpreparecolor\marg{color
        name}\marg{counter}\marg{prefix}}
    This command is used to ``prepare'' the nodes of a certain color
    for random access. The effect is the following: It is counted how
    many nodes there are having color \meta{color name} in the current
    group and the result is stored in \meta{counter}. Next, macros
    named \meta{prefix}|1|, \meta{prefix}|2|, and so on are defined,
    that store the names of the first, second, third, and so on node
    having the color \meta{color name}.

    The net effect is that after you have prepared a color, you can
    quickly iterate over them. This is especially useful when you
    iterate over several color at the same time.

    As an example, let us create an operator then adds a zig-zag path
    between two color classes:
    \begin{codeexample}[]
\newcount\leftshorecount   \newcount\rightshorecount
\newcount\mycount          \newcount\myothercount
\def\zigzag{
  \tikzgraphpreparecolor{left shore}\leftshorecount{left shore prefix}
  \tikzgraphpreparecolor{right shore}\rightshorecount{right shore prefix}
  \mycount=0\relax
  \loop
    \advance\mycount by 1\relax%
    % Add the "forward" edge
    \tikzgraphsset{new ->=
      {\csname left shore prefix\the\mycount\endcsname}
      {\csname right shore prefix\the\mycount\endcsname}{}{}}
    \myothercount=\mycount\relax%
    \advance\myothercount by1\relax%
    \tikzgraphsset{new <-=
      {\csname left shore prefix\the\myothercount\endcsname}
      {\csname right shore prefix\the\mycount\endcsname}{}{}}    
  \ifnum\myothercount<\leftshorecount\relax
  \repeat
}
\begin{tikzpicture}
  \graph [color class=left shore, color class=right shore]
  { [operator=\zigzag]
    { [left shore, Cartesian placement]                      a, b, c },
    { [right shore, Cartesian placement, nodes={xshift=1cm}] d, e, f }
  };
\end{tikzpicture}
\end{codeexample}
    Naturally, in order to turn the above code into a usable operator,
    some more code would be needed (like default values and taking
    care of shores of different sizes).
  \end{command}
\end{key}

There are a number of predefined operators, like |clique| or |cycle|,
see the reference Section~\ref{section-library-graphs-reference} for a
complete list.



\subsubsection{Graph Operators for Joining Groups}
\label{section-library-graphs-join}

When you join two nodes |foo| and |bar| by the edge specification
|->|, it is fairly obvious, what should happen: An edge from |(foo)|
to |(bar)| should be created. However, suppose we use an edge
specification between two node sets like |{a,b,c}| and |{d,e,f}|. In
this case, it is not so clear which edges should be created. One might
argue that all possible edges from any node in the first set to any node
in the second set should be added. On the other hand, one might also
argue that only a matching between these two sets should be
created. Things get even more muddy when a longer chain of node sets
are joined.

Instead of fixing how edges are created between two node sets,
\tikzname\ takes a somewhat more general, but also more complicated
approach, which can be broken into two parts. In the following, assume
that the following chain specification is given:
\begin{quote}
  \meta{spec$_1$} \meta{edge specification} \meta{spec$_2$}
\end{quote}
An example might be |{a,b,c} -> {d, e->f}|.

\medskip
\textbf{The source and target vertices.} Let us start with the
question of which vertices of the first node set should be connected to
vertices in the second node set.

There are two predefined special color classes that are used for this:
|source| and |target|. For every group specification, some vertices
are colored as |source| vertices and some vertices are |target|
vertices (a node can both be a target and a source). Initially, every
vertex is both a source and a target, but that can change as we will
see in a moment.

The intuition behind source and target vertices is that, in some
sense, edges ``from the outside'' lead into the group via the source
vertices and lead out of the group via the target vertices. To be more
precise, the following happens:
\begin{enumerate}
\item The target vertices of the first group are connected to
  the source vertices of the second group.
\item In the group resulting from the union of the nodes from
  \meta{spec$_1$} and \meta{spec$_2$}, the source vertices are only
  those from the first group, and the target vertices are only those
  from the second group.
\end{enumerate}

Let us go over the effect of these rules for the example
|{a,b,c} -> {d, e->f}|. First, each individual node is initially both
a |source| and a |target| vertex. Then, in |{a,b,c}| all nodes are
still both source and target vertices since just grouping vertices
does not change their colors. Now, in |e->f| something interesting
happens for the first time: the target vertices of the ``group'' |e|
(which is just the node |e|) are connected to the source vertices of
the ``group'' |f|. This means, that an edge is added from |e| to
|f|. Then, in the resulting group |e->f| the only source vertex is |e|
and the only target vertex is |f|. This implies that in the group
|{d,e->f}| the sources are |d| and |e| and the targets are |d| and~|f|.

Now, in |{a,b,c} -> {d,e->f}| the targets  of |{a,b,c}| (which are all
three of them) are connected to the sources of |{d,e->f}| (which are
just |d| and~|e|). Finally, in the whole graph only |a|, |b|, and |c|
are sources while only  |d| and |f| are targets.

\begin{codeexample}[]
\def\hilightsource#1{\fill [green, opacity=.25] (#1) circle [radius=2mm]; }  
\def\hilighttarget#1{\fill [red,   opacity=.25] (#1) circle [radius=2mm]; }  
\tikz \graph
  [operator=\tikzgraphforeachcolorednode{source}{\hilightsource},
   operator=\tikzgraphforeachcolorednode{target}{\hilighttarget}]
  { {a,b,c} -> {d, e->f} };
\end{codeexample}

The next objective is to make more precise what it means that ``the
targets of the first graph'' and the ``sources of the second graph''
should be connected. We know already of a general way of connecting
nodes of a graph: operators! Thus, we use an operator for this job.
For instance, the |complete bipartite| operator adds an edge from every node
having a certain color to every node have a certain other color. This
is exactly what we need here: The first color is ``the color |target|
restricted to the nodes of the first graph'' and the second color is
``the color |source| restricted to the nodes of the second graph.''

However, we cannot really specify that only nodes from a certain
subgraph are meant -- the |operator| machinery only operates on all
nodes of the current graph. For this reason, what really happens is
the following: When the |graph| command encounters \meta{spec$_1$}
\meta{edge specification} \meta{spec$_2$}, it first computes and
colors the nodes of the first and the second specification
independently. Then, the |target| nodes of the first graph are
recolored to |target'| and the |source| nodes of the second graph are
recolored to |source'|. Then, the two graphs are united into one
graph and a \emph{joining operator} is executed, which should add
edges between |target'| and |source'|. Once this is done, 
the colors |target'| and |source'| get erased. Note that in the
resulting graph only the |source| nodes from the first graph are still
|source| nodes and likewise for the |target| nodes of the second graph.


\medskip
\textbf{The joining operators.} The job of a joining operator is
to add edges between nodes colored |target'| and |source'|. The
following rule is used to determine which operator should be chosen
for performing this job:
\begin{enumerate}
\item If the \meta{edge specification} explicitly sets the |operator|
  key to something non-empty (and also not to |\relax|), then the
  \meta{code} of this |operator| call is used.
\item Otherwise, the current value of the following key is used: 
  \begin{key}{/tikz/graphs/default edge operator=\meta{key}}
    This key stores the name of a \meta{key} that is executed for every
    \meta{edge specification} whose \meta{options} do not contain the
    |operator| key.
\begin{codeexample}[]
\tikz \graph [default edge operator=matching] {
  {a, b}    ->[matching and star]
  {c, d, e} --[complete bipartite]
  {f, g, h} -- 
  {i, j, k}
};    
\end{codeexample}
  \end{key}
\end{enumerate}

A typical joining operator is |complete bipartite|. It takes the names of two
color classes as input and adds edges from all vertices of the first
class to all vertices of the second class. Now, the trick is that the
default value for the |complete bipartite| key is |{target'}{source'}|. Thus,
if you just write |->[complete bipartite]|, the same happens as if you had
written
\begin{quote}
|->[complete bipartite={target'}{source'}]|    
\end{quote}
This is exactly what we want to happen. The same default values are
also set for other joining operators like |matching| or |butterfly|.

Even though an operator like |complete bipartite| is typically used
together with an edge specification, it can also be used as a normal
operator together with a group specification. In this case, however,
the color classes must be named explicitly:

\begin{codeexample}[]
\begin{tikzpicture}
  \graph [color class=red, color class=green, math nodes]
  { [complete bipartite={red}{green}]
    { [red,   Cartesian placement]                      r_1, r_2, r_3 },
    { [green, Cartesian placement, nodes={xshift=1cm}]  g_1, g_2, g_3 }
  };
\end{tikzpicture}
\end{codeexample}

A list of predefined joining operators can be found in the reference
Section~\ref{section-library-graphs-reference}.

The fact that joining operators can also be used as normal operators
leads to a subtle problem: A normal operator will typically use the
current value of |default edge kind| to decide which kind of edges
should be put between the identified vertices, while a joining
operator should, naturally, use the kind of edge specified by the
\meta{edge specification}. This problem is solved as follows: Like a
normal operator, a joining operator should also use the current value
of |default edge kind| for the edges it produces. The trick is that
this will automatically be set to the current \meta{edge
  specification} when the operator explicitly in the \meta{options} of
the edge specification or implicitly in the |default edge operator|.




\subsection{Graph Macros}
\label{section-library-graphs-macros}

A \emph{graph macro} is a small graph that is inserted at some point
into the graph that is currently being constructed. There is special
support for such graph macros in \tikzname. You might wonder why this
is necessary -- can't one use \TeX's normal macro mechanism? The
answer is ``no'': one cannot insert new nodes into a graph using
normal macros because the chains, groups, and nodes are determined
prior to macro expansion. Thus, any macro encountered where some node
text should go will only be expanded when this node is being named and 
typeset.

A graph macro is declared using the following key:

\begin{key}{/tikz/graphs/declare=\marg{graph name}\marg{specification}}
  This key declares that \meta{graph name} can subsequently be used as
  a replacement for a \meta{node name}. Whenever the \meta{graph name}
  is used in the following, a graph group will be inserted instead
  whose content is exactly \meta{specification}. In case \meta{graph
    name} is used together with some \meta{options}, they are executed
  prior to inserting the \meta{specification}.
\begin{codeexample}[]
\tikz \graph [branch down=4mm, declare={claw}{1 -- {2,3,4}}] {
  a;
  claw;
  b;
};
\end{codeexample}
  In the next example, we use a key to configure a subgraph:
\begin{codeexample}[]
\tikz \graph [ n/.code=\def\n{#1}, branch down=4mm,
               declare={star}{root -- { \foreach \i in {1,...,\n} {\i} }}]
{ star [n=5]; };
\end{codeexample}
  Actually, the |n| key is already defined internally for a similar
  purpose.

  As a last example, let us define a somewhat more complicated graph
  macro. 
\begin{codeexample}[]
\newcount\mycount  
\tikzgraphsset{
  levels/.store in=\tikzgraphlevel,
  levels=1,
  declare={bintree}{%
    [/utils/exec={%
      \ifnum\tikzgraphlevel=1\relax%
        \def\childtrees{ / }%
      \else%
        \mycount=\tikzgraphlevel%
        \advance\mycount by-1\relax%
        \edef\childtrees{
          / -> {
            bintree[levels=\the\mycount],
            bintree[levels=\the\mycount]
          }}
      \fi%
    },
    parse/.expand once=\childtrees
    ]
    % Everything is inside the \childtrees...
  }
}
\tikz \graph [grow down=5mm, branch right=5mm] { bintree [levels=5] };
\end{codeexample}
\end{key}

Note that when you use a graph macro several time inside the same
graph, you will typically have to use the |name| option so that
different copies of the subgraph are created:
\begin{codeexample}[]
\tikz \graph [branch down=4mm, declare={claw}{1 -- {2,3,4}}] {
  claw [name=left],
  claw [name=right]
};
\end{codeexample}

You will find a list of useful graph macros in the reference section,
Section~\ref{section-library-graphs-reference-macros}. 


\subsection{Online Placement Strategies}
\label{section-library-graphs-placement}

The main job of the |graph| library is to make it easy to specify
which nodes are present in a graph and how they are connected. In
contrast, it is \emph{not} the primary job of the library to compute
good positions for nodes in a graph -- use for instance a |\matrix|,
specify good positions ``by hand'' or use the graph drawing
facilities. Nevertheless, some basic support for automatic node
placement is provided for simple cases. The graph library will provide
you with information about the position of nodes inside their groups
and chains. 

As a graph is being constructed, a \emph{placement strategy} is used
to determine a (reasonably good) position for the nodes as they are
created. These placement strategies get some information about what
\tikzname\ has already seen concerning the already constructed nodes,
but it gets no information concerning the upcoming nodes. Because of
this lack of information concerning the future, the strategies need to
be what is called an \emph{online strategy} in computer science. (The
opposite are \emph{offline strategies}, which get information about
the whole graph and all the sizes of the nodes in it. The graph
drawing libraries employ such offline strategies.)

Strategies are selected using keys like |no placement| or
|Cartesian placement|. It is permissible to use different strategies inside
different parts of a graph, even though the different strategies do
not always work together in perfect harmony.


\subsubsection{Manual Placement}
\label{section-graphs-xy}

\begin{key}{/tikz/graphs/no placement}
  This strategy simply ``switches off'' the whole placement
  mechanism, causing all nodes to be placed at the origin by
  default. You need to use this strategy if you position nodes ``by
  hand''. For this, you can use the |at| key, the |shift| keys:
\begin{codeexample}[]
\tikz \graph [no placement]
{
  a[at={(0:0)}] -> b[at={(1,0)}] -> c[yshift=1cm];
};
\end{codeexample}
  Since the syntax and the many braces and parentheses are a bit
  cumbersome, the following two keys might also be useful: 
  \begin{key}{/tikz/graphs/x=\meta{x dimension}}
    When you use this key, it will have the same effect as if you had written
    |at={(|\meta{x dimension}|,|\meta{y dimension}|)}|, where \meta{y
      dimension} is a value set using the |y| key:
\begin{codeexample}[]
\tikz \graph [no placement]
{
  a[x=0,y=0] -> b[x=1,y=0] -> c[x=0,y=1];
};
\end{codeexample}
    Note that you can specify an |x| or a |y| key for a whole scope
    and then vary only the other key:
\begin{codeexample}[]
\tikz \graph [no placement]
{
  a ->
  { [x=1] % group option
    b [y=0] -> c[y=1]
  };
};
\end{codeexample}
    Note that these keys have the path |/tikz/graphs/|, so they will
    be available inside |graph|s and will not clash with the usual |x|
    and |y| keys of \tikzname, which are used to specify the basic
    lengths of vectors.
  \end{key}
  \begin{key}{/tikz/graphs/y=\meta{y dimension}}
    See above.    
  \end{key}
\end{key}




\subsubsection{Placement on a Grid}

\begin{key}{/tikz/graphs/Cartesian placement}
  This strategy is the default strategy. It works, roughly, as
  follows: For each new node on a chain, advance a ``logical width''
  counter and for each new node in a group, advance a ``logical
  depth'' counter. When a chain contains a whole group, then the
  ``logical width'' taken up by the group is the maximum over the
  logical widths taken up by the chains inside the group; and
  symmetrically the logical depth of a chain is the maximum of the
  depths of the groups inside it.
  
  This slightly confusing explanation is perhaps best exemplified. In
  the below example, the two numbers indicate the two logical width
  and depth of each node as computed by the |graph| library. Just
  ignore the arcane code that is used to print these numbers.
\begin{codeexample}[]
\tikz
  \graph [nodes={align=center, inner sep=1pt}, grow right=7mm,
          typeset={\tikzgraphnodetext\\[-4pt]
                   \tiny\mywidth\\[-6pt]\tiny\mydepth},
          placement/compute position/.append code=
            \pgfkeysgetvalue{/tikz/graphs/placement/width}{\mywidth}
            \pgfkeysgetvalue{/tikz/graphs/placement/depth}{\mydepth}]
{
  a,
  b,
  c -> d -> {
    e -> f -> g,
    h -> i
  } -> j,
  k -> l
};
\end{codeexample}  
  You will find a detailed description of how these logical units are
  computed, exactly, in Section~\ref{section-library-graphs-new-online}.
  
  Now, even though we talk about ``widths'' and ``depths'' and even
  though by default a graph ``grows'' to the right and down, this is
  by no means fixed. Instead, you can use the following keys to change
  how widths and heights are interpreted:
  \begin{key}{/tikz/graphs/chain shift=\meta{coordinate} (initially {(1,0)})}
    Under the regime of the |Cartesian placement| strategy, each node is
    shifted by the current logical width times this \meta{coordinate}.
\begin{codeexample}[]
\tikz \graph [chain shift=(45:1)] {
  a -> b -> c;
  d -> e;
  f -> g -> h;
};
\end{codeexample}
  \end{key}  
  \begin{key}{/tikz/graphs/group shift=\meta{coordinate} (initially {(0,-1)})}
    Like for |chain shift|, each node is shifted by the current
    logical depth times this \meta{coordinate}. 
\begin{codeexample}[]
\tikz \graph [chain shift=(45:7mm), group shift=(-45:7mm)] {
  a -> b -> c;
  d -> e;
  f -> g -> h;
};
\end{codeexample}
  \end{key}  
\end{key}


\begin{key}{/tikz/graphs/grow up=\meta{distance} (default 1)}
  Sets the |chain shift| to |(|\meta{distance}|,0)|, so that chains
  ``grow upward.'' The distance by which the center of each new
  element is removed from the center of the previous one is
  \meta{distance}. 
\begin{codeexample}[]
\tikz \graph [grow up=7mm] { a -> b -> c};      
\end{codeexample}
\end{key}
\begin{key}{/tikz/graphs/grow down=\meta{distance} (default 1)}
  Like |grow up|.
\begin{codeexample}[]
\tikz \graph [grow down=7mm] { a -> b -> c};      
\end{codeexample}
\end{key}
\begin{key}{/tikz/graphs/grow left=\meta{distance} (default 1)}
  Like |grow up|.
\begin{codeexample}[]
\tikz \graph [grow left=7mm] { a -> b -> c};      
\end{codeexample}
\end{key}
\begin{key}{/tikz/graphs/grow right=\meta{distance} (default 1)}
  Like |grow up|.
\begin{codeexample}[]
\tikz \graph [grow right=7mm] { a -> b -> c};      
\end{codeexample}
\end{key}
\begin{key}{/tikz/graphs/branch up=\meta{distance} (default 1)}
  Sets the |group shift| so that groups ``branch upward.''  The
  distance by which the center of each new element is removed from
  the center of the previous one is \meta{distance}.
\begin{codeexample}[]
\tikz \graph [branch up=7mm] { a -> b -> {c, d, e} };      
\end{codeexample}
  Note that when you draw a tree, the |branch ...| keys specify how
  siblings (or adjacent branches) are arranged, while the |grow ...|
  keys specify in which direction the branches ``grow''.
\end{key}
\begin{key}{/tikz/graphs/branch down=\meta{distance} (default 1)}
\begin{codeexample}[]
\tikz \graph [branch down=7mm] { a -> b -> {c, d, e}};      
\end{codeexample}
\end{key}
\begin{key}{/tikz/graphs/branch left=\meta{distance} (default 1)}
\begin{codeexample}[]
\tikz \graph [branch left=7mm, grow down=7mm] { a -> b -> {c, d, e}};      
\end{codeexample}
\end{key}
\begin{key}{/tikz/graphs/branch right=\meta{distance} (default 1)}
\begin{codeexample}[]
\tikz \graph [branch right=7mm, grow down=7mm] { a -> b -> {c, d, e}};      
\end{codeexample}
\end{key}

The following keys place nodes in a $N\times M$ grid. 
\begin{key}{/tikz/graphs/grid placement}
  This key works similar to |Cartesian placement|. As for that placement
  strategy, a node has logical width and depth 1. However, the computed
  total width and depth are mapped to a $N\times M$ grid.
  The values of $N$ and $M$ depend on the size of the graph and the
  value of |wrap after|. The number of columns $M$ is either set to 
  |wrap after| explicitly or computed automatically as 
  $\sqrt{\verb!|V|!}$. $N$ is the number of rows needed to lay out the 
  graph in a grid with $M$ columns. 
\begin{codeexample}[]
% An example with 6 nodes, 3 columns and therefor 2 rows
\tikz \graph [grid placement] { subgraph I_n[n=6, wrap after=3] };
\end{codeexample}
\begin{codeexample}[]
% An example with 9 nodes with columns and rows computed automatically
\tikz \graph [grid placement] { subgraph Grid_n [n=9] };
\end{codeexample}
\begin{codeexample}[]
% Directions can be changed
\tikz \graph [grid placement, branch up, grow left] { subgraph Grid_n [n=9] };
\end{codeexample}
  In case a user-defined graph instead of a pre-defined
  |subgraph| is to be layed out using |grid placement|, |n| has to be
  specified explicitly:
\begin{codeexample}[]
\tikz \graph [grid placement] { 
  [n=6, wrap after=3] 
  a -- b -- c -- d -- e -- f 
};
\end{codeexample}
\end{key}



\subsubsection{Placement Taking Node Sizes Into Account}

Options like |grow up| or |branch right| do not take the sizes of the
to-be-positioned nodes into account -- all nodes are placed quite
``dumbly'' at grid positions. It turns out that the
|Cartesian placement| can also be used to place notes in such a way
that their height and/or width is taken into account. Note, however,
that while the following options may yield an adequate placement in
many situations, when you need advanced alignments you should use a
|matrix| or advanced offline strategies to place the nodes.


\begin{key}{/tikz/graphs/grow right sep=\meta{distance} (default 1em)}
  This key has several effects, but let us start with the bottom line:
  Nodes along a chain are placed in such a way that the left end of a
  new node is \meta{distance} from the right end of the previous node: 
\begin{codeexample}[]
\tikz \graph [grow right sep, left anchor=east, right anchor=west] {
  start -- {
    long text -- {short, very long text} -- more text,
    long -- longer -- longest
  } -- end
};      
\end{codeexample}
  What happens internally is the following: First, the |anchor| of the
  nodes is set to |west| (or |north west| or |south west|, see
  below). Second, the logical width of a node is no 
  longer |1|, but set to the actual width of the node (which we define
  as the horizontal difference between the |west| anchor and the
  |east| anchor) in points. Third, the |chain shift| is set to
  |(1pt,0pt)|.
\end{key}
\begin{key}{/tikz/graphs/grow left sep=\meta{distance} (default 1em)}
\begin{codeexample}[]
\tikz \graph [grow left sep] { long -- longer -- longest };      
\end{codeexample}
\end{key}
\begin{key}{/tikz/graphs/grow up sep=\meta{distance} (default 1em)}
\begin{codeexample}[]
\tikz \graph [grow up sep] {
  a / $a=x$ --
  b / {$b=\displaystyle \int_0^1 x dx$} --
  c [draw, circle, inner sep=7mm]
};      
\end{codeexample}
\end{key}
\begin{key}{/tikz/graphs/grow down sep=\meta{distance} (default 1em)}
  As above.
\end{key}

\begin{key}{/tikz/graphs/branch right sep=\meta{distance} (default 1em)}
  This key works like |grow right sep|, only it affects groups rather
  than chains.
\begin{codeexample}[]
\tikz \graph [grow down, branch right sep] {
  start -- {
    an even longer text -- {short, very long text} -- more text,
    long -- longer -- longest,
    some text -- a -- b
  } -- end
};      
\end{codeexample}
  When both this key and, say, |grow down sep| are set, instead of the
  |west| anchor, the |north west| anchor will be selected
  automatically. 
\end{key}

\begin{key}{/tikz/graphs/branch left sep=\meta{distance} (default 1em)}
\begin{codeexample}[]
\tikz \graph [grow down sep, branch left sep] {
  start -- {
    an even longer text -- {short, very long text} -- more text,
    long -- longer,
    some text -- a -- b
  } -- end
};      
\end{codeexample}
\end{key}

\begin{key}{/tikz/graphs/branch up sep=\meta{distance} (default 1em)}
\begin{codeexample}[]
\tikz \graph [branch up sep] { a, b, c[draw, circle, inner sep=7mm] };      
\end{codeexample}
\end{key}

\begin{key}{/tikz/graphs/branch down sep=\meta{distance} (default 1em)}
\end{key}



\subsubsection{Placement On a Circle}

The following keys place nodes on circles. Note that, typically, you
do not use |circular placement| directly, but rather use one of the
two keys |clockwise| or |counterclockwise|.

\begin{key}{/tikz/graphs/circular placement}
  This key works quite similar to |Cartesian placement|. As for that
  placement strategy, a node has logical width and depth |1|. However,
  the computed total width and depth are mapped to polar coordinates
  rather than Cartesian coordinates.
  
  \begin{key}{/tikz/graphs/chain polar shift=|(|\meta{angle}|:|\meta{radius}|)| (initially {(0:1)})}
    Under the regime of the |circular placement| strategy, each node
    on a chain is shifted by |(|\meta{logical
      width}\meta{angle}|:|\meta{logical width}\meta{angle}|)|.
\begin{codeexample}[]
\tikz \graph [circular placement] {
  a -> b -> c;
  d -> e;
  f ->  g -> h;
};
\end{codeexample}
  \end{key}  
  \begin{key}{/tikz/graphs/group polar shift=|(|\meta{angle}|:|\meta{radius}|)| (initially {(45:0)})}
    Like for |group shift|, each node
    on a chain is shifted by |(|\meta{logical
      depth}\meta{angle}|:|\meta{logical depth}\meta{angle}|)|.
\begin{codeexample}[]
\tikz \graph [circular placement, group polar shift=(30:0)] {
  a -> b -> c;
  d -> e;
  f -> g -> h;
};
\end{codeexample}
\begin{codeexample}[]
\tikz \graph [circular placement,
              chain polar shift=(30:0),
              group polar shift=(0:1cm)] {
  a -- b -- c;
  d -- e;
  f -- g -- h;
};
\end{codeexample}
  \end{key}
  \begin{key}{/tikz/graphs/radius=\meta{dimension} (initially 1cm)}
    This is an initial value that is added to the total computed
    radius when the polar shift of a node has been
    calculated. Essentially, this key allows you to set the
    \meta{radius} of the innermost circle.
\begin{codeexample}[]
\tikz \graph [circular placement, radius=5mm] { a, b, c, d };
\end{codeexample}
\begin{codeexample}[]
\tikz \graph [circular placement, radius=1cm] { a, b, c, d };
\end{codeexample}
  \end{key}
  \begin{key}{/tikz/graphs/phase=\meta{angle} (initially 90)}
    This is an initial value that is added to the total computed
    angle when the polar shift of a node has been
    calculated. 
\begin{codeexample}[]
\tikz \graph [circular placement] { a, b, c, d };
\end{codeexample}
\begin{codeexample}[]
\tikz \graph [circular placement, phase=0] { a, b, c, d };
\end{codeexample}
  \end{key}
\end{key}

\label{key-graphs-clockwise}%
\begin{key}{/tikz/graphs/clockwise=\meta{number} (default \string\tikzgraphVnum)}
  This key sets the |group shift| so that if there are exactly
  \meta{number} many nodes in a group, they will form a complete
  circle. If you do not provide a \meta{number}, the current value of
  |\tikzgraphVnum| is used, which is exactly what you want when you
  use predefined graph macros like |subgraph K_n|.
\begin{codeexample}[]
\tikz \graph [clockwise=4] { a, b, c, d };
\end{codeexample}
\begin{codeexample}[]
\tikz \graph [clockwise] { subgraph K_n [n=5] };
\end{codeexample}
\end{key}

\label{key-graphs-counterclockwise}%
\begin{key}{/tikz/graphs/counterclockwise=\meta{number} (default \string\tikzgraphVnum)}
  Works like |clockwise|, only the direction is inverted.
\end{key}


\subsubsection{Levels and Level Styles}

As a graph is being parsed, the |graph| command keeps track of a
parameter called the \emph{level} of a node. Provided that the graph
is actually constructed in a tree-like manner, the level is exactly
equal to the level of the node inside this tree.

\begin{key}{/tikz/graphs/placement/level}
  This key stores a number that is increased for each element on a
  chain, but gets reset at the end of a group:
\begin{codeexample}[]
\tikz \graph [ branch down=5mm, typeset=
    \tikzgraphnodetext:\pgfkeysvalueof{/tikz/graphs/placement/level}]
{  
  a -> {
    b,
    c -> {
      d,
      e -> {f,g},
      h
    },
    j
  }
};
\end{codeexample}
  Unlike the parameters |depth| and |width| described in the next
  section, the key |level| is always available.
\end{key}

In addition to keeping track of the value of the |level| key, the
|graph| command also executes the following keys whenever it creates a
node:

\begin{stylekey}{/tikz/graph/level=\meta{level}}
  This key gets executed for each newly created node with \meta{level}
  set to the current level of the node. You can use this key to, say,
  reconfigure the node distance or the node color.
\end{stylekey}

\begin{stylekey}{/tikz/graph/level \meta{level}}
  This key also gets executed for each newly created node with
  \meta{level} set to the current level of the node.
\begin{codeexample}[]
\tikz \graph [
  branch down=5mm,
  level 1/.style={nodes=red},
  level 2/.style={nodes=green!50!black},
  level 3/.style={nodes=blue}]
{  
  a -> {
    b,
    c -> {
      d,
      e -> {f,g},
      h
    },
    j
  }
};
\end{codeexample}
\begin{codeexample}[]
\tikz \graph [
  branch down=5mm,
  level 1/.style={grow right=2cm},
  level 2/.style={grow right=1cm},
  level 3/.style={grow right=5mm}]
{  
  a -> {
    b,
    c -> {
      d,
      e -> {f,g},
      h
    },
    j
  }
};
\end{codeexample}
\end{stylekey}



\subsubsection{Defining New Online Placement Strategies}

\label{section-library-graphs-new-online}

In the following the details of how to define a new placement strategy
are explained. Most readers may wish to skip this section.

As a graph specification is being parsed, the |graph| library will keep
track of different numbers that identify the positions of the
nodes. Let us start with what happens on a chain. First, the following 
counter is increased for each element of the chain:
\begin{key}{/tikz/graphs/placement/element count}
  This key stores a number that tells us the position of the node on
  the current chain. However, you only have access to this value
  inside the code passed to the macro |compute position|, explained
  later on.
\begin{codeexample}[]
\tikz \graph [
  grow right sep, typeset=\tikzgraphnodetext:\mynum,
  placement/compute position/.append code=
    \pgfkeysgetvalue{/tikz/graphs/placement/element count}{\mynum}]
{
  a -> b -> c,
  d -> {e, f->h} -> j
};
\end{codeexample}
  As can be seen, each group resets the element counter.
\end{key}

The second value that is computed is more complicated to explain, but
it also gives more interesting information:
\begin{key}{/tikz/graphs/placement/width}
  This key stores the ``logical width'' of the nodes parsed up to now
  in the current group or chain (more precisely, parsed since the last
  call of |place| in an enclosing group). This is not necessarily the
  ``total physical width'' of the nodes, but rather a number
  representing how ``big'' the elements prior to the current element
  were. This \emph{may} be their width, but it may also be their
  height or even their number (which, incidentally, is the default). 
  You can use the |width| to perform shifts or rotations of
  to-be-created nodes (to be explained later).
  
  The logical width is defined recursively as follows. First, the
  width of a single node is computed by calling the following key:
  \begin{key}{/tikz/graphs/placement/logical node width=\meta{full
        node name}}
    This key is called to compute a physical or logical width of the
    node \meta{full node name}. You can change the code of this
    key. The code should return the computed value in the macro
    |\pgfmathresult|. By default, this key returns |1|.
  \end{key}
  The width of a chain is the sum of the widths of its elements. The 
  width of a group is the maximum of the widths of its elements.

  To get a feeling what the above rules imply in practice, let us
  first have a look at an example where each node has logical width
  and height |1| (which is the default). The arcane options at the
  beginning of the code just setup things so that the computed width
  and depth of each node is displayed at the bottom of each node.
\begin{codeexample}[]
\tikz
  \graph [nodes={align=center, inner sep=1pt}, grow right=7mm,
          typeset={\tikzgraphnodetext\\[-4pt]
                   \tiny\mywidth\\[-6pt]\tiny\mydepth},
          placement/compute position/.append code=
            \pgfkeysgetvalue{/tikz/graphs/placement/width}{\mywidth}
            \pgfkeysgetvalue{/tikz/graphs/placement/depth}{\mydepth}]
{
  a,
  b,
  c -> d -> {
    e -> f -> g,
    h -> i
  } -> j,
  k -> l
};
\end{codeexample}
  In the next example the ``logical'' width and depth actually match
  the ``physical'' width and height. This is caused by the
  |grow right sep| option, which internally sets the
  |logical node width| key so that it returns the width of its
  parameter in points.
\begin{codeexample}[]
\tikz
  \graph [grow right sep, branch down sep, nodes={align=left, inner sep=1pt},
          typeset={\tikzgraphnodetext\\[-4pt] \tiny Width: \mywidth\\[-6pt] \tiny Depth: \mydepth},
          placement/compute position/.append code=
            \pgfkeysgetvalue{/tikz/graphs/placement/width}{\mywidth}
            \pgfkeysgetvalue{/tikz/graphs/placement/depth}{\mydepth}]
{
  a,
  b,
  c -> d -> {
    e -> f -> g,
    h -> i
  } -> j,
  k -> l
};
\end{codeexample}  
\end{key}

Symmetrically to chains, as a group is being constructed, counters are
available for the number of chains encountered so far in the current
group and for the logical depth of the current group:
\begin{key}{/tikz/graphs/placement/element count}
  This key stores a number that tells us the sequence number of the
  chain in the current group.
\begin{codeexample}[]
\tikz \graph [
  grow right sep, branch down=5mm, typeset=\tikzgraphnodetext:\mynum,
  placement/compute position/.append code=
    \pgfkeysgetvalue{/tikz/graphs/placement/chain count}{\mynum}]
{
  a -> b -> {c,d,e},
  f,
  g -> h
};
\end{codeexample}
\end{key}

\begin{key}{/tikz/graphs/placement/depth}
  Similarly to the |width| key, this key stores the ``logical depth''
  of the nodes parsed up to now in the current group or chain and, also
  similarly, this key may or may not be related to the actual
  depth/height of the current node. 
  As for the |width|, the exact definition is as follows: For a single
  node, the depth is computed by the following key: 
  \begin{key}{/tikz/graphs/placement/logical node depth=\meta{full
        node name}}
    The code behind this key should return the ``logical height'' of
    the node   \meta{full node name} in the macro |\pgfmathresult|. 
  \end{key}
  Second, the depth of a group is the sum of the depths of its
  elements. Third, the depth of a chain is the maximum of the depth of
  its elements.  
\end{key}

The |width|, |depth|, |element count|, and |chain count| keys get
updated automatically, but do not have an effect by themselves. This
is to the following two keys:

\begin{key}{/tikz/graphs/placement/compute position=\meta{code}}
  The \meta{code} is called by the |graph| command just prior to
  creating a new node (the exact moment when this key is called is
  detailed in the description of the |place| key). When the
  \meta{code} is called, all of the keys described above will hold
  numbers computed in the way described above.

  The job of the \meta{code} is to setup node options appropriately so
  that the to-be-created node will be placed correctly. Thus, the
  \meta{code} should typically set the key
  |nodes={shift=|\meta{coordinate}|}| where \meta{coordinate} is the
  computed position for the node. The \meta{code} could
  also set other options like, say, the color of a node depending on
  its depth.

  The following example appends some code to the standard code of
  |compute position| so that ``deeper'' nodes of a tree are
  lighter. (Naturally, the same effect could be achieved much more
  easily using the |level| key.)
\begin{codeexample}[]
\newcount\mycount  
\def\lightendeepernodes{
  \pgfmathsetcount{\mycount}{
    100-20*\pgfkeysvalueof{/tikz/graphs/placement/width}
  }
  \edef\mydepth{\the\mycount}
  \tikzset{nodes={fill=red!\mydepth,circle,text=white}}
}
\tikz
  \graph [placement/compute position/.append code=\lightendeepernodes]
   {
     a -> {
       b -> c -> d,
       e -> {
         f,
         g
       },
       h
     }
   };
\end{codeexample}
\end{key}

\begin{key}{/tikz/graphs/placement/place}
  Executing this key has two effects: First, the key
  |compute position| is called to compute a good 
  position for future nodes (usually, these ``future nodes'' are just
  a single node that is created immediately). Second, all of the above
  counters like |depth| or |width| are reset (but not |level|).

  There are two places where this key is sensibly called: First, just
  prior to creating a node, which happens automatically. Second, when
  you change the online strategy. In this case, the computed width and
  depth values from one strategy typically make no sense in the other
  strategy, which is why the new strategy should proceed ``from a
  fresh start.'' In this case, the implicit call of |compute position|
  ensures that the new strategy gets the last place the old strategy
  would have used as its starting point, while the computation of its
  positions is now relative to this new starting point.

  For these reasons, when an online strategy like
  |Cartesian placement| is called, this key gets called
  implicitly. You will rarely need to call this key directly, except
  when you define a new online strategy.
\end{key}



\subsection{Reference: Predefined Elements}

\label{section-library-graphs-reference}


\subsubsection{Graph Macros}
\label{section-library-graphs-reference-macros}

\begin{tikzlibrary}{graph.standard}
  This library defines a number of graph macros that are often used in
  the literature. When new graphs are added to this collection, they
  will follow the definitions in the Mathematica program, see
  |mathworld.wolfram.com/topics/SimpleGraphs.html|. 
\end{tikzlibrary}


\begin{graph}{subgraph I\_n}
  This graph consists just of $n$ unconnected vertices. The following
  key is used to specify the set of these vertices:
  \begin{key}{/tikz/graphs/V=\marg{list of vertices}}
    Sets a list of vertex names for use with graphs like
    |subgraph I_n| and also other graphs. This list is available in
    the macro |\tikzgraphV|. The number of elements of this list is
    available in |\tikzgraphVnum|.
  \end{key}
  \begin{key}{/tikz/graphs/n=\meta{number}}
    This is an abbreviation for
    |V={1,...,|\meta{number}|}, name shore V={name=V}|.
  \end{key}
\begin{codeexample}[]
\tikz \graph [branch right, nodes={draw, circle}]
  { subgraph I_n [V={a,b,c}] };    
\end{codeexample}
  This graph is not particularly exciting by itself. However, it is
  often used to introduce nodes into a graph that are then connected
  as in the following example:
\begin{codeexample}[]
\tikz \graph [clockwise, clique] { subgraph I_n [n=4] };    
\end{codeexample}
\end{graph}


\begin{graph}{subgraph I\_nm}
  This graph consists of two sets of once $n$ unconnected vertices and
  then $m$ unconnected vertices. The first set consists of the
  vertices set by the key |V|, the other set consists of the vertices
  set by the key |W|.
\begin{codeexample}[]
\tikz \graph { subgraph I_nm [V={1,2,3}, W={a,b,c}] };    
\end{codeexample}
  In order to set the graph path name of the two
  sets, the following keys get executed:
  \begin{stylekey}{/tikz/graphs/name shore V (initially \normalfont empty)}
    Set this style to, say, |name=my V set| in order to set a
    name for the |V| set.    
  \end{stylekey}
  \begin{stylekey}{/tikz/graphs/name shore W (initially \normalfont empty)}
    Same as for |name shore V|.
  \end{stylekey}
  \begin{key}{/tikz/graphs/W=\marg{list of vertices}}
    Sets the list of vertices for the |W| set. The elements and
    their number are available in the macros |\tikzgraphW| and
    |\tikzgraphWnum|, respectively.
  \end{key}
  \begin{key}{/tikz/graphs/m=\meta{number}}
    This is an abbreviation for
    |W={1,...,|\meta{number}|}, name shore W={name=W}|.
  \end{key}
  The main purpose of this subgraph is to setup the nodes in a
  bipartite graph:
\begin{codeexample}[]
\tikz \graph {
  subgraph I_nm [n=3, m=4];

  V 1 -- { W 2, W 3 };
  V 2 -- { W 1, W 3 };
  V 3 -- { W 1, W 4 };
};    
\end{codeexample}
\end{graph}

\begin{graph}{subgraph K\_n}
  This graph is the complete clique on the vertices from the |V| key. 
\begin{codeexample}[]
\tikz \graph [clockwise] { subgraph K_n [n=7] };    
\end{codeexample}
\end{graph}


\begin{graph}{subgraph K\_nm}
  This graph is the complete bipartite graph with the two shores |V|
  and |W| as in |subgraph I_nm|.
\begin{codeexample}[]
\tikz \graph [branch right, grow down]
  { subgraph K_nm [V={6,...,9}, W={b,...,e}] };    
\end{codeexample}
\begin{codeexample}[]
\tikz \graph [simple, branch right, grow down]
{
  subgraph K_nm [V={1,2,3}, W={a,b,c,d}, ->];
  subgraph K_nm [V={2,3},   W={b,c},     <-];
};    
\end{codeexample}
\end{graph}

\begin{graph}{subgraph P\_n}
  This graph is the path on the vertices in |V|.
\begin{codeexample}[]
\tikz \graph [branch right] { subgraph P_n [n=3] };    
\end{codeexample}
\end{graph}


\begin{graph}{subgraph C\_n}
  This graph is the cycle on the vertices in |V|.
\begin{codeexample}[]
\tikz \graph [clockwise] { subgraph C_n [n=7, ->] };    
\end{codeexample}
\end{graph}


\begin{graph}{subgraph Grid\_n}
  This graph is a grid of the vertices in |V|.
  \begin{key}{/tikz/graphs/wrap after=\meta{number}}
    Defines the number of nodes placed in a single row of the grid. This
    value implicitly defines the number of grid columns as well.
    In the following example a |grid placement| is used to visualize the
    edges created between the nodes of a |Grid_n| |subgraph| using
    different values for |wrap after|.
    \begin{codeexample}[]
\tikz \graph [grid placement] { subgraph Grid_n [n=3,wrap after=1] };
\tikz \graph [grid placement] { subgraph Grid_n [n=3,wrap after=3] };
    \end{codeexample}
    \begin{codeexample}[]
\tikz \graph [grid placement] { subgraph Grid_n [n=4,wrap after=2] };
\tikz \graph [grid placement] { subgraph Grid_n [n=4] };
    \end{codeexample}
  \end{key}
\end{graph}


% TODO: Implement the Grid_nm subgraph described here:
%
%\begin{graph}{subgraph Grid\_nm}
%  This graph is a grid built from the cartesian product of the two node
%  sets |V| and |W| which are either defined using the keys
%  |/tikz/graphs/V| and |/tikz/graphs/W| or |/tikz/graphs/n| and
%  |/tikz/graphs/m| or a mixture of both. 
%  
%  The resulting |Grid_nm| subgraph has $n$ ``rows'' and $m$ ``columns'' and
%  the nodes are named |V i W j| with $1\le i\le n$ and $1\le j\le n$.
%  The names of the two shores |V| and |W| can be changed as described in
%  the documentation of the keys |/tikz/graphs/name shore V| and
%  |/tikz/graphs/name shore W|.
%  \begin{codeexample}[]
%\tikz \graph [grid placement] { subgraph Grid_nm [V={1,2,3}, W={4, 5, 6}] };
%  \end{codeexample}
%\end{graph}



\subsubsection{Group Operators}

The following keys use the |operator| key to setup operators that
connect the vertices of the current group having a certain color in a
specific way.

\begin{key}{/tikz/graphs/clique=\meta{color} (default all)}
  Adds an edge between all vertices of the current group having the
  (logical) color \meta{color}. Since, by default, this color is set
  to |all|, which is a color that all nodes get by default, when you
  do not specify anything, all nodes will be connected.
\begin{codeexample}[]
\tikz \graph [clockwise, n=5] {
  a,
  b,
  {
    [clique]
    c, d, e
  }
};    
\end{codeexample}
\begin{codeexample}[]
\tikz \graph [color class=red, clockwise, n=5] {
  [clique=red, ->]
  a, b[red], c[red], d, e[red]
};    
\end{codeexample}
\end{key}

\begin{key}{/tikz/graphs/induced independent set=\meta{color} (default all)}
  This key is the ``opposite'' of a |clique|: It removes all edges in
  the current group having belonging to color class \meta{color}. More
  precisely, an edge of kind |-!-| is added for each pair of
  vertices. This means that edge only get removed if you specify the
  |simple| option.
\begin{codeexample}[]
\tikz \graph [simple] {
  subgraph K_n [<->, n=7, clockwise]; % create lots of edges
  
  { [induced independent set] 1, 3, 4, 5, 6 }
};    
\end{codeexample}
\end{key}


\begin{key}{/tikz/graphs/cycle=\meta{color} (default all)}
  Connects the nodes colored \meta{color} is a cyclic fashion. The
  ordering is the ordering in which they appear in the whole graph
  specification.
\begin{codeexample}[]
\tikz \graph [clockwise, n=6, phase=60] {
  { [cycle, ->] a, b, c },
  { [cycle, <-] d, e, f }
};    
\end{codeexample}
\end{key}

\begin{key}{/tikz/graphs/induced cycle=\meta{color} (default all)}
  While the |cycle| command will only add edges, this key will also
  remove all other edges between the nodes of the cycle, provided we
  are constructing a |simple| graph.
\begin{codeexample}[]
\tikz \graph [simple] {
  subgraph K_n [n=7, clockwise]; % create lots of edges
  
  { [induced cycle, ->, edge=red] 2, 3, 4, 6, 7 },
};    
\end{codeexample}
\end{key}
  
\begin{key}{/tikz/graphs/path=\meta{color} (default all)}
  Works like |cycle|, only there is no edge from the last to the first
  vertex. 
\begin{codeexample}[]
\tikz \graph [clockwise, n=6] {
  { [path, ->] a, b, c },
  { [path, <-] d, e, f }
};    
\end{codeexample}
\end{key}
\begin{key}{/tikz/graphs/induced path=\meta{color} (default all)}
  Works like |induced cycle|, only there is no edge from the last to the first
  vertex. 
\begin{codeexample}[]
\tikz \graph [simple] {
  subgraph K_n [n=7, clockwise]; % create lots of edges
  
  { [induced path, ->, edges=red] 2, 3, 4, 6, 7 },
};    
\end{codeexample}
\end{key}


\subsubsection{Joining Operators}

The following keys are typically used as options of an \meta{edge
  specification}, but can also be called in a group specification
(however, then, the colors need to be set explicitly).

\begin{key}{/tikz/graphs/complete bipartite=\meta{from color}\meta{to
      color} (default \char`\{source'\char`\}\char`\{target'\char`\})}
  Adds all possible edges from every node having color \meta{from color}
  to every node having color \meta{to color}: 
\begin{codeexample}[]
\tikz \graph { {a, b}       ->[complete bipartite]
               {c, d, e}    --[complete bipartite]
               {g, h, i, j} --[complete bipartite]
               k };    
\end{codeexample}
\begin{codeexample}[]
\tikz \graph [color class=red, color class=green, clockwise, n=6] {
  [complete bipartite={red}{green}, ->]
  a [red], b[red], c[red], d[green], e[green], f[green]
};
\end{codeexample}
\end{key}
\begin{key}{/tikz/graphs/induced complete bipartite}
  Works like the |complete bipartite| operator, but in a |simple|
  graph any edges between the vertices in either shore are removed
  (more precisely, they get replaced by |-!-| edges).
\begin{codeexample}[]
\tikz \graph [simple] {
  subgraph K_n [n=5, clockwise];  % Lots of edges
  
  {2, 3} ->[induced complete bipartite] {4, 5}
};
\end{codeexample}  
\end{key}
\begin{key}{/tikz/graphs/matching=\meta{from color}\meta{to
      color} (default \char`\{source'\char`\}\char`\{target'\char`\})}
  This joining operator forms a maximum
  \emph{matching} between the nodes of the two sets of nodes having
  colors \meta{from color} and \meta{to color}, respectively. The first node of
  the from set is connected to the first node of to set, the second
  node of the from set is connected to the second node of the to set,
  and so on. If the sets have the same 
  size, what results is what graph theoreticians call a \emph{perfect
    matching}, otherwise only a maximum, but not perfect matching
  results. 
\begin{codeexample}[]
\tikz \graph {
  {a, b, c} ->[matching]
  {d, e, f} --[matching]
  {g, h}    --[matching]
  {i, j, k}
};    
\end{codeexample}
\end{key}

\begin{key}{/tikz/graphs/matching and star=\meta{from color}\meta{to
      color} (default \char`\{source'\char`\}\char`\{target'\char`\})}
  The |matching and star| connector works like the |matching|
  connector, only it behaves differently when the two to-be-connected
  sets have different size. In this case, all the surplus nodes get
  connected to the last node of the other set, resulting in what is known
  as a \emph{star} in graph theory. This simple rule allows
  for some powerful effects (since this connector is the default,
  there is no need to add it here):
  \begin{codeexample}[]
\tikz \graph { a -> {b, c} -> {d, e} -- f};    
  \end{codeexample}
  The |matching and star| connector also makes it easy to create trees and
  series-parallel graphs.
\end{key}

\begin{key}{/tikz/graphs/butterfly=\opt{\meta{options}}}
  The |butterfly| connector is used to create the kind of connections
  present between layers of a so-called \emph{butterfly network}.
  As for other connectors, two sets of nodes are connected, which are
  the nodes having color |target'| and |source'| by default. In a
  \emph{level $l$} connection, the first $l$ nodes of the first set
  are connected to the second $l$ nodes of the second set, while the
  second $l$ nodes of the first set get connected to the first $l$
  nodes of the second set. Then, for next $2l$ nodes of both sets a
  similar kind of connection is installed. Additionally, each node
  gets connected to the corresponding node in the other set with the
  same index (as in a |matching|):
\begin{codeexample}[]
\tikz \graph [left anchor=east, right anchor=west,
              branch down=4mm, grow right=15mm] {
  subgraph I_n [n=12, name=A] --[butterfly={level=3}]
  subgraph I_n [n=12, name=B] --[butterfly={level=2}]
  subgraph I_n [n=12, name=C]
};
\end{codeexample}
  Unlike most joining operators, the colors of the nodes in the first
  and the second set are not passed as parameters to the |butterfly|
  key. Rather, they can be set using the \meta{options}, which are
  executed with the path prefix |/tikz/graphs/butterfly|.
  \begin{key}{/tikz/graphs/butterfly/level=\meta{level} (initially 1)}
    Sets the level $l$ for the connections.
  \end{key}
  \begin{key}{/tikz/graphs/butterfly/from=\meta{color} (initially target')}
    Sets the color class of the from nodes.
  \end{key}
  \begin{key}{/tikz/graphs/butterfly/to=\meta{color} (initially source')}
    Sets the color class of the to nodes.
  \end{key}
\end{key}



%%% Local Variables: 
%%% mode: latex
%%% TeX-master: "pgfmanual-pdftex-version"
%%% End: 

% Copyright 2011 by Renée Ahrens, Olof Frahm, Jens Kluttig, Matthias Schulz, Stephan Schuster
% Copyright 2011 by Till Tantau
%
% This file may be distributed and/or modified
%
% 1. under the LaTeX Project Public License and/or
% 2. under the GNU Free Documentation License.
%
% See the file doc/generic/pgf/licenses/LICENSE for more details.

\section{Graph Drawing Layouts: Trees}
\label{section-first-graphdrawing-library-in-manual}
\label{section-library-graphdrawing-trees}

{\noindent {\emph{by Till Tantau}}}

\begin{tikzlibrary}{graphdrawing.trees}
  Load this package when you wish to use layout trees. You should load
  the |graphdrawing| library first. 
\end{tikzlibrary}


\tikzname\ offers several different syntax to specify trees (see
Sections \ref{section-library-graphs}
and~\ref{section-trees}). The job of the graph drawing algorithms from
this library is to turn the specification of trees into beautiful
layouts. 

We start this section with a description of algorithms, then we have a
look at how missing children can be specified and at what happens when
the input graph is not a tree.


\subsection{The Reingold--Tilford Tree Layout}

The Reingold--Tilford method is a standard method for drawing
trees. It is described in:
\begin{itemize}
\item
  E.~M.\ Reingold and J.~S.\ Tilford,
  \newblock Tidier drawings of trees,
  \newblock \emph{IEEE Transactions on Software Engineering,}
  7(2), 223--228, 1981.
\end{itemize}
My implementation in |graphdrawing.trees| follows the following paper, which
introduces some nice extensions of the basic algorithm:
\begin{itemize}
\item
  A.\ Br\"uggemann-Klein, D.\ Wood,
  \newblock Drawing trees nicely with \TeX,
  \emph{Electronic Publishing,} 2(2), 101--115, 1989.
\end{itemize}
As a historical remark, Br\"uggemann-Klein and Wood have implemented
their version of the Reingold--Tilford algorithm directly in \TeX\
(resulting in the Tree\TeX\ style). With the power of Lua\TeX\ at
our disposal, the 2012 implementation in the |graphdrawing.tree|
library is somewhat more powerful and cleaner, but it really was an
impressive achievement to implement this algorithm back in 1989
directly in \TeX. 


\begin{gdalgorithm}{tree layout}{pgf.gd.trees.ReingoldTilford1981}
  The basic idea of the Reingold--Tilford algorithm is to use the
  following rules to position the nodes of a tree (the following
  description assumes that the tree grows downwards, other growth
  directions are handled by the automatic orientation mechanisms of
  the graph drawing library): 
  \begin{enumerate}
  \item For a node, recursively compute a layout for each of its children.
  \item Place the tree rooted at the first child somewhere on the page.
  \item Place the tree rooted at the second child to the right of the
    first one as near as possible so that no two nodes touch (and such
    that the |sibling sep| padding is not violated).
  \item Repeat for all subsequent children.
  \item Then place the root above the child trees at the middle
    position, that is, at the half-way point between the left-most and
    the right-most child of the node.
  \end{enumerate}
\begin{codeexample}[]
\tikz [tree layout, sibling distance=8mm]
\graph [nodes={circle, draw, inner sep=1.5pt}]{
  1 -- { 2 -- 3 -- { 4 -- 5, 6 -- { 7, 8, 9 }}, 10 -- 11 -- { 12, 13 } }
};
\end{codeexample}
  The standard keys |level distance|, |level sep|, |sibling distance|,
  and |sibling sep|, as well as the |pre| and |post| versions of these
  keys, as taken into consideration when nodes are positioned. See also
  Section~\ref{subsection-gd-dist-pad} for details on these keys. 
\begin{codeexample}[]
\tikz [tree layout, grow=-30,
       sibling distance=0mm, level distance=0mm,]
\graph [nodes={circle, draw, inner sep=1.5pt}]{
  1 -- { 2 -- 3 -- { 4 -- 5, 6 -- { 7, 8, 9 }}, 10 -- 11 -- { 12, 13 } }
};
\end{codeexample}

  \noindent\textbf{Handling of Missing Children.}
  As described in Section~\ref{section-gd-missing-children}, you can
  specify that some child nodes are ``missing'' in the tree, but some
  space should be reserved for them. This is exactly what happens:
  When the subtrees of the children of a node are arranged, each
  position with a missing child is treated as if a zero-width,
  zero-height subtree were present at that positions:
\begin{codeexample}[]
\tikz [tree layout, nodes={draw,circle}]
\node {r}
  child { node {a}
    child [missing]
    child { node {b} }
  }
  child[missing];
\end{codeexample}
  or in |graph| syntax:
\begin{codeexample}[]
\tikz \graph [tree layout, nodes={draw,circle}]
{
  r -> {
    a -> {
      , %missing
      b},
    % missing
  }
};
\end{codeexample}
  More than one child can go missing:
\begin{codeexample}[]
\tikz \graph [tree layout, nodes={draw,circle}, sibling sep=0pt]
{ r -> { a, , ,b -> {c,d}, ,e} };
\end{codeexample}
  Although missing children are taken into consideration for the
  computation of the placement of the children of a root node relative
  to one another and also for the computation of the position of the
  root node, they are usually \emph{not} considered as part of the
  ``outline'' of a subtree (the \texttt{minimum number of children}
  key ensures that |b|, |c|, |e|, and |f| all have a missing right
  child): 
\begin{codeexample}[]
\tikz \graph [tree layout, minimum number of children=2,
              nodes={draw,circle}]
{ a -> { b -> c -> d, e -> f -> g } };
\end{codeexample}
  This behaviour of ``ignoring'' missing children in later stages of
  the recursion can be changed using the following key:
  \begin{key}{/graph drawing/tree layout/missing nodes get
      space=\meta{true or false}}
    When set to true, the missing children are treated as if there
    where zero-width, zero-height nodes during the whole tree layout
    process: 
\begin{codeexample}[]
\tikz \graph [tree layout={missing nodes get space},
              minimum number of children=2, nodes={draw,circle}]
{ a -> { b -> c -> d, e -> f -> g } };
\end{codeexample}
  \end{key}
  
  \noindent\textbf{Significant Pairs of Siblings.}
  Br\"uggemann-Klein and Wood have proposed an extension of the
  Reingold--Tilford method that is intended to better highlight the
  overall structure of a tree. Consider the following two trees:

\begin{codeexample}[]
\tikz [baseline=(a.base), tree layout, minimum number of children=2,
       sibling distance=5mm, level distance=5mm]
\graph [nodes={circle, inner sep=0pt, minimum size=2mm, fill, as=}]{
  a -- { b -- c -- { d -- e, f -- { g, h }}, i -- j -- k[second] }
};\quad
\tikz [baseline=(a.base), tree layout, minimum number of children=2,
       sibling distance=5mm, level distance=5mm]
\graph [nodes={circle, inner sep=0pt, minimum size=2mm, fill, as=}]{
  a -- { b -- c -- d -- e, i -- j -- { f -- {g,h}, k } }
};
\end{codeexample}  
  As observed by Br\"uggemann-Klein and Wood, the two trees are
  structurally quite different, but the Reingold--Tilford method
  places the nodes at exactly the same positions and only one edge
  ``switches'' positions. In order to better highlight the differences
  between the trees, they propose to add a little extra separation
  between siblings that form a \emph{significant pair}. They define
  such a pair as follows: Consider the subtrees of two adjacent
  siblings. There will be one or more levels where these subtrees have
  a minimum distance. For instance, the following two trees the
  subtrees of the nodes |a| and |b| have a minimum distance only at
  the top level in the left example, and in all levels in the second
  example. A \emph{significant pair} is a pair of siblings where the
  minimum distance is encountered on any level other than the first
  level. Thus, in the first example there is no significant pair,
  while in the second example |a| and |b| form such a pair.
\begin{codeexample}[]
\tikz \graph [tree layout, minimum number of children=2,
              level distance=5mm, nodes={circle,draw}]
{ / -> { a -> / -> /, b -> /[second] -> /[second] }};
\quad
\tikz \graph [tree layout, minimum number of children=2,
              level distance=5mm, nodes={circle,draw}]
{ / -> { a -> / -> /, b -> / -> / }};
\end{codeexample}
  Whenever the algorithm encounters a significant pair, it adds the
  following extra space between the siblings:
  \begin{key}{/graph drawing/tree layout/significant
      sep=\meta{dimension} (initially 0em)}
\begin{codeexample}[]
\tikz [baseline=(a.base), tree layout={significant sep=1em},
       minimum number of children=2,
       sibling distance=5mm, level distance=5mm]
\graph [nodes={circle, inner sep=0pt, minimum size=2mm, fill, as=}]{
  a -- { b -- c -- { d -- e, f -- { g, h }}, i -- j -- k[second] }
};\quad
\tikz [baseline=(a.base), tree layout={significant sep=1em},
       minimum number of children=2,
       sibling distance=5mm, level distance=5mm]
\graph [nodes={circle, inner sep=0pt, minimum size=2mm, fill, as=}]{
  a -- { b -- c -- d -- e, i -- j -- { f -- {g,h}, k } }
};
\end{codeexample}      
  \end{key}
\end{gdalgorithm}


\begin{gdalgorithm}{binary tree layout}{pgf.gd.trees.ReingoldTilford1981}
  This key executes:
  \begin{enumerate}
  \item |tree layout|, thereby selecting the Reingold--Tilford method,
  \item |minimum number of children=2|, thereby ensuring the all nodes
    have (at least) two children or none at all, and
  \item |tree layout/significant sep=1em| to highlight significant pairs.
  \end{enumerate}
\begin{codeexample}[]
\tikz [grow'=up, binary tree layout, sibling distance=7mm, level distance=7mm]
\graph {
  a -- { b -- c -- { d -- e, f -- { g, h }}, i -- j -- k[second] }
};
\end{codeexample}      
  Here  is a further example, a tree taken from the paper of
  Br\"uggemann-Klein and Wood. The right tree demonstrates nicely the
  advantages of having the full power of \tikzname's anchoring and the
  graph drawing engine's orientation mechanisms at one's disposal: 
\begin{codeexample}[]
\tikz \graph [binary tree layout] {
  Knuth -> {
    Beeton -> Kellermann [second] -> Carnes,
    Tobin -> Plass -> { Lamport, Spivak } 
  }
};\qquad
\tikz [>=spaced stealth']
  \graph [binary tree layout, grow'=right, level sep=1.5em,
          nodes={right, fill=blue!50, text=white, chamfered rectangle},
          edges={decorate,decoration={snake, post length=5pt}}] 
{
  Knuth -> {
    Beeton -> Kellermann [second] -> Carnes,
    Tobin -> Plass -> { Lamport, Spivak } 
  }
};
\end{codeexample}
\end{gdalgorithm}


\begin{gdalgorithm}{extended binary tree layout}{pgf.gd.trees.ReingoldTilford1981}
  This algorithm is similar to |binary tree layout|, only the option
  \texttt{missing nodes get space} is executed and the
  \texttt{significant sep} is zero.
\begin{codeexample}[]
\tikz [grow'=up, extended binary tree layout,
       sibling distance=7mm, level distance=7mm]
\graph {
  a -- { b -- c -- { d -- e, f -- { g, h }}, i -- j -- k[second] }
};
\end{codeexample}      
\end{gdalgorithm}


\subsection{Specifying Missing Children}

\label{section-gd-missing-children}

In the present section we discuss keys for specifying missing children
in a tree. For many certain kind of trees, in particular for binary
trees, there are not just ``a certain number of children'' at each
node, but, rather, there is a designated ``first'' (or ``left'') child
and a ``second'' (or ``right'') child. Even if one of these children
is missing and a node actually has only one child, the single child will
still be a ``first'' or ``second'' child and this information should
be taken into consideration when drawing a tree.

The first useful key for specifying missing children is the following:
\begin{key}{/graph drawing/minimum number of children=\meta{number}
    (initially 0)}
  \keyalias{tikz}\keyalias{tikz/graphs}
  When this key is set to 2 or more, the following happens: We first
  compute a spanning tree for the graph, see
  Section~\ref{subsection-gd-spanning-tree}. Then, whenever a node is
  not a leaf in this spanning tree (when it has at least one child),
  we add ``virtual'' or ``dummy'' nodes as children of the node until
  the total number of real and dummy children is at least
  \meta{number}. If there where at least \meta{number} children at the
  beginning, nothing happens.

  The new children are added after the existing children. This means
  that, for instance, in a tree with \meta{number} set to |2|, for
  every node with a single child, this child will be the first child
  and the second child will be missing.
\begin{codeexample}[]
\tikz \graph [binary tree layout,level distance=5mm]
{ a -> { b->c->d, e->f->g } };  
\end{codeexample}
\end{key}

Once the minimum number of children has been set, we still need a way
of specifying ``missing first children'' or, more generally, missing
children that are not ``at the end'' of the list of children. For
this, there are three methods:

\begin{enumerate}
\item When you use the |child| syntax, you can use the |missing| key
  with the |child| command to indicate a missing child:
\begin{codeexample}[]
\tikz [binary tree layout, level distance=5mm]
\node {a}
child { node {b}
  child { node {c}
    child { node {d} }
} }
child { node {e}
  child [missing]
  child { node {f}
    child [missing]
    child { node {g}
} } };
\end{codeexample}
\item When using the |graph| syntax, you can use an ``empty node'',
  which really must be completely empty and may not even contain a
  slash, to indicate a missing node:
\begin{codeexample}[]
\tikz [binary tree layout, level distance=5mm]
\graph { a -> { b -> c -> d, e -> { , f -> { , g} } } };
\end{codeexample}
\item You can simply specify the index of a child directly, see below.
\end{enumerate}

\begin{key}{/graph drawing/desired child index=\meta{number}}
  Pass this key to a node to tell the graph drawing engine which child
  number you ``desired'' for the node. Whenever all desires for the
  children of a node are conflict-free, they will all be met; children
  for which no desired indices were given will remain at their
  position, whenever possible, but will ``make way'' for children with
  a desired position. In detail, the following happens: We first
  determine the total number of children (real or dummy) needed, which
  is the maximum of the actual number of children, of the
  \texttt{minimum number of children}, and of the highest desired
  child index. Then we go over all children that have a desired child
  index and put they at this position. If the position is already
  taken (because some other child had the same desired index), the
  next free position is used with a wrap-around occurring at the
  end. Next, all children without a desired index are place using the
  same mechanism, but they want to be placed at the position they had
  in the original spanning tree.

  While all of this might sound a bit complicated, the application of
  the key in a binary tree is pretty straightforward: To indicate that
  a node is a ``right'' child in a tree, just add \texttt{desired child index=2}
  to it. This will make it a second child, possibly causing the fist
  child to be missing. If there are two nodes specified as children of
  a node, by saying \texttt{desired child index=}\meta{number} for one
  of them, you will cause it be first or second child, depending on
  \meta{number}, and cause the \emph{other} child to become the other
  child.

  Since |desired child index=2| is a bit long, the following shortcuts
  are available:
  \begin{key}{/graph drawing/first}
    \keyalias{tikz}\keyalias{tikz/graphs}
    A shorthand for |desired child index=1|.    
  \end{key}
  \begin{key}{/graph drawing/second}
    \keyalias{tikz}\keyalias{tikz/graphs}
    A shorthand for |desired child index=2|.    
  \end{key}
  \begin{key}{/graph drawing/third}
    \keyalias{tikz}\keyalias{tikz/graphs}
    A shorthand for |desired child index=3|.    
  \end{key}
  \begin{key}{/graph drawing/fourth}
    \keyalias{tikz}\keyalias{tikz/graphs}
    A shorthand for |desired child index=4|.    
  \end{key}
\begin{codeexample}[]
\tikz \graph [binary tree layout, level distance=5mm]
{ a -> b[second] };  
\end{codeexample}
\begin{codeexample}[]
\tikz \graph [binary tree layout, level distance=5mm]
{ a -> { b[second], c} };  
\end{codeexample}
\begin{codeexample}[]
\tikz \graph [binary tree layout, level distance=5mm]
{ a -> { b, c[first]} };  
\end{codeexample}
\begin{codeexample}[]
\tikz \graph [binary tree layout, level distance=5mm]
{ a -> { b[second], c[second]} };  
\end{codeexample}
\begin{codeexample}[]
\tikz \graph [binary tree layout, level distance=5mm]
{ a -> { b[third], c[first], d} };  
\end{codeexample}
  You might wonder why |second| is used rather than |right|. The
  reason is that trees may also grow left and right and, additionally,
  the |right| and |left| keys are already in use for
  anchoring. Naturally, you can locally redefine them, if you want.
\end{key}


\subsection{Spanning Tree Computation}
\label{subsection-gd-spanning-tree}
Although the algorithms of this library are tailored to layout trees,
they will work for any graph as input. First, if the graph is not
connected, it is decomposed into connected components and these are
laid out individiually. Second, for each component, a spanning tree of
the graph is computed first and the layout is computed for this
spanning tree; all other edges will still be drawn, but they have no
impact on the placement of the nodes. If the graph is already a tree,
the spanning tree will be the original graph.

The computation of the spanning tree is a non-trivial process since
a non-tree graph has many different possible spanning trees. You can
choose between different methods for deciding on a spanning tree, it
is even possible to implement new algorithms. (In the future, the
computation of spanning trees and the cylce removal in layered graph
drawing algorithms will be unified, but, currently, they are
implemented differently.) 

\begin{key}{/graph drawing/spanning tree algorithm=\meta{algorithm}}
  \keyalias{tikz}\keyalias{tikz/graphs}
  Selects the (sub)algorithm that is to be used for computing spanning
  trees whenever this is requested by a tree layout algorithm. The
  default algorithm is |pgf.gd.trees.BreadthFirst|.
\begin{codeexample}[]
\tikz \graph [tree layout,
  spanning tree algorithm=pgf.gd.trees.BreadthFirst]
{
  1 -- {2,3,4,5} -- 6;    
};   
\end{codeexample}
\begin{codeexample}[]
\tikz \graph [tree layout,
  spanning tree algorithm=pgf.gd.trees.DepthFirst]
{
  1 --[bend right] {2,3,4,5 [>bend left]} -- 6;    
};   
\end{codeexample} 
\end{key}

Usually, you do not use the |spanning tree subalgorithm| option directly,
but use one of the below keys to select a method.

\begin{gdalgorithm}{breadth first spanning tree}{pgf.gd.trees.BreadthFirst}
  This key selects ``breadth first'' as the (sub)algorithm for
  computing spanning trees. Note that this key does not cause a graph
  drawing scope to start; the key only has an effect in conjunction
  with keys like |tree layout|.

  The algorithm will be called whenever a graph drawing algorithm
  needs a spanning tree on which to operate. It works as follows:
  
  \begin{enumerate}
  \item It looks for a node for which the following graph parameter is
    set:
    \begin{key}{/graph drawing/root}
      \keyalias{tikz}\keyalias{tikz/graphs}
    \end{key}
    If there are several such nodes, the first one is used. If there
    are no such nodes, the first node of indegree 0 (with respect to
    |->| edges) is used. If there is no such node, the first node is
    used.

    Let call the node determined in this way the \emph{root node}.
  \item For every edge, a \emph{priority} is determined, which is a
    number between 1 and 5. How this happens, exactly, will be
    explained in a moment.
  \item Starting from the root node, we now perform a breadth first
    search through the tree, thereby implicitly building a spanning
    tree: Suppose for a moment that all edges have priority~1. Then,
    the algorithm works just the way that a normal breadth first
    search is performed: We keep a queue of to-be-visited nodes and
    while this queue is not empty, we remove its first node. If this
    node has not yet been visited, we add all its neighbors at the
    end of the queue. When a node is taken out of the queue, we make
    it the child of the node whose neighbor it was when it was
    added. Since the queue follows the ``first in, first out''
    principle (it is a fifo queue), the children of the root will be
    all nodes at distance $1$ form the root, their children will be
    all nodes at distance $2$, and so on. 
  \item Now suppose that some edges have a priority different
    from~1, in which case things get more complicated. We now keep
    track of one fifo queue for each of the five possible
    priorities. When we consider the neighbors of a node, we actually
    consider all its incident edges. Each of these edges has a certain
    priority and the neighbor is put into the queue of the edge's
    priority. Now, we still remove nodes normally from the queue for
    priority~1; only if this queue is empty and there is still a node
    in the queue for priority~2 we remove the first element from this
    queue (and proceed as before). If the second queue is also empty,
    we try the third, and so on up to the fifth queue. If all queues
    are empty, the algorithm stops.
  \end{enumerate}

  The effect of the five queues is the following: If the edges of
  priority $1$ span the whole graph, a spanning tree consisting solely
  of these edges will be computed. However, if they do not, once we
  have visited reachable using only priority 1 edges, we will extend
  the spanning tree using a priority 2 edge; but then we once switch
  back to using only priority 1 edges. If neither priority~1 nor
  priority~2 edges suffice to cover the whole graph, priority~3 edges
  are used, and so on.
  
  You can directly specify the priority of an edge using the following
  edge parameter:
  \begin{key}{/graph drawing/edge priority=\meta{number}}
    \keyalias{tikz}\keyalias{tikz/graphs}
    Sets the priority of the edge to \meta{number}, which must be
    |1|, |2|, |3|, |4|, or  |5|.      
  \end{key}
  Remember that priority |1| edges are always preferred over other
  edges. For this reason, there is the following key:
  \begin{key}{/graph drawing/spanning tree edge}
    \keyalias{tikz}\keyalias{tikz/graphs}
    An easy-to-remember shorthand for |edge priority=1|. When this key
    is used with an edge, it will always be preferred over other edges
    when it comes to choosing which edges are part of the spanning tree.    
  \end{key}
  \begin{key}{/graph drawing/no spanning tree edge}
    \keyalias{tikz}\keyalias{tikz/graphs}
    An easy-to-remember shorthand for |edge priority=5|. This causes
    the edge to be used only as a last resort as part of a spanning
    tree. 
  \end{key}
  
  Instead of specifying priorities ``by hand'', you can also let them
  be computed automatically. For all edges without an explicitly
  specified priority the following key is used to determine it:
  \begin{key}{/graph drawing/edge priority method=\meta{method}}
    \keyalias{tikz}\keyalias{tikz/graphs}
    There are currently the following choices for \meta{methods}:
    \begin{itemize}
    \item \declare{|forward first|} (this is the default method)
      
      For two nodes $u$ and $v$ connected by an edge $e$, this method
      assigns the priority 2 to $e$ if it is a |->| edge from $u$
      to~$v$; it assigns $3$ to all undirected and bidirected edges
      (|--| and |<->| edges); and $4$ to all other edges (|<-| edges
      and |->| edge from $v$ to $u$).

      In practice, this means that the algorithm will favor
      forward-going edges and whenever a spanning tree can be computed
      using only the |->| edges, this spanning tree will be used. Only
      when these edges no longer suffice to build a spanning tree, we
      will also consider undirected or bidirected edges. The
      ``backward'' edges are only used as a last resort.
\begin{codeexample}[]
\tikz \graph [tree layout, edge priority method=forward first]        
{
  1 -> 2 -> 3 -- 4, 2 -> 4;
  a -- b -- c -> d, b --[bend right] d;
};
\end{codeexample}
    \item \declare{|undirected first|}
      
      This method assigns priority 2 to |--| and |<->| edges,
      priority~3 to |->| edges, and priority~4 to |<-| edges.
\begin{codeexample}[]
\tikz \graph [tree layout, edge priority method=undirected first]        
{
  1 -> 2 -> 3 -- 4, 2 -> [bend right] 4;
  a -- b -- c -> d, b -- d;
};
\end{codeexample}
      An interesting application of this method is in creating trees
      where siblings are automatically connected by directed edges:
\begin{codeexample}[]
\tikz \graph [tree layout, nodes={draw}, sibling distance=0pt,
              edge priority method=undirected first,
              every group/.style={
                default edge kind=->,
                path=source % connect every group as a path
              }] 
{
  5 -- {
    "1,3" -- {0,2,4},
    11    -- {
      "7,9" -- { 6, 8, 10 }
    }
  }
};
\end{codeexample}     
      Note that in the above example, we could also have used the key
      |no spanning tree edge| inside the |every group|. This would
      also cause the added edges to be ignored:
\begin{codeexample}[]
\tikz \graph [tree layout, nodes={draw}, sibling distance=0pt,
              every group/.style={
                default edge kind=->, no spanning tree edge,
                path=source}] 
{
  5 -> {
    "1,3" -> {0,2,4},
    11    -> {
      "7,9" -> { 6, 8, 10 }
    }
  }
};
\end{codeexample}     
    \item \declare{|directed first|}
      
      Assigns priority 2 to directed edges that ``go along their
      direction''. That is, for an edge between $u$ and $v$, we get
      priority $2$ for |->|, for |<->|, and also for |<-|, but only
      when going from $v$ to $u$. Undirected edges get priority~3, all
      other edges get priority 4.

      This strategy is nice with trees specified with both forward and
      backward edges:
\begin{codeexample}[]
\tikz \graph [tree layout, nodes={draw}, sibling distance=0pt,
              edge priority method=directed first]
{
  3 <- 5[root] -> 8,
  1 <- 3 -> 4,
  7 <- 8 -> 9,
  1 -- 4 -- 7 -- 9
};
\end{codeexample}
    \item \declare{|all the same|}
      
      Assigns priority 3 to all edges, meaning that they are ``all the same.''
    \end{itemize}
  \end{key}
\end{gdalgorithm}

\begin{gdalgorithm}{depth first spanning tree}{pgf.gd.trees.DepthFirst}
  Works exactly like |depth first spanning tree| (same handling of
  priorities by five queues), only the queues are now lifo instead of
  fifo. 
\end{gdalgorithm}


%%% Local Variables: 
%%% mode: latex
%%% TeX-master: "pgfmanual-pdftex-version"
%%% End: 

% Copyright 2008 by Mark Wibrow
%
% This file may be distributed and/or modified
%
% 1. under the LaTeX Project Public License and/or
% 2. under the GNU Public License.
%
% See the file doc/generic/pgf/licenses/LICENSE for more details.

\section{Lindenmayer System Drawing Library}
\subsection{Overview}

Lindenmayer systems (also commonly known as ``L-systems''), were
originally developed by Aristid Lindenmayer as a theory of algae 
growth patterns and then subsequently used to model branching 
patterns in plants and produce fractal patterns.
Typically, an L-system consists of a set of symbols, each of which
is associated with some graphical action (such as ``turn left'' or 
``move forward'') and a set of rules (``production'' or ``rewrite'' 
rules). Given a string of symbols, the rewrite rules are applied 
several times and the when resulting string is processed the action 
associated with each symbol is executed. 

In \pgfname, L-systems can be used to create simple 2-dimensional
fractal patterns\ldots	
\begin{codeexample}[pre={\expandafter\let\csname pgf@lsystem@Koch curve\endcsname=\relax}]
\begin{tikzpicture}
\pgfdeclarelindenmayersystem{Koch curve}{
  \rule{F -> F-F++F-F}
}

\filldraw [fill=blue!50, draw=blue!50!black]
  [l-system={Koch curve, step=2pt, angle=60, axiom=F++F++F, order=3}]
  lindenmayer system -- cycle;
\end{tikzpicture}
\end{codeexample}

\noindent\ldots and ``plant like'' patterns\ldots

\begin{codeexample}[]
\begin{tikzpicture}
\draw [green!50!black, rotate=90]
  [l-system={rules={F -> FF-[-F+F]+[+F-F]}, axiom=F, order=4, step=2pt,
   randomize step percent=25, angle=30, randomize angle percent=5}]
  lindenmayer system;
\end{tikzpicture}
\end{codeexample}

\noindent
\ldots but it is important to bear in mind that even moderately
complex L-systems can exceed the available memory of \TeX, 
and can be very slow. 
If possible, you are advised to increase the main memory and save 
stack to their maximum possible values for your particular 
\TeX{} distribution. 
However, even by doing this you may find you still run out of memory
quite quickly.

For an excellent introduction to L-systems (containing some
``really cool'' pictures -- many of which are sadly not possible in 
\pgfname)
see \emph{The Algorithmic Beauty of Plants} by 
Przemyslaw Prusinkiewicz and Aristid Lindenmayer (which is freely
available via the internet). 

\begin{pgflibrary}{lindenmayersystems}
  This \pgfname-library provides basic commands for defining and using 
  simple L-systems. The \tikzname-library provides, furthermore, a
  front end for using L-systems in  \tikzname. 
\end{pgflibrary}



\subsubsection{Declaring L-systems}
  Before an L-system can be used, it must be declared using the
  following command:
  
\begin{command}{\pgfdeclarelindenmayersystem\marg{name}\marg{specification}}

This command declares a Lindenmayer system called \meta{name}.
The \meta{specification} argument contains a description of the
L-system's symbols and rules. Two commands |\symbol| and |\rule| are
only defined when the \meta{specification} argument is executed.

\begin{command}{\symbol\marg{name}\marg{code}}
  This defines a symbol called \meta{name} for a specific L-system,
  and associates it with \meta{code}. 
  
  A symbol should consist of a single 
  alpha-numeric character (i.e., |A|-|Z|, |a|-|z| or |0|-|9|). 
  The symbols
  |F|, |f|, |+|, |-|, |[| and |]| are available by default so do 
  not need to be defined for each L-system. However, if you are
  feeling adventurous, they can be redefined for specific L-systems 
  if required. The L-system treats the default symbols as follows
  (the commands they execute are described below):
  
  \begin{itemize}
  	\item 
  	|F| move forward a certain distance, drawing a line. Uses
  	|\pgflsystemdrawforward|.
  	
  	\item 
  	|f| move forward a certain distance, without drawing a line.
  	Uses |\pgflsystemmoveforward|.
  	
  	\item 
  	|+| turn left by some angle.
  	Uses |\pgflsystemturnleft|.
  	
  	\item 
  	|-| turn right by some angle.
  	Uses |\pgflsystemturnright|.
  	
  	\item 
  	|[| save the current state (i.e., the position and direction).
  	Uses |\pgflsystemsavestate|.
  	
  	\item 
  	|]| restore the last saved state.
  	Uses |\pgflsystemrestorestate|.
  	 
  \end{itemize}
  
  The symbols |[| and |]| act like a stack: |[| pushes the state of the
  L-system on to the stack, and |]| pops a state off the stack. 
  
   When \meta{code} is executed the transformation matrix is set up
  so that the origin is at the current position and the positive 
  x-axis ``points forward'', so |\pgfpathlineto{\pgfpoint{1cm}{0cm}}| 
  draws a line 1cm forward.

The following keys store values which can alter the production of an
L-system. 

\begin{key}{/pgf/lindenmayer system/step=\meta{length} (initially 5pt)}
  This key sets how far the L-system moves forward.
\end{key}

\begin{key}{/pgf/lindenmayer system/randomize step percent=\meta{percentage} (initially 0)}
  If the step is to be randomized, this key specifies by how much. 
\end{key}

\begin{key}{/pgf/lindenmayer system/left angle=\meta{angle} (initially 90)}
  This key sets the angle through which the L-system turns when it
  turns left.
\end{key}

\begin{key}{/pgf/lindenmayer system/right angle=\meta{angle} (initially 90)}
  This key sets the angle through which the L-system turns when it
  turns right.
\end{key}

\begin{key}{/pgf/lindenmayer system/randomize angle percent=\meta{percentage} (initially 0)}
  If the angles are to be randomized, this key specifies by how much. 
\end{key}

For speed and convenience, when the code for a symbol is executed the 
following commands are available.

\begin{command}{\pgflsystemcurrentstep}
	The current ``step'' of the L-system (i.e., how far the system
	will move foward if required). This is initially set to the
	value in the \TeX-dimensions |\pgflsystemstep|, but the actual 
	value may be changed if |\pgflsystemrandomizestep| is used 
	(see below). 
\end{command}

\begin{command}{\pgflsystemleftangle}
	The angle the L-system will turn when it turns left. 
	The value stored in this macro may be changed if 
	|\pgflsystemrandomizeleftangle| is used. 
\end{command}

\begin{command}{\pgflsystemrightangle}
	The angle the L-system will turn when it turns right. 
	The value stored in this macro may be changed if 
	|\pgflsystemrandomizerightangle| is used. 
\end{command}


The following commands may be useful if you wish to define your own
symbols.

\begin{command}{\pgflsystemrandomizestep}
	Randomizes the value in |\pgflsystemcurrentstep| according to the value of
	the |randomize| |step| |percent| key.
\end{command}

\begin{command}{\pgflsystemrandomizeleftangle}
	Randomizes the value in |\pgflsystemleftangle| according to the value of
	the |randomize| |angle| |percent| key.
\end{command}

\begin{command}{\pgflsystemrandomizerightangle}
	Randomizes the value in |\pgflsystemrightangle| according to the value of
	the |randomize| |angle| key.
\end{command}

\begin{command}{\pgflsystemdrawforward}
	Move forward in the current direction, by |\pgflsystemcurrentstep|,
	drawing a line in the process. This macro calls 
	|\pgflsystemrandomizestep|. Internally, \pgfname{} simply
	shifts the transformation matrix in the positive direction of 
	the current (transformed) x-axis by |\pgflsystemstep| 
	and then executes	a line-to to the (newly transformed) origin.
\end{command}

\begin{command}{\pgflsystemmoveforward}
	Move forward in the current direction, by |\pgflsystemcurrentstep|,
	without drawing a line. This macro calls 
	|\pgflsystemrandomizestep|. \pgfname{} executes a transformation
	as abvove, but executes	a move-to to the (newly transformed) 
	origin.
\end{command}

\begin{command}{\pgflsystemturnleft}
  Turn left by |\pgflsystemleftangle|. This macro calls 
	|\pgflsystemrandomizeleftangle|. Internally, \pgfname{}
	simply rotates the transformation matrix.
\end{command}

\begin{command}{\pgflsystemturnright}
	Turn right by |\pgflsystemrightangle|. This macro calls 
	|\pgflsystemrandomizerightangle|. Internally, \pgfname{}
	simply rotates the transformation matrix.
\end{command}

\begin{command}{\pgflsystemsavestate}
	Save the current position and orientation. Internally,
	\pgfname{} simply starts a new \TeX-group. 
\end{command}

\begin{command}{\pgflsystemrestorestate}
	Restore the lased saved position and orientation. Internally,
	\pgfname{} closes a \TeX-group, restoring the transformation 
	matrix of the outer scope, and a move-to command is executed to
	the (transformed) origin.
\end{command}
  

\end{command}

\begin{command}{\rule{\ttfamily\char`\{}\meta{head}{\ttfamily->}\meta{body}{\ttfamily\char`\}}}
  Declare a rule. \meta{head} should consist of a single symbol, which
  need not have been declared using |\symbol| or exist as a defualt
  symbol (in fact, the more intersting L-systems depend on using
  symbols with no corresponding code, to control the ``growth'' of the
  system).
 	\meta{body} consists of a string of symbols, which again need not
 	necessarily have any code associated with them.
 	
\end{command}

  As an example, the following shows an L-system that uses
  some of these commands. This example illustrates the point
  that some symbols, in this case |A| and |B|, do not have to 
  have code associated with them. They simply control the
  growth of the system.

\begin{codeexample}[pre={\nullfont\expandafter\let\csname pgf@lsystem@Hilbert curve\endcsname=\relax}]
\pgfdeclarelindenmayersystem{Hilbert curve}{
  \symbol{X}{\pgflsystemdrawforward}
  \symbol{+}{\pgflsystemturnright} % Explicitly define + and - symbols.
  \symbol{-}{\pgflsystemturnleft}
  \rule{A -> +BX-AXA-XB+}
  \rule{B -> -AX+BXB+XA-}
}
\tikz\draw[lindenmayer system={Hilbert curve, axiom=A, order=4, angle=90}]
  lindenmayer system;
\end{codeexample}


\end{command}

\subsection{Using Lindenmayer Systems}
\subsubsection{Using L-Systems in PGF}

The following command is used to run an L-system in \pgfname:
\begin{command}{\pgflindenmayersystem\marg{name}\marg{axiom}\marg{order}}
  Runs the L-system called \meta{name} using the input string \meta{axiom}
  for \meta{order} iteraions.
  In general, prior to calling this command the 
  transformation matrix should be set appropriately for shifting and
  rotating, and a move-to to the (transformed) origin should be 
  executed. This origin will be where the L-system starts.
  In addition the relavent keys should be set appropriately.
  
\begin{codeexample}[]
\begin{tikzpicture}
  \draw [help lines] grid (3,2);
  \pgfset{lindenmayer system/.cd, angle=60, step=2pt}
  \foreach \x/\y in {0cm/1cm, 1.5cm/1.5cm, 2.5cm/0.5cm, 1cm/0cm}{
    \pgftransformshift{\pgfqpoint{\x}{\y}}
    \pgfpathmoveto{\pgfpointorigin}
    \pgflindenmayersystem{Koch curve}{F++F++F}{2}
    \pgfusepath{stroke}
  }
\end{tikzpicture}
\end{codeexample}

  Note that, it is perfectly feasible for an L-system to define
  special symbols which perform the move-to and use-path 
  operations.
  
\end{command}

\subsubsection{Using L-Systems in Ti\emph{k}Z}

  In \tikzname, an L-system is created using a path operation. 
  However, \tikzname{} is more flexible regarding the positioning
  of the L-system and also provides keys to create L-systems
  ``on-line''.
  
\begin{pathoperation}{lindenmayer system}{ \opt{|[|\meta{keys}|]|}}
  This will run an L-system according to the parameters
  specified in \meta{keys} (which can also contain normal \tikz{} keys
  such as |draw| or |thin|). The syntax is flexible
  regarding the L-system parameters and the following all do
  the same thing:

\begin{codeexample}[code only]
\draw lindenmayer system [lindenmayer system={Hilbert curve, axiom=4, order=3}];
\end{codeexample} 

\begin{codeexample}[code only]
\draw [lindenmayer system={Hilbert curve, axiom=4, order=3}] lindenmayer system;
\end{codeexample} 

\begin{codeexample}[code only]
\tikzset{lindenmayer system={Hilbert curve, axiom=4, order=3}}
\draw lindenmayer system;
\end{codeexample} 

\end{pathoperation}

\begin{pathoperation}{l-system}{ \opt{|[|\meta{keys}|]|}}
  A more compact version of the |lindenmayer system| path command.
\end{pathoperation}

This library adds some additional keys for specifying L-systems.
These keys only work in \tikzname{} and all
have the same path, namely, |/pgf/lindenmayer| |system|, but so 
you do not have to keep repeating this path the following keys are 
provided:
 
\begin{stylekey}{/pgf/lindenmayer system=\marg{keys}}
\keyalias{tikz}
This key changes the key path to |/pgf/lindenmayer systems| and
executes \meta{keys}.
\end{stylekey}

\begin{stylekey}{/pgf/l-system=\marg{keys}}
\keyalias{tikz}
A more compact version of the previous key.
\end{stylekey}

\begin{key}{/pgf/lindenmayer system/name=\marg{name}}
  Set the name for the L-system. 
\end{key}

\begin{key}{/pgf/lindenmayer system/axiom=\marg{string}}
  Set the axiom (or input string) for the L-system. 
\end{key}

\begin{key}{/pgf/lindenmayer system/order=\marg{integer}}
  Set the number of iterations the L-system will perform.
\end{key}

\begin{key}{/pgf/lindenmayer system/rules=\marg{rule list}}
  This key allows an (anonymous) L-system to be declared ``on-line''.
  There is, however, a restriction that only the defualt symbols can be
  used for drawing (empty symbols can still be used to control
  the growth of the system). The rules in \meta{rule list} should
  be separated by commas.
  
\begin{codeexample}[]
\tikz[rotate=65]\draw [green!60!black] l-system
  [l-system={rules={F -> F[+F]F[-F]}, axiom=F, order=4, angle=25,step=3pt}];
\end{codeexample} 
\end{key}

\begin{key}{/pgf/lindenmayer system/anchor=\meta{anchor}}
  Be default, when this key is not used, the L-system will start from 
  the last specified coordinate. By using this key, the L-system
  will be placed inside a special (rectangle) node which can be
	positioned using \meta{anchor}.

 
\begin{codeexample}[]
\begin{tikzpicture}[l-system={step=1.75pt, order=5, angle=60}]
  \pgfdeclarelindenmayersystem{Sierpinski triangle}{
    \symbol{X}{\pgflsystemdrawforward}
    \symbol{Y}{\pgflsystemdrawforward}
    \rule{X -> Y-X-Y}
    \rule{Y -> X+Y+X}
  }
  \draw [help lines] grid (3,2);
  \draw [red] (0,0) l-system 
    [l-system={Sierpinski triangle, axiom=+++X, anchor=south west}];
  \draw [blue] (3,2) l-system 
    [l-system={Sierpinski triangle, axiom=X, anchor=north east}];
\end{tikzpicture}
\end{codeexample} 
\end{key}

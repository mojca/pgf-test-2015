% Copyright 2011 by Jannis Pohlmann
%
% This file may be distributed and/or modified
%
% 1. under the LaTeX Project Public License and/or
% 2. under the GNU Free Documentation License.
%
% See the file doc/generic/pgf/licenses/LICENSE for more details.

\section{Graph Drawing Layouts: Miscellaneous}
\label{section-last-graphdrawing-library-in-manual}


\begin{tikzlibrary}{graphdrawing.misc}
  Load this package when you wish to use the graph drawing algorithms
  defined in this library. You should load the |graphdrawing| library first.
\end{tikzlibrary}


\begin{gdalgorithm}{circular layout}{Circular Layout Tantau 2012}
  This layout arranges the nodes in a circle. The centers of the nodes
  are placed on a counter-clockwise circle, starting with the first
  node at the |grow| direction (for |grow'|, the circle is
  clockwise). The order of the nodes is the order in which they appear
  in the graph, the edges are not taken into consideration.

\begin{codeexample}[]
\tikz[>=spaced stealth']
  \graph [circular layout, grow'=down, sibling sep=1em,
          nodes={draw,circle}, math nodes]
  {
    x_1 -> x_2 -> x_3 -> x_4 ->
    x_5 -> "\dots"[draw=none] -> "x_{n-1}" -> x_n -> x_1
  };    
\end{codeexample}

  The nodes are placed in such a way that
  \begin{enumerate}
  \item The (angular) distance between the centers of consecutive
    nodes is at least  |sibling distance|,
  \item the distance between the borders of consecutive nodes is at
    least |sibling sep|, and
  \item the radius is at least |circular layout/radius|.
    \begin{key}{/graph drawing/circular layout/radius=\meta{radius}}
      The minimum radius of the circle.
    \end{key}
  \end{enumerate}
  The radius of the circle is chosen near-minimal such that the above
  properties are satisfied. To be more precise, if all nodes are
  circles, the radius is chosen optimally while for, say, rectangular
  nodes there may be too much space between the nodes in order to
  satisfy the second condition.

\begin{codeexample}[]
\tikz \graph [circular layout,
          sibling sep=0pt, sibling distance=0pt,
          nodes={draw,circle}]
  { 1 -- 2 [minimum size=30pt] -- 3 --
    4 [minimum size=50pt] -- 5 [minimum size=40pt] -- 6 -- 7 }; 
\end{codeexample}

\begin{codeexample}[]
\begin{tikzpicture}
  \graph [circular layout={radius=1.25cm},
          sibling sep=0pt, sibling distance=0pt,
          nodes={draw,circle}]
  { 1 -- 2 [minimum size=30pt] -- 3 --
    4 [minimum size=50pt] -- 5 [minimum size=40pt] -- 6 -- 7 }; 
  
  \draw [red] (0,-1.25) circle [radius=1.25cm];
\end{tikzpicture}
\end{codeexample}

\begin{codeexample}[]
\tikz \graph [circular layout,
    sibling sep=0pt, sibling distance=1cm,
    nodes={draw,circle}]
  { 1 -- 2 [minimum size=30pt] -- 3 --
    4 [minimum size=50pt] -- 5 [minimum size=40pt] -- 6 -- 7 }; 
\end{codeexample}

\begin{codeexample}[]
\tikz \graph [circular layout,
    sibling sep=2pt, sibling distance=0pt,
    nodes={draw,circle}]
  { 1 -- 2 [minimum size=30pt] -- 3 --
    4 [minimum size=50pt] -- 5 [minimum size=40pt] -- 6 -- 7 }; 
\end{codeexample}

\begin{codeexample}[]
\tikz \graph [circular layout,
    sibling sep=0pt, sibling distance=0pt,
    nodes={rectangle,draw}]
  { 1 -- 2 [minimum size=30pt] -- 3 --
    4 [minimum size=50pt] -- 5 [minimum size=40pt] -- 6 -- 7 }; 
\end{codeexample}
\end{gdalgorithm}

\begin{gdalgorithm}{simple demo layout}{Simple Demo}
  The algorithm used in the examples of this manual for demonstrating
  how a trivial graph drawing can be implemented.
\end{gdalgorithm}




%%% Local Variables: 
%%% mode: latex
%%% TeX-master: "pgfmanual-pdftex-version"
%%% End: 

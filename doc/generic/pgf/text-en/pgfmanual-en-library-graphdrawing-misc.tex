% Copyright 2011 by Jannis Pohlmann
%
% This file may be distributed and/or modified
%
% 1. under the LaTeX Project Public License and/or
% 2. under the GNU Free Documentation License.
%
% See the file doc/generic/pgf/licenses/LICENSE for more details.

\section{Miscellaneous Graph Drawing Algorithms}


\begin{tikzlibrary}{graphdrawing.misc}
  Load this package when you wish to use the graph drawing algorithms
  defined in this library. You should load the |graphdrawing| library first.
\end{tikzlibrary}

\ifluatex\relax\else{LuaTeX is required for setting this manual section.}\expandafter\endinput\fi


\begin{gdalgorithm}{AhrensFKSS2011 minimize crossings}
  This algorithm is due to  Ren\'ee Ahrens, Olof Frahm, Jens Kluttig,
  Matthias Schulz, Stephan Schuster, who have implemented the graph
  drawing engine.

  For computing the layout of an arbitrary graph you can use the
  |few intersections|-option. As input any graph is feasible. Note
  that for complex graphs the time to compute a layout can be long. 

\begin{codeexample}[]
\tikz [AhrensFKSS2011 minimize crossings, scale=2]
  \graph { 1 -> {2, 3} -> 4};
\end{codeexample}

\begin{codeexample}[]
\tikz [AhrensFKSS2011 minimize crossings, scale=2]
  \graph { 6 -> 3 -> 5 -> 1 ->{2 -> {3, 4, 6}}, 5 -> 2};
\end{codeexample}

  The algorithm that is going to be described next provides a standard
  approach for generating a layout for a generic graph. The resulting
  layout of this algorithm can occupy lots of space. Therefore you can
  limit its height and width using two algorithm-specific graph
  parameters: 

  \begin{key}{/graph drawing/AhrensFKSS2011 minimize crossings/max
      height=\meta{dimension}}
    Limits the height of the graph. The default is the maximum height
    of all nodes multiplied two times by number of nodes.
\begin{codeexample}[]
\tikz [AhrensFKSS2011 minimize crossings={max height=50pt}, scale=2]
  \graph { 6 -> 3 -> 5 -> 1 ->{2 -> {3, 4, 6}}, 5 -> 2};
\end{codeexample}
  \end{key}

  \begin{key}{/graph drawing/AhrensFKSS2011 minimize crossings/max
      width=\meta{dimension}} 
    Limits the width of the graph.
\begin{codeexample}[]
\tikz [AhrensFKSS2011 minimize crossings={ max width=50pt, max height=50pt},
       scale=2]
  \graph { 6 -> 3 -> 5 -> 1 ->{2 -> {3, 4, 6}}, 5 -> 2};
\end{codeexample}
    Since nothing is for free, the compact layout comes at the cost of
    an unwanted intersection of edges. 
  \end{key}

  \medskip
  \noindent\textbf{How Does This Algorithm Work?}
  The algorithm that is going to be described now provides a standard
  approach for generating a layout for a generic graph. It uses the
  principle of local search to find a layout of the nodes where
  primarily the number of intersections of paths is minimized.
  
  To realize the local search algorithm for the node positioning problem an
  initial state has to be defined. A state is defined by the arrangement
  of all nodes on a grid. This grid consists of rows and columns of the
  same size that is set to twice the number of nodes in the graph by
  default, but can be changed by the |max height| and |max width|
  keys. The height of the rows is determined by the heighest node and
  the width of the columns by widest.
  
  In the beginning the nodes are positioned in the middle of the grid in
  shape of a rectangle. As the next step one node has to change its
  position on the grid and the resulting arrangement has to be
  evaluated. The evaluation of a state is done by a cost function which
  counts the number of path intersections in the current
  arrangement. Additional the average length of 33\% of the longest
  paths is added to the number of intersections, so that intersections
  and the path lengths will be minimized. An arrangement is optimal if
  there are no intersections and all nodes are positioned together.
  
  The algorithm checks all possible neighbors of the current state if
  there is a state with lower costs and picks the first neighbor with
  lower costs as the state for the next step. The algorithm determines
  if their is no neighbor with lower costs. 
  
  To implement the local search, the following components are required:
  \begin{itemize}
  \item A representation of the |graph|-object as an arrangement on a
    grid.
  \item A cost function, that represents all aspects which have to be
    minimized. In this implementation that are primarily the
    intersections and the length of the longest paths. 
  \item A neighbor function, which provides all grid arrangements
    resulting from the movement of one node.
  \end{itemize}
  Important for the final layout is at first the initial arrangement of
  the nodes that offers an advantageous starting point for the
  improvement and secondly how to reach a quiet aesthetic and compact
  representation. The initial arrangement is done by grouping the nodes
  in the middle of the grid.
  
  The aesthetics claim is quiet difficult to realize by an algorithm. A
  simple approach is to minimize the longest paths of the layout. 
\end{gdalgorithm}




%%% Local Variables: 
%%% mode: latex
%%% TeX-master: "pgfmanual-pdftex-version"
%%% End: 

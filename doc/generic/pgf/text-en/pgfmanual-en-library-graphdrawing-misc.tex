% Copyright 2011 by Jannis Pohlmann
%
% This file may be distributed and/or modified
%
% 1. under the LaTeX Project Public License and/or
% 2. under the GNU Free Documentation License.
%
% See the file doc/generic/pgf/licenses/LICENSE for more details.

\section{Graph Drawing Layouts: Miscellaneous}
\label{section-last-graphdrawing-library-in-manual}


\begin{tikzlibrary}{graphdrawing.misc}
  Load this package when you wish to use the graph drawing algorithms
  defined in this library. You should load the |graphdrawing| library first.
\end{tikzlibrary}


\begin{gdalgorithm}{circular layout}{Circular Layout Tantau 2012}
  TODO: Document this...

\begin{codeexample}[]
\tikz[>=spaced stealth']
  \graph [circular layout, grow'=down, sibling sep=1em,
          nodes={draw,circle}, math nodes]
  {
    x_1 -> x_2 -> x_3 -> x_4 ->
    x_5 -> "\dots"[draw=none] -> "x_{n-1}" -> x_n -> x_1
  };    
\end{codeexample}

\begin{codeexample}[]
\tikz[>=spaced stealth']
  \graph [circular layout, grow'=30, sibling sep=1em,
          nodes={draw,circle}]
  { subgraph K_n [n=8] }; 
\end{codeexample}
\end{gdalgorithm}

\begin{gdalgorithm}{simple demo layout}{Simple Demo}
  The algorithm used in the examples of this manual for demonstrating
  how a trivial graph drawing can be implemented.
\end{gdalgorithm}




%%% Local Variables: 
%%% mode: latex
%%% TeX-master: "pgfmanual-pdftex-version"
%%% End: 

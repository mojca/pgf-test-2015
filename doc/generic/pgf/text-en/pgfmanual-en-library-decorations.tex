% Copyright 2008 by Mark Wibrow
%
% This file may be distributed and/or modified
%
% 1. under the LaTeX Project Public License and/or
% 2. under the GNU Free Documentation License.
%
% See the file doc/generic/pgf/licenses/LICENSE for more details.

\section{Decoration Library}
\label{section-library-decorations}



\subsection{Overview and Common Options}

The decoration libraries define a number of (more or less useful)
decorations that can be applied to paths. The usage of decorations is
not covered in the present section, please consult 
Sections~\ref{section-tikz-decorations}, which explains how
decorations are used in \tikzname, and
\ref{section-base-decorations}, which  explains how new
decorations can be defined. 


The decorations are influenced by a number of parameters that can be
set using the |decoration| option. These parameters are
typically shared between different decorations. In the following, the
general options are documented, special-purpose keys are documented
with the decoration that uses it.

\begin{key}{/pgf/decoration/amplitude=\meta{dimension} (initially 2.5pt)}
  This key determines the ``desired height'' (or amplitude) of 
  decorations for which this makes sense. For instance, the initial
  value of |2.5pt| means that deforming decorations should deform a
  path by up to 2.5pt away from the original path.
\end{key}

\begin{key}{/pgf/decoration/meta-amplitude=\meta{dimension} (initially 2.5pt)}
  This key determines the amplitude for a meta-decoration. 
\end{key}

\begin{key}{/pgf/decoration/segment length=\meta{dimension} (initially 10pt)}
  Many decorations are made up of small segments. This key determines
  the desired length of such segments. 
\end{key}

\begin{key}{/pgf/decoration/meta-segment length=\meta{dimension} (initially 1cm)}
  This determined the length of the meta-segments from which a
  meta-snake is made up.
\end{key}

\begin{key}{/pgf/decoration/angle=\meta{degree} (initially 45)}
  The way some decorations look like depends on a configurable angle. For
  instance, a |wave| decoration consists of arcs and the opening angle
  of these arcs is given by the |angle|.
\end{key}

\begin{key}{/pgf/decoration/aspect=\meta{factor} (initially 0.5)}
  For some decorations there is a natural aspect ratio. For instance,
  for a |brace| decoration the aspect ratio determines where the brace
  point will be.
\end{key}



\subsection{Path Morphing Decorations}

\begin{pgflibrary}{decorations.pathmorphing}
  A \emph{path morphing decorations} ``morphs'' or ``deforms'' the
  to-be-decorated path. This means that what used to be a straight
  line might afterwards be a snaking curve and have bumps. However, a
  line is still and a line and path deforming decorations do not
  change the number of subpaths. For instance, if the path used to
  consist of two circles and an open arc, the path will after the
  decoration process still consist of two closed subpath and one open
  subpath. 
\end{pgflibrary}


\subsubsection{Decorations Producing Straight Line Paths}

The following deformations use only straight lines in order to morph
the paths.

\begin{decoration}{lineto}
  This decoration replaces the path by straight lines. For each curve,
  the path simply goes directly from the start point to the end point.
  In the following example, the arc actually consist of two
  subcurves.

  This decoration is actually always defined when the decoration
  module is loaded, but it is documented here for consistency.
\begin{codeexample}[]
\begin{tikzpicture}[decoration=lineto]
  \draw [help lines] grid (3,2);
  \draw [decorate,fill=examplefill]
    (0,0) -- (3,1) arc (0:180:1.5 and 1) -- cycle;
\end{tikzpicture}
\end{codeexample}
\end{decoration}


\begin{decoration}{straight zigzag}
  This (meta-)decoration decorates the path by alternating between 
  |curveto| and |zigzag| decorations. It always finishes
  with the |curveto| decoration. The following parameters influence
  the decoration:
  \begin{itemize}
  \item |amplitude|
    determines how much the zig-zag lines raises above and falls below
    a straight line to the target point.
  \item |segment length|
    determines the length of a complete ``up-down'' cycle.
  \item	|meta-segment length|
    determines the length of the |curveto| and the |zigzag| decorations.
  \end{itemize}

\begin{codeexample}[]
\begin{tikzpicture}[decoration={straight zigzag,meta-segment length=1.1cm}]
  \draw [help lines] grid (3,2);
  \draw [decorate,fill=examplefill]
    (0,0) -- (3,1) arc (0:180:1.5 and 1) -- cycle;
\end{tikzpicture}
\end{codeexample}
\end{decoration}


\begin{decoration}{random steps}
  This snake consists of straight line segments. The line segments
  head towards the target, but each step is randomly shifted a little
  bit. The following parameters influence the snake:
  \begin{itemize}
  \item |segment length|
    determines the basic length of each step.
  \item |amplitude|
    The end of each step is perturbed both in $x$- and in
    $y$-direction by two values drawn uniformly from the interval
    $[-d,d]$, where $d$ is the value of |amplitude|.
  \end{itemize}
\begin{codeexample}[]
\begin{tikzpicture}
    [decoration={random steps,segment length=2mm}]
  \draw [help lines] grid (3,2);
  \draw [decorate,fill=examplefill]
    (0,0) -- (3,1) arc (0:180:1.5 and 1) -- cycle;
\end{tikzpicture}
\end{codeexample}
\end{decoration}


\begin{decoration}{saw}
  This decoration looks like the blade of a saw. The following parameters
  influence the decoration:
  \begin{itemize}
  \item |amplitude|
    determines how much each spike raises above the straight line.
  \item |segment length|
    determines the length each spike.
  \end{itemize}
\begin{codeexample}[]
\begin{tikzpicture}[decoration=saw]
  \draw [help lines] grid (3,2);
  \draw [decorate,fill=examplefill]
    (0,0) -- (3,1) arc (0:180:1.5 and 1) -- cycle;
\end{tikzpicture}
\end{codeexample}
\end{decoration}


\begin{decoration}{zigzag}
  This decoration looks like a zig-zag line. The following parameters
  influence the decoration:
  \begin{itemize}
  \item |amplitude|
    determines how much the zig-zag lines raises above and falls below
    a straight line to the target point.
  \item |segment length|
    determines the length of a complete ``up-down'' cycle.
  \end{itemize}
\begin{codeexample}[]
\begin{tikzpicture}[decoration=zigzag]
  \draw [help lines] grid (3,2);
  \draw [decorate,fill=examplefill]
    (0,0) -- (3,1) arc (0:180:1.5 and 1) -- cycle;
\end{tikzpicture}
\end{codeexample}
\end{decoration}



\subsubsection{Decorations Producing Curved Line Paths}

\begin{decoration}{bent}
  This decoration adds a slightly bent line from the start to the
  target. The amplitude of the bent is given |amplitude|
  (an amplitude of zero gives a straight line). 
  \begin{itemize}
  \item |amplitude|
    determines the amplitude of the bent.
  \item |aspect|
    determines how tight the bent is. A good value is around |0.3|. 
  \end{itemize}
  Note that this decoration makes only little sense for curves. You
  should apply it only to straight lines.
\begin{codeexample}[]
\begin{tikzpicture}[decoration=bent]
  \draw [help lines] grid (3,2);
  \draw [decorate] (0,0) -- (3,1) -- (1.5,2) -- (0,1);
\end{tikzpicture}
\end{codeexample}
\begin{codeexample}[]
\begin{tikzpicture}[decoration={bent,aspect=.3}]
  \draw [decorate,fill=examplefill] (0,0) rectangle (3.5,2);
  \node[circle,draw] (A) at (.5,.5) {A};
  \node[circle,draw] (B) at (3,1.5) {B};
  \draw[->,decorate] (A) -- (B);
  \draw[->,decorate] (B) -- (A);
\end{tikzpicture}
\end{codeexample}
\end{decoration}


\begin{decoration}{bumps}
  This decoration replaces the path by little half ellipses. The
  following parameters influence itL
  \begin{itemize}
  \item |amplitude|
    determines the height of the half ellipse.
  \item |segment length|
    determines the width of the half ellipse.
  \end{itemize}
\begin{codeexample}[]
\begin{tikzpicture}[decoration=bumps]
  \draw [help lines] grid (3,2);
  \draw [decorate,fill=examplefill]
    (0,0) -- (3,1) arc (0:180:1.5 and 1) -- cycle;
\end{tikzpicture}
\end{codeexample}
\end{decoration}


\begin{decoration}{coil}
  This decoration replaces the path by a coiled line. To understand how this works,
  imagine a three-dimensional spring. The spring's axis points along
  the path toward the target. Then, we ``view'' the spring from a
  certain angle. If we look ``straight from the side'' we will see a
  perfect sine curve, if we look ``more from the front'' we will see a
  coil. The following parameters influence the snake:  
  \begin{itemize}
  \item |amplitude|
    determines how much the coil rises above the path and falls below
    it. Thus, this is the radius of the coil.
  \item |segment length|
    determines the distance between two consecutive ``curls.'' Thus,
    when the spring is see ``from the side'' this will be the wave
    length of the sine curve. 
  \item |aspect|
    determines the ``viewing direction.'' A value of |0| means
    ``looking from the side'' and a value of |0.5|, which is the
    default, means ``look more from the front.'' 
  \end{itemize}
\begin{codeexample}[]
\begin{tikzpicture}[decoration=coil]
  \draw [help lines] grid (3,2);
  \draw [decorate,fill=examplefill]
    (0,0) -- (3,1) arc (0:180:1.5 and 1) -- cycle;
\end{tikzpicture}
\end{codeexample}
\begin{codeexample}[]
\begin{tikzpicture}
    [decoration={coil,aspect=0.3,segment length=3mm,amplitude=3mm}]
  \draw [help lines] grid (3,2);
  \draw [decorate,fill=examplefill]
    (0,0) -- (3,1) arc (0:180:1.5 and 1) -- cycle;
\end{tikzpicture}
\end{codeexample}
\end{decoration}



\begin{decoration}{curveto}
  This decoration simply yields a line following the original
  path. This means that (ideally) it does not change the path and
  follows any curves in the path (hence the name). In
  reality, due to the internals of how decorations are implemented,
  this decoration actually replaces the path by numerous small
  straight lines.

  This decoration is mostly useful in conjunction with
  meta-decorations. It is also actually defined in the decoration
  module and is always available.

\begin{codeexample}[]
\begin{tikzpicture}[decoration=curveto]
  \draw [help lines] grid (3,2);
  \draw [decorate,fill=examplefill]
    (0,0) -- (3,1) arc (0:180:1.5 and 1) -- cycle;
\end{tikzpicture}
\end{codeexample}
\end{decoration}



\begin{decoration}{snake}
  This decoration replaces the path by a line that looks like a snake
  seen from above. More precisely, the snake is a sine wave with a
  ``softened'' start and ending. The following parameters influence
  the snake: 
  \begin{itemize}
  \item |amplitude|
    determines the sine wave's amplitude.
  \item |segment length|
    determines the sine wave's wave length.
  \end{itemize}
\begin{codeexample}[]
\begin{tikzpicture}[decoration=snake]
  \draw [help lines] grid (3,2);
  \draw [decorate,fill=examplefill]
    (0,0) -- (3,1) arc (0:180:1.5 and 1) -- cycle;
\end{tikzpicture}
\end{codeexample}
\end{decoration}



  
\subsection{Path Replacing Decorations}

\begin{pgflibrary}{decorations.pathreplacing}
  This library defines decorations that replace the to-be-decorated
  path by another path. Unlike morphing decorations, the replaced path
  might be quite different, for instance a straight line might be
  replaced by a set of circles. Note that filling a path that has been
  replaced using one of the decorations in this library typically does
  not fill the original area but, rather, the smaller area of the
  newly-created path segments.
\end{pgflibrary}

\begin{decoration}{border}
  This decoration adds straight lines the path that are at a specific
  angle to the line toward the target. The idea is to add these little
  lines to indicate the ``border'' or an area. The following
  parameters influence the decoration:  
  \begin{itemize}
  \item |segment length|
    determines the distance between consecutive ticks.
  \item |amplitude|
    determines the length of the ticks.
  \item |angle|
    determines the angle between the ticks and the line of the path. 
  \end{itemize}
\begin{codeexample}[]
\begin{tikzpicture}[decoration=border]
  \draw [help lines] grid (3,2);
  \draw [postaction={decorate,draw,red}]
        (0,0) -- (3,1) arc (0:180:1.5 and 1);
\end{tikzpicture}
\end{codeexample}
\end{decoration}


\begin{decoration}{brace}
  This decoration replaces a straight line path by a long brace. The
  left and right end of the brace will be exactly on the start and
  endpoint of the decoration. The decoration really only makes sense
  for paths that are a straight line.
  \begin{itemize}
  \item |amplitude|
    determines how much the brace rises above the path.
  \item |aspect|
    determines the fraction of the total length where the ``middle
    part'' of the brace will be.  
  \end{itemize}
\begin{codeexample}[]
\begin{tikzpicture}[decoration=brace]
  \draw [help lines] grid (3,2);
  \draw [decorate] (0,0) -- (3,1);
\end{tikzpicture}
\end{codeexample}
\end{decoration}



\begin{decoration}{expanding waves}
  This decoration adds arcs to the path that get bigger along the line
  towards the target. The following parameters influence the decoration:
  \begin{itemize}
  \item |segment length|
    determines the distance between consecutive arcs.
  \item |angle|
    determines the opening angle below and above the path. Thus, the
    total opening angle is twice this angle.
  \end{itemize}
\begin{codeexample}[]
\begin{tikzpicture}[decoration={expanding waves,angle=5}]
  \draw [help lines] grid (3,2);
  \draw [decorate] (0,0) -- (3,1) arc (0:180:1.5 and 1);
\end{tikzpicture}
\end{codeexample}
\end{decoration}


\begin{decoration}{ticks}
  This decoration replaces the path by straight lines that are
  orthogonal to the path. The following parameters influence the
  decoration:  
  \begin{itemize}
  \item |segment length|
    determines the distance between consecutive ticks.
  \item |amplitude|
    determines half the length of the ticks.
  \end{itemize}
\begin{codeexample}[]
\begin{tikzpicture}[decoration=ticks]
  \draw [help lines] grid (3,2);
  \draw [decorate] (0,0) -- (3,1) arc (0:180:1.5 and 1);
\end{tikzpicture}
\end{codeexample}
\end{decoration}



\begin{decoration}{waves}
  This decoration replaces the path by arcs that have a constant
  size. The following parameters influence the decoration: 
  \begin{itemize}
  \item |segment length|
    determines the distance between consecutive arcs.
  \item |angle|
    determines the opening angle below and above the path. Thus, the
    total opening angle is twice this angle.
  \item |radius|
    determines the radius of each arc.
  \end{itemize}
\begin{codeexample}[]
\begin{tikzpicture}[decoration={waves,radius=4mm}]
  \draw [help lines] grid (3,2);
  \draw [decorate] (0,0) -- (3,1) arc (0:180:1.5 and 1);
\end{tikzpicture}
\end{codeexample}
\end{decoration}




\subsection{Decorations with Shapes}

\begin{pgflibrary}{decorations.shapes}
  This library defines decorations that use shapes or shape-like
  drawings to decorate a path. The following options are common
  options used by the decorations in this library:
  
  \begin{key}{/pgf/decoration/shape width=\meta{dimension}  (initially 2.5pt)}
    The desired width of the shapes. For decorations that support
    varying shape sizes, this key sets both the start and end width
    (which can be overwritten using options like |shape start width|).
  \end{key}

  \begin{key}{/pgf/decoration/shape height=\meta{dimension} (initially 2.5pt)}
    Works like the previous key, only for the height.
  \end{key}

  \begin{key}{/pgf/decoration/shape size=\meta{dimension}}
    Sets the desired width and height simultaneously.
  \end{key}
\end{pgflibrary}



\begin{decoration}{crosses}
  This decoration replaces the path by (diagonal) crosses. The
  following parameters influence the decoration:  
  \begin{itemize}
  \item |segment length|
    determines the distance between (the centers of) consecutive crosses.
  \item |shape height|
    determines the height of each cross.
  \item |shape width|
    determines the width of each cross.
  \end{itemize}
\begin{codeexample}[]
\begin{tikzpicture}[decoration=crosses]
  \draw [help lines] grid (3,2);
  \draw [decorate] (0,0) -- (3,1) arc (0:180:1.5 and 1);
\end{tikzpicture}
\end{codeexample}
\end{decoration}

\begin{decoration}{triangles}
  This decoration replaces the path by triangles that point along the
  path. The following parameters influence the decoration: 
  \begin{itemize}
  \item |segment length|
    determines the distance between consecutive triangles.
  \item |shape height|
    determines height of the triangle side that is orthogonal
    to the path.
  \item |shape width|
    determines the width of the triangle.
  \end{itemize}
\begin{codeexample}[]
\begin{tikzpicture}[decoration=triangles]
  \draw [help lines] grid (3,2);
  \draw [decorate,fill=examplefill] (0,0) -- (3,1) arc (0:180:1.5 and 1);
\end{tikzpicture}
\end{codeexample}
\end{decoration}



\begin{decoration}{shape backgrounds}
  This is a general decoration that replaces the to-be-decorated path by repeated
  copies of the background path of an arbitrary shape that has
  previously defined using the |\pgfdeclareshape| command (that is,
  you can use any shape in the shape libraries). Please note that the
  background path of the shapes is used, but \emph{no nodes are
    created}. You cannot have text inside the shapes of this path, you
  cannot name them, or refer to them. 

\begin{codeexample}[]
\begin{tikzpicture}[decoration={shape backgrounds,shape=star,shape size=5pt}]
  \draw [help lines] grid (3,2);
  \draw [decorate] (0,0) -- (3,1) arc (0:180:1.5 and 1);
\end{tikzpicture}
\end{codeexample}

\begin{codeexample}[]
\tikzset{paint/.style={ draw=#1!50!black, fill=#1!50 },
         decorate with/.style=
           {decorate,decoration={shape backgrounds,shape=#1,shape size=2mm}}}
\begin{tikzpicture}
  \draw [decorate with=dart,      paint=red]    (0,1.5) -- (3,1.5);
  \draw [decorate with=diamond,   paint=green]  (0,1)   -- (3,1);
  \draw [decorate with=rectangle, paint=blue]   (0,0.5) -- (3,0.5);
  \draw [decorate with=circle,    paint=yellow] (0,0)   -- (3,0);
\end{tikzpicture}
\end{codeexample}

  All shapes are positioned by the anchor that is specified via the
  |anchor| decoration option:

  \begin{key}{/pgf/decoration/anchor=\meta{anchor} (initially center)}
    The anchor used to position the shapes backgrounds.
  \end{key}

  A shape background path is added at the start point of the path and,
  if the distance between the shapes is appropriate, at the end point
  of the path. 
	
\begin{codeexample}[]
\begin{tikzpicture}[decoration={
      shape backgrounds,shape=regular polygon,shape size=4mm}]
  \draw [help lines] grid (3,2);
  \draw [thick] (0,0) -- (2,2) (1,0) -- (3,0);
  \draw [red, decorate, decoration={shape sep=.5cm}]  (1,0) -- (3,0);
  \draw [blue, decorate, decoration={shape sep=.5cm}] (0,0) -- (2,2);
\end{tikzpicture}
\end{codeexample}

  Keys for cusomizing specific shapes can be specified (e.g., 
  |star points|, |cloud puffs|, |kite angles|, and so on). The size of
  the shape is ``enforced'' using transformations. This means that the
  shape is typeset with an empty text box and some default size
  values, resulting in an initial shape. This shape is then rescaled
  using coordinate transformations so that it has the desired size
  (which may vary as we travel along the to-be-decorated path). This
  means that settings involving angles and distances may not appear
  entirely accurate. More general options such as |inner sep| and
  |minimum size| will be ignored,  but transformations can be applied
  to each segment as described below.
  
\begin{codeexample}[]
\tikzset{
  paint/.style={draw=#1!50!black, fill=#1!50},
  my star/.style={decorate,decoration={shape backgrounds,shape=star},
                  star points=#1}
}
\begin{tikzpicture}[decoration={shape sep=.5cm, shape size=.5cm}]
  \draw [my star=9, paint=red]                            (0,1.5) -- (3,1.5);
  \draw [my star=5, paint=blue]                           (0,.75) -- (3,.75);
  \draw [my star=5, paint=yellow, shape border rotate=30] (0,0) -- (3,0);
\end{tikzpicture}
\end{codeexample}

  There are various keys to control the drawing of the shape
  decoration. 
  
  \begin{key}{/pgf/decoration/shape=\meta{shape name} (initially circle)}
    The shape whose background path is used.
  \end{key}
  
  \begin{key}{/pgf/decorations/shape sep=\meta{spacing} (initially {.25cm, between centers})}
    Set the spacing between the shapes on the snaked path. This can be
    just a distance on its own, but the additional keywords 
    |between centers|, and |between borders| (which must be preceded by a 
    comma), specify that the distance	is between the center anchors of 
    the shapes or between the edges of the \emph{boundaries} of
    the shape borders.
	
\begin{codeexample}[]
\begin{tikzpicture}[
    decoration={shape backgrounds,shape size=.5cm,shape=signal},
    signal from=west, signal to=east,
    paint/.style={decorate, draw=#1!50!black, fill=#1!50}]
  \draw [help lines] grid (3,2);
  \draw [paint=red, decoration={shape sep=.5cm}]
    (0,2) -- (3,2);
  \draw [paint=green, decoration={shape sep={1cm, between center}}]
    (0,1) -- (3,1);
  \draw [paint=blue, decoration={shape sep={1cm, between borders}}]
    (0,0) -- (3,0);
\end{tikzpicture}
\end{codeexample}
  \end{key}
  
  \begin{key}{/pgf/decoration/shape evenly spread=\meta{number}}
    This key overides the |shape sep| key and forces the decoration to
    fit \meta{number} shapes evenly across the path. 
    If \meta{number} is less than |1|, then no shapes will be used. 
    If \meta{number} equals |1|, then one shape is put in the middle 
    of the path. 
    The additional keywords |by centers| (the default, if no keyword is
    specified) and |by borders| can be used (both preceded by a comma), 
    to specify how the distance between shapes is determined. These
    keywords will only have a noticable effect if the shapes sizes
    differ over time.
    
\begin{codeexample}[]
\tikzset{
  paint/.style={draw=#1!50!black, fill=#1!50},
  spreading/.style={
    decorate,decoration={shape backgrounds, shape=rectangle,
    shape start size=4mm,shape end size=1mm,shape evenly spread={#1}}}
}
\begin{tikzpicture}
  \fill [paint=green,spreading={5, by borders},
         decoration={shape scaled}]            (0,2)   -- (3,2);
  \fill [paint=blue,spreading={5, by centers},
         decoration={shape scaled}]            (0,1.5) -- (3,1.5);   
  \fill [paint=red,    spreading=5]            (0,1)   -- (3,1);
  \fill [paint=orange, spreading=4]            (0,.5)  -- (3,.5);
  \fill [paint=gray,    spreading=1]            (0,0)   -- (3,0);
\end{tikzpicture}
\end{codeexample}
  \end{key}

  \begin{key}{/pgf/decoration/shape sloped=\opt{\meta{boolean}} (initially true)}
    By default, shapes are rotated to the slope of the snaked path. If 
    \meta{boolean} is the value |false|, then this rotation is turned 
    off. Internally this sets the \TeX-if |\ifpgfshapesnakesloped|
    appropriately.

\begin{codeexample}[]
\tikzset{
  paint/.style={draw=#1!50!black, fill=#1!50}
}
\begin{tikzpicture}[decoration={
    shape width=.65cm, shape height=.45cm,
    shape=isosceles triangle, shape sep=.75cm,
    shape backgrounds}]    
  \draw [help lines] grid (3,2);
  \draw [paint=red,decorate] (0,0) -- (2,2);
  \draw [paint=blue,decorate,decoration={shape sloped=false}]
                             (1,0) -- (3,2);
\end{tikzpicture}
\end{codeexample}
  \end{key}%

  It is possible to scale the width and height of the shapes across the
  length of the snaked path. The shapes are scaled between the starting
  size and the ending size. The following keys customize the way the
  snake shapes are scaled:
	
  \begin{key}{/pgf/decoration/shape scaled=\meta{boolean} (initially false)}
\begin{codeexample}[]
\tikzset{
  bigger/.style={decoration={shape start size=.125cm, shape end size=.5cm}},
  smaller/.style={decoration={shape start size=.5cm, shape end size=.125cm}},
  decoration={shape backgrounds,
              shape sep={.25cm, between borders},shape scaled}
}
\begin{tikzpicture}
  \draw [help lines] grid (3,2);
  \fill [decorate, bigger, red!50]   (0,1) -- (3,2);
  \fill [decorate, smaller, blue!50] (0,0) -- (3,1);
\end{tikzpicture}
\end{codeexample}

    If this key is set to false (which is the default), then only the
    start width and height are used. Note that the keys |shape width|
    and |shape height| set the start and end height simultaneously.
  \end{key}

  \begin{key}{/pgf/decoration/shape start width=\meta{length} (initially 2.5pt)}
    The starting width of the shape.
  \end{key}%

  \begin{key}{/pgf/decoration/shape start height=\meta{length} (initially 2.5pt)}
    The starting height of the shape.
  \end{key}%
  
  \begin{stylekey}{/pgf/decoration/shape start size=\meta{length}}
    Set both the the start height and start width simultaneously.
  \end{stylekey}%

  \begin{key}{/pgf/decoration/shape end width=\meta{length} (initially 2.5pt)}
    The recommended ending width of the shape. Note, that this is the
    width that a shape will take only if it is drawn exactly at the end
    of the path.


\begin{codeexample}[]    
\tikzset{
  bigger/.style={decoration={shape start size=.25cm, shape end size=1cm}},
  smaller/.style={decoration={shape start size=1cm, shape end size=.25cm}},
  decoration={shape backgrounds,
              shape sep={.25cm, between borders},shape scaled}
}
\begin{tikzpicture}
  \draw [help lines] grid (3,2);
  \fill [decorate,bigger,
         decoration={shape sep={.25cm, between borders}}, blue!50] 
    (0,1.5) -- (3,1.5);
  \fill [decorate,smaller,
         decoration={shape sep={1cm, between centers}},   red!50]  
    (0,.5)  -- (3,.5);
  \draw [gray, dotted] (0,1.625) -- (3,2)    (0,1.375) -- (3,1) 
                       (0,1)     -- (3,.625) (0,0)     -- (3,.375); 
\end{tikzpicture}
\end{codeexample}
  \end{key}%

  \begin{key}{/pgf/decoration/shape end height=\meta{length}}
    The recommended ending height of the shape.
  \end{key}%

  \begin{stylekey}{/pgf/decoration/shape snake end size=\meta{length}}
    Set both the the end height and end width simultaneously.
  \end{stylekey}
\end{decoration}





\subsection{Text Decorations}

\begin{pgflibrary}{decorations.text}
  The decoration in this library decorates the path with some
  text. This can be used to draw text that follows a curve.  
\end{pgflibrary}

\begin{decoration}{text along path}
  This decoration decorates the path with text. This drawing of the
  text is a ``side effect'' of the decoration. The to-be-decorated
  path is only used to determine where the characters should be put
  and it is thrown away after the decoration is done. This is why in
  the following example no line is shown.
	
\begin{codeexample}[]
\catcode`\|12
\begin{tikzpicture}[decoration={text along path,
    text={Some long text along a ridiculously long curve that}}] 
  \draw [help lines] grid (3,2);
  \draw [decorate] (0,0) -- (3,1) arc (0:180:1.5 and 1);
\end{tikzpicture}
\end{codeexample}

  \pgfname{} ``does its best'' to typeset the text, however you
  should note the following points:
  \begin{itemize}
  \item
    Each character in the text is typeset in a separate |\hbox|. This
    means that if you want fancy things like kerning or ligatures you
    will have to manually annotate the characters in the decoration 
    text within a group, for example, |W{\kern-1ptA}TER|. 
  \item
    Each character is positioned using the center of its baseline. To
    move the text vertcally (relative to the path), the additional
    transform key should be used.
  \item
    No attempt is made to ensure characters do not overlap when
    the angle between segments is considerably less than 180\textdegree{}
    (this is tricky to do in \TeX{} without a huge processing
    overhead). In general this should not be too much of a problem, 
    but, once again, kerning can be used in most cases to overcome 
    any undesirable effects.
  \item			
    It is only possible to typeset text in math mode under considerable
    restrictions. Math mode is entered and exited using any character	
    of category code 3 (e.g., in plain \TeX{} this is |$|). %$
    Math subscripts and superscripts need to be	contained within braces 
    (e.g., |{^y_i}|) as do commands like |\times| or |\cdot|. 
    However, even modestly complex mathematical	typesetting is unlikely 
    to be sucessful along a path (or even desirable).
  \item
    Some inaccuracies in positioning may be particularly apparent
    at subpath boundaries. This can (unfortunately) only be solved 
    on case by case basis by individually kerning the offending 
    characters within a group.
  \end{itemize}
  
  The following keys are used by the |text| decoration:
  \begin{key}{/pgf/decoration/text=\meta{text}
      (initially \normalfont empty)}
    Set the text to typeset along the curve. 
    Consecutive spaces are ignored, so |\ | (or |\space| in \LaTeX) 
    should be used to insert multiple spaces.	It is possible to
    format the text using normal formating commands, such as |\it|, |\bf|
    and |\color|, within customisable delimiters. Initially these
    delimiters are both {\tt\char`\|} (however, care will be needed 
    regarding	the category codes of delimiters --- see below). 

{\catcode`\|12
\begin{codeexample}[]
\catcode`\|12
\begin{tikzpicture}
  \draw [help lines] grid (3,2);
  \path [decorate,decoration={text along path,
           text={a big |\color{green}|green|| juicy apple.}}] 
    (0,0) .. controls (0,2) and (3,0) .. (3,2);
\end{tikzpicture}
\end{codeexample}
}
    By following the first delimiter
    with |+|, the formatting commands are added to any exisiting 
    formatting.

{\catcode`\|12
\begin{codeexample}[]
\begin{tikzpicture}
  \draw [help lines] grid (3,2);
  \path [decorate,decoration={text along path,
           text={a |\large|big |+\bf\color{red}|red|| juicy apple.}}] 
    (0,0) .. controls (0,2) and (3,0) .. (3,2);
\end{tikzpicture}
\end{codeexample}
}
	
    Internally, the text is stored in the macro |\pgfdecorationtext|. 
    Any characters that have not been typeset when the end of the 
    path has been reached will be stored in |\pgfdecorationrestoftext|.
    
  \end{key}

{\catcode`\|12
  \begin{key}{/pgf/decoration/text format delimiters=\marg{before}\marg{after} (initially \char`\{|\char`\}\char`\{\char`\})}

    \catcode`\|13
	
    Set the characters that the text decoration will use to parse 
    formatting commands. 
    If \meta{after} is empty, then \meta{before} will be used for both
    delimiters.
    In general you should stick to characters	whose category codes are 
    |11| or |12|.
    As |+| is used to indicate that the specifed format commands 
    are added	to any exisiting ones, you should avoid using |+| as
    a delimiter. 

\begin{codeexample}[]
\begin{tikzpicture}
  \draw [help lines] grid (3,2);
  \path [decorate, decoration={text along path,text format delimiters={[}{]}, 
           text={A big [\color{red}]red[] and [\color{green}]green[] apple.}}] 
    (0,0) .. controls (0,2) and (3,0) .. (3,2);
\end{tikzpicture}
\end{codeexample}
  \end{key}
}

  \begin{key}{/pgf/decoration/text color=\meta{color}  (initially black)}
    The color of the text.
  \end{key}
\end{decoration}



\subsection{Fractal Decorations}

\begin{pgflibrary}{decorations.fractals}
  The decorations of this library can be used to create fractal
  lines. To use them, you typically have to apply the decoration
  repeatedly to an originally straight path.
\end{pgflibrary}


\begin{decoration}{Koch curve type 1}
  This decoration replaces a straight line by a ``rectangular bump.''
  By repeatedly applying this replacement, different levels of the
  Koch curve fractal can be created. Its Hausdorff dimension is $\log
  5/\log 3$.
\begin{codeexample}[]
\begin{tikzpicture}[decoration=Koch curve type 1]
  \draw decorate{ (0,0) -- (3,0) };
  \draw decorate{ decorate{ (0,-1.5) -- (3,-1.5) }};
  \draw decorate{ decorate{ decorate{ (0,-3) -- (3,-3) }}};
\end{tikzpicture}
\end{codeexample}
\end{decoration}


\begin{decoration}{Koch curve type 2}
  This decoration replaces a straight line by a ``rectangular sine.''
  Its Hausdorff dimension is $3/2$.
\begin{codeexample}[]
\begin{tikzpicture}[decoration=Koch curve type 2]
  \draw decorate{ (0,0) -- (3,0) };
  \draw decorate{ decorate{ (0,-2) -- (3,-2) }};
  \draw decorate{ decorate{ decorate{ (0,-4) -- (3,-4) }}};
\end{tikzpicture}
\end{codeexample}
\end{decoration}

\begin{decoration}{Koch snowflake}
  This decoration replaces a straight line by a ``line with a spike.''
  Koch's snowflake's Hausdorff dimension is $\log 4/\log 3$.
\begin{codeexample}[]
\begin{tikzpicture}[decoration=Koch snowflake]
  \draw decorate{ (0,0) -- (3,0) };
  \draw decorate{ decorate{ (0,-1) -- (3,-1) }};
  \draw decorate{ decorate{ decorate{ (0,-2) -- (3,-2) }}};
  \draw decorate{ decorate{ decorate{ decorate{ (0,-3) -- (3,-3) }}}};
\end{tikzpicture}
\end{codeexample}
\end{decoration}

\begin{decoration}{Cantor set}
  This decoration replaces a straight line by a ``line with a whole in
  the middle.'' The Hausdorff dimension of the Cantor set is $\log
  2/\log 3$. 
\begin{codeexample}[]
\begin{tikzpicture}[decoration=Cantor set,very thick]
  \draw decorate{ (0,0) -- (3,0) };
  \draw decorate{ decorate{ (0,-.5) -- (3,-.5) }};
  \draw decorate{ decorate{ decorate{ (0,-1) -- (3,-1) }}};
  \draw decorate{ decorate{ decorate{ decorate{ (0,-1.5) -- (3,-1.5) }}}};
\end{tikzpicture}
\end{codeexample}
\end{decoration}










\endinput




% Copyright 2010 by Renée Ahrens, Olof Frahm, Jens Kluttig, Matthias Schulz, Stephan Schuster
% Copyright 2011 by Till Tantau
% Copyright 2011 by Jannis Pohlmann
%
% This file may be distributed and/or modified
%
% 1. under the LaTeX Project Public License and/or
% 2. under the GNU Free Documentation License.
%
% See the file doc/generic/pgf/licenses/LICENSE for more details.


\section{The Algorithm Layer}

\label{section-gd-algorithm-layer}

\noindent{\emph{by Till Tantau}}


\subsection{Overview}

The present section is addressed at readers interested in implementing
new graph drawing algorithms for the graph drawing system. Obviously,
in order to do so, you need to have an algorithm in mind and also some
programming skills; but fortunately only in the Lua programming
language: Even though the graph drawing system was originally
developed as an extension of \tikzname, is has been restructured so
that the ``algorithm layer'' where you define algorithms is
scrupulously separated from \tikzname. In particular, an algorithm
declared and implemented on this layer can be used in with every ``display
layers,'' see the next section, without change. Nevertheless, in
the following we will use the \tikzname\ display layer and syntax in
our examples.

All new graph drawing algorithms can and must be implemented entirely in
the Lua programming language, which is a small, easy-to-learn (and
quite beautiful) language integrated into current versions of
\TeX. Actually, just a ``stub'' needs to be implemented in Lua. This
stub can call code implemented in another programming language such as
C, but this is in some sense ``internal'' to the algorithm and not of
interest to the graph drawing system. From its point of view, all
there is is some Lua code it can call.

In the following, after a small ``hello world'' example of graph
drawing and a discussion of technical details like
how to name files so that \TeX\ will find them, we have a look at the
main parts of the algorithm layer:

\begin{itemize}
\item Section~\ref{section-gd-namespaces} gives and overview of the
  available namespaces and also of naming conventions used in the
  graph drawing system.
\item Section~\ref{section-gd-gd-scope} explores what graph
  drawing scopes ``looks like on the algorithm layer.'' As the graph
  of a graph drawing scope is being parsed on the display layer, a lot
  of information is gathered: The nodes and edges of the graph are
  identified and the 
  object-oriented model is built, but other information is also
  collected. For instance, a sequence of \emph{events} is created
  during the parsing process. As another example, numerous kinds of
  \emph{collections} may be identified by the parser. The parsed graph
  together with the event sequence and the collections are all
  gathered in a single table, called the \emph{scope table} of the
  current graph drawing scope. Algorithms can access this table to
  retrieve information that goes beyond the ``pure'' graph model.

  One entry in this table is of particular importance: The
  \emph{syntactic digraph.} While most graph drawing 
  algorithms are not really interested in the ``details'' of how a
  graph was specified, for some algorithms it makes a big difference
  whether you write |a -> b| or |b <- a| in your specification of the
  graph. These algorithms can access the ``fine details'' of how the
  input graph was specified through the syntactic digraph; all other
  algorithms can access their |digraph| or |ugraph| fields and do not
  have to worry about the difference between |a -> b| and |b <- a|.
\item Section~\ref{section-gd-models} explains the object-oriented
  model of graphs used throughout the graph drawing system. Graph
  drawing algorithms do not get the ``raw'' specification used by the
  user to specify a graph (like |{a -> {b,c}}| in the |graph|
  syntax). Instead, what a graph drawing algorithm sees is ``just'' a
  graph object that provides methods for accessing the vertices and
  arcs.
\item Section~\ref{section-gd-transformations} explains how the
  information in the graph drawing scope is processed. One might
  expect that we simply run the algorithm selected by the user;
  however, things are more involved in practice. When the layout of a
  graph needs to be computed, only very few algorithms will actually
  be able to compute positions for the nodes of \emph{every}
  graph. For instance, most algorithms implicitly assume that the
  input graph is connected; algorithms for computing layouts for trees
  assume that the input is, well, a tree; and so on. For this reason,
  graph drawing algorithms will not actually need the original input
  graph as their input, but some \emph{transformed} version of
  it. Indeed, \emph{all} graph drawing algorithms are treated as graph 
  transformations by the graph drawing engine.

  This section explains how transformations are chosen and which
  transformations are applied by default.
\item Section~\ref{section-gd-interface-to-algorithms} documents the
  interface-to-algorithm class. This interface encapsulates all that
  an algorithm ``sees'' of the graph drawing system (apart from the
  classes in |model| and |lib|).
\item Section~\ref{section-gd-examples} provides a number of complete
  examples that show how graph drawing algorithms can, actually, be
  implemented. 
\item Section~\ref{section-gd-libs} documents the different
  libraries functions that come with the graph drawing engine. For
  instance, there are library functions for computing the (path)
  distance of nodes in a graph; a parameter that is needed by some
  algorithms.
\end{itemize}



\subsection{Getting Started}

In this section, a ``hello world'' example of a graph
drawing algorithm  is given, followed by an overview of the
organization of the whole engine.


\subsubsection{The Hello World of Graph Drawing}

Let us start our tour of the algorithm layer with a ``hello world''
version of graph drawing: An algorithm that simply places all nodes of
a graph in a circle of a fixed radius. Naturally, this is not a
particularly impressive or intelligent graph drawing algorithm; but
neither is the classical ``hello world''\dots\ Here is a minimal
version of the needed code (this is not the typical way of formulating
the code, but it is the shortest; we will have a look at the more
standard and verbose way in a moment):

\begin{codeexample}[code only]
pgf.gd.interface.InterfaceToAlgorithms.declare {
  key = "very simple demo layout",
  algorithm = {
    run =
      function (self)
        local alpha = (2 * math.pi) / #self.ugraph.vertices
        for i,vertex in ipairs(self.ugraph.vertices) do
          vertex.pos.x = math.cos(i * alpha) * 25
          vertex.pos.y = math.sin(i * alpha) * 25
        end
      end
  }
}
\end{codeexample}
\directlua{
pgf.gd.interface.InterfaceToAlgorithms.declare {
  key = "very simple demo layout",
  algorithm = {
    run =
      function (self)
        local alpha = (2 * math.pi) / \luaescapestring{#}self.ugraph.vertices
        for i,vertex in ipairs(self.ugraph.vertices) do
          vertex.pos.x = math.cos(i * alpha) * 25
          vertex.pos.y = math.sin(i * alpha) * 25
        end
      end
  }
}
}

This code \emph {declares} a new algorithm (|very simple demo layout|)
and includes an implementation of the algorithm (through the |run|
field of the |algorithm| field). When the |run| method is called, the
|self| parameter will contain the to-be-drawn graph in its |ugraph|
field. It is now the job of the code to modify the positions of the
vertices in this graph (in the example, this is done by assigning
values to |vertex.pos.x| and |vertex.pos.y|).

In order to actually \emph{use} the algorithm, the above code first
needs to be executed somehow. For \tikzname, one can just call
|\directlua| on it or put it in a file and then use |\directlua| plus
|require| (a better alternative) or you put it in a file like
|simpledemo.lua| and use |\usegdlibrary{simpledemo}| (undoubtedly the
``best'' way). For another display layer, like a graphical editor, the
code could also be executed through the use of |require|.

Executing the code ``just'' declares the algorithm, this is what the
|declare| function does. Inside some internal tables, the algorithm
layer will store the fact that a  |very simple demo layout| is now
available. The algorithm layer will also communicate with the display
layer through the binding layer to advertise this fact to the
``user.'' In the case of \tikzname, this means that the option key
|very simple demo layout| becomes available at this point and we can
use it like this:

\begin{codeexample}[]
\tikz [very simple demo layout]
  \graph { f -> c -> e -> a -> {b -> {c, d, f}, e -> b}};
\end{codeexample}

It turns out, that our little algorithm is already more powerful than
one might expect. Consider the following example:
\begin{codeexample}[]
\tikz [very simple demo layout, componentwise]
  \graph {
    1 -> 2 ->[orient=right] 3 -> 1;
    a -- b --[orient=45]  c -- d -- a;
  };
\end{codeexample}

Note that, in our algorithm, we ``just'' put all nodes on a circle
around the origin. Nevertheless, the graph gets decomposed into two
connected components, the components are rotated so that the edge from
node |2| to node |3| goes from left to right and the edge from |b| to
|c| goes up at an angle of $45^\circ$, and the components are placed
next to each other so that some spacing is achieved.

The ``magic'' that achieves all this behind the scenes is called 
``graph transformations.'' They will heavily pre- and postprocess the
input and output of graph drawing algorithms to achieve the above
results. 

Naturally, some algorithms may not wish their inputs and/or
outputs to be ``tampered'' with. An algorithm can easily configure
which transformations should be applied, by passing appropriate options
to |declare|.


\subsubsection{Declaring an Algorithm}

Let us now have a look at how one would ``really'' implement the
example algorithm. First of all, we place our algorithm in a
separate file called, say, |ExampleLayout.lua|. This way, by putting
it in a separate file, all display layers can easily install the
algorithm at runtime by saying |require "ExampleLayout"|.

Next, the |declare| function is needed quite often, so it makes sense
to create a short local name for it:

\begin{codeexample}[code only]
-- This is the file ExampleLayout.lua
local declare = require "pgf.gd.interface.InterfaceToAlgorithms".declare  
\end{codeexample}

The |declare| function is the work-horse of the algorithm layer. It
takes a table that contains at least a |key| field, which must be a
unique string, and some other fields that specify in more detail what
kind of key is declared. Once declared through a call of |declare|,
the ``key'' can be used on the display layer.

For declaring an algorithm, the table passed to |declare| must contain
a field |algorithm|. This field, in turn, must (normally) be set to a
table that will become the algorithm class. In the above example, our
algorithm was so simple that we could place the whole definition of
the class inside the call of |declare|, but normally the class is
defined in more detail after the call to |declare|:

\begin{codeexample}[code only]
local ExampleClass = {}  -- A local variable holding for the class table

declare {
  key = "very simple demo layout",
  algorithm = ExampleClass
}

function ExampleClass:run ()
  local alpha = (2 * math.pi) / #self.ugraph.vertices
  ...
end
\end{codeexample}

The effect of the |declare| will be that the table stored in
|ExampleClass| is setup to form a class in the sense of
object-oriented programming. In particular,  a static |new| function
is installed.

Now, whenever the user uses the key |very simple demo layout| on a
graph, at some point the graph drawing engine will create a new
instance of the |ExampleClass| using |new| and will then call the
|run| method of this class. The class can have any number of other
methods, but |new| and |run| are the only ones directly called by the
graph drawing system.


\subsubsection{The Run Method}

The |run| method of an algorithm classes lies at the heart of any
graph drawing algorithm. This method will be called whenever a graph
needs to be laid out. Upon this call, the |self| object will have
some important fields set:
\begin{itemize}
\item |ugraph| This stands for ``undirected graph'' and is the
  ``undirected'' version of the to-be-laid out graph. In this graph,
  whenever there is an arc between $u$ and $v$, there is also an arc
  between $v$ and $u$. It is obtained by considering the syntactic
  digraph and then ``forgetting'' about the actual direction of the
  edges.

  When you have set certain |preconditions| in your algorithm class,
  like |connected=true|, the |ugraph| will satisfy these
  conditions. In particular, the |ugraph| typically will not be the
  underlying undirected graph of the complete syntactic digraph, but
  rather of some part of it. The use of (sub)layouts will also modify
  the syntactic digraph is fancy ways.

  Refer to this graph whenever your algorithm is ``most comfortable''
  with an undirected graph, as is the case for instance for most
  force-base algorithms.
\item |digraph| This stands for ``directed graph'' and is the
  ``semantically directed'' version of the to-be-laid out
  graph. Basically, when happens is that reverse edges in the
  syntactic digraph (an edge like |b <- a|) will yield an |Arc| from
  |a| to |b| in the |digraph| while they yield a |b| to |a| arc and
  edge in the syntactic digraph. Also, undirected edges like |a -- b|
  are replaced by directed edges in both directions between the
  vertices.
\item |scope| The graph drawing scope.
\item |layout| The layout object for this graph. This is a collection
  of kind |layout|. 
\end{itemize}


\subsubsection{Loading Algorithms on Demand}

In order to use the |very simple demo layout| on the display layer,
|declare| must have been called for this key. However, we just saw
that the |declare| function takes the actual class table as parameter
and, thus, whenever an algorithm is declared, it is also completely
loaded and compiled at this point.

This is not always desirable. A user may wish to include a number of
libraries in order to declare a large number of potentially useful
algorithms, but will not actually use all of them. Indeed, at least
for large, complex algorithms, it is preferable that the algorithm's
code is loaded only when the algorithm is used for the first time.

Such a ``loading of algorithms on demand'' is supported through the
option of setting the |algorithm| field in a |declare| to a
string. This string must now be the file name of a Lua file that
contains the code of the actual algorithm. When the key is actually
used for the first time, this file will be loaded. It must return a
table that will be plugged into the |algorithm| field; so subsequent
usages of the key will not load the file again.

The net effect of all this is that you can place implementations of
algorithms in files separate from interface files that just contain
the |declare| commands for these algorithms. You will typically do
this only for rather large algorithms.

For our example, the code would look like this:

\begin{codeexample}[code only]
-- File ExampleLayout.lua
local declare = require "pgf.gd.interface.InterfaceToAlgorithms".declare  
declare {
  key = "very simple demo layout",
  algorithm = "ExampleLayoutImplementation"
}
\end{codeexample}

\begin{codeexample}[code only]
-- File ExampleLayoutImplementation.lua
local ExampleClass = {}
function ExampleClass:run ()
  local alpha = (2 * math.pi) / #self.ugraph.vertices
  ...
end
return ExampleClass
\end{codeexample}


\subsubsection{Declaring Options}

Let us now make our example algorithm a bit more ``configurable''. For
this, we use |declare| once more, but instead of the |algorithm|
field, we use a |type| field. This tells the display layer that the
key is not used to select an algorithm, but to configure ``something''
about the graph or about nodes or edges.

In our example, we may wish to configure the radius of the graph. So,
we introduce a |radius| key (actually, this key already exists, so we
would not need to declare it, but let us do so anyway for example
purposes):

\begin{codeexample}[code only]
declare {
  key = "radius",
  type = "length",
  initial = "25"
}
\end{codeexample}

This tells the display layer that there is now an option called
|radius|, that users set it to some ``length'', and that if it is not
set at all, the value |25| should be used by default.

To access what the user has specified for this key, an algorithm can
simple access the |options| field of a graph, vertex, or arc at the
key's name:

\begin{codeexample}[code only]
          vertex.pos.x = math.cos(i * alpha) * vertex.options['radius']
          vertex.pos.y = math.sin(i * alpha) * vertex.options['radius']
\end{codeexample}



\subsubsection{Adding Documentation}

You should always document the keys you |declare|. For this, the
|declare| function allows you to add three fields to its argument
table:
\begin{itemize}
\item |summary| This should be a string that succinctly summarizes the
  effect this key has. The idea is that this text will be shown as a
  ``tooltip'' in a graphical editor or will be printed out by a
  command line tool when a user requests help about the key.
  You can profit from using Lua's |[[| and |]]| syntax for specifying
  multi-line strings.
  
  Also, when the file containing the key is parsed for
  this manual, this text will be shown.
\item |documentation| When present, this field contains a more
  extensive documentation of the key. It will also be shown in this
  manual, but typically not as a tool tip.
\item |examples| This should either me a single string or an array of
  strings. Each string should be an example demonstrating how the key
  is used in \tikzname. They will all be included in the manual, each
  surrounded by a |codeexample| environment.
\end{itemize}

Let us augment our |radius| key with some documentation. The three
dashes before the |declare| are only needed when the declaration is
part of this manual and they will trigger an inclusion of the key in
the manual.

\begin{codeexample}[code only]
--- 
declare {
  key = "radius",
  type = "length",
  initial = "25",
  summary = [[
    Specifies the radius of a circle on which the nodes are placed when
    the |very simple example layout| is used. Each vertex can have a
    different radius.
  ]],
  examples = [[
    \tikz \graph [very simple example layout,
                  radius=2cm] {
      a -- b -- c -- d -- e;
    };
  ]]
}
\end{codeexample}

As a courtesy, all of the strings given in the documentation can start
and end with quotation marks, which will be removed. (This helps
syntax hilighting with editors that do not recognize the |[[| to |]]|
syntax.) Also, the indentation of the strings is removed (we compute
the minimum number of leading spaces on any line and remove this many
spaced from all lines).



\subsection{Namespaces and File Names}

\label{section-gd-namespaces}

\subsubsection{Namespaces}

All parts of the graph drawing library reside in the Lua ``namespace''
|pgf.gd|, which is itself a ``sub-namespace'' of |pgf|. For your own
algorithms, you are free to place them in whatever namespace you like;
only for the official distribution of \pgfname\ everything has been
put into the correct namespace.

Let us now have a more detailed look at these namespaces. A namespace
is just a Lua table, and sub-namespaces are just subtables of
namespace tables. Following the Java convention, namespaces are in
lowercase letters. The following namespaces are part of the core of
the graph drawing engine:
\begin{itemize}
\item |pgf| This namespace is the main namespace of \pgfname. Other
  parts of \pgfname\ and \tikzname\ that also employ Lua should put an
  entry into this table. Since, currently, only the graph drawing
  engine adheres to this rule, this namespace is declared inside the
  graph drawing directory, but this will change.

  The |pgf| table is the \emph{only} entry into the global table of
  Lua generated by the graph drawing engine (or, \pgfname, for that
  matter). If you intend to extend the graph drawing engine, do not
  even \emph{think} of polluting the global namespace. You will be
  fined.
\item |pgf.gd| This namespace is main namespace of the graph drawing
  engine, including the object-oriented models of graphs and the
  layout pipeline. Algorithms that are part of the
  distribution are also inside this namespace, but if you write your
  own algorithms you do not need place them inside this
  namespace. (Indeed, you probably should not before they are made
  part of the official distribution.)
\item |pgf.gd.interface| This namespace handles, on the one hand, the
  communication between the algorithm layer and the binding layer and,
  on the other hand, the communication between the display layer
  (\tikzname) and the binding layer.
\item |pgf.gd.binding| So-called ``bindings'' between display layers
  and the graph drawing system reside in this namespace.
\item |pgf.gd.lib| Numerous useful classes that ``make an algorithm's
  your life easier'' are collected in this namespace. Examples are a
  class for decomposing a graph into connected components or a class
  for computing the ideal distance between two sibling nodes in a
  tree, taking all sorts of rotations and separation parameters into
  account. 
\item |pgf.gd.model| This namespace contains all Lua classes that are
  part of the object-oriented model of graphs employed
  throughout the graph drawing engine. For readers familiar with the
  model--view--controller pattern: This is the namespace containing
  the model-part of this pattern.
\item |pgf.gd.control| This namespace contains the ``control logic''
  of the graph drawing system. It will transform graphs according to
  rules, disassemble layouts and sublayouts and will call the
  appropriate algorithms. For readers still familiar with the
  model--view--controller pattern: This is the namespace containing
  the control-part of this pattern. 
\item |pgf.gd.trees| This namespace contains classes that are useful
  for dealing with graphs that are trees. In particular, it contains a
  class for computing a spanning tree of an arbitrary connected graph;
  an operation that is an important preprocessing step for many
  algorithms.

  In addition to providing ``utility functions for trees,'' the
  namespace \emph{also} includes actual algorithms for computing graph
  layouts like |pgf.gd.trees.ReingoldTilford1981|. It may seem to be a
  bit of an ``impurity'' that a namespace mixes utility classes and
  ``real'' algorithms, but experience has shown that it is better to
  keep things together in this way.

  Concluding the analogy to the model--view--controller pattern, a
  graph drawing algorithm is, in a loose sense, the ``view'' part of
  the pattern.
\item |pgf.gd.layered| This namespace provides classes and functions
  for ``layered'' layouts; the Sugiyama layout method being the most
  well-known one. Again, the namespace contains both algorithms to be
  used by a user and utility functions.
\item |pgf.gd.force| Collects force-based algorithms and, again, also
  utility functions and classes.
\item |pgf.gd.examples| Contains some example algorithms. They are
  \emph{not} intended to be used directly, rather they should serve as
  inspirations for readers wishing to implement their own algorithms.
\end{itemize}

There are further namespaces that also reside in the |pgf.gd|
namespace, these namespaces are used to organize different graph
drawing algorithms into categories.

In Lua, similarly to Java, when a class |SomeClass| is part of, say,
the namespace |pgf.gd.example|, it is customary to put the class's
code in a file |SomeClass.lua| and then put this class in a directory
|example|, that is a subdirectory of a directory |gd|, which is in
turn a subdirectory of a directory |pgf|. When you write
\texttt{require "pgf.gd.example.SomeClass"} the so-called
\emph{loader} will turn this into a request for the file
\texttt{pgf/gd/example/SomeClass.lua} (for Unix systems).



\subsubsection{Defining and Using Namespaces and Classes}

There are a number of rules concerning the structure and naming of
namespaces as well as the naming of files. Let us start with the
rules for naming namespaces, classes, and functions. They follow the
``Java convention'':

\begin{enumerate}
\item A namespace is a short lowercase |word|.
\item A function in a namespace is in |lowercase_with_underscores_between_words|.
\item A class name is in |CamelCaseWithAnUppercaseFirstLetter|.
\item A class method name is in |camelCaseWithALowercaseFirstLetter|.
\end{enumerate}

From Lua's point of view, every namespace and every class is just a
table. However, since these tables will be loaded using Lua's
|require| function, each namespace and each class must be placed
inside a separate file (unless you modify the |package.loaded| table,
but, then, you know what you are doing anyway). Inside such a file, you
should first declare a local variable whose name is the name of the
namespace or class that you intend to define and then assign a
(possibly empty) table to this variable:
\begin{codeexample}[code only]
-- File pgf.gd.example.SomeClass.lua:
local SomeClass = {}
\end{codeexample}
Next, you should add your class to the encompassing namespace. This is
achieved as follows:
\begin{codeexample}[code only]
require("pgf.gd.example").SomeClass = SomeClass
\end{codeexample}
The reason this works is that the |require| will return the table that
is the namespace |pgf.gd.example|. So, inside this namespace, the
|SomeClass| field will be filled with the table stored in the local
variable of the same name -- which happens to be the table
representing the class.

At the end of the file, you must write
\begin{codeexample}[code only]
return SomeClass  
\end{codeexample}
This ensures that the table that is defined in this file gets stored
by Lua in the right places. Note that you need and should not use
Lua's |module| command. The reason is that this command has
disappeared in the new version of Lua and that it is not really
needed. 

Users of your class can import and use your class by writing:
\begin{codeexample}[code only]
...
local SomeClass = require "pgf.gd.examples.SomeClass"
...  
\end{codeexample}




\subsection{The Graph Drawing Scope}

\label{section-gd-gd-scope}

\includeluadocumentationof{pgf.gd.interface.Scope}



\subsection{The Model Classes}

\label{section-gd-models}

All that a graph drawing algorithm will ``see'' of the graph specified
by the user is a ``graph object.'' Such an object is an
object-oriented model of the user's graph that no longer encodes the
specific way in which the user specified the graph; it only encodes
which nodes and edges are present. For instance, the \tikzname\ graph
specification 
\begin{codeexample}[code only]
graph { a -- {b, c} }
\end{codeexample}
\noindent and the graph specification
\begin{codeexample}[code only]
node (a) { a }
child { node (b) {b} }
child { node (c) {c} }
\end{codeexample}
will generate exactly the same graph object.

\begin{luanamespace}{pgf.gd.}{model}
  This namespace contains the classes modeling graphs,
  nodes, and edges. Also, the |Coordinate| class is found here, since
  coordinates are also part of the modeling.  
\end{luanamespace}


\subsubsection{Directed Graphs (Digraphs)}

Inside the graph drawing engine, the only model of a graph that is 
available treats graphs as
\begin{enumerate}
\item directed (all edges have a designated head and a designated
  tail) and
\item simple (there can be at most one edge between any pair of
  nodes). 
\end{enumerate}
These two properties may appear to be somewhat at odds with what users
can specify as graphs and with what some graph drawing algorithms
might expect as input. For instance, suppose a user writes
\begin{codeexample}[code only]
graph { a -- b --[red] c, b --[green, bend right] c }
\end{codeexample}
In this case, it seems that the input graph for a graph drawing
algorithm should actually be an \emph{undirected} graph in which there
are \emph{multiple} edges (namely $2$) between |b| and~|c|.
Nevertheless, the graph drawing engine will turn the user's input a
directed simple graph in ways described later. You do not need to
worry that information gets lost during this process: The
\emph{syntactic digraph,} which is available to graph drawing
algorithms on request, stores all the information about which edges
are present in the original input graph.

The main reasons for only considering directed, simple graphs are speed
and simplicity: The implementation of these graphs has been optimized so
that all operations on these graphs have a guaranteed running time
that is small in practice. 

\includeluadocumentationof{pgf.gd.model.Digraph}

\subsubsection{Vertices}

\includeluadocumentationof{pgf.gd.model.Vertex}

\subsubsection{Arcs}
\label{section-gd-arc-model}

\includeluadocumentationof{pgf.gd.model.Arc}

\subsubsection{Edges}

\includeluadocumentationof{pgf.gd.model.Edge}

\subsubsection{Collections}

\includeluadocumentationof{pgf.gd.model.Collection}

\subsubsection{Coordinates and Transformations}

\includeluadocumentationof{pgf.gd.model.Coordinate}
\includeluadocumentationof{pgf.gd.lib.Transform}

\subsubsection{Options and Data Storages for Vertices, Arcs, and Digraphs}

\includeluadocumentationof{pgf.gd.lib.Storage}

\subsubsection{Events}

\label{section-gd-events}

\includeluadocumentationof{pgf.gd.lib.Event}



\subsection{Graph Transformations}

\label{section-gd-transformations}

\includeluadocumentationof{pgf.gd.control.LayoutPipeline}



\subsection{The Interface To Algorithms}

\label{section-gd-interface-to-algorithms}

\includeluadocumentationof{pgf.gd.interface.InterfaceToAlgorithms}






\subsection{Examples of Implementations of Graph Drawing Algorithms}
\label{section-gd-examples}

\includeluadocumentationof{pgf.gd.examples.library}
\includeluadocumentationof{pgf.gd.examples.SimpleDemo}
\includeluadocumentationof{pgf.gd.examples.SimpleEdgeDemo}
\includeluadocumentationof{pgf.gd.examples.SimpleHuffman}




\subsection{Support Libraries}

\label{section-gd-libs}

The present section lists a number of general-purpose libraries that
are used by different algorithms.

\subsubsection{Basic Functions}

\includeluadocumentationof{pgf}

\includeluadocumentationof{pgf.gd.lib}

\subsubsection{Lookup Tables}

\includeluadocumentationof{pgf.gd.lib.LookupTable}

\subsubsection{Computing Distances in Graphs}

\emph{Still needs to be ported to digraph classes!}

%\includeluadocumentationof{pgf.gd.lib.PathLengths}

\subsubsection{Priority Queues}

\includeluadocumentationof{pgf.gd.lib.PriorityQueue}




% Copyright 2006 by Till Tantau
%
% This file may be distributed and/or modified
%
% 1. under the LaTeX Project Public License and/or
% 2. under the GNU Free Documentation License.
%
% See the file doc/generic/pgf/licenses/LICENSE for more details.

\section{Syntax for Path Specifications}

A \emph{path} is a series of straight and curved line segments. It is
specified following a |\path| command and the specification must
follow a special syntax, which is described in the subsections of the
present section.


\begin{command}{\path\meta{specification}|;|}
  This command is available only inside a |{tikzpicture}| environment.

  The \meta{specification} is a long stream of \emph{path
  operations}. Most of these path operations tell \tikzname\ how the path
  is build. For example, when you write |--(0,0)|, you use a
  \emph{line-to operation} and it means ``continue the path from
  wherever you are to the origin.''

  At any point where \tikzname\ expects a path operation, you can also
  give some graphic options, which is a list of options in brackets,
  such as |[rounded corners]|. These options can have different
  effects:
  \begin{enumerate}
  \item
    Some options take ``immediate'' effect and apply to all subsequent
    path operations on the path. For example, the |rounded corners|
    option will round all following corners, but not the corners
    ``before'' and if the |sharp corners| is given later on the path
    (in a new set of brackets), the rounding effect will end.

\begin{codeexample}[]
\tikz \draw (0,0) -- (1,1)
           [rounded corners] -- (2,0) -- (3,1)
           [sharp corners] -- (3,0) -- (2,1);
\end{codeexample}
    Another example are the transformation options, which also apply
    only to subsequent coordinates.
  \item
    The options that have immediate effect can be ``scoped'' by
    putting part of a path in curly braces. For example, the above
    example could also be written as follows:

\begin{codeexample}[]
\tikz \draw (0,0) -- (1,1)
           {[rounded corners] -- (2,0) -- (3,1)}
           -- (3,0) -- (2,1);
\end{codeexample}
  \item
    Some options only apply to the path as a whole. For example, the
    |color=| option for determining the color used for, say, drawing
    the path always applies to all parts of the path. If several
    different colors are given for different parts of the path, only
    the last one (on the outermost scope) ``wins'':
 
\begin{codeexample}[]
\tikz \draw (0,0) -- (1,1)
           [color=red] -- (2,0) -- (3,1)
           [color=blue] -- (3,0) -- (2,1);
\end{codeexample}

    Most options are of this type. In the above example, we would have
    had to ``split up'' the path into several |\path| commands:
\begin{codeexample}[]
\tikz{\draw (0,0) -- (1,1);
      \draw [color=red] (2,0) -- (3,1);
      \draw [color=blue] (3,0) -- (2,1);}
\end{codeexample}
  \end{enumerate}

  By default, the |\path| command does ``nothing'' with the
  path, it just ``throws it away.'' Thus, if you write
  |\path(0,0)--(1,1);|, nothing is drawn 
  in your picture. The only effect is that the area occupied by the
  picture is (possibly) enlarged so that the path fits inside the
  area. To actually ``do'' something with the path, an option like
  |draw| or |fill| must be given somewhere on the path. Commands like
  |\draw| do this implicitly.
  
  Finally, it is also possible to give \emph{node specifications} on a
  path. Such specifications can come at different locations, but they
  are always allowed when a normal path operation could follow. A node
  specification starts with |node|. Basically, the effect is to
  typeset the node's text as normal \TeX\ text and to place
  it at the ``current location'' on the path. The details are explained
  in Section~\ref{section-nodes}.

  Note, however, that the nodes are \emph{not} part of the path in any
  way. Rather, after everything has been done with the path what is
  specified by the path options (like filling and drawing the path due
  to a |fill| and a |draw| option somewhere in the
  \meta{specification}), the nodes are added in a post-processing
  step.   
  
  The following style influences scopes:
  \begin{stylekey}{/tikz/every path (initially \normalfont empty)}
    This style is installed at the beginning of every path. This can
    be useful for (temporarily) adding, say, the |draw| option to
    everything in a scope.
\begin{codeexample}[]
\begin{tikzpicture}
  [fill=examplefill,          % only sets the color
   every path/.style={draw}]  % all paths are drawn
  \fill  (0,0) rectangle +(1,1);
  \shade (2,0) rectangle +(1,1);
\end{tikzpicture}
\end{codeexample}
  \end{stylekey}

\end{command}




\subsection{The Move-To Operation}

The perhaps simplest operation is the move-to operation, which is
specified by just giving a coordinate where a path operation is
expected.

\begin{pathoperation}[noindex]{}{\meta{coordinate}}
  \index{empty@\protect\meta{empty} path operation}%
  \index{Path operations!empty@\protect\texttt{\meta{empty}}}%
  The move-to operation normally starts a path at a certain
  point. This does not cause a line segment to be created, but it  
  specifies the starting point of the next segment. If a path is
  already under construction, that is, if several segments have
  already been created, a move-to operation will start a new part of the
  path that is not connected to any of the previous segments.

\begin{codeexample}[]
\begin{tikzpicture}
  \draw (0,0) --(2,0) (0,1) --(2,1);
\end{tikzpicture}
\end{codeexample}

  In the specification |(0,0) --(2,0) (0,1) --(2,1)| two move-to
  operations are specified: |(0,0)| and |(0,1)|. The other two
  operations, namely |--(2,0)| and |--(2,1)| are line-to operations,
  described next.
\end{pathoperation}


\subsection{The Line-To Operation}


\subsubsection{Straight Lines}

\begin{pathoperation}{--}{\meta{coordinate}}
  The line-to operation extends the current path from the current
  point in a straight line to the given coordinate. The ``current
  point'' is the endpoint of the previous drawing operation or the point
  specified by a prior move-to operation.

  You use two minus signs followed by a coordinate in round
  brackets. You can add spaces before and after the~|--|.

  When a line-to operation is used and some path segment has just been
  constructed, for example by another line-to operation, the two line
  segments become joined. This means that if they are drawn, the point
  where they meet is ``joined'' smoothly. To appreciate the difference,
  consider the following two examples: In the left example, the path
  consists of two path segments that are not joined, but that happen to
  share a point, while in the right example a smooth join is shown.

\begin{codeexample}[]
\begin{tikzpicture}[line width=10pt]
  \draw (0,0) --(1,1)  (1,1) --(2,0);
  \draw (3,0) -- (4,1) -- (5,0);
  \useasboundingbox (0,1.5); % make bounding box higher
\end{tikzpicture}
\end{codeexample}

\end{pathoperation}


\subsubsection{Horizontal and Vertical Lines}

Sometimes you want to connect two points via straight lines that are
only horizontal and vertical. For this, you can use two path
construction operations.

{\catcode`\|=12
\begin{pathoperation}[noindex]{-|}{\meta{coordinate}}
  \index{--1@\protect\texttt{-\protect\pgfmanualbar} path operation}%
  \index{Path operations!--1@\protect\texttt{-\protect\pgfmanualbar}}%
  This operation means ``first horizontal, then vertical.''

  \begin{codeexample}[]
\begin{tikzpicture}
  \draw (0,0) node(a) [draw] {A}  (1,1) node(b) [draw] {B};
  \draw (a.north) |- (b.west);
  \draw[color=red] (a.east) -| (2,1.5) -| (b.north);
\end{tikzpicture}
\end{codeexample}
\end{pathoperation}
\begin{pathoperation}[noindex]{|-}{\meta{coordinate}}
  \index{--2@\protect\texttt{\protect\pgfmanualbar-} path operation}%
  \index{Path operations!--2@\protect\texttt{\protect\pgfmanualbar-}}%
  This operations means  ``first vertical, then horizontal.''
\end{pathoperation}
}


\subsubsection{Snaked Lines}
\label{section-tikz-snakes}

The line-to operation can not only be used to append straight lines to
the path, but also ``snaked'' lines like this one:
\tikz\draw[snake=snake] (0,0) -- (1,0);. They are called ``snakes''
because they look a little bit like snakes seen from above.

\tikzname\ and \pgfname\ use a concept that I termed \emph{snakes}
for appending such ``squiggly'' lines. A snake specifies a way of
extending a path between two points in a ``fancy manner.''

Normally, a snake will just connect the start point to the end point
without starting new subpaths. Thus, a path containing a snaked line
can, nevetheless, still be used for filling. However, this is not
always the case. Some snakes consist of numerous unconnected
segments. ``Lines'' consisting of such snakes cannot be used as the
borders of enclosed areas.

Here are some examples of snakes in action:

\begin{codeexample}[]
\begin{tikzpicture}[thick]
  \draw                                        (0,3)   -- (3,3);
  \draw[snake=zigzag]                          (0,2.5) -- (3,2.5);
  \draw[snake=brace]                           (0,2)   -- (3,2);
  \draw[snake=triangles]                       (0,1.5) -- (3,1.5);
  \draw[snake=coil,segment length=4pt]         (0,1)   -- (3,1);
  \draw[snake=coil,segment aspect=0]           (0,.5)  -- (3,.5);
  \draw[snake=expanding waves,segment angle=7] (0,0)   -- (3,0);
\end{tikzpicture}
\end{codeexample}

\begin{codeexample}[]
\begin{tikzpicture}
  \filldraw[fill=red!20,snake=bumps] (0,0) rectangle (3,2);
\end{tikzpicture}
\end{codeexample}

\begin{codeexample}[]
\begin{tikzpicture}
  \filldraw[fill=blue!20]              (0,3)
  [snake=saw]                       -- (3,3)
  [snake=coil,segment aspect=0]     -- (2,1)
  [snake=bumps]                     -| (0,3);
\end{tikzpicture}
\end{codeexample}

\begin{codeexample}[]
\begin{tikzpicture}
  \filldraw [fill=yellow!50,
             snake=random steps,segment length=3pt,segment amplitude=1pt]
     (0,0) rectangle (3,2);
  \node at (1.5,1) {Saved from trash};
\end{tikzpicture}
\end{codeexample}

\begin{codeexample}[]
\begin{tikzpicture}
  \shade [left color=green,right color=black,
          snake=random steps,segment length=1mm,segment amplitude=3mm]
    (0,0) rectangle (3,2);
\end{tikzpicture}
% You do not want to meet this snake on a dark street...          
\end{codeexample}

No special path operation is needed to use a snake. Instead, you use
the following option to ``switch on'' snaking:

\begin{key}{/tikz/snake=\meta{snake name} (default \normalfont is scope's
  snake)}
  This option causes the snake \meta{snake name} to be used for
  subsequent line-to operations. So, whenever you use the |--| syntax
  to specify that a straight line should be added to the path, a snake
  to this path will be added instead. Snakes will also be used when
  you use the \verb!-|! and \verb!|-! syntax and also when you use the
  |rectangle| operation. Snakes will \emph{not} be used when you use
  the curve-to operation nor when any other ``curved'' line is added
  to the path.

  This option has to be given anew for each path. However, you can
  also leave out the \meta{snake name}. In this case, the enclosing
  scope's \meta{snake name} is used. Thus, you can specify a
  ``standard'' snake name for scope and then just say |\draw[snake]|
  every time this snake should actually be used.

  The \meta{snake name} |none| is special. It can be used to switch
  off snaking after it has been switched on on a path.

  A bit strangely, no valid \meta{snake names} are defined by
  \tikzname\ by default. Instead, you have to include the library
  package |pgflibrarysnakes|. This package defines numerous snakes,
  see Section~\ref{section-library-snakes} for the complete list.
\end{key}

Most snakes can be configured. For example, for a snake that looks
like a sine curve, you might wish to change the amplitude or the
frequency. There are numerous options that influence these
parameters. Not all options apply to all snakes, see
Section~\ref{section-library-snakes} once more for details.

\begin{key}{/tikz/gap before snakes=\meta{dimension}}
  This option allows you to add a certain ``gap'' to the snake at its
  beginning. The snake will not start at the current point; instead
  the start point of the snake is move be \meta{dimension} in the
  direction of the target.
\begin{codeexample}[]
\begin{tikzpicture}
  \draw[help lines] (0,0) grid (3,2);
  \draw[snake=zigzag]                      (0,1) -- ++(3,1);
  \draw[snake=zigzag,gap before snake=1cm] (0,0) -- ++(3,1);
\end{tikzpicture}
\end{codeexample}
\end{key}

\begin{key}{/tikz/gap after snake=\meta{dimension}}
  This option has the same effect as |gap before snake|, only it
  affects the end of the snake, which will ``end early.''
\end{key}
\begin{key}{/tikz/gap around snake=\meta{dimension}}
  This option sets the gap before and after the gap to
  \meta{dimension}. 
\begin{codeexample}[]
\begin{tikzpicture}
  \draw[help lines] (0,0) grid (3,2);
  \draw[snake=brace]                      (0,1) -- ++(3,1);
  \draw[snake=brace,gap around snake=5mm] (0,0) -- ++(3,1);
\end{tikzpicture}
\end{codeexample}
\end{key}
\begin{key}{/tikz/line before snake=\meta{dimension}}
  This option works like |gap before snake|, only it will connect the
  current point with a straight line to the start of the snake.
\begin{codeexample}[]
\begin{tikzpicture}
  \draw[help lines] (0,0) grid (3,2);
  \draw[snake=zigzag]                       (0,1) -- ++(3,1);
  \draw[snake=zigzag,line before snake=1cm] (0,0) -- ++(3,1);
\end{tikzpicture}
\end{codeexample}
\end{key}
\begin{key}{/tikz/line after snake=\meta{dimension}}
  Works line |gap after snake|, only it adds a straight line.
\end{key}
\begin{key}{/tikz/line around snake=\meta{dimension}}
  Works line |gap around snake|, only it adds straight lines.
\end{key}
\begin{key}{/tikz/raise snake=\meta{dimension}}
  This option can be used with all snakes. It will offset the snake by
  ``raising'' it by \meta{dimension}. A negative \meta{dimension} will
  lower the snake. Raising and lowering is always relative to the line
  along which the snake is drawn. Here is an example:
\begin{codeexample}[]
\begin{tikzpicture}
  \node (a) {A};
  \node (b) at (2,1) {B};
  \draw                                  (a) -- (b);
  \draw[snake=brace]                     (a) -- (b);
  \draw[snake=brace,raise snake=5pt,red] (a) -- (b);
\end{tikzpicture}
\end{codeexample}
\end{key}
\begin{key}{/tikz/mirror snake}
  This option causes the snake to be ``reflected along the path.''
  This is best understood by looking at an example:
\begin{codeexample}[]
\begin{tikzpicture}
  \node (a) {A};
  \node (b) at (2,1) {B};
  \draw                                     (a) -- (b);
  \draw[snake=brace]                        (a) -- (b);
  \draw[snake=brace,mirror snake,red,thick] (a) -- (b);
\end{tikzpicture}
\end{codeexample}
  This option can be used with every snake and can be combined with
  the |raise snake| option.
\end{key}


\begin{key}{/pgf/segment amplitude=\meta{dimension} (initially 2.5pt)}
  \keyalias{tikz}  
  This option sets the ``amplitude'' of the snake. For a snake that is
  a sine wave this would be the amplitude of this line. For other
  snakes this value typically describes how far the snakes ``rises
  above'' or ``falls below'' the path. For some snakes, this value is
  ignored. 
\begin{codeexample}[]
\begin{tikzpicture}
  \node (a) {A}   node (b) at (2,1) {B}  node (c) at (2,-1) {C};
  \draw[snake=zigzag]                                 (a) -- (b);
  \draw[snake=zigzag,segment amplitude=5pt,red,thick] (a) -- (c);
\end{tikzpicture}
\end{codeexample}
\end{key}

\begin{key}{/pgf/segment length=\meta{dimension} (initially 10pt)}
  \keyalias{tikz}
  This option sets the length of each ``segment'' of a snake. For a
  sine wave this would be the wave length, for other snakes it is the
  length of each ``repetitive part'' of the snake.
\begin{codeexample}[]
\begin{tikzpicture}
  \node (a) {A}   node (b) at (2,1) {B}  node (c) at (2,-1) {C};
  \draw[snake=zigzag]                               (a) -- (b);
  \draw[snake=zigzag,segment length=20pt,red,thick] (a) -- (c);
\end{tikzpicture}
\end{codeexample}
\begin{codeexample}[]
\begin{tikzpicture}
  \node (a) {A}   node (b) at (2,1) {B}  node (c) at (2,-1) {C};
  \draw[snake=bumps]                               (a) -- (b);
  \draw[snake=bumps,segment length=20pt,red,thick] (a) -- (c);
\end{tikzpicture}
\end{codeexample}
\end{key}

\begin{key}{/pgf/segment object length=\meta{dimension} (initially
    \normalfont same as |/pgf/segment amplitude|)}
  \keyalias{tikz}
  This option sets the length of the objects inside each segment of a
  snake. This option is only used for snakes in which each segment
  contains an object like a triangle or a star. 
\begin{codeexample}[]
\begin{tikzpicture}
  \node (a) {A}   node (b) at (2,1) {B}  node (c) at (2,-1) {C};
  \draw[snake=triangles]                                     (a) -- (b);
  \draw[snake=triangles,segment object length=8pt,red,thick] (a) -- (c);
\end{tikzpicture}
\end{codeexample}
\end{key}

\begin{key}{/pgf/segment angle=\meta{degrees} (intially 45)}
  \keyalias{tikz}
  This option sets an angle that is interpreted in a snake-specific
  way. For example, the |waves| and |expanding waves| snakes interpret
  this as (half the) opening angle of the wave. The |border| snake
  uses this value for the angle of the little ticks.
\begin{codeexample}[]
\begin{tikzpicture}[segment amplitude=10pt]
  \node (a) {A}   node (b) at (2,0) {B};
  \draw[snake=border]                            (a) -- (b);
  \draw[snake=border,segment angle=20,red,thick] (a) -- (b);
\end{tikzpicture}
\end{codeexample}
\begin{codeexample}[]
\begin{tikzpicture}[segment amplitude=10pt]
  \node (a)            {A}   node (b)  at (2,0)  {B};
  \node (a') at (0,-1) {A}   node (b') at (2,-1) {B};
  \draw[snake=expanding waves]                            (a)  -- (b);
  \draw[snake=expanding waves,segment angle=20,red,thick] (a') -- (b');
\end{tikzpicture}
\end{codeexample}
\end{key}

\begin{key}{/pgf/segment aspect=\meta{ratio} (initially 0.5)}
  \keyalias{tikz}
  This option sets an aspect ratio that is interpreted in a
  snake-specific way. For example, for the |coils| snake this
  describes the ``direction'' from which the coil is viewed.
\begin{codeexample}[]
\begin{tikzpicture}[segment amplitude=5pt,segment length=5pt]
  \node (a) {A}   node (b) at (2,1) {B}  node (c) at (2,-1) {C};
  \draw[snake=coil]                            (a) -- (b);
  \draw[snake=coil,segment aspect=0,red,thick] (a) -- (c);
\end{tikzpicture}
\end{codeexample}
\end{key}

It is possible to define new snakes, but this cannot be done inside
\tikzname. You need to use the command |\pgfdeclaresnake| from the
basic level directly, see Section~\ref{section-base-snakes}.

The following styles define combinations of segment settings that may
be useful:
\begin{stylekey}{/tikz/snake triangles 45}
  Installs a snake the consists of little triangles with an opening
  angle of $45^\circ$.
\end{stylekey}

\begin{stylekey}{/tikz/snake triangles 60}
  Installs a snake the consists of little triangles with an opening
  angle of $60^\circ$.
\end{stylekey}

\begin{stylekey}{/tikz/snake triangles 90}
  Installs a snake the consists of little triangles with an opening
  angle of $90^\circ$.
\end{stylekey}



\subsection{The Curve-To Operation}

The curve-to operation allows you to extend a path using a B�zier
curve.

\begin{pathoperation}{..}{\declare{|controls|}\meta{c}\opt{|and|\meta{d}}\declare{|..|\meta{y}}}
  This operation extends the current path from the current
  point, let us call it $x$, via a curve to a the current point~$y$.
  The curve is a cubic B�zier curve. For such a curve, 
  apart from $y$, you also specify two control points $c$ and $d$. The
  idea is that the curve starts at $x$, ``heading'' in the direction
  of~$c$. Mathematically spoken, the tangent of the curve at $x$ goes
  through $c$. Similarly, the curve ends at $y$, ``coming from'' the
  other control point,~$d$. The larger the distance between $x$ and~$c$
  and between $d$ and~$y$, the larger the curve will be.

  If the ``|and|\meta{d}'' part is not given, $d$ is assumed to be
  equal to $c$.

\begin{codeexample}[]
\begin{tikzpicture}
  \draw[line width=10pt] (0,0) .. controls (1,1) .. (4,0)
                               .. controls (5,0) and (5,1) .. (4,1);
  \draw[color=gray] (0,0) -- (1,1) -- (4,0) -- (5,0) -- (5,1) -- (4,1);
\end{tikzpicture}
\end{codeexample}

  As with the line-to operation, it makes a difference whether two curves
  are joined because they resulted from consecutive curve-to or line-to
  operations, or whether they just happen to have the same ending:

\begin{codeexample}[]
\begin{tikzpicture}[line width=10pt]
  \draw (0,0) -- (1,1) (1,1) .. controls (1,0) and (2,0) .. (2,0);
  \draw (3,0) -- (4,1) .. controls (4,0) and (5,0) .. (5,0);
  \useasboundingbox (0,1.5); % make bounding box higher
\end{tikzpicture}
\end{codeexample}
\end{pathoperation}


\subsection{The Cycle Operation}

\begin{pathoperation}{--cycle}{}
  This operation adds a straight line from the current
  point to the last point specified by a move-to operation. Note that
  this need not be the beginning of the path. Furthermore, a smooth join
  is created between the first segment created after the last move-to
  operation and the straight line appended by the cycle operation.

  Consider the following example. In the left example, two triangles are
  created using three straight lines, but they are not joined at the
  ends. In the second example cycle operations are used.

\begin{codeexample}[]
\begin{tikzpicture}[line width=10pt]
  \draw (0,0) -- (1,1) -- (1,0) -- (0,0) (2,0) -- (3,1) -- (3,0) -- (2,0);
  \draw (5,0) -- (6,1) -- (6,0) -- cycle (7,0) -- (8,1) -- (8,0) -- cycle;
  \useasboundingbox (0,1.5); % make bounding box higher
\end{tikzpicture}
\end{codeexample}
\end{pathoperation}



\subsection{The Rectangle Operation}

A rectangle can obviously be created using four straight lines and a
cycle operation. However, since rectangles are needed so often, a
special syntax is available for them.

\begin{pathoperation}{rectangle}{\meta{corner}}
  When this operation is used, one corner will be the current point,
  another corner is given by \meta{corner}, which becomes the new
  current point.

\begin{codeexample}[]
\begin{tikzpicture}
  \draw (0,0) rectangle (1,1);
  \draw (.5,1) rectangle (2,0.5) (3,0) rectangle (3.5,1.5) -- (2,0);
\end{tikzpicture}
\end{codeexample}
\end{pathoperation}


\subsection{Rounding Corners}

All of the path construction operations mentioned up to now are
influenced by the following option:
\begin{key}{/tikz/rounded corners=\meta{inset} (default 4pt)}
  When this option is in force, all corners (places where a line is
  continued either via line-to or a curve-to operation) are replaced by
  little arcs so that the corner becomes smooth. 

\begin{codeexample}[]
\tikz \draw [rounded corners] (0,0) -- (1,1)
           -- (2,0) .. controls (3,1) .. (4,0);
\end{codeexample}

  The \meta{inset} describes how big the corner is. Note that the
  \meta{inset} is \emph{not} scaled along if you use a scaling option
  like |scale=2|. 

\begin{codeexample}[]
\begin{tikzpicture}
  \draw[color=gray,very thin] (10pt,15pt) circle (10pt);
  \draw[rounded corners=10pt] (0,0) -- (0pt,25pt) -- (40pt,25pt);
\end{tikzpicture}
\end{codeexample}

  You can switch the rounded corners on and off ``in the middle of
  path'' and different corners in the same path can have different
  corner radii:

\begin{codeexample}[]
\begin{tikzpicture}
  \draw (0,0) [rounded corners=10pt] -- (1,1) -- (2,1)
                     [sharp corners] -- (2,0)
               [rounded corners=5pt] -- cycle;
\end{tikzpicture}
\end{codeexample}

Here is a rectangle with rounded corners:
\begin{codeexample}[]
\tikz \draw[rounded corners=1ex] (0,0) rectangle (20pt,2ex);
\end{codeexample}

  You should be aware, that there are several pitfalls when using this
  option. First, the rounded corner will only be an arc (part of a
  circle) if the angle is $90^\circ$. In other cases, the rounded
  corner will still be round, but ``not as nice.''

  Second, if there are very short line segments in a path, the
  ``rounding'' may cause inadverted effects. In such case it may be
  necessary to temporarily switch off the rounding using
  |sharp corners|. 
\end{key}

\begin{key}{/tikz/sharp corners}
  This options switches off any rounding on subsequent corners of the
  path.   
\end{key}



\subsection{The Circle and Ellipse Operations}

A circle can be approximated well using four B�zier curves. However,
it is difficult to do so correctly. For this reason, a special syntax
is available for adding such an approximation of a circle to the
current path.

\begin{pathoperation}{circle}{|(|\meta{radius}|)|}
  The center of the circle is given by the current point. The new
  current point of the path will remain to be the center of the
  circle.  
\end{pathoperation}

\begin{pathoperation}{ellipse}{|(|\meta{half width}| and |\meta{half height}|)|}
  Note that you can add spaces after |ellipse|, but you have to place
  spaces around |and|.

\begin{codeexample}[]
\begin{tikzpicture}
  \draw (1,0) circle (.5cm);
  \draw (3,0) ellipse (1cm and .5cm) -- ++(3,0) circle (.5cm)
    -- ++(2,-.5) circle (.25cm);
\end{tikzpicture}
\end{codeexample}
\end{pathoperation}


\subsection{The Arc Operation}

The \emph{arc operation} allows you to add an arc to the current
path.
\begin{pathoperation}{arc}{|(|\meta{start angle}|:|\meta{end
    angle}|:|\meta{radius}\opt{| and |\meta{half height}}|)|}
  The arc operation adds a part of a circle of the given radius
  between the given angles. The arc will start at the current point
  and will end at the end of the arc.

  \begin{codeexample}[]
\begin{tikzpicture}
  \draw (0,0) arc (180:90:1cm) -- (2,.5) arc (90:0:1cm);
  \draw (4,0) -- +(30:1cm) arc (30:60:1cm) -- cycle;
  \draw (8,0) arc (0:270:1cm and .5cm) -- cycle;
\end{tikzpicture}
\end{codeexample}

\begin{codeexample}[]
\begin{tikzpicture}
  \draw (-1,0) -- +(3.5,0);
  \draw (1,0) ++(210:2cm) -- +(30:4cm);
  \draw (1,0) +(0:1cm) arc (0:30:1cm);      
  \draw (1,0) +(180:1cm) arc (180:210:1cm);
  \path (1,0) ++(15:.75cm) node{$\alpha$};
  \path (1,0) ++(15:-.75cm) node{$\beta$};
\end{tikzpicture}
\end{codeexample}
\end{pathoperation}


\subsection{The Grid Operation}

You can add a grid to the current path using the |grid| path
operation. 

\begin{pathoperation}{grid}{\opt{\oarg{options}}\meta{corner}}
  This operations adss a grid filling a rectangle whose two corners
  are given by \meta{corner} and by the previous coordinate. Thus, the
  typical way in which a grid is drawn is |\draw (1,1) grid (3,3);|,
  which yields a grid filling the rectangle whose corners are at
  $(1,1)$ and $(3,3)$. All coordinate transformations apply to the
  grid.

\begin{codeexample}[]
\tikz[rotate=30] \draw[step=1mm] (0,0) grid (2,2);
\end{codeexample}

  The \meta{options}, which are local to the |grid| operation, can be
  used to influence the appearance of the grid. The stepping of the
  grid is governed by the following options: 

\begin{key}{/tikz/step=\meta{number or dimension or coordinate}
    (initially 1cm)}
  Sets the stepping in both the
  $x$ and $y$-direction. If a dimension is provided, this is used
  directly. If a number is provided, this number is interpreted in the
  $xy$-coordinate system. For example, if you provide the number |2|,
  then the $x$-step is twice the $x$-vector and the $y$-step is twice
  the $y$-vector set by the |x=| and |y=| options. Finally, if you
  provide a coordinate, then the $x$-part of this coordinate will be
  used as the $x$-step and the $y$-part will be used as the
  $y$-coordinate.

\begin{codeexample}[]
\begin{tikzpicture}[x=.5cm]
  \draw[thick] (0,0) grid [step=1]     (3,2);
  \draw[red]   (0,0) grid [step=.75cm] (3,2);
\end{tikzpicture}
\begin{tikzpicture}
  \draw        (0,0) circle (1);
  \draw[blue]  (0,0) grid [step=(45:1)] (3,2);
\end{tikzpicture}
\end{codeexample}  

  A complication arises when the $x$- and/or $y$-vector do not point
  along the axes. Because of this, the actual rule for computing the
  $x$-step and the $y$-step is the following: As the $x$- and
  $y$-steps we use the $x$- and $y$-components or the following two
  vectors: The first vector is either $(\meta{x-grid-step-number},0)$
  or $(\meta{x-grid-step-dimension},0\mathrm{pt})$, the second vector
  is  $(0,\meta{y-grid-step-number})$ or
  $(0\mathrm{pt},\meta{x-grid-step-dimension})$. 
\end{key}  

\begin{key}{/tikz/xstep=\meta{dimension or number} (initially 1cm)}
  Sets the stepping in the $x$-direction. 
\begin{codeexample}[]
\tikz \draw (0,0) grid [xstep=.5,ystep=.75] (3,2);
\end{codeexample}
\end{key}

\begin{key}{/tikz/ystep=\meta{dimension or number} (initially 1cm)}
  Sets the stepping in the $y$-direction. 
\end{key}

  It is important to note that the grid is always ``phased'' such that
  it contains the point $(0,0)$ if that point happens to be inside the
  rectangle. Thus, the grid does \emph{not} always have an intersection
  at the corner points; this occurs only if the corner points are
  multiples of the stepping. Note that due to rounding errors, the
  ``last'' lines of a grid may be omitted. In this case, you have to
  add an epsilon to the corner points.

  The following style is useful for drawing grids:
\begin{stylekey}{/tikz/help lines (initially {line width=0.2pt,gray})}
  This style makes lines ``subdued'' by using thin gray lines for
  them. However, this style is not installed automatically and you
  have to say for example:
\begin{codeexample}[]
\tikz \draw[help lines] (0,0) grid (3,3);
\end{codeexample}
\end{stylekey}

\end{pathoperation}



\subsection{The Parabola Operation}

The |parabola| path operation continues the current path with a
parabola. A parabola is a (shifted and scaled) curve defined by the
equation $f(x) = x^2$ and looks like this: \tikz \draw (-1ex,1.5ex)
parabola[parabola height=-1.5ex] +(2ex,0ex);.

\begin{pathoperation}{parabola}{\opt{\oarg{options}|bend|\meta{bend
        coordinate}}\meta{coordinate}}
  This operation adds a parabola through the current point and the
  given \meta{coordinate}. If the |bend| is given, it specifies where
  the bend should go; the \meta{options} can also be used to specify
  where the bend is. By default, the bend is at the old current point. 

\begin{codeexample}[]
\begin{tikzpicture}
  \draw               (0,0) rectangle                (1,1.5)
                      (0,0) parabola                 (1,1.5);
  \draw[xshift=1.5cm] (0,0) rectangle                (1,1.5)
                      (0,0) parabola[bend at end]    (1,1.5);
  \draw[xshift=3cm]   (0,0) rectangle                (1,1.5)
                      (0,0) parabola bend (.75,1.75) (1,1.5);
\end{tikzpicture}
\end{codeexample}

  The following options influence parabolas:
\begin{key}{/tikz/bend=\meta{coordinate}}
  Has the same effect as saying |bend|\meta{coordinate} outside the
  \meta{options}. The option specifies that the bend of the parabola
  should be at the given \meta{coordinate}. You have to take care
  yourself that the bend position is a ``valid'' position; which means
  that if there is no parabola of the form $f(x) = a x^2 + b x + c$
  that goes through the old current point, the given bend, and the new
  current point, the result will not be a parabola.

  There is one special property of the \meta{coordinate}: When a
  relative coordinate is given like |+(0,0)|, the position relative
  to which this coordinate is ``flexible.'' More precisely, this
  position lies somewhere on a line from the old current point to the
  new current point. The exact position depends on the next
  option.
\end{key}

\begin{key}{/tikz/bend pos=\meta{fraction}}
  Specifies where the ``previous'' point is relative to which the bend
  is calculated. The previous point will be at the \meta{fraction}th
  part of the line from the old current point to the new current
  point.

  The idea is the following: If you say |bend pos=0| and
  |bend +(0,0)|, the bend will be at the old current point. If you say
  |bend pos=1| and |bend +(0,0)|, the bend will be at the new current
  point. If you say |bend pos=0.5| and |bend +(0,2cm)| the bend will
  be 2cm above the middle of the line between the start and end
  point. This is most useful in situations such as the following:
\begin{codeexample}[]
\begin{tikzpicture}
  \draw[help lines] (0,0) grid (3,2);
  \draw (-1,0) parabola[bend pos=0.5] bend +(0,2) +(3,0);
\end{tikzpicture}
\end{codeexample}

  In the above example, the |bend +(0,2)| essentially means ``a
  parabola that is 2cm high'' and |+(3,0)| means ``and 3cm wide.''
  Since this situation arises often, there is a special shortcut
  option:
  \begin{key}{/tikz/parabola height=\meta{dimension}}
    This option has the same effect as
    |[bend pos=0.5,bend={+(0pt,|\meta{dimension}|)}]|. 
\begin{codeexample}[]
\begin{tikzpicture}
  \draw[help lines] (0,0) grid (3,2);
  \draw (-1,0) parabola[parabola height=2cm] +(3,0);
\end{tikzpicture}
\end{codeexample}
  \end{key}
\end{key}

The following styles are useful shortcuts:
\begin{stylekey}{/tikz/bend at start}
  This places the bend at the start of a
  parabola. It is a shortcut for the following options:
  |bend pos=0,bend={+(0,0)}|.
\end{stylekey}

\begin{stylekey}{/tikz/bend at end}
  This places the bend at the end of a parabola.
\end{stylekey}

\end{pathoperation}


\subsection{The Sine and Cosine Operation}

The |sin| and |cos| operations are similar to the |parabola|
operation. They, too, can be used to draw (parts of) a sine or cosine
curve.

\begin{pathoperation}{sin}{\meta{coordinate}}
  The effect of |sin| is to draw a scaled and shifted version of a sine
  curve in the interval $[0,\pi/2]$. The scaling and shifting is done in
  such a way that the start of the sine curve in the interval is at the
  old current point and that the end of the curve in the interval is at
  \meta{coordinate}. Here is an example that should clarify this:

\begin{codeexample}[]
\tikz \draw (0,0) rectangle (1,1)     (0,0) sin (1,1)
            (2,0) rectangle +(1.57,1) (2,0) sin +(1.57,1);
\end{codeexample}
\end{pathoperation}

\begin{pathoperation}{cos}{\meta{coordinate}}
  This operation works similarly, only a cosine in the interval
  $[0,\pi/2]$ is drawn. By correctly alternating |sin| and |cos|
  operations, you can create a complete sine or cosine curve:

\begin{codeexample}[]
\begin{tikzpicture}[xscale=1.57]
  \draw (0,0) sin (1,1) cos (2,0) sin (3,-1) cos (4,0) sin (5,1);
  \draw[color=red] (0,1.5) cos (1,0) sin (2,-1.5) cos (3,0) sin (4,1.5) cos (5,0);
\end{tikzpicture}
\end{codeexample}
\end{pathoperation}

Note that there is no way to (conveniently) draw an interval on a sine
or cosine curve whose end points are not multiples of $\pi/2$.



\subsection{The Plot Operation}

The |plot| operation can be used to append a line or curve to the path
that goes through a large number of coordinates. These coordinates are
either given in a simple list of coordinates, read from some file, or
they are computed on the fly.

Since the syntax and the behaviour of this command are a bit complex,
they are described in the separated Section~\ref{section-tikz-plots}.

\subsection{The To Path Operation}

The |to| operation is used to add a user-defined path
from the previous coordinate to the following coordinate. When you
write |(a) to (b)|, a straight line is added from |a|
to |b|, exactly as if you had written |(a) -- (b)|. However, if you
write |(a) to [out=135,in=45] (b)| a curve is added to the path,
which leaves at an angle of 135$^\circ$ at |a| and arrives at an angle
of 45$^\circ$ at |b|. This is because the options |in| and |out|
trigger a special path to be used instead of the straight line. 

\begin{pathoperation}{to}{\opt{|[|\meta{options}|]|}
    \opt{\meta{nodes}} |(|\meta{coordinate}|)|}

  This path operation inserts the path current set via the |to path|
  option at the current position. The \meta{options} can be used to
  modify (perhaps implicitly) the |to path| and to setup how the path
  will be rendered. 

  Before the |to path| is inserted, a number of macros are setup that
  can ``help'' the |to path|. These are |\tikztostart|,
  |\tikztotarget|, and |\tikztonodes|; they are explained in the
  following. 
  
  \medskip
  \textbf{Start and Target Coordinates.}\ \ 
  The |to| operation is always followed by a \meta{coordinate}, called
  the target coordinate. The macro |\tikztotarget| is set to this
  coordinate (without the parantheses). There is also a \emph{start
    coordinate}, which is the coordinate preceding the |to|
  operation. This coordinate can be accessed via the macro
  |\tikztostart|. In the following example, for the first |to|, the
  macro |\tikztostart| is |0pt,0pt| and the |\tikztotarget| is
  |0,2|. For the second |to|, the macro |\tikztostart| is |10pt,10pt|
  and |\tikztotarget| is |a|.
  
\begin{codeexample}[]
\begin{tikzpicture}
  \draw[help lines] (0,0) grid (3,2);
  
  \draw      (0,0)       to (0,2);
  \node      (a)         at (2,2) {a};
  \draw[red] (10pt,10pt) to (a);
\end{tikzpicture}
\end{codeexample}


  \medskip
  \textbf{Nodes on tos.}\ \
  It is possible to add nodes to the paths constructed by a |to|
  operation. To do so, you specify the nodes between the |to|
  keyword and the coordinate (if there are options to the |to|
  operation, these come first). The effect of |(a) to node {x} (b)| 
  (typically) is the same as if you had written
  |(a) -- node {x} (b)|, namely that the node is placed on the
  to. This can be used to add labels to tos: 
  
\begin{codeexample}[]
\begin{tikzpicture}
  \draw (0,0) to node [sloped,above] {x} (3,2);

  \draw (0,0) to[out=90,in=180] node [sloped,above] {x} (3,2);
\end{tikzpicture}
\end{codeexample}

  \medskip
  \textbf{Styles for nodes.}\ \
  In addition to the \meta{options} given after the |to| operation,
  the following style is also set at the beginning of the to path:
  \begin{stylekey}{/tikz/every to (initially \normalfont empty)}
    This style is installed at the beginning of every to. By
    default, it is set to |draw|. 
\begin{codeexample}[]
\tikz[every to/.style={draw,dashed}]
  \path (0,0) to (3,2);
\end{codeexample}
  \end{stylekey}

  \medskip
  \textbf{Options.}\ \ 
  The \meta{options} given with the |to| allow you to influence the
  appearance of the |to path|. Mostly, these options are used to
  change the |to path|. This can be used to change the path from a 
  straight line to, say, a curve.

  The path used is set using the following option:
  \begin{key}{/tikz/to path=\meta{path}}
    Whenever an |to| operation is used, the \meta{path} is
    inserted. More precisely, the following path is added:

    \begin{quote}
      |[every to,|\meta{options}|] |\meta{path}
    \end{quote}
  
    The \meta{options} are the options given to the |to| operation,
    the \meta{path} is the path set by this option |to path|.

    Inside the \meta{path}, different macros are used to reference the
    from- and to-coordinates. In detail, these are:
    \begin{itemize}
    \item \declare{|\tikztostart|} will expand to the from-coordinate
      (without the parantheses).
    \item \declare{|\tikztotarget|} will expand to the to-coordinate.
    \item \declare{|\tikztonodes|} will expand to the nodes between
      the |to| operation and the coordinate. Furthermore, these
      nodes will have the |pos| option set implicitly.      
    \end{itemize}

    Let us have a look at a simple example. The standard straight line
    for an to is achieved by the following \meta{path}:
    \begin{quote}
      |-- (\tikztotarget) \tikztonodes|
    \end{quote}

    Indeed, this is the default setting for the path. When we write
    |(a) to (b)|, the \meta{path} will expand to |(a) -- (b)|, when
    we write
    \begin{quote}
      |(a) to[red] node {x} (b)|
    \end{quote}
    the \meta{path} will expand to
    \begin{quote}
      |(a) -- (b) node[pos] {x}|
    \end{quote}

    It is not possible to specify the path
    \begin{quote}
      |-- \tikztonodes (\tikztotarget)|
    \end{quote}
    since \tikzname\ does not allow one to have a macro after |--|
    that expands to a node.

    Now let us have a look at how we can modify the \meta{path}
    sensibly. The simplest way is to use a curve.
    
\begin{codeexample}[]
\begin{tikzpicture}[to path={
    .. controls +(1,0) and +(1,0) .. (\tikztotarget) \tikztonodes}]

  \node (a) at (0,0) {a};
  \node (b) at (2,1) {b};
  \node (c) at (1,2) {c};
                  
  \draw (a) to node {x} (b)
        (a) to          (c);
\end{tikzpicture}
\end{codeexample}

    Here is another example:

\begin{codeexample}[]
\tikzset{
  my loop/.style={->,to path={
    .. controls +(80:1) and +(100:1) .. (\tikztotarget) \tikztonodes}},
  my state/.style={circle,draw}}
                           
\begin{tikzpicture}[shorten >=2pt]
  \node [my state] (a) at (210:1) {$q_a$};
  \node [my state] (b) at (330:1) {$q_b$};
                  
  \draw (a) to           node[below]       {1} (b)
            to [my loop] node[above right] {0} (b);
\end{tikzpicture}
\end{codeexample}

    \begin{key}{/tikz/execute at begin to=\meta{code}}
      The \meta{code} is executed prior to the to. This can be used to
      draw one or more additional paths or to do additional
      computations.
    \end{key}

    \begin{key}{/tikz/executed at end to=\meta{code}}
      Works like the previous option, only this code is executed after
      the to path has been added.
    \end{key}


    \begin{stylekey}{/tikz/every to (initially \normalfont empty)}
      This style is installed at the beginning of every to.
    \end{stylekey}
  \end{key}
\end{pathoperation}

There are a number of predefined |to path|s, see
Section~\ref{library-to-paths} for a reference.


  

\subsection{The Scoping Operation}

When \tikzname\ encounters and opening or a closing brace (|{| or~|}|) at
some point where a path operation should come, it will open or close a
scope. All options that can be applied ``locally'' will be scoped
inside the scope. For example, if you apply a transformation like
|[xshift=1cm]| inside the scoped area, the shifting only applies to
the scope. On the other hand, an option like |color=red| does not have
any effect inside a scope since it can only be applied to the path as
a whole. 


\subsection{The Node Operation}

There are teo more operations that can be found in paths:
|node| and |edge|. The first is used to add a so-called node to a
path. This operation is special in the following sense: It does not 
change the current path in any way. In other words, this operation 
is not really a path operation, but has an effect that is
``external''  to the path. The |edge| operation has similar effect in
that it adds something \emph{after} the main parth has been
drawn. However, it works like the |to| operation, that is, it adds a
|to| path to the picture after the main path has been drawn.

Since these operations are quite complex, they are described in the
separate Section~\ref{section-nodes}.  



\subsection{The PGF-Extra Operation}

In some cases you may need to ``do some calculations or some other
stuff'' while a path is constructed. For this, you would like to
suspend the construction of the path and suspend \tikzname's parsing
of the path, you would then like to have some \TeX\ code executed, and
would then like to resume the parsing of the path. This effect can be
achieved using the following path operation |\pgfextra|. Note that
this operation should only be used by real experts and should only be
used deep inside clever macros, not on normal paths.

\begin{command}{\pgfextra\marg{code}}
  This command may only be used inside a \tikzname\ path. There it is
  used like a normal path operation. The construction of the path is
  temporarily suspended and the \meta{code} is executed. Then, the
  path construction is resumed.

\begin{codeexample}[]
\newdimen\mydim
\begin{tikzpicture}
  \mydim=1cm
  \draw (0pt,\mydim) \pgfextra{\mydim=2cm} -- (0pt,\mydim);
\end{tikzpicture}
\end{codeexample}
\end{command}

\begin{command}{\pgfextra \meta{code} \texttt{\char`\\endpgfextra}}
  This is an alternative syntax for the |\pgfextra| command. If the
  code following |\pgfextra| does not start with a brace, the
  \meta{code} is executed until |\endpgfextra| is encountered. What
  actually happens is that |\pgfextra| that is not followed by a brace
  completely shuts down the \tikzname\ parse and |\endpgfextra| is a
  normal macro that restarts the parser.

\begin{codeexample}[]
\newdimen\mydim
\begin{tikzpicture}
  \mydim=1cm
  \draw (0pt,\mydim)
    \pgfextra \mydim=2cm \endpgfextra -- (0pt,\mydim);
\end{tikzpicture}
\end{codeexample}
\end{command}

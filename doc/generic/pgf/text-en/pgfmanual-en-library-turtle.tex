% Copyright 2008 by Mark Wibrow
%
% This file may be distributed and/or modified
%
% 1. under the LaTeX Project Public License and/or
% 2. under the GNU Free Documentation License.
%
% See the file doc/generic/pgf/licenses/LICENSE for more details.

\section{Turtle Graphics Library}
\label{section-library-tutrle}


\begin{pgflibrary}{turtle}
  This little library defines some keys to create simple turtle
  graphics in the tradition of the Logo programming language. These
  commands are mostly for fun, but they can also be used for more
  ``serious'' business.
\end{pgflibrary}


All keys of the turtle graphics library start with |turtle|. Each such
key should be given inside a separate graphic option block as the
following example shows:

\begin{codeexample}[]
\tikz \draw
  [turtle home] [turtle forward] [turtle right] [turtle forward];
\end{codeexample}

These keys use a bit of magic (actually the |after option path|) to do 
the actual drawing. 

The basic drawing model behind the turtle graphics is very simple:
There is a (virtual) \emph{turtle} that crawls around the page,
thereby extending the path. The turtle always heads in a certain
direction. When you move the turtle forward you extend the path in
that direction; turning the turtle just changes the direction, it does
not cause anything to be drawn.

The turtle always moves relative to the last current point of the
path and you can mix normal path commands with turtle
commands. However, the direction of the turtle is managed
independently of other path commands.

\begin{key}{/tikz/turtle home}
  Places the turtle at the origin and lets it head upward. This key,
  like all other turtle keys, should be given as a graphic option on a
  path. Multiple turtle keys should be placed in separate gaphic
  option blocks.

  Note that the use of graphic option blocks does not really honour
  the spirit of graphic options since, in reality, this key is no
  graphic option. But it works nicely, so why bother?
\end{key}

\begin{key}{/tikz/turtle forward=\meta{distance} (default \normalfont see text)}
  Makes the turtle move forward by the given \meta{distance}. If no
  \meta{distance} is specified, the current value of the following key
  is used:
  \begin{key}{/tikz/turtle distance=\meta{distance} (initially 1cm)}
    The default distance by which the turtle advances.
  \end{key}
  ``Moving forward the turtle'' actually means that, relative to the
  current last point on the path, a point at the given \meta{distance}
  in the direction the turtle is currently heading is computed. Then,
  a |to| path operation is used to extend the path to this point with
  the to path set so that it draws a straight line (the |line to|
  option is used). The |to| path can be influenced by the following keys:
  \begin{stylekey}{/tikz/turtle to path options (initially \normalfont empty)}
    This style stores (additional) the options that are passed to the
    |to| path operation. By setting this style you can change the
    to-path: 
\begin{codeexample}[]
\tikz \draw [turtle home]
            [turtle forward,turtle to path options={bend left}]
            [turtle right]
            [turtle forward];
\end{codeexample}    
    This key will be reset by every turtle movement.
  \end{stylekey}
  \begin{key}{/tikz/how=\meta{keys}}
    This key is defined only after the |turtle forward| or
    |turtle backward| key has been used inside a graphic option
    block. Its effect is to set the |turtle to path options| style to
    the value of \meta{keys}. The effect will be local to one
    turtle movement:
\begin{codeexample}[]
\tikz \draw [turtle home]
            [turtle forward,how=bend left]
            [turtle right]
            [turtle forward,how=bend right]
            [turtle right]
            [turtle forward];
\end{codeexample}    
  \end{key}
  \begin{stylekey}{/tikz/every turtle movement (initially \normalfont empty)}
    This key is executed at the beginning of every turtle
    movement. You can use it, for instance, to set the turtle to path
    for all movements:
\begin{codeexample}[]
\tikz \draw [every turtle movement/.style={how=bend left}]
            [turtle home]
            [turtle forward]
            [turtle right]
            [turtle forward,how=move to]
            [turtle right]
            [turtle forward];
\end{codeexample}    
  \end{stylekey}
\end{key}

\begin{key}{/tikz/turtle back=\meta{distance} (default \normalfont see text)}
  This has the same effect as a |turtle forward| for the negated
  \meta{distance} value.
\end{key}

\begin{key}{/tikz/turlte left=\meta{angle} (default 90)}
  Turns the turtle left by the given angle. 
\end{key}

\begin{key}{/tikz/turlte right=\meta{angle} (default 90)}
  Turns the turtle right by the given angle. 
\end{key}

Turtle graphics are especially nice in conjunction with the |\foreach|
statement:

\begin{codeexample}[]
\tikz \filldraw [thick,blue,fill=blue!20]
  [turtle home]
  \foreach \i in {1,...,5}
  {
    [turtle forward]
    [turtle right=144]
  };
\end{codeexample}

\endinput




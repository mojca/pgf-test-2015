% Copyright 2012 by Till Tantau
%
% This file may be distributed and/or modified
%
% 1. under the LaTeX Project Public License and/or
% 2. under the GNU Free Documentation License.
%
% See the file doc/generic/pgf/licenses/LICENSE for more details.

\section{Graph Drawing Algorithms: Layered Layouts}

{\emph{by  Till Tantau and Jannis Pohlmann}}

\begin{tikzlibrary}{graphdrawing.layered}
  This library provides keys for drawing graphs using the Sugiyama
  method, which is especially useful for drawing hierachical graphs.
  You should load the |graphdrawing| library first.
\end{tikzlibrary}



\subsection{Overview}

A ``layered'' layout of a graph tries to arrange the nodes in
consecutive horizontal layers (naturally, by rotating the graph, this
can be changed in to vertical layers) such that edges tend to be only
between nodes on adjacent layers. Trees, for instance, can always be
laid out in this way.




http://www.tcs.uni-luebeck.de/downloads/papers/2011/2011-configurable-graph-drawing-algorithms-jannis-pohlmann.pdf



\begin{gdalgorithm}{layered layout}{modular layered}
  A modular version of the popular Sugiyama method...
\begin{codeexample}[]
\tikz [rounded corners] \graph [layered layout] { a -> {b,c -> {d,e,f} } -> h -> a };    
\end{codeexample}
\end{gdalgorithm}



%%% Local Variables: 
%%% mode: latex
%%% TeX-master: "pgfmanual-pdftex-version"
%%% End: 

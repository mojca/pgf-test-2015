% Copyright 2007 by Mark Wibrow
%
% This file may be distributed and/or modified
%
% 1. under the LaTeX Project Public License and/or
% 2. under the GNU Free Documentation License.
%
% See the file doc/generic/pgf/licenses/LICENSE for more details.
%

\part{Mathetical Operations}
\pgfname{} supports mathematical operations beyond simple addition,
subtraction, multiplication and division, using both integers and 
non-integers. In addition, \pgfname{} can calculate functions such
as square-roots, sine, cosine, and generate pseudo-random numbers, 
both as ``stand-alone'' macros, and as infix operations in the 
arguments of certain commands.
\newpage

\section{Mathematics in PGF}
Those familiar with the \calcname{} package, know that it is 
possible to parse basic infix mathematical arithmatic operations, 
in the arguments of certain macros. 
However, \calcname{} operates under fairly constrained input. 
For example, only the |+|, |-|, |*| and |/| operators are supported, 
all non-integers must be followed by a \TeX{} dimension, and the 
second operand in a multiplication or division operation can only be
 an integer. 

\pgfname{} provides enhanced functionality, which permits the parsing
of mathematical operations involving integers and non-integers 
with or without units. Futhermore, various functions, including
trigonometic functions and random number generators can also be 
parsed (see Section~\ref{pgfmath-parsing}). 
The \calcname{} macros |\setlength| and friends have \pgfname{} versions 
which can parse these operations and functions 
(see Section~\ref{pgfmath-registers}). Additionally, each operation
and function has an independent \pgfname{} command associated with it
(see Section~\ref{pgfmath-commands}), and can be 
accessed outside the parser.

Currently, the mathematical algorithms are all implemented in \TeX.
This poses some intriguing programming challenges as \TeX{} is a
typesetting language not a mathematical one, and as with any 
programming language, there is a trade-off between accuracy and 
efficiency. Some levels of accuracy may not be your liking. 
However, it is very simple to replace them. In fact, the interface to 
the mathematical operations and functions has been deliberately 
implemented in a transparently enacpsualted manner, for precisely this 
purpose. For more details see Section~\ref{pgfmath-reimplement}.

\subsection{Setting Registers}

\label{pgfmath-registers}

These macros are very similar to their cousins in the \calcname{} 
package. The only difference is that the expressions that than can be
used to determine the values assigned to the registers, can be more
complex mathematical expressions, including functions and 
non-integers, as described in Section~\ref{pgfmath-parsing}. 

\begin{command}{\pgfmathsetlength\marg{dimension register}\marg{expression}}
Sets the length of the \TeX{} \meta{dimension register}, to the value
(in points) specified by \meta{expression}. 
Section~\ref{pgfmath-parsing}, provides details of syntax for
\meta{expression}.
\end{command}

\begin{command}{\pgfmathaddtolength\marg{dimension register}\marg{expression}}
	Adds the value (in points) of \meta{expression} to the \TeX{} 
	\meta{dimension register}.
\end{command}

\begin{command}{\pgfmathsetcount\marg{count register}\marg{expression}}
	Sets the value of the \TeX{} \meta{count register}, to the 
	\emph{truncated} value specified by \meta{expression}. 
\end{command}

\begin{command}{\pgfmathaddtocount\marg{count register}\marg{expression}}
	Adds the \emph{truncated} value  of \meta{expression} to the \TeX{} 
\	meta{count register}.
\end{command}

\begin{command}{\pgfmathsetcounter\marg{counter}\marg{expression}}
	Sets the value of the \meta{counter}, to the \emph{truncated} value 
	specified by \meta{expression}. 
\end{command}

\begin{command}{\pgfmathaddtocounter\marg{counter}\marg{expression}}
	Adds the \emph{truncated} value  of \meta{expression} to 
	\meta{counter}.
\end{command}

\begin{command}{\pgfmathnewcounter\marg{counter}}
	This is simply a version of the \LaTeX{} macro |\newcounter|, 
	implemented	entirely to maintain consistency (consistency is good,
	inconsistency is evil). Considering |\pgfmathnewcounter{foo}|, this
	creates a new count register |\c@foo|, and a macro |\thefoo|, which
	returns the value in |\c@foo|.
\end{command}

\begin{command}{\pgfmathsetmacro\marg{macro}\marg{expression}}
	Defines \meta{macro} as the  value of \meta{expression}. The result
	is a decimal \emph{without} units.
\end{command}

All the fancy mathematical ``bells-and-whistles'' that the parser 
provides, come with an additional processing cost, and in some
instances, such as simply setting a length to |1cm|, with no other
operations involved, the addition processing time is undesirable. 
To overcome this, the following feature is implemented: when no
mathematical operations are required, the value in \meta{expression}
can be preceded by |+|. This will bypass the parsing process and the 
assignment will be orders of magnitude faster. This feature 
\emph{only} works with the macros for setting registers described in
this section.

\begin{codeexample}[code only]
\pgfmathsetlength\mydimen{1cm}  % parsed     : slower.
\pgfmathsetlength\mydimen{+1cm} % not parsed : much faster.
\end{codeexample}

% Copyright 2007 by Mark Wibrow
%
% This file may be distributed and/or modified
%
% 1. under the LaTeX Project Public License and/or
% 2. under the GNU Free Documentation License.
%
% See the file doc/generic/pgf/licenses/LICENSE for more details.
%

\section{Mathematical Expressions}

\label{pgfmath-syntax}

The easiest way of using \pgfname's mathematical engine is to provide
a mathematical expression given in familiar infix notation, for
example, |1cm+4*2cm/5.5| or |2*3+3*sin(30)|. This expression can be
parsed by the mathematical engine and the result can be placed in a
dimension register, a counter, or a macro.

It should be noted that all
calculations must not exceed $\pm16383.99999$ at \emph{any} point,
because the underlying computations rely on \TeX{} dimensions. This
means that many of the underlying computations are necessarily
approximate and, in addition, not very fast. \TeX{} is,
after all, a typesetting language and not ideally
suited to relatively advanced mathematical operations. However, it
is possible to change the computations as described in
Section~\ref{pgfmath-reimplement}.

In the present section, the high-level macros for parsing an
expression are explained first, then the syntax for expression is
explained.


\subsection{Parsing Expressions}

\label{pgfmath-registers}

\label{pgfmath-parsing}

\subsubsection{Commands}

The basic command for invoking the parser of \pgfname's mathematical
engine is the following:

\begin{command}{\pgfmathparse\marg{expression}}
  This macro parses \meta{expression} and returns the result without
  units in  the macro |\pgfmathresult|.

  \example |\pgfmathparse{2pt+3.5pt}| will set |\pgfmathresult| to the
  text |5.5|.

  In the following, the special properties of this command are
  explained. The exact syntax of mathematical expressions is explained
  in Sections \ref{pgfmath-operators} and~\ref{pgfmath-functions}.

  \begin{itemize}
  \item
    The result stored in the macro |\pgfmathresult| is a decimal
    \emph{without units}. This is true regardless of whether the
    \meta{expression} contains any unit specification. All numbers
    with units are converted to points first. See
    Section~\ref{pgfmath-units} for details on units.
  \item
    The parser will recognize \TeX{} registers and box dimensions,
    so |\mydimen|, |0.5\mydimen|, |\wd\mybox|, |0.5\dp\mybox|,
    |\mycount\mydimen| and so on can be parsed.
    
  \item
    The $\varepsilon$-TeX\ extensions |\dimexpr|, |\numexpr|, |\glueexpr|, and
    |\muexpr| are recognized and evaluated. The values they
    result in will be used in the further evaluation, as if you had
    put |\the| before them. 
    
  \item
    Parenthesis can be used to change the order of the evaluation.
    
  \item
    Various functions are recognized, so it is possible to parse
    |sin(.5*pi r)*60|, which means ``the sine of $0.5$ times $\pi$
    radians, multiplied by 60''. The argument of functions can
    be any expression.
    
  \item
    Scientific notation in the form |1.234e+4| is recognized (but
    the restriction on the range of values still applies). The exponent
    symbol can be upper or lower case (i.e., |E| or |e|).
    
  \item
    An integer with a zero-prefix (excluding, of course zero itself),
    is interpreted as an octal number and is automatically converted
    to base 10.
    
  \item
    An integer with prefix |0x| or |0X| is interpreted as a hexadecimal
    number and is automatically converted to base 10. Alphabetic digits
    can be in uppercase or lowercase.
    
  \item
    An integer with prefix |0b| or |0B| is interpreted as a binary
    number and is automatically converted to base 10.
    
  \item
    An expression (or part of an expression) surrounded with double
    quotes (i.e., the character |"|) will not be evaluated.
    Obviously this should be used with great care.
    
  \end{itemize}

\end{command}



\begin{command}{\pgfmathqparse\marg{expression}}
  This macro is similar to |\pgfmathparse|: it parses
  \meta{expression} and returns the result in the macro
  |\pgfmathresult|. It differs in two respects. Firstly,
  |\pgfmathqparse| does not parse functions, scientific
  notation, the prefixes for binary octal, or hexadecimal numbers,
  nor does it accept the special use of |"|, |?| or |:| characters.
  Secondly, numbers in \meta{expression} \emph{must}
  specify a \TeX{} unit (except in such instances as |0.5\pgf@x|),
  which greatly simplifies the problem of parsing real numbers.
  As a result of these restrictions |\pgfmathqparse|
  is about twice as fast as |\pgfmathparse|. Note that the result
  will still be a number without units.	
\end{command}

\begin{command}{\pgfmathpostparse}

  At the end of the parse this command is executed, allowing some
  custom action to be performed on the result of the parse. When this
  command is executed, the macro |\pgfmathresult| will hold the result
  of the parse (as always, without units). The result of the custom
  action should be used to redefine |\pgfmathresult| appropriately.
  By default, this command is equivalent to |\relax|. This differs
  from previous versions, where, if the parsed expression contained
  no units, the result of the parse was scaled according to the value
  in |\pgfmathresultunitscale| (which by default was |1|).

  This scaling can be  turned on again using:
  |\let\pgfmathpostparse=\pgfmathscaleresult|.
  Note, however that by scaling the result, the base conversion
  functions will not work, and the |"| character should not be
  used to quote parts of an expression.

\end{command}

Instead of the |\pgfmathparse| macro you can also use wrapper commands,
whose usage is very similar to their cousins in the \calcname{}
package. The only difference is that the expressions can be any
expression that is handled by |\pgfmathparse|.
For all of the following commands, if \meta{expression} starts with
|+|, no parsing is done and a simple assignment or increment is done
using normal \TeX\ assignments or increments. This will be orders of
magnitude faster than calling the parser.

The effect of the following commands is always local to the current
\TeX\ scope.

\begin{command}{\pgfmathsetlength\marg{register}\marg{expression}}
  Basically, this command sets the length of the \TeX{}
  \meta{register} to the value specified by
  \meta{expression}. However, there is some fine print:

  First, in case \meta{expression} starts with a |+|, a simple \TeX\
  assignment is done. In particular, \meta{register} can be a glue
  register and \meta{expression} be something like |+1pt plus 1fil|
  and the \meta{register} will be assigned the expected value.

  Second, when the \meta{expression} does not start with |+|, it is
  first parsed using |\pgfmathparse|, resulting in a (dimensionless)
  value |\pgfmathresult|. Now, if the parser encountered the unit |mu|
  somewhere in the expression, it assumes that \meta{register} is a
  |\muskip| register and will try to assign to \meta{register} the
  value |\pgfmathresult| followed by |mu|. Otherwise, in case |mu| was
  not encountered, it is assumed that \meta{register} is a dimension
  register or a glue register and we assign |\pgfmathresult| followed
  by |pt| to it.

  The net effect of the above is that you can write things like
\begin{codeexample}[]
  \muskipdef\mymuskip=0
  \pgfmathsetlength{\mymuskip}{1mu+3*4mu} \the\mymuskip 
\end{codeexample}
\begin{codeexample}[]  
  \dimendef\mydimen=0  
  \pgfmathsetlength{\mydimen}{1pt+3*4pt}  \the\mydimen
\end{codeexample}
\begin{codeexample}[]  
  \skipdef\myskip=0  
  \pgfmathsetlength{\myskip}{1pt+3*4pt}  \the\myskip
\end{codeexample}

  One thing that will \emph{not} work is
  |\pgfmathsetlength{\myskip}{1pt plus 1fil}| since the parser does
  not support fill's. You can, however, use the |+| notation in this
  case: 
\begin{codeexample}[]  
  \skipdef\myskip=0  
  \pgfmathsetlength{\myskip}{+1pt plus 1fil}  \the\myskip
\end{codeexample}
\end{command}

\begin{command}{\pgfmathaddtolength\marg{register}\marg{expression}}
  Adds the value of \meta{expression} to the \TeX{}
  \meta{register}. All of the special consideration mentioned for
  |\pgfmathsetlength| also apply here in the same way.
\end{command}

\begin{command}{\pgfmathsetcount\marg{count register}\marg{expression}}
  Sets the value of the \TeX{} \meta{count register}, to the
  \emph{truncated} value specified by \meta{expression}.
\end{command}

\begin{command}{\pgfmathaddtocount\marg{count register}\marg{expression}}
  Adds the \emph{truncated} value  of \meta{expression} to the \TeX{}
  \meta{count register}.
\end{command}

\begin{command}{\pgfmathsetcounter\marg{counter}\marg{expression}}
  Sets the value of the \LaTeX{} \meta{counter} to the \emph{truncated}
  value specified by \meta{expression}.
\end{command}

\begin{command}{\pgfmathaddtocounter\marg{counter}\marg{expression}}
  Adds the \emph{truncated} value  of \meta{expression} to
  \meta{counter}.
\end{command}

\begin{command}{\pgfmathsetmacro\marg{macro}\marg{expression}}
  Defines \meta{macro} as the  value of \meta{expression}. The result
  is a decimal without units.
\end{command}

\begin{command}{\pgfmathsetlengthmacro\marg{macro}\marg{expression}}
  Defines \meta{macro} as the value of \meta{expression}
  \LaTeX{} \emph{in points}.
\end{command}

\begin{command}{\pgfmathtruncatemacro\marg{macro}\marg{expression}}
  Defines \meta{macro} as the truncated value of \meta{expression}.
\end{command}


\subsubsection{Considerations Concerning Units}
\label{pgfmath-units}

As was explained earlier, the parser commands like |\pgfmathparse|
will always return a result without units in it and all dimensions
that have a unit like |10pt| or |1in| will first be converted to \TeX\
points (|pt|) and, then, the unit is dropped.

Sometimes it is useful, nevertheless, to find out whether an
expression or not. For this, you can use the following commands:

{\let\ifpgfmathunitsdeclared\relax
  \begin{command}{\ifpgfmathunitsdeclared}
    After a call  of |\pgfmathparse| this if will be true exactly if
    some unit was encountered in the expression. It is always set
    globally in each call.
    
    Note that \emph{any} ``mentioning'' of a unit inside an
    expression will set this \TeX-if to true. In particular, even an
    expressionlike |2pt/1pt|, which arguably should be considered
    ``scalar'' or ``unit-free'' will still have this \TeX-if set to
    true. However, see the |scalar| function for a way to change
    this. 
  \end{command}
}

\begin{math-function}{scalar(\mvar{value})}
  \mathcommand
  
  This function is the identity function on its input, but it will
  reset the \TeX-if |\ifpgfmathunitsdeclared|. Thus, it can be used to
  indicate that the given \meta{value} should be considered as a
  ``scalar'' even when it contains units; but note that it will work
  even when the \meta{value} is a string or something else. The only
  effect of this function is to clear the unit declaration.

\begin{codeexample}[]
\pgfmathparse{scalar(1pt/2pt)} \pgfmathresult\
\ifpgfmathunitsdeclared with \else without \fi unit
\end{codeexample}

  Note, however, that this command (currently) really just clears the
  \TeX-if as the input is scanned from left-to-right. Thus, even if
  there is a use of a unit before the |scalar| function is used, the
  \TeX-if will be cleared:

\begin{codeexample}[]
\pgfmathparse{1pt+scalar(1pt)} \pgfmathresult\
\ifpgfmathunitsdeclared with \else without \fi unit
\end{codeexample}

  The other way round, a use of a unit after the |scalar| function
  will set the units once more.
\begin{codeexample}[]
\pgfmathparse{scalar(1pt)+1pt} \pgfmathresult\
\ifpgfmathunitsdeclared with \else without \fi unit
\end{codeexample}

  For these reasons, you should use the function only on the outermost
  level of an expression.
  
  A typical use of this function is the following:
\begin{codeexample}[]  
\tikz{
  \coordinate["$A$"]       (A) at (2,2);%$
  \coordinate["$B$" below] (B) at (0,0);
  \coordinate["$C$" below] (C) at (3,0);
  \draw (A) -- (B) -- (C) -- cycle;
  \path
    let \p1 =($(A)-(B)$), \p2 =($(A)-(C)$),
        \n1 = {veclen(\x1,\y1)}, \n2 = {veclen(\x2,\y2)}
    in  coordinate ["$D$" below] (D) at ($ (B)!scalar(\n1/(\n1+\n2))!(C) $);
  \draw (A) -- (D);
}
\end{codeexample}
\end{math-function}

A special kind of units are \TeX's ``math units'' (|mu|). It will
be treated as if |pt| had been used, but you can
check whether an expression contained a math unit using the
following: 
{\let\ifpgfmathmathunitsdeclared\relax
  \begin{command}{\ifpgfmathmathunitsdeclared}
    This \TeX-if is similar to |\ifpgfmathunitsdeclared|, but it
    is only set when the unit |mu| is encountered at least
    once. In this case, |\ifpgfmathunitsdeclared| will \emph{also}
    be set to true. The |scalar| function has no effect on this \TeX-if.
  \end{command}
}
  
\subsection{Syntax for Mathematical Expressions: Operators}

The syntax for the expressions recognized by |\pgfmathparse| and
friends is rather straightforward. Let us start with the operators.

\label{pgfmath-operators}

The following operators (presented in the context in which they are used)
are recognized:

\begin{math-operator}{+}{infix}{add}
 Adds \mvar{x} to \mvar{y}.
\end{math-operator}

\begin{math-operator}{-}{infix}{subtract}
  Subtracts \mvar{y} from \mvar{x}.
\end{math-operator}

\begin{math-operator}{-}{prefix}{neg}
  Reverses the sign of \mvar{x}.
\end{math-operator}

\begin{math-operator}{*}{infix}{multiply}
  Multiples \mvar{x} by \mvar{y}.
\end{math-operator}

\begin{math-operator}{/}{infix}{divide}
  Divides \mvar{x} by \mvar{y}. An error will result if \mvar{y} is 0,
  or if the result of the division is too big for the mathematical
  engine. Please remember when using this command that accurate (and
  reasonably quick) division of real numbers that are not integers
  is particularly tricky in \TeX.
\end{math-operator}

\begin{math-operator}{\char`\^}{infix}{pow}
  Raises \mvar{x} to the power \mvar{y}.
\end{math-operator}

\begin{math-operator}{\protect\exclamationmarktext}{postfix}{factorial}
  Calculates the factorial of \mvar{x}.
\end{math-operator}

\begin{math-operator}{r}{postfix}{deg}
  Converts \mvar{x} to degrees (\mvar{x} is assumed to be in radians).
  This operator has the same precedence as multiplication.
\end{math-operator}

\begin{math-operators}{?}{:}{conditional}{ifthenelse}

  |?| and |:| are special operators which can be used as a shorthand
  for |if| \mvar{x} |then| \mvar{y} |else| \mvar{z} inside the parser.
  The expression \mvar{x} is taken to be true if it evaluates to any
  non-zero value.

\end{math-operators}

\begin{math-operator}{==}{infix}{equal}
  Returns |1| if \mvar{x}$=$\mvar{y}, |0| otherwise.
\end{math-operator}

\begin{math-operator}{>}{infix}{greater}
  Returns |1| if \mvar{x}$>$\mvar{y}, |0| otherwise.
\end{math-operator}

\begin{math-operator}{<}{infix}{less}
  Returns |1| if \mvar{x}$<$\mvar{y}, |0| otherwise.
\end{math-operator}

\begin{math-operator}{\protect\exclamationmarktext=}{infix}{notequal}
  Returns |1| if \mvar{x}$\neq$\mvar{y}, |0| otherwise.
\end{math-operator}

\begin{math-operator}{>=}{infix}{notless}
  Returns |1| if \mvar{x}$\geq$\mvar{y}, |0| otherwise.
\end{math-operator}

\begin{math-operator}{<=}{infix}{notgreater}
  Returns |1| if \mvar{x}$\leq$\mvar{y}, |0| otherwise.
\end{math-operator}

\begin{math-operator}{{\char`\&}{\char`\&}}{infix}{and}
  Returns |1| if both \mvar{x} and \mvar{y} evaluate to some
  non-zero value. Both arguments are evaluated.
\end{math-operator}



{
 \catcode`\|=12
\begin{math-operator}[no index]{||}{infix}{or}
	\index{*pgfmanualvbarvbarr@\protect\texttt{\protect\pgfmanualvbarvbar} math operator}%
  \index{Math operators!*pgfmanualvbarvbar@\protect\texttt{\protect\pgfmanualvbarvbar}}%
  Returns {\tt 1} if either \mvar{x} or \mvar{y} evaluate to some
  non-zero value.
\end{math-operator}
}

\begin{math-operator}{\protect\exclamationmarktext}{prefix}{not}
  Returns |1| if \mvar{x} evaluates to zero, |0| otherwise.
\end{math-operator}


\begin{math-operators}{(}{)}{group}{}

These operators act in the usual way, that is, to control the order
in which operators are executed, for example, |(1+2)*3|. This
includes the grouping of arguments for functions, for example,
|sin(30*10)| or |mod(72,3)| (the comma character is also treated
as an operator).

Parentheses for functions with one argument are not always
necessary, |sin 30| (note the space) is the same as |sin(30)|.
However, functions have the highest precedence so, |sin 30*10|
is the same as |sin(30)*10|.

\end{math-operators}


\begin{math-operators}{\char`\{}{\char`\}}{array}{}

These operators are used to process array-like structures (within an
expression these characters do not act like \TeX{} grouping tokens).
The \meta{array specification} consists of comma separated elements,
for example, |{1, 2, 3, 4, 5}|. Each element in the array will be
evaluated as it is parsed, so expressions can be used.
In addition, an element of an array can be an array itself,
allowing multiple dimension arrays to be simulated:
|{1, {2,3}, {4,5}, 6}|.
When storing an array in a macro, do not forget the surrounding
braces: |\def\myarray{{1,2,3}}| not |\def\myarray{1,2,3}|.

\begin{codeexample}[]
\def\myarray{{1,"two",2+1,"IV","cinq","sechs",sin(\i*5)*14}}
\foreach \i in  {0,...,6}{\pgfmathparse{\myarray[\i]}\pgfmathresult, }
\end{codeexample}

\end{math-operators}

\begin{math-operators}{\char`\[}{\char`\]}{array access}{array}

|[| and |]| are two operators used in one particular circumstance: to
access an array (specified using the |{| and |}| operators)
using the index \mvar{x}. Indexing starts from zero,
so, if the index is greater than, or equal to, the number of values in
the array, an error will occur, and zero will be returned.

\begin{codeexample}[]
\def\myarray{{7,-3,4,-9,11}}
\pgfmathparse{\myarray[3]} \pgfmathresult
\end{codeexample}

If the array is defined to have multiple dimensions, then the array
access operators can be immediately repeated.

\begin{codeexample}[]
\def\print#1{\pgfmathparse{#1}\pgfmathresult}
\def\identitymatrix{{{1,0,0},{0,1,0},{0,0,1}}}
\tikz[x=0.5cm,y=0.5cm]\foreach \i in {0,1,2} \foreach \j in {0,1,2}
  \node at (\j,-\i) [anchor=base] {\print{\identitymatrix[\i][\j]}};
\end{codeexample}

\end{math-operators}


\begin{math-operators}{\char`\"}{\char`\"}{group}{}

These operators are used to quote \mvar{x}. However, as every
expression is expanded with |\edef| before it is parsed, macros
(e.g., font commands like |\tt| or |\Huge|) may need to be
``protected'' from this expansion (e.g., |\noexpand\Huge|). Ideally,
you should avoid such macros anyway.
Obviously, these operators should be used with great care as further
calculations are unlikely to be possible with the result.

\begin{codeexample}[]
\def\x{5}
\foreach \y in {0,10}{
  \pgfmathparse{\x > \y ? "\noexpand\Large Bigger" : "\noexpand\tiny smaller"}
  \x\ is \pgfmathresult\ than \y.
}
\end{codeexample}

\end{math-operators}




\subsection{Syntax for Mathematical Expressions: Functions}

\label{pgfmath-functions}

The following functions are recognized:

\medskip
\def\mathlink#1{\hyperlink{math:#1}{\tt#1}}
\begin{tikzpicture}
\foreach \f [count=\i from 0] in
{abs,acos,add,and,array,asin,atan,atan2,bin,ceil,cos,
 cosec,cosh,cot,deg,depth,div,divide,e,equal,factorial, false,
 floor,frac,gcd,greater,height,hex,Hex,int,ifthenelse,iseven,isodd,isprime,
 less,ln,log10,log2,max,min,mod,Mod,multiply,
 neg,not,notequal,notgreater,notless,
 oct,or,pi,pow,rad,rand,random,real,rnd,round,
 scalar,sec,sin,sinh,sqrt,subtract,tan,tanh,true, veclen,width}
\node [anchor=base west] at ({int(\i/12)*2.5cm},{-mod(\i,12)*1.1*\baselineskip}) {\mathlink{\f}};
\end{tikzpicture}
\bigskip

Each function has a \pgfname{} command associated with it (which is
also shown with the function below). In general, the command
is simply the name of the function prefixed with |\pgfmath|, for
example, |\pgfmathadd|, but there are some notable exceptions.

\subsubsection{Basic arithmetic functions}

\label{pgfmath-functions-basic}

\begin{math-function}{add(\mvar{x},\mvar{y})}
\mathcommand

  Adds $x$ and $y$.

\begin{codeexample}[]
\pgfmathparse{add(75,6)} \pgfmathresult
\end{codeexample}
\end{math-function}

\begin{math-function}{subtract(\mvar{x},\mvar{y})}
\mathcommand

  Subtract $x$ from $y$.

\begin{codeexample}[]
\pgfmathparse{subtract(75,6)} \pgfmathresult
\end{codeexample}
\end{math-function}

\begin{math-function}{neg(\mvar{x})}
\mathcommand

	This returns $-\mvar{x}$.
	
\begin{codeexample}[]
\pgfmathparse{neg(50)} \pgfmathresult
\end{codeexample}

\end{math-function}

\begin{math-function}{multiply(\mvar{x},\mvar{y})}
\mathcommand

  Multiply $x$ by $y$.

\begin{codeexample}[]
\pgfmathparse{multiply(75,6)} \pgfmathresult
\end{codeexample}
\end{math-function}

\begin{math-function}{divide(\mvar{x},\mvar{y})}
\mathcommand

  Divide $x$ by $y$.

\begin{codeexample}[]
\pgfmathparse{divide(75,6)} \pgfmathresult
\end{codeexample}
\end{math-function}

\begin{math-function}{div(\mvar{x},\mvar{y})}
\mathcommand

  Divide $x$ by $y$ and round to the nearest integer

\begin{codeexample}[]
\pgfmathparse{div(75,9)} \pgfmathresult
\end{codeexample}
\end{math-function}

\begin{math-function}{factorial(\mvar{x})}
\mathcommand

  Return \mvar{x}!.

\begin{codeexample}[]
\pgfmathparse{factorial(5)} \pgfmathresult
\end{codeexample}

\end{math-function}

\begin{math-function}{sqrt(\mvar{x})}
\mathcommand

 Calculates $\sqrt{\textrm{\mvar{x}}}$.

\begin{codeexample}[]
\pgfmathparse{sqrt(10)} \pgfmathresult
\end{codeexample}

\begin{codeexample}[]
\pgfmathparse{sqrt(8765.432)}  \pgfmathresult
\end{codeexample}

\end{math-function}

\begin{math-function}{pow(\mvar{x},\mvar{y})}
\mathcommand

 Raises \mvar{x} to the power \mvar{y}. For greatest accuracy,
 \mvar{y} should be an integer. If \mvar{y} is not an integer,
 the actual calculation will be an approximation of $e^{y\text{ln}(x)}$.

\begin{codeexample}[]
\pgfmathparse{pow(2,7)} \pgfmathresult
\end{codeexample}

\end{math-function}


\begin{math-function}{e}
\mathcommand

  Returns the value 2.718281828.
{
\catcode`\^=7

\begin{codeexample}[]
\pgfmathparse{(e^2-e^-2)/2} \pgfmathresult
\end{codeexample}

}
\end{math-function}

\begin{math-function}{exp(\mvar{x})}
\mathcommand

{
\catcode`\^=7

	Maclaurin series for $e^x$.
}	
\begin{codeexample}[]
\pgfmathparse{exp(1)} \pgfmathresult
\end{codeexample}

\begin{codeexample}[]
\pgfmathparse{exp(2.34)} \pgfmathresult
\end{codeexample}

\end{math-function}


\begin{math-function}{ln(\mvar{x})}
\mathcommand

{
\catcode`\^=7

	An approximation for $\ln(\textrm{\mvar{x}})$.
	This uses an algorithm of Rouben Rostamian, and coefficients
	suggested by Alain Matthes.
}	
\begin{codeexample}[]
\pgfmathparse{ln(10)} \pgfmathresult
\end{codeexample}

\begin{codeexample}[]
\pgfmathparse{ln(exp(5))} \pgfmathresult
\end{codeexample}

\end{math-function}

\begin{math-function}{log10(\mvar{x})}
\mathcommand[logten(\mvar{x})]

	An approximation for $\log_{10}(\textrm{\mvar{x}})$.

\begin{codeexample}[]
\pgfmathparse{log10(100)} \pgfmathresult
\end{codeexample}

\end{math-function}

\begin{math-function}{log2(\mvar{x})}
\mathcommand[logtwo(\mvar{x})]

	An approximation for $\log_2(\textrm{\mvar{x}})$.

\begin{codeexample}[]
\pgfmathparse{log2(128)} \pgfmathresult
\end{codeexample}

\end{math-function}

\begin{math-function}{abs(\mvar{x})}
\mathcommand

	Evaluates the absolute value of $x$.
	
\begin{codeexample}[]
\pgfmathparse{abs(-5)} \pgfmathresult
\end{codeexample}

\begin{codeexample}[]
\pgfmathparse{-abs(4*-3)} \pgfmathresult
\end{codeexample}
\end{math-function}

\begin{math-function}{mod(\mvar{x},\mvar{y})}
\mathcommand

	This evaluates \mvar{x} modulo \mvar{y}, using truncated division.
	The sign of the result is the same as the sign of
	$\frac{\textrm{\mvar{x}}}{\textrm{\mvar{y}}}$.

\begin{codeexample}[]
\pgfmathparse{mod(20,6)} \pgfmathresult
\end{codeexample}

\begin{codeexample}[]
\pgfmathparse{mod(-100,30)} \pgfmathresult
\end{codeexample}

\end{math-function}

\begin{math-function}{Mod(\mvar{x},\mvar{y})}
\mathcommand

	This evaluates \mvar{x} modulo \mvar{y}, using floored division.
	The sign of the result is never negative.

\begin{codeexample}[]
\pgfmathparse{Mod(-100,30)} \pgfmathresult
\end{codeexample}

\end{math-function}





\subsubsection{Rounding functions}

\label{pgfmath-functions-rounding}

\begin{math-function}{round(\mvar{x})}
\mathcommand

	Rounds \mvar{x} to the nearest integer. It uses ``asymmetric half-up''
	rounding. So |1.5| is rounded to |2|, but |-1.5| is rounded to |-2|
	(\emph{not} |-1|).

\begin{codeexample}[]
\pgfmathparse{round(32.5/17)} \pgfmathresult
\end{codeexample}

\begin{codeexample}[]
\pgfmathparse{round(398/12)} \pgfmathresult
\end{codeexample}

\end{math-function}

\begin{math-function}{floor(\mvar{x})}
\mathcommand

	Rounds \mvar{x} down to the nearest integer.
	
\begin{codeexample}[]
\pgfmathparse{floor(32.5/17)} \pgfmathresult
\end{codeexample}

\begin{codeexample}[]
\pgfmathparse{floor(398/12)} \pgfmathresult
\end{codeexample}

\begin{codeexample}[]
\pgfmathparse{floor(-398/12)} \pgfmathresult
\end{codeexample}
\end{math-function}

\begin{math-function}{ceil(\mvar{x})}
\mathcommand

	Rounds \mvar{x} up to the nearest integer.

\begin{codeexample}[]
\pgfmathparse{ceil(32.5/17)} \pgfmathresult
\end{codeexample}

\begin{codeexample}[]
\pgfmathparse{ceil(398/12)} \pgfmathresult
\end{codeexample}

\begin{codeexample}[]
\pgfmathparse{ceil(-398/12)} \pgfmathresult
\end{codeexample}

\end{math-function}

\begin{math-function}{int(\mvar{x})}
\mathcommand

	Returns the integer part of \mvar{x}.

\begin{codeexample}[]
\pgfmathparse{int(32.5/17)} \pgfmathresult
\end{codeexample}

\end{math-function}

\begin{math-function}{frac(\mvar{x})}
\mathcommand

	Returns the fractional part of \mvar{x}.

\begin{codeexample}[]
\pgfmathparse{frac(32.5/17)} \pgfmathresult
\end{codeexample}

\end{math-function}

\begin{math-function}{real(\mvar{x})}
\mathcommand

	Ensures \mvar{x} contains a decimal point.

\begin{codeexample}[]
\pgfmathparse{real(4)} \pgfmathresult
\end{codeexample}

\end{math-function}


\subsubsection{Integer arithmetics functions}

\label{pgfmath-functions-integerarithmetics}

\begin{math-function}{gcd(\mvar{x},\mvar{y})}
\mathcommand

  Computes the greatest common divider of \mvar{x} and \mvar{y}. 

\begin{codeexample}[]
\pgfmathparse{gcd(42,56)} \pgfmathresult
\end{codeexample}

\end{math-function}

\begin{math-function}{isodd(\mvar{x})}
\mathcommand

  Returns |1| if the integer part of \mvar{x} is odd. Otherwise, returns |0|.

\begin{codeexample}[]
\pgfmathparse{isodd(2)} \pgfmathresult, 
\pgfmathparse{isodd(3)} \pgfmathresult
\end{codeexample}

\end{math-function}

\begin{math-function}{iseven(\mvar{x})}
\mathcommand

  Returns |1| if the integer part of \mvar{x} is even. Otherwise, returns |0|.

\begin{codeexample}[]
\pgfmathparse{iseven(2)} \pgfmathresult, 
\pgfmathparse{iseven(3)} \pgfmathresult
\end{codeexample}

\end{math-function}

\begin{math-function}{isprime(\mvar{x})}
\mathcommand

  Returns |1| if the integer part of \mvar{x} is prime. Otherwise, returns |0|.

\begin{codeexample}[]
\pgfmathparse{isprime(1)} \pgfmathresult, 
\pgfmathparse{isprime(2)} \pgfmathresult,
\pgfmathparse{isprime(31)} \pgfmathresult,
\pgfmathparse{isprime(64)} \pgfmathresult
\end{codeexample}

\end{math-function}


\subsubsection{Trigonometric functions}

\label{pgfmath-functions-trigonometric}

\begin{math-function}{pi}
\mathcommand

	Returns the value $\pi=3.141592654$.
	
\begin{codeexample}[]
\pgfmathparse{pi} \pgfmathresult
\end{codeexample}

\begin{codeexample}[]
\pgfmathparse{pi r} \pgfmathresult
\end{codeexample}

\end{math-function}

\begin{math-function}{rad(\mvar{x})}
\mathcommand

	Convert \mvar{x} to radians. \mvar{x} is assumed to be in degrees.
	
\begin{codeexample}[]
\pgfmathparse{rad(90)} \pgfmathresult
\end{codeexample}

\end{math-function}

\begin{math-function}{deg(\mvar{x})}
\mathcommand

	Convert \mvar{x} to degrees. \mvar{x} is assumed to be in radians.
	
\begin{codeexample}[]
\pgfmathparse{deg(3*pi/2)} \pgfmathresult
\end{codeexample}

\end{math-function}

\begin{math-function}{sin(\mvar{x})}
\mathcommand

	Sine of \mvar{x}. By employing the |r| operator, \mvar{x} can be in
	radians.
	
\begin{codeexample}[]
\pgfmathparse{sin(60)} \pgfmathresult
\end{codeexample}

\begin{codeexample}[]
\pgfmathparse{sin(pi/3 r)} \pgfmathresult
\end{codeexample}

\end{math-function}

\begin{math-function}{cos(\mvar{x})}
\mathcommand

	Cosine of \mvar{x}. By employing the |r| operator, \mvar{x} can be in
	radians.

\begin{codeexample}[]
\pgfmathparse{cos(60)} \pgfmathresult
\end{codeexample}

\begin{codeexample}[]
\pgfmathparse{cos(pi/3 r)} \pgfmathresult
\end{codeexample}

\end{math-function}

\begin{math-function}{tan(\mvar{x})}
\mathcommand

	Tangent of \mvar{x}. By employing the |r| operator, \mvar{x} can be in
	radians.
	
\begin{codeexample}[]
\pgfmathparse{tan(45)} \pgfmathresult
\end{codeexample}

\begin{codeexample}[]
\pgfmathparse{tan(2*pi/8 r)} \pgfmathresult
\end{codeexample}

\end{math-function}


\begin{math-function}{sec(\mvar{x})}
\mathcommand

	Secant of \mvar{x}. By employing the |r| operator, \mvar{x} can be in
	radians.

\begin{codeexample}[]
\pgfmathparse{sec(45)} \pgfmathresult
\end{codeexample}

\end{math-function}

\begin{math-function}{cosec(\mvar{x})}
\mathcommand

	Cosecant of \mvar{x}. By employing the |r| operator, \mvar{x} can be in
	radians.
	
\begin{codeexample}[]
\pgfmathparse{cosec(30)} \pgfmathresult
\end{codeexample}

\end{math-function}

\begin{math-function}{cot(\mvar{x})}
\mathcommand

	Cotangent of \mvar{x}. By employing the |r| operator, \mvar{x} can be in
	radians.
	
\begin{codeexample}[]
\pgfmathparse{cot(15)} \pgfmathresult
\end{codeexample}

\end{math-function}

\begin{math-function}{asin(\mvar{x})}
\mathcommand

	Arcsine of \mvar{x}. The result is in degrees and in the range $\pm90^\circ$.

\begin{codeexample}[]
\pgfmathparse{asin(0.7071)} \pgfmathresult
\end{codeexample}

\end{math-function}

\begin{math-function}{acos(\mvar{x})}
\mathcommand

	Arccosine of \mvar{x} in degrees. The result is in the range $[0^\circ,180^\circ]$.

\begin{codeexample}[]
\pgfmathparse{acos(0.5)} \pgfmathresult
\end{codeexample}

\end{math-function}

\begin{math-function}{atan(\mvar{x})}
\mathcommand

	Arctangent of $x$ in degrees.

\begin{codeexample}[]
\pgfmathparse{atan(1)} \pgfmathresult
\end{codeexample}

\end{math-function}

\begin{math-function}{atan2(\mvar{y},\mvar{x})}
\mathcommand[atantwo(\mvar{y},\mvar{x})]

	Arctangent of $y\div x$ in degrees. This also takes into account the
	quadrants.

\begin{codeexample}[]
\pgfmathparse{atan2(-4,3)} \pgfmathresult
\end{codeexample}

\end{math-function}

\subsubsection{Comparison and logical functions}

\label{pgfmath-functions-comparison}

\begin{math-function}{equal(\mvar{x},\mvar{y})}
\mathcommand

	This returns |1| if $\mvar{x}=\mvar{y}$ and |0| otherwise.
	
\begin{codeexample}[]
\pgfmathparse{equal(20,20)} \pgfmathresult
\end{codeexample}

\end{math-function}

\begin{math-function}{greater(\mvar{x},\mvar{y})}
\mathcommand

	This returns |1| if $\mvar{x}>\mvar{y}$ and |0| otherwise.
	
\begin{codeexample}[]
\pgfmathparse{greater(20,25)} \pgfmathresult
\end{codeexample}

\end{math-function}

\begin{math-function}{less(\mvar{x},\mvar{y})}
\mathcommand

	This returns |1| if $\mvar{x}<\mvar{y}$ and |0| otherwise.
	
\begin{codeexample}[]
\pgfmathparse{greater(20,25)} \pgfmathresult
\end{codeexample}
\end{math-function}

\begin{math-function}{notequal(\mvar{x},\mvar{y})}
\mathcommand

	This returns |0| if $\mvar{x}=\mvar{y}$ and |1| otherwise.
	
\begin{codeexample}[]
\pgfmathparse{notequal(20,25)} \pgfmathresult
\end{codeexample}

\end{math-function}

\begin{math-function}{notgreater(\mvar{x},\mvar{y})}
\mathcommand

	This returns |1| if $\mvar{x}\leq\mvar{y}$ and |0| otherwise.
	
\begin{codeexample}[]
\pgfmathparse{notgreater(20,25)} \pgfmathresult
\end{codeexample}
\end{math-function}

\begin{math-function}{notless(\mvar{x},\mvar{y})}
\mathcommand

	This returns |1| if $\mvar{x}\geq\mvar{y}$ and |0| otherwise.
	
\begin{codeexample}[]
\pgfmathparse{notless(20,25)} \pgfmathresult
\end{codeexample}

\end{math-function}

\begin{math-function}{and(\mvar{x},\mvar{y})}
\mathcommand

	This returns |1| if \mvar{x} and \mvar{y} both evaluate to
	non-zero values. Otherwise |0| is returned.
	
\begin{codeexample}[]
\pgfmathparse{and(5>4,6>7)} \pgfmathresult
\end{codeexample}

\end{math-function}

\begin{math-function}{or(\mvar{x},\mvar{y})}
\mathcommand

	This returns |1| if either \mvar{x} or \mvar{y} evaluate to
	non-zero values. Otherwise |0| is returned.
	
\begin{codeexample}[]
\pgfmathparse{and(5>4,6>7)} \pgfmathresult
\end{codeexample}

\end{math-function}

\begin{math-function}{not(\mvar{x})}
\mathcommand

	This returns |1| if $\mvar{x}=0$, otherwise |0|.
	
\begin{codeexample}[]
\pgfmathparse{not(true)} \pgfmathresult
\end{codeexample}

\end{math-function}


\begin{math-function}{ifthenelse(\mvar{x},\mvar{y},\mvar{z})}
\mathcommand

	This returns \mvar{y} if \mvar{x} evaluates to some non-zero value,
	otherwise \mvar{z} is returned.
	
\begin{codeexample}[]
\pgfmathparse{ifthenelse(5==4,"yes","no")} \pgfmathresult
\end{codeexample}

\end{math-function}

\begin{math-function}{true}
\mathcommand

	This evaluates to |1|.
	
\begin{codeexample}[]
\pgfmathparse{true ? "yes" : "no"} \pgfmathresult
\end{codeexample}

\end{math-function}

\begin{math-function}{false}
\mathcommand

	This evaluates to |0|.
	
\begin{codeexample}[]
\pgfmathparse{false ? "yes" : "no"} \pgfmathresult
\end{codeexample}

\end{math-function}



\subsubsection{Pseudo-random functions}

\label{pgfmath-functions-random}

\begin{math-function}{rnd}
\mathcommand

	Generates a pseudo-random number between $0$ and $1$ with a uniform distribution.

\begin{codeexample}[]
\foreach \x in {1,...,10}{\pgfmathparse{rnd}\pgfmathresult, }
\end{codeexample}

\end{math-function}

\begin{math-function}{rand}
\mathcommand

	Generates a pseudo-random number between $-1$ and $1$ with a uniform distribution.

\begin{codeexample}[]
\foreach \x in {1,...,10}{\pgfmathparse{rand}\pgfmathresult, }
\end{codeexample}

\end{math-function}

\begin{math-function}{random(\opt{\mvar{x},\mvar{y}})}
\mathcommand
  This function takes zero, one or two arguments. If there are zero
  arguments, a uniform random number between $0$ and $1$ is generated. If there is
  one argument \mvar{x}, a random integer between $1$ and \mvar{x} is
  generated. Finally, if there are two arguments, a random integer
  between \mvar{x} and \mvar{y} is generated. If there are no
  arguments, the \pgfname{} command should be called as follows:
  |\pgfmathrandom{}|.

\begin{codeexample}[]
\foreach \x in {1,...,10}{\pgfmathparse{random()}\pgfmathresult, }
\end{codeexample}

\begin{codeexample}[]
\foreach \x in {1,...,10}{\pgfmathparse{random(100)}\pgfmathresult, }
\end{codeexample}

\begin{codeexample}[]
\foreach \x in {1,...,10}{\pgfmathparse{random(232,762)}\pgfmathresult, }
\end{codeexample}
\end{math-function}

\subsubsection{Base conversion functions}

\label{pgfmath-functions-base}

\begin{math-function}{hex(\mvar{x})}
\mathcommand

  Convert \mvar{x}{} (assumed to be an integer in base 10) to a
  hexadecimal representation, using lower case alphabetic digits.
	No further calculation will be possible with the result.
	
\begin{codeexample}[]
\pgfmathparse{hex(65535)} \pgfmathresult
\end{codeexample}
\end{math-function}

\begin{math-function}{Hex(\mvar{x})}
\mathcommand

  Convert \mvar{x}{} (assumed to be an integer in base 10) to a
  hexadecimal representation, using upper case alphabetic digits.
  No further calculation will be possible with the result.

\begin{codeexample}[]
\pgfmathparse{Hex(65535)} \pgfmathresult
\end{codeexample}
\end{math-function}

\begin{math-function}{oct(\mvar{x})}
\mathcommand

  Convert \mvar{x}{} (assumed to be an integer in base 10) to an
  octal representation.
  No further calculation should be attempted with the result, as
  the parser can only process numbers converted to base 10.
\begin{codeexample}[]
\pgfmathparse{oct(63)} \pgfmathresult
\end{codeexample}
\end{math-function}

\begin{math-function}{bin(\mvar{x})}
\mathcommand

  Convert \mvar{x}{} (assumed to be an integer in base 10) to a
  binary representation.
  No further calculation should be attempted with the result, as
  the parser can only process numbers converted to base 10.

\begin{codeexample}[]
\pgfmathparse{bin(185)} \pgfmathresult
\end{codeexample}
\end{math-function}

\subsubsection{Miscellaneous functions}

\label{pgfmath-functions-misc}

\begin{math-function}{min(\mvar{x$_1$},\mvar{x$_2$},\ldots,\mvar{x$_n$})}
\mathcommand[min({\mvar{x$_1$},\mvar{x$_2$},\ldots},{\ldots,\mvar{x$_{n-1}$},\mvar{x$_n$}})]

  Return the minimum value from \mvar{x$_1$}\ldots\mvar{x$_n$}.
  For historical reasons, the command |\pgfmathmin| takes two
  arguments, but each of these can contain an arbitrary number
  of comma separated values.

\begin{codeexample}[]
\pgfmathparse{min(3,4,-2,250,-8,100)} \pgfmathresult
\end{codeexample}

\end{math-function}


\begin{math-function}{max(\mvar{x$_1$},\mvar{x$_2$},\ldots,\mvar{x$_n$})}
\mathcommand[max({\mvar{x$_1$},\mvar{x$_2$},\ldots},{\ldots,\mvar{x$_{n-1}$},\mvar{x$_n$}})]

  Return the maximum value from \mvar{x$_1$}\ldots\mvar{x$_n$}.
  Again, for historical reasons, the command |\pgfmathmax| takes two
  arguments, but each of these can contain an arbitrary number
  of comma separated values.

\begin{codeexample}[]
\pgfmathparse{max(3,4,-2,250,-8,100)} \pgfmathresult
\end{codeexample}

\end{math-function}


\begin{math-function}{veclen(\mvar{x},\mvar{y})}
\mathcommand

 Calculates $\sqrt{\left(\textrm{\mvar{x}}^2+\textrm{\mvar{y}}^2\right)}$.
 This uses a polynomial approximation, based on ideas of Rouben Rostamian
\begin{codeexample}[]
\pgfmathparse{veclen(12,5)} \pgfmathresult
\end{codeexample}

\end{math-function}





\begin{math-function}{array(\mvar{x},\mvar{y})}
\mathcommand

	This accesses the array \mvar{x} at the index \mvar{y}. The
	array must begin and end with braces (e.g., |{1,2,3,4}|) and
	array indexing starts at |0|.
	
\begin{codeexample}[]
\pgfmathparse{array({9,13,17,21},2)} \pgfmathresult
\end{codeexample}

\end{math-function}


The following hyperbolic functions were adapted from code
suggested by Martin Heller:

\begin{math-function}{sinh(\mvar{x})}
\mathcommand

	The hyperbolic sine of \mvar{x}%
	
\begin{codeexample}[]
\pgfmathparse{sinh(0.5)} \pgfmathresult
\end{codeexample}

\end{math-function}

\begin{math-function}{cosh(\mvar{x})}
\mathcommand

	The hyperbolic cosine of \mvar{x}%
	
\begin{codeexample}[]
\pgfmathparse{cosh(0.5)} \pgfmathresult
\end{codeexample}

\end{math-function}

\begin{math-function}{tanh(\mvar{x})}
\mathcommand

	The hyperbolic tangent of \mvar{x}%
	
\begin{codeexample}[]
\pgfmathparse{tanh(0.5)} \pgfmathresult
\end{codeexample}

\end{math-function}

\begin{math-function}{width("\mvar{x}")}
\mathcommand

  Return the width of a \TeX{} (horizontal) box containing \mvar{x}.
  The quote characters are necessary to prevent \mvar{x}{} from being
  parsed.
  It is important to remember that any expression is expanded with
  |\edef| before being parsed, so any macros (e.g., font commands
  like |\tt| or |\Huge|) will need to be ``protected'' (e.g.,
  |\noexpand\Huge| is usually sufficient).

\begin{codeexample}[]
\pgfmathparse{width("Some Lovely Text")} \pgfmathresult
\end{codeexample}

	Note that results of this method are provided in points.
\end{math-function}

\begin{math-function}{height("\mvar{x}")}
\mathcommand

  Return the height of a box containing \mvar{x}.

\begin{codeexample}[]
\pgfmathparse{height("Some Lovely Text")} \pgfmathresult
\end{codeexample}
\end{math-function}

\begin{math-function}{depth("\mvar{x}")}
\mathcommand

  Returns the depth of a box containing \mvar{x}.

\begin{codeexample}[]
\pgfmathparse{depth("Some Lovely Text")} \pgfmathresult
\end{codeexample}
\end{math-function}


% Copyright 2007 by Mark Wibrow
%
% This file may be distributed and/or modified
%
% 1. under the LaTeX Project Public License and/or
% 2. under the GNU Free Documentation License.
%
% See the file doc/generic/pgf/licenses/LICENSE for more details.

\section{Additional Mathematical Commands}

\label{pgfmath-commands}

Instead of parsing and evaluating complex expressions, you can also
use the mathematical engine to evaluate a single mathematical
operation. The macros used for many of these computations are listed 
above in Section~\ref{pgfmath-functions}. \pgfname{} also provides 
some additional commands which are shown below:

\subsection{Basic arithmetic functions}

\label{pgfmath-commands-basic}

In addition to the commands described in 
Section~\ref{pgfmath-functions-basic}, the following command is
provided:

\begin{command}{\pgfmathreciprocal\marg{x}}         
	Defines |\pgfmathresult| as $1\div\meta{x}$. This provides 
	greatest accuracy when \mvar{x} is small.                  
\end{command}

\subsection{Comparison and logical functions}

In addition to the commands described in 
Section~\ref{pgfmath-functions-comparison}, 
the following command was provided by Christian Feuers\"anger:

\begin{command}{\pgfmathapproxequalto\marg{x}\marg{y}}       
	Defines |\pgfmathresult| 1.0 if $ \rvert \meta{x} - \meta{y} \lvert < 0.0001$, but 0.0 otherwise.                    
	As a side-effect, the global boolean |\ifpgfmathcomparison| will be set accordingly.
\end{command}

\subsection{Pseudo-Random Numbers}

\label{pgfmath-random}

In addition to the commands described in 
Section~\ref{pgfmath-functions-random}, 
the following commands are provided:

\begin{command}{\pgfmathgeneratepseudorandomnumber}
	Defines |\pgfmathresult| as a pseudo-random integer between 1 and 
	$2^{31}-1$. This uses a linear congruency generator, based on ideas
	of Erich Janka.
\end{command}

\begin{command}{\pgfmathrandominteger\marg{macro}\marg{minimum}\marg{maximum}}
	This defines \meta{macro} as a pseudo-randomly generated integer from 
	the range \meta{minimum} to \marg{maximum} (inclusive).
	
\begin{codeexample}[]
\begin{pgfpicture}
   \foreach \x in {1,...,50}{
      \pgfmathrandominteger{\a}{1}{50}
      \pgfmathrandominteger{\b}{1}{50}
      \pgfpathcircle{\pgfpoint{+\a pt}{+\b pt}}{+2pt}
      \color{blue!40!white}
      \pgfsetstrokecolor{blue!80!black}
      \pgfusepath{stroke, fill}
   }	  
\end{pgfpicture}
\end{codeexample}
\end{command}

\begin{command}{\pgfmathdeclarerandomlist\marg{list name}\{\marg{item-1}\marg{item 2}...\}}
	This creates a list of items with the name \meta{list name}.
\end{command}

\begin{command}{\pgfmathrandomitem\marg{macro}\marg{list name}}
	Select an item from a random list \meta{list name}. The
	selected item is placed in \meta{macro}.
\end{command}

\begin{codeexample}[]
\begin{pgfpicture}
   \pgfmathdeclarerandomlist{color}{{red}{blue}{green}{yellow}{white}}
   \foreach \a in {1,...,50}{
      \pgfmathrandominteger{\x}{1}{85}
      \pgfmathrandominteger{\y}{1}{85}
      \pgfmathrandominteger{\r}{5}{10}
      \pgfmathrandomitem{\c}{color}
      \pgfpathcircle{\pgfpoint{+\x pt}{+\y pt}}{+\r pt}
      \color{\c!40!white}
      \pgfsetstrokecolor{\c!80!black}
      \pgfusepath{stroke, fill}
   }	  
\end{pgfpicture}
\end{codeexample}

\begin{command}{\pgfmathsetseed\marg{integer}}
  Explicitly sets the seed for the pseudo-random number generator. By
  default it is set to the value of |\time|$\times$|\year|.
\end{command}


      
\subsection{Base Conversion}
	
\label{pgfmath-bases}

\pgfname{} provides limited support for conversion between 
\emph{representations} of numbers. Currently the numbers must be
positive integers in the range $0$ to $2^{31}-1$, and the bases in the
range $2$ to $36$. All digits representing numbers greater than 9 (in
base ten), are alphabetic, but may be upper or lower case. 

In addition to the commands described in 
Section~\ref{pgfmath-functions-base}, 
the following commands are provided:

\begin{command}{\pgfmathbasetodec\marg{macro}\marg{number}\marg{base}}
	Defines \meta{macro} as the result of converting \meta{number} from
	base \meta{base} to base 10. Alphabetic digits can be upper or lower
	case.

\medskip{\def\medskip{}

\begin{codeexample}[]
\pgfmathbasetodec\mynumber{107f}{16} \mynumber
\end{codeexample}
	
	\noindent Note that, as usual in \TeX, the braces around an argument can be omitted if the argument is just a single token (a macro name is a single token).


\begin{codeexample}[]
\pgfmathbasetodec\mynumber{33FC}{20} \mynumber
\end{codeexample}

}\medskip

\end{command}

\begin{command}{\pgfmathdectobase\marg{macro}\marg{number}\marg{base}}
	Defines \meta{macro} as the result of converting \meta{number} from
	base 10 to base \meta{base}. Any resulting alphabetic digits are in
	\emph{lower case}.
	
\begin{codeexample}[]
\pgfmathdectobase\mynumber{65535}{16} \mynumber
\end{codeexample}

\end{command}

\begin{command}{\pgfmathdectoBase\marg{macro}\marg{number}\marg{base}}
	Defines \meta{macro} as the result of converting \meta{number} from
	base 10 to base \meta{base}. Any resulting alphabetic digits are in
	\emph{upper case}.
	
\begin{codeexample}[]
\pgfmathdectoBase\mynumber{65535}{16} \mynumber
\end{codeexample}

\end{command}

\begin{command}{\pgfmathbasetobase\marg{macro}\marg{number}\marg{base-1}\marg{base-2}}
	Defines \meta{macro} as the result of converting \meta{number} from
	base \meta{base-1} to base \meta{base-2}. Alphabetic digits in 
	\meta{number} can be upper or lower case, but any resulting 
	alphabetic digits are in \emph{lower case}.
	
\begin{codeexample}[]
\pgfmathbasetobase\mynumber{11011011}{2}{16} \mynumber
\end{codeexample}

\end{command}

\begin{command}{\pgfmathbasetoBase\marg{macro}\marg{number}\marg{base-1}\marg{base-2}}
	Defines \meta{macro} as the result of converting \meta{number} from
	base \meta{base-1} to base \meta{base-2}. Alphabetic digits in 
	\meta{number} can be upper or lower case, but any resulting 
	alphabetic digits are in \emph{upper case}.
	
\begin{codeexample}[]
\pgfmathbasetoBase\mynumber{121212}{3}{12} \mynumber
\end{codeexample}

\end{command}


\begin{command}{\pgfmathsetbasenumberlength\marg{integer}}
	Sets the number of digits in the result of a base conversion to 
	\meta{integer}. If the result of a conversion has less digits
	than this number, it is prefixed with zeros.

\begin{codeexample}[]
\pgfmathsetbasenumberlength{8}
\pgfmathdectobase\mynumber{15}{2} \mynumber
\end{codeexample}
\end{command}




\subsection{Angle Computations}

Unlike the rest of the math engine, which is a ``standalone'' package,
the following commands only work in conjunction with the core of
\pgfname.



\begin{command}{\pgfmathanglebetweenpoints\marg{p}\marg{q}}
  Returns the angle of a line from \meta{p} to \meta{q} relative to a
  line going straight right from \meta{p}.
  
\begin{codeexample}[]
\pgfmathanglebetweenpoints{\pgfpoint{1cm}{3cm}}{\pgfpoint{2cm}{4cm}}
\pgfmathresult
\end{codeexample}
\end{command}

\begin{command}{\pgfmathanglebetweenlines\marg{$p_1$}\marg{$q_1$}\marg{$p_2$}\marg{$q_2$}}
  Returns the clockwise angle between a line going through $p_1$ and
  $q_1$ and a line going through $p_2$ and $q_2$.
  
\begin{codeexample}[]
\pgfmathanglebetweenlines{\pgfpoint{1cm}{3cm}}{\pgfpoint{2cm}{4cm}}
                         {\pgfpoint{0cm}{1cm}}{\pgfpoint{1cm}{0cm}}  
\pgfmathresult
\end{codeexample}
\end{command}


% Copyright 2007 by Mark Wibrow
%
% This file may be distributed and/or modified
%
% 1. under the LaTeX Project Public License and/or
% 2. under the GNU Free Documentation License.
%
% See the file doc/generic/pgf/licenses/LICENSE for more details.


\section[Reimplementing the Computations of the Mathematical Engine]
  {Reimplementing the Computations of the\\ Mathematical Engine}

\label{pgfmath-reimplement}

Perhaps you are not satisfied with the Newton-Raphson approximation to
square-roots. Perhaps you have a fantastically more accurate
and efficient way of calculating the sine or cosine of angles. Perhaps
 you would like the library to interface with a package such as |fp| 
 for fixed-point arithmetic (but note that |fp| is \emph{very} slow).
In these case you will want to replace the current implementations of
the computations done by the mathematical engine by your own code. 

The mathematical engine was designed with such a replacement in
mind. For this reason, the operations and functions like |\pgfmathadd|
are implemented in the following manner: 

\begin{itemize}
\item |\pgfmath|\meta{function name} 

  This macro is the ``public'' interface for the function
  \meta{function name}. All arguments passed to this macro are 
  evaluated using |\pgfmathparse| and then passed on to the following
  function:
  
\item |\pgfmath|\meta{function name}|@|
  
  This macro is the ``non-public'' implementation of the functions 
  algorithm (but note that, for speed, the parser calls this macro 
  rather than the ``public'' one). Arguments passed to this macro 
  are expected to be numbers \emph{without units}. This is the macro 
  which should be rewritten with your prize-winning new algorithm.
	
\end{itemize}

The effect of |\pgfmath|\meta{function name}|@| should be to set the
macro |\pgfmathresult| to the correct value (namely to the result of
the computation without units). Furthermore, the function should have
no other side effects, that is, it should not change any global
values. One way to achieve this is to use the following code:

\begin{codeexample}[code only]
\def\pgfmath...@#1#2...{%
   \begingroup%
      ... code for algorithm XXX ...
      \pgfmath@returnone\pgfmath@x%
   \endgroup%
}
\end{codeexample}


The macro |\pgfmath@returnone<macro>| uses some |\aftergroup| magic to
save result of the algorithm, by defining |\pgfmathresult| as the 
expansion of |<macro>| \emph{without units}. |<macro>| can be a macro
containing a number (with or without units), or a dimension or count
(or possibly a skip) register. By performing the algorithm within a
\TeX{} group, \pgfname{} registers such as |\pgf@x|, |\pgf@y| and 
|\c@pgf@counta|, |\c@pgfcountb|, and so forth can be used at will.
Note that current the implementation uses |\pgfmath@x|, |\pgfmath@y|, 
and |\c@pgfmath@counta|, |\c@pgfmath@countb| throughout, so for 
consistency these should be employed. Whilst they are currently |\let|
to their \pgfname{} equivalents (see |pgfmathutil.code.tex|), this 
could change (as could the \pgfname{} registers), so keeping things
consistent is probably a good idea.

% Copyright 2007 by Mark Wibrow
%
% This file may be distributed and/or modified
%
% 1. under the LaTeX Project Public License and/or
% 2. under the GNU Free Documentation License.
%
% See the file doc/generic/pgf/licenses/LICENSE for more details.

\section{Conversion Between Bases}
	
\label{pgfmath-bases}

\pgfname{} provides limited support for conversion between 
\emph{representations} of numbers. Currently the numbers must be
positive integers in the range $0$ to $2^{31}-1$, and the bases in the
range $2$ to $36$. All digits representing numbers greater than 9 (in
base ten), are alphabetic, but may be upper or lower case. Note, that
again, examples in this section, show the result of the calculation
\emph{not} what is shown on screen.

\begin{command}{\pgfmathbasetodec\marg{macro}\marg{number}\marg{base}}
	Defines \meta{macro} as the result of converting \meta{number} from
	base \meta{base} to base 10. Alphabetic digits can be upper or lower
	case.

\medskip{\def\medskip{}

\begin{codeexample}[post=\tt\footnotesize\mynumber]
\pgfmathbasetodec\mynumber{107f}{16}
\end{codeexample}


\begin{codeexample}[post=\tt\footnotesize\mynumber]
\pgfmathbasetodec\mynumber{33FC}{20}
\end{codeexample}

}\medskip

\end{command}

\begin{command}{\pgfmathdectobase\marg{macro}\marg{number}\marg{base}}
	Defines \meta{macro} as the result of converting \meta{number} from
	base 10 to base \meta{base}. Any resulting alphabetic digits are in
	\emph{lower case}.
	
\begin{codeexample}[post=\tt\footnotesize\mynumber]
\pgfmathdectobase\mynumber{65535}{16}
\end{codeexample}

\end{command}

\begin{command}{\pgfmathdectoBase\marg{macro}\marg{number}\marg{base}}
	Defines \meta{macro} as the result of converting \meta{number} from
	base 10 to base \meta{base}. Any resulting alphabetic digits are in
	\emph{upper case}.
	
\begin{codeexample}[post=\tt\footnotesize\mynumber]
\pgfmathdectoBase\mynumber{65535}{16}
\end{codeexample}

\end{command}

\begin{command}{\pgfmathbasetobase\marg{macro}\marg{number}\marg{base-1}\marg{base-2}}
	Defines \meta{macro} as the result of converting \meta{number} from
	base \meta{base-1} to base \meta{base-2}. Alphabetic digits in 
	\meta{number} can be upper or lower case, but any resulting 
	alphabetic digits are in \emph{lower case}.
	
\begin{codeexample}[post=\tt\footnotesize\mynumber]
\pgfmathbasetobase\mynumber{11011011}{2}{16}
\end{codeexample}

\end{command}

\begin{command}{\pgfmathbasetoBase\marg{macro}\marg{number}\marg{base-1}\marg{base-2}}
	Defines \meta{macro} as the result of converting \meta{number} from
	base \meta{base-1} to base \meta{base-2}. Alphabetic digits in 
	\meta{number} can be upper or lower case, but any resulting 
	alphabetic digits are in \emph{upper case}.
	
\begin{codeexample}[post=\tt\footnotesize\mynumber]
\pgfmathbasetoBase\mynumber{121212}{3}{12}
\end{codeexample}

\end{command}


\begin{command}{\pgfmathsetbasenumberlength\marg{integer}}
	Set the number of digits in the result of a base conversion to 
	\meta{integer}. If the result of a conversion has less than this 
	number it is prefixed with zeros.

\begin{codeexample}[post=\tt\footnotesize\mynumber]
\pgfmathsetbasenumberlength{8}
\pgfmathdectobase\mynumber{15}{2}
\end{codeexample}

\end{command}
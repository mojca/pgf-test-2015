% Copyright 2008 by Mark Wibrow
%
% This file may be distributed and/or modified
%
% 1. under the LaTeX Project Public License and/or
% 2. under the GNU Free Documentation License.
%
% See the file doc/generic/pgf/licenses/LICENSE for more details.

\section{Decorated Paths}

\label{section-tikz-decorations}


\subsection{Overview}

Decorations are a general concept to make (sub)paths ``more
interesting.'' The general idea is the following: First, you construct
a path using the usual path construction commands. The resulting path
is, in essence, a series of straight and curved lines. Instead of
directly using this path for filling or drawing, you can then specify
that it should form the basis for a decoration. In this case,
depending on which decoration you use, a new path is constructed
``along'' the path you specified. For instance, with the |zigzag|
decoration, the new path is a zigzagging line that goes along the old
path.

Let us have a look at an example: In the first picture, we see a path
that consists of a line, an arc, and a line. In the second picture,
this path has been used as the basis of a decoration.

\begin{codeexample}[]
\tikz \fill
  [fill=blue!20,draw=blue,thick] (0,0) -- (2,1) arc (90:-90:.5) -- cycle;
\end{codeexample}
\begin{codeexample}[]
\tikz \fill [decorate,decoration={zigzag}]
  [fill=blue!20,draw=blue,thick] (0,0) -- (2,1) arc (90:-90:.5) -- cycle;
\end{codeexample}

It is also possible to decorate only a subpath (the exact syntax will
be explained later in this section).
\begin{codeexample}[]
\tikz \fill [decoration={zigzag}]
  [fill=blue!20,draw=blue,thick] (0,0) -- (2,1)
    decorate { arc (90:-90:.5) } -- cycle;
\end{codeexample}

The |zigzag| decoration will be called a \emph{path
  morphing} decoration because it morphs a path into a different, but
topologically equivalent path. Not all decorations are path
morphing; rather there are three kinds of decorations.


\begin{enumerate}
\item The just-mentioned \emph{path morphing} decorations morph the
  path in the sense that what used to be a straight  line might
  afterwards be a squiggly line or might have bumps. However, a line
  is still and a line and path deforming decorations do not change
  the number of subpaths. 

  Examples of such decorations are the |snake| or the |zigzag|
  decoration. Many such decorations are defined in the library
  |decorations.pathmorphing|.
  
\item \emph{Path replacing} decorations completely replace the
  path by a different path that is only ``loosely based'' on the
  original path. For instance, the |crosses| decoration replaces a path
  by a path consisting of a sequence of crosses. Note how in the
  following example filling the path has no effect since the path
  consist only of (numerous) unconnected straight line subpaths:
\begin{codeexample}[]
\tikz \fill [decorate,decoration={crosses}]
  [fill=blue!20,draw=blue,thick] (0,0) -- (2,1) arc (90:-90:.5) -- cycle;
\end{codeexample}

  Examples of path replacing decorations are |crosses| or |ticks| or
  |shape backgrounds|. Such decorations are defined in the library
  |decorations.pathreplacing|, but also in |decorations.shapes|.
  
\item \emph{Path removing} decorations completely remove the
  to-be-decorated path. Thus, they have no effect on the main path
  that is being constructed. Instead, they typically have numerous
  \emph{side  effects}. For instance, they might ``write some text''
  along the (removed) path or they might place nodes along this
  path. Note that for such decorations the path usage command for the
  main path have no influence on how the decoration looks like.

\begin{codeexample}[]
\tikz \fill [decorate,decoration={text along path,
               text=This is a text along a path. Note how the path is lost.}]
  [fill=blue!20,draw=blue,thick] (0,0) -- (2,1) arc (90:-90:.5) -- cycle;
\end{codeexample}
\end{enumerate}

Decorations are defined in different decoration libraries, see
Section~\ref{section-library-decorations} for details. It is also
possible to define your own decorations, see
Section~\ref{section-base-decorations}, but you need to use the
\pgfname\ basic layer and a bit of theory is involved.

Decorations can be used to decorate already decorated paths. In the
following three graphics, we start with a simple path, then decorate
it once, and then decorate the decorated path once more.

\begin{codeexample}[]
\tikz \fill [fill=blue!20,draw=blue,thick]
  (0,0) rectangle (3,2);
\end{codeexample}
\begin{codeexample}[]
\tikz \fill [fill=blue!20,draw=blue,thick]
  decorate[decoration={zigzag,segment length=10mm,amplitude=2.5mm}]
    { (0,0) rectangle (3,2) };
\end{codeexample}
\begin{codeexample}[]
\tikz \fill [fill=blue!20,draw=blue,thick]
  decorate[decoration={crosses,segment length=2mm}] {
    decorate[decoration={zigzag,segment length=10mm,amplitude=2.5mm}] {
      (0,0) rectangle (3,2) 
    }
  };
\end{codeexample}

One final word of warning: Decorations can be pretty slow to
typeset and they can be inaccurate. The reason is that \pgfname\ has
to a \emph{lot} of rather difficult computations in the background and
\TeX\ is not very good at doing math. Decorations are fastest when
applied to straight line segments, but even then they are much slower
than other alternative. For instance, the |ticks| decoration can be
simulated by clever use of a dashing pattern and the dashing pattern
will literally be thousands of times faster to typeset. However, for
most decorations there are no real alternatives.

\begin{tikzlibrary}{decorations}
  In order to use decorations, you first have to load a decoration
  library. This |decoration| library defines the basic options
  described in the following, but it does not define any new
  decorations. This is done by libraries like
  |decorations.text|. Since these more specialized libraries include
  the |decoration| library automatically, you usually do not have to
  bother about it.
\end{tikzlibrary}



\subsection{Decorating a Subpath Using the Decorate Path Command}

The most general way to decorate a (sub)path is the following path
command.

\begin{pathoperation}{decorate}{\opt{\oarg{options}}\marg{subpath}}
  This path operation causes the \meta{subpath} to be
  decorated using the current decoration. Depending on the decoration,
  this may or may not extend the current path.
\begin{codeexample}[]
\begin{tikzpicture}
  \draw [help lines] grid (3,2);
  \draw decorate [decoration={name=zigzag}]
         { (0,0) .. controls (0,2) and (3,0) .. (3,2) |- (0,0) };
\end{tikzpicture}
\end{codeexample}
  The path can include straight lines, curves,
  rectangles, arcs, circles, ellipses, and even already decorated
  paths (that is, you can nest applications of the |decorate| path
  command, see below).

  Closed subpaths (like  rectangles or circles) may not be decorated
  succesfully with ``continuous'' decorations (those that do not
  create multiple segmented subpaths). In addition, due to the limits
  on the precision in  \TeX, some inaccuraces in positioning when
  crossing subpath boundaries may occasionally be found.

  You can use nodes normally inside the \meta{subpath}.
\begin{codeexample}[]
\begin{tikzpicture}
  \draw [help lines] grid (3,2);
  \draw decorate [decoration={name=zigzag}]
    { (0,0) -- (2,2) node (hi) [left,draw=red] {Hi!} arc(90:0:1)};

  \draw [blue] decorate [decoration={crosses}] {(3,0) -- (hi)};
\end{tikzpicture}
\end{codeexample}
  
  The following key is used to select the decoration and also to
  select further ``rendering options'' for the decoration.

  \begin{key}{/pgf/decoration=\meta{decoration options}}
    \keyalias{tikz}
    This option is used to specify which decoration is used and how it
    will look like. Note that his key will \emph{not} cause any
    decorations to be applied, immediately. It takes the |decorate| path
    command or the |decorate| option to actually decorate a path. The
    |decoration| option is only used to specify which decoration should
    be used, in principle. You can also use this option at the
    beginning of a picture or a scope to specify the decoration to be
    used with each invocation of the |decorate| path
    command. Naturally, any local options of the |decorate| path
    command override these ``global'' options.
\begin{codeexample}[]
\begin{tikzpicture}[decoration=zigzag]
  \draw       decorate                      {(0,0) -- (3,2)};
  \draw [red] decorate [decoration=crosses] {(0,2) -- (3,0)};
\end{tikzpicture}
\end{codeexample}
    
    The \meta{decoration options} are special options
    (which have the path prefix |/pgf/decoration/|) that determine the
    properties of the decoration. Which options are appropriate for a
    decoration depend strongly on the decoration, you will have to look
    up the appropriate options in the documentation of the decoration,
    see Section~\ref{section-library-decorations}.
    
    There is one option (available only in \tikzname) that is special:
    \begin{key}{/pgf/decoration/name=\meta{name} (initially none)}
      Use this key to set which decoration is to be used. The
      \meta{name} can both be a decoration or a meta-decoration (you
      need to worry about the difference only if you wish to define
      your own decorations).
      
      If you set \meta{name} to |none|, no decorations are added.
\begin{codeexample}[]
\begin{tikzpicture}
  \draw [help lines] grid (3,2);
  \draw decorate [decoration={name=zigzag}]
         { (0,0) .. controls (0,2) and (3,0) .. (3,2) };
\end{tikzpicture}
\end{codeexample}
      Since this option is used so often, you can also leave out the
      |name=| part. Thus, the above example can be rewritten more
      succinctly: 
\begin{codeexample}[]
\begin{tikzpicture}
  \draw [help lines] grid (3,2);
  \draw decorate [decoration=zigzag]
         { (0,0) .. controls (0,2) and (3,0) .. (3,2) };
\end{tikzpicture}
\end{codeexample}
      In general, when \meta{decoration options} are parsed, for each
      unknown key it is checked whether that key happens to be a
      (meta-)decoration and, if so, the |name| option is executed for
      this key.
    \end{key}

    Further options allow you to adjust the position of decorations
    relative to the to-be-decorated path. See
    Section~\ref{section-decorations-adjust} below for details.
  \end{key}

  Recall that some decorations actually completely remove the
  to-be-decorated path. In such cases, the construction of the main
  path is resumed after the |decorate| path command ends.
  
\begin{codeexample}[]
\begin{tikzpicture}[decoration={text along path,text=
      around and around and around and around we go}]

  \draw (0,0) -- (1,1) decorate { -- (2,1) } -- (3,0);
\end{tikzpicture}
\end{codeexample}

  It is permissible to nest |decorate| commands. In this case, the
  path resulting from the first decoration process is used as the
  to-be-decorated path for the second decoration process. This is
  especially useful for drawing fractals. The |Koch snowflake|
  decoration replaces a straight line like \tikz\draw (0,0) -- (1,0);
  by \tikz[decoration=Koch snowflake] \draw decorate{(0,0) --
    (1,0)};. Repeatedly applying this transformation to a triangle
  yields a fractal that looks a bit like a snowflake, hence the name. 
\begin{codeexample}[]
\begin{tikzpicture}[decoration=Koch snowflake,draw=blue,fill=blue!20,thick]
  \filldraw (0,0) -- ++(60:1) -- ++(-60:1) -- cycle ;
  \filldraw decorate{ (0,-1) -- ++(60:1) -- ++(-60:1) -- cycle };
  \filldraw decorate{ decorate{ (0,-2.5) -- ++(60:1) -- ++(-60:1) -- cycle }};
\end{tikzpicture}
\end{codeexample}
\end{pathoperation}



\subsection{Decorating a Complete Path}

You may sometimes wish to decorate a path over whose construction you
have no control. For instance, the path of the background of a node is
created without your having a chance to issue a |decorate| path
command. In such cases you can use the following option, which allows
you to decorate a path ``after the fact.''

\begin{key}{/tikz/decorate=\opt{\meta{boolean}} (default true)}
  When this key is set, the whole path is decorated after it has been
  finished. The decoration used for decorating the path is set via the
  |decoration| way, in exactly the same way as for the |decorate| path
  command. Indeed, the following two commands have the same effect:
  \begin{enumerate}
  \item |\path decorate[|\meta{options}|] {|\meta{path}|};|
  \item |\path [decorate,|\meta{options}|] |\meta{path}|;|
  \end{enumerate}
  The main use or the |decorate| option is the you can also use it
  with the nodes. It then causes the background path of the node to be
  decorated. Note that you decorate a background path only once in
  this manner. That is, in contrast to the |decorate| path command you
  cannot apply this option twice (this would just set it to |true|,
  once more).

\begin{codeexample}[]
\begin{tikzpicture}[decoration=zigzag]
  \draw [help lines] (0,0) grid (3,5);
  
  \draw [fill=blue!20,decorate] (1.5,4) circle (1cm);

  \node at (1.5,2.5) [fill=red!20,decorate,ellipse] {Ellipse};

  \node at (1.5,1) [inner sep=6mm,fill=red!20,decorate,ellipse,decoration=
    {text along path,text={This is getting silly}}] {Ellipse};
\end{tikzpicture}
\end{codeexample}

  In the last example, the |text along path| decoration removes the
  path. In such cases it is useful to use a pre- or postaction to
  cause the decoration to be applied only before or after the main
  path has been used. Incidentally, this is another application of the
  |decorate| option that you cannot achieve with the decorate path
  command. 
\begin{codeexample}[]
\begin{tikzpicture}[decoration=zigzag]
  \node at (1.5,1) [inner sep=6mm,fill=red!20,ellipse,
    postaction={decorate,decoration=
    {text along path,text={This is getting silly}}}] {Ellipse};
\end{tikzpicture}
\end{codeexample}
  Here is more useful example, where a postaction is used to add the
  path after the main path has been drawn.
\begin{codeexample}[]
\catcode`\|12
\begin{tikzpicture}
\draw [help lines] grid (3,2);
\fill [draw=red,fill=red!20,
         postaction={decorate,decoration={raise=2pt,text along path,
           text=around and around and around and around we go}}] 
  (0,1) arc (180:-180:1.5cm and 1cm);
\end{tikzpicture}
\end{codeexample} 
\end{key}


\subsection{Adjusting Decorations}

\label{section-decorations-adjust}

\subsubsection{Positioning Decorations Relative to the To-Be-Decorate Path}

The following option, which are only available with \tikzname, allow
you to modify the positioning of decorations relative to the
to-be-decorated path.

\begin{key}{/pgf/decoration/raise=\meta{dimension} (initially 0pt)}
  The segments of the decoration are raised by \meta{dimension}
  relative to the to-be-decorated path. More precisely, the segments
  of the path are offset by this much ``to the left'' of the path as
  we travel along the path. This raising is done after and in addition
  to any transformations set using the |transform| option (see below).

  A negative \meta{dimension} will offset the decoration ``to the
  right'' of the to-be-decorated path.
\begin{codeexample}[]
\begin{tikzpicture}
  \draw [help lines] (0,0) grid (3,2);

  \draw (0,0) -- (1,1) arc (90:0:2 and 1);
  \draw      decorate [decoration=crosses]
        { (0,0) -- (1,1) arc (90:0:2 and 1) };
  \draw[red] decorate [decoration={crosses,raise=5pt}]
        { (0,0) -- (1,1) arc (90:0:2 and 1) };
\end{tikzpicture}
\end{codeexample}
\end{key}

\begin{key}{/pgf/decoration/mirror=\opt{\meta{boolean}}}
  Causes the segments of the decoration to be mirrored along the
  to-be-decorated path. This is done after and in addition to any
  transformations set using the |transform| and/or |raise| options.
\begin{codeexample}[]
\begin{tikzpicture}
  \node (a)          {A};
  \node (b) at (2,1) {B};
  \draw                                                    (a) -- (b);
  \draw[decorate,decoration=brace]                         (a) -- (b);
  \draw[decorate,decoration={brace,mirror},red]            (a) -- (b);
  \draw[decorate,decoration={brace,mirror,raise=5pt},blue] (a) -- (b);
\end{tikzpicture}
\end{codeexample}
\end{key}


\begin{key}{/pgf/decoration/transform=\meta{transformations}}
  This key allows you to specify general \meta{transformations} to be
  applied to the segments of a decoration. These transformations are
  applied before and independently of |raise| and |mirror|
  transformations. The \meta{transformations} should be normal
  \tikzname\ transformations like |shift| or |rotate|.

  In the following example the |shift only| transformation is used to
  make sure that the crosses are \emph{not} sloped along the path.
\begin{codeexample}[]
\begin{tikzpicture}
  \draw [help lines] (0,0) grid (3,2);

  \draw (0,0) -- (1,1) arc (90:0:2 and 1);
  \draw[red,very thick] decorate [decoration={
               crosses,transform={shift only},shape size=1.5mm}]
        { (0,0) -- (1,1) arc (90:0:2 and 1) };
\end{tikzpicture}
\end{codeexample}
\end{key}


\subsubsection{Starting and Ending Decorations Early or Late}

You sometimes may wish to ``end'' a decoration a bit early on the
path. For instance, you might wish a |snake| decoration to stop 5mm
before the end of the path and to continue in a straight line. There
are different ways of achieving this effect, but the easiest may be
the |pre| and |post| options, which only have an effect in
\tikzname. Note, however, that they can only be used with decorations,
not with meta-decorations.

\begin{key}{/pgf/decoration/pre=\meta{decoration} (initially lineto)}
  This key sets a decoration that should be used before the main
  decoration starts. The \meta{decoration} will be used for a length
  of |pre length|, which |0pt| by default. Thus, for the |pre| option
  to have any effect, you also need to set the |pre length| option.
\begin{codeexample}[]
\begin{tikzpicture}
\tikz [decoration={zigzag,pre=lineto,pre length=1cm}]
  \draw [decorate] (0,0) -- (2,1) arc (90:0:1);
\end{tikzpicture}
\end{codeexample}
\begin{codeexample}[]
\begin{tikzpicture}
\tikz [decoration={zigzag,pre=moveto,pre length=1cm}]
  \draw [decorate] (0,0) -- (2,1) arc (90:0:1);
\end{tikzpicture}
\end{codeexample}
\begin{codeexample}[]
\begin{tikzpicture}
\tikz [decoration={zigzag,pre=crosses,pre length=1cm}]
  \draw [decorate] (0,0) -- (2,1) arc (90:0:1);
\end{tikzpicture}
\end{codeexample}

  Note that the default |pre| option is |lineto|, not |curveto|. This
  means that the default |pre| decoration will not follow curves (for
  efficiency reasons). Change the |pre| key to |curveto| if you have a
  curved path. 
\begin{codeexample}[]
\begin{tikzpicture}
\tikz [decoration={zigzag,pre length=3cm}]
  \draw [decorate] (0,0) -- (2,1) arc (90:0:1);
\end{tikzpicture}
\end{codeexample}
\begin{codeexample}[]
\begin{tikzpicture}
\tikz [decoration={zigzag,pre=curveto,pre length=3cm}]
  \draw [decorate] (0,0) -- (2,1) arc (90:0:1);
\end{tikzpicture}
\end{codeexample}
\end{key}

\begin{key}{/pgf/decoration/pre length=\meta{dimension} (initially 0pt)}
  This key sets the distance along which the pre-decoration should be
  used. If you do not need/wish a pre-decoration, set this key to
  |0pt| (exactly this string, not just to something that evaluated to
  the same things such as |0cm|).
\end{key}

\begin{key}{/pgf/decorations/post=\meta{decoration} (initially
    lineto)}
  Works like |pre|, only for the end of the decoration.  
\end{key}

\begin{key}{/pgf/decorations/post length=\meta{dimension} (initially
    0pt)}
  Works like |pre length|, only for the end of the decoration.  
\end{key}

Here is a typical example that shows how these keys can be used:

\begin{codeexample}[]
\begin{tikzpicture}
  [decoration=snake,
   line around/.style={decoration={pre length=#1,post length=#1}}]

  \draw[->,decorate]                  (0,0)    -- ++(3,0);
  \draw[->,decorate,line around=5pt]  (0,-5mm) -- ++(3,0);
  \draw[->,decorate,line around=1cm]  (0,-1cm) -- ++(3,0);
\end{tikzpicture}
\end{codeexample}



\endinput

\subsection{Snakes}

\label{section-tikz-snakes}

To be revised...


The line-to operation can not only be used to append straight lines to
the path, but also ``snaked'' lines like this one:
\tikz\draw[snake=snake] (0,0) -- (1,0);. They are called ``snakes''
because they look a little bit like snakes seen from above.

\tikzname\ and \pgfname\ use a concept that I termed \emph{snakes}
for appending such ``squiggly'' lines. A snake specifies a way of
extending a path between two points in a ``fancy manner.''

Normally, a snake will just connect the start point to the end point
without starting new subpaths. Thus, a path containing a snaked line
can, nevetheless, still be used for filling. However, this is not
always the case. Some snakes consist of numerous unconnected
segments. ``Lines'' consisting of such snakes cannot be used as the
borders of enclosed areas.

Here are some examples of snakes in action:

\begin{codeexample}[]
\begin{tikzpicture}[thick]
  \draw                                        (0,3)   -- (3,3);
  \draw[snake=zigzag]                          (0,2.5) -- (3,2.5);
  \draw[snake=brace]                           (0,2)   -- (3,2);
  \draw[snake=triangles]                       (0,1.5) -- (3,1.5);
  \draw[snake=coil,segment length=4pt]         (0,1)   -- (3,1);
  \draw[snake=coil,segment aspect=0]           (0,.5)  -- (3,.5);
  \draw[snake=expanding waves,segment angle=7] (0,0)   -- (3,0);
\end{tikzpicture}
\end{codeexample}

\begin{codeexample}[]
\begin{tikzpicture}
  \filldraw[fill=red!20,snake=bumps] (0,0) rectangle (3,2);
\end{tikzpicture}
\end{codeexample}

\begin{codeexample}[]
\begin{tikzpicture}
  \filldraw[fill=blue!20]              (0,3)
  [snake=saw]                       -- (3,3)
  [snake=coil,segment aspect=0]     -- (2,1)
  [snake=bumps]                     -| (0,3);
\end{tikzpicture}
\end{codeexample}

\begin{codeexample}[]
\begin{tikzpicture}
  \filldraw [fill=yellow!50,
             snake=random steps,segment length=3pt,segment amplitude=1pt]
     (0,0) rectangle (3,2);
  \node at (1.5,1) {Saved from trash};
\end{tikzpicture}
\end{codeexample}

\begin{codeexample}[]
\begin{tikzpicture}
  \shade [left color=green,right color=black,
          snake=random steps,segment length=1mm,segment amplitude=3mm]
    (0,0) rectangle (3,2);
\end{tikzpicture}
% You do not want to meet this snake on a dark street...          
\end{codeexample}

No special path operation is needed to use a snake. Instead, you use
the following option to ``switch on'' snaking:

\begin{key}{/tikz/snake=\meta{snake name} (default \normalfont is scope's
  snake)}
  This option causes the snake \meta{snake name} to be used for
  subsequent line-to operations. So, whenever you use the |--| syntax
  to specify that a straight line should be added to the path, a snake
  to this path will be added instead. Snakes will also be used when
  you use the \verb!-|! and \verb!|-! syntax and also when you use the
  |rectangle| operation. Snakes will \emph{not} be used when you use
  the curve-to operation nor when any other ``curved'' line is added
  to the path.

  This option has to be given anew for each path. However, you can
  also leave out the \meta{snake name}. In this case, the enclosing
  scope's \meta{snake name} is used. Thus, you can specify a
  ``standard'' snake name for scope and then just say |\draw[snake]|
  every time this snake should actually be used.

  The \meta{snake name} |none| is special. It can be used to switch
  off snaking after it has been switched on on a path.

  A bit strangely, no valid \meta{snake names} are defined by
  \tikzname\ by default. Instead, you have to include the library
  package |pgflibrarysnakes|. This package defines numerous snakes,
  see Section~\ref{section-library-snakes} for the complete list.
\end{key}

Most snakes can be configured. For example, for a snake that looks
like a sine curve, you might wish to change the amplitude or the
frequency. There are numerous options that influence these
parameters. Not all options apply to all snakes, see
Section~\ref{section-library-snakes} once more for details.

\begin{key}{/tikz/gap before snakes=\meta{dimension}}
  This option allows you to add a certain ``gap'' to the snake at its
  beginning. The snake will not start at the current point; instead
  the start point of the snake is move be \meta{dimension} in the
  direction of the target.
\begin{codeexample}[]
\begin{tikzpicture}
  \draw[help lines] (0,0) grid (3,2);
  \draw[snake=zigzag]                      (0,1) -- ++(3,1);
  \draw[snake=zigzag,gap before snake=1cm] (0,0) -- ++(3,1);
\end{tikzpicture}
\end{codeexample}
\end{key}

\begin{key}{/tikz/gap after snake=\meta{dimension}}
  This option has the same effect as |gap before snake|, only it
  affects the end of the snake, which will ``end early.''
\end{key}
\begin{key}{/tikz/gap around snake=\meta{dimension}}
  This option sets the gap before and after the gap to
  \meta{dimension}. 
\begin{codeexample}[]
\begin{tikzpicture}
  \draw[help lines] (0,0) grid (3,2);
  \draw[snake=brace]                      (0,1) -- ++(3,1);
  \draw[snake=brace,gap around snake=5mm] (0,0) -- ++(3,1);
\end{tikzpicture}
\end{codeexample}
\end{key}
\begin{key}{/tikz/line before snake=\meta{dimension}}
  This option works like |gap before snake|, only it will connect the
  current point with a straight line to the start of the snake.
\begin{codeexample}[]
\begin{tikzpicture}
  \draw[help lines] (0,0) grid (3,2);
  \draw[snake=zigzag]                       (0,1) -- ++(3,1);
  \draw[snake=zigzag,line before snake=1cm] (0,0) -- ++(3,1);
\end{tikzpicture}
\end{codeexample}
\end{key}
\begin{key}{/tikz/line after snake=\meta{dimension}}
  Works line |gap after snake|, only it adds a straight line.
\end{key}
\begin{key}{/tikz/line around snake=\meta{dimension}}
  Works line |gap around snake|, only it adds straight lines.
\end{key}
\begin{key}{/tikz/raise snake=\meta{dimension}}
  This option can be used with all snakes. It will offset the snake by
  ``raising'' it by \meta{dimension}. A negative \meta{dimension} will
  lower the snake. Raising and lowering is always relative to the line
  along which the snake is drawn. Here is an example:
\begin{codeexample}[]
\begin{tikzpicture}
  \node (a) {A};
  \node (b) at (2,1) {B};
  \draw                                  (a) -- (b);
  \draw[snake=brace]                     (a) -- (b);
  \draw[snake=brace,raise snake=5pt,red] (a) -- (b);
\end{tikzpicture}
\end{codeexample}
\end{key}
\begin{key}{/tikz/mirror snake}
  This option causes the snake to be ``reflected along the path.''
  This is best understood by looking at an example:
\begin{codeexample}[]
\begin{tikzpicture}
  \node (a) {A};
  \node (b) at (2,1) {B};
  \draw                                     (a) -- (b);
  \draw[snake=brace]                        (a) -- (b);
  \draw[snake=brace,mirror snake,red,thick] (a) -- (b);
\end{tikzpicture}
\end{codeexample}
  This option can be used with every snake and can be combined with
  the |raise snake| option.
\end{key}


\begin{key}{/pgf/segment amplitude=\meta{dimension} (initially 2.5pt)}
  \keyalias{tikz}  
  This option sets the ``amplitude'' of the snake. For a snake that is
  a sine wave this would be the amplitude of this line. For other
  snakes this value typically describes how far the snakes ``rises
  above'' or ``falls below'' the path. For some snakes, this value is
  ignored. 
\begin{codeexample}[]
\begin{tikzpicture}
  \node (a) {A}   node (b) at (2,1) {B}  node (c) at (2,-1) {C};
  \draw[snake=zigzag]                                 (a) -- (b);
  \draw[snake=zigzag,segment amplitude=5pt,red,thick] (a) -- (c);
\end{tikzpicture}
\end{codeexample}
\end{key}

\begin{key}{/pgf/segment length=\meta{dimension} (initially 10pt)}
  \keyalias{tikz}
  This option sets the length of each ``segment'' of a snake. For a
  sine wave this would be the wave length, for other snakes it is the
  length of each ``repetitive part'' of the snake.
\begin{codeexample}[]
\begin{tikzpicture}
  \node (a) {A}   node (b) at (2,1) {B}  node (c) at (2,-1) {C};
  \draw[snake=zigzag]                               (a) -- (b);
  \draw[snake=zigzag,segment length=20pt,red,thick] (a) -- (c);
\end{tikzpicture}
\end{codeexample}
\begin{codeexample}[]
\begin{tikzpicture}
  \node (a) {A}   node (b) at (2,1) {B}  node (c) at (2,-1) {C};
  \draw[snake=bumps]                               (a) -- (b);
  \draw[snake=bumps,segment length=20pt,red,thick] (a) -- (c);
\end{tikzpicture}
\end{codeexample}
\end{key}

\begin{key}{/pgf/segment object length=\meta{dimension} (initially
    \normalfont same as |/pgf/segment amplitude|)}
  \keyalias{tikz}
  This option sets the length of the objects inside each segment of a
  snake. This option is only used for snakes in which each segment
  contains an object like a triangle or a star. 
\begin{codeexample}[]
\begin{tikzpicture}
  \node (a) {A}   node (b) at (2,1) {B}  node (c) at (2,-1) {C};
  \draw[snake=triangles]                                     (a) -- (b);
  \draw[snake=triangles,segment object length=8pt,red,thick] (a) -- (c);
\end{tikzpicture}
\end{codeexample}
\end{key}

\begin{key}{/pgf/segment angle=\meta{degrees} (initially 45)}
  \keyalias{tikz}
  This option sets an angle that is interpreted in a snake-specific
  way. For example, the |waves| and |expanding waves| snakes interpret
  this as (half the) opening angle of the wave. The |border| snake
  uses this value for the angle of the little ticks.
\begin{codeexample}[]
\begin{tikzpicture}[segment amplitude=10pt]
  \node (a) {A}   node (b) at (2,0) {B};
  \draw[snake=border]                            (a) -- (b);
  \draw[snake=border,segment angle=20,red,thick] (a) -- (b);
\end{tikzpicture}
\end{codeexample}
\begin{codeexample}[]
\begin{tikzpicture}[segment amplitude=10pt]
  \node (a)            {A}   node (b)  at (2,0)  {B};
  \node (a') at (0,-1) {A}   node (b') at (2,-1) {B};
  \draw[snake=expanding waves]                            (a)  -- (b);
  \draw[snake=expanding waves,segment angle=20,red,thick] (a') -- (b');
\end{tikzpicture}
\end{codeexample}
\end{key}

\begin{key}{/pgf/segment aspect=\meta{ratio} (initially 0.5)}
  \keyalias{tikz}
  This option sets an aspect ratio that is interpreted in a
  snake-specific way. For example, for the |coils| snake this
  describes the ``direction'' from which the coil is viewed.
\begin{codeexample}[]
\begin{tikzpicture}[segment amplitude=5pt,segment length=5pt]
  \node (a) {A}   node (b) at (2,1) {B}  node (c) at (2,-1) {C};
  \draw[snake=coil]                            (a) -- (b);
  \draw[snake=coil,segment aspect=0,red,thick] (a) -- (c);
\end{tikzpicture}
\end{codeexample}
\end{key}

It is possible to define new snakes, but this cannot be done inside
\tikzname. You need to use the command |\pgfdeclaresnake| from the
basic level directly, see Section~\ref{section-base-snakes}.

The following styles define combinations of segment settings that may
be useful:
\begin{stylekey}{/tikz/snake triangles 45}
  Installs a snake the consists of little triangles with an opening
  angle of $45^\circ$.
\end{stylekey}

\begin{stylekey}{/tikz/snake triangles 60}
  Installs a snake the consists of little triangles with an opening
  angle of $60^\circ$.
\end{stylekey}

\begin{stylekey}{/tikz/snake triangles 90}
  Installs a snake the consists of little triangles with an opening
  angle of $90^\circ$.
\end{stylekey}




\subsection{Decorations}

This will be the main text of this section.


\begin{tikzlibrary}{decorations}
  This library enables decorations to be used in \tikzname\ and also
  installs a number of decorations, see
  Section~\ref{section-library-decorations} for a complete list.
\end{tikzlibrary}

In order to decorate (part of) a path, you use the following path
command:

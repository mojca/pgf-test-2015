% Copyright 2003 by Till Tantau <tantau@cs.tu-berlin.de>.
%
% This program can be redistributed and/or modified under the terms
% of the LaTeX Project Public License Distributed from CTAN
% archives in directory macros/latex/base/lppl.txt.




\section{Automata Drawing Library}

\begin{package}{pgflibrarytikzautomata}
  This packages provides shapes and styles for drawing finite state
  automata. 
\end{package}


\subsection{Drawing Automata}

The automata drawing library is intended to make it easy to draw
finite automata. It does not cover every imaginable situation, but
most finite automata found in text books can be drawn in a nice and
convenient fashion using this library.

To draw an automaton, proceed as follows:
\begin{enumerate}
\item For each state of the automaton, there should be one node with
  the option |state|.
\item To place the states, you can either use absolute positions or
  relative positions, using options like |above of| or |right of|.
\item Give a unique name to each state node.
\item Accepting and initial states are indicated by adding the
  options |accepting| and |initial|, respectively, to the state
  nodes.
\item Once the states are fixed, the edges can be added. For this, the
  |edge| operation is most useful. It is, however, also possible to
  add edges after each node has been placed.
\item For loops, use the |edge [loop]| operation.
\end{enumerate}

Let us now see how this works for a real example. Let us consider a
nondeterminsitic four state automaton that checks whether an contains
the sequence $0^*1$ or the sequence $1^*0$. 
\begin{codeexample}[]
\begin{tikzpicture}[shorten >=1pt,node distance=2cm,auto]
  \draw[help lines] (0,0) grid (3,2);

  \node[state,initial]  (q_0)                      {$q_0$};
  \node[state]          (q_1) [above right of=q_0] {$q_1$};
  \node[state]          (q_2) [below right of=q_0] {$q_2$};
  \node[state,accepting](q_3) [below right of=q_1] {$q_3$};

  \path[->] (q_0) edge              node        {0} (q_1)
                  edge              node [swap] {1} (q_2)
            (q_1) edge              node        {1} (q_3)
                  edge [loop above] node        {0} ()
            (q_2) edge              node [swap] {0} (q_3)
                  edge [loop below] node        {1} ();
\end{tikzpicture}
\end{codeexample}


\subsection{States With and Without Output}

The |state| style actually just ``selects'' a default underlying
style. Thus, you can define multiple new complicated state style and
then simply set the |state| style to your given style to get the
desired kind of styles.

By default, the following state styles are defined:
\begin{itemize}
  \itemstyle{state without output}
  This node style causes nodes to be drawn circles. Also, this style
  calls |every state|.
  \itemstyle{state with output}
  This node style causes nodes to be drawn as split circles, that is,
  using the |circle split| shape. In the upper part of the shape you
  have the name of the style, in the lower part the output is
  placed. To specify the output, use the command |\nodepart{lower}|
  inside the node. This style also calls |every state|.
\begin{codeexample}[]
\begin{tikzpicture}
  \draw[help lines] (0,0) grid (3,2);

  \node[state without output] {$q_0$};
  
  \node[state with output] at (2,0) {$q_1$ \nodepart{lower} $00$};
\end{tikzpicture}
\end{codeexample}

  \itemstyle{state}
  This style is set to |state without output| by default. You should
  redefine it to something else, if you wish to use states of a
  different nature.
\begin{codeexample}[]
\begin{tikzpicture}
  \tikzstyle{state}=[state with output]
  \node[state]          {$q_0$ \nodepart{lower} $11$};
  \node[state] at (2,0) {$q_1$ \nodepart{lower} $00$};
\end{tikzpicture}
\end{codeexample}

  \itemstyle{every state}
  This style is used by |state with output| and also by
  |state without output|. By default, it does nothing, but you can use
  it to make your state look more fancy:
\begin{codeexample}[]
\begin{tikzpicture}[shorten >=1pt,node distance=2cm,>=stealth']
  \tikzstyle{every state}=[draw=blue!50,very thick,fill=blue!20]

  \node[state,initial]  (q_0)                      {$q_0$};
  \node[state]          (q_1) [above right of=q_0] {$q_1$};
  \node[state]          (q_2) [below right of=q_0] {$q_2$};

  \path[->] (q_0) edge              node [above left]  {0} (q_1)
                  edge              node [below left]  {1} (q_2)
            (q_1) edge [loop above] node               {0} ()
            (q_2) edge [loop below] node               {1} ();
\end{tikzpicture}
\end{codeexample}
\end{itemize}


\subsection{Initial and Accepting States}

The styles |initial| and |accepting| are similar to the |state| style
as they also just select an ``underlying'' style, which installs the
actual settings for initial and accepting states.

For initial states, there is only one predefined way of drawing them,
so the two-stage mechanism is not really necessary, but, perhaps,
I will find another way of drawing them in the literature some time.

Let us start with the initial states.
\begin{itemize}
  \itemstyle{initial}
  This style simply selects |initial by arrow|.
  \itemstyle{initial by arrow}
  This style causes an arrow and, possibly, some text to be added to
  the node. The arrow points from the text to the node. The node text
  and the direction and the distance can be set using the following
  options:
  \begin{itemize}
  \itemoption{initial text}|=|\meta{text} sets the text to be
  used. Use an empty text to suppress all text. The default is
  |start|. 
  \itemoption{initial where}|=|\meta{direction} set the place where
  the text should be shown. Allowed values are |above|, |below|,
  |left|, and |right|.
  \itemoption{intial distance}|=|\meta{distance} is the length of the
  arrow leading from the text to the state node.
  \end{itemize}
  \itemstyle{every initial by arrow}
  This style is executed at the beginning of every path that contains
  the arrow and the text. You can use it to, say, make the text red or
  whatever. 
\begin{codeexample}[]
\begin{tikzpicture}
  \tikzstyle{every initial by arrow}=[text=red,->>]
  \node[state,initial,initial distance=2cm] {$q_0$};
\end{tikzpicture}
\end{codeexample}
  \itemstyle{initial above} is a shorthand for
  |initial by arrow,initial where=above|
  \itemstyle{initial below} works similarly to the previous option.
  \itemstyle{initial left} works similarly to the previous option.
  \itemstyle{initial right} works similarly to the previous option.
\end{itemize}

For the accepting states, the sitation is similar: There is also an
|accepting| style that selects the way accepting states are
rendered. However, there are now two options: First,
|accepting by arrow|, which works the same way as |initial by arrow|,
only with the direction of arrow reversed, and |accepting by double|,
where accepting states get a double line around them.

\begin{itemize}
  \itemstyle{accepting}
  This style selects |accepting by double| by default. You can replace
  this by the style |accepting by arrow| to get accepting states with
  an arrow leaving them.
  \itemstyle{accepting by double}
  Specifies that the node should get a double line around it.
  \itemstyle{accepting by arrow}
  This style causes an arrow and, possibly, some text to be added to
  the node. The arrow points to the text from the node.
  \begin{itemize}
  \itemoption{accepting text}|=|\meta{text} sets the text to be
  used. Use an empty text to suppress all text. This is the default.
  \itemoption{accepting where}|=|\meta{direction} set the place where
  the text should be shown. Allowed values are |above|, |below|,
  |left|, and |right|.
  \itemoption{intial distance}|=|\meta{distance} is the length of the
  arrow.
  \end{itemize}
\begin{codeexample}[]
\begin{tikzpicture}
  [shorten >=1pt,node distance=2cm,>=stealth',initial text=]

  \tikzstyle{every state}=[draw=blue!50,very thick,fill=blue!20]
  \tikzstyle{accepting}=[accepting by arrow]

  \node[state,initial]  (q_0)                      {$q_0$};
  \node[state]          (q_1) [above right of=q_0] {$q_1$};
  \node[state]          (q_2) [below right of=q_0] {$q_2$};
  \node[state,accepting](q_3) [below right of=q_1] {$q_3$};

  \path[->] (q_0) edge              node [above left]  {0} (q_1)
                  edge              node [below left]  {1} (q_2)
            (q_1) edge              node [above right] {1} (q_3)
                  edge [loop above] node               {0} ()
            (q_2) edge              node [below right] {0} (q_3)
                  edge [loop below] node               {1} ();
\end{tikzpicture}
\end{codeexample}
  \itemstyle{every accepting by arrow}
  This style is executed at the beginning of every path that contains
  the arrow and the text.
  \itemstyle{accepting above} is a shorthand for
  |accepting by arrow,accepting where=above|
  \itemstyle{accepting below} works similarly to the previous option.
  \itemstyle{accepting left} works similarly to the previous option.
  \itemstyle{accepting right} works similarly to the previous option.
\end{itemize}

As a final example, let us again typeset the automaton. This time, we
use the following rule for accepting/initial state: Initial states are
red, accepting states are green, and normal states are orange. Then,
we must find a path from a red state to a green state.

\begin{codeexample}[]
\begin{tikzpicture}[shorten >=1pt,node distance=2cm,>=stealth',thick]
  \tikzstyle{every state}=[fill,draw=none,orange,text=white]
  \tikzstyle{accepting}=[green!50!black,text=white]
  \tikzstyle{initial}=  [red,text=white]

  \node[state,initial]  (q_0)                      {$q_0$};
  \node[state]          (q_1) [above right of=q_0] {$q_1$};
  \node[state]          (q_2) [below right of=q_0] {$q_2$};
  \node[state,accepting](q_3) [below right of=q_1] {$q_3$};

  \path[->] (q_0) edge              node [above left]  {0} (q_1)
                  edge              node [below left]  {1} (q_2)
            (q_1) edge              node [above right] {1} (q_3)
                  edge [loop above] node               {0} ()
            (q_2) edge              node [below right] {0} (q_3)
                  edge [loop below] node               {1} ();
\end{tikzpicture}
\end{codeexample}


%%% Local Variables: 
%%% mode: latex
%%% TeX-master: "pgfmanual-pdftex-version"
%%% End: 

% Copyright 2010 by Renée Ahrens, Olof Frahm, Jens Kluttig, Matthias Schulz, Stephan Schuster
% Copyright 2011 by Till Tantau
% Copyright 2011 by Jannis Pohlmann
%
% This file may be distributed and/or modified
%
% 1. under the LaTeX Project Public License and/or
% 2. under the GNU Free Documentation License.
%
% See the file doc/generic/pgf/licenses/LICENSE for more details.


\section{Implementing Graph Drawing Algorithms: The Lua Side}
\label{section-gd-implementation-first}

\noindent{\emph{by Till Tantau}}
\label{section-gd-own-algorithm}
\label{section-library-graphdrawing-ownAlgorithm}

File status: I'm working on it...


\subsection{Overview}

The present section is the first of three sections addressed at
readers interested in implementing new graph drawing
algorithms. Obviously, in order to do so, you need to have an
algorithm in mind and also some programming skills; but no deep
knowledge of \TeX\ will be needed. All graph drawing algorithms can
and must be implemented entirely in the Lua programming language,
which is a small, easy-to-learn (and quite beautiful) language
integrated into current versions of \TeX. The present section focuses
on the Lua side of the graph drawing engine; the next section
addresses the interface between this Lua side and \TeX\ side (most
readers will wish to skip this section); the last of the three
sections presents complete example code of simple graph drawing
algorithms.  

In the following, after a small ``hello world'' example of graph
drawing and a discussion of technical details like
how to name files so that \TeX\ will find them, we have a look at the
main parts of the graph drawing engine:

\begin{itemize}
\item Section~\ref{section-gd-models} explains the object-oriented
  model of graphs used throughout the graph drawing engine. Graph
  drawing algorithms do not get the ``raw'' specification used by the
  user to specify a graph (like |{a -> {b,c}}| in the |graph|
  syntax). Instead, what a graph drawing algorithm sees is ``just'' a
  graph object that provides methods for accessing the vertices and
  arcs.
\item Section~\ref{section-gd-gd-scope} explores what the graph
  drawing scope ``looks like on the Lua side.'' As the graph of a
  graph drawing scope is being parsed, a lot of information is
  gathered: The nodes and edges of the graph are identified and the
  object-oriented model is built, but other information is also
  collected. For instance, a sequence of \emph{events} is created
  during the parsing process. As another example, numerous kinds of
  \emph{collections} may be identified by the parser. The parsed graph
  together with the event sequence and the collections are all
  gathered in a single table, called the \emph{scope table} of the
  current graph drawing scope. Algorithms can access this table to
  retrieve information that goes beyond the ``pure'' graph model.

  On entry in this table is of particular importance: The
  \emph{syntactic digraph.} While most graph drawing 
  algorithms are not really interested in the ``details'' of how a
  graph was specified, for some algorithms it makes a big difference
  whether you write |a -> b| or |b <- a| in your specification of the
  graph. These algorithms can access the ``fine details'' of how the
  input graph was specified through the syntactic digraph; all other
  algorithms can access their |digraph| or |ugraph| fields and do not
  have to worry about the difference between |a -> b| and |b <- a|.
\item Section~\ref{section-gd-transformations} explains how the
  information in the graph drawing scope is processed. One might
  expect that we simply run the algorithm selected by the user;
  however, things are more involved in practice. When the layout of a
  graph needs to be computed, only very few algorithms will actually
  be able to compute positions for the nodes of \emph{every}
  graph. For instance, most algorithms implicitly assume that the
  input graph is connected; algorithms for computing layouts for trees
  assume that the input is, well, a tree; and so on. For this reason,
  graph drawing algorithms will not actually need the original input
  graph as their input, but some \emph{transformed} version of
  it. Indeed, \emph{all} graph drawing algorithms are treated as graph 
  transformations by the graph drawing engine.

  This section explains how transformations are chosen and which
  transformations are applied by default.
\item Finally, Section~\ref{section-gd-libs} documents the different
  libraries functions that come with the graph drawing engine. For
  instance, there are library functions for computing the (path)
  distance of nodes in a graph; a parameter that is needed by some
  algorithms. 
\end{itemize}



\subsection{Getting Started}

In this section, a ``hello world'' example of a graph
drawing algorithm  is given, followed by an overview of the
organization of the whole engine.



\subsubsection{The Hello World of Graph Drawing}

Let us start our tour of the Lua side of \tikzname's graph drawing
engine with a ``hello world'' version of graph drawing: An algorithm
that simply places all nodes of a graph in a 
circle of a fixed radius. Naturally, this is not a particularly
impressive or intelligent graph drawing algorithm; but neither is the
classical ``hello world''\dots\ Here is the complete code:

\begin{codeexample}[code only]
pgf.gd.interface.InterfaceToAlgorithms.declare {
  key = "very simple demo layout",
  algorithm = {
    run =
      function (self)
        local alpha = (2 * math.pi) / #self.ugraph.vertices
        for i,vertex in ipairs(self.ugraph.vertices) do
          vertex.pos.x = math.cos(i * alpha) * 25
          vertex.pos.y = math.sin(i * alpha) * 25
        end
      end
  }
}
\end{codeexample}
\directlua{
pgf.gd.interface.InterfaceToAlgorithms.declare {
  key = "very simple demo layout",
  algorithm = {
    run =
      function (self)
        local alpha = (2 * math.pi) / \luaescapestring{#}self.ugraph.vertices
        for i,vertex in ipairs(self.ugraph.vertices) do
          vertex.pos.x = math.cos(i * alpha) * 25
          vertex.pos.y = math.sin(i * alpha) * 25
        end
      end
  }
}
}


We can employ this algorithm as follows. First, we need to somehow
execute the above code. You can either just use |\directlua| on it
(you need to put the |#| in |\luaescapestring|, however) or
put it in a file and then use |\directlua| plus |require| (a better
alternative) or you put it in a file like |simpledemo.lua| and use
|\usegdlibrary{simpledemo}| (undoubtedly the ``best'' way). Whatever
way you choose, you can now use it as follows:

\begin{codeexample}[]
\tikz [very simple demo layout]
  \graph { f -> c -> e -> a -> {b -> {c, d, f}, e -> b}};
\end{codeexample}

It turns out, that our little algorithm is already more powerful than
one might expect. Consider the following example:
\begin{codeexample}[]
\tikz [very simple demo layout, componentwise]
  \graph {
    1 -> 2 ->[orient=right] 3 -> 1;
    a -- b --[orient=45]  c -- d -- a;
  };
\end{codeexample}

Note that, in our algorithm, we ``just'' put all nodes on a circle
around the origin. Nevertheless, the graph gets decomposed into two
connected components, the components are rotated so that the edge from
node |2| to node |3| goes from left to right and the edge from |b| to
|c| goes up at an angle of $45^\circ$, and the components are placed
next to each other so that some spacing is achieved.

The ``magic'' that achieves all this behind the scenes is called 
``graph transformations.'' They will heavily pre- and postprocess the
input and output of graph drawing algorithms to achieve the above
results. 

Naturally, some algorithms may not wish their inputs and/or
outputs to be ``tampered'' with. An algorithm can easily configure
which transformations should be applied, by passing appropriate options
to |declare|.



\subsubsection{Namespaces}

All parts of the graph drawing library reside in the Lua ``namespace''
|pgf.gd|, which is itself a ``sub-namespace'' of |pgf|. For your own
algorithms, you are free to place them in whatever namespace you like;
only for the official distribution of \pgfname\ everything has been
put into the correct namespace.

Let us now have a more detailed look at these namespaces. A namespace
is just a Lua table, and sub-namespaces are just subtables of
namespace tables. Following the Java convention, namespaces are in
lowercase letters. The following namespaces are part of the core of
the graph drawing engine:
\begin{itemize}
\item |pgf| This namespace is the main namespace of \pgfname. Other
  parts of \pgfname\ and \tikzname\ that also employ Lua should put an
  entry into this table. Since, currently, only the graph drawing
  engine adheres to this rule, this namespace is declared inside the
  graph drawing directory, but this will change.

  The |pgf| table is the \emph{only} entry into the global table of
  Lua generated by the graph drawing engine (or, \pgfname, for that
  matter). If you intend to extend the graph drawing engine, do not
  even \emph{think} of polluting the global namespace. You will be
  fined.
\item |pgf.gd| This namespace is main namespace of the graph drawing
  engine, including the object-oriented models of graphs and the
  layout pipeline. Algorithms that are part of the
  distribution are also inside this namespace, but if you write your
  own algorithms you do not need place them inside this
  namespace. (Indeed, you probably should not.)
\item |pgf.gd.model| This namespace contains all Lua classes that are
  part of the object-oriented model of graphs employed
  throughout the graph drawing engine. For readers familiar with the
  model--view--controller pattern: This is the namespace containing
  the model-part of this pattern.
\item |pgf.gd.interface| This namespace handles, on the one hand, the
  communication between the algorithm layer and the binding layer and,
  on the other hand, the communication between the display layer
  (\tikzname) and the binding layer.
\item |pgf.gd.binding| So-called ``bindings'' between display layers
  and the graph drawing system reside in this namespace.
\item |pgf.gd.lib| Numerous useful classes that ``make an algorithm's
  your life easier'' are collected in this namespace. Examples are a
  class for decomposing a graph into connected components or a class
  for computing the ideal distance between two sibling nodes in a
  tree, taking all sorts of rotations and separation parameters into
  account. 
\item |pgf.gd.trees| This namespace contains classes that are useful
  for dealing with graphs that are trees. In particular, it contains a
  class for computing a spanning tree of an arbitrary connected graph;
  an operation that is an important preprocessing step for many
  algorithms.

  In addition to providing ``utility functions for trees,'' the
  namespace \emph{also} includes actual algorithms for computing graph
  layouts like |pgf.gd.trees.ReingoldTilford1981|. It may seem to be a
  bit of an ``impurity'' that a namespace mixes utility classes and
  ``real'' algorithms, but experience has shown that it is better to
  keep things together in this way.

  Concluding the analogy to the model--view--controller pattern, a
  graph drawing algorithm is, in a loose sense, the ``view'' part of
  the pattern.
\item |pgf.gd.layered| This namespace provides classes and functions
  for ``layered'' layouts; the Sugiyama layout method being the most
  well-known one. Again, the namespace contains both algorithms to be
  used by a user and utility functions.
\item |pgf.gd.force| Collects force-based algorithms and, again, also
  utility functions and classes.
\item |pgf.gd.examples| Contains some example algorithms. They are
  \emph{not} intended to be used directly, rather they should serve as
  inspirations for readers wishing to implement their own algorithms.
\end{itemize}

There are further namespaces that also reside in the |pgf.gd|
namespace, these namespaces are used to organize different graph
drawing algorithms into categories.


\subsubsection{Defining and Using Namespaces and Classes}

There are a number of rules concerning the structure and naming of
namespaces as well as the naming of files. Let us start with the
rules for naming namespaces, classes, and functions. They follow the
``Java convention'':

\begin{enumerate}
\item A namespace is a short lowercase |word|.
\item A function in a namespace is in |lowercase_with_underscores_between_words|.
\item A class name is in |CamelCaseWithAnUppercaseFirstLetter|.
\item A class method name is in |camelCaseWithALowercaseFirstLetter|.
\end{enumerate}

From Lua's point of view, every namespace and every class is just a
table. However, since these tables will be loaded using Lua's
|require| function, each namespace and each class must be placed
inside a separate file (unless you modify the |package.loaded| table,
but, then, you know what you are doing anyway). Inside such a file, you
should first declare a local variable whose name is the name of the
namespace or class that you intend to define and then assign a
(possibly empty) table to this variable:
\begin{codeexample}[code only]
-- File pgf.gd.example.SomeClass.lua:
local SomeClass = {}
\end{codeexample}
Next, you should add your class to the encompassing namespace. This is
achieved as follows:
\begin{codeexample}[code only]
require("pgf.gd.example").SomeClass = SomeClass
\end{codeexample}
The reason this works is that the |require| will return the table that
is the namespace |pgf.gd.example|. So, inside this namespace, the
|SomeClass| field will be filled with the table stored in the local
variable of the same name -- which happens to be the table
representing the class.

At the end of the file, you must write
\begin{codeexample}[code only]
return SomeClass  
\end{codeexample}
This ensures that the table that is defined in this file gets stored
by Lua in the right places. Note that you need and should not use
Lua's |module| command. The reason is that this command has
disappeared in the new version of Lua and that it is not really
needed. 

Users of your class can import and use your class by writing:
\begin{codeexample}[code only]
...
local SomeClass = require "pgf.gd.examples.SomeClass"
...  
\end{codeexample}


\subsubsection{File Names of Lua Classes and Namespaces}

In Lua, similarly to Java, when a class |SomeClass| is part of, say,
the namespace |pgf.gd.example|, it is customary to put the class's
code in a file |SomeClass.lua| and then put this class in a directory
|example|, that is a subdirectory of a directory |gd|, which is in
turn a subdirectory of a directory |pgf|. When you write
\texttt{require "pgf.gd.example.SomeClass"} it is the job of a
so-called \emph{loader} to turn this into a request for the file
\texttt{pgf/gd/example/SomeClass.lua} (for Unix systems).

Unfortunately, this approach does not work well with (Lua)\TeX\ due to
the way \TeX\ looks for files. Although these days a typical \TeX\
installation consists of thousands of files scattered over hundreds of
sub-sub-sub-directories, from \TeX's point of view the directory
structure is in some sense irrelevant: If there are two files
|foo.lua| in two different directories, there is really no way of
knowing which one will be used when you request it; trying to prefix a
file request with a path prefix in a portable way is difficult to
impossible, especially when different |texmf| trees are involved.

For these reasons the files of the graph drawing engine are named and
organized in a somehwat redundant way: A class like
|pgf.gd.examples.SomeClass| will be placed in a file called
\texttt{pgf.gd.examples.SomeClass.\penalty0lua} (so the complete namespace path is
part of the file name). \emph{Additionally}, the file is placed in a
directory path is named according to the ``usual'' way, namely
|pgf/gd/examples/|. Note, however, that the file could just as well
reside in any other directory that is part of some |texmf| tree.

The bottom line of all this is: Filenames of Lua files of the graph
drawing engine contain the complete namespace path and, additionally,
they are placed in a directory path also representing the complete
namespace path. 


\subsection{The Interface To Algorithms}

\includeluadocumentationof{pgf.gd.interface.InterfaceToAlgorithms}


\subsection{The Model Classes}

\label{section-gd-models}

All that a graph drawing algorithm will ``see'' of the graph specified
by the user is a ``graph object.'' Such an object is an
object-oriented model of the user's graph that no longer encodes the
specific way in which the user specified the graph; it only encodes
which nodes and edges are present. For instance, the graph
specification 
\begin{codeexample}[code only]
graph { a -- {b, c} }
\end{codeexample}
\noindent and the graph specification
\begin{codeexample}[code only]
node (a) { a }
child { node (b) {b} }
child { node (c) {c} }
\end{codeexample}
will generate exactly the same graph object.

\begin{luanamespace}{pgf.gd.}{model}
  This namespace contains the classes modeling graphs,
  nodes, and edges. Also, the |Coordinate| class is found here, since
  coordinates are also part of the modeling.  
\end{luanamespace}

\subsubsection{Directed Graphs (Digraphs)}

Inside the graph drawing engine, the only model of a graph that is 
available treats graphs as
\begin{enumerate}
\item directed (all edges have a designated head and a designated
  tail) and
\item simple (there can be at most one edge between any pair of
  nodes). 
\end{enumerate}
These two properties may appear to be somewhat at odds with what users
can specify as graphs and with what some graph drawing algorithms
might expect as input. For instance, suppose a user writes
\begin{codeexample}[code only]
graph { a -- b --[red] c, b --[green, bend right] c }
\end{codeexample}
In this case, it seems that the input graph for a graph drawing
algorithm should actually be a \emph{undirected} graph in which there
are \emph{multiple} edges (namely $2$) between |b| and~|c|.
Nevertheless, the graph drawing engine will turn the user's input a
directed simple graph in ways described later. You do not need to
worry that information gets lost during this process: The
\emph{syntactic digraph,} which is available to graph drawing
algorithms on request, stores all the information about which edges
are present in the original input graph.

The main reason for only considering directed, simple graphs is speed
and simplicity: The implementation of these graphs has been optimized so
that all operations on these graphs have a guaranteed running time
that is small in practice. 

\includeluadocumentationof{pgf.gd.model.Digraph}

\subsubsection{Vertices}

\includeluadocumentationof{pgf.gd.model.Vertex}

\subsubsection{Arcs}
\label{section-gd-arc-model}

\includeluadocumentationof{pgf.gd.model.Arc}

\subsubsection{Collections}

\includeluadocumentationof{pgf.gd.model.Collection}

\subsubsection{Coordinates and Transformations}

\includeluadocumentationof{pgf.gd.model.Coordinate}
\includeluadocumentationof{pgf.gd.lib.Transform}

\subsubsection{Options and Data Storages for Vertices, Arcs, and Digraphs}

\includeluadocumentationof{pgf.gd.lib.Storage}



\subsection{The Graph Drawing Scope}

\label{section-gd-gd-scope}

\includeluadocumentationof{pgf.gd.interface.Scope}


\subsubsection{The Syntactic Digraph}

\label{section-gd-syntactic-digraph}

\subsubsection{Accessing the Syntactic Digraph: Finding Nodes and Edges}

\subsubsection{Accessing the Syntactic Digraph: Node Parameters}

\subsubsection{Accessing the Syntactic Digraph: Graph Parameters}

\subsubsection{Accessing the Syntactic Digraph: Edge Parameters}

\subsubsection{Modifying the Syntactic Digraph: Generating Options}

\subsubsection{Modifying the Syntactic Digraph: Adding Edges}

\subsubsection{Modifying the Syntactic Digraph: Adding Nodes}

\subsubsection{Events}

\label{section-gd-events}

\includeluadocumentationof{pgf.gd.lib.Event}



\subsection{Graph Transformations}

\label{section-gd-transformations}

\includeluadocumentationof{pgf.gd.control.LayoutPipeline}



\subsection{Support Libraries}

\label{section-gd-libs}

The present section lists a number of general-purpose libraries that
are used by different algorithms.

\subsubsection{Basic Functions}

\includeluadocumentationof{pgf}

\includeluadocumentationof{pgf.gd.lib}

\subsubsection{Lookup Tables}

\includeluadocumentationof{pgf.gd.lib.LookupTable}

\subsubsection{Computing Distances in Graphs}

\emph{Still needs to be ported to digraph classes!}

%\includeluadocumentationof{pgf.gd.lib.PathLengths}

\subsubsection{Priority Queues}

\includeluadocumentationof{pgf.gd.lib.PriorityQueue}



% Copyright 2010 by Renée Ahrens, Olof Frahm, Jens Kluttig, Matthias Schulz, Stephan Schuster
% Copyright 2011 by Till Tantau
% Copyright 2011 by Jannis Pohlmann
%
% This file may be distributed and/or modified
%
% 1. under the LaTeX Project Public License and/or
% 2. under the GNU Free Documentation License.
%
% See the file doc/generic/pgf/licenses/LICENSE for more details.


\section{Implementing Graph Drawing Algorithms}
\noindent{\emph{by Till Tantau and Ren\'ee Ahrens, Olof-Joachim
      Frahm, Jens Kluttig, Jannis Pohlmann, Matthias Schulz, Stephan
      Schuster}}
\label{section-gd-own-algorithm}
\label{section-library-graphdrawing-ownAlgorithm}

File status: Needs to be partly rewritten (many things changed) and
class descriptions need yet to be written

\subsection{A First Example}

This section presents a simple example of how a graph drawing
algorithm can be implemented. For each graph drawing algorithm
there must be a class of the name given to the |algorithm| key. This
class should usually reside in a file called
|pgfgd-algorithm-|\meta{algorithm name}. This class must provide (at
least) the two methods |new| and |run|. Each time a layout needs to
be computed for a graph, a new object of this algorithm class is
instantiated using the class's |new| method. For the newly created
object, an attribute |graph| will be set to an object representing the
graph. Then, the |constructor| method of the object is called,
provided it exists. Then, the |run| method is called, which should do
the actual work. (The separation into a constructor and a run method
is purely for convenience.) The |run| method should modify the
coordinates of the nodes of its |graph| attribute.

To simplify the creating of classes and constructors, the graph
drawing engine provides the function |pgf.gd.new_algorithm_class|, which
takes a table of infos about the algorithm as input and will create a
class and a |new| method for you.

As a complete example, the following code fragment implements a
trivial graph drawing algorithm that just places all nodes on a
fixed-size circle.  It is accessed with the name 
|SimpleDemo|.

\pgfgddeclareforwardedkeys{/graph drawing}{
  radius/.graph parameter=evaluate math expression,
  radius/.parameter initial=1cm,
  node radius/.node parameter=evaluate math expression
}

\begin{codeexample}[code only]
-- File SimpleDemo.lua

local SimpleDemo = pgf.gd.new_algorithm_class {}

function SimpleDemo:run()
  local radius = 28.908  -- this is 1cm in points
  local alpha = (2 * math.pi) / #self.graph.nodes
  for i=1,#self.graph.nodes do
    self.graph.nodes[i].pos.x = radius * math.cos((i-1) * alpha)
    self.graph.nodes[i].pos.y = radius * math.sin((i-1) * alpha)
  end
end

return SimpleDemo  
\end{codeexample}

The algorithm computes a circular layout like in the following.

\tikz [graph drawing scope, /graph drawing/algorithm=pgf.gd.examples.SimpleDemo]
  \graph { f -> c -> e -> a -> {b -> {c, d, f}, e -> b}};

\begin{codeexample}[code only]
\tikz [graph drawing scope, /graph drawing/algorithm=SimpleDemo]
  \graph { f -> c -> e -> a -> {b -> {c, d, f}, e -> b}};
\end{codeexample}


\subsection{Lua Layer: Overview}

All of the graph drawing engine resides in the directory
|graphdrawing| of |pgf|. Inside, there are the following
subdirectories:

\begin{itemize}
\item |tex|, with the \pgfname\ and \tikzname\ library files and
\item |lua|, with all the Lua code, including both support files and
  actual algorithms.
\end{itemize}

TODO: Details...

drawing algorithms are implemented in Lua, these directories also
contain mainly Lua files, ...

The job of this library is to make the graph
parameters of the algorithms visible to \pgfname, so this file
typically just contains calls of |\pgfgddeclarealgorithmkey| and
|\pgfgddeclareforwardedkeys|.

In the following, we first describe which steps are necessary to
implement a new graph drawing algorithm. We then have a look at the
classes made available to graph drawing algorithms by the
engine. Finally, the section concludes with a class and function
reference.



\subsection{Lua Layer: Installing Graph Drawing Algorithms}
\label{section-gd-implementing-algorithms}

In the following we describe in detail how a new graph drawing
algorithm can be implemented and installed. 


\subsubsection{Starting the Graph Drawing Engine}

First, before any graph drawing algorithms can be used, the graph
drawing engine needs to be loaded. This is done by loading the
\pgfname\ library |pgflibrarygraphdrawing|. This will
initialise the Lua graph drawing subsystem by invoking the Lua loader
class.   

In the most basic cases, no further \TeX\ code needs to be written to
use a new graph drawing algorithm; but we will see later on, that a
small entry in an appropriate \pgfname\ library of graph drawing
algorithms will make the use of the algorithm somewhat simpler.


\subsubsection{Main File of the Graph Drawing Algorithm}

As indicated at in the description of the |algorithm| key on
page~\ref{section-gd-algorithm-key}, each graph drawing algorithm  
must be implemented in a class. The class name will be the value
passed to the |algorithm| key, albeit without any spaces. This class
is a normal class in the sense of Lua and it will be loaded by calling
|require| on the value. The class file must return the class.

During |\pgfgdendscope|, the algorithm's class will be loaded, if
necessary. Then, the following things happen, in turn, in normal
operation mode: (Let \meta{class} by the class name.)

\begin{enumerate}
\item Some preprocessing is done, if the ``static'' attributes of the
  class specify this.
\item A new instance of the class is created by calling
  \meta{class}|:new(graph)|, where |graph| is a variable holding the
  graph object. 
\item The |run| method of the instance is called. This method should now
  compute ``good'' positions for the nodes in the graph represented by
  the |graph|.
\item Post layout operations are performed, namely orienting the
  graph and then anchoring the graph. Both operations are performed
  automatically, but it is possible to configure them.
\end{enumerate}

TODO: Work on this:

Let us now look at the |graph_drawing_algorithm| function in more
detail. It takes a single parameter \meta{info}, which must be a
table, and does the following:

\begin{enumerate}
\item It declares a new class.
\item It declares a |new| method for this class, which takes two
  parameters |g| and |algo| and returns a new \meta{instance} of the
  class. The first parameter will be installed in the attribute
  |graph| of \meta{instance}, the second in the |parent_algorithm|
  attribute. The |new| method tests whether a key called 
  |graph_options| is defined in the \meta{info} table. If so, the
  value for this key must be a table of \meta{options}. This table is
  processed as follows: For each pair \meta{key} |=| \meta{value}
  inside the \meta{options} table, an attribute \meta{key} is created
  in \meta{instance}. The \meta{value} may contain a \meta{type} at
  the end in square brackets. If \meta{type} is |string|, then the
  following happens:
  \begin{quote}
    |graph:getOption('/graph drawing/' .. |\meta{value}|)|
  \end{quote}
  For \meta{type} being |boolean|, the option is first converted to a
  boolean, similarly for \meta{type} being |number|. For \meta{type}
  being |algorithm| or |require| or |load|, the option must store the
  name of a file. This file will be loaded using |require| and the
  result is stored in the key.
\item If the key |properties| is defined inside the \meta{info} table,
  its value should be table of ``static'' or ``default'' values for
  the class. More precisely, this table is used as the metatable of
  the class.
\end{enumerate}

Let us have a look at an example: We redo our |SimpleDemo|, but this
time using the full power of the |graph_drawing_algorithm| function:

\begin{codeexample}[code only]
local SimpleDemo = pgf.gd.new_algorithm_class = {
  
  -- Declare a property:
  properties = {
    -- Ensures, that the graph is always connected when the graph
    -- drawing algorithm is called
    works_only_on_connected_graphs = true
  },

  -- Declare a graph parameter:
  graph_parameters = {
    label  = 'label [string]',
    radius = 'radius [number]',
  }
}
\end{codeexample}

The code is equivalent to the following:
\begin{codeexample}[code only]
local SimpleDemo = { works_only_on_connected_graphs = true }
SimpleDemo.__index = SimpleDemo

function SimpleDemo:new(g)
  local obj = { graph = g }
  setmetatable(obj, SimpleDemo)
  
  obj.label  = g:getOption('/graph drawing/label')
  obj.radius = tonumber(g:getOption('/graph drawing/radius'))
  
  return obj  
end  
\end{codeexample}



\subsubsection{Coordinate Systems in Lua}

\label{section-gd-lua-coordinates}

The main job of a graph drawing algorithm is to modify the
coordinates of the nodes of the graph object in the |graph|
attribute. Before we have a look at how this can be done, let us 
first clarify how the different coordinate systems of \pgfname\
interact with the graph drawing engine.

Let us start with the case that there is no special transformation
matrix setup is setup. In this case, all coordinates inside the Lua
layer are pairs of numbers that will be interpreted as dimensions in
\TeX\ points (one \TeX\ point equals 1/72.27 inches). The first number
will be interpreted as the $x$-coordinate (going right) and the second
number will be interpreted as the $y$-coordinate (going up). This is
true both for the bounding boxes of the nodes that are passed down to
the Lua layer, but also also for the coordinates that are computed by
the algorithms inside the Lua layer.

When graph parameters are set using the |evaluate math expression|
syntax, the dimensions will already have been converted into this
coordinate system. For instance, when a user writes
|node distance=1in|, then |getOption('/graph drawing/node distance')|
will yield the string |'72.27'|.

When a transformation matrix is
set, such as a shift by 1cm to the right and a rotation by
30$^\circ$, the following happens: At the beginning of a
graph drawing scope, the transformation matrix is reset. Thus, for
instance all nodes created inside the graph drawing scope for which no
scaling or shifting is setup will be centered on the origin. When
|\pgfgdendscope| is reached, the transformation matrix is immediately
restored, \emph{prior} to inserting the nodes at the computed
positions. This means, in particular, that the coordinates computed by
the graph drawing algorithms will be transformed by the transformation
matrix that was in force at the beginning of the graph drawing
scope. Continuing the example, all coordinates computed by the graph
drawing algorithms would be shifted by 1cm and then rotated by
30$^\circ$.

The bottom line is that graph drawing algorithms do not need to worry
about \pgfname's transformation matrix.



\subsubsection{Example of a Graph Drawing Algorithm's Code}

The following code fragment (taken and slightly altered
from the file |pgfgd-algorithm-SimpleDemo.lua|)
implements a trivial graph drawing algorithm that just places all
nodes on a fixed-size circle.  

\pgfgddeclareforwardedkeys{/graph drawing}{
  radius/.graph parameter=evaluate math expression
}
\pgfgdset{radius/.parameter initial=1cm}


\begin{codeexample}[code only]
local SimpleDemo = pgf.gd.new_algorithm_class {}

function SimpleDemo:run()
  local radius = 28.908  -- this is 1cm in points
  local nodeCount = table.count_pairs(self.graph.nodes)

  local alpha = (2 * math.pi) / nodeCount
  for i,node in ipairs(self.graph.nodes) do
    -- the interesting part...
    node.pos.x = radius * math.cos(i * alpha)
    node.pos.y = radius * math.sin(i * alpha)
  end
end

return SimpleDemo
\end{codeexample}


\subsubsection{Setting Up a Key for Selecting the Algorithm}

Users will typically wish to write something shorter than
|graph drawing scope...| in order to run a graph drawing algorithm on
a graph. For this reason, you should setup a style on the \TeX\ side
that calls the above keys. For instance, you could create a small
\tikzname\ library and place the following in the library:

\begin{codeexample}[code only]
\tikzset{circle layout/.style={
    graph drawing scope,
    /graph drawing/algorithm=SimpleDemo}}
\end{codeexample}

However, there is a better command for this:

\begin{codeexample}[code only]
% Place this in a file like pgflibrarygraphdrawing.circular.code.tex
\pgfgddeclarealgorithmkey
{circle layout}
{circle layout}
{algorithm=SimpleDemo}
\end{codeexample}
\pgfgddeclarealgorithmkey
{circle layout}
{circle layout}
{algorithm=pgf.gd.examples.SimpleDemo}

The |\pgfgddeclarealgorithmkey| takes care of setting up your style
key in appropriate ways (including some ways you will not have thought
of) and installs some additional handlers.


\subsubsection{Setting Up Graph Parameters}

Returning to the algorithm, it would be better if we could
``configure'' the radius of the circle. The graph drawing engine
provides for this case: You can declare certain \pgfname\ keys to be
so-called ``graph parameters''. When a key is declared as a graph
parameter, it will be available inside the algorithm: 

\begin{codeexample}[code only]
local radius = tonumber(self.graph:getOption("/graph drawing/radius"))
\end{codeexample}

Using the |getOption| method we obtain the value of the
graph parameter, but we must first register the key on the \TeX\ side
as follows: 

\begin{codeexample}[code only]
\pgfgddeclareforwardedkeys{/graph drawing}{
  radius/.graph parameter=evaluate math expression,
  radius/.parameter initial=1cm
}
\end{codeexample}

The |evaluate math expression| tells \TeX\ that whenever you assign
something to the |radius| option, the mathematical expression should
be evaluated and the result should be passed down to the graph drawing
algorithm. For instance, when you write |radius=20pt+3.5pt|, the
algorithm will get the value |23.5| as a result to calling
|getOption|. The |getOption| function will return a |nil| value for
keys that have not been set. While this sometimes is desired
behaviour, in our example we would want the radius to be set to a
default value (1cm in this case) when nothing has been specified. This
is achieved by the second line. The result of the above modifications
can be seen in the following example:

\begin{codeexample}[]
\tikz \graph [circle layout, radius=1.5cm]
  {f -> c -> e ->[bend right] a -> {b -> {c, d, f}, e -> b}};
\end{codeexample}

Since graph parameters are used quite frequently, there is special
support for them: In the declaration of class via
|pgf.gd.new_algorithm_class|, you can provide a key
|graph_parameters|. This key will take a table of key--value pairs,
where the key is interpreted as an attribute of the algorithm object
and the value is a string possibly suffixed by a type of the parameter
in square brackets. The string is interpreted as the name of a graph
parameter (without the |/graph drawing/| part of the path) and the
attribute will be setup to the value of this graph parameter, possibly
after a typecast. For our example, we would write:

\begin{codeexample}[code only]
local SimpleDemo = pgf.gd.new_algorithm_class {
  graph_parameters = { radius = "radius [number]" }
}
\end{codeexample}
In the main program, we can now write |self.radius|.

In addition to graph parameters, we can also have \emph{node
  parameters}. These are setup similarly to |.graph parameter|, but
with |.node parameter| and they are then accessed via the
|node:getOption| function.

As a slightly artificial example, let us introduce a |node radius|
key, which allows us to change the radius of a single node. For this,
we check for a node whether its radius key is set:

\begin{codeexample}[code only]
-- In SimpleDemo.lua:

  for i,node in ipairs(self.graph.nodes) do
    -- the interesting part...
    local node_radius = tonumber(node:getOption('/graph drawing/node radius')
                                 or self.radius)
    node.pos.x = node_radius * math.cos(i * alpha)
    node.pos.y = node_radius * math.sin(i * alpha)
  end
   
% In pgflibrarygraphdrawing.circle.code.tex
\pgfgddeclareforwardedkeys{/graph drawing}{
  node radius/.node parameter=evaluate math expression
}
\end{codeexample}
\pgfgddeclareforwardedkeys{/graph drawing}{
  node radius/.node parameter=evaluate math expression
}

\begin{codeexample}[]
\tikz \graph [circle layout]
  { a -> b -> c -> d [node radius=2cm] -> e -> a };
\end{codeexample}

Here is the complete code of the final algorithm:
\begin{codeexample}[code only]
-- File SimpleDemo.lua

local SimpleDemo = pgf.gd.new_algorithm_class {
  graph_parameters = { radius = "radius [number]" }
}

function SimpleDemo:run()
  local alpha = (2 * math.pi) / #self.graph.nodes
  for i,node in ipairs(self.graph.nodes) do
    -- the interesting part...
    local node_radius = tonumber(node:getOption('/graph drawing/node radius')
                                 or self.radius)
    node.pos.x = node_radius * math.cos(i * alpha)
    node.pos.y = node_radius * math.sin(i * alpha)
  end
end
   
return SimpleDemo  
\end{codeexample}

\begin{codeexample}[code only]
% File pgflibrarygraphdrawing.circle.code.tex
  
\pgfgddeclarealgorithmkey
  {circle layout}
  {circle layout}
  {algorithm=SimpleDemo}

\pgfgddeclareforwardedkeys{/graph drawing}{
  radius/.graph parameter=evaluate math expression,
  radius/.parameter initial=1cm,
  node radius/.node parameter=evaluate math expression
}
\end{codeexample}




\subsection{Lua Layer: Pre- and Postprocessing}

A number of tasks in graph drawing can be performed independently of
the actual algorithm used. For instance, many algorithms require that
the graph is connected. In this case, unconnected input graphs first
need to be decomposed into their connected components, which should
then be processed independently. Such a step would be
\emph{preprocessing} step. Similarly, once a graph has been laid out
by an algorithm, it often still needs to be shifted around to its
anchoring position. This step is the same for any algorithm and can be
done in a \emph{postprocessing} step.

It turns out that some pre- or postprocessing steps make sense for
certain algorithms, but not for other algorithms. For this reason, an
algorithm can specify which steps should (not) be performed by setting
certain attributes in the algorithm's class. Usually, these attributes
will be set using the |properties| key in the declaration of the
algorithm's class.

In the following, we describe which steps are performed and which keys
influence them.


\subsubsection{Preprocessing}

The following preprocessing steps are performed for every graph:
\begin{enumerate}
\item If the |works_only_on_connected_graphs| property is set, the
  connected components of the graph will first be computed.
\item For each component or, if the property is not set, for the whole
  graph, a new algorithm object is created.
\item The |run| method is then called for each component,
  \emph{unless} the size of the component is |1|. If, however, the
  |run_also_for_single_node| property is set, the algorithm is even
  invoked for a 1-node graph.
\end{enumerate}

\subsubsection{Postprocessing}

Each time the |run| method finishes, the following postprocessing
operations are performed:
\begin{enumerate}
\item The graph is oriented, see
  Section~\ref{subsection-library-graphdrawing-standard-orientation}. A
  graph drawing algorithm can set the |growth_direction| property in
  case the graph has a natural growth direction.
\item The graph is anchored, see
  Section~\ref{subsection-library-graphdrawing-anchoring}. 
\end{enumerate}
The above steps are applied to each connected component individually
if the splitting key has been set.

The components then need to be ``packed'', but this is not yet
implemented.




\subsection{Lua Layer: The Main Classes}

In the following, details of the different main classes that are
useful for graph drawing algorithms are documented.


\subsubsection{The Graph Class}

The class |Graph| is used to represent graphs and contains
references to the nodes and edges stored in a graph.

A graph drawing algorithm gets passed a |Graph| object that represents
the to-be-layouted graph. However, you can also create new graph
objects, for instance to decompose the graph into connected
components. 

To create a new graph, you can use the |copy| method, which creates a 
shallow copy (without coying nodes or edges), and the
|subGraphParent| method, which creates a deep copy of the graph, edge
and node objects starting at a designated parent node. If you need
more control by supplying your own set of already visited nodes, use
the underlying function |subGraph|.

A graph allows you to add and remove nodes and edges via |addNode|,
|addEdge|, |removeNode| and |removeEdge| respectively.  There are also
variants which remove all incident edges on a node removal and
conversely, |deleteNode| and |deleteEdge|.

Nodes can be looked up by name with |findNode|. The more generic
|findNodeIf| allows you to search for a node passing a test
predicate. 

The |walkDepth| and |walkBreadth| methods may be used to get
iterators over all nodes and edges in a depth-first or breadth-first
order (other traversal orders may require a rewrite or extension of the
|walkAux| method).

Positions are represented using the class |Vector|.

The following tasks are typical for manipulating the graph.

\begin{itemize}
\item Iterate over all nodes.
\begin{codeexample}[code only]
for node in table.value_iter(self.graph.nodes) do
   ...
end
\end{codeexample}
\item Get width or height of a node:
\begin{codeexample}[code only]
local width, height = node.width, node.height
\end{codeexample}
\item Get or set the coordinates of a node. The final values of these
  coordinates will be used during as the actual positions of the nodes
  on the page.
\begin{codeexample}[code only]
node.pos.x = node.pos.x + 1
node.pos.y = node.pos.y + 1
\end{codeexample}
\item Iterate over all edges and all nodes of the current edge.
\begin{codeexample}[code only]
for _,edge in ipairs(self.graph.edges) do
   for _,node in ipairs(edge.nodes) do
      ...
   end
end
\end{codeexample}
\item Get the nodes connected by an edge.
\begin{codeexample}[code only]
local nodeLeft = edge.nodes[1]
local nodeRight = edge.nodes[2]
\end{codeexample}
\end{itemize}

%% This file has been generated from the lua sources using LuaDoc.
% To regenerate it call "make genluadoc" in
% doc/generic/pgf/version-for-luatex/en.

\begin{filedescription}{pgflibrarygraphdrawing-graph.lua}


\begin{luacommand}{{Graph:\textunderscore{}\textunderscore{}tostring}()}
Returns a string representation of this graph including all nodes and edges. 


Return value:
\begin{itemize} \item[] Graph as string.  \end{itemize}


\end{luacommand}\begin{luacommand}{{Graph:addEdge}(\meta{edge})}
Adds an edge to the graph. 

Parameters:
\begin{parameterdescription}
	\item[\meta{edge}] The edge to be added. 
\end{parameterdescription}



\end{luacommand}\begin{luacommand}{{Graph:addNode}(\meta{node})}
Adds a node to the graph. 

Parameters:
\begin{parameterdescription}
	\item[\meta{node}] The node to be added. 
\end{parameterdescription}



\end{luacommand}\begin{luacommand}{{Graph:copy}()}
Creates a shallow copy of a graph.  The nodes and edges of the original graph are not preserved in the copy. 


Return value:
\begin{itemize} \item[] A shallow copy of the graph.  \end{itemize}


\end{luacommand}\begin{luacommand}{{Graph:createEdge}(\meta{nodeA},\meta{nodeB},\meta{direction},\meta{edgenodes},\meta{options},\meta{tikzoptions})}
Creates and adds a new edge to the graph. 

Parameters:
\begin{parameterdescription}
	\item[\meta{nodeA}] The first node of the new edge.\item[\meta{nodeB}] The second node of the new edge.\item[\meta{direction}] The direction of the new edge. Possible values are |Edge.UNDIRECTED|, |Edge.LEFT|, |Edge.RIGHT|, |Edge.BOTH| and |Edge.NONE| (for invisible edges).\item[\meta{edgenodes}] A string of \tikzname\ edge nodes that needs to be passed back to the \TeX layer unmodified.\item[\meta{options}] The options of the new edge.\item[\meta{tikzoptions}] A table of \tikzname\ options to be used by graph drawing algorithms to treat the edge in special ways. 
\end{parameterdescription}


Return value:
\begin{itemize} \item[] The newly created edge.  \end{itemize}


\end{luacommand}\begin{luacommand}{{Graph:deleteEdge}(\meta{edge})}
Like removeEdge, but also removes the edge from its adjacent nodes. 

Parameters:
\begin{parameterdescription}
	\item[\meta{edge}] The edge to be deleted. 
\end{parameterdescription}


Return value:
\begin{itemize} \item[] The removed edge or |nil| if it was not found in the graph.  \end{itemize}


\end{luacommand}\begin{luacommand}{{Graph:deleteNode}(\meta{node})}
Like removeNode, but also deletes all adjacent edges of the removed node.  This function also removes the deleted adjacent edges from all neighbours of the removed node. 

Parameters:
\begin{parameterdescription}
	\item[\meta{node}] The node to be deleted together with its adjacent edges. 
\end{parameterdescription}


Return value:
\begin{itemize} \item[] The removed node or |nil| if the node was not found in the graph.  \end{itemize}


\end{luacommand}\begin{luacommand}{{Graph:findNode}(\meta{name})}
If possible, looks up the node with the given name in the graph. 

Parameters:
\begin{parameterdescription}
	\item[\meta{name}] Name of the node to look up. 
\end{parameterdescription}


Return value:
\begin{itemize} \item[] The node with the given name or |nil| if it was not found in the graph.  \end{itemize}


\end{luacommand}\begin{luacommand}{{Graph:findNodeIf}(\meta{test})}
Looks up the first node for which the function \meta{test} returns |true|. 

Parameters:
\begin{parameterdescription}
	\item[\meta{test}] A function that takes one parameter (a |Node|) and returns |true| or |false|. 
\end{parameterdescription}


Return value:
\begin{itemize} \item[] The first node for which \meta{test} returns |true|.  \end{itemize}


\end{luacommand}\begin{luacommand}{{Graph:getOption}(\meta{name})}
Returns the value of the graph option \meta{name}. 

Parameters:
\begin{parameterdescription}
	\item[\meta{name}] Name of the option. 
\end{parameterdescription}


Return value:
\begin{itemize} \item[] The value of the graph option \meta{name} or |nil|.  \end{itemize}


\end{luacommand}\begin{luacommand}{{Graph:mergeOptions}(\meta{options})}
Merges the given options into the options of the graph. 

Parameters:
\begin{parameterdescription}
	\item[\meta{options}] The options to be merged. 
\end{parameterdescription}



See also:
\begin{itemize}
	\item[] |mergeTable |
\end{itemize}

\end{luacommand}\begin{luacommand}{{Graph:new}(\meta{values})}
Creates a new graph. 

Parameters:
\begin{parameterdescription}
	\item[\meta{values}] Values to override default graph settings. The following parameters can be set:\par |nodes|: The nodes of the graph.\par |edges|: The edges of the graph.\par |pos|: Initial position of the graph.\par |options|: A table of node options passed over from \tikzname. 
\end{parameterdescription}


Return value:
\begin{itemize} \item[] A newly-allocated graph.  \end{itemize}


\end{luacommand}\begin{luacommand}{{Graph:removeEdge}(\meta{edge})}
If possible, removes an edge from the graph and returns it. 

Parameters:
\begin{parameterdescription}
	\item[\meta{edge}] The edge to be removed. 
\end{parameterdescription}


Return value:
\begin{itemize} \item[] The removed edge or |nil| if it was not found in the graph.  \end{itemize}


\end{luacommand}\begin{luacommand}{{Graph:removeNode}(\meta{node})}
If possible, removes a node from the graph and returns it. 

Parameters:
\begin{parameterdescription}
	\item[\meta{node}] The node to remove. 
\end{parameterdescription}


Return value:
\begin{itemize} \item[] The removed node or |nil| if it was not found in the graph.  \end{itemize}


\end{luacommand}\begin{luacommand}{{Graph:setOption}(\meta{name},\meta{value})}
Sets the graph option \meta{name} to \meta{value}. 

Parameters:
\begin{parameterdescription}
	\item[\meta{name}] Name of the option to be changed.\item[\meta{value}] New value for the graph option \meta{name}. 
\end{parameterdescription}



\end{luacommand}\begin{luacommand}{{Graph:subGraph}(\meta{root},\meta{graph},\meta{visited})}
Returns a subgraph.  The resulting subgraph begins at the node root, excludes all nodes and edges that are marked as visited. 

Parameters:
\begin{parameterdescription}
	\item[\meta{root}] Root node where the operation starts.\item[\meta{graph}] Result graph object or |nil| if the original graph should be used as the parent graph.\item[\meta{visited}] Set of already visited nodes/edges or |nil|. This set will be modified so make sure not to use a table that you want to remain untouched. 
\end{parameterdescription}



\end{luacommand}\begin{luacommand}{{Graph:subGraphParent}(\meta{root},\meta{parent},\meta{graph})}
Creates a new subgraph with \meta{parent} marked as visited.  This function can be useful if the graph is a tree structure (and \meta{parent} is the parent node of \meta{root}). 

Parameters:
\begin{parameterdescription}
	\item[\meta{root}] Root node where the operation starts.\item[\meta{parent}] Parent of the recursion step before.\item[\meta{graph}] Result graph object or |nil| if the original graph should be used as the parent graph. 
\end{parameterdescription}



See also:
\begin{itemize}
	\item[] |subGraph |
\end{itemize}

\end{luacommand}\begin{luacommand}{{Graph:walkAux}(\meta{root},\meta{visited},\meta{removeIndex})}
Auxiliary function to walk a graph. Does nothing if no nodes exist. 

Parameters:
\begin{parameterdescription}
	\item[\meta{root}] The first node to be visited.  If nil, chooses some node.\item[\meta{visited}] Set of already visited nodes and edges. |visited[v] == true| indicates that the node or edge |v| has already been visited.\item[\meta{removeIndex}] A numeric value or |nil| that defines the order in which nodes and edges are visited while traversing the graph. |nil| results in queue behavior, |1| in stack behavior. 
\end{parameterdescription}



See also:
\begin{itemize}
	\item[] |walkDepth|\item[] |walkBreadth |
\end{itemize}

\end{luacommand}\begin{luacommand}{{Graph:walkBreadth}(\meta{root},\meta{visited})}
Returns an iterator to walk the graph in a breadth-first traversal.  The iterator returns all edges and nodes one at a time. In case only the nodes are of interest, a filter function like |iter.filter| can be used to ignore edges. 

Parameters:
\begin{parameterdescription}
	\item[\meta{root}] The first node to be visited.  If nil, chooses some node.\item[\meta{visited}] Set of already visited nodes and edges. |visited[v] == true| indicates that the node or edge |v| has already been visited. 
\end{parameterdescription}



See also:
\begin{itemize}
	\item[] |iter.filter |
\end{itemize}

\end{luacommand}\begin{luacommand}{{Graph:walkDepth}(\meta{root},\meta{visited})}
Returns an iterator to walk the graph in a depth-first traversal.  The iterator returns all edges and nodes one at a time. In case only the nodes are of interest, a filter function like |iter.filter| can be used to ignore edges. 

Parameters:
\begin{parameterdescription}
	\item[\meta{root}] The first node to be visited.  If nil, chooses some node.\item[\meta{visited}] Set of already visited nodes and edges. |visited[v] == true| indicates that the node or edge |v| has already been visited. 
\end{parameterdescription}



See also:
\begin{itemize}
	\item[] |iter.filter |
\end{itemize}

\end{luacommand}
\end{filedescription}
%
%The following module simplifies the traversal of graphs:
%
%% This file has been generated from the lua sources using LuaDoc.
% To regenerate it call "make genluadoc" in
% doc/generic/pgf/version-for-luatex/en.

\begin{filedescription}{pgflibrarygraphdrawing-traversal-helpers.lua}


\begin{luacommand}{{traversal.depth\textunderscore{}first\textunderscore{}dag}(\meta{graph},\meta{initial\_nodes})}
Iterator for traversing a directed acyclic \meta{graph} in depth-first order. 

Parameters:
\begin{parameterdescription}
	\item[\meta{graph}] A directed acyclic graph. 
\end{parameterdescription}


Return value:
\begin{parameterdescription} 
  \item[] An iterator for traversing \meta{graph} in a depth-first order. 
\end{parameterdescription}


\end{luacommand}
\begin{luacommand}{{traversal.topological\textunderscore{}sorting}(\meta{graph})}
Iterator for traversing a directed \meta{graph} using a topological sorting.  A topological sorting of a directed graph is a linear ordering of its nodes such that, for every edge |(u,v)|, |u| comes before |v|.  Important note: if performed on a graph with at least one cycle a topological sorting is impossible. Thus, the nodes returned from the iterator are not guaranteed to satisfy the ``|u| comes before |v|'' criterion. The iterator may even terminate early or loop forever. 

Parameters:
\begin{parameterdescription}
	\item[\meta{graph}] A directed acyclic graph. 
\end{parameterdescription}


Return value:
\begin{parameterdescription} 
  \item[] An iterator for traversing \meta{graph} in a topological order. 
\end{parameterdescription}


\end{luacommand}

\end{filedescription}



\subsubsection{Nodes}

Nodes serve as direct representations of the \TeX\ level nodes and
include information about incident edges, the calculated position and
the \TeX\ box used.  Typically one'll use its methods to navigate
through the graph or to add and remove edges in an intermediary graph.
Using the information from the \TeX\ side, this class is also able to
provide layout information, i.e. the dimensions of the corresponding
\TeX\ box.

%% This file has been generated from the lua sources using LuaDoc.
% To regenerate it call "make genluadoc" in
% doc/generic/pgf/version-for-luatex/en.

\begin{filedescription}{pgflibrarygraphdrawing-node.lua}


\begin{luacommand}{{Node:\textunderscore{}\textunderscore{}eq}(\meta{object},\meta{other})}
Compares two nodes by their name. 

Parameters:
\begin{parameterdescription}
	\item[\meta{other}] Another node to compare with. 
\end{parameterdescription}


Return value:
\begin{itemize} \item[] |true| if both nodes have the same name. |false| otherwise.  \end{itemize}


\end{luacommand}\begin{luacommand}{{Node:\textunderscore{}\textunderscore{}tostring}()}
Returns a formated string representation of the node. 


Return value:
\begin{itemize} \item[] String represenation of the node.  \end{itemize}


\end{luacommand}\begin{luacommand}{{Node:addEdge}(\meta{edge})}
Adds new edge to the node. 

Parameters:
\begin{parameterdescription}
	\item[\meta{edge}] The edge to be added. 
\end{parameterdescription}



\end{luacommand}\begin{luacommand}{{Node:copy}()}
Creates a shallow copy of the node.  Most notably, the edges adjacent are not preserved in the copy. 


Return value:
\begin{itemize} \item[] Copy of the node.  \end{itemize}


\end{luacommand}\begin{luacommand}{{Node:getDegree}()}
Counts the adjacent edges of the node. 


Return value:
\begin{itemize} \item[] The number of adjacent edges of the node.  \end{itemize}


\end{luacommand}\begin{luacommand}{{Node:getEdges}()}
Returns all edges of the node.  Instead of calling |node:getEdges()| the edges can alternatively be accessed directly with |node.edges|. 


Return value:
\begin{itemize} \item[] All edges of the node.  \end{itemize}


\end{luacommand}\begin{luacommand}{{Node:getInDegree}(\meta{ignorereversed})}
Returns the number of incoming edges of the node. 

Parameters:
\begin{parameterdescription}
	\item[\meta{ignorereversed}] Optional parameter to consider reversed edges not reversed for this method call. Defaults to |false|. 
\end{parameterdescription}


Return value:
\begin{itemize} \item[] The number of incoming edges of the node.  \end{itemize}


See also:
\begin{itemize}
	\item[] |Node:getIncomingEdges(reversed) |
\end{itemize}

\end{luacommand}\begin{luacommand}{{Node:getIncomingEdges}(\meta{ignorereversed})}
Returns the incoming edges of the node. Undefined result for hyperedges. 

Parameters:
\begin{parameterdescription}
	\item[\meta{ignorereversed}] Optional parameter to consider reversed edges not reversed for this method call. Defaults to |false|. 
\end{parameterdescription}


Return value:
\begin{itemize} \item[] Incoming edges of the node. This includes undirected edges and directed edges pointing to the node.  \end{itemize}


\end{luacommand}\begin{luacommand}{{Node:getOption}(\meta{name})}
Returns the value of the node option \meta{name}. 

Parameters:
\begin{parameterdescription}
	\item[\meta{name}] Name of the node option. 
\end{parameterdescription}


Return value:
\begin{itemize} \item[] The value of the node option \meta{name} or |nil|.  \end{itemize}


\end{luacommand}\begin{luacommand}{{Node:getOutDegree}(\meta{ignorereversed})}
Returns the number of edges starting at the node. 

Parameters:
\begin{parameterdescription}
	\item[\meta{ignorereversed}] Optional parameter to consider reversed edges not reversed for this method call. Defaults to |false|. 
\end{parameterdescription}


Return value:
\begin{itemize} \item[] The number of outgoing edges of the node.  \end{itemize}


See also:
\begin{itemize}
	\item[] |Node:getOutgoingEdges() |
\end{itemize}

\end{luacommand}\begin{luacommand}{{Node:getOutgoingEdges}(\meta{ignorereversed})}
Returns the outgoing edges of the node. Undefined result for hyperedges. 

Parameters:
\begin{parameterdescription}
	\item[\meta{ignorereversed}] Optional parameter to consider reversed edges not reversed for this method call. Defaults to |false|. 
\end{parameterdescription}


Return value:
\begin{itemize} \item[] Outgoing edges of the node. This includes undirected edges and directed edges leaving the node.  \end{itemize}


\end{luacommand}\begin{luacommand}{{Node:getTexHeight}()}
Computes the heigth of the node. 


Return value:
\begin{itemize} \item[] Height of the node.  \end{itemize}


\end{luacommand}\begin{luacommand}{{Node:getTexWidth}()}
Computes the width of the node. 


Return value:
\begin{itemize} \item[] Width of the node.  \end{itemize}


\end{luacommand}\begin{luacommand}{{Node:new}(\meta{values})}
Creates a new node. 

Parameters:
\begin{parameterdescription}
	\item[\meta{values}] Values to override default node settings. The following parameters can be set:\par |name|: The name of the node. It is obligatory to define this.\par |tex|:  Information about the corresponding \TeX\ node.\par |edges|: Edges adjacent to the node.\par |pos|: Initial position of the node.\par |options|: A table of node options passed over from \tikzname. 
\end{parameterdescription}


Return value:
\begin{itemize} \item[] A newly allocated node.  \end{itemize}


\end{luacommand}\begin{luacommand}{{Node:removeEdge}(\meta{edge})}
Removes an edge from the node. 

Parameters:
\begin{parameterdescription}
	\item[\meta{edge}] The edge to remove. 
\end{parameterdescription}



\end{luacommand}\begin{luacommand}{{Node:setOption}(\meta{name},\meta{value})}
Sets the node option \meta{name} to \meta{value}. 

Parameters:
\begin{parameterdescription}
	\item[\meta{name}] Name of the node option to be changed.\item[\meta{value}] New value for the node option \meta{name}. 
\end{parameterdescription}



\end{luacommand}
\end{filedescription}


\subsubsection{The Edge Class}

|Edge| objects contain references to incident nodes, including the
possibility to create hyperedges with more than two nodes for an edge.
Edges can be undirected or directed (denoted by the constants
|Edge.UNDIRECTED| or |Edge.LEFT|, |Edge.RIGHT|, |Edge.BOTH| and
|Edge.NONE| for invisible edges, see |Interface:drawEdge|). 

%% This file has been generated from the lua sources using LuaDoc.
% To regenerate it call "make genluadoc" in
% doc/generic/pgf/version-for-luatex/en.

\begin{filedescription}{pgflibrarygraphdrawing-edge.lua}


\begin{luacommand}{{Edge:\textunderscore{}\textunderscore{}eq}(\meta{other})}
Returns whether or not the two edges have the same adjacent nodes. 

Parameters:
\begin{parameterdescription}
	\item[\meta{other}] Another edge to compare with. 
\end{parameterdescription}


Return value:
\begin{parameterdescription} 
  \item[] |true| if the two edges have exactly the same adjacent nodes. 
\end{parameterdescription}


\end{luacommand}
\begin{luacommand}{{Edge:\textunderscore{}\textunderscore{}tostring}()}
Returns a readable string representation of the edge. 


Return value:
\begin{parameterdescription} 
  \item[] String representation of the edge. 
\end{parameterdescription}


\end{luacommand}
\begin{luacommand}{{Edge:addNode}(\meta{node})}
If possible, adds a node to the edge. 

Parameters:
\begin{parameterdescription}
	\item[\meta{node}] The node to be added to the edge. 
\end{parameterdescription}



\end{luacommand}
\begin{luacommand}{{Edge:containsNode}(\meta{node})}
Returns whether or not a node is adjacent to the edge. 

Parameters:
\begin{parameterdescription}
	\item[\meta{node}] The node to check. 
\end{parameterdescription}


Return value:
\begin{parameterdescription} 
  \item[] |true| if the node is adjacent to the edge. |false| otherwise. 
\end{parameterdescription}


\end{luacommand}
\begin{luacommand}{{Edge:copy}()}
Copies an edge (preventing accidental use).  The nodes of the edge are not preserved and have to be added to the copy manually if necessary. 


Return value:
\begin{parameterdescription} 
  \item[] Shallow copy of the edge. 
\end{parameterdescription}


\end{luacommand}
\begin{luacommand}{{Edge:getDegree}()}
Counts the nodes on this edge. 


Return value:
\begin{parameterdescription} 
  \item[] The number of nodes on the edge. 
\end{parameterdescription}


\end{luacommand}
\begin{luacommand}{{Edge:getNeighbour}(\meta{node})}
Gets first neighbour of the node (disregarding hyperedges). 

Parameters:
\begin{parameterdescription}
	\item[\meta{node}] The node which first neighbour should be returned. 
\end{parameterdescription}


Return value:
\begin{parameterdescription} 
  \item[] The first neighbour of the node. 
\end{parameterdescription}


\end{luacommand}
\begin{luacommand}{{Edge:getNeighbours}(\meta{node})}
Returns all neighbours of a node adjacent to the edge.  The edge direction is not taken into account, so this method always returns all neighbours even if called on a directed edge. 

Parameters:
\begin{parameterdescription}
	\item[\meta{node}] A node. Typically but not necessarily adjacent to the edge. If the node is not an intermediate or end point of the edge, an empty array is returned. 
\end{parameterdescription}


Return value:
\begin{parameterdescription} 
  \item[] An array of nodes that are adjacent to the input node via the edge the method is called on. 
\end{parameterdescription}


\end{luacommand}
\begin{luacommand}{{Edge:getNodes}()}
Returns all nodes of the edge.  Instead of calling |edge:getNodes()| the nodes can alternatively be accessed directly with |edge.nodes|. 


Return value:
\begin{parameterdescription} 
  \item[] All edges of the node. 
\end{parameterdescription}


\end{luacommand}
\begin{luacommand}{{Edge:getOption}(\meta{name})}
Returns the value of the edge option \meta{name}. 

Parameters:
\begin{parameterdescription}
	\item[\meta{name}] Name of the option. 
\end{parameterdescription}


Return value:
\begin{parameterdescription} 
  \item[] The value of the edge option \meta{name} or |nil|. 
\end{parameterdescription}


\end{luacommand}
\begin{luacommand}{{Edge:isHead}(\meta{node},\meta{ignore\_reversed})}
Checks whether a node is the head of the edge. Does not work for hyperedges.  This method only works for edges with two adjacent nodes.  For undirected edges or edges that point into both directions, the result will always be true. Directed edges may be reversed internally, so their head and tail might be switched. Whether or not this internal reversal is handled by this method can be specified with the optional second \meta{ignore\_reversed} parameter which is |false| by default. 

Parameters:
\begin{parameterdescription}
	\item[\meta{node}] The node to check.\item[\meta{ignore\_reversed}] Optional parameter. Set this to true if reversed edges should not be considered reversed for this method call. 
\end{parameterdescription}


Return value:
\begin{parameterdescription} 
  \item[] True if the node is the head of the edge. 
\end{parameterdescription}


\end{luacommand}
\begin{luacommand}{{Edge:isHyperedge}()}
Returns whether or not the edge is a hyperedge.  A hyperedge is an edge with more than two adjacent nodes. 


Return value:
\begin{parameterdescription} 
  \item[] |true| if the edge is a hyperedge. |false| otherwise. 
\end{parameterdescription}


\end{luacommand}
\begin{luacommand}{{Edge:isTail}(\meta{node},\meta{ignore\_reversed})}
Checks whether a node is the tail of the edge. Does not work for hyperedges.  This method only works for edges with two adjacent nodes.  For undirected edges or edges that point into both directions, the result will always be true.  Directed edges may be reversed internally, so their head and tail might be switched. Whether or not this internal reversal is handled by this method can be specified with the optional second \meta{ignore\_reversed} parameter which is |false| by default. 

Parameters:
\begin{parameterdescription}
	\item[\meta{node}] The node to check.\item[\meta{ignore\_reversed}] Optional parameter. Set this to true if reversed edges should not be considered reversed for this method call. 
\end{parameterdescription}


Return value:
\begin{parameterdescription} 
  \item[] True if the node is the tail of the edge. 
\end{parameterdescription}


\end{luacommand}
\begin{luacommand}{{Edge:new}(\meta{values})}
Creates an edge between nodes of a graph. 

Parameters:
\begin{parameterdescription}
	\item[\meta{values}] Values to override default edge settings. The following parameters can be set:\par |nodes|: TODO \par |edge_nodes|: TODO \par |options|: TODO \par |tikz_options|: TODO \par |direction|: TODO \par |bend_points|: TODO \par |bend_nodes|: TODO \par |reversed|: TODO \par 
\end{parameterdescription}


Return value:
\begin{parameterdescription} 
  \item[] A newly-allocated edge. 
\end{parameterdescription}


\end{luacommand}
\begin{luacommand}{{Edge:setOption}(\meta{name},\meta{value})}
Sets the edge option \meta{name} to \meta{value}. 

Parameters:
\begin{parameterdescription}
	\item[\meta{name}] Name of the option to be changed.\item[\meta{value}] New value for the edge option \meta{name}. 
\end{parameterdescription}



\end{luacommand}

\end{filedescription}


\subsubsection{Positions and Vectors}

TT: More documentation is needed here!

%% This file has been generated from the lua sources using LuaDoc.
% To regenerate it call "make genluadoc" in
% doc/generic/pgf/version-for-luatex/en.

\begin{filedescription}{pgflibrarygraphdrawing-position.lua}


\begin{luacommand}{{Position.calcCoordsTo}(\meta{posFrom},\meta{posTo})}
Returns a vector between two positions.

Parameters:
\begin{parameterdescription}
	\item[\meta{posFrom}] Position A.\item[\meta{posTo}] Position B.
\end{parameterdescription}


Return value:
\begin{parameterdescription} 
  \item[] x- and y-coordinates of the vector between posFrom and posTo.
\end{parameterdescription}


\end{luacommand}
\begin{luacommand}{{Position:\textunderscore{}\textunderscore{}tostring}()}
Returns a readable string representation of the position.


Return value:
\begin{parameterdescription} 
  \item[] string representation of the position.
\end{parameterdescription}


\end{luacommand}
\begin{luacommand}{{Position:copy}()}
Creates a copy of this position object.


Return value:
\begin{parameterdescription} 
  \item[] Copy of the position.
\end{parameterdescription}


\end{luacommand}
\begin{luacommand}{{Position:equals}(\meta{pos})}
Returns a boolean value whether the object is equal to the given position.


Return value:
\begin{parameterdescription} 
  \item[] true if the position is equal to the given position pos.
\end{parameterdescription}


\end{luacommand}
\begin{luacommand}{{Position:getAbsCoordinates}(\meta{x},\meta{y})}
Computes absolute coordinates of a position.

Parameters:
\begin{parameterdescription}
	\item[\meta{x}] Just used internally for recrusion.\item[\meta{y}] Just used internally for recrusion.
\end{parameterdescription}


Return value:
\begin{parameterdescription} 
  \item[] Absolute position.
\end{parameterdescription}


\end{luacommand}
\begin{luacommand}{{Position:isAbsPosition}()}
Determines if the position is absolute.


Return value:
\begin{parameterdescription} 
  \item[] True if the position is absolute, else false.
\end{parameterdescription}


\end{luacommand}
\begin{luacommand}{{Position:new}(\meta{values})}
Represents a relative postion.

Parameters:
\begin{parameterdescription}
	\item[\meta{values}] Values (e.g. x- and y-coordinate) to be merged with the default-metatable of a position.
\end{parameterdescription}


Return value:
\begin{parameterdescription} 
  \item[] A new position object.
\end{parameterdescription}


\end{luacommand}
\begin{luacommand}{{Position:relateTo}(\meta{pos},\meta{keepAbsPosition})}
Relates a position to the given position.

Parameters:
\begin{parameterdescription}
	\item[\meta{pos}] The relative position.\item[\meta{keepAbsPosition}] If true, the coordinates of the position are computed in the relation to the given position pos.
\end{parameterdescription}



\end{luacommand}

\end{filedescription}
%% This file has been generated from the lua sources using LuaDoc.
% To regenerate it call "make genluadoc" in
% doc/generic/pgf/version-for-luatex/en.

\begin{filedescription}{pgflibrarygraphdrawing-vector.lua}


\begin{luacommand}{{Vector:copy}()}
Creates a copy of the vector that holds the same elements as the original. 


Return value:
\begin{parameterdescription} 
  \item[] A newly-allocated copy of the vector holding exactly the same elements. 
\end{parameterdescription}


\end{luacommand}
\begin{luacommand}{{Vector:dividedBy}(\meta{other})}
Performs a vector division and returns the result in a new vector. 

Parameters:
\begin{parameterdescription}
	\item[\meta{other}] Vector to divide by. 
\end{parameterdescription}


Return value:
\begin{parameterdescription} 
  \item[] A new vector with the result of the division. 
\end{parameterdescription}


\end{luacommand}
\begin{luacommand}{{Vector:dividedByScalar}(\meta{scalar})}
Divides a vector by a scalar value and returns the result in a new vector. 

Parameters:
\begin{parameterdescription}
	\item[\meta{scalar}] Scalar value to divide the vector by. 
\end{parameterdescription}


Return value:
\begin{parameterdescription} 
  \item[] A new vector with the result of the division. 
\end{parameterdescription}


\end{luacommand}
\begin{luacommand}{{Vector:dotProduct}(\meta{other})}
Performs the dot product of two vectors and returns the result in a new vector. 

Parameters:
\begin{parameterdescription}
	\item[\meta{other}] Vector to perform the dot product with. 
\end{parameterdescription}


Return value:
\begin{parameterdescription} 
  \item[] A new vector with the result of the dot product. 
\end{parameterdescription}


\end{luacommand}
\begin{luacommand}{{Vector:get}(\meta{index})}
Returns the element at the given \meta{index}. 


Return value:
\begin{parameterdescription} 
  \item[] The element at the given \meta{index}. 
\end{parameterdescription}


\end{luacommand}
\begin{luacommand}{{Vector:limit}(\meta{limit\_function})}
Limits all elements of the vector in-place. 

Parameters:
\begin{parameterdescription}
	\item[\meta{limit\_function}] A function that is called for each index/element pair. It is supposed to return minimum and maximum values for the element. The element is then clamped to these values. 
\end{parameterdescription}



\end{luacommand}
\begin{luacommand}{{Vector:minus}(\meta{other})}
Subtracts two vectors and returns the result in a new vector. 

Parameters:
\begin{parameterdescription}
	\item[\meta{other}] Vector to subtract. 
\end{parameterdescription}


Return value:
\begin{parameterdescription} 
  \item[] A new vector with the result of the subtraction. 
\end{parameterdescription}


\end{luacommand}
\begin{luacommand}{{Vector:minusScalar}(\meta{scalar})}
Subtracts a scalar value from a vector and returns the result in a new vector. 

Parameters:
\begin{parameterdescription}
	\item[\meta{scalar}] Scalar value to subtract from all elements. 
\end{parameterdescription}


Return value:
\begin{parameterdescription} 
  \item[] A new vector with the result of the subtraction. 
\end{parameterdescription}


\end{luacommand}
\begin{luacommand}{{Vector:new}(\meta{n},\meta{fill\_function})}
Creates a new vector with \meta{n} values using an optional \meta{fill\_function}. 

Parameters:
\begin{parameterdescription}
	\item[\meta{n}] The number of elements of the vector.\item[\meta{fill\_function}] Optional function that takes a number between 1 and \meta{n} and is expected to return a value for the corresponding element of the vector. If omitted, all elements of the vector will be initialized with 0. 
\end{parameterdescription}


Return value:
\begin{parameterdescription} 
  \item[] A newly-allocated vector with \meta{n} elements. 
\end{parameterdescription}


\end{luacommand}
\begin{luacommand}{{Vector:norm}()}
Computes the Euclidean norm of the vector. 


Return value:
\begin{parameterdescription} 
  \item[] The Euclidean norm of the vector. 
\end{parameterdescription}


\end{luacommand}
\begin{luacommand}{{Vector:normalized}()}
Normalizes the vector and returns the result in a new vector. 


Return value:
\begin{parameterdescription} 
  \item[] Normalized version of the original vector. 
\end{parameterdescription}


\end{luacommand}
\begin{luacommand}{{Vector:plus}(\meta{other})}
Performs a vector addition and returns the result in a new vector. 

Parameters:
\begin{parameterdescription}
	\item[\meta{other}] The vector to add. 
\end{parameterdescription}


Return value:
\begin{parameterdescription} 
  \item[] A new vector with the result of the addition. 
\end{parameterdescription}


\end{luacommand}
\begin{luacommand}{{Vector:plusScalar}(\meta{scalar})}
Performs an addition with a scalar value and returns the result in a new vector.  The scalar value is added to all elements of the vector. 

Parameters:
\begin{parameterdescription}
	\item[\meta{scalar}] Scalar value to add to all elements. 
\end{parameterdescription}


Return value:
\begin{parameterdescription} 
  \item[] A new vector with the result of the addition. 
\end{parameterdescription}


\end{luacommand}
\begin{luacommand}{{Vector:reset}()}
Resets all vector elements to 0 in-place. 



\end{luacommand}
\begin{luacommand}{{Vector:set}(\meta{index},\meta{value})}
Changes the element at the given \meta{index}. 

Parameters:
\begin{parameterdescription}
	\item[\meta{index}] The index of the element to change.\item[\meta{value}] New value of the element. 
\end{parameterdescription}



\end{luacommand}
\begin{luacommand}{{Vector:timesScalar}(\meta{scalar})}
Multiplies a vector by a scalar value and returns the result in a new vector. 

Parameters:
\begin{parameterdescription}
	\item[\meta{scalar}] Scalar value to multiply the vector with. 
\end{parameterdescription}


Return value:
\begin{parameterdescription} 
  \item[] A new vector with the result of the multiplication. 
\end{parameterdescription}


\end{luacommand}
\begin{luacommand}{{Vector:update}(\meta{update\_function})}
Updates the values of the vector in-place. 

Parameters:
\begin{parameterdescription}
	\item[\meta{update\_function}] A function that is called for each element of the vector. The elements are replaced by the values returned from this function. 
\end{parameterdescription}



\end{luacommand}
\begin{luacommand}{{Vector:x}()}
Convenience method that returns the first element of the vector. 


Return value:
\begin{parameterdescription} 
  \item[] The first element of the vector. 
\end{parameterdescription}


\end{luacommand}
\begin{luacommand}{{Vector:y}()}
Convenience method that returns the second element of the vector. 


Return value:
\begin{parameterdescription} 
  \item[] The second element of the vector. 
\end{parameterdescription}


\end{luacommand}

\end{filedescription}


\subsubsection{The Interface and System Classes}

The class |Interface| is the main entry point in Lua. Every
communication from \TeX\ to Lua is done here. It provides methods to
create graphs, add nodes and edges to graphs, and finally to invoke the
selected algorithm. The |Interface| class manages the stack of
graphs. When the |newGraph()| function is called, it generates a new graph
object and pushes it on the graph stack. The methods |addNode()| and
|addEdge()| are called for each node and each edge, creating the
actual Lua objects and adding them to the current graph. 

After adding nodes and edges, when the scope ends, the interface
invokes the actual algorithm to layout the graph. This is done in the
|drawGraph()| function. The next step is to put the nodes back in the
\TeX\ output stream. This is invoked by the |finishGraph()| method. 

%% This file has been generated from the lua sources using LuaDoc.
% To regenerate it call "make genluadoc" in
% doc/generic/pgf/version-for-luatex/en.

\begin{filedescription}{pgflibrarygraphdrawing-interface.lua}


\begin{luacommand}{{Interface:addEdge}(\meta{from},\meta{to},\meta{direction},\meta{edge\_nodes},\meta{options},\meta{tikz\_options})}
Adds an edge from one node to another by name.  Both parameters are node names and have to exist before an edge can be created between them. 

Parameters:
\begin{parameterdescription}
	\item[\meta{from}] Name of the node the edge begins at.\item[\meta{to}] Name of the node the edge ends at.\item[\meta{direction}] Direction of the edge (e.g. |--| for an undirected edge or |->| for a directed edge from the first to the second node).\item[\meta{edge\_nodes}] A string for \tikzname\ to generate the edge label nodes later. Needs to be passed back to TikZ unmodified.\item[\meta{options}] A string of |{key}{value}| pairs of edge options that are relevant to graph drawing algorithms.\item[\meta{tikz\_options}] A string of |{key}{value}| pairs that need to be passed back to \tikzname\ unmodified. 
\end{parameterdescription}



See also:
\begin{itemize}
	\item[] |addNode |
\end{itemize}

\end{luacommand}
\begin{luacommand}{{Interface:addNode}(\meta{name},\meta{xMin},\meta{yMin},\meta{xMax},\meta{yMax},\meta{options})}
Adds a new node to the graph.  The options string of |{key}{value}| pairs is parsed and assigned to the node. Graph drawing algorithms may use these options to treat the node in special ways. 

Parameters:
\begin{parameterdescription}
	\item[\meta{name}] Name of the node.\item[\meta{xMin}] Minimum x point of the bouding box.\item[\meta{yMin}] Minimum y point of the bouding box.\item[\meta{xMax}] Maximum x point of the bouding box.\item[\meta{yMax}] Maximum y point of the bouding box.\item[\meta{options}] Options for the node. 
\end{parameterdescription}



\end{luacommand}
\begin{luacommand}{{Interface:drawEdge}(\meta{edge})}
Passes an edge back to the \TeX\ layer.  Edges with a direction of |Edge.NONE| are skipped and not passed back to \TeX. 

Parameters:
\begin{parameterdescription}
	\item[\meta{edge}] The edge to pass back to the \TeX\ layer. 
\end{parameterdescription}



\end{luacommand}
\begin{luacommand}{{Interface:drawGraph}()}
Arranges the current graph using the specified algorithm.  The algorithm is derived from the graph options and is loaded on demand from the corresponding algorithm file. For a fictitious algorithm |simple| this file is per convention called |pgflibrarygraphdrawing-algorithms-simple.lua|. It is required to define at least one function as an entry point to the algorithm. The name of the function is again predetermined as |graph_drawing_algorithm_simple|. When a graph is to be layed out, this function is called with the graph as its only parameter. 



\end{luacommand}
\begin{luacommand}{{Interface:drawNode}(\meta{node})}
Passes a node back to the \TeX\ layer. 

Parameters:
\begin{parameterdescription}
	\item[\meta{node}] The node to pass back to the \TeX\ layer. 
\end{parameterdescription}



\end{luacommand}
\begin{luacommand}{{Interface:finishGraph}()}
Passes the current graph back to the \TeX\ layer and removes it from the stack. 



\end{luacommand}
\begin{luacommand}{{Interface:getOption}(\meta{name})}
Returns the value of the graph option \meta{name}. 

Parameters:
\begin{parameterdescription}
	\item[\meta{name}] Name of the option. 
\end{parameterdescription}


Return value:
\begin{parameterdescription} 
  \item[] The value of the \meta{name} option or |nil|. 
\end{parameterdescription}


\end{luacommand}
\begin{luacommand}{{Interface:loadAlgorithm}(\meta{name})}
Attempts to load the algorithm with the given \meta{name}.  This function tries to look up the corresponding algorithm file |pgflibrarygraphdrawing-algorithms-name.lua| and attempts to look up the main function for calling the algorithm. 

Parameters:
\begin{parameterdescription}
	\item[\meta{name}] Name of the algorithm. 
\end{parameterdescription}


Return value:
\begin{parameterdescription} 
  \item[] The algorithm function or nil. 
\end{parameterdescription}


\end{luacommand}
\begin{luacommand}{{Interface:newGraph}(\meta{options})}
Creates a new graph and adds it to the graph stack.  The options string consisting of |{key}{value}| pairs is parsed and assigned to the graph. These options are used to configure the different graph drawing algorithms shipped with \tikzname. 

Parameters:
\begin{parameterdescription}
	\item[\meta{options}] A string containing |{key}{value}| pairs of \tikzname\ options. 
\end{parameterdescription}



See also:
\begin{itemize}
	\item[] |finishGraph |
\end{itemize}

\end{luacommand}
\begin{luacommand}{{Interface:setOption}(\meta{name},\meta{value})}
Sets the graph option \meta{name} to \meta{value}. Only affects the current graph. 

Parameters:
\begin{parameterdescription}
	\item[\meta{name}] The name of the option to set.\item[\meta{value}] New value for the option. 
\end{parameterdescription}



\end{luacommand}

\end{filedescription}

Communication with \TeX\ on a basic layer is done in the |Sys|
class. The |beginShipout()| function opens a new scope in \pgfname\
to put all graph drawing nodes into. This prevents other graph objects
outside the graph drawing scope from referencing these nodes. The
|endShipout()| method closes the scope. Nodes and edges are put in the
output stream by the methods |putTeXBox()| and |putEdge()|, which
invoke callbacks to \TeX. 


\subsubsection{Support Classes and Functions}

Most classes in the framework (including the module objects) implement
the |__tostring| method, meaning that you can get a somewhat useful
string representation of the object via the standard |tostring|
function.

%% This file has been generated from the lua sources using LuaDoc.
% To regenerate it call "make genluadoc" in
% doc/generic/pgf/version-for-luatex/en.

\begin{filedescription}{pgflibrarygraphdrawing-helper.lua}


\begin{luacommand}{{parseBraces}(\meta{str},\meta{default})}
Parses a braced list of {key}{value} pairs and returns a table mapping keys to values.



\end{luacommand}

\end{filedescription}
%% This file has been generated from the lua sources using LuaDoc.
% To regenerate it call "make genluadoc" in
% doc/generic/pgf/version-for-luatex/en.

\begin{filedescription}{pgflibrarygraphdrawing-table-helpers.lua}


\begin{luacommand}{{table.combine\textunderscore{}pairs}(\meta{table},\meta{combine\_func},\meta{initial\_value})}
Combine all key/value pairs of \meta{table} to a single value using a combine function.  This is a very powerful function. It can be used for combining the key/value pairs of a table into a single string but can also be used to compute mathematical operations on tables, such as finding the maximum value in a table etc.  The main difference to |table.combine_values| is that keys and values are used to determine the combination value and that the key/value pairs are are passed to \meta{combine\_func} in a random order. 

Parameters:
\begin{parameterdescription}
	\item[\meta{table}] Table to iterate over.\item[\meta{combine\_func}] Function to be called for each key/value pair. It takes three parameters, the current combination value and the key/value pair. It is supposed to return a new combination value.\item[\meta{initial\_value}] Initial combination value. 
\end{parameterdescription}


Return value:
\begin{parameterdescription} 
  \item[] The final combination value after all key/value pairs have been passed over to \meta{combine\_func}. 
\end{parameterdescription}


\end{luacommand}
\begin{luacommand}{{table.combine\textunderscore{}values}(\meta{input},\meta{combine\_func},\meta{initial\_value})}
Combine all values of \meta{input} to a single value using a combine function.  This is a very powerful function. It can be used for combining the values of a table into a single string but can also be used to compute mathematical operations on tables, such as finding the maximum value in a table etc.  The main difference to |table.combine_pairs| is that the keys are ignored and that the values are passed to \meta{combine\_func} in the order they appear in the table. 

Parameters:
\begin{parameterdescription}
	\item[\meta{input}] Table to iterate over.\item[\meta{combine\_func}] Function to be called for each value. It takes two parameters, the current combination value and the current value. It is supposed to return a new combination value.\item[\meta{initial\_value}] Initial combination value. 
\end{parameterdescription}


Return value:
\begin{parameterdescription} 
  \item[] The final combination value after all values of \meta{input} have been passed over to \meta{combine\_func}. 
\end{parameterdescription}


\end{luacommand}
\begin{luacommand}{{table.copy}(\meta{source},\meta{target})}
Copies a table while preserving its metatable. 

Parameters:
\begin{parameterdescription}
	\item[\meta{source}] The table to copy.\item[\meta{target}] The table to which values are to be copied or |nil| if a new table is to be allocated. 
\end{parameterdescription}


Return value:
\begin{parameterdescription} 
  \item[] The \meta{target} table or a newly allocated table containing all keys and values of the \meta{source} table. 
\end{parameterdescription}


\end{luacommand}
\begin{luacommand}{{table.count\textunderscore{}pairs}(\meta{input})}
Count the key/value pairs in the table. 

Parameters:
\begin{parameterdescription}
	\item[\meta{input}] The table whose key/value pairs to count. 
\end{parameterdescription}


Return value:
\begin{parameterdescription} 
  \item[] Number of key/value pairs in the table. 
\end{parameterdescription}


\end{luacommand}
\begin{luacommand}{{table.filter\textunderscore{}keys}(\meta{table},\meta{filter\_func})}
Copies a table and filters out all keys using a function. 

Parameters:
\begin{parameterdescription}
	\item[\meta{table}] The table whose values are to be filtered.\item[\meta{filter\_func}] The test function to be called for each key of \meta{table}. If it returns |false| or |nil| for a key, that key will not be part of the result table. 
\end{parameterdescription}


Return value:
\begin{parameterdescription} 
  \item[] Copy of \meta{table} with its keys filtered using \meta{filter\_func}. 
\end{parameterdescription}


\end{luacommand}
\begin{luacommand}{{table.filter\textunderscore{}pairs}(\meta{table},\meta{filter\_func})}
Copies a table and filters out all key/value pairs using a function. 

Parameters:
\begin{parameterdescription}
	\item[\meta{table}] The table whose values are to be filtered.\item[\meta{filter\_func}] The test function to be called for each pair of \meta{table}. If it returns |false| or |nil| for a pair, that pair will not be part of the result table. 
\end{parameterdescription}


Return value:
\begin{parameterdescription} 
  \item[] Copy of \meta{table} with its pairs filtered using \meta{filter\_func}. 
\end{parameterdescription}


\end{luacommand}
\begin{luacommand}{{table.filter\textunderscore{}values}(\meta{input},\meta{filter\_func})}
Copies a table and filters out all values using a function. 

Parameters:
\begin{parameterdescription}
	\item[\meta{input}] The table whose values are to be filtered.\item[\meta{filter\_func}] The test function to be called for each value of the input table. If it returns |false| or |nil| for a value, that value will not be part of the result table. 
\end{parameterdescription}


Return value:
\begin{parameterdescription} 
  \item[] Copy of \meta{input} with its values filtered using \meta{filter\_func}. 
\end{parameterdescription}


\end{luacommand}
\begin{luacommand}{{table.find}(\meta{table},\meta{find\_func})}
Returns the first value in \meta{table} for which \meta{find\_func} returns |true|. 

Parameters:
\begin{parameterdescription}
	\item[\meta{table}] The table to search in.\item[\meta{find\_func}] A function to test values with. It receives a single parameter (a value of \meta{table}) and is supposed to return either |true| or |false|. 
\end{parameterdescription}


Return value:
\begin{parameterdescription} 
  \item[] The first value of \meta{table} for which \meta{find\_func} returns true. Returns |nil| if the function was |false| for al of the values in \meta{table}. 
\end{parameterdescription}


\end{luacommand}
\begin{luacommand}{{table.find\textunderscore{}index}(\meta{table},\meta{find\_func})}
Returns the index of the first value in \meta{table} for which \meta{find\_func} returns |true|. 

Parameters:
\begin{parameterdescription}
	\item[\meta{table}] The table to search in.\item[\meta{find\_func}] A function to test values with. It receives a single parameter (a value of \meta{table}) and is supposed to return either |true| or |false|. 
\end{parameterdescription}


Return value:
\begin{parameterdescription} 
  \item[] Index of the first value of \meta{table} for which \meta{find\_func} returns |true|. Returns |nil| if the function was |false| for all of the values in \meta{table}. 
\end{parameterdescription}


\end{luacommand}
\begin{luacommand}{{table.key\textunderscore{}iter}(\meta{table})}
Iterate over all keys of a table in random order. 

Parameters:
\begin{parameterdescription}
	\item[\meta{table}] The table whose keys to iterate over. 
\end{parameterdescription}


Return value:
\begin{parameterdescription} 
  \item[] An iterator for the keys of the table. 
\end{parameterdescription}


\end{luacommand}
\begin{luacommand}{{table.map}(\meta{input},\meta{map\_func})}
Maps key/value pairs of an \meta{input} table to a flat table of new values. 

Parameters:
\begin{parameterdescription}
	\item[\meta{input}] Table whose key/value pairs are to be mapped to new values.\item[\meta{map\_func}] The mapping function to be called for each key/value pair of \meta{input}. The value it returns for a pair will be inserted into the result table. 
\end{parameterdescription}


Return value:
\begin{parameterdescription} 
  \item[] A new table containing all values returned by \meta{map\_func} for the key/value pairs of the \meta{input} table. 
\end{parameterdescription}


\end{luacommand}
\begin{luacommand}{{table.map\textunderscore{}keys}(\meta{table},\meta{map\_func})}
Maps keys of a table to new keys in a copy of the table. 

Parameters:
\begin{parameterdescription}
	\item[\meta{table}] The table whose keys are to be mapped to new keys.\item[\meta{map\_func}] A function to be called for each key of \meta{table} in order to generate a new key to replace the old one in the result table. 
\end{parameterdescription}


Return value:
\begin{parameterdescription} 
  \item[] A new table with all keys of \meta{table} having been replaced with the keys returned from \meta{map\_func}. The original values are preserved. 
\end{parameterdescription}


\end{luacommand}
\begin{luacommand}{{table.map\textunderscore{}pairs}(\meta{table},\meta{map\_func})}
Maps keys and values of a table to new pairs of keys and values. 

Parameters:
\begin{parameterdescription}
	\item[\meta{table}] The table whose key and value pairs are to be replaced.\item[\meta{map\_func}] A function to be called for each key and value pair of \meta{table} in order to generate a new pair to replace the old one. 
\end{parameterdescription}


Return value:
\begin{parameterdescription} 
  \item[] A new table with all key and value pairs of \meta{table} having been replaced with the pairs returned from \meta{map\_func}. 
\end{parameterdescription}


\end{luacommand}
\begin{luacommand}{{table.map\textunderscore{}values}(\meta{input},\meta{map\_func})}
Maps values of a table to new values in a new table. 

Parameters:
\begin{parameterdescription}
	\item[\meta{input}] The table whose values are to be mapped to new values.\item[\meta{map\_func}] A function to be called for each value in order to generate a new value to replace the old one in the result table. 
\end{parameterdescription}


Return value:
\begin{parameterdescription} 
  \item[] A new table with all values of the \meta{input} table having been replaced with the values returned from \meta{map\_func}. 
\end{parameterdescription}


\end{luacommand}
\begin{luacommand}{{table.merge}(\meta{table1},\meta{table2},\meta{first\_metatable})}
Merges the key/value pairs of two tables.  This function merges the key/value pairs of the two input tables.  All |nil| values of the first table are overwritten by the corresponding values of the second table.  By default the metatable of the second input table is applied to the resulting table. If \meta{first\_metatable} is set to |true| however, the metatable of the first input table will be used. 

Parameters:
\begin{parameterdescription}
	\item[\meta{table1}] First table with key/value pairs.\item[\meta{table2}] Second table with key/value pairs.\item[\meta{first\_metatable}] Whether to inherit the metatable of \meta{table1} or not. 
\end{parameterdescription}


Return value:
\begin{parameterdescription} 
  \item[] A new table with the key/value pairs of the two input tables merged together. 
\end{parameterdescription}


\end{luacommand}
\begin{luacommand}{{table.randomized\textunderscore{}pair\textunderscore{}iter}(\meta{table})}
Iterate over the key/value pairs of \meta{table} in a truely random order. 

Parameters:
\begin{parameterdescription}
	\item[\meta{table}] The table whose key/value pairs to iterate over. 
\end{parameterdescription}


Return value:
\begin{parameterdescription} 
  \item[] A randomized iterator for the values of \meta{table}. 
\end{parameterdescription}


\end{luacommand}
\begin{luacommand}{{table.randomized\textunderscore{}value\textunderscore{}iter}(\meta{table})}
Iterate over the values of \meta{table} in a truely random order. 

Parameters:
\begin{parameterdescription}
	\item[\meta{table}] The table whose values to iterate over. 
\end{parameterdescription}


Return value:
\begin{parameterdescription} 
  \item[] A randomized iterator for the values of the table. 
\end{parameterdescription}


\end{luacommand}
\begin{luacommand}{{table.remove\textunderscore{}values}(\meta{input},\meta{remove\_func})}
Removes all values from \meta{input} for which \meta{remove\_func} returns |true|.  Important note: this method does not work with dictionaries. Make sure only to process number-indexed arrays with it. 

Parameters:
\begin{parameterdescription}
	\item[\meta{input}] The table to remove values from.\item[\meta{remove\_func}] Function to be called for each value of \meta{input}. If it returns |false|, the value will be removed from the table in-place. 
\end{parameterdescription}


Return value:
\begin{parameterdescription} 
  \item[] \meta{input} which was edited in-place. 
\end{parameterdescription}


\end{luacommand}
\begin{luacommand}{{table.update\textunderscore{}values}(\meta{table},\meta{update\_func})}
Update values of \meta{table} in-place using an update function. 

Parameters:
\begin{parameterdescription}
	\item[\meta{table}] The table whose values are to be updated.\item[\meta{update\_func}] A function that takes two parameters, the key/value pairs of \meta{table} and returns a new value to replace the old one. 
\end{parameterdescription}


Return value:
\begin{parameterdescription} 
  \item[] The input \meta{table}. 
\end{parameterdescription}


\end{luacommand}
\begin{luacommand}{{table.value\textunderscore{}iter}(\meta{table})}
Iterate over all values of a table.  FIXME: The iterators stops if a key's value is nil. But we actually want to continue iterating until the end of the table. 

Parameters:
\begin{parameterdescription}
	\item[\meta{table}] The table whose values to iterate over. 
\end{parameterdescription}


Return value:
\begin{parameterdescription} 
  \item[] An iterator for the values of the table. 
\end{parameterdescription}


\end{luacommand}

\end{filedescription}
%% This file has been generated from the lua sources using LuaDoc.
% To regenerate it call "make genluadoc" in
% doc/generic/pgf/version-for-luatex/en.

\begin{filedescription}{pgflibrarygraphdrawing-iter-helpers.lua}


\begin{luacommand}{{iter.filter}(\meta{iterator},\meta{filter\_func})}
Skips all values of an iterator for which \meta{filter\_func} returns |false|. 

Parameters:
\begin{parameterdescription}
	\item[\meta{iterator}] Original \meta{iterator} of values.\item[\meta{filter\_func}] Filter function that takes a value of the original \meta{iterator} and is expected to return |false| if the value should be skipped. 
\end{parameterdescription}


Return value:
\begin{parameterdescription} 
  \item[] A modified iterator that skips values of \meta{iterator} for which \meta{filter\_func} returns |false|. 
\end{parameterdescription}


\end{luacommand}
\begin{luacommand}{{iter.map}(\meta{iterator},\meta{map\_func})}
Maps all values of an iterator to new values.  This function will cause loops to iterate over the values of the original \meta{iterator} replaced by the values returned from \meta{map\_func}. 

Parameters:
\begin{parameterdescription}
	\item[\meta{iterator}] Original iterator whose values are to be mapped to new ones.\item[\meta{map\_func}] Mapping function that takes a value of the original \meta{iterator} and maps it to a new value that is then returned to the loop instead. 
\end{parameterdescription}


Return value:
\begin{parameterdescription} 
  \item[] A modified iterator. 
\end{parameterdescription}


\end{luacommand}
\begin{luacommand}{{iter.times}(\meta{n})}
Causes a loop to run multiple times.  Use this iterator like this to perform 100 loops:\\ |for n in iter.times(100) do ... end|.  To iterate over the values $0, 10, 20, 30, ..., 100$ do:\\ |for n in iter.filter(iter.times(100), function (n) return n % 10 == 0 end)| 

Parameters:
\begin{parameterdescription}
	\item[\meta{n}] Number of loops. 
\end{parameterdescription}



\end{luacommand}

\end{filedescription}



 % Copyright 2008 by Till Tantau and Mark Wibrow
%
% This file may be distributed and/or modified
%
% 1. under the LaTeX Project Public License and/or
% 2. under the GNU Free Documentation License.
%
% See the file doc/generic/pgf/licenses/LICENSE for more details.

\section{Snakes and Decorations}
\label{section-base-snakes-and-decorations}

\subsection{Overview}

Snakes and decorations are two ways of ``making paths more fancy''. 
This could be as simple as extending a path with a ``zigzagged'' 
line\ldots 

\begin{codeexample}[]
\begin{tikzpicture}
  \draw[snake=zigzag] (0,0) -- (3,0);
  \draw[snake=zigzag] (0,-.5) -- (3,-1);
\end{tikzpicture}
\end{codeexample}

\ldots but could also be as complex as typesetting text along a path:

{\catcode`\|12
\begin{codeexample}[]
\begin{tikzpicture}
  \draw [red, dashed]
    [postaction={decoration=text, decoration text={Some text along a path}}]
    (0,2) .. controls (2,2) and (1,0) .. (3,0);
\end{tikzpicture}
\end{codeexample}
}

The most common use for a snake or a decoration is to ``repeat
something along a path'' (for example, the |zigzag| snake 
repeats \tikz\draw[snake=zigzag](0,0)--(\pgfsnakesegmentlength,0);
along a straight line). However, it not necessarily the case that
only one thing be repeated: a snake or a decoration can consist
of different parts, or \emph{segments}, repeated in a particular
order.

Before a snake or a decoration can be used, it must first be declared. 
This declaration contains a detailed description of how each segement 
looks like and in what ordering the different possible 
segments are used.


\subsubsection{Segments}


When you declare a snake or a decoration, you provide a description 
of how their different segments will be rendered. The description of
each segment should be given in a way as if the ``x-axis'' of the
segment is the tangent to the path at a particular point,
and that point is the origin of the segment.
Thus, for example, the segment of the |zigzag| snake might be
defined using the following code: 
\begin{codeexample}[code only]
\pgfpathlineto{\pgfpoint{5pt}{5pt}}
\pgfpathlineto{\pgfpoint{15pt}{-5pt}}
\pgfpathlineto{\pgfpoint{20pt}{0pt}}
\end{codeexample}

\pgfname\ will ensure that an appropriate coordinate transformation
is in place when the segment is rendered such that
the segment actually points in the right direction. Also
subsequent segments will be transformed such that they are
``further along the path'' toward the end of the path.
All transformations are setup automatically.

Note that we did not use a |\pgfpathmoveto{\pgfpointorigin}| at the
beginning of the segment code. Doing so would subdivide the path into
numerous subpaths. Rather, we assume that the previous segment caused
the current point to be at the origin.

While we can now describe segments, we still need a way to ``stop''
a snake or a decoration. Also, \pgfname\ needs to know the width of 
the different segments such that it can translate and rotate the 
next segment correctly.
Even though it would be possible to compute all this information
automatically, \pgfname\ does not do so and you have to provide the
information ``by hand'' for each segment. 


\subsubsection{Finite State Automata}

The |zagzag| snake only has one segment that is repeated again and
again. However, we might also like to have \emph{different} segments
and use rules to describe which segment should be used where. For
example, we might have special segments at the start and at the end.

Both decorations and snakes use a mechanism known in computer science 
as \emph{finite state automata} to describe which segment is used at a
particular point. The idea is the following: For the first segment we
start in a special \emph{state} called the \emph{initial state}. In
this state, and also in all other state later, \pgfname\ first
computes how much space is left on the path. That is, \pgfname\ keeps
track of the distance to the end of the path. Attached to each state 
there is a set of rules of the following form: ``If the remaining 
distance is less than $x$, switch to state~$q$.'' \pgfname\ checks 
for each of these rules whether it applies and, if so, immediately 
switches to state~$q$.

Only if none of the rules tell us to switch to another
state, \pgfname\ will execute the state's code. This code will add a
segment to the path. In addition to the rules there is also width
parameter attached to each state. \pgfname\ then translates the
coordinate system by this width and reduces the remaining distance.
Then, \pgfname\ either stays in the current state or switches to
another state, depending on yet another property attached of the
state.

The whole process stops when a special state called |final| is
reached. The segment of this state is added to the path immediately
(it is often empty, though) and the process ends.


\subsection{Snakes}

\label{section-base-snakes}

A snake is used to extend the path and the
commands for using snakes start with |\pgfpath|. However, snakes do
not necessarily extend the path using line-to and curve-to operations;
rather, they can also contain move-to operations and, thereby, cause
the path to be split into many subpaths.

Once a snake has been declared, it can be used. For this you specify a
target point and a snaked line will be added from the last point
current point to the target point.

\subsubsection{The Snake Declaration Command}

\begin{command}{\pgfdeclaresnake\marg{name}\marg{initial
      state}\marg{states}}
  This command declares a new snake called \meta{name}. The
  \meta{states} argument contains a description of the snake
  automaton's states and the transitions between them. The
  \meta{initial state} is the state in which the automaton starts.

  The \meta{states} argument should consist of |\state| commands, one
  for each state of the snake automaton. The |\state| command is
  defined only when the \meta{states} argument is executed.

\begin{codeexample}[]
\pgfdeclaresnake{example snake}{initial}
{
  \state{initial}[width=10pt]
  {
    \pgfpathlineto{\pgfpoint{0pt}{5pt}}
    \pgfpathlineto{\pgfpoint{5pt}{5pt}}
    \pgfpathlineto{\pgfpoint{5pt}{-5pt}}
    \pgfpathlineto{\pgfpoint{10pt}{-5pt}}
    \pgfpathlineto{\pgfpoint{10pt}{0pt}}
  }
  \state{final}
  {
    \pgfpathlineto{\pgfpoint{\pgfsnakeremainingdistance}{0pt}}
  }
}
\tikz{\draw[snake=example snake] (0,0) -- (3,0);
      \draw[snake=example snake] (0,-1) -- (3,-3);}
\end{codeexample}

 The |\state| command works as follows:
  \begin{command}{\state\marg{name}\oarg{options}\marg{code}}
    This command declares a new state inside the current snake
    automaton. The state is names \meta{name}.

    When \pgfname\ is in state \meta{name}, the following things
    happen:
    \begin{enumerate}
    \item
      The \meta{options} are parsed. This may lead, see below, to a 
      state switch. When this happens, the following steps are not
      executed. The \meta{options} are executed one after the other in
      the given order. If an option causes a state switch, the switch
      is immediate, even if later options might cause a different
      state switch.
    \item
      The \meta{code} is executed. When this happens, the coordinate
      transformation will be setup such that the origin is at the
      start of the segment and such that the target lies in the
      direction ``east.''
    \item
      After the \meta{code} has been executed, the state switches to
      whatever state has been specified inside the \meta{options}
      using the |next state| option. If no |next state| has been
      specified, the state stays the same.
    \end{enumerate}

    The \meta{options} are executed with the key path set to
    |/pgf/snakes/|. The following keys are defined for this path: 
    \begin{key}{/pgf/snakes/switch if less than=\meta{dimension}| to |\meta{new state}}
      When this key is encountered, \pgfname\ checks whether the
      remaining distance to the target is less than
      \meta{dimension}. If so, an immediate state switch to \meta{new
        state} occurs.
    \end{key}
    \begin{key}{/pgf/snakes/switch if sub-path less than=\meta{dimension}| to |\meta{new state}}
      When this key is encountered, \pgfname\ checks whether the
      remaining distance to the target is less than
      \meta{dimension}. If so, an immediate state switch to \meta{new
        state} occurs.
    \end{key}
    \begin{key}{/pgf/snakes/width=\meta{dimension}}
      This option tells \pgfname\ the width of the segment if it is
      appended to the path. The coordinate system will, afterward, be
      translated by \meta{dimension} to ensure that the next segment
      starts where the current one ended.

      In addition, this option will cause an immediate switch to the
      state |final| if the remaining distance is less than
      \meta{dimension}. The effect is the same as if you had said
      |switch if less than=|\meta{dimension}| to final| just before
      the |width| option.
    \end{key}
    \begin{key}{/pgf/snakes/next state=\meta{new state}}
      After the segment has been added to the path, a state switch to
      \meta{new state} is performed. If this option is not given, the
      next state is the same as the current state.
    \end{key}
  
    There are two dimensions that are useful inside snake automata:
    \begin{command}{\pgfsnakeremainingdistance}
      This \TeX\ dimension holds the remaining distance to the target.
    \end{command}
    \begin{command}{\pgfsnakecompleteddistance}
      This \TeX\ dimension holds the distance already completed on the
      snake. 
    \end{command}
  \end{command}

  As a final example we present a more complicated snake that makes
  use of the different options:
\begin{codeexample}[]
\pgfdeclaresnake{complicated example snake}{initial}
{
  \state{initial}[width=5pt,next state=up]
  { \pgfpathlineto{\pgfpoint{5pt}{0pt}} }
  
  \state{up}[width=5pt,next state=down]
  {
    \ifdim\pgfsnakeremainingdistance>\pgfsnakecompleteddistance
      % Growing
      \pgfpathlineto{\pgfpoint{0pt}{\pgfsnakecompleteddistance}}
      \pgfpathlineto{\pgfpoint{5pt}{\pgfsnakecompleteddistance}}
      \pgfpathlineto{\pgfpoint{5pt}{0pt}}
    \else
      % Shrinking
      \pgfpathlineto{\pgfpoint{0pt}{\pgfsnakeremainingdistance}}
      \pgfpathlineto{\pgfpoint{5pt}{\pgfsnakeremainingdistance}}
      \pgfpathlineto{\pgfpoint{5pt}{0pt}}
    \fi%      
  }
  \state{down}[width=5pt,next state=up]
  {
    \ifdim\pgfsnakeremainingdistance>\pgfsnakecompleteddistance
      % Growing
      \pgfpathlineto{\pgfpoint{0pt}{-\pgfsnakecompleteddistance}}
      \pgfpathlineto{\pgfpoint{5pt}{-\pgfsnakecompleteddistance}}
      \pgfpathlineto{\pgfpoint{5pt}{0pt}}
    \else
      % Shrinking
      \pgfpathlineto{\pgfpoint{0pt}{-\pgfsnakeremainingdistance}}
      \pgfpathlineto{\pgfpoint{5pt}{-\pgfsnakeremainingdistance}}
      \pgfpathlineto{\pgfpoint{5pt}{0pt}}
    \fi%      
  }
  \state{final}
  {
    \pgfpathlineto{\pgfpoint{\pgfsnakeremainingdistance}{0pt}}
  }
}
\begin{tikzpicture}[snake=complicated example snake]
  \draw[snake] (0,0) -- (3,0);
  \fill[red!50,snake,rounded corners=2pt] (.5,-2) -- ++(2.5,-2.5)
                             [snake=none] -- ++(0,-.5) -| (0,-2) -- cycle;
\end{tikzpicture}
\end{codeexample}
\end{command}

\subsubsection{Predefined Snakes}

Only two very simple and basic snakes are predefined when you load
\pgfname. For more interesting snakes you can use the package
|pgflibrarysnakes|.

\begin{snake}{lineto}
  This most trivial of all snakes is simply a straight line. This
  snake is typically not used in a ``stand alone'' fashion. Rather, it
  is useful as a ``subsnake'' of the |\pgfsnakesto| command.
\end{snake}

\begin{snake}{moveto}
  Arguably, this snake is even simpler than the previous snake. This
  snake consists of a simple move-to operation. Like the |lineto|
  snake, it is useful as a ``subsnake'' of the |\pgfsnakesto|
  command. 
\end{snake}


\subsubsection{Snake Parameters}

The appearance of snakes is usually influenced by the current settings
of a whole bunch of keys. For efficiency reasons, the following keys
(which are the most important ones) do not actually store their
setting ``inside themselves,'' but set \TeX\ macros and dimensions:
\begin{itemize}
\item |/pgf/segment amplitude| sets the \TeX-dimension |\pgfsnakesegmentamplitude|
\item |/pgf/segment length| sets the \TeX-dimension |\pgfsnakesegmentlength|
\item |/pgf/segment angle| sets the \TeX-macro |\pgfsnakesegmentangle|
\item |/pgf/segment aspect| sets the \TeX-macro   |\pgfsnakesegmentaspect|
\item |/pgf/segment object length| sets the \TeX-macro (!) |\pgfsnakesegmentobjectlength|
\end{itemize}

This means that when you define your own snake, you should refer to
the above macros, not the the keys.



\subsubsection{Using Snakes}

Three commands can be used to use a snake.

\begin{command}{\pgfpathsnakealongvector\marg{snake}\marg{length}\marg{vector}}
  This command will append the \meta{snake} to the current path. The
  length of the snake is given by \meta{length}. The \meta{vector}
  should be a normalizes vector (a vector having length |1pt|)
  pointing in the direction in which the snake should grow.
\begin{codeexample}[]
\begin{tikzpicture}
  \draw[help lines] (0,0) grid (3,2);
  \pgfpathmoveto{\pgfpoint{1cm}{1cm}}
  \pgfpathsnakealongvector{zigzag}{2cm}{\pgfpointpolar{30}{1pt}}
  \pgfusepath{stroke}
\end{tikzpicture}    
\end{codeexample}
\end{command}

\begin{command}{\pgfpathsnakesto\marg{snake list}\marg{target}}
  This command will append the snakes in the \meta{snake list}  to the
  current path such that it ends at \meta{point}. This command calls the previous 
  one (repeatedly, possibly) after having computed the distance from
  the current point to \meta{target} and normalized the vector
  connecting the current point to the target.

  The \meta{snake list} is a comma-separated list of pairs consisting
  of a snake name in curly braces and a distance in curly braces. For
  example, a snake list might be |{lineto}{1cm},{moveto}{2cm}|. This
  is interpreted as follows: Use a |lineto| snake for the first
  centimeter of the way to the \meta{target}, then use the |moveto|
  snake for the next two centimeters.

  No attempt is made to check or ensure that the distances of the
  ``subsnakes' add up to the distnace from the current point to the
  \meta{target}. However, it is possible to refer to this distance
  inside the \meta{snake list}: While the snake list is processed, the
  two \TeX-dimensions |\pgfsnakeremainingdistance| and
  |\pgfsnakecompleteddistance| will be set to the correct values. For
  example, consider the example from a above, consisting of a |lineto|
  and a |moveto| snake. Suppose the distance to \meta{target} where
  5cm. Then, when the length of the first subsnake is computed, the
  remaining distance willbe set to 5cm and the completed distance to
  0cm. When the length of the second subsnake is computed, the
  remaining distance wil be 4cm and the completed distance will be
  1cm.

  Here are some useful examles: |{zigzag}{\pgfsnakeremainingdistance}|
  is a \meta{snake list} that consists only of a |zigzag| snake whose
  length is the total distance from the current point to the
  \meta{target}. Next, to get a snake that is a |zigzag| snake that is
  preceeded and succeeded by 3mm of |lineto|, you can use the
  following:
\begin{codeexample}[]
\begin{tikzpicture}
  \draw[help lines](0,0) grid (3,2);
  \pgfpathmoveto{\pgfpointorigin}
  \pgfpathsnakesto{%
    {lineto}{3mm},%
    {zigzag}{\pgfsnakeremainingdistance-6mm},%
    {lineto}{3mm}}
  {\pgfpoint{3cm}{2cm}}
  \pgfusepath{stroke}
\end{tikzpicture}
\end{codeexample}

  Note that the computation of the distance may be imprecise. In
  general, the placement precision of the snakes will not be perfect. 
\end{command}

\begin{command}{\pgfpathsnaketo\marg{snake}\marg{target}}
  This command is just a shortcut for calling |\pgfpathsnakesto| for a
  single snake whose length is |\pgfsnakeremainingdistance|.
\begin{codeexample}[]
\begin{tikzpicture}
  \draw[help lines] (0,0) grid (3,2);
  \pgfpathmoveto{\pgfpoint{1cm}{1cm}}
  \pgfpathsnaketo{zigzag}{\pgfpoint{3cm}{2cm}}
  \pgfusepath{stroke}
\end{tikzpicture}    
\end{codeexample}
\end{command}

As was already mentioned, when each segment of the snake is added to
the path, an appropriate coordinate transformation will be in
force. It is sometimes useful to add an additional transformation
locally. For example, by reflecting everything around the $x$-axis
right before each segment is added, the snake will effectively be
mirrored along the path. The following command allows you to install
such a ``last minute transformation.''

\begin{command}{\pgfsetsnakesegmenttransformation\marg{code}}
  The \meta{code} will be executed at the very beginning of each
  segment. Normally, this be a transformation command that changes the
  $y$-axis in some way.
\end{command}


\subsection{Decorations}
\label{section-base-decorations}

Decorations are ``grown up'' snakes. Like snakes,
decorations consist of segments and are declared as finite 
state automata. However, whereas snakes can
only be used to extend a path in a straight line, decorations
are much more flexible. Decorations can also extend a path in a
``snake-like'' manner (everthing that can be done with a snake
can be done with a decoration), but they are not limited to
single straight lines. Multiple paths consiting of lines, and 
curves can be decorated. 

In addition, a decoration can stroke or fill the path it creates 
allowing for different colors, line widths, fills and shades all 
within the same decoration. Alternatively, a decoration can not 
create a path at all: the |text| decoration simply typesets text
along a path.
Further versitility is provided by ``meta-decorations''. These
are automata that decorate a path with decorations.

Naturally, all of this comes at some cost. Measuring the 
length of an arbitrary path, and moving along it is very tricky 
to do accurately and quickly in \TeX. So you should be aware that 
decorations can be much slower than snakes. Also, some inaccuracies
in positioning may be noticable when crossing a subpath boundary.

In order to prevent confusion with different types of path, such
as those that are extended, those that are decorated and those that 
are created, the following conventions will be used:

\begin{itemize}
	\item 
		the \emph{existing} path refers to any unused path in existence 
		before a decoration environment is encountered.
	\item
		the \emph{decorated} path refers to ``the path that is to be 
		decorated''	or ``the path that has been decorated'', that is, 
		the path that	the decoration automaton moves along. The decorated
		path may consist of many line and curve subpaths (for example,
		a circle or an ellipse consists of four curves).
	\item
		the \emph{decoration} path refers to ``the path that the decoration 
		makes'', that is, the path created by the decoration automaton.
		When the automaton terminates, this path may be empty (as it has
		been used by the decoration).
\end{itemize}

\subsubsection{The Decoration Declaration Command}

As with snakes, decorations must be declared. The command for
declaring decorations is very similar to the |\pgfdeclaresnake| 
command, and operates in the same way. However, a state in a
decoration automaton has more keys, macros and dimensions
avaiblable, than a state in a snake automaton.

\begin{command}{\pgfdeclaredecoration\marg{name}\marg{initial
      state}\marg{states}}
  This command declares a new decoration called \meta{name}. The
  \meta{states} argument contains a description of the decoration
  automaton's states and the transitions between them. The
  \meta{initial state} is the state in which the automaton starts.
  
\begin{codeexample}[]
\pgfdeclaredecoration{stars}{initial}{
  \state{initial}[width=15pt]
  {
  	\pgfmathparse{round(rnd*100)}
  	\pgfsetfillcolor{yellow!\pgfmathresult!orange}
    \pgfsetstrokecolor{yellow!\pgfmathresult!red}
    \pgfnode{star}{center}{}{}{\pgfusepath{stroke,fill}}
  }
  \state{final}
  {
  	\pgfpathmoveto{\pgfpointdecoratedpathlast}
  }
}
\tikz\path[decoration=stars, star point ratio=2, star points=5,
           inner sep=0, minimum size=rnd*10pt+2pt]            
  (0,0) .. controls (0,2)  and (3,2)  .. (3,0)
        .. controls (3,-3) and (0,0)  .. (0,-3)
        .. controls (0,-5) and (3,-5) .. (3,-3);
\end{codeexample}

  The |\state| command works as follows:
  \begin{command}{\state\marg{name}\oarg{options}\marg{code}}
    This command declares a new state inside the current decoration
    automaton. The state is names \meta{name}.

    The \meta{options} are executed with the key path set to
    |/pgf/decoration/|. The following keys are defined for this path: 
    \begin{key}{/pgf/decoration/switch if less than=\meta{dimension}| to |\meta{new state}}
      When this key is encountered, \pgfname\ checks whether the
      remaining distance to the end of the decorated path is less than
      \meta{dimension}. If so, an immediate state switch to \meta{new
        state} occurs.
    \end{key}
    \begin{key}{/pgf/decoration/switch if subpath less than=\meta{dimension}| to |\meta{new state}}
      When this key is encountered, \pgfname\ checks whether the
      remaining distance to the end of the current decorated subpath is 
      less than \meta{dimension}. If so, an immediate state switch to 
      \meta{new state} occurs.
    \end{key}
    \begin{key}{/pgf/decoration/width=\meta{dimension}}
      This option tells \pgfname\ the width of the segment.
      The coordinate system will, afterward, be
      translated to a point which is \meta{dimension} along the 
      decorated path (and also rotated to an angle of the tangent
      to the decorated path).

      In addition, this option will cause an immediate switch to the
      state |final| if the remaining distance is less than
      \meta{dimension}. The effect is the same as if you had said
      |switch if less than=|\meta{dimension}| to final| just before
      the |width| option.
    \end{key}
    \begin{key}{/pgf/decoration/next state=\meta{new state}}
      After the code for state has been executed, a state switch to
      \meta{new state} is performed. If this option is not given, the
      next state is the same as the current state.
    \end{key}
    \begin{key}{/pgf/decoration/repeat state=\meta{repetitions} (initially 0)}
      Repeat the current state \meta{repetitions} times. This is reset
      if one of the |switch if| keys causes a state switch. 
      Note, that the maximum number of times the state will be executed 
      is $\meta{repetitions}+1$.
    \end{key}
    
		There are a number of macros and dimensions which may be useful
		inside a decoration automaton. The following macros are available:
		
		\begin{command}{\pgfdecoratedpathlength}
				The length of the decorated path. Note that if multiple
				decorations are being applied, this refers to the length
				of the current segment of the decorated path.
		\end{command}
		
		\begin{command}{\pgfdecoratedsubpathlength}
				The length of the current decorated sub-path.
		\end{command}
		
		\begin{command}{\pgfpointdecoratedpathlast}
				The final point of the decorated path.
		\end{command}
				
		\begin{command}{\pgfpointdecoratedsubpathlast}
				The final point of the current decorated sub-path.
		\end{command}
		
		\begin{command}{\pgfdecoratedangle}
		   The angle of the tangent to the decorated path at the \emph{origin}
		   of the current segment. The transformation matrix applied at 
		   the beginning of a state includes a rotation equivalent to 
		   this angle.
		\end{command}
		
		 The following dimension registers are also available inside the 
		 automaton:
		
		\begin{command}{\pgfdecoratedremainingdistance}
				The remaining distance on the decorated path. When multiple
				decorations are applied, this is the distance to the end
				of the segment of the decorated path that is currently being
				decorated.
		\end{command}
		
		\begin{command}{\pgfdecoratedcompleteddistance}
				The completed distance on the decorated path.
		\end{command}
		
		\begin{command}{\pgfdecoratedsubpathremainingdistance}
				The remaining distance on the current decorated sub-path.
		\end{command}
		
		\begin{command}{\pgfdecoratedsubpathcompleteddistance}
				The completed distance on the current decorated sub-path.
		\end{command}
	\end{command}
\end{command}


\subsubsection{Predefined Decorations}

Only two very simple decorations are predefined when you 
load \pgfname{}. For more interesting decorations you can load
the library |pgflibrarydecorations|.

\begin{decoration}{lineto}

	This decoration is analogous to the snake |lineto|: it simply 
	decorates the path as if it was stroked. In reality, lots
	of very small stright lines are used. This decoration is quite
	handy when using mutiple decorations, or as part of a
	meta-decoration.

\end{decoration}


\begin{decoration}{moveto}

 This decoration is analogous to the snake |moveto|, it just
 moves along the path (actually in very small steps). Again,
 this decoration is most useful in multiple decorations or in
 a meta-decoration.
 
\end{decoration}

\subsubsection{Decoration Parameters}

Like snakes, the appearance of decorations can influenced by the 
settings of various keys. They are analogous to the keys used 
by snakes, and similarly, do not actually store their
setting ``inside themselves,'' but set \TeX\ macros and dimensions.

\begin{itemize}
	\item |/pgf/decoration segment amplitude| 
		sets the \TeX-dimension |\pgfdecorationsegmentamplitude|
		
	\item |/pgf/decoration segment length| 
		sets the \TeX-dimension |\pgfdecorationsegmentlength|
		
	\item |/pgf/decoration segment angle| 
		sets the macro |\pgfdecorationsegmentangle|
		
	\item |/pgf/decoration segment aspect| 
		sets the macro |\pgfdecorationsegmentaspect|
		
	\item |/pgf/decoration segment object length| 
		sets the macro |\pgfdecorationsegmentobjectlength|
\end{itemize}

When defining decorations, you should refer to the above macros, 
not the the keys.


\subsubsection{Using Decorations}

One of the biggest difference between snakes and decorations
is in the way they are used in a picture. Whereas snakes
extend an existing path using |\pgfpath| commands, a decoration 
uses the |{pgfdecorate}| environment, in which a path is specified 
and then decorated, creating and possibly using other paths. 


\begin{codeexample}[]
\begin{pgfpicture}
  \pgfdecorationsegmentlength.125cm
  \begin{pgfdecoration}{{zigzag}{\pgfdecoratedpathlength}}
    \pgfpathmoveto{\pgfpoint{1.5cm}{1cm}}
    \pgfpatharc{45}{270}{1.5cm and 1cm}
    \pgfpatharc{90}{-135}{1.5cm and 1cm}
  \end{pgfdecoration}
  \pgfusepath{stroke}
\end{pgfpicture}
\end{codeexample}

\begin{environment}{{pgfdecoration}\marg{decoration list}}
	This environment decorates the path specified in 
	\meta{environment contents} with the 
	decorations specified in \marg{decoration list}. In effect, the
	decorated path is divided into segments, each with its own 
	decoration.
	
	When this environment is entered, the sequence of 
	events will be as follows:
		\begin{enumerate}
			\item
				Any existing unused path is saved.
			\item 
				The path commands specified in \meta{environment contents} are
				executed and this resulting path (the ``decorated'' path) is 
				saved. 
			\item
				The saved existing path is reinstalled.
			\item
				The decoration automaton moves along the decorated path using
				the automaton \meta{name}, thus creating (and
				possibly using) the ``decoration'' path.
		\end{enumerate}
	 
	Some important points should be noted regarding the use of this
	environment:
	
	\begin{itemize}
		\item
			If \meta{environment contents} does not begin with 
			|\pgfpathmoveto|,	the last known point on the existing path is 
			assumed as the starting point.
		\item
			All except the last of any sequence of consecutive move-to commands 
			in \meta{environment contents} are discarded.
		\item
			Any move-to commands at end of \meta{environment contents} are 
			ignored.
		\item
			Any close-path commands on the decorated path are interpreted as 
			straight lines.
			Internally somthing a little more complicated is going on,
			however, a closed path on the decorated path has no effect on the 
			decoration path, other than causing the automaton to travel in a 
			straight line towards the location of the last move-to command on 
			the decorated path.
		\item
			Although a the decorated path assumes the last known point on the
			path, it does not automatically insert a move-to for the decoration
			path. 
			If a decoration	path commences with a line-to or curve-to when the 
			existng path is empty, an appropriate move-to command should be 
			inserted before the decoration commences (this can be done quite
			simply --- see below).
		\item
			If a decoration uses its own path, any unused existing path will
			be used by the decoration. 
	\end{itemize}
	
	
	The format for each item in \marg{decoration list} is as follows:
	
	\marg{name}\marg{length}\marg{before code}\marg{after code}
	
	This instructs \pgfname{} to decorate the path for a distance
	of \marg{length} using the decoration \meta{name}. Before the
	decoration is applied \meta{before code} is executed. After the
	decoration has finished, \meta{after code} is executed.
			
\begin{codeexample}[]
\begin{tikzpicture}[decoration segment length=5pt]
  \draw [help lines] grid (3,2);
  \begin{pgfdecoration}{{lineto}{1cm},{zigzag}{2cm},{lineto}{1cm}}
    \pgfpathmoveto{\pgfpointorigin}
    \pgfpathcurveto
      {\pgfpoint{0cm}{2cm}}{\pgfpoint{3cm}{2cm}}{\pgfpoint{3cm}{0cm}}
  \end{pgfdecoration}
\pgfusepath{stroke}
\end{tikzpicture}
\end{codeexample}

	The above example illustrates an important point: if the sum of 
	the lengths specified in \meta{decoration list} is not equal to 
	the length of the decorated path then the decoration will simply 
	stop at the point it has reached (if the sum of the lengths is 
	greater than the decorated path length, later decorations in the 
	list are simply ignored). 
		
	Fortuntely when the lengths are evaluated, the dimension
	|\pgfdecoratedremainingdistance| holds the remaining distance on
	the entire decorated path, and |\pgfdecoratedpathlength| holds the
	total length of the path. Thus, it is possible to specify lengths
	like |\pgfdecoratedpathlength/3|.

\begin{codeexample}[]
\begin{tikzpicture}[decoration segment length=5pt]
  \draw [help lines] grid (3,2);
  \begin{pgfdecoration}{
      {lineto}{\pgfdecoratedpathlength/3},
      {zigzag}{\pgfdecoratedpathlength/3},
      {lineto}{\pgfdecoratedremainingdistance}
    }
    \pgfpathmoveto{\pgfpointorigin}
    \pgfpathcurveto
      {\pgfpoint{0cm}{2cm}}{\pgfpoint{3cm}{2cm}}{\pgfpoint{3cm}{0cm}}
  \end{pgfdecoration}
  \pgfusepath{stroke}
\end{tikzpicture}
\end{codeexample}

	When \meta{before code} is executed, the macro 
	|\pgfpointdecoratedpathfirst| will return the point corresponding
	to the start of the current segment of the decorated path.
	When \meta{after code} is executed the macro 
	|\pgfpointdecoratedpathlast| returns the point corresponding to
	the end of the current segment of the decorated path.
	This means that if decorations do not use their own path, it is 
	possible to do something with them and and continue from the
	correct place. 
	
\begin{codeexample}[]
\begin{tikzpicture}
  \draw [help lines] grid (3,2);
  \begin{pgfdecoration}{
	    {lineto}{\pgfdecoratedpathlength/3}
	      {}
	      {
	        \pgfusepath{stroke}
	      },
	    {zigzag}{\pgfdecoratedpathlength/3}
  	    {
	        \pgfpathmoveto{\pgfpointdecoratedpathfirst}
	        \pgfdecorationsegmentlength=5pt
	      }
	      {
	        \pgfsetstrokecolor{red}
	        \pgfusepath{stroke}
	        \pgfpathmoveto{\pgfpointdecoratedpathlast}
	        \pgfsetstrokecolor{black}
	      },
	    {lineto}{\pgfdecoratedremainingdistance}
    }
    \pgfpathmoveto{\pgfpointorigin}
    \pgfpathcurveto
      {\pgfpoint{0cm}{2cm}}{\pgfpoint{3cm}{2cm}}{\pgfpoint{3cm}{0cm}}
  \end{pgfdecoration}
\pgfusepath{stroke}
\end{tikzpicture}
\end{codeexample}
	
	If \meta{before code} and \meta{after code} are 
	empty they can be omitted. However, if \meta{before code} is
	empty, braces must be included if \meta{after code} is not,
	for example, |{zigzag}{1cm}{}{\pgfusepath{stroke}|.
	
	After the |{decoration}| environment has finished, the following 
	macros are available:
	
	\begin{command}{\pgfdecorateexistingpath}
		Any unused path existing before the environment was entered.
	\end{command}
	
	\begin{command}{\pgfdecoratedpath}
		The path that was decorated.
	\end{command}
	
	\begin{command}{\pgfdecorationpath}
	  Any unused path created by the automaton.
	\end{command}
	
	\begin{command}{\pgfpointdecoratedpathlast}
		The final point of the decorated path.
	\end{command}
	
	\begin{command}{\pgfpointdecorationpathlast}
		The final point of any unused decoration path.
	\end{command}

\end{environment}

\begin{plainenvironment}{{pgfdecoration}\marg{name}}
  The plain-\TeX{} version of the |{pgfdecorate}| environment.
\end{plainenvironment}

\begin{contextenvironment}{{pgfdecoration}\marg{name}}
  The Con\TeX t version of the |{pgfdecoration}| environment.
\end{contextenvironment}

For convenience, the following macros provide a ``shorthand''
for decorations (internally, they all use the |{pgfdecoration}|
environment).

\begin{command}{\pgfdecoratepath\marg{name}\marg{path commands}}
	Decorate the path described by \meta{path commands} with the
	decoration \meta{name}. This is equivalent to
\end{command}

\begin{command}{\pgfdecoratecurrentpath\marg{name}}
	Decorate the existing path with the	decoration \meta{name}.
\end{command}

Both the above commands use the current definitons of the following
macros:

\begin{command}{\pgfdecoratebeforecode}
  Specifies code to be executed immediately before a decoration is 
  applied.
  When this macro is executed, |\pgfpointdecoratedpathfirst| will 
  have been set up to return the point corresponding to the start 
  of the current segment of the decorated path.
\end{command}

\begin{command}{\pgfdecorateaftercode}
  Specifies code to be executed immediately after a decoration is 
  applied.
  When this macro is executed, |\pgfpointdecoratedpathlast| will 
  have been set up  to return the point corresponding	to the end
  of the current segment of the decorated path.
\end{command}


It may sometimes be useful to add an additional transformation
for each segment of a decoration. The following command allows 
you to define such a ``last minute transformation.''

\begin{command}{\pgfsetdecorationsegmenttransformation\marg{code}}
  The \meta{code} will be executed at the very beginning of each
  segment. Note when applying multiple decorations, this will
  be reset between decorations, so needs to be specified for
  each decortion.

\begin{codeexample}[]
\begin{tikzpicture}
  \draw [help lines] grid (3,2);
  \begin{pgfdecoration}{
	    {lineto}{\pgfdecoratedpathlength/3},
	    {zigzag}{\pgfdecoratedpathlength/3}
        {
          \pgfdecorationsegmentlength=5pt
          \pgfsetdecorationsegmenttransformation{\pgftransformyshift{.5cm}}
        },
	    {lineto}{\pgfdecoratedremainingdistance}
    }
    \pgfpathmoveto{\pgfpointorigin}
    \pgfpathcurveto
      {\pgfpoint{0cm}{2cm}}{\pgfpoint{3cm}{2cm}}{\pgfpoint{3cm}{0cm}}
  \end{pgfdecoration}
\pgfusepath{stroke}
\end{tikzpicture}
\end{codeexample}
  
\end{command}


\subsubsection{The Meta-Decorations command}
\label{section-base-meta-decorations}

A meta-decoration provides an alternative way to decorate a path with 
mutiple decorations. It is, in essence, an automaton that decorates
a path with decoration automatons. In general, however, the end effect 
is still that a path is decorated with other paths, and the decorated 
path should be thought of as being divided into segments each with 
their own decoration. Like snakes and ordinary decorations, before a 
meta-decoration can be used it must be declared.

\begin{command}{\pgfdeclaremetadecorate\marg{name}\marg{initial state}\marg{states}}

	This command declares a new meta-decoration called \meta{name}. The
  \meta{states} argument contains a description of the meta-decoration
  automaton's states and the transitions between them. The
  \meta{initial state} is the state in which the automaton starts.
  
  The |\state| command is similar to the one found in snake and
  decoration declarations, and takes the same form:
  
  \begin{command}{\state\marg{name}\oarg{options}\marg{code}}
  	Declares the state \meta{name} inside the current meta-decoration
    automaton.
    Unlike states in snakes or decorations, states in 
	  meta-decorations are not executed within a group. This permits a 
	  rather useful trick: By using an initial state with |width=0pt| some
	  calculations can be done in advance which could speed the execution
	  of the meta-decoration.
    
    The \meta{options} are executed with the key path set to
    |/pgf/meta-decorations/|, and the following keys are defined for 
    this path: 
    
    \begin{key}{/pgf/meta-decoration/switch if less than=\meta{dimension}| to |\meta{new state}}
      This causes \pgfname\ to check whether the
      remaining distance to the end of the decorated path is less than
      \meta{dimension}, and If so, to immediatly switch to the state 
      \meta{new state}. When this key is evaluated, the macro 
      |\pgfmetadecoratedpathlength| will be defined as the total length of 
      the decoration path, allowing for values such as
      |\pgfmetadecoratedpathlength/8|.
    \end{key}
    
    \begin{key}{/pgf/meta-decoration/width=\meta{dimension}}
      This option tells \pgfname\ the width of the segment, that is,
      the length of the path which the decoration automaton specified 
      in \meta{code} will decorate.
      
      In addition, this option will cause an immediate switch to the
      state |final| if the remaining distance is less than
      \meta{dimension}. 
    \end{key}
    
    \begin{key}{/pgf/meta-decoration/next state=\meta{new state}}
      After the code for a state has been executed, a state switch to
      \meta{new state} is performed. If this option is not given, the
      next state is the same as the current state.
    \end{key}
    
    The code in \meta{code} is quite different from the code in a 
    snake or a decoration state. In almost all cases only the 
    following three macros will be required:
    
    \begin{command}{\decoration\marg{name}}
    	This sets the decoration for the current state to \meta{name}.
    	If this command is omitted, the |moveto| decoration will be
    	used.
    \end{command}
    
    \begin{command}{\beforedecoration\marg{code}}
    	Defines \marg{code} as (typically) \pgfname{} commands to be
    	executed before the decoration is applied to the current segment.
    	This command can be omited. When this code is executed,
    	the macro |\pgfpointmetadecoratedpathfirst| will be set up as
    	the first point on the current segment of the decorated path.
    	If you wish to set up some decoration specific parameters 
    	such as segment length, or segment amplitude, then they
    	can be set in \meta{code}.
    	
    \end{command}
    
    \begin{command}{\afterdecoration\marg{code}}
    	Defines \marg{code} as commands to be	executed afer the 
    	decoration has been applied to the current segment.
    	This command can be omited.
    	When this code is executed, |\pgfpointmetadecoratedpathlast| will 
    	be set up as the last point on the current segment of the 
    	decorated path.
    \end{command}
    
    There are some macros that may be usedful when creating 
    meta-decorations (note that they are all macros):
    
    \begin{command}{\pgfmetadecoratedpathlength}
      The entire length of the decorated path.
    \end{command}
    
    
    \begin{command}{\pgfmetadecoratedcompleteddistance}
    	The completed distance on the entire decorated path.
    \end{command}
    
    \begin{command}{\pgfmetadecoratedremainingdistance}
    	The remaining distance on the entire decorated path.
    \end{command}
    
    \begin{command}{\pgfmetadecoratedsubpathcompleteddistance}
    	The completed distance on the current decorated subpath.
    \end{command}
    
    \begin{command}{\pgfmetadecoratedsubpathremainingdistance}
    	The remaining distance on the current decorated subpath.
    \end{command}
  \end{command}
   
   To finish with, a simple example of a meta-decoration:

\begin{codeexample}[]
\pgfdeclaremetadecoration{arrows}{initial}{
  \state{initial}[width=0pt, next state=arrow]
  {	
    \pgfmathdivide{100}{\pgfmetadecoratedpathlength}
    \let\factor\pgfmathresult
    \pgfsetlinewidth{1pt}
    \pgfset{/pgf/decoration segment length=4pt}
  }
  \state{arrow}[
    switch if less than=\pgfmetadecorationsegmentlength to final,
    width=\pgfmetadecorationsegmentlength/3, 
    next state=zigzag]
  {
    \decoration{lineto}
    \beforedecoration
    {   
      \pgfmathparse{\pgfmetadecoratedcompleteddistance*\factor}
      \pgfsetcolor{red!\pgfmathresult!yellow}
      \pgfpathmoveto{\pgfpointmetadecoratedpathfirst}
    }	
  }
  \state{zigzag}[width=\pgfmetadecorationsegmentlength/3, next state=end arrow]
  {
  	\decoration{zigzag}
  }
  \state{end arrow}[width=\pgfmetadecorationsegmentlength/3, next state=move]
  {
    \decoration{lineto}
    \beforedecoration{\pgfpathmoveto{\pgfpointmetadecoratedpathfirst}}
    \afterdecoration
    {	
      \pgfsetarrowsend{to}
      \pgfusepath{stroke}	
    }
  }  
  \state{move}[width=\pgfmetadecorationsegmentlength/2, next state=arrow]{}
  \state{final}{}
}

\tikz\draw[meta-decoration segment length=2cm, meta-decoration=arrows]
  (0,0) .. controls (0,2)   and (3,2)   .. (3,0)
        .. controls (3,-2)  and (0,-2)  .. (0,-4)
        .. controls (0,-6)  and (3,-6)  .. (3,-8)
        .. controls (3,-10) and (0,-10) .. (0,-8);
\end{codeexample}

\end{command}


\subsubsection{Predefined Meta-decorations}

There are no predefined meta-decorations loaded with \pgfname{}.

\subsubsection{Meta-decoration Parameters}

The appearance of meta-decorations decorations can influenced by the 
settings of two keys. They are analogous to the keys used 
by both snakes and ordinary decorations, and similarly, do not 
actually store their setting ``inside themselves,'' but set 
\TeX\ macros.

\begin{itemize}
	\item |/pgf/meta-decoration segment amplitude| 
		sets the macro |\pgfmetadecorationsegmentamplitude|
		
	\item |/pgf/meta-decoration segment length| 
		sets the macro |\pgfmetadecorationsegmentlength|
		
\end{itemize}

When defining meta-decorations, you should refer to the above macros, 
not the the keys.



\subsubsection{Using Meta-Decorations}

Using meta-decorations is much simpler than using decorations, because
you can only use one meta-decoration per path.

\begin{environment}{{pgfmetadecoration}\marg{name}}

	This environment decorates the path described in 
	\meta{environment contents}, with the	meta-decoration \meta{name}.
	
\end{environment}

\begin{plainenvironment}{{pgfmetadecoration}\marg{name}}

	The plain \TeX{} version of the |{pgfmetadecoration}| environment.
	
\end{plainenvironment}

\begin{contextenvironment}{{pgfmetadecoration}\marg{name}}

	The Con\TeX t version of the |{pgfmetadecoration}| environment.
	
\end{contextenvironment}

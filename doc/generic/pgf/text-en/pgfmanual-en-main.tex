% Copyright 2006 by Till Tantau
%
% This file may be distributed and/or modified
%
% 1. under the LaTeX Project Public License and/or
% 2. under the GNU Free Documentation License.
%
% See the file doc/generic/pgf/licenses/LICENSE for more details.


% pgf version is defined in \pgfversion in file
% generic/pgf/utilities/pgfrcs.code.tex

\def\xcolorversion{2.00}

\usepackage[version=latest]{pgf}

\usepackage{xkeyval,calc,listings,tikz,fp}

\usepackage{hyperref}
\hypersetup{%
	colorlinks=false, % use true to enable colors below:
	linkcolor=blue,%red,
	filecolor=blue,%magenta,
	pagecolor=blue,%red,
	urlcolor=blue,%cyan,
	citecolor=blue,
	pdfborder=0 0 0,
}

% We need lots of libraries...
\usetikzlibrary{
  arrows,
  arrows.spaced,
  calc,
  fit,
  patterns,
  plotmarks,
  shapes.geometric,
  shapes.misc,
  shapes.symbols,
  shapes.arrows,
  shapes.callouts,
  shapes.multipart,
  shapes.gates.logic.US,
  shapes.gates.logic.IEC,
  circuits.logic.US,
  circuits.logic.IEC,
  circuits.logic.CDH,
  circuits.ee.IEC,
  datavisualization,
  datavisualization.polar,
  datavisualization.formats.functions,
  er,
  automata,
  backgrounds,
  chains,
  topaths,
  trees,
  petri,
  mindmap,
  matrix,
  calendar,
  folding,
  fadings,
  shadings,
  spy,
  through,
  turtle,
  positioning,
  scopes,
  decorations.fractals,
  decorations.shapes,
  decorations.text,
  decorations.pathmorphing,
  decorations.pathreplacing,
  decorations.footprints,
  decorations.markings,
  shadows,
  lindenmayersystems,
  intersections,
  fixedpointarithmetic,
  fpu,
  svg.path,
  external,
  graphs,
  graphs.standard,
  quotes
}

\usepackage{ifluatex}
\ifluatex
\usetikzlibrary{
  graphdrawing,
  graphdrawing.misc,
  graphdrawing.trees,
  graphdrawing.force,
  graphdrawing.circular,
  graphdrawing.layered
}
\else
	\def\LuaTeX{Lua\TeX}%
\fi



\iffalse
%\iftrue
	\tikzexternalize[
		mode=list only,export=true,% simply skips EVERY picture -> good for debugging the text.
	]
		{pgfmanual}

	\tikzifexternalizing{%
		\pgfkeys{/pdflinks/codeexample links=false}%
	}{}%
\fi


\usepackage[a4paper,left=2.25cm,right=2.25cm,top=2.5cm,bottom=2.5cm,nohead]{geometry}
\usepackage{amsmath,amssymb}
\usepackage{xxcolor}
\usepackage{pifont}
\usepackage{makeidx}

\ifluatex
  \usepackage{luatextra}
  % \filedescription is defined in expl3, required by fontspec,
  % required by luatextra. Needs to be \relaxed since
  % pgfmanual-en-macros.tex defines an environment named filedescription
  \let\filedescription\relax
  \usepackage[latin1]{luainputenc}
\else 
  \usepackage[latin1]{inputenc}
\fi
\usepackage{amsmath}

\graphicspath{{../../images/}}              % TODOsp: under windows this would go up 2 directories, but the file is only one directory up
% Copyright 2006 by Till Tantau
%
% This file may be distributed and/or modified
%
% 1. under the LaTeX Project Public License and/or
% 2. under the GNU Free Documentation License.
%
% See the file doc/generic/pgf/licenses/LICENSE for more details.

% $Header: /cvsroot/pgf/pgf/doc/generic/pgf/macros/pgfmanual-en-macros.tex,v 1.13 2007/02/12 18:24:34 tantau Exp $

\providecommand\href[2]{\texttt{#1}}

\colorlet{examplefill}{yellow!80!black}
\definecolor{graphicbackground}{rgb}{0.96,0.96,0.8}
\definecolor{codebackground}{rgb}{0.8,0.8,1}

\newenvironment{pgfmanualentry}{\list{}{\leftmargin=2em\itemindent-\leftmargin\def\makelabel##1{\hss##1}}}{\endlist}
\newcommand\pgfmanualentryheadline[1]{\itemsep=0pt\parskip=0pt\item\strut#1\par\topsep=0pt}
\newcommand\pgfmanualbody{\parskip3pt}



\newenvironment{pgflayout}[1]{
  \begin{pgfmanualentry}
    \pgfmanualentryheadline{\texttt{\string\pgfpagesuselayout\char`\{\declare{#1}\char`\}}\oarg{options}}
    \index{#1@\protect\texttt{#1} layout}%
    \index{Page layouts!#1@\protect\texttt{#1}}%
    \pgfmanualbody
}
{
  \end{pgfmanualentry}
}


\newenvironment{command}[1]{
  \begin{pgfmanualentry}
    \extractcommand#1\@@
    \pgfmanualbody
}
{
  \end{pgfmanualentry}
}

%% M.W: START MATH MACROS
\def\mvar#1{{\rmfamily\textit{#1}}}

\def\extractmathfunctionname#1{\extractmathfunctionname@#1(,)\tmpa\tmpb}
\def\extractmathfunctionname@#1(#2)#3\tmpb{\def\mathname{#1}}

\def\extractmathoperatorname{\begingroup\def\mvar##1{}\def\ {}\extractmathoperatorname@}
\def\extractmathoperatorname@#1{\xdef\mathname{#1}\endgroup}

	
\def\vskipspecial#1{\vskip#1\vskip0em}

\newenvironment{math-function}[1]{
	\begin{pgfmanualentry}
		\extractmathfunctionname{#1}
		\pgfmanualentryheadline{\item{\ttfamily#1}}%
		\index{\mathname @\protect\texttt{\mathname} math function}%
		\index{Math functions!\mathname @\protect\texttt{\mathname}}
		\pgfmanualbody
}
{
	\end{pgfmanualentry}\vskipspecial{-3em}
}

\newenvironment{math-operator}[1]{	
	\begin{pgfmanualentry}
		\extractmathoperatorname{#1}
		\pgfmanualentryheadline{\item{\ttfamily#1}}%
		\index{\mathname @\protect\texttt{\mathname} math operator}%
		\index{Math operators!\mathname @\protect\texttt{\mathname}}
    	\pgfmanualbody
}
{%
	\end{pgfmanualentry}\vskipspecial{-3em}
}

\newenvironment{math-constant}[1]{
	\begin{pgfmanualentry}
		\pgfmanualentryheadline{\item{\ttfamily#1}}%
		\index{#1@\protect\texttt{#1} math constant}%
		\index{Math constants!#1@\protect\texttt{#1}}
		\pgfmanualbody
}
{
	\end{pgfmanualentry}\vskipspecial{-3em}
}
\def\calcname{\textsc{calc}}

%% M.W: END MATH MACROS

\def\extractcommand#1#2\@@{%
  \pgfmanualentryheadline{\declare{\texttt{\string#1}}#2}%
  \removeats{#1}%
  \index{\strippedat @\protect\myprintocmmand{\strippedat}}}


\renewenvironment{environment}[1]{
  \begin{pgfmanualentry}
    \extractenvironement#1\@@
    \pgfmanualbody
}
{
  \end{pgfmanualentry}
}

\def\extractenvironement#1#2\@@{%
  \pgfmanualentryheadline{{\ttfamily\char`\\begin\char`\{\declare{#1}\char`\}}#2}%
  \pgfmanualentryheadline{{\ttfamily\ \ }\meta{environment contents}}%
  \pgfmanualentryheadline{{\ttfamily\char`\\end\char`\{\declare{#1}\char`\}}}%
  \index{#1@\protect\texttt{#1} environment}%
  \index{Environments!#1@\protect\texttt{#1}}}


\newenvironment{plainenvironment}[1]{
  \begin{pgfmanualentry}
    \extractplainenvironement#1\@@
    \pgfmanualbody
}
{
  \end{pgfmanualentry}
}

\def\extractplainenvironement#1#2\@@{%
  \pgfmanualentryheadline{{\ttfamily\declare{\char`\\#1}}#2}%
  \pgfmanualentryheadline{{\ttfamily\ \ }\meta{environment contents}}%
  \pgfmanualentryheadline{{\ttfamily\declare{\char`\\end#1}}}%
  \index{#1@\protect\texttt{#1} environment}%
  \index{Environments!#1@\protect\texttt{#1}}}


\newenvironment{contextenvironment}[1]{
  \begin{pgfmanualentry}
    \extractcontextenvironement#1\@@
    \pgfmanualbody
}
{
  \end{pgfmanualentry}
}

\def\extractcontextenvironement#1#2\@@{%
  \pgfmanualentryheadline{{\ttfamily\declare{\char`\\start#1}}#2}%
  \pgfmanualentryheadline{{\ttfamily\ \ }\meta{environment contents}}%
  \pgfmanualentryheadline{{\ttfamily\declare{\char`\\stop#1}}}%
  \index{#1@\protect\texttt{#1} environment}%
  \index{Environments!#1@\protect\texttt{#1}}}


\newenvironment{shape}[1]{
  \begin{pgfmanualentry}
    \pgfmanualentryheadline{Shape {\ttfamily\declare{#1}}}%
    \index{#1@\protect\texttt{#1} shape}%
    \index{Shapes!#1@\protect\texttt{#1}}
    \pgfmanualbody
}
{
  \end{pgfmanualentry}
}

\newenvironment{predefinednode}[1]{
  \begin{pgfmanualentry}
    \pgfmanualentryheadline{Predefined node {\ttfamily\declare{#1}}}%
    \index{#1@\protect\texttt{#1} node}%
    \index{Predefined node!#1@\protect\texttt{#1}}
    \pgfmanualbody
}
{
  \end{pgfmanualentry}
}

\newenvironment{coordinatesystem}[1]{
  \begin{pgfmanualentry}
    \pgfmanualentryheadline{Coordinate system {\ttfamily\declare{#1}}}%
    \index{#1@\protect\texttt{#1} coordinate system}%
    \index{Coordinate systems!#1@\protect\texttt{#1}}
    \pgfmanualbody
}
{
  \end{pgfmanualentry}
}

\newenvironment{snake}[1]{
  \begin{pgfmanualentry}
    \pgfmanualentryheadline{Snake {\ttfamily\declare{#1}}}%
    \index{#1@\protect\texttt{#1} snake}%
    \index{Snakes!#1@\protect\texttt{#1}}
    \pgfmanualbody
}
{
  \end{pgfmanualentry}
}

\def\pgfmanualbar{\char`\|}
\makeatletter
\newenvironment{pathoperation}[3][]{
  \begin{pgfmanualentry}
    \pgfmanualentryheadline{\textcolor{gray}{{\ttfamily\char`\\path}\
        \ \dots}
      \declare{\texttt{#2}}#3\ \textcolor{gray}{\dots\texttt{;}}}%
    \def\pgfmanualtest{#1}%
    \ifx\pgfmanualtest\@empty%
      \index{#2@\protect\texttt{#2} path operation}%
      \index{Path operations!#2@\protect\texttt{#2}}%
    \fi%
    \pgfmanualbody
}
{
  \end{pgfmanualentry}
}
\makeatother

\def\extractcommand#1#2\@@{%
  \pgfmanualentryheadline{\declare{\texttt{\string#1}}#2}%
  \removeats{#1}%
  \index{\strippedat @\protect\myprintocmmand{\strippedat}}}

\def\doublebs{\texttt{\char`\\\char`\\}}


\newenvironment{package}[1]{
  \begin{pgfmanualentry}
    \pgfmanualentryheadline{{\ttfamily\char`\\usepackage\char`\{\declare{#1}\char`\}\space\space \char`\%\space\space  \LaTeX}}
    \index{#1@\protect\texttt{#1} package}%
    \index{Packages and files!#1@\protect\texttt{#1}}%
    \pgfmanualentryheadline{{\ttfamily\char`\\input \declare{#1}.tex\space\space\space \char`\%\space\space  plain \TeX}}
    \pgfmanualentryheadline{{\ttfamily\char`\\usemodule[\declare{#1}]\space\space \char`\%\space\space  Con\TeX t}}
    \pgfmanualbody
}
{
  \end{pgfmanualentry}
}


\newenvironment{pgflibrary}[1]{
  \begin{pgfmanualentry}
    \pgfmanualentryheadline{{\ttfamily\char`\\usepgflibrary\char`\{\declare{#1}\char`\}\space\space\space
        \char`\%\space\space  \LaTeX\space and plain \TeX\space and pure pgf}}
    \index{#1@\protect\texttt{#1} library}%
    \index{Libraries!#1@\protect\texttt{#1}}%
    \pgfmanualentryheadline{{\ttfamily\char`\\usepgflibrary[\declare{#1}]\space\space \char`\%\space\space  Con\TeX t\space and pure pgf}}
    \pgfmanualentryheadline{{\ttfamily\char`\\usetikzlibrary\char`\{\declare{#1}\char`\}\space\space
        \char`\%\space\space  \LaTeX\space and plain \TeX\space when using \tikzname}}
    \pgfmanualentryheadline{{\ttfamily\char`\\usetikzlibrary[\declare{#1}]\space
        \char`\%\space\space  Con\TeX t\space when using \tikzname}}
    \pgfmanualbody
}
{
  \end{pgfmanualentry}
}

\newenvironment{tikzlibrary}[1]{
  \begin{pgfmanualentry}
    \pgfmanualentryheadline{{\ttfamily\char`\\usetikzlibrary\char`\{\declare{#1}\char`\}\space\space \char`\%\space\space  \LaTeX\space and plain \TeX}}
    \index{#1@\protect\texttt{#1} library}%
    \index{Libraries!#1@\protect\texttt{#1}}%
    \pgfmanualentryheadline{{\ttfamily\char`\\usetikzlibrary[\declare{#1}]\space \char`\%\space\space Con\TeX t}}
    \pgfmanualbody
}
{
  \end{pgfmanualentry}
}



\newenvironment{filedescription}[1]{
  \begin{pgfmanualentry}
    \pgfmanualentryheadline{File {\ttfamily\declare{#1}}}%
    \index{#1@\protect\texttt{#1} file}%
    \index{Packages and files!#1@\protect\texttt{#1}}%
    \pgfmanualbody
}
{
  \end{pgfmanualentry}
}


\newenvironment{packageoption}[1]{
  \begin{pgfmanualentry}
    \pgfmanualentryheadline{{\ttfamily\char`\\usepackage[\declare{#1}]\char`\{pgf\char`\}}}
    \index{#1@\protect\texttt{#1} package option}%
    \index{Package options for \textsc{pgf}!#1@\protect\texttt{#1}}%
    \pgfmanualbody
}
{
  \end{pgfmanualentry}
}



\newcommand\opt[1]{{\color{black!50!green}#1}}
\newcommand\ooarg[1]{{\ttfamily[}\meta{#1}{\ttfamily]}}

\def\opt{\afterassignment\pgfmanualopt\let\next=}
\def\pgfmanualopt{\ifx\next\bgroup\bgroup\color{black!50!green}\else{\color{black!50!green}\next}\fi}



\def\beamer{\textsc{beamer}}
\def\pdf{\textsc{pdf}}
\def\pgfname{\textsc{pgf}}
\def\tikzname{Ti\emph{k}Z}
\def\pstricks{\textsc{pstricks}}
\def\prosper{\textsc{prosper}}
\def\seminar{\textsc{seminar}}
\def\texpower{\textsc{texpower}}
\def\foils{\textsc{foils}}

{
  \makeatletter
  \global\let\myempty=\@empty
  \global\let\mygobble=\@gobble
  \catcode`\@=12
  \gdef\getridofats#1@#2\relax{%
    \def\getridtest{#2}%
    \ifx\getridtest\myempty%
      \expandafter\def\expandafter\strippedat\expandafter{\strippedat#1}
    \else%
      \expandafter\def\expandafter\strippedat\expandafter{\strippedat#1\protect\printanat}
      \getridofats#2\relax%
    \fi%
  }

  \gdef\removeats#1{%
    \let\strippedat\myempty%
    \edef\strippedtext{\stripcommand#1}%
    \expandafter\getridofats\strippedtext @\relax%
  }
  
  \gdef\stripcommand#1{\expandafter\mygobble\string#1}
}

\def\printanat{\char`\@}

\def\declare{\afterassignment\pgfmanualdeclare\let\next=}
\def\pgfmanualdeclare{\ifx\next\bgroup\bgroup\color{red!75!black}\else{\color{red!75!black}\next}\fi}


\let\textoken=\command
\let\endtextoken=\endcommand

\def\myprintocmmand#1{\texttt{\char`\\#1}}

\def\example{\par\smallskip\noindent\textit{Example: }}
\def\themeauthor{\par\smallskip\noindent\textit{Theme author: }}

\def\itemoption#1{\item \declare{\texttt{#1}}%
  \indexoption{#1}%
}

\def\indexoption#1{%
  \index{#1@\protect\texttt{#1} option}%
  \index{Graphic options!#1@\protect\texttt{#1}}%
}

\def\itemstyle#1{\item \texttt{style=}\declare{\texttt{#1}}%
  \index{#1@\protect\texttt{#1} style}%
  \index{Styles!#1@\protect\texttt{#1}}%
}

\def\itemcalendaroption#1{\item \declare{\texttt{#1}}%
  \index{#1@\protect\texttt{#1} date test}%
  \index{Date tests!#1@\protect\texttt{#1}}%
}



\def\class#1{\list{}{\leftmargin=2em\itemindent-\leftmargin\def\makelabel##1{\hss##1}}%
\extractclass#1@\par\topsep=0pt}
\def\endclass{\endlist}
\def\extractclass#1#2@{%
\item{{{\ttfamily\char`\\documentclass}#2{\ttfamily\char`\{\declare{#1}\char`\}}}}%
  \index{#1@\protect\texttt{#1} class}%
  \index{Classes!#1@\protect\texttt{#1}}}

\def\partname{Part}

\makeatletter
\def\index@prologue{\section*{Index}\addcontentsline{toc}{section}{Index}
  This index only contains automatically generated entries. A good
  index should also contain carefully selected keywords. This index is
  not a good index.
  \bigskip
}
\c@IndexColumns=2
  \def\theindex{\@restonecoltrue
    \columnseprule \z@  \columnsep 35\p@
    \twocolumn[\index@prologue]%
       \parindent -30pt
       \columnsep 15pt
       \parskip 0pt plus 1pt
       \leftskip 30pt
       \rightskip 0pt plus 2cm
       \small
       \def\@idxitem{\par}%
    \let\item\@idxitem \ignorespaces}
  \def\endtheindex{\onecolumn}
\def\noindexing{\let\index=\@gobble}


\newcommand\patternindex[1]{
  \index{#1@\protect\texttt{#1} pattern}%
  \index{Patterns!#1@\protect\texttt{#1}}
  \texttt{#1}& 
  \begin{tikzpicture}
    \path[draw=black!50,very thin,pattern=#1,rounded corners]
    (0pt,0pt) rectangle (1cm,1.5em);
  \end{tikzpicture}
}


\newcommand\symarrow[1]{
  \index{#1@\protect\texttt{#1} arrow tip}%
  \index{Arrow tips!#1@\protect\texttt{#1}}
  \texttt{#1}& yields thick  
  \begin{tikzpicture}[arrows={#1-#1},thick,baseline]
    \useasboundingbox (0pt,-0.5ex) rectangle (1cm,2ex);
    \draw (0pt,.5ex) -- (1cm,.5ex);
  \end{tikzpicture} and thin
  \begin{tikzpicture}[arrows={#1-#1},thin,baseline]
    \useasboundingbox (0pt,-0.5ex) rectangle (1cm,2ex);
    \draw (0pt,.5ex) -- (1cm,.5ex);
  \end{tikzpicture}
}

\newcommand\sarrow[2]{
  \index{#1@\protect\texttt{#1} arrow tip}%
  \index{Arrow tips!#1@\protect\texttt{#1}}
  \index{#2@\protect\texttt{#2} arrow tip}%
  \index{Arrow tips!#2@\protect\texttt{#2}}
  \texttt{#1-#2}& yields thick  
  \begin{tikzpicture}[arrows={#1-#2},thick,baseline]
    \useasboundingbox (0pt,-0.5ex) rectangle (1cm,2ex);
    \draw (0pt,.5ex) -- (1cm,.5ex);
  \end{tikzpicture} and thin
  \begin{tikzpicture}[arrows={#1-#2},thin,baseline]
    \useasboundingbox (0pt,-0.5ex) rectangle (1cm,2ex);
    \draw (0pt,.5ex) -- (1cm,.5ex);
  \end{tikzpicture}
}

\newcommand\carrow[1]{
  \index{#1@\protect\texttt{#1} arrow tip}%
  \index{Arrow tips!#1@\protect\texttt{#1}}
  \texttt{#1}& yields for line width 1ex
  \begin{tikzpicture}[arrows={#1-#1},line width=1ex,baseline]
    \useasboundingbox (0pt,-0.5ex) rectangle (1.5cm,2ex);
    \draw (0pt,.5ex) -- (1.5cm,.5ex);
  \end{tikzpicture}
}
\def\myvbar{\char`\|}
\newcommand\plotmarkentry[1]{%
  \index{#1@\protect\texttt{#1} plot mark}%
  \index{Plot marks!#1@\protect\texttt{#1}}
  \texttt{\char`\\pgfuseplotmark\char`\{\declare{#1}\char`\}} &
  \tikz\draw[color=black!25] plot[mark=#1,mark options={fill=examplefill,draw=black}] coordinates{(0,0) (.5,0.2) (1,0) (1.5,0.2)};\\
}
\newcommand\plotmarkentrytikz[1]{%
  \index{#1@\protect\texttt{#1} plot mark}%
  \index{Plot marks!#1@\protect\texttt{#1}}
  \texttt{mark=\declare{#1}} & \tikz\draw[color=black!25]
  plot[mark=#1,mark options={fill=examplefill,draw=black}] 
    coordinates {(0,0) (.5,0.2) (1,0) (1.5,0.2)};\\
}



\ifx\scantokens\@undefined
  \PackageError{pgfmanual-macros}{You need to use extended latex
    (elatex) or (pdfelatex) to process this document}{}
\fi

\begingroup
\catcode`|=0
\catcode`[= 1
\catcode`]=2
\catcode`\{=12
\catcode `\}=12
\catcode`\\=12 |gdef|find@example#1\end{codeexample}[|endofcodeexample[#1]]
|endgroup

\begingroup
\catcode`\^=7
\catcode`\^^M=13
\catcode`\ =13%
\gdef\returntospace{\catcode`\ =13\def {\space}\catcode`\^^M=13\def^^M{}}%
\endgroup

\begingroup
\catcode`\%=13
\catcode`\^^M=13
\gdef\commenthandler{\catcode`\%=13\def%{\@gobble@till@return}}
\gdef\@gobble@till@return#1^^M{}
\gdef\@gobble@till@return@ignore#1^^M{\ignorespaces}
\gdef\typesetcomment{\catcode`\%=13\def%{\@typeset@till@return}}
\gdef\@typeset@till@return#1^^M{{\def%{\char`\%}\textsl{\char`\%#1}}\par}
\endgroup

\define@key{codeexample}{width}{\setlength\codeexamplewidth{#1}}
\define@key{codeexample}{graphic}{\colorlet{graphicbackground}{#1}}
\define@key{codeexample}{code}{\colorlet{codebackground}{#1}}
\define@key{codeexample}{execute code}{\csname code@execute#1\endcsname}
\define@key{codeexample}{code only}[]{\code@executefalse}
\define@key{codeexample}{pre}{\def\code@pre{#1}}
\define@key{codeexample}{post}{\def\code@post{#1}}
\define@key{codeexample}{vbox}[]{\def\code@pre{\vbox\bgroup\setlength{\hsize}{\linewidth-6pt}}\def\code@post{\egroup}}
\define@key{codeexample}{ignorespaces}[]{\let\@gobble@till@return=\@gobble@till@return@ignore}
\define@key{codeexample}{leave comments}[]{\def\code@catcode@hook{\catcode`\%=12}\let\commenthandler=\relax\let\typesetcomment=\relax}

\def\code@pre{}
\def\code@post{}
\def\code@catcode@hook{}

\newdimen\codeexamplewidth
\newif\ifcode@execute
\newbox\codeexamplebox
\def\codeexample[#1]{%
  \begingroup%
  \code@executetrue
  \setlength\codeexamplewidth{4cm+7pt}
  \setkeys{codeexample}{#1}%
  \parindent0pt
  \begingroup%
  \par%
  \medskip%
  \let\do\@makeother%
  \dospecials%
  \obeylines%
  \@vobeyspaces%
  \catcode`\%=13%
  \catcode`\^^M=13%
  \code@catcode@hook%
  \relax%
  \find@example}
\def\endofcodeexample#1{%
  \endgroup%
  \ifcode@execute%
    \setbox\codeexamplebox=\hbox{%
      {%
        {%
          \returntospace%
          \commenthandler%
          \xdef\code@temp{#1}% removes returns and comments
        }%
        \colorbox{graphicbackground}{\color{black}\ignorespaces%
          \code@pre\expandafter\scantokens\expandafter{\code@temp\ignorespaces}\code@post\ignorespaces}%
      }%
    }%
    \ifdim\wd\codeexamplebox>\codeexamplewidth%
      \def\code@start{\par}%
      \def\code@flushstart{}\def\code@flushend{}%
      \def\code@mid{\parskip2pt\par\noindent}%
      \def\code@width{\linewidth-6pt}%
      \def\code@end{}%
    \else%
      \def\code@start{%
        \linewidth=\textwidth%
        \parshape \@ne 0pt \linewidth
        \leavevmode%
        \hbox\bgroup}%
      \def\code@flushstart{\hfill}%
      \def\code@flushend{\hbox{}}%
      \def\code@mid{\hskip6pt}%
      \def\code@width{\linewidth-12pt-\codeexamplewidth}%
      \def\code@end{\egroup}%
    \fi%
    \code@start%
    \noindent%
    \begin{minipage}[t]{\codeexamplewidth}\raggedright
      \hrule width0pt%
      \footnotesize\vskip-1em%
      \code@flushstart\box\codeexamplebox\code@flushend%
      \vskip-1ex
      \leavevmode%
    \end{minipage}%
  \else%
    \def\code@mid{\par}
    \def\code@width{\linewidth-6pt}
    \def\code@end{}
  \fi%
  \code@mid%  
  \colorbox{codebackground}{%
    \begin{minipage}[t]{\code@width}%
      {%
        \let\do\@makeother
        \dospecials
        \frenchspacing\@vobeyspaces
        \normalfont\ttfamily\footnotesize
        \typesetcomment%
        \@tempswafalse
        \def\par{%
          \if@tempswa
          \leavevmode \null \@@par\penalty\interlinepenalty
          \else
          \@tempswatrue
          \ifhmode\@@par\penalty\interlinepenalty\fi
          \fi}%
        \obeylines
        \everypar \expandafter{\the\everypar \unpenalty}%
        #1}
    \end{minipage}}%
  \code@end%
  \par%
  \medskip
  \end{codeexample}
}

\def\endcodeexample{\endgroup}


\makeatother


%%% Local Variables: 
%%% mode: latex
%%% TeX-master: "beameruserguide"
%%% End: 
    % TODOsp: same here

\makeindex

\makeatletter
\renewcommand*\l@section[2]{%
  \ifnum \c@tocdepth >\z@
    \addpenalty\@secpenalty
    \addvspace{1.0em \@plus\p@}%
    \setlength\@tempdima{2.5em}%
    \begingroup
      \parindent \z@ \rightskip \@pnumwidth
      \parfillskip -\@pnumwidth
      \leavevmode \bfseries
      \advance\leftskip\@tempdima
      \hskip -\leftskip
      #1\nobreak\hfil \nobreak\hb@xt@\@pnumwidth{\hss #2}\par
    \endgroup
  \fi}
\renewcommand*\l@subsection{\@dottedtocline{2}{2.5em}{3.3em}}
\renewcommand*\l@subsubsection{\@dottedtocline{3}{5.8em}{4.2em}}
\makeatother

%\includeonly{pgfmanual-en-library-profiler}

% Global styles:
\tikzset{
  every plot/.style={prefix=plots/pgf-},
  shape example/.style={
    color=black!30,
    draw,
    fill=yellow!30,
    line width=.5cm,
    inner xsep=2.5cm,
    inner ysep=0.5cm}
}

\index{Options for graphics|see{Graphic options and styles}}
\index{Styles for graphics|see{Graphic options and styles}}
\index{Options for packages|see{Package options}}
\index{Handlers for keys|see{Key handlers}}
\index{File|see{Packages and files}}
\index{Layout|see{Page layout}}
\index{Node|see{Predefined node}}
\index{Data formats|see{Formats}}

\begin{document}

% Copyright 2010 by Renée Ahrens, Olof Frahm, Jens Kluttig, Matthias Schulz, Stephan Schuster
% Copyright 2011 by Till Tantau
% Copyright 2011 by Jannis Pohlmann
%
% This file may be distributed and/or modified
%
% 1. under the LaTeX Project Public License and/or
% 2. under the GNU Free Documentation License.
%
% See the file doc/generic/pgf/licenses/LICENSE for more details.

\section{Using Algorithmic Graph Drawing}

{\noindent {\emph{by Till Tantau and Ren\'ee Ahrens, Olof-Joachim
      Frahm, Jens Kluttig, Jannis Pohlmann, Matthias Schulz, Stephan
      Schuster}}} 

\label{section-library-graphdrawing}

\begin{tikzlibrary}{graphdrawing}
  This package provides capabilities for automatic graph drawing.

  \medskip
  \textbf{Note:} Graph drawing requires that the document is typeset
  using Lua\TeX. This package should work with \LuaTeX\ 0.4 or
  higher, which is included in all current \TeX\ distributions.
\end{tikzlibrary}

\ifluatex\relax\else{LuaTeX is required for setting this manual
  section.}\expandafter\endinput\fi 


\subsection{Overview}

\emph{Algorithmic graph drawing} (or just \emph{graph drawing} in the
following) means that algorithms are used to decide where the nodes of
a graph are positioned on a page so that the graph ``looks nice.'' The
idea is that you, as human (or you, as a machine, if you happen to be
a machine and happen to be reading this document) just specify which
nodes are present in a graph and which edges are
present. Additionally, you may add some ``hints'' like ``this node
should be near the center'' or ``this edge is pretty important.'' You
do \emph{not} specify where, exactly, the nodes and edges should
be. This is something you leave to a \emph{graph drawing
  algorithm}. The algorithm gets your description of the graph as an
input and then decides where the nodes should go on the page.

Naturally, graph drawing is a bit of a (black?) art. There is no
``perfect'' way of drawing a graph, rather, depending on the
circumstances there are several different ways of drawing the same
graph and often it will just depend on the aesthetic sense of the
reader which layout he or she would prefer. For this reason, there is
a huge number of graph drawing algorithms ``out there'' and there are
scientific conference devoted to such algorithms, where each
year dozens of new algorithms are proposed.

Unlike the rest of \pgfname\ and \tikzname, which is implemented
purely in \TeX, the graph drawing algorithms are simply too complex to
implement them in \TeX. Instead, the programming language Lua is used
by the graph drawing library -- a programming language that has been
integrated into recent versions of \TeX. This means that (a) as a user
of the graph drawing engine you will can run \TeX\ on your documents
in the usual way, no external programs are called since Lua is already
integrated into \TeX\ and (b) it is pretty easy to implement new graph
drawing algorithms for \tikzname\ since Lua can be used and no \TeX\
programming knowledge is needed. 

The graph drawing engine of \tikzname\ provides two main features:
\begin{enumerate}
\item ``Users'' of the graph drawing engine can invoke the graph
  drawing algorithms often by just adding a single option to their
  picture. Here is a typical example, where the |layered layout| option
  tells \tikzname\ that the graph should be drawn (``should be layed
  out'') using a so-called ``layered graph drawing algorithm'' (what
  these are will be explained later):
\begin{codeexample}[]
\tikz
  \graph [layered layout, components go right top aligned, nodes={draw, rounded corners=2pt}]
  {
    first root -> {1 -> {2, 3} -> {4, 5}, 6 }, 4 -- 5;
    second root -> x -> {a -> {/,/}, b, c -> d -> {/,/} };
    third root -> child -> grandchild -> youngster -> third root;    
  };
\end{codeexample}
  Here is another example, where a different layout method is used
  that is more appropriate for trees:
\begin{codeexample}[]
\tikz [grow'=up, binary tree layout, nodes={circle,draw}]
  \node {1}
  child { node {2}
    child { node {3} }
    child { node {4}
      child { node {5} }
      child { node {6} }
    }
  }
  child { node {7}
    child { node {8}
      child[missing]
      child { node {9} }
    }
  };
\end{codeexample}
  An a final example, this time using a ``spring electrical layout''
  (whatever that might be\dots):
\begin{codeexample}[]
\tikz [spring electrical layout]
{
  \foreach \i in {1,...,6}
    \node (node \i) [fill=blue!50, text=white, circle] {\i};
    
  \draw (node 1) edge (node 2)
        (node 2) edge (node 3)
        (node 3) edge (node 4)
                 edge (node 5)
                 edge (node 6);
}
\end{codeexample}
  In all of the example, the positions of the nodes have only been
  computed \emph{after} all nodes have been created and the edges have
  been specified. For instance, in the last example, without the
  option |spring electrical layout|, all of the nodes would have been
  placed on top of each other.
\item The graph drawing engine is also intended to make is
  (relatively) easy to implement new graph drawing algorithms. These
  algorithms can and must be implemented in the Lua programming
  language (which is \emph{much} easier to program than \TeX\
  itself). The Lua code for a graph drawing algorithm gets an
  object-oriented model of the input graph as an input and must just
  compute the desired new positions of the nodes. The complete
  handling of passing options and configurations back-and-forth
  between the different \tikzname\ and \pgfname\ layers is handled by
  the graph drawing engine.

  The bottom line is that the graph drawing engine makes it easy
  to try out new graph drawing algorithms for medium sized graphs (up
  to a few hundred nodes).
\end{enumerate}

The documentation of the graph drawing engine is structured as
follows: The current section explains the graph drawing engine from
``the user's point of few'' and also describes the basic steps
necessary to implement a new graph drawing algorithm. The libraries
containing the different graph drawing algorithms are documented in
Sections on graph drawing
algorithms. Section~\ref{section-gd-own-algorithm} covers the
internals of how the graph drawing engine works. 



\subsection{Usage}

To use the graph drawing engine, you first need to load some
libraries. First, you should always load the |graphdrawing| library,
which will setup the basic keys. Next, you need to load another
library like |graphdrawing.trees|, see the following
Sections~\ref{section-first-graphdrawing-library-in-manual} to
\ref{section-last-graphdrawing-library-in-manual} for the different
libraries that are available. The actual graph drawing
algorithms reside in these libraries. Finally, you may also wish to
load the |graphs| library, but this is only necessary if you wish to
use the |graph| path command, which provides an easy-to-use syntax for
specifying graphs. You can also use the graph drawing engine
independently of the |graphs| library, for instance in conjunction
with the |child| or the |edge| syntax. Here is a typical setup:

\begin{codeexample}[code only]
\usetikzlibrary{graphs,graphdrawing,graphdrawing.trees}  
\end{codeexample}

Having setup things, you must then specify for which scopes the
graph drawing engine should apply an layout algorithm to the nodes in
the scope. Typically, you just add an option ending with |... layout|
to the |graph| path operation and then let the graph drawing do its
magic:

\begin{codeexample}[]
\tikz [rounded corners]
  \graph [layered layout, sibling distance=8mm, level distance=8mm]
  {
    a -> {
      b,
      c -> { d, e }
    } ->
    f -> 
    a
  };    
\end{codeexample}

Whenever you use such an option, (to be more precise, inside every
scope with the |graph drawing scope| key set either
explicitly or implicitly, which is exactly what happens when one such
an option is used) you can:
\begin{itemize}
\item Create nodes in the usual way. The nodes will be created
  completely, but then tucked away in an internal table. This means
  that all of \tikzname's options for nodes can be applied. You can
  also name a node and reference it later.
\item Create edges using either the syntax of the |graph| command
  (using |--|, |<-|, |->|, or |<->|), or using the |edge| command,
  or using the |child| command. These edges will, however, not be
  created immediately. Instead, the basic layer's command
  |\pgfgdedge| will be called, which stores ``all the information
  concerning the edge.'' The actual drawing of the edge will only
  happen after all nodes have been positioned.
\item Most of the keys that can be passed to an edge will work as
  expected. In particular, you can add labels to edges using the
  usual |node| syntax for edges.
\item The |label| and |pin| options can be used in the usual manner
  with nodes inside a graph drawing scope. Only, the labels and
  nodes will play no role in the positioning of the nodes and they
  are added when the nodes are finally positioned.
\item Similarly, nodes that are placed ``on an edge'' using the
  implicit positioning syntax can be used in the usual manner. 
\end{itemize}
Here are some things that will \emph{not} work:
\begin{itemize}
\item Only edges created using the graph syntax, the |edge| command,
  or the |child| command will correctly pass their connection
  information to the basic layer. When you write |\draw (a)--(b);|
  inside a graph drawing scope, where |a| and |b| are nodes that
  have been created inside the scope, you will get an error
  message / things will look wrong. The reason is that the usual
  |--| is not ``caught'' by the graph drawing engine and, thus,
  tries to immediately connect two nodes that do not yet exist
  (except inside some internal table).
\item The options of edges are executed twice: Once when the edge is
  ``examined'' by the |\pgfgdedge| command (using some magic to shield
  against the side effects) and then once more when the edge is
  actually created. Fortunately, in almost all cases, this will not be
  a problem; but if you do very evil magic inside your edge options,
  you must roll a D100 to see what strange things will happen. (Do no
  evil, by the way.)
\end{itemize}

The rest of this subsection describes the ``fine print'' of what
happens, in detail. You may wish to skip it.

\medskip
\noindent\textbf{The Details.}
Let us start with some background knowledge on how the graph drawing
engine works might be useful: Using a special internal key called
|graph drawing scope|, which you typically will not call directly,
the graph drawing engine can be switched on for a |{scope}|. When this
happens, a lot of things change inside \pgfname\ and \tikzname\ for
this scope: First, all nodes created inside the scope are not
immediately placed at the position where they were created. Instead,
they are ``spirited away'' to some internal table of the graph drawing
engine. Second, all edges created inside the scope using either the
|graph| command, the |edge| command, or the |child| command are also
``spirited away'' to another internal table. Then, at the end of the
scope, the graph drawing algorithm is started, which has access to
these internal tables of nodes and edge of the graph that has been
specified inside the scope. The algorithm will then compute new,
better, positions for the nodes. Finally, once the positions have been
computed, the graph drawing engine will then retrieve the nodes from
the internal table and place them at the computed positions and it
also retrieves the edges from the internal table and also adds them to
the picture.

While this theory may sound complicated, the use of the graph library
is, fortunately, pretty simple: Just add a key like |tree layout| or
|spring layout| to a scope and leave out any explicit positioning via
things like |at| -- the positioning will be done automatically by the
graph drawing algorithm.

The keys like |tree layout| or |spring layout| are explained in more detail
in the chapters on the different libraries. They all internally call
(at least) two keys: |graph drawing scope| and |algorithm|. These
keys are documented in the following, but you typically will not use
them explicitly. In addition to setting up the scope and setting the
correct algorithms, keys like |tree layout| and |spring layout| also take
some \meta{options} as arguments. These \meta{options} allow you setup
special graph parameters for the algorithm.

\begin{key}{/tikz/graph drawing scope}
  This key can (only) be used as an option when a \tikzname\ scope is
  started. Thus, you can pass it to |\tikz|, to |{tikzpicture}|, to
  |\scoped|, to |{scope}|, to |graph|, and to |{graph}|. For instance,
  the |tree layout| option (which uses |graph drawing scope| internally) can
  be used in the following ways:
\begin{codeexample}[]
\tikz [tree layout] \graph {a -> {b,c}};  

\tikz \graph [tree layout] {a -> {b,c}};

\tikz \path graph [tree layout] {a -> {b,c}};

\begin{tikzpicture}[tree layout]
  \graph {a -> {b,c}};
\end{tikzpicture}

\begin{tikzpicture}
  \draw [help lines] (0,0) grid (3,1);
  
  \scoped [tree layout] \graph {a -> {b,c}};
    
  \begin{scope}[tree layout, xshift=1cm, rotate=90]
    \graph {a -> {b,c}};
  \end{scope}
\end{tikzpicture}
\end{codeexample}

  You can \emph{not} use the |graph drawing scope| key with a single
  node or on a path. In particular, to typeset a tree given in the
  |child| syntax somewhere inside a |{tikzpicture}|, you must prefix
  it with the |\scoped| command:
\begin{codeexample}[]
\begin{tikzpicture}
  \scoped [tree layout]
    \node {root}
    child { node {left child} }
    child { node {right child} };
\end{tikzpicture}
\end{codeexample}
  Naturally, the above could have been written more succinctly as
\begin{codeexample}[]
\tikz [tree layout]
  \node {root}
  child { node {left child} }
  child { node {right child} };
\end{codeexample}
  Or even more succinctly:
\begin{codeexample}[]
\tikz \graph [tree layout] { root -- {left child, right child} };
\end{codeexample}

  In detail, adding the |graph drawing scope| command to a scope has
  the following effects:
  \begin{itemize}
  \item The basic layer is informed, using the
    |execute at begin scope| key, that the current scope will contain
    nodes that should be positioned by a graph drawing engine. Which
    algorithm is used depends on the value of the |algorithm| key.
  \item If the |graphs| library has been loaded, the default
    positioning mechanisms of this library are switched off, leaving
    the positioning to the graph drawing engine. Also, when an edge is
    created by the |graphs| library, this is signalled to the graph
    drawing library. (To be more precise: The keys |new ->| and so on
    are redefined so that they call |\pgfgdedge| instead of creating
    an edge.
  \item The |edge| path command is modified so that it also calls
    |\pgfgdedge| instead of immediately creating any edges.
  \item The |edge from parent| path command is modified so that is
    also calls |\pgfgdedge|.
  \item The keys |append after command| and |prefix after command|
    keys are modified so that they are executed only via
    |late options| when the node has ``reached its final parking
    position''. 
  \end{itemize}
\end{key}

\begin{key}{/graph drawing/algorithm=\meta{algorithm's name}}
  \label{section-gd-algorithm-key}%
  This key specifies which algorithm should be used for typesetting a
  graph. The names of these algorithm's are often a bit cryptic (like
  |Walshaw2000| or something similar), which is why you typically do
  not call this key directly. Instead, styles with more
  easy-to-remember names internally set this key.

  Setting this key has the following effects: When a scope with the
  |graph drawing scope| command is started, the current value of
  \meta{algorithm's name} is examined. Lua will try to find a class
  named \meta{algorithm's name} that has been declared using the
  |graph_drawing_algorithm| command (but any spaces inside the
  \meta{algorithm's name} are deleted). If it does not find such a
  class, the engine tries to  
  load the file called |pgfgd-algorithm-|\meta{algorithm's name}|.lua|
  (again, spaces are deleted) and then, again, tries to lookup the
  class. Thus, any class \meta{algorithm's name} mentioned inside a
  document must either have 
  already been defined in some Lua file loaded by some library or it
  must reside in a file with the corresponding name. Once the class
  has been found, the class's |new| method is called, the |graph|
  attribute of the object is set to the to-be-layouted graph, and the
  |constructor| method of the class is called for the object (if it
  exists) and, finally, the |run| method is called. Details of what
  should be done in these methods are given in
  Section~\ref{section-gd-own-algorithm}).  

  Here is an example where we switch on the graph drawing engine
  explicitly and explicitly select an algorithm:
\begin{codeexample}[]
\tikz [graph drawing scope,
       /graph drawing/algorithm=Spring Electrical Walshaw 2000]
  \graph { a <-> {b, c} };  
\end{codeexample}

  The reference of the available algorithms is in
  Sections~\ref{section-first-graphdrawing-library-in-manual} to 
  \ref{section-last-graphdrawing-library-in-manual}.
\end{key}




\subsection{Graph, Node, and Edge Parameters}

Graph drawing algorithms can typically be configured in some way. For
instance, for a graph drawing algorithm that visualizes its nodes as a
tree, it will typically be useful when the user can change the
so-called \emph{level distance} and the \emph{sibling distance}. For
other algorithms, like force-based algorithms, a large number of
parameters influence the way the algorithms work.

Options that influence graph drawing algorithms will be called
\emph{graph drawing parameters} in the following. There are three kinds of
graph drawing parameters:
\begin{itemize}
\item Graph parameters,
\item node parameters, and
\item edge parameters.
\end{itemize}
A graph drawing graph parameter influences the layout of the whole
graph. A graph drawing node parameter is an option that is attached to
a single node and should only have a direct influence on this node
(like ``place this node exactly at this position, no matter what''). A
graph drawing edge parameter in important for a single edge (like
``this edge must be exactly |2cm| long'').

A graph drawing algorithm may or may not take the different graph
parameters into account. After all, these options may even outright
contradict each other, so an algorithm can only try to ``do its
best''.

While many graph parameters are very specific to a single algorithm, a
number of graph parameters will be important for many algorithms. Such
graph parameters are called \emph{common} graph parameters, the most
important of which are documented in the following. The common graph
parameters can be used like any normal \tikzname\ option. In contrast,
specific options for algorithms must be passed to the key that
installs the algorithm. For example, the orientation of a graph
is setup with the common key |orient|, which is given alongside a key
like |spring layout|:

\begin{codeexample}[]
\tikz \graph [spring layout, orient=1|2] { 1--2--3--1 };  
\end{codeexample}

In contrast, the very specific option |iterations| must be
passed to the |spring layout| key:

\begin{codeexample}[]
\tikz \graph [spring layout={iterations=3}] { 1--2--3--1 };  
\end{codeexample}





\subsection{Padding and Node Distances}

\label{subsection-gd-dist-pad}

In many drawings, you may wish to specify how ``near'' two nodes should
be placed by a graph drawing algorithm. Naturally, this depends
strongly on the specifics of the algorithm, but there are a number of
general keys that will be used by many algorithms.



\subsubsection{Distances and Paddings Between Layers}

A number of graph drawing algorithms arrange nodes in layers; we refer
to the nodes on the same layer as siblings (although, in a tree,
siblings are only nodes with the same parent; nevertheless we use
``sibling'' loosely also for nodes that are more like ``cousins'').

\begin{key}{/graph drawing/level distance=\meta{dimension} (initially 1cm)}
  \keyalias{tikz}\keyalias{tikz/graphs}
  This is minimum distance that the centers of nodes on one
  level should have from the centers of nodes on the next level. It
  will not always be possible to satisfy this desired distance, for
  instance in case the nodes are too big. In this case, the
  \meta{dimension} is just considered as a lower bound.
\begin{codeexample}[]
\tikz \graph [layered layout, level distance=1cm] { 1--2--3--1 };  
\tikz \graph [layered layout, level distance=5mm] { 1--2--3--1 };  
\end{codeexample}
\end{key}

\begin{key}{/graph drawing/level pre sep=\meta{dimension} (initially 0.5pt)}
  \keyalias{tikz}\keyalias{tikz/graphs}
  This is a minimum ``padding'' or ``separation'' between the border
  of the nodes on a level to any nodes on the previous level. Thus, if
  nodes are so big that nodes on consecutive levels would overlap (or
  just come with \meta{dimension} distance of one another), their
  distance is enlarged so that this distance is still satisfied.

  (If a node on the previous level also has a |level post sep|, this
  post padding and the \meta{dimension} add up. Thus, these keys
  behave like the ``padding'' keys rather
  than the ``margin'' key of cascading style sheets.)
  
  Currently, this option is not yet implemented.
\end{key}

\begin{key}{/graph drawing/level post sep=\meta{dimension} (initially 0.5pt)}
  \keyalias{tikz}\keyalias{tikz/graphs}
  Works like |level pre sep|.
\end{key}

\begin{key}{/graph drawing/level sep=\meta{dimension}}
  \keyalias{tikz}\keyalias{tikz/graphs}
  Sets both |level pre sep| and |level post sep| to
  $\meta{dimension}/2$.
\end{key}

Note that if you set |level distance=0| and |level distance=1em|, you get
a layout where any two consecutive layers are ``spaced apart'' by
|1em|.


\subsubsection{Distances and Paddings Between Siblings}

The following keys work much like the |level ...| keys, only for
sibling:

\begin{key}{/graph drawing/sibling distance=\meta{dimension} (initially 1cm)}
  \keyalias{tikz}\keyalias{tikz/graphs}
  This is minimum distance that the centers of node should have to the
  center of the next node on the same level. As for levels, this is
  just a lower bound.

  For some layouts, like a circular layout, the \meta{dimension} is
  measured as the distance on the circle:
\begin{codeexample}[]
\tikz \graph [circular layout, sibling distance=1cm, nodes={circle,draw}]
  { 1--2--3--4--5--6--1 };  
\end{codeexample}
\begin{codeexample}[]
\tikz \graph [circular layout, sibling distance=0cm, sibling sep=0pt,
              nodes={circle,draw}]
  { 1--2--3--4--5--6--1 };  
\end{codeexample}
\begin{codeexample}[]
\tikz \graph [circular layout, sibling distance=0cm, sibling sep=0pt,
              nodes={circle,draw}]
  { 1--2--3[sibling distance=3cm]--4--5--6--1 };  
\end{codeexample}
\end{key}


\begin{key}{/graph drawing/sibling pre sep=\meta{dimension} (initially 0.5pt)}
  \keyalias{tikz}\keyalias{tikz/graphs}
  Works like |level pre sep|, only for siblings.
\begin{codeexample}[]
\tikz \graph [circular layout, sibling distance=0cm, nodes={circle,draw},
              sibling sep=0pt]
  { 1--2--3--4--5--6--1 };  
\end{codeexample}
\begin{codeexample}[]
\tikz \graph [circular layout, sibling distance=0cm, nodes={circle,draw},
              sibling pre sep=1em]
  { 1--2--3--4--5--6--1 };  
\end{codeexample}
\begin{codeexample}[]
\tikz \graph [circular layout, sibling distance=0cm, nodes={circle,draw},
              sibling pre sep=1em]
  { 1--2--3[sibling pre sep=1cm]--4--5--6--1 };  
\end{codeexample}
\end{key}

\begin{key}{/graph drawing/level sibling sep=\meta{dimension} (initially 0.5pt)}
  \keyalias{tikz}\keyalias{tikz/graphs}
  Works like |level sibling sep|.
\end{key}

\begin{key}{/graph drawing/sibling sep=\meta{dimension}}
  \keyalias{tikz}\keyalias{tikz/graphs}
  Sets both |sibling pre sep| and |sibling post sep| to
  $\meta{dimension}/2$.
\end{key}




\subsubsection{Paddings Between Components}

When a graph consists of several connected component, many graph
drawing algorithms will layout these components individually. The
different components will then be arranged next to each other, see
Section~\ref{section-gd-packing} for the details, such that between
the nodes of any two components the following padding is avaiable:

\begin{key}{/graph drawing/component sep=\meta{dimension} (initially 2em)}
  \keyalias{tikz}\keyalias{tikz/graphs}
  This is distance between the bounding boxes that nodes of different
  connected components will have when they are placed next to each
  other:
\begin{codeexample}[]
\tikz \graph [binary tree layout, sibling distance=4mm, level distance=8mm,
              components go right top aligned,
              component sep=1pt, nodes=draw]  
{
  1 -> 2 -> {3->4[second]->5,6,7};
  a -> b[second] -> c[second] -> d -> e;
  x -> y[second] -> z -> u[second] -> v;
};  
\end{codeexample}
\begin{codeexample}[]
\tikz \graph [binary tree layout, sibling distance=4mm, level distance=8mm,
              components go right top aligned,
              component sep=1em, nodes=draw]  
{
  1 -> 2 -> {3->4[second]->5,6,7};
  a -> b[second] -> c[second] -> d -> e;
  x -> y[second] -> z -> u[second] -> v;
};  
\end{codeexample}
\end{key}



\subsection{Anchoring a Graph}

\label{subsection-library-graphdrawing-anchoring}

A graph drawing algorithm must compute positions of the nodes of a
graph, but the computed positions are only \emph{relative} (``this
node is left of this node, but above that other node''). It is not
immediately obvious where the ``the whole graph'' should be placed
\emph{absolutely} once all relative positions have been computed. In
case that the graph consists of several unconnected components, the
situation is even more complicated.

In order to determine the absolute position of a graph, the graph
drawing engine relies on the following key:

\begin{key}{/graph drawing/desired at=\marg{coordinate}}
  \keyalias{tikz}\keyalias{tikz/graphs}
  When you add this key to a node in a graph, you ``desire'' that the
  node should be placed at the \meta{coordinate} by the graph drawing
  algorithm. Now, when you set this key for a single node of a graph,
  then, by shifting the graph around, this ``wish'' can obviously
  always be fulfill:
\begin{codeexample}[]
\begin{tikzpicture}
  \draw [help lines] (0,0) grid (3,2);
  \graph [spring layout]
  {
    a [desired at={(1,0)}] -- b -- c -- a;
  };
\end{tikzpicture}
\end{codeexample}
\begin{codeexample}[]
\begin{tikzpicture}
  \draw [help lines] (0,0) grid (3,2);
  \graph [spring layout]
  {
    a -- b[desired at={(2,1)}] -- c -- a;
  };
\end{tikzpicture}
\end{codeexample}
\begin{codeexample}[]
\begin{tikzpicture}
  \draw [help lines] (0,0) grid (3,2);
  \graph [layered layout]
  {
    a -- b[desired at={(2,1)}] -- c -- a;
  };
\end{tikzpicture}
\end{codeexample}
  Since the key's name is a but long and since the many braces and
  parentheses are a bit cumbersome, there is a special support for
  this key inside a |graph|: The standard |/tikz/at| key is redefined
  inside a |graph| so that it points to |/graph drawing/desired at|
  instead. (Which is more logical anyway, since it makes no sense to
  specify an |at| position for a node whose position it to be computed
  by a graph drawing algorithm.) A nice side effect of this is that
  you can use the |x| and |y| keys (see
  Section~\ref{section-graphs-xy}) to specify desired positions:
\begin{codeexample}[]
\begin{tikzpicture}
  \draw [help lines] (0,0) grid (3,2);
  \graph [spring layout]
  {
    a -- b[x=2,y=1] -- c -- a;
  };
\end{tikzpicture}
\end{codeexample}
\begin{codeexample}[]
\begin{tikzpicture}
  \draw [help lines] (0,0) grid (3,2);
  \graph [layered layout]
  {
    a [x=1,y=2] -- { b, c } -- {e, f} -- a
  };
\end{tikzpicture}
\end{codeexample}

  A problem arises when two or more nodes have this key set
  and when these nodes are in the same connected component, because
  then your ``desires'' for placement and the positions computed by
  the graph drawing algorithm may clash. Graph drawing algorithms are
  ``told'' about the desired positions. Most algorithms will simply
  ignore these desired positions since they will be taken care of in
  the so-called post-anchoring phase, see below. However, for some
  algorithms it makes a lot of sense to fix the positions of some
  nodes and only compute the positions 
  of the other nodes relative to these nodes. For instance, for a
  |spring layout| it makes perfect sense that some nodes are
  ``nailed to the canvas'' while other nodes can ``move freely''.
\begin{codeexample}[]
\tikz \graph [spring layout]
{
  a -- { b, c, d, e -- {f,g,h} };
  { h, g } -- a;
};
\end{codeexample}
\begin{codeexample}[]
\tikz \graph [spring layout]
{
  a -- { b, c, d[x=0], e -- {f[x=2], g, h[x=1]} };
  { h, g } -- a;
};
\end{codeexample}
\begin{codeexample}[]
\tikz \graph [spring layout]
{
  a -- { b, c, d[x=0], e -- {f[x=2,y=1], g, h[x=1]} };
  { h, g } -- a;
};
\end{codeexample}
\end{key}


\begin{key}{/graph drawing/anchor node=\meta{node name}}
  \keyalias{tikz}\keyalias{tikz/graphs}
  This option can be used with a graph to specify a node that should
  be used for anchoring the whole graph. When this option is
  specified, after the layout has been computed, the whole graph will
  be shifted in such a way that the \meta{node name} is either
  \begin{itemize}
  \item at the current value of |anchor at| or 
  \item at the position that is specified in the form of a
    |desired at| for the \meta{node name}.
  \end{itemize}
\begin{codeexample}[]
\tikz \draw (0,0)
  -- (1,0.5) graph [edges=red,  layered layout, anchor node=a] { a -> {b,c} }
  -- (2,0)   graph [edges=blue, layered layout,
                    anchor node=y, anchor at={(2,0)}]          { x -> {y,z} };
\end{codeexample}
\begin{codeexample}[]
\begin{tikzpicture}
  \draw [help lines] (0,0) grid (3,2);
  
  \graph [layered layout, anchor node=c, edges=rounded corners]
    { a -- {b [x=1,y=1], c [x=1,y=1] } -- d -- a};
\end{tikzpicture}
\end{codeexample}
  Note how in the above example |c| is placed at |(1,1)| rather than
  |b| as would happen by default.
\end{key}

\begin{key}{/graph drawing/anchor at=\meta{coordinate} (initially the origin)}
  \keyalias{tikz}\keyalias{tikz/graphs}
  The coordinate at which the graph should be anchored when no
  explicit anchor is given for any node.
\begin{codeexample}[]
\begin{tikzpicture}
  \draw [help lines] (0,0) grid (2,2);
  
  \graph [layered layout, edges=rounded corners, anchor at={(1,2)}]
    { a -- {b, c [anchor here] } -- d -- a};
\end{tikzpicture}
\end{codeexample}
\end{key}

\begin{key}{/graph drawing/anchor here=\opt{\meta{true or false}} (default true)}
  \keyalias{tikz}\keyalias{tikz/graphs}
  This option can be passed to a single node (rather than the graph as
  a whole) in order to specify that this node should be used for the
  anchoring process.
\begin{codeexample}[]
\begin{tikzpicture}
  \draw [help lines] (0,0) grid (2,2);
  
  \graph [layered layout, edges=rounded corners]
    { a -- {b, c [anchor here] } -- d -- a};
\end{tikzpicture}
\end{codeexample}
  In the above example, |c| is placed at the origin since this is the
  default |anchor at| position.
\end{key}

Let us briefly summarize the order in which \tikzname\ tries to
determine the node at which the graph should be anchored:
\begin{enumerate}
\item If the |anchor node=|\meta{node name} option given to the graph
  as a whole, the graph is anchored at \meta{node name}, provided
  there is a node of this name in the graph. (If there is no node of
  this name or if it is mispelled, the effect is the same as if this
  option had not been given at all.)
\item Otherwise, if there is a node where the |anchor here| option is
  specified, the first node with this option set is used.
\item Otherwise, if there is a node where the |desired at| option is
  set (perhaps implicitly through keys like |x|), the first such node
  is used.
\item Finally, in all other cases, the first node is used.
\end{enumerate}

In the above description, the ``first'' node refers to the node first
encountered in the specification of the graph.



\subsection{Orienting a Graph}

\label{subsection-library-graphdrawing-standard-orientation}

Just as a graph drawing algorithm cannot know \emph{where} a graph
should be placed on a page, it is also often unclear which
\emph{orientation} it should have. Some graphs, like trees, have a
natural direction in which they ``grow'', but for an ``arbitrary''
graph the ``natural orientation'' is, well, arbitrary.

As for anchoring, the graph drawing algorithm is ``told'' about
the desired ``orientation of certain edges and certain nodes'': for
each node and each edge, keys may specify a ``desired
orientation''. For edges, you can, for instance, request that an
edge should be vertical an go upwards by saying |slope=up|. For a
node, ``specifying an orientation'' means that the rest of the graph
should be rotated in such a way that certain other nodes or
vectors should lie at a certain angle relative to the current node.


\subsubsection{Orienting a Graph by Fixing the Slope of Edges}

The following keys are used to specify orientations:
\begin{key}{/graph drawing/orient=\meta{angle}}
  \keyalias{tikz}\keyalias{tikz/graphs}
  Adding this key to an edge tells the graph drawing engine that the
  edge should have a slope of the given \meta{angle}. This ``slope''
  is defined as the angle of the line connecting the start of the edge
  to the end of the edge (independently of the actual to-path of the
  edge, which might define a bend or more complicated shapes). For
  instance, a \meta{angle} of |45| requests that the end node is ``up
  and right'' relative to the start node.
  
  Instead of an \meta{angle}, you can also specify the standard
  direction texts |north| or |south east| and so forth and also
  |up|, |down|, |left|, and |right|.
    
\begin{codeexample}[]
\tikz \graph [spring layout]
{
  a -- { b, c, d, e -- {f, g, h} };
  h -- [orient=30] a;
};
\end{codeexample}
\end{key}

\begin{key}{/graph drawing/orient'=\meta{angle}}
  Same as above, only the rest of the graph should be flipped relative
  to the edge.
    
\begin{codeexample}[]
\tikz \graph [spring layout]
{
  a -- { b, c, d, e -- {f, g, h} };
  h -- [orient'=30] a;
};
\end{codeexample}
\end{key}


\subsubsection{Orienting a Graph by Fixing the Slope Between Nodes}

\begin{key}{/graph drawing/orient=\opt{|:|}\meta{angle}|:|\meta{another node}}
  \keyalias{tikz}\keyalias{tikz/graphs}
  Adding this version of the |orient| key (it is detected by the
  presence of the colon) to a node requests that the graph drawing
  engine should ensure that the straight line from the origin (typically
  the center) of the node to the origin of \meta{another node}
  should have a slope of \meta{angle}. Note that the current node
  and the \meta{another node} need not be connected by an edge.
\begin{codeexample}[]
\tikz \graph [spring layout]
{
  a [orient=:-90:f] -- { b, c, d, e -- {f, g, h} };
  { h, g } -- a;
};
\end{codeexample}
  
  Instead of an \meta{angle}, you can also specify the standard
  direction texts |north| or |south east| and so forth and also
  |up|, |down|, |left|, and |right|. Furthermore, the leading colon is
  optional: 
\begin{codeexample}[]
\tikz \graph [spring layout]
{
  a [orient=down:h] -- { b, c, d, e -- {f, g, h} };
  { h, g } -- a;
};
\end{codeexample}

  As special features, a dash somewhere inside the |orient| key is
  replaced by |:0:| and a vertical bar by |:-90:|. Thus, |orient=-a|
  is the same as |orient=:0:a|. Similarly:
\begin{codeexample}[]
\tikz \graph [spring layout]
{
  a [orient=|h] -- { b, c, d, e -- {f, g, h} };
  { h, g } -- a;
};
\end{codeexample}
\end{key}

\begin{key}{/graph drawing/orient'=\meta{angle}:\meta{another node}}
  Same as above, only the rest of the graph should be flipped relative
  to the edge.
\end{key}

Instead of specifying the slope between two nodes ``at the nodes'' it 
is sometimes more natural to specify it at the beginning of the
graph. For this, the following special key is available:

\begin{key}{/graph drawing/orient=\meta{node1}|:|\meta{angle}|:|\meta{node2}}
  \keyalias{tikz}\keyalias{tikz/graphs}
  This has nearly the same effect as specifying
  |orient=|\meta{angle}|:|\meta{node2} as an option for the node
  \meta{node1}. The only difference is that |orient| options given at
  a node always take precedence over this ``global'' option.

  As above, \meta{node1}|-|\meta{node2} gets replaced by
  \meta{node1}|:0:|\meta{node2} and \meta{node1}\verb!|!\meta{node2}
  \meta{node1}|:-90:|\meta{node2}.
\begin{codeexample}[]
\tikz \graph [spring layout] { a -- b -- c -- a };
\tikz \graph [spring layout,orient=a-b] { a -- b -- c -- a };
\tikz \graph [spring layout,orient=b-a] { a -- b -- c -- a };
\tikz \graph [spring layout,orient=b|a] { a -- b -- c -- a };
\tikz \graph [spring layout,orient=a:10:b] { a -- b -- c -- a };
\tikz \graph [spring layout,orient=1-2] { subgraph K_n[n=5] };
\tikz \graph [spring layout,orient=2-1] { subgraph K_n[n=5] };
\end{codeexample}
\end{key}

\begin{key}{/graph drawing/orient'=\meta{orientation}}
  \keyalias{tikz}\keyalias{tikz/graphs}
  Does the same as |orient| except that the nodes are flipped over the
  principal axis.
\begin{codeexample}[]
\tikz \graph [spring layout,orient=a-b]  { a -- b -- c -- a };
\tikz \graph [spring layout,orient'=a-b] { a -- b -- c -- a };
\end{codeexample}
\end{key}



\subsubsection{Orienting a Graph by Fixing the Direction of Growth of the Children}

\begin{key}{/graph drawing/grow=\meta{angle}}
  \keyalias{tikz}\keyalias{tikz/graphs}
  This key specifies in which direction the neighbors of a node
  ``should grow.'' For some graph drawing algorithms, especially for
  those that layout trees, but also for those that produce layered
  layouts, there is a natural direction in which the ``children'' of
  a node should be placed. For instance, saying |grow=down| will cause
  the children of a node in a tree to be placed in a left-to-right
  line below the node (as always, you can replace the \meta{angle}
  by direction texts). The children are requested to be placed in a
  counter-clockwise fashion, the |grow'| key will place them in a
  clockwise fashion.
  
  Note that when you say |grow=down| it is not necessarily the case
  that any particular node is actually directly below the current
  node; the key just requests that the direction of growth is
  downward.
  
  In principle, you can specify the direction of growth for each node 
  individually, but do not count on graph drawing algorithms to
  honour these wishes.
  
\begin{codeexample}[]
\tikz \graph [layered layout, sibling distance=5mm]
{
  a [grow=right] -- { b, c, d, e -- {f, g, h} };
  { h, g } -- a;
};
\end{codeexample}

  When you give the |grow=right| key to the graph as a whole, it will
  be applied to all nodes. This happens to be exactly what you want:
  
\begin{codeexample}[]
\tikz \graph [layered layout, grow=right, sibling distance=5mm]
{
  a -- { b, c, d, e -- {f, g, h} };
  { h, g } -- a;
};
\end{codeexample}
  
\begin{codeexample}[]
\tikz
  \graph [layered layout, grow'=right]
  {
    {a,b,c} --[complete bipartite] {e,d,f}
            --[complete bipartite] {g,h,i};
  };
\end{codeexample}
\end{key}
  
\begin{key}{/graph drawing/grow'=\meta{angle}}
  Same as above, only with the children in clockwise order.
\begin{codeexample}[]
\tikz \graph [layered layout, sibling distance=5mm]
{
  a [grow'=right] -- { b, c, d, e -- {f, g, h} };
  { h, g } -- a;
};
\end{codeexample}
\end{key}



\subsubsection{The Phases of the Orientation Procedure}
\label{subsection-graph-orientation-phases}

As for anchoring a graph, the different keys for orienting graphs may
easily produce conflicting demands, which need to be
resolved. The following steps are normally performed for each
connected component of the graph independently (see 
Section~\ref{section-gd-packing} for details on connected components),
but algorithms may choose to consider the graph ``as a whole''. In
this case, the following steps are performed only once for the whole
graph. 

\begin{enumerate}
\item
  The graph drawing algorithm is ``told'' about the desired
  orientations in the form of graph, node, and edge parameters. 
  An algorithm may or may not try to honor the desired
  orientations. As for the anchoring of graphs, for some algorithms it
  is natural and easy to restrict the way nodes are placed so as to
  honor orientation requests, for others this makes sense, at best, on
  a global scale. Algorithms will internally tell the graph drawing
  engine when they grow the graph in some direction. They can even
  indicate that they have already taken care of all growth demands for
  individual nodes.

  Nevertheless, the following steps are always performed:
\item
  The engine checks whether there is an edge in the graph whose
  slope has been fixed using the |orient| key. If there is at least one
  such edge, the first such edge is considered. The graph is rotated
  such that this edge has the desired slope. The orientation process
  stops at this point, all other orientation requests are ignored.
\item
  Otherwise, if there is no specified slope for any edge, it is
  checked whether there is a node with a specified |orient| to another
  node. If this is the case, the first such specification is taken for
  which the other node exists and the graph is rotated so that this
  orientation is satisfied. Again, the process stops in this case.
\item
  It is next checked whether there is a |orient| request for the graph
  as a whole like |orient=a-b|. Provided |a| and |b| exist, this
  request is honoured and the process stops.
\item
  If the algorithm has indicated that it has already taken care of all
  |grow| requests (using an internal function), the process stops at
  this point.
\item
  Otherwise, if none of the above cases is encountered, we look for
  a node with a |grow| key attached to it. If there is such a node,
  the graph is rotated so that the direction of growth of the graph is
  the desired growth direction. For this, the orientation procedure
  obviously needs to know what the direction of growth the algorithm
  was using; the algorithm signals this internally by setting the
  |growth_direction| of the algorithm object or by attaching a
  |growth_direction| to nodes. If an algorithm fails to attach such a
  direction, the direction of the first edge of the node is chosen
  and, for an isolated node, the direction is a line to the first node
  in the graph other than the current node.
  
  If no node has |grow| specified, the orientation is chosen in such a
  way as if |grow=down| had been specified for the first node of the
  graph.   
\end{enumerate}



\subsection{Packing of Connected Components}

\label{subsection-gd-component-packing}

Graphs may be composed of subgraphs or \emph{components} that are not
connected to each other. In order to draw these nicely, most graph
drawing algorithms split them into separate graphs, computes
their layouts with the same graph drawing algorithm independently and,
in a postprocessing step, arranges them next to each other. (Some
graph drawing algorithms will treat a graph ``as a whole''; for such
algorithms the following options do not apply.)

The default method for placing the different components works as
follows:

\begin{enumerate}
\item For each component, a layout is determined and the component is
  oriented as described
  Section~\ref{subsection-library-graphdrawing-standard-orientation}
  on orientation of graphs. 
\item Then, we differentiate between two kinds of components: Those
  that contain an anchored node and those that do not. As described in
  Section~\ref{subsection-library-graphdrawing-anchoring}, you can
  directly specify for a node where it should be placed on the
  page. If a component contains such a node, it is clear where the
  component should go (we call it ``anchored'').
  
  The packing collects and considers all unanchored nodes, plus the
  first anchored component, if such a component exists. The second and
  further anchored component are not considered during the packing
  process, they are simply anchored according to their anchor nodes. 
\item If the previous step has yielded two or more components that now
  need to be packed, they are sorted as prescribed by the
  |component order| key.
\item The first component is now placed (conceptually) at the
  origin. (The final position of this and all other components will be
  determined later, namely in the anchoring phase, but let us imagine
  that the first component lies at the origin at this point.)
\item The second component is now positioned relative to the first
  component. The ``direction'' in which the next component is placed
  relative to the first one is determined by the |component direction|
  key, so components can be placed from left to right or up to down or
  even in any angular direction. However, both internally and in the
  following description, we assume that the components are placed from
  left to right; other directions are achieved by doing some (clever)
  rotating of the arrangement achieved in this way.

  So, we now wish to place the second component to the right of the
  first component. The component is first shifted vertically according
  to some alignment strategy. For instance, it can be shifted so that
  the topmost node of the first component and the topmost node of the
  second component have the same vertical position. Alternatively, we
  might require that certain ``alignment nodes'' in both components
  have the same vertical position. There are several other strategies,
  which can be configured using the |component align| key.

  One the vertical position has been fixed, the horizontal position is
  computed. Here, two different strategies are available: First, image
  rectangular bounding boxed to be drawn around both components. Then
  we shift the second component such that the right border of the
  bounding box of the first component touches the left border of the
  bounding box of the second component. Instead of having the bounding
  boxes ``touch,'' we can also have a padding of |component sep|
  between them. The second strategy is more involved and also known as
  a ``skyline'' strategy. It works as follows: Imaging the second
  component to be placed far to the right of the first component. Now
  start moving the second component to the left until one of the nodes
  of the second component touches a node of the first component, and
  stop. Again, the padding |component sep| can be used to avoid the
  nodes actually touching each other. (In case you are worried that
  the second component might actually ``pass through'' the first
  component because the nodes of the two component have totally
  different vertical heights, rest assured, that such ``vertical
  holes'' are taken care of to avoid this case.)
\item
  After the second component has been placed, the third component is
  considered and positioned relative to the second one, and so on.
\item
  At the end, as hinted at earlier, the whole arrangement is rotate so
  that instead of ``going right'' the component go in the direction of
  |component direction|. Note, however, the this rotation applies only
  to the ``shift'' of the components; the components themselves are
  not rotated. Fortunately, this whole rotation process happens in the
  background and the result is normally exactly what you would expect.
\end{enumerate}

In the following, we go over the different keys that can be used to
configure the component packing.


\subsubsection{Ordering the Component}

The different connected components of the graph are collected in a
list. The ordering of the nodes in this list can be configured using
the following key:

\begin{key}{/graph drawing/component order=\meta{strategy} (initially
    by first specified node)}
  \keyalias{tikz}\keyalias{tikz/graphs}
  The following values are permissible for \meta{strategy}
  \begin{itemize}
  \item \declare{|by first specified node|}

    The components are ordered ``in the way they appear in the input
    specification of the graph''. More precisely, for each component
    consider the node that is first encountered in the description
    of the graph. Order the components in the same way as these nodes
    appear in the graph description.
\begin{codeexample}[]
\tikz \graph [tree layout, nodes={inner sep=1pt,draw,circle}]
{ a, b, c -- d -- e, f -- g };
\end{codeexample}
  \item \declare{|increasing node number|}
    
    The components are ordered by increasing number of nodes. For
    components with the same number of nodes, the first node in each
    component is considered and they are ordered according to the
    sequence in which these nodes appear in the input.
\begin{codeexample}[]
\tikz \graph [tree layout, nodes={inner sep=1pt,draw,circle},
              component order=increasing node number]
{ a, b, c -- d -- e, f -- g };
\end{codeexample}
    \begin{key}{/graph drawing/small components first}
      \keyalias{tikz}
      \keyalias{tikz/graphs}
      A shorthand for |component order=increasing node number|.
    \end{key}
  \item \declare{|decreasing node number|}
    As above, on in decreasing order.  
    \begin{key}{/graph drawing/large components first}
      \keyalias{tikz}
      \keyalias{tikz/graphs}
      A shorthand for |component order=decreasing node number|.
\begin{codeexample}[]
\tikz \graph [tree layout, nodes={inner sep=1pt,draw,circle},
              large components first]
{ a, b, c -- d -- e, f -- g };
\end{codeexample}
    \end{key}
  \end{itemize}
\end{key}


\subsubsection{Arranging Components in a Certain Direction}

\begin{key}{/graph drawing/component direction=\meta{angle} (initially 0)}
  \keyalias{tikz}\keyalias{tikz/graphs}
  The \meta{angle} is used to determine the relative position of each
  component relative to the previous one. The direction need not be a
  multiple of |90|.
\begin{codeexample}[]
\tikz \graph [tree layout, nodes={inner sep=1pt,draw,circle},
              component direction=left]
  { a, b, c -- d -- e, f -- g };
\end{codeexample}
\begin{codeexample}[]
\tikz \graph [tree layout, nodes={inner sep=1pt,draw,circle},
              component direction=10]
  { a, b, c -- d -- e, f -- g };
\end{codeexample}
  As usual, you can use texts like |up| or |right| instead of a
  number.

  As the example shows, the direction only has an influence on the
  relative positions of the components, not on the direction of growth
  inside the components. In particular, the components are not rotated
  by this option in any way. You can use the |grow| option or |orient|
  options to orient individual components:
\begin{codeexample}[]
\tikz \graph [tree layout, nodes={inner sep=1pt,draw,circle},
              component direction=up]
  { a, b, c [grow=right] -- d -- e, f[grow=45] -- g };
\end{codeexample}
\end{key}



\subsubsection{Aligning Components}

When components are placed next to each from left to right, it
is not immediately clear how the components should be aligned
vertically. What happens is the in each component a horizontal line is
determined and then all components are shifted vertically so that the
lines are aligned. There are different strategies for choosing these
``lines'', see the description of the options described later on.
When the |component direction| option is used to change the direction
in which components are placed, it certainly make no longer sense to
talk about ``horizontal'' and ``vertical'' lines. Instead, what
actually happens is that the alignment does not consider
``horizontal'' lines, but lines that go in the direction specified by
|component direction| and aligns them by moving components along a
line that is perpendicular to the line. For these reasons, let us call
the line in the component direction the \emph{alignment line} and a
line that is perpendicular to it the \emph{shift line}.

The first way of specifying through which point of a component the
alignment line should get is to use the following option:

\begin{key}{/graph drawing/align here}
  \keyalias{tikz}
  \keyalias{tikz/graphs}
  When this option is given to a node, this alignment line will go
  through the origin of this node. If this option is passed to more
  than one node of a component, the node encountered first in the
  component is used.
\begin{codeexample}[]
\tikz \graph [binary tree layout, nodes={draw}]
{ a, b -- c[align here], d -- e[second, align here] -- f };
\end{codeexample}
\end{key}

In many cases, however, you will not wish to specify an alignment node
manually in each component. Instead, you will use the following key to
specify a \emph{strategy} that should be used to automatically
determine such a node:

\begin{key}{/graph drawing/component align=\meta{strategy} (initially first node)}
  \keyalias{tikz}
  \keyalias{tikz/graphs}
  The following values are permissible:
  \begin{itemize}
  \item \declare{|first node|}
    In each component, the alignment line goes through the center of
    the first node of the component encountered during specification
    of the component.
\begin{codeexample}[]
\tikz \graph [binary tree layout, nodes={draw},
              component align=first node]
{ a, b -- c, d -- e[second] -- f };
\end{codeexample}
  \item \declare{|center|}
    
    The nodes of the component are projected onto the shift line. The
    alignment line is now chosen so that is exactly in the middle
    between the maximum and minimum value that the projected nodes
    have on the shift line.
\begin{codeexample}[]
\tikz \graph [binary tree layout, nodes={draw},
              component align=center]
{ a, b -- c, d -- e[second] -- f };
\end{codeexample}
\begin{codeexample}[]
\tikz \graph [binary tree layout, nodes={draw},
              component direction=90,
              component align=center]
{ a, b -- c, d -- e[second] -- f };
\end{codeexample}
  \item \declare{|counterclockwise|}

    As for |center|, we project the nodes of the component onto the
    shift line. The alignment line is now chosen so that it goes
    through the center of the node whose center has the highest
    projected value.
\begin{codeexample}[]
\tikz \graph [binary tree layout, nodes={draw},
              component align=counterclockwise]
{ a, b -- c, d -- e[second] -- f };
\end{codeexample}
\begin{codeexample}[]
\tikz \graph [binary tree layout, nodes={draw},
              component direction=90,
              component align=counterclockwise]
{ a, b -- c, d -- e[second] -- f };
\end{codeexample}
    The name |counterclockwise| is intended to indicate that the align
    line goes through the node that comes last if we go from the
    alignment direction in a counter-clockwise direction.
  \item \declare{|clockwise|}
    
    Works like |counterclockwise|, only in the other direction:
\begin{codeexample}[]
\tikz \graph [binary tree layout, nodes={draw},
              component align=clockwise]
{ a, b -- c, d -- e[second] -- f };
\end{codeexample}
\begin{codeexample}[]
\tikz \graph [binary tree layout, nodes={draw},
              component direction=90,
              component align=clockwise]
{ a, b -- c, d -- e[second] -- f };
\end{codeexample}
  \item \declare{|counterclockwise bounding box|}

    This method is quite similar to |counterclockwise|, only the
    alignment line does not go through the center of the node with a
    maximum projected value on the shift line, but through the maximum
    value of the projected bounding boxes. For a left-to-right
    packing, this means that the components are aligned so that the
    bounding boxes of the components are aligned at the top.
\begin{codeexample}[]
\tikz \graph [tree layout, nodes={draw, align=center},
              component align=counterclockwise]
{ a, "high\\node" -- b};
\tikz \graph [tree layout, nodes={draw, align=center},
              component align=counterclockwise bounding box]
{ a, "high\\node" -- b};
\end{codeexample}
  \item \declare{|clockwise bounding box|}
    
    Works like |counterclockwise bounding box|.
  \end{itemize}
\end{key}

Using a combination of |component direction| and |component align|,
numerous different packing strategies can be configured. However,
since names like |counterclockwise| are a bit hard to remember and to
apply in practice, a number of easier-to-remember keys are predefined
that combine an alignment and a direction:

\begin{key}{/graph drawing/components go right top aligned}
  \keyalias{tikz/graphs}
  \keyalias{tikz}
  Shorthand for |component direction=right| and
  |component align=counterclockwise|. This means that, as the name
  suggest, the components will be placed left-to-right and they are
  aligned such that their top nodes are in a line.  
\begin{codeexample}[]  
\tikz \graph [tree layout, nodes={draw, align=center},
              components go right top aligned]
  { a, "high\\node" -- b};
\end{codeexample}
\end{key}

\begin{key}{/graph drawing/components go right absolute top aligned}
  \keyalias{tikz/graphs}
  \keyalias{tikz}
  Like the previous key, but with
  |component align=counterclockwise bounding box|. This means that the
  components will be aligned with their bounding boxed being
  top-aligned: 
\begin{codeexample}[]  
\tikz \graph [tree layout, nodes={draw, align=center},
              components go right absolute top aligned]
  { a, "high\\node" -- b};
\end{codeexample}
\end{key}

\begin{key}{/graph drawing/components go right bottom aligned}
  \keyalias{tikz/graphs}
  \keyalias{tikz}
  As above, only with an alignment of the bottom nodes.
\end{key}

\begin{key}{/graph drawing/components go right absolute bottom aligned}
  \keyalias{tikz/graphs}
  \keyalias{tikz}
\end{key}

\begin{key}{/graph drawing/components go right center aligned}
  \keyalias{tikz/graphs}
  \keyalias{tikz}
  As above, but with the alignment at the centers.
\end{key}

\begin{key}{/graph drawing/components go right}
  \keyalias{tikz/graphs}
  \keyalias{tikz}
  Shorthand for |component direction=right| and
  |component align=first node|.
\end{key}

The above options are all available also with |right| replaced by
|left|. Here is an example:
\begin{codeexample}[]
\tikz \graph [tree layout, nodes={draw, align=center},
              components go left top aligned]
  { a, "high\\node" -- b};
\end{codeexample}
Next, the options are also available with |right| replaced by |up| and
also by |down|. Then, instead of |top| or |bottom| for the alignment,
|left| and |right| must be used:
\begin{codeexample}[]
\tikz \graph [tree layout, nodes={draw, align=center},
              components go down left aligned]
  { a, hello -- {world,s} };
\end{codeexample}
\begin{codeexample}[]
\tikz \graph [tree layout, nodes={draw, align=center},
              components go up absolute left aligned]
  { a, hello -- {world,s}};
\end{codeexample}



\subsubsection{The Distance Between Components}

Once the components of a graph have been oriented, sorted, aligned,
and a direction has been chosen, it remains to determine the distance
between adjacent components. Two methods are available for computing
this distance, as specified by the following option:

\begin{key}{/graph drawing/component packing=\meta{method} (initially
    skyline)}
  \keyalias{tikz}
  \keyalias{tikz/graphs}
  Given two components, their distance is computed as follows in
  depencende of \meta{method}:
  \begin{itemize}
  \item \declare{|rectangular|}

    Imagine a bounding box to be drawn around both components. They
    are then shifted such that the padding (separating distance)
    between the two boxes is the current value of |component sep|.
\begin{codeexample}[]
\tikz \graph [tree layout, nodes={draw}, component sep=0pt,
              component packing=rectangular]
  { a -- long text, longer text -- b};
\end{codeexample}
  \item \declare{|skyline|}

    The ``skyline method'' described in the introduction of 
    Section~\ref{subsection-library-graphdrawing-anchoring} is used to
    compute the distance.
\begin{codeexample}[]
\tikz \graph [tree layout, nodes={draw}, component sep=0pt,
              component packing=skyline]
  { a -- long text, longer text -- b};
\end{codeexample}
  \end{itemize}
\end{key}



\subsection{Implementing Graph Drawing Algorithms}

\label{section-gd-own-algorithm}
\label{section-library-graphdrawing-ownAlgorithm}

This section presents a simple example of how a graph drawing
algorithm can be implemented. A more detailed explanation of how
new graph drawing algorithms can be integrated and configured in given
in Section~\ref{section-gd-implementing-algorithms}.

As explained for the |algorithm| key, for each graph drawing algorithm
there must be a class of the name given to the |algorithm| key. This
class should usually reside in a file called
|pgfgd-algorithm-|\meta{algorithm name}. This class must provide (at
least) the two methods |new| and |run|. Each time a layout needs to
be computed for a graph, a new object of this algorithm class is
instantiated using the class's |new| method. For the newly created
object, an attribute |graph| will be set to an object representing the
graph. Then, the |constructor| method of the object is called,
provided it exists. Then, the |run| method is called, which should do
the actual work. (The separation into a constructor and a run method
is purely for convenience.) The |run| method should modify the
coordinates of the nodes of its |graph| attribute.

To simplify the creating of classes and constructors, the graph
drawing engine provides the function |graph_drawing_algorithm|, which
takes a table of infos about the algorithm as input and will create a
class and a constructor.

As a complete example, the following code fragment implements a
trivial graph drawing algorithm that just places all nodes on a
fixed-size circle.  It is accessed with the name 
|Simple Demo|.

\pgfgddeclareforwardedkeys{/graph drawing}{
  radius/.graph parameter=evaluate math expression,
  radius/.parameter initial=1cm,
  node radius/.node parameter=evaluate math expression
}

\begin{codeexample}[code only]
-- File pgfgd-algorithm-SimpleDemo.lua

graph_drawing_algorithm { name = "SimpleDemo" }

function SimpleDemo:run()
  local radius = 28.908  -- this is 1cm in points
  local alpha = (2 * math.pi) / #self.graph.nodes
  for i=1,#self.graph.nodes do
    self.graph.nodes[i].pos.x = radius * math.cos((i-1) * alpha)
    self.graph.nodes[i].pos.y = radius * math.sin((i-1) * alpha)
  end
end
\end{codeexample}

The algorithm computes a circular layout like in the following.

\begin{codeexample}[]
\tikz [graph drawing scope, /graph drawing/algorithm=Simple Demo]
  \graph { f -> c -> e -> a -> {b -> {c, d, f}, e -> b}};
\end{codeexample}

For details on how to make things like |radius| configurable and how
to setup keys so that users can just write
|\tikz[circular layout] ...|, please see Section~\ref{section-gd-implementing-algorithms}.


\endinput


%% Copyright 2011 by Renée Ahrens, Olof Frahm, Jens Kluttig, Matthias Schulz, Stephan Schuster
% Copyright 2011 by Till Tantau
%
% This file may be distributed and/or modified
%
% 1. under the LaTeX Project Public License and/or
% 2. under the GNU Free Documentation License.
%
% See the file doc/generic/pgf/licenses/LICENSE for more details.

\section{Graph Drawing Layouts: Trees}
\label{section-first-graphdrawing-library-in-manual}
\label{section-library-graphdrawing-trees}


\begin{tikzlibrary}{graphdrawing.trees}
  Load this package when you wish to use layout trees. You should load
  the |graphdrawing| library first. 
\end{tikzlibrary}

\ifluatex\relax\else{LuaTeX is required for setting this manual section.}\expandafter\endinput\fi


\subsection{Overview}

\tikzname\ offers several different syntax to specify trees (see
Sections \ref{section-trees} and~\ref{section-library-graphs}). By
default, \tikzname\ will attempt to produce a reasonable layout of the
specified trees, but since \TeX's algorithmic capabilities are quite
limited, no advanced layout of the tree is done by the standard
algorithms. This is where the graph drawing algorithms from this
library come in: Having the full power of Lua\TeX\ at their disposal,
they will produce a far better layout of the trees.


\subsection{The Reingold--Tilford Tree Layout}

\begin{gdalgorithm}{tree layout}{Tree Reingold Tilford 1981}
  This layout arranges nodes in a tree according to the
  Reingold--Tilford method.
\begin{codeexample}[]
\tikz [binary tree layout, sibling distance=7mm, level distance=10mm]
\graph [nodes={circle, inner sep=0pt, minimum size=2mm, fill}]{
  / -- { / -- / -- { / -- /, / -- { /, / }}, / -- / -- /[second] }
};
\end{codeexample}
\begin{codeexample}[]
\tikz \graph [binary tree layout, level distance=10mm] {
  Knuth -> {
    Beeton -> Kellermann [second] -> Carnes,
    Tobin -> Plass -> { Lamport, Spivak } 
  }
};\qquad
\tikz \graph [binary tree layout, grow'=right, sibling distance=5mm] {
  Knuth -> {
    Beeton -> Kellermann [second] -> Carnes,
    Tobin -> Plass -> { Lamport, Spivak } 
  }
};
\end{codeexample}
\begin{codeexample}[]
\tikz \graph [binary tree layout, grow'=30, sibling distance=5mm] {
  Knuth -> {
    Beeton -> Kellermann [second] -> Carnes,
    Tobin -> Plass -> { Lamport, Spivak } 
  }
};
\end{codeexample}
\end{gdalgorithm}

% \begin{gdalgorithm}{centered root tree layout}{A proof-of-concept by Ahrens et al., 2011}{AhrensFKSS2011 tree}
%   This algorithm was implemented as a proof-of-concept by Ahrens,
%   Frahm, Kluttig, Schulz, and Schuster, who have implemented the graph
%   drawing engine. The idea is that the 
%   root of each subtree is centered horizontally above the child
%   trees. The |sibling distance| and |level distance| keys are taken
%   into consideration.
  
% \begin{codeexample}[]
% \tikz[centered root tree layout, nodes=draw] \graph { a -> {b -> {c,d,f->{i,j,k}}, e}};
% \end{codeexample}

% As you can see, the text nodes aren't quite aligned, so the common fix
%   is to use the |text depth| and |text height| keys to force the text
%   nodes to a specific size.

% \begin{codeexample}[]
% \tikz[AhrensFKSS2011 tree, text depth=.2em, text height=.8em]
%   \graph { a -> {b -> {c,d}, e}};
% \end{codeexample}

%   \medskip
%   \noindent\textbf{Parameters.} 
%   The keys affecting the algorithm are the following common graph
%   drawing parameters:

%   \begin{key}{/graph drawing/root}
%     \keyalias{tikz}\keyalias{tikz/graphs}
%     This is a node parameter. At most one node should have this key
%     set. If no node has it set, the first node in the graph will be
%     used. 
%   \end{key}

%   \begin{key}{/graph drawing/level distance=\meta{leveldistance} (default 1cm)}
%     \keyalias{tikz}\keyalias{tikz/graphs}
%     Determines the vertical space between the nodes on different levels:
% \begin{codeexample}[]
% \tikz [AhrensFKSS2011 tree, level distance=1cm]
%   \graph { 1 -> {2 , 3}};
% \tikz [AhrensFKSS2011 tree, level distance=2cm]
%   \graph { 1 -> {2 , 3}};
% \end{codeexample}
%   \end{key}

%   \begin{key}{/graph drawing/sibling distance=\meta{siblingdistance} (default 1cm)}
%     \keyalias{tikz}\keyalias{tikz/graphs}
%     This determines the horizontal space between the nodes. 
% \begin{codeexample}[]
% \tikz [AhrensFKSS2011 tree, sibling distance=1cm]
%   \graph { 1 ->{2 , 3}};
% \tikz [AhrensFKSS2011 tree, sibling distance=2cm]
%   \graph { 1 ->{2 , 3}};
% \end{codeexample}
% \end{key}

%   \medskip
%   \textbf{How does this algorithm work?}
%   The tree algorithm works recursively. During the recursion one step is
%   performed for each subgraph of the tree.  

%   The process builds a kind of a box structure of the given graph. This
%   means a leaf of a tree returns itself as a box. Its parent returns
%   itself and its children in a bigger box etc. as shown in the following
%   figure. 

% \begin{quote}
% \begin{tikzpicture}[
%     level 0/.style={draw=black!50,very thick},
%     level 1/.style={draw=orange!50,very thick},
%     level 2/.style={draw=blue!50,very thick},
%     level 3/.style={draw=green!50,very thick}]

%     \node[level 1] (1) {1}
%       child {node[level 2] (3) {3}
%         child {node[level 3] (4) {4}
%           child{node[level 3] (6) {6}}
%           child{node[level 3] (7) {7}}
%         }
%         child {node[level 2] (5) {5}}
%       }
%       child {node[level 1] (2) {2}};

%     \begin{pgfonlayer}{background}
%         \node [level 0, fit=(1) (6) (2)] {};
%         \node [level 1, fit=(3) (5) (6) (7)] {};
%         \node [level 2,fit=(4) (6) (7)] {};
%     \end{pgfonlayer}
% \end{tikzpicture}
% \end{quote}

%   In each step the current boxes can be compared by their size, sorted
%   and positioned. In the figure above the boxes of one step are
%   represented in the same color. 
  
%   In the tree algorithm the boxes of each tree level are first sorted
%   ascendingly by their size and then arranged as follows: The the
%   biggest box is positioned in the middle. Then the following boxes are
%   positioned alternately left and right. 
  
%   After this arrangement the relative coordinates for the position of
%   each box have to be computed. The \emph{y}-coordinate of a box (except
%   for the root node of the step) are determined by the maximum height of
%   all boxes to guarantee a uniform layout of the tree. Nodes on the same
%   level in the tree are positioned at the same height. The
%   \emph{x}-coordinate of a box depends on the coordinates of its left
%   neighbour box and an additional spacing (by default 10pt), which can
%   be influenced by the |sibling distance| key. The \emph{y}-coordinate
%   of the root node of each step is set to the maximum \emph{y}-value of
%   the other boxes adding the same spacing meantioned above (by default
%   10pt, influenced by |level distance|). Its \emph{x}-coordinate is
%   determined by the width of the other boxes divided by 2. This means
%   the root node is positioned in the middle above the other boxes. 
  
%   Because each box knows its root node, it is possible to determine the
%   absolute position of each box or node afterwards.  
  
%   At the end of the step the current boxes are added to a result box and
%   returned. 
%\end{gdalgorithm}


%%% Local Variables: 
%%% mode: latex
%%% TeX-master: "pgfmanual-pdftex-version"
%%% End: 

%% Copyright 2012 by Till Tantau
%
% This file may be distributed and/or modified
%
% 1. under the LaTeX Project Public License and/or
% 2. under the GNU Free Documentation License.
%
% See the file doc/generic/pgf/licenses/LICENSE for more details.

\section{Graph Drawing Algorithms: Layered Layouts}

{\emph{by  Till Tantau and Jannis Pohlmann}}

\begin{tikzlibrary}{graphdrawing.layered}
  This library provides keys for drawing graphs using the Sugiyama
  method, which is especially useful for drawing hierachical graphs.
  You should load the |graphdrawing| library first.
\end{tikzlibrary}



\subsection{Overview}

A ``layered'' layout of a graph tries to arrange the nodes in
consecutive horizontal layers (naturally, by rotating the graph, this
can be changed in to vertical layers) such that edges tend to be only
between nodes on adjacent layers. Trees, for instance, can always be
laid out in this way. This method of laying out a graph is especially
useful for hierarchical graphs.

The method implemented in this library is often called the
\emph{Sugiyama method}, which is a rather advanced method of
assigning nodes to layers and positions on these layers. The same
method is also used in the popular GraphViz program, indeed, the
implementation in \tikzname\ is based on the same pseudo-code from the
same paper as the implementation used in GraphViz and both programs
will often generate the same layout (but not always, as explained
below). The current implementation is due to Jannis Pohlmann, who
implemented it as part of his Diploma thesis. Please consult this
thesis for a detailed explanation of the Sugiyama method and its
history:

\begin{itemize}
\item
  Jannis Pohlmann,
  \newblock \emph{Configurable Graph Drawing Algorithms
    for the \tikzname\ Graphics Description Language,}
  \newblock Diploma Thesis,
  \newblock Institute of Theoretical Computer Science, Univerist\"at
  zu L\"ubeck, 2011.\\[.5em]
  \newblock Online at 
  \url{http://www.tcs.uni-luebeck.de/downloads/papers/2011/2011-configurable-graph-drawing-algorithms-jannis-pohlmann.pdf}
  \\[.5em]
  (Note that since the publication of this thesis some option names
  have been changed. Most noticeably, the option name
  |layered drawing| was changed to |layered layout|, which is somewhat
  more consistent with other names used in the graph drawing
  libraries.) 
\end{itemize}

The Sugiyama methods lays out a graph in five steps:
\begin{enumerate}
\item Cycle removal.
\item Layer assignment (sometimes called node ranking).
\item Crossing minimization (also referred to as node ordering).
\item Node positioning (or coordinate assignment).
\item Edge routing.
\end{enumerate}
It turns out that behind each of these steps there lurks an
NP-complete problem, which means, in practice, that each step is
impossible to perform optimally for larger graphs. For this reason,
heuristics and approximation algorithms are used to find a ``good''
way of performing the steps.

A distinctive feature of Pohlmann's implementation of the Sugiyama
method for \tikzname\ is that the algorithms used for each of the
steps can easily be exchanged, just specify a different option. For
the user, this means that by specifying a different 
option and thereby using a different heuristic for one of the steps, a
better layout can often be found. For the researcher, this means that
one can very easily test new approaches and new heuristics without
having to implement all of the other steps anew. 



\subsection{Standard Layered Layout}

In order to compute a layered layout of a graph, use the following option:

\begin{gdalgorithm}{layered layout}{Sugiyama Modular Layered}
  The |layered layout| is the key used to select the modular Sugiyama
  layout algorithm. As explained in the overview of this section, this
  algorithm consists of five consecutive steps, each of which can be
  configured independently of the other ones (how this is done is
  explained later in this section). Naturally, the ``best'' heuristics
  are selected by default, so there is typically no need to change the
  settings, but what is the ``best'' method for one graph need not be
  the best one for another graph.
  
\begin{codeexample}[]
\tikz \graph [layered layout, sibling distance=7mm]
{
  a -> {
    b,
    c -> { d, e, f }
  } ->
  h ->
  a
};    
\end{codeexample}

  As can be seen in the above example, the algorithm will not only
  position the nodes of a graph, but will also perform an edge
  routing. This will look visually quite pleasing if you add the
  |rounded corners| option:

\begin{codeexample}[]
\tikz [rounded corners] \graph [layered layout, sibling distance=7mm]
{
  a -> {
    b,
    c -> { d, e, f }
  } ->
  h -> 
  a
};    
\end{codeexample}


\end{gdalgorithm}



%%% Local Variables: 
%%% mode: latex
%%% TeX-master: "pgfmanual-pdftex-version"
%%% End: 

%% Copyright 2011 by Jannis Pohlmann
%
% This file may be distributed and/or modified
%
% 1. under the LaTeX Project Public License and/or
% 2. under the GNU Free Documentation License.
%
% See the file doc/generic/pgf/licenses/LICENSE for more details.

\section{Force-Based Graph Drawing Algorithms}
\label{section-library-graphdrawing-force-based}

{\emph{by Jannis Pohlmann}}


\begin{tikzlibrary}{graphdrawing.force}
  Load this package when you wish to use force-based graph drawing
  algorithms. You should load the |graphdrawing| library first.
\end{tikzlibrary}

\ifluatex\relax\else{LuaTeX is required for setting this manual section.}\expandafter\endinput\fi


\subsection{Overview}

% TODO Jannis: Explain ideas and concepts behind force-based graph
% drawing algorithms. Briefly explain the various approaches in that
% specific area of graph drawing algorithms (e.g. spring,
% spring-electrical and multidimensional embedding). 

...

\subsubsection{Spring and Spring-Electrical Layouts}

% TODO Jannis: Explain ideas and concepts behind spring and
% spring-electrical algorithms. Describe the technical as well as visual
% differences between the two techniques (think: no attractive forces
% and no peripheral effects in spring layouts). Explain why they were
% consolidated in the common family 'spring layout'.

\begin{key}{/graph drawing/spring layout=\meta{options}}
  \keyalias{tikz}\keyalias{tikz/graphs}
  Similar to the |>| option, this ``generic'' name for a spring layout
  algorithm is not hardwired to any specific algorithm. Rather, users
  can select an algorithm somewhere at the beginning of their program
  and then just write |\graph[spring layout]| to draw a tree.

  The \meta{options} will be forwarded to the currently selected
  algorithm.
\begin{codeexample}[]
\tikz \graph [spring layout] { a -> {b,c} };    
\end{codeexample}
  
  To change the algorithm, change the following key:
  \begin{key}{/graph drawing/spring layout/default algorithm=\meta{algorithm}}
    Set this key to the tree drawing algorithm of your choice. The
    default is currently set to the algorithm
    |Walshaw2000 spring electrical|, but this will change. 
  \end{key}
\end{key}


\subsection{Common Options}

The spring and and spring-electrical drawing algorithms are very similar
in terms of their parameters and the constraints they can handle. They
thus share a number of common \tikzname\ options for fine-tuning. These
options are split up into \emph{graph options} that can be specified
once for a graph, \emph{node options} that can be specified for each
node and \emph{edge options} that can be specified for each edge.

\subsubsection{Graph Options}

% TODO Jannis: This might be worth implementing. It's not very useful in
% the Hu2006 algorithm as it uses the Barnes-Hut algorithm, but the 
% Walshaw2000 algorithm can benefit from it.
%
%\begin{key}{/tikz/influence cutoff distance=\meta{dimension} (initially
%  0pt)}
%  Specifies a distance beyond which the attractive and repulsive forces 
%  between two nodes are assumed to be virtually non-existent. If 
%  \meta{dimension} is set to |0pt|, the cutoff distance is computed 
%  automatically.
%
%  Depending on the graph drawing algorithm being used, the distance
%  between two nodes is computed either based on the graph distance
%  (spring algorithm) or based on the Euclidean distance
%  (spring-electrical algorithm).
%  \begin{codeexample}[]
%  \end{codeexample}
%\end{key}

\begin{key}{/graph drawing/spring layout/maximum 
  iterations=\meta{number} (initially 500)}
  Depending on the characteristics of the input graph and the parameters
  chosen for the spring or spring-electrical algorithm, minimizing the
  system energy may require many iterations.

  In these situations it may come in handy to limit the number of
  iterations. This feature can also be useful to draw the same graph
  after different iterations and thereby demonstrate how the spring or
  spring-electrical algorithm improves the drawing step by step.
  \begin{codeexample}[]
\tikz \graph [spring layout={maximum iterations=1}]   { a -- b -- c -- a };
\tikz \graph [spring layout={maximum iterations=10}]  { a -- b -- c -- a };
\tikz \graph [spring layout={maximum iterations=500}] { a -- b -- c -- a };
  \end{codeexample}
\end{key}

\begin{key}{/graph drawing/spring layout/random seed=\meta{number} 
  (initially 42)}
  Specifies the seed used for Lua's pseudo-random number generator. If
  set to something other than |0|, the random number sequence generated
  by the pseudo-random number generator will be the same at every run.
  The resulting graph drawings will be reproducible in consecutive runs,
  despite randomized elements used in the algorithm.
  If set to |0|, the results are not guaranteed to be reproducible.
  \begin{codeexample}[width=5.5cm]
\tikz \graph [spring layout={random seed=1}] { 
  subgraph K_n[n=4]
};
\tikz \graph [spring layout={random seed=10}] { 
  subgraph K_n[n=4]
};
  \end{codeexample}
\end{key}

\begin{key}{/graph drawing/spring layout/natural spring
  dimension=\meta{dimension} (initially 1cm)}
\end{key}

\begin{key}{/graph drawing/spring layout/spring constant=\meta{number}}
\end{key}

\begin{key}{/graph drawing/spring layout/approximate repulsive
  forces=\opt{\meta{boolean}} (initially true)}
\end{key}

\begin{key}{/graph drawing/spring layout/cooling factor=\meta{number}
  (initially 0.95)}
\end{key}

\begin{key}{/graph drawing/spring layout/convergence
  tolerance=\meta{number} (initially 0.01)}
\end{key}

\begin{key}{/graph drawing/approximate repulsive 
  forces=\opt{\meta{boolean}} (initially false)}
  
  Computing the repulsive forces of the nodes in a graph requires 
  $\mathcal{O}(n^2)$ operations in each iteration of spring- and
  spring-electrical algorithms, where $n$ is the number of nodes in
  the graph. For $l$ coarse graphs, this may increase the runtime of 
  such an algorithm by up to $l\cdot\mathcal{O}(n^2)$ operations. 

  With |approximate repulsive forces| set to |true|, repulsive forces 
  are approximated using the Barnes-Hut algorithm known from solving the
  so-called $N$-body problem. This reduces the number of operations
  needed to compute the repulsive forces to $\mathcal{O}(n\log n)$ per 
  iteration and can thus lead to a significant improvement of the 
  algorithm runtime.
  
  However, this optimization \emph{can} come at the cost of slightly 
  less appealing drawings which is noticable with small graphs in 
  particular. This is why it is turned off by default. Enable it if you
  want to lay out large graphs.

  Here is an example where this disadvantage can be noticed:
  \begin{codeexample}[]
\tikz \graph [spring layout] { 
  { [clique] 3, 5, 1 } -- { [clique] 2, 4, 6 }
};
\tikz \graph [spring layout={approximate repulsive forces}] { 
  { [clique] 3, 5, 1 } -- { [clique] 2, 4, 6 }
};
  \end{codeexample}

  Sometimes, the negative effect is very subtle. Notice how the angle
  of the |2| |4| |6| clique is slightly less appealing in the drawing
  with repulsive forces approximated.
  \begin{codeexample}[]
\tikz \graph [spring layout,orient=2|1] { 
  { 1 -- 3 -- 5 -- 1, 1 -- 2, 2 -- 4 -- 6 -- 2}
};
\tikz \graph [spring layout={approximate repulsive forces},orient=2|1] { 
  { 1 -- 3 -- 5 -- 1, 1 -- 2, 2 -- 4 -- 6 -- 2}
};
  \end{codeexample}

  In the the following example the opposite is the case even. Here,
  approximating the repulsive force generates a better layout than
  computing them accurately:
  \begin{codeexample}[width=6cm]
\tikz \graph [spring layout,orient=1|2] { 
  subgraph Grid_n[n=9]
};
\tikz \graph [spring layout={approximate repulsive forces},
              orient=1|2] { 
  subgraph Grid_n[n=9]
};
  \end{codeexample}
  
  As you can see it is dependent on the graph and other parameters of
  the spring and spring-electrical algorithms as to whether or not it
  makes sense to enable |approximate repulsive forces|.
\end{key}

\begin{key}{/graph drawing/spring layout/coarsen=\opt{\meta{boolean}}
  (initially true)}
  Defines whether or not a multilevel approach is used that
  iteratively coarsens the input graph into graphs $G_1,\dots,G_l$ with 
  a smaller and smaller number of nodes. The coarsening stops as soon as
  a minimum number of nodes is reached, as set via the 
  |minimum graph size| option or when, in the last iteration, the 
  number of nodes was not reduced by at least the ratio specified via 
  |downsize ratio|. 

  A random initial layout is computed for the coarsest graph $G_l$ first.
  Afterwards, it is laid out by computing the attractive and repulsive
  forces between its nodes. 
  
  In the subsequent steps, the previous coarse graph $G_{l-1}$ is 
  restored and its node positions are interpolated from the nodes in 
  $G_l$. $G_{l-1}$ is again laid out by computing the forces between 
  its nodes. These steps are repeated with $G_{l-2},\dots,G_1$ until 
  the original input graph $G_0$ has been restored, interpolated 
  and laid out.

  There are a number of options to fine-tune the coarsening approach.
  They are consolidated in the |/graph drawing/spring layout/coarsening|
  prefix described below.
\end{key}

\begin{key}{/graph drawing/spring layout/coarsening=\marg{options}}
  Executes the \meta{options} with the path prefix 
  |/graph drawing/spring layout/coarsening|.

  These options can be used to configure the coarsening approach
  described in the documentation of the 
  |/graph drawing/spring layout/coarsen| option.
\end{key}

\begin{key}{/graph drawing/spring layout/coarsening/minimum graph
  size=\meta{number} (initially 2)}
  Defines the number of nodes down to which the graph is coarsened
  iteratively. The first graph that has a lesser or equal number of
  nodes becomes the coarsest graph $G_l$, where $l$ is the number of
  coarsening steps. The algorithm proceeds with the steps described in
  the documentation of the |/graph drawing/spring layout/coarsen|
  option.

  In the following example the same graph is coarsened down to two
  and three nodes, respectively. The layout of the original graph is 
  interpolated from the random initial layout and is not changed
  because the forces are not computed. Thus, in the first graph, the
  nodes can have exactly two (or three) possible coordinates in the
  final drawing:
  \begin{codeexample}[]
\tikz \graph [spring layout={maximum iterations=0,coarsen,
                             coarsening={minimum graph size=2}}] { 
  1 -- 2 -- 3 -- 4 
};
\tikz \graph [spring layout={maximum iterations=0,coarsen,
                             coarsening={minimum graph size=3}}] { 
  1 -- 2 -- 3 -- 4 
};
  \end{codeexample}
\end{key}

\begin{key}{/graph drawing/spring layout/coarsening/downsize
  ratio=\meta{number} (initially 0.25)}
\end{key}

\begin{key}{/graph drawing/spring layout/coarsening/collapse independent
  edges=\opt{\meta{boolean}} (initially true)}
\end{key}

\begin{key}{/graph drawing/spring layout/coarsening/connected
  independent nodes=\opt{\meta{boolean}} (initially false)}
\end{key}

%\begin{key}{/tikz/coarsening=\marg{options}}
%  Executes the \meta{options} with the path prefix |/tikz/coarsening|.
%  
%  These options define whether a multilevel approach is used that
%  successively coarsend into graphs with smaller and smaller number
%  of nodes. These graphs are arranged first and are then interpolated
%  into the finer graphs at the previous level. How this is done exactly
%  can be configured using the |coarsening| options described below.
%\end{key}
%
%\begin{key}{/tikz/coarsening/randomized=\opt{\meta{boolean}} (default
%  true, initially false)}
%  If set to |true|, nodes will be inspected in a random order. The
%  effect on the final drawing can only be seen by experimenting with the
%  option.
%  \begin{codeexample}[]
%  \end{codeexample}
%\end{key}
%
%\begin{key}{/tikz/coarsening/minimum size=\meta{number} (default 0)}
%  Defines the minimum number of nodes in a coarsened graph. If a
%  coarsened graph has less than \meta{number} nodes, then... % TODO
%  \begin{codeexample}[] 
%% the same graph with different minimum size values
%  \end{codeexample}
%\end{key}
%
%\begin{key}{/tikz/coarsening/nodes=\opt{\meta{boolean}} (default true,
%  initially false)}
%  \begin{codeexample}[]
%  \end{codeexample}
%\end{key}
%
%\begin{key}{/tikz/coarsening/nearby nodes=\opt{\meta{boolean}} (default
%  true, initially false)}
%  \begin{codeexample}[]
%  \end{codeexample}
%\end{key}
%
%\begin{key}{/tikz/coarsening/nodes with more 
%  neighbors=\opt{\meta{boolean}} (default true, initially false)}
%  \begin{codeexample}[]
%  \end{codeexample}
%\end{key}
%
%\begin{key}{/tikz/coarsening/nearby nodes with more 
%  neighbors=\opt{\meta{boolean}} (default true, initially false)}
%  \begin{codeexample}[]
%  \end{codeexample}
%\end{key}
%
%\begin{key}{/tikz/coarsening/edges=\opt{\meta{boolean}} (default true,
%  initially false)}
%  \begin{codeexample}[]
%  \end{codeexample}
%\end{key}
%
%\begin{key}{/tikz/coarsening/heavy edges=\opt{\meta{boolean}} (default
%  true, initially false)}
%  \begin{codeexample}[]
%  \end{codeexample}
%\end{key}
%
%\begin{key}{/tikz/coarsening/edges with light nodes=\opt{\meta{boolean}}
%  (default true, initially false)}
%  \begin{codeexample}[]
%  \end{codeexample}
%\end{key}
%
%\begin{key}{/tikz/minimum energy delta=\meta{number} (default TODO)}
%  \begin{codeexample}[]
%  \end{codeexample}
%\end{key}
%
%\begin{key}{/tikz/initial step size=\meta{dimension} (default TODO)}
%  \begin{codeexample}[]
%  \end{codeexample}
%\end{key}
%
%\begin{key}{/tikz/step control=\meta{text} (default TODO)}
%  Possible values: |monotonic|, |non-monotonic|, |strictly monotonic|.
%  \begin{codeexample}[]
%  \end{codeexample}
%\end{key}

\subsubsection{Node Options}

%\begin{key}{/tikz/electric charge=\meta{number} (default 1)}
%  Defines the electric charge of the node. The stronger the electric
%  charge, the higher the repulsive force between two nodes. Set this to
%  something between |0| and |1| to reduce the charge compared to the
%  normal setup. Values larger than |1| will generate stronger repulsion
%  between the node and the others.
%  \begin{codeexample}[] 
%\tikz \graph [spring electrical layout,orient=1:90:2] {
%  1 -- 2 -- 3 -- 4 -- 1,
%  1 -- 3, 2 -- 4,
%};
%\tikz \graph [spring electrical layout,orient=1:90:2] {
%  1 [electric charge=1] -- 2 -- 3 -- 4 -- 1,
%  1 -- 3, 2 -- 4,
%};
%\tikz \graph [spring electrical layout,orient=1:90:2] {
%  1 [electric charge=1000] -- 2 [electric charge=1000] -- 3 -- 4 -- 1,
%  1 -- 3, 2 -- 4,
%};
%  \end{codeexample}
%\end{key}

%\end{document}

% TODO Jannis: Explain this one better. Also, compare it to /tikz/at,
% which will move the node after the drawing has been computed, as
% opposed to /graph drawing/desired at, which will only move the node
% while computing the layout.
%
%\begin{key}{/tikz/desired at=\meta{coordinate}}
%  Nails the node down at the specified \meta{coordinate}. It will not
%  move from there despite the repulsive and attractive forces in the
%  system. Note that, while sometimes generating a similar effect, using
%  |at| is very different from altering the orientation of a graph
%  drawing (see section~\ref{subsection-library-graphdrawing-standard-orientation}).
%  Also, if an orientation is specified, it is given priority over
%  the |at| option in that nodes are first fixated at their |at|
%  coordinates but are later moved in order to satisfy the orientation 
%  desired by the user.
%  \begin{codeexample}[width=6.0cm]
%\tikz \graph [spring layout] {
%  1 -- 2 -- 3 -- 4 -- 2
%};
%\tikz \graph [spring layout] {
%  1 [at={(0,0)}] -- 2 [at={(0,1)}] -- 3 -- 4 -- 2
%};
%  \end{codeexample}
%\end{key}


% TODO what about node groups / clusters? This works via color classes
% but how do we define their layouts (cluster, line, circle)?

\subsubsection{Edge Options}

\begin{key}{/tikz/natural length=\meta{dimension} (default 10pt)}
  Defines the natural (zero energy) length of the edge. The smaller the
  length, the stronger the attractive force of the adjacent nodes. The
  \meta{dimension} has a strong influence of how far the nodes will be
  placed from each other in the final drawing.
  \begin{codeexample}[]
% two examples with the same graph
% notably change the natural length of one of the edges
  \end{codeexample}
\end{key}

\begin{key}{/tikz/stiffness=\meta{number} (default 0.5)}
  Defines how flexible the spring associated with the edge is. The
  higher this value is, the closer the final edge length will be to its
  |natural length|.
  \begin{codeexample}[]
% two examples with the same graph
% notably change the stiffness of one of the edges
  \end{codeexample}
\end{key}

\subsection{Options for the Spring Algorithm}

\subsubsection{Graph Options}

...

\subsubsection{Node Options}

...

\subsubsection{Edge Options}

...

\subsection{Options for the Spring-Electrical Algorithm}

\subsubsection{Graph Options}

...

\subsubsection{Node Options}

...

\subsubsection{Edge Options}

...

\begin{codeexample}[]
\vbox{ \hsize=16cm \rightskip=0cm plus 1fill
  \foreach \iterations in {1,...,20,100,500}
  {
    \tikz \graph [spring layout={maximum iterations=\iterations}, orient=1-2] 
      { subgraph K_n[n=4] };
    \penalty0
  }
}
\end{codeexample}

\begin{codeexample}[]
\vbox{ \hsize=16cm \rightskip=0cm plus 1fill
  \foreach \iterations in {1,...,20,100,500}
  {
    \tikz \graph [spring layout={maximum iterations=\iterations}, orient=1-2] 
      { subgraph C_n[n=7] };
    \penalty0
  }
}
\end{codeexample}

\endinput

%% TODO
%% Explain the following concepts:
%% - separation of graph drawing options and regular TikZ options
%% - generic graph drawing options:
%%   - component packing
%%   - orientation
%% - pre-defined graph drawing styles
%% - graph drawing options for fine-tuning the different algorithms

%%% Local Variables: 
%%% mode: latex
%%% TeX-master: "pgfmanual-pdftex-version"
%%% End: 

% Copyright 2011 by Jannis Pohlmann
%
% This file may be distributed and/or modified
%
% 1. under the LaTeX Project Public License and/or
% 2. under the GNU Free Documentation License.
%
% See the file doc/generic/pgf/licenses/LICENSE for more details.

\section{Graph Drawing Algorithms: Circular Layouts}


\begin{tikzlibrary}{graphdrawing.circular}
  Load this package when you wish to use the graph drawing algorithms
  that place nodes on circles. You should load the |graphdrawing| library first.
\end{tikzlibrary}


\begin{gdalgorithm}{circular layout}{Circular Layout Tantau 2012}
  This layout arranges the nodes in a circle. The centers of the nodes
  are placed on a counter-clockwise circle, starting with the first
  node at the |grow| direction (for |grow'|, the circle is
  clockwise). The order of the nodes is the order in which they appear
  in the graph, the edges are not taken into consideration.

\begin{codeexample}[]
\tikz[>=spaced stealth']
  \graph [circular layout, grow'=down, sibling sep=1em,
          nodes={draw,circle}, math nodes]
  {
    x_1 -> x_2 -> x_3 -> x_4 ->
    x_5 -> "\dots"[draw=none] -> "x_{n-1}" -> x_n -> x_1
  };    
\end{codeexample}

  The nodes are placed in such a way that
  \begin{enumerate}
  \item The (angular) distance between the centers of consecutive
    nodes is at least  |sibling distance|,
  \item the distance between the borders of consecutive nodes is at
    least |sibling sep|, and
  \item the radius is at least |circular layout/radius|.
    \begin{key}{/graph drawing/circular layout/radius=\meta{radius}}
      The minimum radius of the circle.
    \end{key}
  \end{enumerate}
  The radius of the circle is chosen near-minimal such that the above
  properties are satisfied. To be more precise, if all nodes are
  circles, the radius is chosen optimally while for, say, rectangular
  nodes there may be too much space between the nodes in order to
  satisfy the second condition.

\begin{codeexample}[]
\tikz \graph [circular layout,
          sibling sep=0pt, sibling distance=0pt,
          nodes={draw,circle}]
  { 1 -- 2 [minimum size=30pt] -- 3 --
    4 [minimum size=50pt] -- 5 [minimum size=40pt] -- 6 -- 7 }; 
\end{codeexample}

\begin{codeexample}[]
\begin{tikzpicture}
  \graph [circular layout={radius=1.25cm},
          sibling sep=0pt, sibling distance=0pt,
          nodes={draw,circle}]
  { 1 -- 2 [minimum size=30pt] -- 3 --
    4 [minimum size=50pt] -- 5 [minimum size=40pt] -- 6 -- 7 }; 
  
  \draw [red] (0,-1.25) circle [radius=1.25cm];
\end{tikzpicture}
\end{codeexample}

\begin{codeexample}[]
\tikz \graph [circular layout,
    sibling sep=0pt, sibling distance=1cm,
    nodes={draw,circle}]
  { 1 -- 2 [minimum size=30pt] -- 3 --
    4 [minimum size=50pt] -- 5 [minimum size=40pt] -- 6 -- 7 }; 
\end{codeexample}

\begin{codeexample}[]
\tikz \graph [circular layout,
    sibling sep=2pt, sibling distance=0pt,
    nodes={draw,circle}]
  { 1 -- 2 [minimum size=30pt] -- 3 --
    4 [minimum size=50pt] -- 5 [minimum size=40pt] -- 6 -- 7 }; 
\end{codeexample}

\begin{codeexample}[]
\tikz \graph [circular layout,
    sibling sep=0pt, sibling distance=0pt,
    nodes={rectangle,draw}]
  { 1 -- 2 [minimum size=30pt] -- 3 --
    4 [minimum size=50pt] -- 5 [minimum size=40pt] -- 6 -- 7 }; 
\end{codeexample}
\end{gdalgorithm}




%%% Local Variables: 
%%% mode: latex
%%% TeX-master: "pgfmanual-pdftex-version"
%%% End: 

%% Copyright 2011 by Jannis Pohlmann
%
% This file may be distributed and/or modified
%
% 1. under the LaTeX Project Public License and/or
% 2. under the GNU Free Documentation License.
%
% See the file doc/generic/pgf/licenses/LICENSE for more details.

\section{Graph Drawing Layouts: Miscellaneous}
\label{section-last-graphdrawing-library-in-manual}


\begin{tikzlibrary}{graphdrawing.misc}
  Load this package when you wish to use the graph drawing algorithms
  defined in this library. You should load the |graphdrawing| library first.
\end{tikzlibrary}


\begin{gdalgorithm}{circular layout}{Circular Layout Tantau 2012}
  TODO: Document this...

\begin{codeexample}[]
\tikz[>=spaced stealth']
  \graph [circular layout, grow'=down, sibling sep=1em,
          nodes={draw,circle}, math nodes]
  {
    x_1 -> x_2 -> x_3 -> x_4 ->
    x_5 -> "\dots"[draw=none] -> "x_{n-1}" -> x_n -> x_1
  };    
\end{codeexample}

\begin{codeexample}[]
\tikz[>=spaced stealth']
  \graph [circular layout, grow'=30, sibling sep=1em,
          nodes={draw,circle}]
  { subgraph K_n [n=8] }; 
\end{codeexample}
\end{gdalgorithm}

\begin{gdalgorithm}{simple demo layout}{Simple Demo}
  The algorithm used in the examples of this manual for demonstrating
  how a trivial graph drawing can be implemented.
\end{gdalgorithm}




%%% Local Variables: 
%%% mode: latex
%%% TeX-master: "pgfmanual-pdftex-version"
%%% End: 

%% Copyright 2010-2011 by Renée Ahrens
% Copyright 2010-2011 by Olof Frahm
% Copyright 2010-2011 by Jens Kluttig
% Copyright 2010-2011 by Matthias Schulz
% Copyright 2010-2011 by Stephan Schuster
% Copyright 2011 by Jannis Pohlmann
%
% This file may be distributed and/or modified
%
% 1. under the LaTeX Project Public License and/or
% 2. under the GNU Free Documentation License.
%
% See the file doc/generic/pgf/licenses/LICENSE for more details.

\section{Graph Drawing Internals}
\label{section-base-graphdrawing}

\ifluatex\relax\else{LuaTeX is required for setting this manual section.}\endinput\fi

As mentioned before (\ref{section-library-graphdrawing}), the graph
drawing library makes use of Lua. But where does the control flow
leave \TeX\ and what happens to your \tikzname\ nodes? The subsequent
sections will discuss this process in deep. The general approach is to
intercept the immediate placement of the nodes and pass them down to
Lua, which does all the placement stuff. After the selected graph
drawing algorithm has finished, it writes the nodes back to
\tikzname\ to have the graph drawn.

This proceeding consists of a front end layer for \tikzname, an
interface to Lua and of course a set of Lua classes to represent the
graph. An algorithms can be developed independently. Only knowledge
about the Lua interface is required; specific \TeX\ programming skills
not necessary.

\subsubsection{The Front End Layer}
Let's have a look at a simple example to see what the front end looks
like:

\begin{codeexample}[]
\tikzpicture[graph drawing={standard tree},scale=2]
  \graph{root [as=Hello,root] -> World[fill=blue!20]};
\endtikzpicture
\end{codeexample}

As you may see, the syntax is exactly the same as described in the
chapter about specifying graphs (section~\ref{section-library-graphs}).

You enable this library with the key |graph drawing|, which sets the
algorithm to use and its specific parameters. All other
\tikzname\ keys are accepted as well, like |scale| in the example
above. Each algorithm has its own keys to parametrize it. Please refer to
the appropriate sections for more information.

The keys are given within the |graph drawing| key family for graph options and per node for node specific options. Furthermore you can use any valid \tikzname\ keys as usual. 

There are some things which will not work with the graph drawing
library, like preordering the nodes. Consider for example the
|chain shift| key of the graphs library to place the nodes on a
certain grid: 

\begin{codeexample}[]
\tikzpicture
  \graph[chain shift=(45:1)] {
    a -> b -> c;
    d -> e;
  };
\endtikzpicture
\end{codeexample}

The graph drawing library does not take care of any predefined layout options by now, so the above example will be set differently:

\begin{codeexample}[]
\tikzpicture[graph drawing={few intersections}, scale=2]
  \graph[chain shift=(45:1)] {
    a -> b -> c;
    d -> e;
  };
\endtikzpicture
\end{codeexample}

A graph drawing algorithm will always place the nodes in its own manner. 

% what is happening in the tikz..tex file. Matthias

\subsubsection{The Interface to Lua}
The main entry point for the library to Lua is defined in the
appropriate |code| file of the library. It employs three Lua classes
to create graphs, pass down nodes and to communicate the given
options.

An overview of what happens is illustrated by the following call graph:

\begin{tikzpicture}[
    class name/.style={draw,minimum size=20pt, fill=blue!20},
    object node/.style={draw,minimum size=15pt, fill=yellow!20},
    p/.style={->,>=stealth},
    livespan/.style={thick,double},
    scale=0.9]
  % class names above
  \node (tikz) at (0,4) [class name] {\tikzname\ graph};
  \node (tex) at (5,4) [class name] {\TeX\ Interface};
  \node (interface) at (10,4) [class name] {Lua Interface};
  \node (sys) at (15,4) [class name] {Sys};
  % lines from the class names to the bottom of the picture
  \draw[livespan] (tikz) -- (0,-6.5);
  \draw[livespan] (tex) -- (5,-6.5);
  \draw[livespan] (interface) -- (10,-6.5);
  \draw[livespan] (sys) -- (15,-6.5);
  % first command: \graph{  -- generates new graph in lua interface
  \node (tikz-begin-graph) at (0,3) [object node] {|\graph{|}; %}
  \node (tex-begin-graph) at (5,3) [object node] {|\pgfgdbeginscope|};  
  \node (interface-new-graph) at (10,3) [object node] {|newGraph(|...|)|};
  \draw [p] (tikz-begin-graph.east) -- (tex-begin-graph.west);
  \draw [p] (tex-begin-graph.east) -- (interface-new-graph.west);    
  % second command: a -> b   -- generates two nodes in lua
  % and one edge
  \node (tikz-node) at (0,2) [object node] {|a -> b;|};
  \node (tex-node) at (5,2) [object node] {|\pgf@gd@positionnode@callback|};
  \node (interface-add-node-behind) at (10.1,1.9) [object node] {|addNode(|...|)|};
  \draw[p] (tikz-node.east) -- (tex-node.west);
  
  \node (interface-add-node) at (10,2) [object node] {|addNode(|...|)|};
  \draw[p] (tex-node.east) -- (interface-add-node.west);

  \node (tex-add-edge) at (5,1) [object node] {|\pgfgdaddedge|};
  \node (interface-add-edge) at (10,1) [object node] {|addEdge(|...|)|};
  \draw[p] (tikz-node.east) -- (1.5,2) -- (1.5,1) -- (tex-add-edge.west);
  \draw[p] (tex-add-edge.east) -- (interface-add-edge.west);

  % scope ends -- cloes graph, layouts it and draws it
  \node (tikz-end) at (0,0) [object node] {|};|};
  \node (tex-end) at (5,0) [object node] {|\pgfgdendscope|};
  \node (interface-draw-graph) at (10,0) [object node] {|drawGraph()|};
  \node (interface-finish-graph) at (10,-2) [object node] {|finishGraph()|};

  \node (invoke-algorithm) at (12.5,-1) [object node] {invoke algorithm};
  \draw[p] (tikz-end.east) -- (tex-end.west);
  \draw[p] (tex-end.east) -- (interface-draw-graph.west);
  \draw[p] (interface-draw-graph.east) -- (12.5,0) -- (invoke-algorithm.north);
  \draw[p] (tex-end.east) -- (7.5,0) -- (7.5,-2) -- (interface-finish-graph.west);

  % begin shipout
  \node (sys-begin-shipout) at (15,-2) [object node] {|beginShipout()|};
  \draw[p] (interface-finish-graph.east) -- (sys-begin-shipout.west);
  \node (tex-begin-shipout) at (5,-3) [object node] {|\pgfgdbeginshipout|};
  \draw[p] (sys-begin-shipout.187) -- (12,-2.2) -- (12,-3) -- (tex-begin-shipout.east);

  % put tex box
  \node (sys-puttexbox-behind) at (15.1,-4.1) [object node] {|putTeXBox(|...|)|};
  \node (sys-puttexbox) at (15,-4) [object node] {|putTeXBox(|...|)|};
  \node (tex-puttexbox) at (5,-4) [object node] {|\pgfgdinternalshipoutnode|};

  \draw[p] (12.5,-2) -- (12.5,-4) -- (sys-puttexbox.west);
  %(interface-finish-graph.east) -- (12.5,-2) -- (12.5,-4) -- (sys-puttexbox.west);
  \draw[p] (sys-puttexbox.187) -- (12,-4.2) -- (12,-4) -- (tex-puttexbox.east);

  % put edge
  \node (sys-put-edge-behind) at (15.1,-5.1) [object node] {|putEdge(|...|)|};
  \node (sys-put-edge) at (15,-5) [object node] {|putEdge(|...|)|};
  \draw[p] (12.5,-4) -- (12.5,-5) -- (sys-put-edge.west);
  %(interface-finish-graph.east) -- (12.5,-2) -- (12.5,-5) -- (sys-put-edge.west);
  % end shipout
  \node (sys-end-shipout) at (15,-6) [object node] {|endShipout()|};
  \draw[p] (12.5,-5) -- (12.5,-6) -- (sys-end-shipout.west);
  %(interface-finish-graph.east) -- (12.5,-2) -- (12.5,-6) -- (sys-end-shipout.west);
  \node (tex-end-shipout) at (5,-6) [object node] {|\pgfgdendshipout|};
  \draw[p] (sys-end-shipout.187) -- (12,-6.175) -- (12,-6) -- (tex-end-shipout.east);
\end{tikzpicture}


\paragraph{The \TeX\ side.}
\label{section-library-graphdrawing-the-tex-side}

In order to layout a graph, we need to keep \tikzname\ from placing the nodes immediately. This is done using the macro
|\pgfpositionnodelater| as described in chapter~\ref{section-shapes},
subchapter~\ref{section-shapes-deferred-node-positioning}. 

In short terms this works as follows: This macro takes another \meta{macro} as
first argument. If this is |\relax|, the behaviour is to immediately
place the node into the current picture. Any other \meta{macro} that is passed
will be executed. It works like a
callback function -- the node will be put into a box register, the
name of the node and the bounding box coordinates are stored in
separate macros and afterwards \meta{macro} will be called.

As we have to make sure, that the unplaced node will not be referenced
by \tikzname\ keys like |right of|, it is temporarily renamed to
|not yet positionedPGFGDINTERNAL|\meta{nodename}.

To finally
insert the node into the picture, we need to set the mentioned macros
and put the node into the box register. Then we can call
|\pgfpositionnodenow| with the target coordinates of the node.

The code file of the graph drawing library sets the callback function
at the beginning of a graph drawing scope, e.g.\ when a |\graph|
starts. This can also be triggered using |\pgfgdbeginscope| and
|\pgfgdendscope|, which can be used to create a sub scope in an
existing graph drawing scope. Opening a scope yields in creating a new
graph on the Lua graph stack. All subsequent operations (like adding
nodes or edges) apply to the top of the stack. 

%by now this leads to an infinite loop . when its fixed, the example
%can be uncommented :)
% \begin{codeexample}[]
% \tikzpicture[graph drawing={few intersections}, scale=2]
% \graph{
%   a->b;
% %  \graph{c->d;}; TODO: triggers an infinite loop.
%   };
% \endtikzpicture
% \end{codeexample}

The callback function gets all option keys in
|/tikz/graphs/graph drawing/|, copies the box register and passes all information down to the Lua interface class.

When the library is loaded, it initialises the Lua subsystem. This takes place by checking if \LuaTeX\ is present and then invoking the Lua loader class. 

The library code file consists mainly the following macros:

\begin{command}{\pgfgdbeginscope}
  The begin scope macro opens a new graph drawing scope. This creates a new graph object on the top of the Lua graph stack. All subsequent operations will work on this graph until |\pgfgdendscope| will be called.

It is not necessary to call it manually, because in a graph drawing environment it is executed by default at the beginning of a |\graph| statement.
\end{command}


\makeatletter
\begin{command}{\pgf@gd@positionnode@callback}
  This macro saves the keys from |/tikz/graphs/graph drawing/| into a temporary macro, sets the box register |\pgf@gd@box| to the |\pgfpositionnodelaterbox| and passes these informations down to Lua. Additionally the node name and the bounding box is passed down, too. This macro is only used internally.
\end{command}
\makeatother

\begin{command}{\pgfgdaddedge\marg{from}\marg{to}\marg{direction}}
  Adds an edge to the Lua graph object. It requires the name of the target node \meta{from}, the destination node \meta{to} with a distinct \meta{direction} like |->|.

  It is called when a |->|, |--|, |<-| or |-!-| is encountered in a graph.
\end{command}

\begin{command}{\pgfgdendscope}
  At the end of a graph drawing scope the selected algorithm runs and layouts the graph. After finishing this task the macro pops the graph from the stack.
\end{command}

\begin{command}{\pgfgdbeginshipout}
  When the layout is completed and the scope ended, this macro places a |\scope| into the output stream. The layouted graph will be placed inside an extra scope.
\end{command}

\begin{command}{\pgfgdinternalshipoutnode\marg{name}\marg{x min}\marg{x max}\marg{y min}\marg{y max}\marg{x pos}\marg{y pos}\marg{box}}
  When the algorithm finished the layout and the scope ended, the nodes have to be passed back to \tikzname. This macro takes the name of the node, the bounding box, the newly computed position and a box register number. It restores the macros set by |\pgfpositionnodelater| as mentioned above, fills the box register |\pgfpositionnodelaterbox| and then calls |\pgfpositionnodenow| with the coordinates of the node. This macro inserts the node into the current picture.
\end{command}

\begin{command}{\pgfgdendshipout}
  Issues a |\endscope| macro to close the scope opened by |\pgfgdbeginshipout|.
\end{command}

\paragraph{Lua interface class.}

The class |Interface| is the main entry point in Lua. Every communication from \TeX\ to Lua is done here.
It provides methods to create graphs, add nodes and edges to graphs and finally to invoke the selected algorithm. The |Interface| class manages the stack of graphs.

When the |newGraph()| function is called, it generates a new graph object and pushes it on the graph stack. The methods |addNode()| and |addEdge()| are called for each node and each edge, creating the actual Lua objects and adding them to the current graph.

After adding nodes and edges, when the scope ends, the interface invokes the actual algorithm to layout the graph. This is done in the |drawGraph()| function. The next step is to put the nodes back in the \TeX\ output stream. This is invoked by the |finishGraph()| method.

For a reference about the functions and their usage, please refer to section~\ref{section-library-graphdrawing-lua-documentation-interface}.

\paragraph{Lua system class.}

Communication with \TeX\ on a basic layer is done in the |Sys| class. The |beginShipout()| function opens a new scope in \tikzname\ to put all graph drawing nodes into. This prevents other graph objects outside the graph drawing scope from referencing these nodes. The |endShipout()| method closes the scope.

Nodes and edges are put in the output stream by the methods |putTeXBox()| and |putEdge()|. The first calls the |\pgfgdinternalshipoutnode| macro, which is explained in section~\ref{section-library-graphdrawing-the-tex-side}. The latter method writes the appropriate |\draw| directly to the output stream. 

For a reference about the functions and their usage, please refer to section~\ref{section-library-graphdrawing-lua-documentation-sys}.

\subsubsection{Lua Graph Representation}
Most classes in the framework (including the module objects) implement
the |__tostring| method, meaning that you can get a somewhat useful
string representation of the object via the standard |tostring|
function.

The main class which contains references to all other objects is
|Graph|.  New graphs are usually created automatically, so common ways
to get new graph objects are the |copy| method, which creates a
shallow copy (without coying nodes or edges), and the
|subGraphParent| method, which creates a deep copy of the graph, edge
and node objects starting at a designated parent node. If you need
more control by supplying your own set of already visited nodes, use
the underlying function |subGraph|.

A graph allows you to add and remove nodes and edges via |addNode|,
|addEdge|, |removeNode| and |removeEdge| respectively.  There are also
variants which remove all incident edges on a node removal and
conversely, |deleteNode| and |deleteEdge|.

Only nodes can be looked up by name with |findNode|, a
method implemented using the more generic |findNodeIf|, which supports
an arbitrary test predicate.

Lastly the |walkDepth| and |walkBreadth| methods may be used to get
iterators over all nodes and edges in a depth-first or breadth-first
order (other traversal orders may require a rewrite or extension of the
|walkAux| method).

Positions are represented using the dedicated class |Position|, the member
variables |x| and |y| contain the coordinates.  Positions can also be
relative to other positions, which can be tested using |isAbsPosition|.
The conversion to absolute coordinates is done with |getAbsCoordinates|.
The equality test method implements comparing two positions by using their
absolute positions.

For a detailed description of the mentioned classes and methods refer
to section~\ref{section-library-graphdrawing-lua-documentation-graphrep}.

\paragraph{Common graph operations.}
The following tasks are typical for manipulating the graph.
Those snippets will get you started even if you do not have any Lua
experience.

\begin{itemize}
\item Iterate over all nodes.
\begin{codeexample}[code only]
for node in values(graph.nodes) do
   ...
end
\end{codeexample}
\item Get or set width/height of a node, e.g.\ for measuring.
\begin{codeexample}[code only]
local width, height = node.width, node.height
\end{codeexample}
\item Get or set x-y-coordinates of a node.
\begin{codeexample}[code only]
node.pos.x = node.pos.x + 1
node.pos.y = node.pos.y + 1
\end{codeexample}
\item Relate the position of node to the position of another.
\begin{codeexample}[code only]
newNode.pos.x, newNode.pos.y = 1, 1
--sets position of newNode 1 pt in y- and x-direction relative to node
newNode.pos:relateTo(node.pos)
\end{codeexample}
\item Get absolute x-y-coordinate of node, with or without relative coordinates.
\begin{codeexample}[code only]
absX, absY = newNode:getAbsCoordinates()
\end{codeexample}
\item Iterate over all edges and all nodes of the current edge.
\begin{codeexample}[code only]
for edge in values(graph.edges) do
   for node in values(edge:getNodes()) do
      ...
   end
end
\end{codeexample}
\item Get the nodes connected by an edge.
\begin{codeexample}[code only]
local nodeLeft = edge:getNodes()[1]
local nodeRight = edge:getNodes()[2]
\end{codeexample}
\end{itemize}

A full example for a user-defined algorithm is shown in
section~\ref{section-library-graphdrawing-ownAlgorithm}.

\subsection{Registering graph drawing keys}
\label{section-base-graphdrawing-registerKeys}

Graphs and nodes in Lua have specific options, like the name of the
algorithm to use. These keys are registered on the \tikzname\ layer.


\begin{stylekey}{/tikz/graphs/graph drawing/register key}
  The argument of this style is registered as a new key for a
  graph. The name of the key and it's value will be passed down to the
  Lua graph object and should be used for algorithm-wide options. 

  An example is the |algorithm| key, which is required for each graph
  drawing context. 

  The key/value pair will be stored in |/tikz/graphs/graph drawing/@options/|.
\end{stylekey}

\begin{stylekey}{/tikz/graphs/graph drawing/register math key}
  Registering a new math key is like registering a new key, except
  that it's a parseable value. When a value is assigned to the key,
  pgf will parse the value. 

  Math keys can be used if a option holds a dimension value, like the
  |scale| option of \tikzname\. The value will be expanded and
  computed to the dimension |pt|. 

  A sample math key is introduced in the simpleexample algorithm
  (see \ref{section-library-graphdrawing-ownAlgorithm}) below.
\end{stylekey}

\begin{stylekey}{/tikz/graphs/graph drawing/register node key}
  A node key is not stored graph-wide; it is designated for a single
  node. The name/value pair is accessible from the node object in Lua;
  in \tikzname\ it will be stored in the key family |/tikz/graphs/graph drawing/@node@options/|.
\end{stylekey}

\begin{stylekey}{/tikz/graphs/graph drawing/register node math key}
  Like node key, but with parsing of it's value (see |register math key|).
\end{stylekey}

\subsection{Creating your own Algorithm}
\label{section-library-graphdrawing-ownAlgorithm}
There are two ways to make a user-definded algorithm
available to the graph drawing library.
You can create your own graph drawing algorithm by naming it like
|drawGraphAlgorithm_xyz| and placing it into the |pgf.graphdrawing|
Lua module, where |xyz| is the string which is supplied to the
\TeX\ interface.  This way the function is looked up before the
framework tries to load a file named
|pgflibrarygraphdrawing-algorithms-xyz.lua| anywhere in the accessible
path, which is the second way to define your algorithm.

You may load a file named according to the above-mentioned scheme that
contains an algorithm on your own using the |Interface:loadAlgorithm()|
function, which accepts the name string as single argument. This will
usually modify the module entry of the function name, so you have to
be aware of that behaviour if you rely on it to test whether an
algorithm was loaded (e.g. if you want to define a wrapper around the
loaded algorithm).

The algorithm will be called with the graph object as single argument
and should do its work by modifying this object. Any return
values are discarded.

For example, the following code fragment (taken and slightly altered
from the file |pgflibrarygraphdrawing-algorithms-simpleexample.lua|)
implements a rather simple algorithm, placing all nodes on a fixed-size
circle.  It is accessed with the name |simpleexample|, so both the
file- and function name agree on that.

\begin{codeexample}[code only]
pgf.module("pgf.graphdrawing")

--- A very, very simple node placing algorithm for demonstration purposes.
-- All nodes are positioned on a fixed-size circle.
function drawGraphAlgorithm_simpleexample(graph)
   local radius = 20
   local nodeCount = 0

   -- count nodes
   for _ in values(graph.nodes) do
      nodeCount = nodeCount + 1
   end

   local alpha = (2 * math.pi) / nodeCount
   local i = 0
   for node in values(graph.nodes) do
      -- the interesting part...
      node.pos.x = radius * math.cos(i * alpha)
      node.pos.y = radius * math.sin(i * alpha)
      i = i + 1
   end
end
\end{codeexample}

It is important not to use a |local| declaration before the function
header, because it wouldn't be available in the |pgf.graphdrawing|
module anymore.

The algorithm computes a circular layout like in the following.

\begin{codeexample}[]
\tikzpicture [graphs/.cd, graph drawing engine, algorithm=simpleexample]
  \graph { f -> c -> e -> a ->{b -> {c, d, f}, e -> b}};
\endtikzpicture
\end{codeexample}

The invocation above also shows how to use an algorithm which is not
registered as a \tikzname\ key.  In general, you will probably want to
register your algorithm with |\tikzgraphsset| to make your code more
succinct, but also to be able to change algorithm options by manipulating
\tikzname\ keys, which is not possible without registration.  

To do so, we have to modify the first line of the example algorithm.

\begin{codeexample}[code only]
   local radius = graph:getOption("radius") or 20
\end{codeexample}

Using the |getOption| method we obtain the value of the
\tikzname\ option or a |nil| value, therefore there has to be a
default value for any option or more elaborated error handling.  The 
following code block can be used to register this algorithm and
its single option.

\begin{codeexample}[code only]
\tikzgraphsset{
  simpleexample/.style={
    graph drawing engine,
    algorithm=simpleexample
  },
  graph drawing/register math key=radius
}
\end{codeexample}

\tikzgraphsset{
  simpleexample/.style={
    graph drawing engine,
    algorithm=simpleexample
  },
  graph drawing/register math key=radius
}

Eventually this fragment will have to be entered into the
|tikzlibrarygraphdrawing.code.tex| file if it is to be included in the
\pgfname\ source code.

Once registered, specifying the algorithm gets a bit easier. Note the
increased radius compared to the previous example.

\begin{codeexample}[]
\tikzpicture [graph drawing={simpleexample, radius = 30}]
  \graph { f -> c -> e -> a ->{b -> {c, d, f}, e -> b}};
\endtikzpicture
\end{codeexample}

\subsection{Module System}
The package defines its own Lua module system, which is characterised by a
more dynamic view on importing symbols.  Basically, each module has a
set of imported modules and the lookup for names first happens in the local
scope, then in the current module and subsequently in all imported
modules.  Since no name is statically imported, newly assigned
variables in other modules are still visible when those were
previously imported.

Modules are accessed with the |pgf.module| call, which enables the
module for the current context, i.e. the current file. If a module
does not exist, it will be created.  Importing modules is done via
|pgf.import|.  Both functions accept a string argument for the
module name.

Modules are named hierarchically and defined modules are exported into
each parent module.  If the module name contains no period, it is
exported into the global environment.  Nevertheless, importing is only
done on request; importing a module twice doesn't do anything.
It is recommended to dedicate a single module definition file
to create it and import other modules.  For example, the package
contains a single file containing only the following two lines for
creating the |pgf.graphdrawing| module in the first place.

\begin{codeexample}[code only]
pgf.module("pgf.graphdrawing")
pgf.import("pgf")
\end{codeexample}

Symbol lookup first happens in the local namespace, then in the
current module and subsequently in all imported modules and the global
namespace.  Assignment of new variables happens in the current module
(or for variables declared |local| in the local namespace).  If you
need to assign values to the global environment use the special table
|_G| as you'd normally do in Lua.

The |pgf| module is created during the definition of the module system
and mostly contains functions for loading and debugging.  Developers
probably shouldn't touch the |pgf| namespace and instead add new
functionality to modules below this level or in new top-level
modules.

\subsubsection{Module Examples}
Let's see what consequences this module system has in praxis.  The
following code fragment starts from a clean state after rendering it
with \LuaTeX\ and then enters the |pgf.graphdrawing| module,
overwriting the global |pgf| binding and then again reverting this
change.

\begin{codeexample}[code only]
  \input tikz

  \usetikzlibrary{graphdrawing}

  \directlua{
    pgf.graphdrawing.Sys:log("1: pgf is " .. tostring(pgf))
    pgf.graphdrawing.Sys:log("1: graphdrawing is " .. tostring(graphdrawing))
    
    pgf.module("pgf.graphdrawing")

    Sys:log("2: pgf is " .. tostring(pgf))
    Sys:log("2: graphdrawing is " .. tostring(graphdrawing))

    pgf = 1

    Sys:log("3: pgf is " .. tostring(pgf))
    Sys:log("3: graphdrawing is " .. tostring(graphdrawing))

    pgf = nil

    Sys:log("4: pgf is " .. tostring(pgf))

    pgf.graphdrawing = nil

    Sys:log("5: pgf is " .. tostring(pgf))

    _G.pgf = nil

    Sys:log("6: pgf is " .. tostring(pgf))
  }
\end{codeexample}

The result will be as follows:

\begin{codeexample}[code only]
1: pgf is <module 'pgf', table: 0x7979600>
1: graphdrawing is nil

2: pgf is <module 'pgf', table: 0x7979600>
2: graphdrawing is <module 'pgf.graphdrawing', table: 0x7973c60>

3: pgf is 1
3: graphdrawing is <module 'pgf.graphdrawing', table: 0x7973c60>

4: pgf is <module 'pgf', table: 0x7979600>
5: pgf is <module 'pgf', table: 0x7979600>
6: pgf is nil
\end{codeexample}

As you can see the |pgf| table is available in the global environment
and also after using the |pgf.graphdrawing| module, although we don't
refer to it with its full name.  Assigning a new value to |pgf|
doesn't overwrite the global object, but introduces a local binding
shadowing the global one. Assigning |nil| then removes the local
binding, therefore in the next line the global variable is available
again.

Note that in all but the first case the binding to |graphdrawing|
stays the same.  Also, using these assignments, you can't accidentally
remove your access to the |pgf| or any imported modules as the last
two assignments show (the |Sys:log| method still works).

\subsection{Lua Documentation}
This sections provides a full documentation of all relevant Lua classes
used.

Every class and function in the package (except for module handling in
|pgf|) is available in the |pgf.graphdrawing| module.

\label{section-library-graphdrawing-lua-documentation}
\subsubsection{Graph Representation}
\label{section-library-graphdrawing-lua-documentation-graphrep}
% This file has been generated from the lua sources using LuaDoc.
% To regenerate it call "make genluadoc" in
% doc/generic/pgf/version-for-luatex/en.

\begin{filedescription}{pgflibrarygraphdrawing-graph.lua}


\begin{luacommand}{{Graph:\textunderscore{}\textunderscore{}tostring}()}
Returns a string representation of this graph including all nodes and edges. 


Return value:
\begin{itemize} \item[] Graph as string.  \end{itemize}


\end{luacommand}\begin{luacommand}{{Graph:addEdge}(\meta{edge})}
Adds an edge to the graph. 

Parameters:
\begin{parameterdescription}
	\item[\meta{edge}] The edge to be added. 
\end{parameterdescription}



\end{luacommand}\begin{luacommand}{{Graph:addNode}(\meta{node})}
Adds a node to the graph. 

Parameters:
\begin{parameterdescription}
	\item[\meta{node}] The node to be added. 
\end{parameterdescription}



\end{luacommand}\begin{luacommand}{{Graph:copy}()}
Creates a shallow copy of a graph.  The nodes and edges of the original graph are not preserved in the copy. 


Return value:
\begin{itemize} \item[] A shallow copy of the graph.  \end{itemize}


\end{luacommand}\begin{luacommand}{{Graph:createEdge}(\meta{nodeA},\meta{nodeB},\meta{direction},\meta{edgenodes},\meta{options},\meta{tikzoptions})}
Creates and adds a new edge to the graph. 

Parameters:
\begin{parameterdescription}
	\item[\meta{nodeA}] The first node of the new edge.\item[\meta{nodeB}] The second node of the new edge.\item[\meta{direction}] The direction of the new edge. Possible values are |Edge.UNDIRECTED|, |Edge.LEFT|, |Edge.RIGHT|, |Edge.BOTH| and |Edge.NONE| (for invisible edges).\item[\meta{edgenodes}] A string of \tikzname\ edge nodes that needs to be passed back to the \TeX layer unmodified.\item[\meta{options}] The options of the new edge.\item[\meta{tikzoptions}] A table of \tikzname\ options to be used by graph drawing algorithms to treat the edge in special ways. 
\end{parameterdescription}


Return value:
\begin{itemize} \item[] The newly created edge.  \end{itemize}


\end{luacommand}\begin{luacommand}{{Graph:deleteEdge}(\meta{edge})}
Like removeEdge, but also removes the edge from its adjacent nodes. 

Parameters:
\begin{parameterdescription}
	\item[\meta{edge}] The edge to be deleted. 
\end{parameterdescription}


Return value:
\begin{itemize} \item[] The removed edge or |nil| if it was not found in the graph.  \end{itemize}


\end{luacommand}\begin{luacommand}{{Graph:deleteNode}(\meta{node})}
Like removeNode, but also deletes all adjacent edges of the removed node.  This function also removes the deleted adjacent edges from all neighbours of the removed node. 

Parameters:
\begin{parameterdescription}
	\item[\meta{node}] The node to be deleted together with its adjacent edges. 
\end{parameterdescription}


Return value:
\begin{itemize} \item[] The removed node or |nil| if the node was not found in the graph.  \end{itemize}


\end{luacommand}\begin{luacommand}{{Graph:findNode}(\meta{name})}
If possible, looks up the node with the given name in the graph. 

Parameters:
\begin{parameterdescription}
	\item[\meta{name}] Name of the node to look up. 
\end{parameterdescription}


Return value:
\begin{itemize} \item[] The node with the given name or |nil| if it was not found in the graph.  \end{itemize}


\end{luacommand}\begin{luacommand}{{Graph:findNodeIf}(\meta{test})}
Looks up the first node for which the function \meta{test} returns |true|. 

Parameters:
\begin{parameterdescription}
	\item[\meta{test}] A function that takes one parameter (a |Node|) and returns |true| or |false|. 
\end{parameterdescription}


Return value:
\begin{itemize} \item[] The first node for which \meta{test} returns |true|.  \end{itemize}


\end{luacommand}\begin{luacommand}{{Graph:getOption}(\meta{name})}
Returns the value of the graph option \meta{name}. 

Parameters:
\begin{parameterdescription}
	\item[\meta{name}] Name of the option. 
\end{parameterdescription}


Return value:
\begin{itemize} \item[] The value of the graph option \meta{name} or |nil|.  \end{itemize}


\end{luacommand}\begin{luacommand}{{Graph:mergeOptions}(\meta{options})}
Merges the given options into the options of the graph. 

Parameters:
\begin{parameterdescription}
	\item[\meta{options}] The options to be merged. 
\end{parameterdescription}



See also:
\begin{itemize}
	\item[] |mergeTable |
\end{itemize}

\end{luacommand}\begin{luacommand}{{Graph:new}(\meta{values})}
Creates a new graph. 

Parameters:
\begin{parameterdescription}
	\item[\meta{values}] Values to override default graph settings. The following parameters can be set:\par |nodes|: The nodes of the graph.\par |edges|: The edges of the graph.\par |pos|: Initial position of the graph.\par |options|: A table of node options passed over from \tikzname. 
\end{parameterdescription}


Return value:
\begin{itemize} \item[] A newly-allocated graph.  \end{itemize}


\end{luacommand}\begin{luacommand}{{Graph:removeEdge}(\meta{edge})}
If possible, removes an edge from the graph and returns it. 

Parameters:
\begin{parameterdescription}
	\item[\meta{edge}] The edge to be removed. 
\end{parameterdescription}


Return value:
\begin{itemize} \item[] The removed edge or |nil| if it was not found in the graph.  \end{itemize}


\end{luacommand}\begin{luacommand}{{Graph:removeNode}(\meta{node})}
If possible, removes a node from the graph and returns it. 

Parameters:
\begin{parameterdescription}
	\item[\meta{node}] The node to remove. 
\end{parameterdescription}


Return value:
\begin{itemize} \item[] The removed node or |nil| if it was not found in the graph.  \end{itemize}


\end{luacommand}\begin{luacommand}{{Graph:setOption}(\meta{name},\meta{value})}
Sets the graph option \meta{name} to \meta{value}. 

Parameters:
\begin{parameterdescription}
	\item[\meta{name}] Name of the option to be changed.\item[\meta{value}] New value for the graph option \meta{name}. 
\end{parameterdescription}



\end{luacommand}\begin{luacommand}{{Graph:subGraph}(\meta{root},\meta{graph},\meta{visited})}
Returns a subgraph.  The resulting subgraph begins at the node root, excludes all nodes and edges that are marked as visited. 

Parameters:
\begin{parameterdescription}
	\item[\meta{root}] Root node where the operation starts.\item[\meta{graph}] Result graph object or |nil| if the original graph should be used as the parent graph.\item[\meta{visited}] Set of already visited nodes/edges or |nil|. This set will be modified so make sure not to use a table that you want to remain untouched. 
\end{parameterdescription}



\end{luacommand}\begin{luacommand}{{Graph:subGraphParent}(\meta{root},\meta{parent},\meta{graph})}
Creates a new subgraph with \meta{parent} marked as visited.  This function can be useful if the graph is a tree structure (and \meta{parent} is the parent node of \meta{root}). 

Parameters:
\begin{parameterdescription}
	\item[\meta{root}] Root node where the operation starts.\item[\meta{parent}] Parent of the recursion step before.\item[\meta{graph}] Result graph object or |nil| if the original graph should be used as the parent graph. 
\end{parameterdescription}



See also:
\begin{itemize}
	\item[] |subGraph |
\end{itemize}

\end{luacommand}\begin{luacommand}{{Graph:walkAux}(\meta{root},\meta{visited},\meta{removeIndex})}
Auxiliary function to walk a graph. Does nothing if no nodes exist. 

Parameters:
\begin{parameterdescription}
	\item[\meta{root}] The first node to be visited.  If nil, chooses some node.\item[\meta{visited}] Set of already visited nodes and edges. |visited[v] == true| indicates that the node or edge |v| has already been visited.\item[\meta{removeIndex}] A numeric value or |nil| that defines the order in which nodes and edges are visited while traversing the graph. |nil| results in queue behavior, |1| in stack behavior. 
\end{parameterdescription}



See also:
\begin{itemize}
	\item[] |walkDepth|\item[] |walkBreadth |
\end{itemize}

\end{luacommand}\begin{luacommand}{{Graph:walkBreadth}(\meta{root},\meta{visited})}
Returns an iterator to walk the graph in a breadth-first traversal.  The iterator returns all edges and nodes one at a time. In case only the nodes are of interest, a filter function like |iter.filter| can be used to ignore edges. 

Parameters:
\begin{parameterdescription}
	\item[\meta{root}] The first node to be visited.  If nil, chooses some node.\item[\meta{visited}] Set of already visited nodes and edges. |visited[v] == true| indicates that the node or edge |v| has already been visited. 
\end{parameterdescription}



See also:
\begin{itemize}
	\item[] |iter.filter |
\end{itemize}

\end{luacommand}\begin{luacommand}{{Graph:walkDepth}(\meta{root},\meta{visited})}
Returns an iterator to walk the graph in a depth-first traversal.  The iterator returns all edges and nodes one at a time. In case only the nodes are of interest, a filter function like |iter.filter| can be used to ignore edges. 

Parameters:
\begin{parameterdescription}
	\item[\meta{root}] The first node to be visited.  If nil, chooses some node.\item[\meta{visited}] Set of already visited nodes and edges. |visited[v] == true| indicates that the node or edge |v| has already been visited. 
\end{parameterdescription}



See also:
\begin{itemize}
	\item[] |iter.filter |
\end{itemize}

\end{luacommand}
\end{filedescription}
% This file has been generated from the lua sources using LuaDoc.
% To regenerate it call "make genluadoc" in
% doc/generic/pgf/version-for-luatex/en.

\begin{filedescription}{pgflibrarygraphdrawing-node.lua}


\begin{luacommand}{{Node:\textunderscore{}\textunderscore{}eq}(\meta{object},\meta{other})}
Compares two nodes by their name. 

Parameters:
\begin{parameterdescription}
	\item[\meta{other}] Another node to compare with. 
\end{parameterdescription}


Return value:
\begin{itemize} \item[] |true| if both nodes have the same name. |false| otherwise.  \end{itemize}


\end{luacommand}\begin{luacommand}{{Node:\textunderscore{}\textunderscore{}tostring}()}
Returns a formated string representation of the node. 


Return value:
\begin{itemize} \item[] String represenation of the node.  \end{itemize}


\end{luacommand}\begin{luacommand}{{Node:addEdge}(\meta{edge})}
Adds new edge to the node. 

Parameters:
\begin{parameterdescription}
	\item[\meta{edge}] The edge to be added. 
\end{parameterdescription}



\end{luacommand}\begin{luacommand}{{Node:copy}()}
Creates a shallow copy of the node.  Most notably, the edges adjacent are not preserved in the copy. 


Return value:
\begin{itemize} \item[] Copy of the node.  \end{itemize}


\end{luacommand}\begin{luacommand}{{Node:getDegree}()}
Counts the adjacent edges of the node. 


Return value:
\begin{itemize} \item[] The number of adjacent edges of the node.  \end{itemize}


\end{luacommand}\begin{luacommand}{{Node:getEdges}()}
Returns all edges of the node.  Instead of calling |node:getEdges()| the edges can alternatively be accessed directly with |node.edges|. 


Return value:
\begin{itemize} \item[] All edges of the node.  \end{itemize}


\end{luacommand}\begin{luacommand}{{Node:getInDegree}(\meta{ignorereversed})}
Returns the number of incoming edges of the node. 

Parameters:
\begin{parameterdescription}
	\item[\meta{ignorereversed}] Optional parameter to consider reversed edges not reversed for this method call. Defaults to |false|. 
\end{parameterdescription}


Return value:
\begin{itemize} \item[] The number of incoming edges of the node.  \end{itemize}


See also:
\begin{itemize}
	\item[] |Node:getIncomingEdges(reversed) |
\end{itemize}

\end{luacommand}\begin{luacommand}{{Node:getIncomingEdges}(\meta{ignorereversed})}
Returns the incoming edges of the node. Undefined result for hyperedges. 

Parameters:
\begin{parameterdescription}
	\item[\meta{ignorereversed}] Optional parameter to consider reversed edges not reversed for this method call. Defaults to |false|. 
\end{parameterdescription}


Return value:
\begin{itemize} \item[] Incoming edges of the node. This includes undirected edges and directed edges pointing to the node.  \end{itemize}


\end{luacommand}\begin{luacommand}{{Node:getOption}(\meta{name})}
Returns the value of the node option \meta{name}. 

Parameters:
\begin{parameterdescription}
	\item[\meta{name}] Name of the node option. 
\end{parameterdescription}


Return value:
\begin{itemize} \item[] The value of the node option \meta{name} or |nil|.  \end{itemize}


\end{luacommand}\begin{luacommand}{{Node:getOutDegree}(\meta{ignorereversed})}
Returns the number of edges starting at the node. 

Parameters:
\begin{parameterdescription}
	\item[\meta{ignorereversed}] Optional parameter to consider reversed edges not reversed for this method call. Defaults to |false|. 
\end{parameterdescription}


Return value:
\begin{itemize} \item[] The number of outgoing edges of the node.  \end{itemize}


See also:
\begin{itemize}
	\item[] |Node:getOutgoingEdges() |
\end{itemize}

\end{luacommand}\begin{luacommand}{{Node:getOutgoingEdges}(\meta{ignorereversed})}
Returns the outgoing edges of the node. Undefined result for hyperedges. 

Parameters:
\begin{parameterdescription}
	\item[\meta{ignorereversed}] Optional parameter to consider reversed edges not reversed for this method call. Defaults to |false|. 
\end{parameterdescription}


Return value:
\begin{itemize} \item[] Outgoing edges of the node. This includes undirected edges and directed edges leaving the node.  \end{itemize}


\end{luacommand}\begin{luacommand}{{Node:getTexHeight}()}
Computes the heigth of the node. 


Return value:
\begin{itemize} \item[] Height of the node.  \end{itemize}


\end{luacommand}\begin{luacommand}{{Node:getTexWidth}()}
Computes the width of the node. 


Return value:
\begin{itemize} \item[] Width of the node.  \end{itemize}


\end{luacommand}\begin{luacommand}{{Node:new}(\meta{values})}
Creates a new node. 

Parameters:
\begin{parameterdescription}
	\item[\meta{values}] Values to override default node settings. The following parameters can be set:\par |name|: The name of the node. It is obligatory to define this.\par |tex|:  Information about the corresponding \TeX\ node.\par |edges|: Edges adjacent to the node.\par |pos|: Initial position of the node.\par |options|: A table of node options passed over from \tikzname. 
\end{parameterdescription}


Return value:
\begin{itemize} \item[] A newly allocated node.  \end{itemize}


\end{luacommand}\begin{luacommand}{{Node:removeEdge}(\meta{edge})}
Removes an edge from the node. 

Parameters:
\begin{parameterdescription}
	\item[\meta{edge}] The edge to remove. 
\end{parameterdescription}



\end{luacommand}\begin{luacommand}{{Node:setOption}(\meta{name},\meta{value})}
Sets the node option \meta{name} to \meta{value}. 

Parameters:
\begin{parameterdescription}
	\item[\meta{name}] Name of the node option to be changed.\item[\meta{value}] New value for the node option \meta{name}. 
\end{parameterdescription}



\end{luacommand}
\end{filedescription}
% This file has been generated from the lua sources using LuaDoc.
% To regenerate it call "make genluadoc" in
% doc/generic/pgf/version-for-luatex/en.

\begin{filedescription}{pgflibrarygraphdrawing-edge.lua}


\begin{luacommand}{{Edge:\textunderscore{}\textunderscore{}eq}(\meta{other})}
Returns whether or not the two edges have the same adjacent nodes. 

Parameters:
\begin{parameterdescription}
	\item[\meta{other}] Another edge to compare with. 
\end{parameterdescription}


Return value:
\begin{parameterdescription} 
  \item[] |true| if the two edges have exactly the same adjacent nodes. 
\end{parameterdescription}


\end{luacommand}
\begin{luacommand}{{Edge:\textunderscore{}\textunderscore{}tostring}()}
Returns a readable string representation of the edge. 


Return value:
\begin{parameterdescription} 
  \item[] String representation of the edge. 
\end{parameterdescription}


\end{luacommand}
\begin{luacommand}{{Edge:addNode}(\meta{node})}
If possible, adds a node to the edge. 

Parameters:
\begin{parameterdescription}
	\item[\meta{node}] The node to be added to the edge. 
\end{parameterdescription}



\end{luacommand}
\begin{luacommand}{{Edge:containsNode}(\meta{node})}
Returns whether or not a node is adjacent to the edge. 

Parameters:
\begin{parameterdescription}
	\item[\meta{node}] The node to check. 
\end{parameterdescription}


Return value:
\begin{parameterdescription} 
  \item[] |true| if the node is adjacent to the edge. |false| otherwise. 
\end{parameterdescription}


\end{luacommand}
\begin{luacommand}{{Edge:copy}()}
Copies an edge (preventing accidental use).  The nodes of the edge are not preserved and have to be added to the copy manually if necessary. 


Return value:
\begin{parameterdescription} 
  \item[] Shallow copy of the edge. 
\end{parameterdescription}


\end{luacommand}
\begin{luacommand}{{Edge:getDegree}()}
Counts the nodes on this edge. 


Return value:
\begin{parameterdescription} 
  \item[] The number of nodes on the edge. 
\end{parameterdescription}


\end{luacommand}
\begin{luacommand}{{Edge:getNeighbour}(\meta{node})}
Gets first neighbour of the node (disregarding hyperedges). 

Parameters:
\begin{parameterdescription}
	\item[\meta{node}] The node which first neighbour should be returned. 
\end{parameterdescription}


Return value:
\begin{parameterdescription} 
  \item[] The first neighbour of the node. 
\end{parameterdescription}


\end{luacommand}
\begin{luacommand}{{Edge:getNeighbours}(\meta{node})}
Returns all neighbours of a node adjacent to the edge.  The edge direction is not taken into account, so this method always returns all neighbours even if called on a directed edge. 

Parameters:
\begin{parameterdescription}
	\item[\meta{node}] A node. Typically but not necessarily adjacent to the edge. If the node is not an intermediate or end point of the edge, an empty array is returned. 
\end{parameterdescription}


Return value:
\begin{parameterdescription} 
  \item[] An array of nodes that are adjacent to the input node via the edge the method is called on. 
\end{parameterdescription}


\end{luacommand}
\begin{luacommand}{{Edge:getNodes}()}
Returns all nodes of the edge.  Instead of calling |edge:getNodes()| the nodes can alternatively be accessed directly with |edge.nodes|. 


Return value:
\begin{parameterdescription} 
  \item[] All edges of the node. 
\end{parameterdescription}


\end{luacommand}
\begin{luacommand}{{Edge:getOption}(\meta{name})}
Returns the value of the edge option \meta{name}. 

Parameters:
\begin{parameterdescription}
	\item[\meta{name}] Name of the option. 
\end{parameterdescription}


Return value:
\begin{parameterdescription} 
  \item[] The value of the edge option \meta{name} or |nil|. 
\end{parameterdescription}


\end{luacommand}
\begin{luacommand}{{Edge:isHead}(\meta{node},\meta{ignore\_reversed})}
Checks whether a node is the head of the edge. Does not work for hyperedges.  This method only works for edges with two adjacent nodes.  For undirected edges or edges that point into both directions, the result will always be true. Directed edges may be reversed internally, so their head and tail might be switched. Whether or not this internal reversal is handled by this method can be specified with the optional second \meta{ignore\_reversed} parameter which is |false| by default. 

Parameters:
\begin{parameterdescription}
	\item[\meta{node}] The node to check.\item[\meta{ignore\_reversed}] Optional parameter. Set this to true if reversed edges should not be considered reversed for this method call. 
\end{parameterdescription}


Return value:
\begin{parameterdescription} 
  \item[] True if the node is the head of the edge. 
\end{parameterdescription}


\end{luacommand}
\begin{luacommand}{{Edge:isHyperedge}()}
Returns whether or not the edge is a hyperedge.  A hyperedge is an edge with more than two adjacent nodes. 


Return value:
\begin{parameterdescription} 
  \item[] |true| if the edge is a hyperedge. |false| otherwise. 
\end{parameterdescription}


\end{luacommand}
\begin{luacommand}{{Edge:isTail}(\meta{node},\meta{ignore\_reversed})}
Checks whether a node is the tail of the edge. Does not work for hyperedges.  This method only works for edges with two adjacent nodes.  For undirected edges or edges that point into both directions, the result will always be true.  Directed edges may be reversed internally, so their head and tail might be switched. Whether or not this internal reversal is handled by this method can be specified with the optional second \meta{ignore\_reversed} parameter which is |false| by default. 

Parameters:
\begin{parameterdescription}
	\item[\meta{node}] The node to check.\item[\meta{ignore\_reversed}] Optional parameter. Set this to true if reversed edges should not be considered reversed for this method call. 
\end{parameterdescription}


Return value:
\begin{parameterdescription} 
  \item[] True if the node is the tail of the edge. 
\end{parameterdescription}


\end{luacommand}
\begin{luacommand}{{Edge:new}(\meta{values})}
Creates an edge between nodes of a graph. 

Parameters:
\begin{parameterdescription}
	\item[\meta{values}] Values to override default edge settings. The following parameters can be set:\par |nodes|: TODO \par |edge_nodes|: TODO \par |options|: TODO \par |tikz_options|: TODO \par |direction|: TODO \par |bend_points|: TODO \par |bend_nodes|: TODO \par |reversed|: TODO \par 
\end{parameterdescription}


Return value:
\begin{parameterdescription} 
  \item[] A newly-allocated edge. 
\end{parameterdescription}


\end{luacommand}
\begin{luacommand}{{Edge:setOption}(\meta{name},\meta{value})}
Sets the edge option \meta{name} to \meta{value}. 

Parameters:
\begin{parameterdescription}
	\item[\meta{name}] Name of the option to be changed.\item[\meta{value}] New value for the edge option \meta{name}. 
\end{parameterdescription}



\end{luacommand}

\end{filedescription}
% This file has been generated from the lua sources using LuaDoc.
% To regenerate it call "make genluadoc" in
% doc/generic/pgf/version-for-luatex/en.

\begin{filedescription}{pgflibrarygraphdrawing-position.lua}


\begin{luacommand}{{Position.calcCoordsTo}(\meta{posFrom},\meta{posTo})}
Returns a vector between two positions.

Parameters:
\begin{parameterdescription}
	\item[\meta{posFrom}] Position A.\item[\meta{posTo}] Position B.
\end{parameterdescription}


Return value:
\begin{parameterdescription} 
  \item[] x- and y-coordinates of the vector between posFrom and posTo.
\end{parameterdescription}


\end{luacommand}
\begin{luacommand}{{Position:\textunderscore{}\textunderscore{}tostring}()}
Returns a readable string representation of the position.


Return value:
\begin{parameterdescription} 
  \item[] string representation of the position.
\end{parameterdescription}


\end{luacommand}
\begin{luacommand}{{Position:copy}()}
Creates a copy of this position object.


Return value:
\begin{parameterdescription} 
  \item[] Copy of the position.
\end{parameterdescription}


\end{luacommand}
\begin{luacommand}{{Position:equals}(\meta{pos})}
Returns a boolean value whether the object is equal to the given position.


Return value:
\begin{parameterdescription} 
  \item[] true if the position is equal to the given position pos.
\end{parameterdescription}


\end{luacommand}
\begin{luacommand}{{Position:getAbsCoordinates}(\meta{x},\meta{y})}
Computes absolute coordinates of a position.

Parameters:
\begin{parameterdescription}
	\item[\meta{x}] Just used internally for recrusion.\item[\meta{y}] Just used internally for recrusion.
\end{parameterdescription}


Return value:
\begin{parameterdescription} 
  \item[] Absolute position.
\end{parameterdescription}


\end{luacommand}
\begin{luacommand}{{Position:isAbsPosition}()}
Determines if the position is absolute.


Return value:
\begin{parameterdescription} 
  \item[] True if the position is absolute, else false.
\end{parameterdescription}


\end{luacommand}
\begin{luacommand}{{Position:new}(\meta{values})}
Represents a relative postion.

Parameters:
\begin{parameterdescription}
	\item[\meta{values}] Values (e.g. x- and y-coordinate) to be merged with the default-metatable of a position.
\end{parameterdescription}


Return value:
\begin{parameterdescription} 
  \item[] A new position object.
\end{parameterdescription}


\end{luacommand}
\begin{luacommand}{{Position:relateTo}(\meta{pos},\meta{keepAbsPosition})}
Relates a position to the given position.

Parameters:
\begin{parameterdescription}
	\item[\meta{pos}] The relative position.\item[\meta{keepAbsPosition}] If true, the coordinates of the position are computed in the relation to the given position pos.
\end{parameterdescription}



\end{luacommand}

\end{filedescription}
% This file has been generated from the lua sources using LuaDoc.
% To regenerate it call "make genluadoc" in
% doc/generic/pgf/version-for-luatex/en.

\begin{filedescription}{pgflibrarygraphdrawing-vector.lua}


\begin{luacommand}{{Vector:copy}()}
Creates a copy of the vector that holds the same elements as the original. 


Return value:
\begin{parameterdescription} 
  \item[] A newly-allocated copy of the vector holding exactly the same elements. 
\end{parameterdescription}


\end{luacommand}
\begin{luacommand}{{Vector:dividedBy}(\meta{other})}
Performs a vector division and returns the result in a new vector. 

Parameters:
\begin{parameterdescription}
	\item[\meta{other}] Vector to divide by. 
\end{parameterdescription}


Return value:
\begin{parameterdescription} 
  \item[] A new vector with the result of the division. 
\end{parameterdescription}


\end{luacommand}
\begin{luacommand}{{Vector:dividedByScalar}(\meta{scalar})}
Divides a vector by a scalar value and returns the result in a new vector. 

Parameters:
\begin{parameterdescription}
	\item[\meta{scalar}] Scalar value to divide the vector by. 
\end{parameterdescription}


Return value:
\begin{parameterdescription} 
  \item[] A new vector with the result of the division. 
\end{parameterdescription}


\end{luacommand}
\begin{luacommand}{{Vector:dotProduct}(\meta{other})}
Performs the dot product of two vectors and returns the result in a new vector. 

Parameters:
\begin{parameterdescription}
	\item[\meta{other}] Vector to perform the dot product with. 
\end{parameterdescription}


Return value:
\begin{parameterdescription} 
  \item[] A new vector with the result of the dot product. 
\end{parameterdescription}


\end{luacommand}
\begin{luacommand}{{Vector:get}(\meta{index})}
Returns the element at the given \meta{index}. 


Return value:
\begin{parameterdescription} 
  \item[] The element at the given \meta{index}. 
\end{parameterdescription}


\end{luacommand}
\begin{luacommand}{{Vector:limit}(\meta{limit\_function})}
Limits all elements of the vector in-place. 

Parameters:
\begin{parameterdescription}
	\item[\meta{limit\_function}] A function that is called for each index/element pair. It is supposed to return minimum and maximum values for the element. The element is then clamped to these values. 
\end{parameterdescription}



\end{luacommand}
\begin{luacommand}{{Vector:minus}(\meta{other})}
Subtracts two vectors and returns the result in a new vector. 

Parameters:
\begin{parameterdescription}
	\item[\meta{other}] Vector to subtract. 
\end{parameterdescription}


Return value:
\begin{parameterdescription} 
  \item[] A new vector with the result of the subtraction. 
\end{parameterdescription}


\end{luacommand}
\begin{luacommand}{{Vector:minusScalar}(\meta{scalar})}
Subtracts a scalar value from a vector and returns the result in a new vector. 

Parameters:
\begin{parameterdescription}
	\item[\meta{scalar}] Scalar value to subtract from all elements. 
\end{parameterdescription}


Return value:
\begin{parameterdescription} 
  \item[] A new vector with the result of the subtraction. 
\end{parameterdescription}


\end{luacommand}
\begin{luacommand}{{Vector:new}(\meta{n},\meta{fill\_function})}
Creates a new vector with \meta{n} values using an optional \meta{fill\_function}. 

Parameters:
\begin{parameterdescription}
	\item[\meta{n}] The number of elements of the vector.\item[\meta{fill\_function}] Optional function that takes a number between 1 and \meta{n} and is expected to return a value for the corresponding element of the vector. If omitted, all elements of the vector will be initialized with 0. 
\end{parameterdescription}


Return value:
\begin{parameterdescription} 
  \item[] A newly-allocated vector with \meta{n} elements. 
\end{parameterdescription}


\end{luacommand}
\begin{luacommand}{{Vector:norm}()}
Computes the Euclidean norm of the vector. 


Return value:
\begin{parameterdescription} 
  \item[] The Euclidean norm of the vector. 
\end{parameterdescription}


\end{luacommand}
\begin{luacommand}{{Vector:normalized}()}
Normalizes the vector and returns the result in a new vector. 


Return value:
\begin{parameterdescription} 
  \item[] Normalized version of the original vector. 
\end{parameterdescription}


\end{luacommand}
\begin{luacommand}{{Vector:plus}(\meta{other})}
Performs a vector addition and returns the result in a new vector. 

Parameters:
\begin{parameterdescription}
	\item[\meta{other}] The vector to add. 
\end{parameterdescription}


Return value:
\begin{parameterdescription} 
  \item[] A new vector with the result of the addition. 
\end{parameterdescription}


\end{luacommand}
\begin{luacommand}{{Vector:plusScalar}(\meta{scalar})}
Performs an addition with a scalar value and returns the result in a new vector.  The scalar value is added to all elements of the vector. 

Parameters:
\begin{parameterdescription}
	\item[\meta{scalar}] Scalar value to add to all elements. 
\end{parameterdescription}


Return value:
\begin{parameterdescription} 
  \item[] A new vector with the result of the addition. 
\end{parameterdescription}


\end{luacommand}
\begin{luacommand}{{Vector:reset}()}
Resets all vector elements to 0 in-place. 



\end{luacommand}
\begin{luacommand}{{Vector:set}(\meta{index},\meta{value})}
Changes the element at the given \meta{index}. 

Parameters:
\begin{parameterdescription}
	\item[\meta{index}] The index of the element to change.\item[\meta{value}] New value of the element. 
\end{parameterdescription}



\end{luacommand}
\begin{luacommand}{{Vector:timesScalar}(\meta{scalar})}
Multiplies a vector by a scalar value and returns the result in a new vector. 

Parameters:
\begin{parameterdescription}
	\item[\meta{scalar}] Scalar value to multiply the vector with. 
\end{parameterdescription}


Return value:
\begin{parameterdescription} 
  \item[] A new vector with the result of the multiplication. 
\end{parameterdescription}


\end{luacommand}
\begin{luacommand}{{Vector:update}(\meta{update\_function})}
Updates the values of the vector in-place. 

Parameters:
\begin{parameterdescription}
	\item[\meta{update\_function}] A function that is called for each element of the vector. The elements are replaced by the values returned from this function. 
\end{parameterdescription}



\end{luacommand}
\begin{luacommand}{{Vector:x}()}
Convenience method that returns the first element of the vector. 


Return value:
\begin{parameterdescription} 
  \item[] The first element of the vector. 
\end{parameterdescription}


\end{luacommand}
\begin{luacommand}{{Vector:y}()}
Convenience method that returns the second element of the vector. 


Return value:
\begin{parameterdescription} 
  \item[] The second element of the vector. 
\end{parameterdescription}


\end{luacommand}

\end{filedescription}
% This file has been generated from the lua sources using LuaDoc.
% To regenerate it call "make genluadoc" in
% doc/generic/pgf/version-for-luatex/en.

\begin{filedescription}{pgflibrarygraphdrawing-box.lua}


\begin{luacommand}{{Box:addBox}(\meta{box})}
Adds new internal Box.

Parameters:
\begin{parameterdescription}
	\item[\meta{box}] The box to be added.
\end{parameterdescription}



\end{luacommand}\begin{luacommand}{{Box:getPaths}()}
Provides all Paths this box contains.


Return value:
\begin{itemize} \item[] Recursive iteration over all paths. \end{itemize}


\end{luacommand}\begin{luacommand}{{Box:getPosAt}(\meta{place},\meta{absolute})}
Calculates the coordinates of the box according to the place parameter.

Parameters:
\begin{parameterdescription}
	\item[\meta{place}] Determines of which position of the box the coordinates should be returned (e.g. the center of the box requires the param Box.CENTER).  Possible values are: \begin{itemize} \item Box.UPPERLEFT \item Box.UPPERRIGHT \item Box.CENTER \item Box.LOWERRIGHT \item Box.LOWERLEFT \end{itemize}\item[\meta{absolute}] If true the absolute coordinates of the box will be returned, otherwise its relative coordinates.
\end{parameterdescription}


Return value:
\begin{itemize} \item[] X- and y-coordinates of the box. \end{itemize}


\end{luacommand}\begin{luacommand}{{Box:new}(\meta{values})}
Creates a new box.

Parameters:
\begin{parameterdescription}
	\item[\meta{values}] Values (e.g. height) to be merged with the default-metatable of a box.
\end{parameterdescription}


Return value:
\begin{itemize} \item[] The new box. \end{itemize}


\end{luacommand}\begin{luacommand}{{Box:recalculateSize}()}
Checks internal Boxes and resets width and height.



\end{luacommand}\begin{luacommand}{{Box:removeBox}(\meta{box})}
Removes internal Box.

Parameters:
\begin{parameterdescription}
	\item[\meta{box}] The box to remove.
\end{parameterdescription}



\end{luacommand}
\end{filedescription}
%% This file has been generated from the lua sources using LuaDoc.
% To regenerate it call "make genluadoc" in
% doc/generic/pgf/version-for-luatex/en.

\begin{filedescription}{pgflibrarygraphdrawing-path.lua}


\begin{luacommand}{{Path:\textunderscore{}\textunderscore{}tostring}()}
Returns a readable string representation of the path.


Return value:
\begin{parameterdescription} 
  \item[] String representation of the path
\end{parameterdescription}


\end{luacommand}
\begin{luacommand}{{Path:\textunderscore{}intersects}(\meta{a1},\meta{a2},\meta{b1},\meta{b2},\meta{allowedIntersections})}
Checks if the lines a1a2 and b1b2 intersect.

Parameters:
\begin{parameterdescription}
	\item[\meta{a1}] Start of the first line.\item[\meta{a2}] End of the first line.\item[\meta{b1}] Start of the second line.\item[\meta{b2}] End of the second line.\item[\meta{allowedIntersections}] A boolean table with the keys a1, a2, b1 and b2. If two or three of those values are true, the corresponding start and/or end points are allowed to match without being seen as intersection. If all four keys are true any matching of start and end points is allowed as long as the two lines are not coincedent. If three of the keys are true or start and end of a line are allowed to match, nill will be returned. If this optional parameter is not given, any matching points will be seen as intersections.
\end{parameterdescription}


Return value:
\begin{parameterdescription} 
  \item[] true, if lines intersect, false otherwise. If allowedIntersections contained an invalid value, nil will be returned.
\end{parameterdescription}


\end{luacommand}
\begin{luacommand}{{Path:addPoint}(\meta{point},\meta{keepAbsPosition})}
Appends new point at the end of path.

Parameters:
\begin{parameterdescription}
	\item[\meta{point}] Point to be added to the path\item[\meta{keepAbsPosition}] true if the coordinates of the point are absolute
\end{parameterdescription}



\end{luacommand}
\begin{luacommand}{{Path:createPath}(\meta{posStart},\meta{posEnd},\meta{keepAbsPosition})}
Adds a new segment to the path.

Parameters:
\begin{parameterdescription}
	\item[\meta{posStart}] Startposition of the new segment\item[\meta{posEnd}] Endposition of the new segment
\end{parameterdescription}



\end{luacommand}
\begin{luacommand}{{Path:getLastPoint}()}
Returns last point in path.


Return value:
\begin{parameterdescription} 
  \item[] last point
\end{parameterdescription}


\end{luacommand}
\begin{luacommand}{{Path:getLength}()}
Returns the length of the whole path.


Return value:
\begin{parameterdescription} 
  \item[] Length of the whole path.
\end{parameterdescription}


\end{luacommand}
\begin{luacommand}{{Path:getPoints}()}
Copies the internal points of a path.


Return value:
\begin{parameterdescription} 
  \item[] array of points
\end{parameterdescription}


\end{luacommand}
\begin{luacommand}{{Path:intersects}(\meta{path})}
Tests if the path is intersected by path.

Parameters:
\begin{parameterdescription}
	\item[\meta{path}] other path
\end{parameterdescription}



\end{luacommand}
\begin{luacommand}{{Path:move}(\meta{x},\meta{y})}
Adds new point with x,y relative to last point.

Parameters:
\begin{parameterdescription}
	\item[\meta{x}] x-coordinate of the new point\item[\meta{y}] y-coordinate of the new point
\end{parameterdescription}



\end{luacommand}
\begin{luacommand}{{Path:new}(\meta{values})}
Creates a new path.

Parameters:
\begin{parameterdescription}
	\item[\meta{values}] Values to be merged with the default-metatable of a path
\end{parameterdescription}


Return value:
\begin{parameterdescription} 
  \item[] A new path.
\end{parameterdescription}


\end{luacommand}

\end{filedescription}

\subsubsection{Base Layer}
% This file has been generated from the lua sources using LuaDoc.
% To regenerate it call "make genluadoc" in
% doc/generic/pgf/version-for-luatex/en.

\begin{filedescription}{pgflibrarygraphdrawing-interface.lua}


\begin{luacommand}{{Interface:addEdge}(\meta{from},\meta{to},\meta{direction},\meta{edge\_nodes},\meta{options},\meta{tikz\_options})}
Adds an edge from one node to another by name.  Both parameters are node names and have to exist before an edge can be created between them. 

Parameters:
\begin{parameterdescription}
	\item[\meta{from}] Name of the node the edge begins at.\item[\meta{to}] Name of the node the edge ends at.\item[\meta{direction}] Direction of the edge (e.g. |--| for an undirected edge or |->| for a directed edge from the first to the second node).\item[\meta{edge\_nodes}] A string for \tikzname\ to generate the edge label nodes later. Needs to be passed back to TikZ unmodified.\item[\meta{options}] A string of |{key}{value}| pairs of edge options that are relevant to graph drawing algorithms.\item[\meta{tikz\_options}] A string of |{key}{value}| pairs that need to be passed back to \tikzname\ unmodified. 
\end{parameterdescription}



See also:
\begin{itemize}
	\item[] |addNode |
\end{itemize}

\end{luacommand}
\begin{luacommand}{{Interface:addNode}(\meta{name},\meta{xMin},\meta{yMin},\meta{xMax},\meta{yMax},\meta{options})}
Adds a new node to the graph.  The options string of |{key}{value}| pairs is parsed and assigned to the node. Graph drawing algorithms may use these options to treat the node in special ways. 

Parameters:
\begin{parameterdescription}
	\item[\meta{name}] Name of the node.\item[\meta{xMin}] Minimum x point of the bouding box.\item[\meta{yMin}] Minimum y point of the bouding box.\item[\meta{xMax}] Maximum x point of the bouding box.\item[\meta{yMax}] Maximum y point of the bouding box.\item[\meta{options}] Options for the node. 
\end{parameterdescription}



\end{luacommand}
\begin{luacommand}{{Interface:drawEdge}(\meta{edge})}
Passes an edge back to the \TeX\ layer.  Edges with a direction of |Edge.NONE| are skipped and not passed back to \TeX. 

Parameters:
\begin{parameterdescription}
	\item[\meta{edge}] The edge to pass back to the \TeX\ layer. 
\end{parameterdescription}



\end{luacommand}
\begin{luacommand}{{Interface:drawGraph}()}
Arranges the current graph using the specified algorithm.  The algorithm is derived from the graph options and is loaded on demand from the corresponding algorithm file. For a fictitious algorithm |simple| this file is per convention called |pgflibrarygraphdrawing-algorithms-simple.lua|. It is required to define at least one function as an entry point to the algorithm. The name of the function is again predetermined as |graph_drawing_algorithm_simple|. When a graph is to be layed out, this function is called with the graph as its only parameter. 



\end{luacommand}
\begin{luacommand}{{Interface:drawNode}(\meta{node})}
Passes a node back to the \TeX\ layer. 

Parameters:
\begin{parameterdescription}
	\item[\meta{node}] The node to pass back to the \TeX\ layer. 
\end{parameterdescription}



\end{luacommand}
\begin{luacommand}{{Interface:finishGraph}()}
Passes the current graph back to the \TeX\ layer and removes it from the stack. 



\end{luacommand}
\begin{luacommand}{{Interface:getOption}(\meta{name})}
Returns the value of the graph option \meta{name}. 

Parameters:
\begin{parameterdescription}
	\item[\meta{name}] Name of the option. 
\end{parameterdescription}


Return value:
\begin{parameterdescription} 
  \item[] The value of the \meta{name} option or |nil|. 
\end{parameterdescription}


\end{luacommand}
\begin{luacommand}{{Interface:loadAlgorithm}(\meta{name})}
Attempts to load the algorithm with the given \meta{name}.  This function tries to look up the corresponding algorithm file |pgflibrarygraphdrawing-algorithms-name.lua| and attempts to look up the main function for calling the algorithm. 

Parameters:
\begin{parameterdescription}
	\item[\meta{name}] Name of the algorithm. 
\end{parameterdescription}


Return value:
\begin{parameterdescription} 
  \item[] The algorithm function or nil. 
\end{parameterdescription}


\end{luacommand}
\begin{luacommand}{{Interface:newGraph}(\meta{options})}
Creates a new graph and adds it to the graph stack.  The options string consisting of |{key}{value}| pairs is parsed and assigned to the graph. These options are used to configure the different graph drawing algorithms shipped with \tikzname. 

Parameters:
\begin{parameterdescription}
	\item[\meta{options}] A string containing |{key}{value}| pairs of \tikzname\ options. 
\end{parameterdescription}



See also:
\begin{itemize}
	\item[] |finishGraph |
\end{itemize}

\end{luacommand}
\begin{luacommand}{{Interface:setOption}(\meta{name},\meta{value})}
Sets the graph option \meta{name} to \meta{value}. Only affects the current graph. 

Parameters:
\begin{parameterdescription}
	\item[\meta{name}] The name of the option to set.\item[\meta{value}] New value for the option. 
\end{parameterdescription}



\end{luacommand}

\end{filedescription}
\label{section-library-graphdrawing-lua-documentation-interface}
% This file has been generated from the lua sources using LuaDoc.
% To regenerate it call "make genluadoc" in
% doc/generic/pgf/version-for-luatex/en.

\begin{filedescription}{pgflibrarygraphdrawing-sys.lua}


\begin{luacommand}{{Sys:beginShipout}()}
Begins the shipout of nodes by opening a scope in pgf.



\end{luacommand}\begin{luacommand}{{Sys:endShipout}()}
Ends the shipout by closing the opened scope.



See also:
\begin{itemize}
	\item[] |Sys:beginShipout()|
\end{itemize}

\end{luacommand}\begin{luacommand}{{Sys:escapeTeXNodeName}(\meta{nodename})}
Adds a ``not yet positionedPGFGDINTERNAL'' prefix to a node name. The prefix is required by pgf to place the node. Actually, when deferring the node placement, the prefix is added to avoid references to the node.

Parameters:
\begin{parameterdescription}
	\item[\meta{nodename}] Name of the node to prefix.
\end{parameterdescription}


Return value:
\begin{itemize} \item[] A newly composed string. \end{itemize}


\end{luacommand}\begin{luacommand}{{Sys:getTeXBox}()}
Retrieves a box from the transfer box register.



See also:
\begin{itemize}
	\item[] |putTeXBox|
\end{itemize}

\end{luacommand}\begin{luacommand}{{Sys:getVerboseMode}()}
Checks the verbosity of the subsystems output.


Return value:
\begin{itemize} \item[] Boolean value specifying the verbosity. \end{itemize}


\end{luacommand}\begin{luacommand}{{Sys:logMessage}(\meta{...})}
Prints objects to the TeX output, formatting them with tostring and separated by spaces.

Parameters:
\begin{parameterdescription}
	\item[\meta{...}] List of parameters.
\end{parameterdescription}



\end{luacommand}\begin{luacommand}{{Sys:putEdge}(\meta{edge},\meta{Edge})}
Assembles and outputs the TeX command to draw an edge.

Parameters:
\begin{parameterdescription}
	\item[\meta{Edge}] A lua edge object.
\end{parameterdescription}



\end{luacommand}\begin{luacommand}{{Sys:putTeXBox}(\meta{nodename},\meta{texnode},\meta{minX},\meta{minY},\meta{maxX},\meta{maxY},\meta{posX},\meta{posY},\meta{nodeName})}
Saves a box from the transfer box register.

Parameters:
\begin{parameterdescription}
	\item[\meta{texnode}] The box which contains the \TeX\ node.\item[\meta{minX}] Maximum y of the bounding box.\item[\meta{minY}] Minimal y of the bounding box.\item[\meta{posX}] X coordinate where to put the node in the output.\item[\meta{posY}] Y coordinate where to put the node in the output.\item[\meta{nodeName}] The name of the node in the box.
\end{parameterdescription}



\end{luacommand}\begin{luacommand}{{Sys:setBoxNumber}(\meta{bn})}
Init method, sets the box register number. This method is called when the \tikzname\ (pgf) library is loaded.

Parameters:
\begin{parameterdescription}
	\item[\meta{bn}] Number of the box register used for transfering boxes of the current graph.
\end{parameterdescription}



\end{luacommand}\begin{luacommand}{{Sys:setVerboseMode}(\meta{mode})}
Enables or disables verbose logging for the graph drawing library.

Parameters:
\begin{parameterdescription}
	\item[\meta{mode}] If true, enable verbose logging. Otherwise it'll be disabled.
\end{parameterdescription}



\end{luacommand}\begin{luacommand}{{Sys:unescapeTeXNodeName}(\meta{nodename})}
Removes the ``not yet positionedPGFGDINTERNAL'' prefix from a node name.

Parameters:
\begin{parameterdescription}
	\item[\meta{nodename}] Nodename without prefix.
\end{parameterdescription}


Return value:
\begin{itemize} \item[] The substring in question. \end{itemize}


See also:
\begin{itemize}
	\item[] |Sys:escapeTeXNodeName(nodename)|
\end{itemize}

\end{luacommand}
\end{filedescription}
\label{section-library-graphdrawing-lua-documentation-sys}
% This file has been generated from the lua sources using LuaDoc.
% To regenerate it call "make genluadoc" in
% doc/generic/pgf/version-for-luatex/en.

\begin{filedescription}{pgflibrarygraphdrawing-texboxregister.lua}


\begin{luacommand}{{TeXBoxRegister:getBox}(\meta{boxReference})}
Gets a box by its reference.

Parameters:
\begin{parameterdescription}
	\item[\meta{boxReference}] Reference id of the box to get.
\end{parameterdescription}



See also:
\begin{itemize}
	\item[] |TeXBoxRegister:insertBox(texbox)|
\end{itemize}

\end{luacommand}
\begin{luacommand}{{TeXBoxRegister:insertBox}(\meta{texbox})}
Adds the content of a \TeX\ box to the box register class. Contents of the box will be stored. 



\end{luacommand}

\end{filedescription}

\subsubsection{Helper Classes}
% This file has been generated from the lua sources using LuaDoc.
% To regenerate it call "make genluadoc" in
% doc/generic/pgf/version-for-luatex/en.

\begin{filedescription}{pgflibrarygraphdrawing-helper.lua}


\begin{luacommand}{{parseBraces}(\meta{str},\meta{default})}
Parses a braced list of {key}{value} pairs and returns a table mapping keys to values.



\end{luacommand}

\end{filedescription}
% This file has been generated from the lua sources using LuaDoc.
% To regenerate it call "make genluadoc" in
% doc/generic/pgf/version-for-luatex/en.

\begin{filedescription}{pgflibrarygraphdrawing-table-helpers.lua}


\begin{luacommand}{{table.combine\textunderscore{}pairs}(\meta{table},\meta{combine\_func},\meta{initial\_value})}
Combine all key/value pairs of \meta{table} to a single value using a combine function.  This is a very powerful function. It can be used for combining the key/value pairs of a table into a single string but can also be used to compute mathematical operations on tables, such as finding the maximum value in a table etc.  The main difference to |table.combine_values| is that keys and values are used to determine the combination value and that the key/value pairs are are passed to \meta{combine\_func} in a random order. 

Parameters:
\begin{parameterdescription}
	\item[\meta{table}] Table to iterate over.\item[\meta{combine\_func}] Function to be called for each key/value pair. It takes three parameters, the current combination value and the key/value pair. It is supposed to return a new combination value.\item[\meta{initial\_value}] Initial combination value. 
\end{parameterdescription}


Return value:
\begin{parameterdescription} 
  \item[] The final combination value after all key/value pairs have been passed over to \meta{combine\_func}. 
\end{parameterdescription}


\end{luacommand}
\begin{luacommand}{{table.combine\textunderscore{}values}(\meta{input},\meta{combine\_func},\meta{initial\_value})}
Combine all values of \meta{input} to a single value using a combine function.  This is a very powerful function. It can be used for combining the values of a table into a single string but can also be used to compute mathematical operations on tables, such as finding the maximum value in a table etc.  The main difference to |table.combine_pairs| is that the keys are ignored and that the values are passed to \meta{combine\_func} in the order they appear in the table. 

Parameters:
\begin{parameterdescription}
	\item[\meta{input}] Table to iterate over.\item[\meta{combine\_func}] Function to be called for each value. It takes two parameters, the current combination value and the current value. It is supposed to return a new combination value.\item[\meta{initial\_value}] Initial combination value. 
\end{parameterdescription}


Return value:
\begin{parameterdescription} 
  \item[] The final combination value after all values of \meta{input} have been passed over to \meta{combine\_func}. 
\end{parameterdescription}


\end{luacommand}
\begin{luacommand}{{table.copy}(\meta{source},\meta{target})}
Copies a table while preserving its metatable. 

Parameters:
\begin{parameterdescription}
	\item[\meta{source}] The table to copy.\item[\meta{target}] The table to which values are to be copied or |nil| if a new table is to be allocated. 
\end{parameterdescription}


Return value:
\begin{parameterdescription} 
  \item[] The \meta{target} table or a newly allocated table containing all keys and values of the \meta{source} table. 
\end{parameterdescription}


\end{luacommand}
\begin{luacommand}{{table.count\textunderscore{}pairs}(\meta{input})}
Count the key/value pairs in the table. 

Parameters:
\begin{parameterdescription}
	\item[\meta{input}] The table whose key/value pairs to count. 
\end{parameterdescription}


Return value:
\begin{parameterdescription} 
  \item[] Number of key/value pairs in the table. 
\end{parameterdescription}


\end{luacommand}
\begin{luacommand}{{table.filter\textunderscore{}keys}(\meta{table},\meta{filter\_func})}
Copies a table and filters out all keys using a function. 

Parameters:
\begin{parameterdescription}
	\item[\meta{table}] The table whose values are to be filtered.\item[\meta{filter\_func}] The test function to be called for each key of \meta{table}. If it returns |false| or |nil| for a key, that key will not be part of the result table. 
\end{parameterdescription}


Return value:
\begin{parameterdescription} 
  \item[] Copy of \meta{table} with its keys filtered using \meta{filter\_func}. 
\end{parameterdescription}


\end{luacommand}
\begin{luacommand}{{table.filter\textunderscore{}pairs}(\meta{table},\meta{filter\_func})}
Copies a table and filters out all key/value pairs using a function. 

Parameters:
\begin{parameterdescription}
	\item[\meta{table}] The table whose values are to be filtered.\item[\meta{filter\_func}] The test function to be called for each pair of \meta{table}. If it returns |false| or |nil| for a pair, that pair will not be part of the result table. 
\end{parameterdescription}


Return value:
\begin{parameterdescription} 
  \item[] Copy of \meta{table} with its pairs filtered using \meta{filter\_func}. 
\end{parameterdescription}


\end{luacommand}
\begin{luacommand}{{table.filter\textunderscore{}values}(\meta{input},\meta{filter\_func})}
Copies a table and filters out all values using a function. 

Parameters:
\begin{parameterdescription}
	\item[\meta{input}] The table whose values are to be filtered.\item[\meta{filter\_func}] The test function to be called for each value of the input table. If it returns |false| or |nil| for a value, that value will not be part of the result table. 
\end{parameterdescription}


Return value:
\begin{parameterdescription} 
  \item[] Copy of \meta{input} with its values filtered using \meta{filter\_func}. 
\end{parameterdescription}


\end{luacommand}
\begin{luacommand}{{table.find}(\meta{table},\meta{find\_func})}
Returns the first value in \meta{table} for which \meta{find\_func} returns |true|. 

Parameters:
\begin{parameterdescription}
	\item[\meta{table}] The table to search in.\item[\meta{find\_func}] A function to test values with. It receives a single parameter (a value of \meta{table}) and is supposed to return either |true| or |false|. 
\end{parameterdescription}


Return value:
\begin{parameterdescription} 
  \item[] The first value of \meta{table} for which \meta{find\_func} returns true. Returns |nil| if the function was |false| for al of the values in \meta{table}. 
\end{parameterdescription}


\end{luacommand}
\begin{luacommand}{{table.find\textunderscore{}index}(\meta{table},\meta{find\_func})}
Returns the index of the first value in \meta{table} for which \meta{find\_func} returns |true|. 

Parameters:
\begin{parameterdescription}
	\item[\meta{table}] The table to search in.\item[\meta{find\_func}] A function to test values with. It receives a single parameter (a value of \meta{table}) and is supposed to return either |true| or |false|. 
\end{parameterdescription}


Return value:
\begin{parameterdescription} 
  \item[] Index of the first value of \meta{table} for which \meta{find\_func} returns |true|. Returns |nil| if the function was |false| for all of the values in \meta{table}. 
\end{parameterdescription}


\end{luacommand}
\begin{luacommand}{{table.key\textunderscore{}iter}(\meta{table})}
Iterate over all keys of a table in random order. 

Parameters:
\begin{parameterdescription}
	\item[\meta{table}] The table whose keys to iterate over. 
\end{parameterdescription}


Return value:
\begin{parameterdescription} 
  \item[] An iterator for the keys of the table. 
\end{parameterdescription}


\end{luacommand}
\begin{luacommand}{{table.map}(\meta{input},\meta{map\_func})}
Maps key/value pairs of an \meta{input} table to a flat table of new values. 

Parameters:
\begin{parameterdescription}
	\item[\meta{input}] Table whose key/value pairs are to be mapped to new values.\item[\meta{map\_func}] The mapping function to be called for each key/value pair of \meta{input}. The value it returns for a pair will be inserted into the result table. 
\end{parameterdescription}


Return value:
\begin{parameterdescription} 
  \item[] A new table containing all values returned by \meta{map\_func} for the key/value pairs of the \meta{input} table. 
\end{parameterdescription}


\end{luacommand}
\begin{luacommand}{{table.map\textunderscore{}keys}(\meta{table},\meta{map\_func})}
Maps keys of a table to new keys in a copy of the table. 

Parameters:
\begin{parameterdescription}
	\item[\meta{table}] The table whose keys are to be mapped to new keys.\item[\meta{map\_func}] A function to be called for each key of \meta{table} in order to generate a new key to replace the old one in the result table. 
\end{parameterdescription}


Return value:
\begin{parameterdescription} 
  \item[] A new table with all keys of \meta{table} having been replaced with the keys returned from \meta{map\_func}. The original values are preserved. 
\end{parameterdescription}


\end{luacommand}
\begin{luacommand}{{table.map\textunderscore{}pairs}(\meta{table},\meta{map\_func})}
Maps keys and values of a table to new pairs of keys and values. 

Parameters:
\begin{parameterdescription}
	\item[\meta{table}] The table whose key and value pairs are to be replaced.\item[\meta{map\_func}] A function to be called for each key and value pair of \meta{table} in order to generate a new pair to replace the old one. 
\end{parameterdescription}


Return value:
\begin{parameterdescription} 
  \item[] A new table with all key and value pairs of \meta{table} having been replaced with the pairs returned from \meta{map\_func}. 
\end{parameterdescription}


\end{luacommand}
\begin{luacommand}{{table.map\textunderscore{}values}(\meta{input},\meta{map\_func})}
Maps values of a table to new values in a new table. 

Parameters:
\begin{parameterdescription}
	\item[\meta{input}] The table whose values are to be mapped to new values.\item[\meta{map\_func}] A function to be called for each value in order to generate a new value to replace the old one in the result table. 
\end{parameterdescription}


Return value:
\begin{parameterdescription} 
  \item[] A new table with all values of the \meta{input} table having been replaced with the values returned from \meta{map\_func}. 
\end{parameterdescription}


\end{luacommand}
\begin{luacommand}{{table.merge}(\meta{table1},\meta{table2},\meta{first\_metatable})}
Merges the key/value pairs of two tables.  This function merges the key/value pairs of the two input tables.  All |nil| values of the first table are overwritten by the corresponding values of the second table.  By default the metatable of the second input table is applied to the resulting table. If \meta{first\_metatable} is set to |true| however, the metatable of the first input table will be used. 

Parameters:
\begin{parameterdescription}
	\item[\meta{table1}] First table with key/value pairs.\item[\meta{table2}] Second table with key/value pairs.\item[\meta{first\_metatable}] Whether to inherit the metatable of \meta{table1} or not. 
\end{parameterdescription}


Return value:
\begin{parameterdescription} 
  \item[] A new table with the key/value pairs of the two input tables merged together. 
\end{parameterdescription}


\end{luacommand}
\begin{luacommand}{{table.randomized\textunderscore{}pair\textunderscore{}iter}(\meta{table})}
Iterate over the key/value pairs of \meta{table} in a truely random order. 

Parameters:
\begin{parameterdescription}
	\item[\meta{table}] The table whose key/value pairs to iterate over. 
\end{parameterdescription}


Return value:
\begin{parameterdescription} 
  \item[] A randomized iterator for the values of \meta{table}. 
\end{parameterdescription}


\end{luacommand}
\begin{luacommand}{{table.randomized\textunderscore{}value\textunderscore{}iter}(\meta{table})}
Iterate over the values of \meta{table} in a truely random order. 

Parameters:
\begin{parameterdescription}
	\item[\meta{table}] The table whose values to iterate over. 
\end{parameterdescription}


Return value:
\begin{parameterdescription} 
  \item[] A randomized iterator for the values of the table. 
\end{parameterdescription}


\end{luacommand}
\begin{luacommand}{{table.remove\textunderscore{}values}(\meta{input},\meta{remove\_func})}
Removes all values from \meta{input} for which \meta{remove\_func} returns |true|.  Important note: this method does not work with dictionaries. Make sure only to process number-indexed arrays with it. 

Parameters:
\begin{parameterdescription}
	\item[\meta{input}] The table to remove values from.\item[\meta{remove\_func}] Function to be called for each value of \meta{input}. If it returns |false|, the value will be removed from the table in-place. 
\end{parameterdescription}


Return value:
\begin{parameterdescription} 
  \item[] \meta{input} which was edited in-place. 
\end{parameterdescription}


\end{luacommand}
\begin{luacommand}{{table.update\textunderscore{}values}(\meta{table},\meta{update\_func})}
Update values of \meta{table} in-place using an update function. 

Parameters:
\begin{parameterdescription}
	\item[\meta{table}] The table whose values are to be updated.\item[\meta{update\_func}] A function that takes two parameters, the key/value pairs of \meta{table} and returns a new value to replace the old one. 
\end{parameterdescription}


Return value:
\begin{parameterdescription} 
  \item[] The input \meta{table}. 
\end{parameterdescription}


\end{luacommand}
\begin{luacommand}{{table.value\textunderscore{}iter}(\meta{table})}
Iterate over all values of a table.  FIXME: The iterators stops if a key's value is nil. But we actually want to continue iterating until the end of the table. 

Parameters:
\begin{parameterdescription}
	\item[\meta{table}] The table whose values to iterate over. 
\end{parameterdescription}


Return value:
\begin{parameterdescription} 
  \item[] An iterator for the values of the table. 
\end{parameterdescription}


\end{luacommand}

\end{filedescription}
% This file has been generated from the lua sources using LuaDoc.
% To regenerate it call "make genluadoc" in
% doc/generic/pgf/version-for-luatex/en.

\begin{filedescription}{pgflibrarygraphdrawing-iter-helpers.lua}


\begin{luacommand}{{iter.filter}(\meta{iterator},\meta{filter\_func})}
Skips all values of an iterator for which \meta{filter\_func} returns |false|. 

Parameters:
\begin{parameterdescription}
	\item[\meta{iterator}] Original \meta{iterator} of values.\item[\meta{filter\_func}] Filter function that takes a value of the original \meta{iterator} and is expected to return |false| if the value should be skipped. 
\end{parameterdescription}


Return value:
\begin{parameterdescription} 
  \item[] A modified iterator that skips values of \meta{iterator} for which \meta{filter\_func} returns |false|. 
\end{parameterdescription}


\end{luacommand}
\begin{luacommand}{{iter.map}(\meta{iterator},\meta{map\_func})}
Maps all values of an iterator to new values.  This function will cause loops to iterate over the values of the original \meta{iterator} replaced by the values returned from \meta{map\_func}. 

Parameters:
\begin{parameterdescription}
	\item[\meta{iterator}] Original iterator whose values are to be mapped to new ones.\item[\meta{map\_func}] Mapping function that takes a value of the original \meta{iterator} and maps it to a new value that is then returned to the loop instead. 
\end{parameterdescription}


Return value:
\begin{parameterdescription} 
  \item[] A modified iterator. 
\end{parameterdescription}


\end{luacommand}
\begin{luacommand}{{iter.times}(\meta{n})}
Causes a loop to run multiple times.  Use this iterator like this to perform 100 loops:\\ |for n in iter.times(100) do ... end|.  To iterate over the values $0, 10, 20, 30, ..., 100$ do:\\ |for n in iter.filter(iter.times(100), function (n) return n % 10 == 0 end)| 

Parameters:
\begin{parameterdescription}
	\item[\meta{n}] Number of loops. 
\end{parameterdescription}



\end{luacommand}

\end{filedescription}
% This file has been generated from the lua sources using LuaDoc.
% To regenerate it call "make genluadoc" in
% doc/generic/pgf/version-for-luatex/en.

\begin{filedescription}{pgflibrarygraphdrawing-traversal-helpers.lua}


\begin{luacommand}{{traversal.depth\textunderscore{}first\textunderscore{}dag}(\meta{graph},\meta{initial\_nodes})}
Iterator for traversing a directed acyclic \meta{graph} in depth-first order. 

Parameters:
\begin{parameterdescription}
	\item[\meta{graph}] A directed acyclic graph. 
\end{parameterdescription}


Return value:
\begin{parameterdescription} 
  \item[] An iterator for traversing \meta{graph} in a depth-first order. 
\end{parameterdescription}


\end{luacommand}
\begin{luacommand}{{traversal.topological\textunderscore{}sorting}(\meta{graph})}
Iterator for traversing a directed \meta{graph} using a topological sorting.  A topological sorting of a directed graph is a linear ordering of its nodes such that, for every edge |(u,v)|, |u| comes before |v|.  Important note: if performed on a graph with at least one cycle a topological sorting is impossible. Thus, the nodes returned from the iterator are not guaranteed to satisfy the ``|u| comes before |v|'' criterion. The iterator may even terminate early or loop forever. 

Parameters:
\begin{parameterdescription}
	\item[\meta{graph}] A directed acyclic graph. 
\end{parameterdescription}


Return value:
\begin{parameterdescription} 
  \item[] An iterator for traversing \meta{graph} in a topological order. 
\end{parameterdescription}


\end{luacommand}

\end{filedescription}


\end{document}

% The titlepage

\newbox\mybox
{
  \parindent0pt
  \null
  \colorlet{mintgreen}{green!50!black!50}

  \thispagestyle{empty}
  \vskip3cm
  \vfill
  \hfil
  \begin{tikzpicture}[overlay]
    \coordinate (front) at (0,0);
    \coordinate (horizon) at (0,.31\paperheight);
    \coordinate (bottom) at (0,-.6\paperheight);
    \coordinate (sky) at (0,.57\paperheight);
    \coordinate (left) at (-.51\paperwidth,0);
    \coordinate (right) at (.51\paperwidth,0);

    \shade [bottom color=blue!30!black!10,top color=blue!30!black!50]
      ([yshift=-5mm]horizon -|  left) rectangle (sky -| right);
    \shade [bottom color=black!70!green!25,top color=black!70!green!10]
      (front -| left) -- (horizon -| left)
      decorate [decoration=random steps] { -- (horizon -| right) }
      -- (front -| right) -- cycle;
    \shade [top color=black!70!green!25,bottom color=black!25]
      ([yshift=-5mm-1pt]front -| left) rectangle ([yshift=1pt]front -| right);
    \fill [black!25] (bottom -| left) rectangle ([yshift=-5mm]front -| right);

    \def\nodeshadowed[#1]#2;{\node[scale=2,above,#1]{\global\setbox\mybox=\hbox{#2}\copy\mybox};
      \node[scale=2,above,#1,yscale=-1,scope fading=south,opacity=0.4]{\box\mybox};}

    \nodeshadowed [at={(-5,5  )},yslant=0.05] {\Huge Ti\textcolor{orange}{\emph{k}}Z};
    \nodeshadowed [at={( 0,5.3)}] {\huge \textcolor{mintgreen}{\&}};
    \nodeshadowed [at={( 5,5  )},yslant=-0.05] {\Huge \textsc{PGF}};
    \nodeshadowed [at={( 0,2  )}] {Manual for Version \pgftypesetversion};

    \foreach \where in {-9cm,9cm}
    {\nodeshadowed [at={(\where,5cm)}] {
    \tikz \draw [green!20!black, rotate=90]
    [l-system={rule set={F -> FF-[-F+F]+[+F-F]}, axiom=F, order=4,
      step=2pt, randomize step percent=50, angle=30, randomize angle percent=5}]
    lindenmayer system;};}

    \foreach \i in {0.5,0.6,...,2}
      \fill [white,decoration=Koch snowflake,opacity=.9]
            [shift=(horizon),shift={(rand*11,rnd*7)},scale=\i]
            [double copy shadow={opacity=0.2,shadow xshift=0pt,shadow
              yshift=3*\i pt,fill=white,draw=none}]
        decorate {
          decorate {
            decorate {
              (0,0) -- ++(60:1) -- ++(-60:1) -- cycle
            }
          }
        };

  \node (left text) [text width=.5\paperwidth-2cm,below right,at={(-.5\paperwidth+1cm,-1.5cm)}]
  {
    \fontfamily{pcr}
    \def\textbraceleft{\char`\{}
    \def\textbraceright{\char`\}}
    \def\textbackslash{\char`\\}
    \begin{lstlisting}[basicstyle=\scriptsize\color{black},
                       keywordstyle=\bfseries\color{white},
                       identifierstyle=\bfseries\color{black},
                       keywords={tikzpicture,shade,fill,draw,path,node},
                       literate={-}{{-}}1]
\begin{tikzpicture}
  \coordinate (front) at (0,0);
  \coordinate (horizon) at (0,.31\paperheight);
  \coordinate (bottom) at (0,-.6\paperheight);
  \coordinate (sky) at (0,.57\paperheight);
  \coordinate (left) at (-.51\paperwidth,0);
  \coordinate (right) at (.51\paperwidth,0);

  \shade [bottom color=white,
          top color=blue!30!black!50]
              ([yshift=-5mm]horizon -|  left)
    rectangle (sky -| right);

  \shade [bottom color=black!70!green!25,
          top color=black!70!green!10]
    (front -| left) -- (horizon -| left)
    decorate [decoration=random steps] {
      -- (horizon -| right)  }
    -- (front -| right) -- cycle;

  \shade [top color=black!70!green!25,
         bottom color=black!25]
              ([yshift=-5mm-1pt]front -| left)
    rectangle ([yshift=1pt]front -| right);

  \fill [black!25]
              (bottom -| left)
    rectangle ([yshift=-5mm]front -| right);

  \def\nodeshadowed[#1]#2;{
    \node[scale=2,above,#1]{
      \global\setbox\mybox=\hbox{#2}
      \copy\mybox};
    \node[scale=2,above,#1,yscale=-1,
          scope fading=south,opacity=0.4]{\box\mybox};
  }
\end{lstlisting}
};

  \node (right text) [text width=.5\paperwidth-2cm,below right,at={(1cm,-1.5cm)}]
  {
    \fontfamily{pcr}
    \def\textbraceleft{\char`\{}
    \def\textbraceright{\char`\}}
    \def\textbackslash{\char`\\}
    \begin{lstlisting}[basicstyle=\scriptsize\color{black},
                       keywordstyle=\bfseries\color{white},
                       identifierstyle=\bfseries\color{black},
                       keywords={tikzpicture,shade,fill,draw,path,node},
                       literate={-}{{-}}1]
  \nodeshadowed [at={(-5,8  )},yslant=0.05]
    {\Huge Ti\textcolor{orange}{\emph{k}}Z};
  \nodeshadowed [at={( 0,8.3)}]
    {\huge \textcolor{green!50!black!50}{\&}};
  \nodeshadowed [at={( 5,8  )},yslant=-0.05]
    {\Huge \textsc{PGF}};
  \nodeshadowed [at={( 0,5  )}]
    {Manual for Version \pgftypesetversion};

  \foreach \where in {-9cm,9cm} {
    \nodeshadowed [at={(\where,5cm)}] { \tikz
      \draw [green!20!black, rotate=90,
             l-system={rule set={F -> FF-[-F+F]+[+F-F]},
               axiom=F, order=4,step=2pt,
               randomize step percent=50, angle=30,
               randomize angle percent=5}] l-system; }}

  \foreach \i in {0.5,0.6,...,2}
    \fill
      [white,opacity=\i/2,
       decoration=Koch snowflake,
       shift=(horizon),shift={(rand*11,rnd*7)},
       scale=\i,double copy shadow={
         opacity=0.2,shadow xshift=0pt,
         shadow yshift=3*\i pt,fill=white,draw=none}]
      decorate {
        decorate {
          decorate {
            (0,0)- ++(60:1) -- ++(-60:1) -- cycle
          } } };

   \node (left text) ...
   \node (right text) ...

   \fill [decorate,decoration={footprints,foot of=gnome},
          opacity=.5,brown]        (rand*8,-rnd*10)
     to [out=rand*180,in=rand*180] (rand*8,-rnd*10);
\end{tikzpicture}
  \end{lstlisting}
  };

  \fill [decorate,decoration=footprints,
         decoration={footprints,foot of=gnome},
         opacity=.5,brown]        (rand*8,-rnd*10)
    to [out=rand*180,in=rand*180] (rand*8,-rnd*10);
\end{tikzpicture}
\vfill
\vbox{}
\clearpage
}

{
  \vbox{}
  \vskip0pt plus 1fill
  F\"ur meinen Vater, damit er noch viele sch\"one \TeX-Graphiken
  erschaffen kann.
  \vskip1em
  \hfill\emph{Till}
  \vskip0pt plus 3fill

  \parindent=0pt
  Copyright 2007 by Till Tantau

  \medskip
  Permission is granted to copy, distribute and/or modify \emph{the documentation}
  under the terms of the \textsc{gnu} Free Documentation License, Version 1.2
  or any later version published by the Free Software Foundation;
  with no Invariant Sections, no Front-Cover Texts, and no Back-Cover Texts.
  A copy of the license is included in the section entitled \textsc{gnu}
  Free Documentation License.

  \medskip
  Permission is granted to copy, distribute and/or modify \emph{the
    code of the package} under the terms of the \textsc{gnu} Public License, Version 2
  or any later version published by the Free Software Foundation.
  A copy of the license is included in the section entitled \textsc{gnu}
  Public License.

  \medskip
  Permission is also granted to distribute and/or modify \emph{both
    the documentation and the code} under the conditions of the LaTeX
  Project Public License, either version 1.3 of this license or (at
  your option) any later version. A copy of the license is included in
  the section entitled \LaTeX\ Project Public License.

  \vbox{}
  \clearpage
}


\title{\bfseries The \tikzname\ and {\Large PGF} Packages\\
  \large Manual for version \pgfversion\\[1mm]
\large\href{http://sourceforge.net/projects/pgf}{\texttt{http://sourceforge.net/projects/pgf}}}
\author{Till Tantau\footnote{Editor of this documentation. Parts of
    this documentation have been written by other authors as indicated
    in these parts or chapters and in Section~\ref{section-authors}.}\\
  \normalsize Institut f\"ur Theoretische Informatik\\[-1mm]
  \normalsize Universit\"at zu L\"ubeck}

\maketitle

\tableofcontents

\clearpage


% Copyright 2003 by Till Tantau <tantau@cs.tu-berlin.de>.
%
% This program can be redistributed and/or modified under the terms
% of the LaTeX Project Public License Distributed from CTAN
% archives in directory macros/latex/base/lppl.txt.



\section{Introduction}

The \pgfname\ package, where ``\pgfname'' is supposed to mean ``portable
graphics format'' (or ``pretty, good, functional'' if you
prefer\dots), is a package for creating graphics in an ``inline''  
manner. It defines a number of \TeX\ commands that draw
graphics. For example, the code |\tikz \draw (0pt,0pt) -- (20pt,6pt);|
yields the line \tikz \draw (0pt,0pt) -- (20pt,6pt); and the code
|\tikz \fill[orange] (1ex,1ex) circle (1ex);| yields \tikz
\fill[orange] (1ex,1ex) circle (1ex);.

In a sense, when you use \pgfname\ you ``program'' your graphics, just
as you ``program'' your document when you use \TeX.  You get all  
the advantages of the ``\TeX-approach to typesetting'' for your 
graphics: quick creation of simple graphics, precise positioning, the
use of macros, often superior typography. You also inherit all the
disadvantages: steep learning curve, no \textsc{wysiwyg}, small
changes require a long recompilation time, and the code does not
really ``show'' how things will look like. 



\subsection{Structure of the System}

The \pgfname\ system consists of different layers:

\begin{description}
\item[System layer:] This layer provides a complete abstraction of what is
  going on ``in the driver.'' The driver is a program like |dvips| or
  |dvipdfm| that takes a |.dvi| file as input and generates a |.ps| or
  a |.pdf| file. (The |pdftex| program also counts as a driver, even
  though it does not take a |.dvi| file as input. Never mind.) Each
  driver has its own syntax for the generation of graphics, causing
  headaches to everyone who wants to create graphics in a portable
  way. \pgfname's system layer ``abstracts away'' these
  differences. For example, the system command
  |\pgfsys@lineto{10pt}{10pt}| extends the current path  to the coordinate
  $(10\mathrm{pt},10\mathrm{pt})$ of the current
  |{pgfpicture}|. Depending on whether |dvips|, 
  |dvipdfm|, or |pdftex| is used to process the document, the system
  command will be converted to different |\special| commands.
  The system layer is as ``minimalistic'' as possible since each
  additional command makes it more work to port \pgfname\ to a new
  driver.

  As a user, you will not use the system layer directly.
\item[Basic layer:]
  The basic layer provides a set of basic commands that allow
  you to produce complex graphics in a much easier manner than by using
  the system layer directly. For example,  the system layer provides
  no commands for creating circles since circles can be composed from
  the more basic B�zier curves (well, almost). However, as a user you
  will want to have a simple command to create circles
  (at least I do) instead of having to write down half a page of
  B�zier  curve  support coordinates. Thus, the basic layer provides a
  command |\pgfpathcircle| that generates the necessary curve
  coordinates for you.

  The basic layer is consists of a \emph{core}, which consists of
  several interdependent packages that can only be loaded \emph{en
    bloc,} and additional packages that extend the core by more
  special-purpose commands like node management or a plotting
  interface. For instance, the \textsc{beamer} package uses the core, 
  but not all of the additional packages of the basic layer.
\item[Frontend layer:]
  A frontend (of which there can be several) is a set of commands
  or a special syntax that makes using the basic layer easier. A
  problem with directly using the basic layer is that code written for
  this layer is often too ``verbose.'' For example, to draw a simple
  triangle, you may need as many as five commands when using the basic
  layer: One for beginning a path at the first corner of the triangle,
  one for extending the path to the second corner, one for going to
  the third, one for closing the path, and one for actually painting
  the triangle (as opposed to filling it). With the |tikz| frontend
  all this boils down to a single simple \textsc{metafont}-like
  command: 
\begin{verbatim}
\draw (0,0) -- (1,0) -- (1,1) -- cycle;
\end{verbatim}

  There are different frontends:
  \begin{itemize}
  \item
    The \tikzname\ frontend is the ``natural'' frontend for \pgfname. It gives
    you access to all features of \pgfname, but it is intended to be
    easy to use. The syntax is a mixture of \textsc{metafont} and
    \textsc{pstricks} and some ideas of myself. This frontend is
    \emph{neither} a complete \textsc{metafont} compatibility layer nor
    a \textsc{pstricks} compatibility layer and it is not intended to
    become either. 
  \item
    The |pgfpict2e| frontend reimplements the standard \LaTeX\
    |{picture}|  environment and commands like |\line| or |\vector|
    using the \pgfname\ basic layer. This layer is not really ``necessary''
    since the |pict2e.sty| package does at least as good a job at
    reimplementing the |{picture}| environment. Rather, the idea
    behind this package is to have a simple demonstration of how a
    frontend can be implemented.
  \end{itemize}

  It would be possible to implement a |pgftricks| frontend that maps
  \textsc{pstricks} commands to \pgfname\ commands. However, I have not
  done this and even if fully implemented, many things that work in
  \pstricks\ will not work, namely whenever some \pstricks\ command
  relies too heavily on PostScript trickery. Nevertheless, such a
  package might be useful in some situations.
\end{description}

As a user of \pgfname\ you will use the commands of a
frontend plus perhaps some commands of the basic layer. For this
reason, this manual explains the frontends first, then the basic
layer, and finally the system layer.



\subsection{Comparison with Other Graphics Packages}

There were two main motivations for creating \pgfname:
\begin{enumerate}
\item
  The standard \LaTeX\ |{picture}| environment is not powerful enough to
  create anything but really simple graphics. This is certainly not
  due to a lack of knowledge or imagination on the part of
  \LaTeX's designer(s). Rather, this is the price paid for the
  |{picture}| environment's portability: It works together with all
  backend drivers.
\item
  The |{pstricks}| package is certainly powerful enough to create
  any conceivable kind of graphic, but it is not portable at all. Most
  importantly, it does not work with |pdftex| nor with any other
  driver that produces anything but PostScript code.
\end{enumerate}

The \pgfname\ package is a trade-off between portability and expressive
power. It is not as portable as |{picture}| and perhaps not quite as
powerful as |{pspicture}|. However, it is more powerful than
|{picture}| and  more portable than |{pspicture}|.

\subsection{Utilities: Page Management}

The \pgfname\ package include a special subpackage called |pgfpages|,
which is used to assemble several pages into a single page. This
package is not really about creating graphics, but it is part of \pgfname\
nevertheless, mostly because its implementation uses \pgfname\ heavily.

The subpackage |pgfpages| provides commands for assembling several
``virtual pages'' into a single ``physical page.'' The idea is that
whenever \TeX\ has a page ready for ``shipout,'' |pgfpages| interrupts
this shipout and instead stores the page to be shipped out in a
special box. When enough ``virtual pages'' have been accumulated in
this way, they are scaled down and arranged on a ``physical page,''
which then \emph{really} shipped out. This mechanism allows you to
create ``two page on one page'' versions of a document directly inside
\LaTeX\ without the use of any external programs.

However, |pgfpages| can do quite a lot more than that. You can use it
to put logos and watermark on pages, print up to 16 pages on one page,
add borders to pages, and more.




\subsection{How to Read This Manual}

This manual describes both the design of the \pgfname\ system and
its usage. The organization is  very roughly according to
``user-friendliness.'' The commands and subpackages that are easiest
and most frequently used are described first, more low-level and
esoteric features are discussed later.

If you have not yet installed \pgfname, please read the installation
first. Second, it might be a good idea to read the tutorial. Finally,
you might wish to skim through the description of \tikzname. Typically, 
you will not need to read the sections on the basic layer. You will
only need to read the part on the system layer if you intend to write
your own frontend or if you wish to port \pgfname\ to a new driver.

The ``public'' commands and environments provided by the |pgf| package
are described throughout the text. In each such description, the
described command, environment or option is printed in red. Text shown
in green is optional and can be left out.



\subsection{Getting Help}

When you need help with \pgfname\ and \tikzname, please do the
following:

\begin{enumerate}
\item
  Read the manual, at least the part that has to do with your problem.
\item
  If that does not solve the problem, try having a look at the
  sourceforge development page for \pgfname\ and \tikzname\ (see the
  title of this document). Perhaps someone has already reported a
  similar problem and someone has found a solution.
\item
  On the website you will find numerous forums for getting
  help. There, you can write to help forums, file bug reports, join
  mailing lists, and so on.
\item
  Before you file a bug report, especially a bug report concerning the
  installation, make sure that this is really a bug. In particular,
  have a look at the |.log| file that results when you \TeX\ your
  files. This |.log| file should show that all the right files are
  loaded from the right directories. Nearly all installation problems
  can be resolved by looking at the |.log| file.
\item
  \emph{As a last resort} you can try to email me (the author). I do
  not mind getting emails, I simply get way too many of them. Because
  of this, I cannot guarantee that your emails will be answered timely
  or even at all. Your chances that your problem will be fixed are
  somewhat higher if you mail to the \pgfname\ mailing list
  (naturally, I read this list and answer questions when I have the
  time).
\item
  Please, do not phone me in my office. If you need a hotline, buy a
  commercial product.
\end{enumerate}




\part{Tutorials and Guidelines}

{\Large \emph{by Till Tantau}}

\bigskip
\noindent
To help you get started with \tikzname, instead of a long installation
and configuration section, this manual starts with tutorials. They
explain all the basic and some of the more advanced features of the
system, without going into all the details. This part also contains
some guidelines on how you should proceed when creating graphics using
\tikzname.

\vskip3cm

\begin{codeexample}[graphic=white,width=0pt]
\tikz \draw[thick,rounded corners=8pt]
  (0,0) -- (0,2) -- (1,3.25) -- (2,2) -- (2,0) -- (0,2) -- (2,2) -- (0,0) -- (2,0);
\end{codeexample}

% Copyright 2006 by Till Tantau
%
% This file may be distributed and/or modified
%
% 1. under the LaTeX Project Public License and/or
% 2. under the GNU Free Documentation License.
%
% See the file doc/generic/pgf/licenses/LICENSE for more details.

\section{Tutorial: A Picture for Karl's Students}

This tutorial is intended for new users of \tikzname. It
does not give an exhaustive account of all the features of \tikzname,
just of those that you are likely to use right away. 

Karl is a math and chemistry high-school teacher. He used to create
the graphics in his worksheets and exams using \LaTeX's |{picture}|
environment. While the results were acceptable, creating the graphics
often turned out to be a lengthy process. Also, there tended to be
problems with lines having slightly wrong angles and circles also
seemed to be hard to get right. Naturally, his students could not care
less whether the lines had the exact right angles and they find
Karl's exams too difficult no matter how nicely they were drawn. But
Karl was never entirely satisfied with the result.

Karl's son, who was even less satisfied with the results (he did not
have to take the exams, after all),  told Karl that he might wish
to try out a new package for creating graphics. A bit confusingly,
this package seems to have two names: First, Karl had to download and
install a package called \pgfname. Then it turns out that inside this
package there is another package called \tikzname, which is supposed to
stand for ``\tikzname\ ist \emph{kein}  Zeichenprogramm.'' Karl finds this
all a bit strange and \tikzname\ seems to indicate that the package
does not do what he needs. However, having used \textsc{gnu}
software for quite some time and ``\textsc{gnu} not being Unix,''
there seems to be hope yet. His son assures him that \tikzname's name is
intended to warn people that \tikzname\ is not a program that you can
use to draw graphics with your mouse or tablet. Rather, it is more
like a ``graphics language.''


\subsection{Problem Statement}

Karl wants to put a graphic on the next worksheet for his
students. He is currently teaching his students about sine and
cosine. What he would like to have is something that looks like this
(ideally):

\noindent
\begin{tikzpicture}
  [scale=3,line cap=round,
   % Styles
   axes/.style=,
   important line/.style={very thick},
   information text/.style={rounded corners,fill=red!10,inner sep=1ex}]

  % Local definitions
  \def\costhirty{0.8660256}

  % Colors
  \colorlet{anglecolor}{green!50!black}
  \colorlet{sincolor}{red}
  \colorlet{tancolor}{orange!80!black}
  \colorlet{coscolor}{blue}

  % The graphic
  \draw[help lines,step=0.5cm] (-1.4,-1.4) grid (1.4,1.4);

  \draw (0,0) circle [radius=1cm];

  \begin{scope}[axes]
    \draw[->] (-1.5,0) -- (1.5,0) node[right] {$x$};
    \draw[->] (0,-1.5) -- (0,1.5) node[above] {$y$};

    \foreach \x/\xtext in {-1, -.5/-\frac{1}{2}, 1}
      \draw[xshift=\x cm] (0pt,1pt) -- (0pt,-1pt) node[below,fill=white] {$\xtext$};

    \foreach \y/\ytext in {-1, -.5/-\frac{1}{2}, .5/\frac{1}{2}, 1}
      \draw[yshift=\y cm] (1pt,0pt) -- (-1pt,0pt) node[left,fill=white] {$\ytext$};
  \end{scope}

  \filldraw[fill=green!20,draw=anglecolor] (0,0) -- (3mm,0pt) arc(0:30:3mm);
  \draw (15:2mm) node[anglecolor] {$\alpha$};

  \draw[important line,sincolor]
    (30:1cm) -- node[left=1pt,fill=white] {$\sin \alpha$} +(0,-.5);

  \draw[important line,coscolor]
    (0,0) -- node[below=2pt,fill=white] {$\cos \alpha$} (\costhirty,0);

  \draw[important line,tancolor] (1,0) --
    node [right=1pt,fill=white]
    {
      $\displaystyle \tan \alpha \color{black}=
      \frac{{\color{sincolor}\sin \alpha}}{\color{coscolor}\cos \alpha}$
    } (intersection of 0,0--30:1cm and 1,0--1,1) coordinate (t);

  \draw (0,0) -- (t);

  \draw[xshift=1.85cm] node [right,text width=6cm,information text]
    {
      The {\color{anglecolor} angle $\alpha$} is $30^\circ$ in the
      example ($\pi/6$ in radians). The {\color{sincolor}sine of
        $\alpha$}, which is the height of the red line, is
      \[
      {\color{sincolor} \sin \alpha} = 1/2.
      \]
      By the Theorem of Pythagoras we have ${\color{coscolor}\cos^2 \alpha} +
      {\color{sincolor}\sin^2\alpha} =1$. Thus the length of the blue
      line, which is the {\color{coscolor}cosine of $\alpha$}, must be
      \[
      {\color{coscolor}\cos\alpha} = \sqrt{1 - 1/4} = \textstyle
      \frac{1}{2} \sqrt 3.
      \]%
      This shows that {\color{tancolor}$\tan \alpha$}, which is the
      height of the orange line, is
      \[
      {\color{tancolor}\tan\alpha} = \frac{{\color{sincolor}\sin
          \alpha}}{\color{coscolor}\cos \alpha} = 1/\sqrt 3.
      \]%
    };
\end{tikzpicture}


\subsection{Setting up the Environment}

In \tikzname, to draw a picture, at the start of the picture
you need to tell \TeX\ or \LaTeX\ that you want to start a picture. In
\LaTeX\ this is done using the environment |{tikzpicture}|, in plain
\TeX\ you just use |\tikzpicture| to start the picture and
|\endtikzpicture| to end it.

\subsubsection{Setting up the Environment in \LaTeX}

Karl, being a \LaTeX\ user, thus sets up his file as follows:

\begin{codeexample}[code only]
\documentclass{article} % say
\usepackage{tikz}
\begin{document}
We are working on
\begin{tikzpicture}
  \draw (-1.5,0) -- (1.5,0);
  \draw (0,-1.5) -- (0,1.5);
\end{tikzpicture}.
\end{document}
\end{codeexample}

When executed, that is, run via |pdflatex| or via |latex| followed by
|dvips|, the resulting will contain something that looks like this:

\begin{codeexample}[width=7cm]
We are working on
\begin{tikzpicture}
  \draw (-1.5,0) -- (1.5,0);
  \draw (0,-1.5) -- (0,1.5);
\end{tikzpicture}.
\end{codeexample}

Admittedly, not quite the whole picture, yet, but we
do have the axes established. Well, not quite, but we have the lines
that make up the axes drawn. Karl suddenly has a sinking feeling
that the picture is still some way off.

Let's have a more detailed look at the code. First, the package
|tikz| is loaded. This package is a so-called ``frontend'' to the
basic \pgfname\ system. The basic layer, which is also described in this
manual, is somewhat more, well, basic and thus harder to use. The
frontend makes things easier by providing a simpler syntax.

Inside the environment there are two |\draw| commands. They mean:
``The path, which is specified following the command up to the
semicolon, should be drawn.'' The first path is specified
as |(-1.5,0) -- (0,1.5)|, which means ``a straight line from the point
at position $(-1.5,0)$ to the point at position $(0,1.5)$.'' Here, the
positions are specified within a special coordinate system in which,
initially, one unit is 1cm.

Karl is quite pleased to note that the environment automatically
reserves enough space to encompass the picture.


\subsubsection{Setting up the Environment in Plain \TeX}

Karl's wife Gerda, who also happens to be a math teacher, is not a
\LaTeX\ user, but uses plain \TeX\ since she prefers to do things
``the old way.'' She can also use \tikzname. Instead of
|\usepackage{tikz}| she has to write |\input tikz.tex| and instead of
|\begin{tikzpicture}| she writes |\tikzpicture| and  instead of
  |\end{tikzpicture}| she writes |\endtikzpicture|.

Thus, she would use:
\begin{codeexample}[code only]
%% Plain TeX file
\input tikz.tex
\baselineskip=12pt
\hsize=6.3truein
\vsize=8.7truein
We are working on
\tikzpicture
  \draw (-1.5,0) -- (1.5,0);
  \draw (0,-1.5) -- (0,1.5);
\endtikzpicture.
\bye
\end{codeexample}

Gerda can typeset this file using either |pdftex| or |tex| together
with |dvips|. \tikzname\ will automatically discern which driver she is
using. If she wishes to use |dvipdfm| together with |tex|, she
either needs to modify the file |pgf.cfg| or can write
|\def\pgfsysdriver{pgfsys-dvipdfm.def}| somewhere \emph{before} she
inputs |tikz.tex| or |pgf.tex|.



\subsubsection{Setting up the Environment in Con\TeX t}

Karl's uncle Hans uses Con\TeX t. Like Gerda, Hans can also use
\tikzname. Instead of |\usepackage{tikz}| he says
|\usemodule[tikz]|. Instead of |\begin{tikzpicture}| he writes
  |\starttikzpicture| and  instead of |\end{tikzpicture}| he writes
|\stoptikzpicture|.

His version of the example looks like this:
\begin{codeexample}[code only]
%% ConTeXt file
\usemodule[tikz]

\starttext
  We are working on
  \starttikzpicture
    \draw (-1.5,0) -- (1.5,0);
    \draw (0,-1.5) -- (0,1.5);
  \stoptikzpicture.
\stoptext
\end{codeexample}

Hans will now typeset this file in the usual way using
|texexec| or |context|.



\subsection{Straight Path Construction}

The basic building block of all pictures in \tikzname\ is the path.
A \emph{path} is a series of straight lines and curves that are
connected (that is not the whole picture, but let us ignore the
complications for the moment). You start a path by specifying the
coordinates of the start position as a point in round brackets, as in
|(0,0)|. This is followed by a series of ``path extension
operations.'' The simplest is |--|, which we used already. It must be
followed by another coordinate and it extends the path in a straight
line to this new position. For example, if we were to turn the two
paths of the axes into one path, the following would result:

\begin{codeexample}[]
\tikz \draw (-1.5,0) -- (1.5,0) -- (0,-1.5) -- (0,1.5);
\end{codeexample}

Karl is a bit confused by the fact that there is no |{tikzpicture}|
environment, here. Instead, the little command |\tikz| is used. This
command either takes one argument (starting with an opening brace as in
|\tikz{\draw (0,0) -- (1.5,0)}|, which yields \tikz{\draw (0,0)
 --(1.5,0);}) or collects everything up to the next semicolon and
puts it inside a |{tikzpicture}| environment. As a rule of thumb, all
\tikzname\ graphic drawing commands must occur as an argument of |\tikz|
or inside a |{tikzpicture}| environment. Fortunately, the command
|\draw| will only be defined inside this environment, so there is
little chance that you will accidentally do something wrong here.



\subsection{Curved Path Construction}

The next thing Karl wants to do is to draw the circle. For this,
straight lines obviously will not do. Instead, we need some way to
draw curves. For this, \tikzname\ provides a special syntax. One or two
``control points'' are needed. The math behind them is not quite
trivial, but here is the basic idea: Suppose you are at point $x$ and
the first control point is $y$. Then the curve will start ``going in
the direction of~$y$ at~$x$,'' that is, the tangent of the curve at $x$
will point toward~$y$. Next, suppose the curve should end at $z$ and
the second support point is $w$. Then the curve will, indeed, end at
$z$ and the tangent of the curve at point $z$ will go through $w$.

Here is an example (the control points have been added for clarity):
\begin{codeexample}[]
\begin{tikzpicture}
  \filldraw [gray] (0,0) circle [radius=2pt]
                   (1,1) circle [radius=2pt]
                   (2,1) circle [radius=2pt]
                   (2,0) circle [radius=2pt];
  \draw (0,0) .. controls (1,1) and (2,1) .. (2,0);
\end{tikzpicture}
\end{codeexample}

The general syntax for extending a path in a ``curved'' way is
|.. controls| \meta{first control point} |and| \meta{second control
  point} |..| \meta{end point}. You can leave out the |and|
\meta{second control point}, which causes the first one to be used
twice.

So, Karl can now add the first half circle to the picture:

\begin{codeexample}[]
\begin{tikzpicture}
  \draw (-1.5,0) -- (1.5,0);
  \draw (0,-1.5) -- (0,1.5);
  \draw (-1,0) .. controls (-1,0.555) and (-0.555,1) .. (0,1)
               .. controls (0.555,1) and (1,0.555) .. (1,0);
\end{tikzpicture}
\end{codeexample}

Karl is happy with the result, but finds specifying circles in this
way to be extremely awkward. Fortunately, there is a much simpler way.


\subsection{Circle Path Construction}

In order to draw a circle, the path construction operation |circle| can
be used. This operation is followed by a radius in brackets as in
the following example: (Note that the previous position is used as the
\emph{center} of the circle.)

\begin{codeexample}[]
\tikz \draw (0,0) circle [radius=10pt];
\end{codeexample}

You can also append an ellipse to the path using the |ellipse|
operation. Instead of a single radius you can specify two of them:

\begin{codeexample}[]
\tikz \draw (0,0) ellipse [x radius=20pt, y radius=10pt];
\end{codeexample}

To draw an ellipse whose axes are not horizontal and vertical, but
point in an arbitrary direction (a ``turned ellipse'' like \tikz
\draw[rotate=30] (0,0) ellipse [x radius=6pt, y radius=3pt];) you can use
transformations, which are explained later. The code for the little
ellipse is |\tikz \draw[rotate=30] (0,0) ellipse [x radius=6pt, y radius=3pt];|, by
the way.

So, returning to Karl's problem, he can write
|\draw (0,0) circle [radius=1cm];| to draw the circle:

\begin{codeexample}[]
\begin{tikzpicture}
  \draw (-1.5,0) -- (1.5,0);
  \draw (0,-1.5) -- (0,1.5);
  \draw (0,0) circle [radius=1cm];
\end{tikzpicture}
\end{codeexample}


At this point, Karl is a bit alarmed that the circle is so small when
he wants the final picture to be much bigger. He is pleased to learn
that \tikzname\ has powerful transformation options and scaling
everything by a factor of three is very easy. But let us leave the
size as it is for the moment to save some space.




\subsection{Rectangle Path Construction}

The next things we would like to have is the grid in the background.
There are several ways to produce it. For example, one might draw lots of
rectangles. Since rectangles are so common, there is a special syntax
for them: To add a rectangle to the current path, use the |rectangle|
path construction operation. This operation should be followed by another
coordinate and will append a rectangle to the path such that the
previous coordinate and the next coordinates are corners of the
rectangle. So, let us add two rectangles to the picture:

\begin{codeexample}[]
\begin{tikzpicture}
  \draw (-1.5,0) -- (1.5,0);
  \draw (0,-1.5) -- (0,1.5);
  \draw (0,0) circle [radius=1cm];
  \draw (0,0) rectangle (0.5,0.5);
  \draw (-0.5,-0.5) rectangle (-1,-1);
\end{tikzpicture}
\end{codeexample}

While this may be nice in other situations, this is not really leading
anywhere with Karl's problem: First, we would need an awful lot of
these rectangles and then there is the border that is not ``closed.''

So, Karl is about to resort to simply drawing four vertical and four
horizontal lines using the nice |\draw| command, when he learns that
there is a |grid| path construction operation.



\subsection{Grid Path Construction}

The |grid| path operation adds a grid to the current path. It will add
lines making up a grid that fills the rectangle whose one corner is
the current point and whose other corner is the point following the
|grid| operation. For example, the code
|\tikz \draw[step=2pt] (0,0) grid (10pt,10pt);| produces \tikz
\draw[step=2pt] (0,0) grid (10pt,10pt);. Note how the optional
argument for |\draw| can be used to specify a grid width (there are
also |xstep| and |ystep| to define the steppings independently). As
Karl will learn soon, there are \emph{lots} of things that can be
influenced using such options.

For Karl, the following code could be used:

\begin{codeexample}[]
\begin{tikzpicture}
  \draw (-1.5,0) -- (1.5,0);
  \draw (0,-1.5) -- (0,1.5);
  \draw (0,0) circle [radius=1cm];
  \draw[step=.5cm] (-1.4,-1.4) grid (1.4,1.4);
\end{tikzpicture}
\end{codeexample}

Having another look at the desired picture, Karl notices that it would
be nice for the grid to be more subdued. (His son told him that grids
tend to be distracting if they are not subdued.) To subdue the grid,
Karl adds two more options to the |\draw| command that draws the
grid. First, he uses the color |gray| for the grid lines. Second, he
reduces the line width to |very thin|. Finally, he swaps the ordering
of the commands so that the grid is drawn first and everything else on
top.

\begin{codeexample}[]
\begin{tikzpicture}
  \draw[step=.5cm,gray,very thin] (-1.4,-1.4) grid (1.4,1.4);
  \draw (-1.5,0) -- (1.5,0);
  \draw (0,-1.5) -- (0,1.5);
  \draw (0,0) circle [radius=1cm];
\end{tikzpicture}
\end{codeexample}


\subsection{Adding a Touch of  Style}

Instead of the options |gray,very thin| Karl could also have
said |help lines|. \emph{Styles} are predefined sets of options
that can be used to organize how a graphic is drawn. By saying
|help lines| you say ``use the style that I (or someone else)
has set for drawing help lines.'' If Karl decides, at some later
point, that grids should be drawn, say, using the color |blue!50|
instead of |gray|, he could provide the following option somewhere:
\begin{codeexample}[code only]
help lines/.style={color=blue!50,very thin}
\end{codeexample}
The effect of this ``style setter'' is that in the current
scope or environment the |help lines| option has the same effect as
|color=blue!50,very thin|.

Using styles makes your graphics code more flexible. You can
change the way things look easily in a consistent manner.
Normally, styles are defined at the beginning of a picture. However,
you may sometimes wish to define a style globally, so that all
pictures of your document can use this style. Then you can easily
change the way all graphics look by changing this one style. In this
situation you can use the |\tikzset| command at the beginning of the
document as in
\begin{codeexample}[code only]
\tikzset{help lines/.style=very thin}
\end{codeexample}

To build a hierarchy of styles you can have one style use
another. So in order to define a style |Karl's grid| that is based on
the |grid| style Karl could say
\begin{codeexample}[code only]
\tikzset{Karl's grid/.style={help lines,color=blue!50}}
...
\draw[Karl's grid] (0,0) grid (5,5);
\end{codeexample}

Styles are made even more powerful by parametrization. This means
that, like other options, styles can also be used with a
parameter. For instance, Karl could parameterize his grid so that, by
default, it is blue, but he could also use another color.

\begin{codeexample}[code only]
\begin{tikzpicture}
  [Karl's grid/.style  ={help lines,color=#1!50},
   Karl's grid/.default=blue]

  \draw[Karl's grid]     (0,0) grid (1.5,2);
  \draw[Karl's grid=red] (2,0) grid (3.5,2);
\end{tikzpicture}
\end{codeexample}


\subsection{Drawing Options}

Karl wonders what other options there are that influence how a path is
drawn. He saw already that the |color=|\meta{color} option can be used
to set the line's color. The option |draw=|\meta{color} does nearly
the same, only it sets the color for the lines only and a different
color can be used for filling (Karl will need this when he fills the
arc for the angle).

He saw that the style |very thin| yields very thin lines. Karl is not
really surprised by this and neither is he surprised to learn that |thin|
yields thin lines,  |thick| yields thick lines, |very thick| yields
very thick lines, |ultra thick| yields really, really thick lines and
|ultra thin| yields lines that are so thin that low-resolution printers
and displays will have trouble showing them. He wonders what gives
lines of ``normal'' thickness. It turns out that |thin| is the correct
choice, since it gives the same thickness as \TeX's |\hrule|
command. Nevertheless, Karl would like to know whether there is 
anything ``in the middle'' between |thin| and |thick|. There is:
|semithick|.

Another useful thing one can do with lines is to dash or dot them. For
this, the two styles |dashed| and |dotted| can be used, yielding
\tikz[baseline] \draw[dashed] (0,.5ex) -- ++(2em,0pt); and
\tikz[baseline] \draw[dotted] (0,.5ex) 
-- ++(2em,0pt);. Both options also exist in a loose and a dense
version, called |loosely dashed|, |densely dashed|, |loosely dotted|,
and |densely dotted|. If he really, really  needs to, Karl can also
define much more complex dashing patterns with the |dash pattern|
option, but his son insists that dashing is to be used with utmost
care and mostly distracts. Karl's son claims that complicated dashing
patterns are evil. Karl's students do not care about dashing patterns.



\subsection{Arc Path Construction}

Our next obstacle is to draw the arc for the angle. For this, the
|arc| path construction operation is useful, which draws part of a
circle or ellipse. This |arc| operation is followed by options in
brackets that specify the arc. An example would be \texttt{arc[start
  angle=10, end angle=80, radius=10pt]}, which means exactly what it
says. Karl obviously
needs an arc from $0^\circ$ to $30^\circ$. The radius should be
something relatively small, perhaps around one third of the circle's
radius. When one uses the arc path construction operation, the
specified arc will be added with its starting point at the current
position. So, we first have to ``get there.''

\begin{codeexample}[]
\begin{tikzpicture}
  \draw[step=.5cm,gray,very thin] (-1.4,-1.4) grid (1.4,1.4);
  \draw (-1.5,0) -- (1.5,0);
  \draw (0,-1.5) -- (0,1.5);
  \draw (0,0) circle [radius=1cm];
  \draw (3mm,0mm) arc [start angle=0, end angle=30, radius=3mm];
\end{tikzpicture}
\end{codeexample}

Karl thinks this is really a bit small and he cannot continue unless
he learns how to do scaling. For this, he can add the |[scale=3]|
option. He could add this option to each |\draw| command, but that
would be awkward. Instead, he adds it to the whole environment, which
causes this option to apply to everything within.

\begin{codeexample}[]
\begin{tikzpicture}[scale=3]
  \draw[step=.5cm,gray,very thin] (-1.4,-1.4) grid (1.4,1.4);
  \draw (-1.5,0) -- (1.5,0);
  \draw (0,-1.5) -- (0,1.5);
  \draw (0,0) circle [radius=1cm];
  \draw (3mm,0mm) arc [start angle=0, end angle=30, radius=3mm];
\end{tikzpicture}
\end{codeexample}

As for circles, you can specify ``two'' radii in order to get an
elliptical arc.

\begin{codeexample}[]
  \tikz \draw (0,0) 
    arc [start angle=0, end angle=315, 
         x radius=1.75cm, y radius=1cm];
\end{codeexample}


\subsection{Clipping a Path}

In order to save space in this manual, it would be nice to clip Karl's
graphics a bit so that we can focus on the ``interesting''
parts. Clipping is pretty easy in \tikzname. You can use the |\clip|
command to clip all subsequent drawing. It works like |\draw|, only it
does not draw anything, but uses the given path to clip everything
subsequently.

\begin{codeexample}[]
\begin{tikzpicture}[scale=3]
  \clip (-0.1,-0.2) rectangle (1.1,0.75);
  \draw[step=.5cm,gray,very thin] (-1.4,-1.4) grid (1.4,1.4);
  \draw (-1.5,0) -- (1.5,0);
  \draw (0,-1.5) -- (0,1.5);
  \draw (0,0) circle [radius=1cm];
  \draw (3mm,0mm) arc [start angle=0, end angle=30, radius=3mm];
\end{tikzpicture}
\end{codeexample}

You can also do both at the same time: Draw \emph{and} clip a
path. For this, use the |\draw| command and add the |clip|
option. (This is not the whole picture: You can also use the |\clip|
command and add the |draw| option. Well, that is also not the whole
picture: In reality, |\draw| is just a shorthand for |\path[draw]|
and |\clip| is a shorthand for |\path[clip]| and you could also say
|\path[draw,clip]|.) Here is an example:

\begin{codeexample}[]
\begin{tikzpicture}[scale=3]
  \clip[draw] (0.5,0.5) circle (.6cm);
  \draw[step=.5cm,gray,very thin] (-1.4,-1.4) grid (1.4,1.4);
  \draw (-1.5,0) -- (1.5,0);
  \draw (0,-1.5) -- (0,1.5);
  \draw (0,0) circle [radius=1cm];
  \draw (3mm,0mm) arc [start angle=0, end angle=30, radius=3mm];
\end{tikzpicture}
\end{codeexample}


\subsection{Parabola and Sine Path Construction}

Although Karl does not need them for his picture, he is pleased to
learn that there are |parabola| and |sin| and |cos| path operations for
adding parabolas and sine and cosine curves to the current path. For the
|parabola| operation, the current point will lie on the parabola as
well as the point given after the parabola operation. Consider
the following example:

\begin{codeexample}[]
\tikz \draw (0,0) rectangle (1,1)  (0,0) parabola (1,1);
\end{codeexample}

It is also possible to place the bend somewhere else:

\begin{codeexample}[]
\tikz \draw[x=1pt,y=1pt] (0,0) parabola bend (4,16) (6,12);
\end{codeexample}

The operations |sin| and |cos| add a sine or cosine curve in the interval
$[0,\pi/2]$ such that the previous current point is at the start of
the curve and the curve ends at the given end point. Here are two
examples:
\begin{codeexample}[]
A sine \tikz \draw[x=1ex,y=1ex] (0,0) sin (1.57,1); curve.
\end{codeexample}

\begin{codeexample}[]
\tikz \draw[x=1.57ex,y=1ex] (0,0) sin (1,1) cos (2,0) sin (3,-1) cos (4,0)
                            (0,1) cos (1,0) sin (2,-1) cos (3,0) sin (4,1);
\end{codeexample}



\subsection{Filling and Drawing}

Returning to the picture, Karl now wants the angle to be ``filled''
with a very light green. For this he uses |\fill| instead of
|\draw|. Here is what Karl does:

\begin{codeexample}[]
\begin{tikzpicture}[scale=3]
  \clip (-0.1,-0.2) rectangle (1.1,0.75);
  \draw[step=.5cm,gray,very thin] (-1.4,-1.4) grid (1.4,1.4);
  \draw (-1.5,0) -- (1.5,0);
  \draw (0,-1.5) -- (0,1.5);
  \draw (0,0) circle [radius=1cm];
  \fill[green!20!white] (0,0) -- (3mm,0mm)
    arc [start angle=0, end angle=30, radius=3mm] -- (0,0);
\end{tikzpicture}
\end{codeexample}

The color |green!20!white| means 20\% green and 80\% white mixed
together. Such color expression are possible since \tikzname\ uses Uwe
Kern's |xcolor| package, see the documentation of that package for
details on color expressions.

What would have happened, if Karl had not ``closed'' the path using
|--(0,0)| at the end? In this case, the path is closed automatically,
so this could have been omitted. Indeed, it would even have been
better to write the following, instead:
\begin{codeexample}[code only]
  \fill[green!20!white] (0,0) -- (3mm,0mm)
    arc [start angle=0, end angle=30, radius=3mm] -- cycle;
\end{codeexample}
The |--cycle| causes the current path to be closed (actually the
current part of the current path) by smoothly joining the first and
last point. To appreciate the difference, consider the following
example:

\begin{codeexample}[]
\begin{tikzpicture}[line width=5pt]
  \draw (0,0) -- (1,0) -- (1,1) -- (0,0);
  \draw (2,0) -- (3,0) -- (3,1) -- cycle;
  \useasboundingbox (0,1.5); % make bounding box higher
\end{tikzpicture}
\end{codeexample}

You can also fill and draw a path at the same time using the
|\filldraw| command. This will first draw the path, then fill it. This
may not seem too useful, but you can specify different colors to be
used for filling and for stroking. These are specified as optional
arguments like this:

\begin{codeexample}[]
\begin{tikzpicture}[scale=3]
  \clip (-0.1,-0.2) rectangle (1.1,0.75);
  \draw[step=.5cm,gray,very thin] (-1.4,-1.4) grid (1.4,1.4);
  \draw (-1.5,0) -- (1.5,0);
  \draw (0,-1.5) -- (0,1.5);
  \draw (0,0) circle [radius=1cm];
  \filldraw[fill=green!20!white, draw=green!50!black] (0,0) -- (3mm,0mm)
    arc [start angle=0, end angle=30, radius=3mm] -- cycle;
\end{tikzpicture}
\end{codeexample}



\subsection{Shading}

Karl briefly considers the possibility of making the angle ``more
fancy'' by \emph{shading} it. Instead of filling the area with a uniform
color, a smooth transition between different colors is used. For this,
|\shade| and |\shadedraw|, for shading and drawing at the same time,
can be used:

\begin{codeexample}[]
  \tikz \shade (0,0) rectangle (2,1)  (3,0.5) circle (.5cm);
\end{codeexample}
The default shading is a smooth transition from gray to white. To
specify different colors, you can use options:

\begin{codeexample}[]
\begin{tikzpicture}[rounded corners,ultra thick]
  \shade[top color=yellow,bottom color=black] (0,0) rectangle +(2,1);
  \shade[left color=yellow,right color=black] (3,0) rectangle +(2,1);
  \shadedraw[inner color=yellow,outer color=black,draw=yellow] (6,0) rectangle +(2,1);
  \shade[ball color=green] (9,.5) circle (.5cm);
\end{tikzpicture}
\end{codeexample}

For Karl, the following might be appropriate:

\begin{codeexample}[]
\begin{tikzpicture}[scale=3]
  \clip (-0.1,-0.2) rectangle (1.1,0.75);
  \draw[step=.5cm,gray,very thin] (-1.4,-1.4) grid (1.4,1.4);
  \draw (-1.5,0) -- (1.5,0);
  \draw (0,-1.5) -- (0,1.5);
  \draw (0,0) circle [radius=1cm];
  \shadedraw[left color=gray,right color=green, draw=green!50!black]
    (0,0) -- (3mm,0mm)
    arc [start angle=0, end angle=30, radius=3mm] -- cycle;
\end{tikzpicture}
\end{codeexample}

However, he wisely decides that shadings usually only distract without
adding anything to the picture.


\subsection{Specifying Coordinates}

Karl now wants to add the sine and cosine lines. He knows already that
he can use the |color=| option to set the lines' colors. So, what is
the best way to specify the coordinates?

There are different ways of specifying coordinates. The easiest way is
to say something like |(10pt,2cm)|. This means 10pt in $x$-direction
and 2cm in $y$-directions. Alternatively, you can also leave out the
units as in |(1,2)|, which means ``one times the current $x$-vector
plus twice the current $y$-vector.'' These vectors default to 1cm in
the $x$-direction and 1cm in the $y$-direction, respectively.

In order to specify points in polar coordinates, use the notation
|(30:1cm)|, which means 1cm in direction 30 degree. This is obviously
quite useful to ``get to the point $(\cos 30^\circ,\sin 30^\circ)$ on
the circle.''

You can add a single |+| sign in front of a coordinate or two of
them as in |+(1cm,0cm)| or |++(0cm,2cm)|. Such coordinates are interpreted
differently: The first form means ``1cm upwards from the previous
specified position'' and the second means ``2cm to the right of the
previous specified position, making this the new specified position.''
For example, we can draw the sine line as follows:

\begin{codeexample}[]
\begin{tikzpicture}[scale=3]
  \clip (-0.1,-0.2) rectangle (1.1,0.75);
  \draw[step=.5cm,gray,very thin] (-1.4,-1.4) grid (1.4,1.4);
  \draw (-1.5,0) -- (1.5,0);
  \draw (0,-1.5) -- (0,1.5);
  \draw (0,0) circle [radius=1cm];
  \filldraw[fill=green!20,draw=green!50!black] (0,0) -- (3mm,0mm)
      arc [start angle=0, end angle=30, radius=3mm] -- cycle;
  \draw[red,very thick] (30:1cm) -- +(0,-0.5);
\end{tikzpicture}
\end{codeexample}

Karl used the fact $\sin 30^\circ = 1/2$. However, he very much
doubts that his students know this, so it would be nice to have a way
of specifying ``the point straight down from |(30:1cm)| that lies on
the $x$-axis.'' This is, indeed, possible using a special syntax: Karl
can write \verb!(30:1cm |- 0,0)!. In general, the meaning of
|(|\meta{p}\verb! |- !\meta{q}|)| is ``the intersection of a vertical
line through $p$ and a horizontal line through $q$.''

Next, let us draw the cosine line. One way would be to say
\verb!(30:1cm |- 0,0) -- (0,0)!. Another way is the following: we
``continue'' from where the sine ends:

\begin{codeexample}[]
\begin{tikzpicture}[scale=3]
  \clip (-0.1,-0.2) rectangle (1.1,0.75);
  \draw[step=.5cm,gray,very thin] (-1.4,-1.4) grid (1.4,1.4);
  \draw (-1.5,0) -- (1.5,0);
  \draw (0,-1.5) -- (0,1.5);
  \draw (0,0) circle [radius=1cm];
  \filldraw[fill=green!20,draw=green!50!black] (0,0) -- (3mm,0mm)
      arc [start angle=0, end angle=30, radius=3mm] -- cycle;
  \draw[red,very thick]  (30:1cm) -- +(0,-0.5);
  \draw[blue,very thick] (30:1cm) ++(0,-0.5) -- (0,0);
\end{tikzpicture}
\end{codeexample}

Note that there is no |--| between |(30:1cm)| and |++(0,-0.5)|. In
detail, this path is interpreted as follows: ``First, the |(30:1cm)|
tells me to move by pen to $(\cos 30^\circ,1/2)$. Next, there comes
another coordinate specification, so I move my pen there without drawing
anything. This new point is half a unit down from the last position,
thus it is at $(\cos 30^\circ,0)$. Finally, I move the pen to the
origin, but this time drawing something (because of the |--|).''

To appreciate the difference between |+| and |++| consider the
following example:

\begin{codeexample}[]
\begin{tikzpicture}
  \def\rectanglepath{-- ++(1cm,0cm)  -- ++(0cm,1cm)  -- ++(-1cm,0cm) -- cycle}
  \draw (0,0) \rectanglepath;
  \draw (1.5,0) \rectanglepath;
\end{tikzpicture}
\end{codeexample}

By comparison, when using a single |+|, the coordinates are different:

\begin{codeexample}[]
\begin{tikzpicture}
  \def\rectanglepath{-- +(1cm,0cm)  -- +(1cm,1cm)  -- +(0cm,1cm) -- cycle}
  \draw (0,0) \rectanglepath;
  \draw (1.5,0) \rectanglepath;
\end{tikzpicture}
\end{codeexample}


Naturally, all of this could have been written more clearly and more
economically like this (either with a single of a double |+|):
\begin{codeexample}[]
\tikz \draw (0,0) rectangle +(1,1)  (1.5,0) rectangle +(1,1);
\end{codeexample}


\subsection{Intersecting Paths}

Karl is left with the line for $\tan \alpha$, which seems difficult to
specify using transformations and polar coordinates. The first -- and
easiest -- thing he can do is so simply use the coordinate
|(1,{tan(30)})| since \tikzname's math engine knows how to compute
things like |tan(30)|. Note the added braces since, otherwise,
\tikzname's parser would think that the first closing parenthesis ends
the coordinate (in general, you need to add braces around components
of coordinates when these components contain parentheses). 

Karl can, however, also use a more elaborate, but also more
``geometric'' way of computing the length of the orange line: He can
specify intersections of paths as coordinates. The line for $\tan
\alpha$ starts at $(1,0)$ 
and goes upward to a point that is at the intersection of a line going
``up'' and a line going from the origin through |(30:1cm)|. Such
computations are made available by the |intersections| library.

What Karl must do is to create two ``invisible'' paths that intersect
at the position of interest. Creating paths that are not otherwise
seen can be done using the |\path| command without any options like
|draw| or |fill|. Then, Karl can add the |name path| option to the
path for later reference. Once the paths have been constructed, Karl
can use the |name intersections| to assign names to the coordinate for
later reference.

\begin{codeexample}[code only]
\path [name path=upward line] (1,0) -- (1,1);
\path [name path=sloped line] (0,0) -- (30:1.5cm); % a bit longer, so that there is an intersection

\draw [name intersections={of=upward line and sloped line, by=x}]
  [very thick,orange] (1,0) -- (x);
\end{codeexample}


\subsection{Adding Arrow Tips}

Karl now wants to add the little arrow tips at the end of the axes. He has
noticed that in many plots, even in scientific journals, these arrow tips
seem to be missing, presumably because the generating programs cannot
produce them. Karl thinks arrow tips belong at the end of axes. His
son agrees. His students do not care about arrow tips.

It turns out that adding arrow tips is pretty easy: Karl adds the option
|->| to the drawing commands for the axes:

\begin{codeexample}[]
\begin{tikzpicture}[scale=3]
  \clip (-0.1,-0.2) rectangle (1.1,1.51);
  \draw[step=.5cm,gray,very thin] (-1.4,-1.4) grid (1.4,1.4);
  \draw[->] (-1.5,0) -- (1.5,0);
  \draw[->] (0,-1.5) -- (0,1.5);
  \draw (0,0) circle [radius=1cm];
  \filldraw[fill=green!20,draw=green!50!black] (0,0) -- (3mm,0mm)
        arc [start angle=0, end angle=30, radius=3mm] -- cycle;
  \draw[red,very thick]    (30:1cm) -- +(0,-0.5);
  \draw[blue,very thick]   (30:1cm) ++(0,-0.5) -- (0,0);

  \path [name path=upward line] (1,0) -- (1,1);
  \path [name path=sloped line] (0,0) -- (30:1.5cm);
  \draw [name intersections={of=upward line and sloped line, by=x}]
        [very thick,orange] (1,0) -- (x);
\end{tikzpicture}
\end{codeexample}

If Karl had used the option |<-| instead of |->|, arrow tips would
have been put at the beginning of the path. The option |<->| puts
arrow tips at both ends of the path.

There are certain restrictions to the kind of paths to which arrow tips
can be added. As a rule of thumb, you can add arrow tips only to a
single open ``line.'' For example, you cannot add tips to,
say, a rectangle or a circle. However, you can add arrow
tips to curved paths and to paths that have several segments, as in
the following examples:

\begin{codeexample}[]
\begin{tikzpicture}
  \draw [<->] (0,0) arc [start angle=180, end angle=30, radius=10pt];
  \draw [<->] (1,0) -- (1.5cm,10pt) -- (2cm,0pt) -- (2.5cm,10pt);
\end{tikzpicture}
\end{codeexample}

Karl has a more detailed look at the arrow that \tikzname\ puts at the
end. It looks like this when he zooms it: \tikz[baseline]
\draw[->,line width=1pt] (0pt,.5ex) -- ++(10pt,0pt);. The shape seems
vaguely familiar and, indeed, this is exactly the end of \TeX's
standard arrow used in something like $f\colon A \to B$.

Karl likes the arrow, especially since it is not ``as thick'' as the
arrows offered by many other packages. However, he expects that,
sometimes, he might need to use some other kinds of arrow.
To do so, Karl can say |>=|\meta{kind of end arrow tip}, where
\meta{kind of end arrow tip} is a special arrow tip specification. For
example, if Karl says |>=Stealth|, then he tells \tikzname\
that he would like  ``stealth-fighter-like'' arrow tips:

\begin{codeexample}[]
\begin{tikzpicture}[>=Stealth]
  \draw [->] (0,0) arc [start angle=180, end angle=30, radius=10pt];
  \draw [<<-,very thick] (1,0) -- (1.5cm,10pt) -- (2cm,0pt) -- (2.5cm,10pt);
\end{tikzpicture}
\end{codeexample}%>>

Karl wonders whether such a military name for the arrow type is really
necessary. He is not really mollified when his son tells him that
Microsoft's PowerPoint uses the same name. He decides to have his
students discuss this at some point.

In addition to |Stealth| there are several other predefined kinds of
arrow tips Karl can choose from, see Section~\ref{section-arrows}. Furthermore,
he can define arrows types himself, if he needs new ones.


\subsection{Scoping}

Karl saw already that there are numerous graphic options that affect how
paths are rendered. Often, he would like to apply certain options to
a whole set of graphic commands. For example, Karl might wish to draw
three paths using a |thick| pen, but would like everything else to
be drawn ``normally.''

If Karl wishes to set a certain graphic option for the whole picture,
he can simply pass this option to the |\tikz| command or to the
|{tikzpicture}| environment (Gerda would pass the options to
|\tikzpicture| and Hans passes them to |\starttikzpicture|). However,
if Karl wants to apply graphic options to a local group, he put these
commands inside a |{scope}| environment (Gerda uses |\scope| and
|\endscope|, Hans uses |\startscope| and |\stopscope|). This
environment takes graphic options as an optional argument and these
options apply to everything inside the scope, but not to anything outside.

Here is an example:

\begin{codeexample}[]
\begin{tikzpicture}[ultra thick]
  \draw (0,0) -- (0,1);
  \begin{scope}[thin]
    \draw (1,0) -- (1,1);
    \draw (2,0) -- (2,1);
  \end{scope}
  \draw (3,0) -- (3,1);
\end{tikzpicture}
\end{codeexample}

Scoping has another interesting effect: Any changes to the clipping
area are local to the scope. Thus, if you say |\clip| somewhere inside
a scope, the effect of the |\clip| command ends at the end of the
scope. This is useful since there is no other way of ``enlarging'' the
clipping area.

Karl has also already seen that giving options to commands like
|\draw| apply only to that command. In turns out that the situation is
slightly more complex. First, options to a command like |\draw| are
not really options to the command, but they are ``path options'' and
can be given anywhere on the path. So, instead of
|\draw[thin] (0,0) -- (1,0);| one can also write
|\draw (0,0) [thin] -- (1,0);| or |\draw (0,0) -- (1,0) [thin];|; all
of these have the same effect. This might seem strange since in the
last case, it would appear that the |thin| should take effect only
``after'' the line from $(0,0)$ to $(1,0)$ has been drawn. However,
most graphic options only apply to the whole path. Indeed, if you say
both |thin| and |thick| on the same path, the last option given will
``win.''

When reading the above, Karl notices that only ``most'' graphic
options apply to the whole path. Indeed, all transformation options do
\emph{not} apply to the whole path, but only to ``everything following
them on the path.'' We will have a more detailed look at this in a
moment. Nevertheless, all options given during a path construction
apply only to this path.



\subsection{Transformations}

When you specify a  coordinate like |(1cm,1cm)|, where is that
coordinate placed on the page? To determine the position, \tikzname,
\TeX, and \textsc{pdf} or PostScript all apply certain transformations
to the given coordinate in order to determine the final position on
the page.

\tikzname\ provides numerous options that allow you to transform
coordinates in \tikzname's private coordinate system. For example, the
|xshift| option allows you to shift all subsequent points by a certain
amount:

\begin{codeexample}[]
\tikz \draw (0,0) -- (0,0.5) [xshift=2pt] (0,0) -- (0,0.5);
\end{codeexample}

It is important to note that you can change transformation ``in the
middle of a path,'' a feature that is not supported by \pdf\
or PostScript. The reason is that \tikzname\ keeps track of its own
transformation matrix.

Here is a more complicated example:
\begin{codeexample}[]
\begin{tikzpicture}[even odd rule,rounded corners=2pt,x=10pt,y=10pt]
  \filldraw[fill=yellow!80!black] (0,0)   rectangle (1,1)
        [xshift=5pt,yshift=5pt]   (0,0)   rectangle (1,1)
                    [rotate=30]   (-1,-1) rectangle (2,2);
\end{tikzpicture}
\end{codeexample}

The most useful transformations are |xshift| and |yshift| for
shifting, |shift| for shifting to a given point as in |shift={(1,0)}|
or |shift={+(0,0)}| (the braces are necessary so that \TeX\ does not
mistake the comma for separating options), |rotate| for rotating by a
certain angle (there is also a |rotate around| for rotating around a
given point), |scale| for scaling by a certain factor, |xscale| and
|yscale| for scaling only in the $x$- or $y$-direction (|xscale=-1| is
a flip), and |xslant| and |yslant| for slanting. If these
transformation and those that I have not mentioned are not
sufficient,  the |cm| option allows you to apply an arbitrary
transformation matrix. Karl's students, by the way, do not know what a
transformation matrix is.



\subsection{Repeating Things: For-Loops}

Karl's next aim is to add little ticks on the axes at positions $-1$,
$-1/2$, $1/2$, and $1$. For this, it would be nice to use some kind of
``loop,'' especially since he wishes to do the same thing at each of
these positions. There are different packages for doing this. \LaTeX\
has its own internal command for this, |pstricks| comes along with the
powerful |\multido| command. All of these can be used together with
\tikzname, so if you are familiar with them, feel free to
use them. \tikzname\ introduces yet another command, called |\foreach|,
which I introduced since I could never remember the syntax of the other
packages. |\foreach| is defined in the package |pgffor| and can be used
independently \tikzname, but \tikzname\ includes it automatically.

In its basic form, the |\foreach| command is easy to use:
\begin{codeexample}[]
\foreach \x in {1,2,3} {$x =\x$, }
\end{codeexample}

The general syntax is |\foreach| \meta{variable}| in {|\meta{list of
    values}|} |\meta{commands}. Inside the \meta{commands}, the
\meta{variable} will be assigned to the different values. If the
\meta{commands} do not start with a brace, everything up to the
next semicolon is used as \meta{commands}.

For Karl and the ticks on the axes, he could use the following code:

\begin{codeexample}[]
\begin{tikzpicture}[scale=3]
  \clip (-0.1,-0.2) rectangle (1.1,1.51);
  \draw[step=.5cm,gray,very thin] (-1.4,-1.4) grid (1.4,1.4);
  \filldraw[fill=green!20,draw=green!50!black] (0,0) -- (3mm,0mm)
      arc [start angle=0, end angle=30, radius=3mm] -- cycle;
  \draw[->] (-1.5,0) -- (1.5,0);
  \draw[->] (0,-1.5) -- (0,1.5);
  \draw (0,0) circle [radius=1cm];

  \foreach \x in {-1cm,-0.5cm,1cm}
    \draw (\x,-1pt) -- (\x,1pt);
  \foreach \y in {-1cm,-0.5cm,0.5cm,1cm}
    \draw (-1pt,\y) -- (1pt,\y);
\end{tikzpicture}
\end{codeexample}

As a matter of fact, there are many different ways of creating the
ticks. For example, Karl could have put the |\draw ...;| inside curly
braces. He could also have used, say,
\begin{codeexample}[code only]
\foreach \x in {-1,-0.5,1}
  \draw[xshift=\x cm] (0pt,-1pt) -- (0pt,1pt);
\end{codeexample}

Karl is curious what would happen in a more complicated situation
where there are, say, 20 ticks. It seems bothersome to explicitly
mention all these numbers in the set for |\foreach|. Indeed, it is
possible to use |...| inside the |\foreach| statement to iterate over
a large number of values (which must, however, be dimensionless
real numbers) as in the following example:

\begin{codeexample}[]
\tikz \foreach \x in {1,...,10}
        \draw (\x,0) circle (0.4cm);
\end{codeexample}

If you provide \emph{two} numbers before the |...|, the |\foreach|
statement will use their difference for the stepping:

\begin{codeexample}[]
\tikz \foreach \x in {-1,-0.5,...,1}
       \draw (\x cm,-1pt) -- (\x cm,1pt);
\end{codeexample}

We can also nest loops to create interesting effects:

\begin{codeexample}[]
\begin{tikzpicture}
  \foreach \x in {1,2,...,5,7,8,...,12}
    \foreach \y in {1,...,5}
    {
      \draw (\x,\y) +(-.5,-.5) rectangle ++(.5,.5);
      \draw (\x,\y) node{\x,\y};
    }
\end{tikzpicture}
\end{codeexample}

The |\foreach| statement can do even trickier stuff, but the above
gives the idea.




\subsection{Adding Text}

Karl is, by now, quite satisfied with the picture. However, the most
important parts, namely the labels, are still missing!

\tikzname\ offers an easy-to-use and powerful system for adding text and,
more generally, complex shapes to a picture at specific positions. The
basic idea is the following: When \tikzname\ is constructing a path and
encounters the keyword |node| in the middle of a path, it
reads a \emph{node specification}. The keyword |node| is typically
followed by some options and then some text between curly braces. This
text is put inside a normal \TeX\ box (if the node specification
directly follows a coordinate, which is usually the case, \tikzname\ is
able to perform some magic so that it is even possible to use verbatim
text inside the boxes) and then placed at the current position, that
is, at the last specified position (possibly shifted a bit, according
to the given options). However, all nodes are drawn only after the
path has been completely drawn/filled/shaded/clipped/whatever.

\begin{codeexample}[]
\begin{tikzpicture}
  \draw (0,0) rectangle (2,2);
  \draw (0.5,0.5) node [fill=yellow!80!black]
                       {Text at \verb!node 1!}
     -- (1.5,1.5) node {Text at \verb!node 2!};
\end{tikzpicture}
\end{codeexample}

Obviously, Karl would not only like to place nodes \emph{on} the last
specified position, but also to the left or the
right of these positions. For this, every node object that you
put in your picture is equipped with several \emph{anchors}. For
example, the |north| anchor is in the middle at the upper end of the shape,
the |south| anchor is at the bottom and the |north east| anchor is in
the upper right corner. When you give the option |anchor=north|, the
text will be placed such that this northern anchor will lie on the
current position and the text is, thus, below the current
position. Karl uses this to draw the ticks as follows:

\begin{codeexample}[]
\begin{tikzpicture}[scale=3]
  \clip (-0.6,-0.2) rectangle (0.6,1.51);
  \draw[step=.5cm,help lines] (-1.4,-1.4) grid (1.4,1.4);
  \filldraw[fill=green!20,draw=green!50!black] (0,0) -- (3mm,0mm)
    arc [start angle=0, end angle=30, radius=3mm] -- cycle;
  \draw[->] (-1.5,0) -- (1.5,0);   \draw[->] (0,-1.5) -- (0,1.5);
  \draw (0,0) circle [radius=1cm];

  \foreach \x in {-1,-0.5,1}
    \draw (\x cm,1pt) -- (\x cm,-1pt) node[anchor=north] {$\x$};
  \foreach \y in {-1,-0.5,0.5,1}
    \draw (1pt,\y cm) -- (-1pt,\y cm) node[anchor=east] {$\y$};
\end{tikzpicture}
\end{codeexample}

This is quite nice, already. Using these anchors, Karl can now add
most of the other text elements. However, Karl thinks that, though
``correct,'' it is quite counter-intuitive that in order to place something
\emph{below} a given point, he has to use the \emph{north} anchor. For
this reason, there is an option called |below|, which does the
same as |anchor=north|. Similarly, |above right| does the same as
|anchor=south east|. In addition, |below| takes an optional
dimension argument. If given, the shape will additionally be shifted
downwards by the given amount. So, |below=1pt| can be used to put
a text label below some point and, additionally shift it  1pt
downwards.

Karl is not quite satisfied with the ticks. He would like to have
$1/2$ or $\frac{1}{2}$ shown instead of $0.5$, partly to show off the
nice capabilities of \TeX\ and \tikzname, partly because for positions
like $1/3$ or $\pi$ it is certainly very much preferable to have the
``mathematical'' tick there instead of just the ``numeric'' tick.
His students, on the other hand, prefer $0.5$ over $1/2$
since they are not too fond of fractions in general.

Karl now faces a problem: For the |\foreach| statement, the position
|\x| should still be given as |0.5| since \tikzname\ will not know where
|\frac{1}{2}| is supposed to be. On the other hand, the typeset text
should really be  |\frac{1}{2}|. To solve this problem, |\foreach|
offers a special syntax: Instead of having one variable |\x|, Karl can
specify two (or even more) variables separated by a slash as in
|\x / \xtext|. Then, the elements in the set over which |\foreach|
iterates must also be of the form \meta{first}|/|\meta{second}. In
each iteration, |\x| will be set to \meta{first} and |\xtext| will be
set to \meta{second}. If no \meta{second} is given, the \meta{first}
will be used again. So, here is the new code for the ticks:

\begin{codeexample}[]
\begin{tikzpicture}[scale=3]
  \clip (-0.6,-0.2) rectangle (0.6,1.51);
  \draw[step=.5cm,help lines] (-1.4,-1.4) grid (1.4,1.4);
  \filldraw[fill=green!20,draw=green!50!black] (0,0) -- (3mm,0mm)
      arc [start angle=0, end angle=30, radius=3mm] -- cycle;
  \draw[->] (-1.5,0) -- (1.5,0); \draw[->] (0,-1.5) -- (0,1.5);
  \draw (0,0) circle [radius=1cm];

  \foreach \x/\xtext in {-1, -0.5/-\frac{1}{2}, 1}
    \draw (\x cm,1pt) -- (\x cm,-1pt) node[anchor=north] {$\xtext$};
  \foreach \y/\ytext in {-1, -0.5/-\frac{1}{2}, 0.5/\frac{1}{2}, 1}
    \draw (1pt,\y cm) -- (-1pt,\y cm) node[anchor=east] {$\ytext$};
\end{tikzpicture}
\end{codeexample}

Karl is quite pleased with the result, but his son points out that
this is still not perfectly satisfactory: The grid and the circle
interfere with the numbers and decrease their legibility. Karl is not
very concerned by this (his students do not even notice), but his son
insists that there is an easy solution: Karl can add the
|[fill=white]| option to fill out the background of the text shape
with a white color.

The next thing Karl wants to do is to add the labels like $\sin
\alpha$. For this, he would like to place a label ``in the middle of
the line.'' To do so, instead of specifying the label
|node {$\sin\alpha$}|  directly after one of the endpoints of the line
(which would place
the label at that endpoint), Karl can give the label directly after
the |--|, before the coordinate. By default, this places the label in
the middle of the line, but the |pos=| options can be used to modify
this. Also, options like |near start| and |near end| can be used to
modify this position:


\begin{codeexample}[]
\begin{tikzpicture}[scale=3]
  \clip (-2,-0.2) rectangle (2,0.8);
  \draw[step=.5cm,gray,very thin] (-1.4,-1.4) grid (1.4,1.4);
  \filldraw[fill=green!20,draw=green!50!black] (0,0) -- (3mm,0mm)
    arc [start angle=0, end angle=30, radius=3mm] -- cycle;
  \draw[->] (-1.5,0) -- (1.5,0) coordinate (x axis);
  \draw[->] (0,-1.5) -- (0,1.5) coordinate (y axis);
  \draw (0,0) circle [radius=1cm];

  \draw[very thick,red]
    (30:1cm) -- node[left=1pt,fill=white] {$\sin \alpha$} (30:1cm |- x axis);
  \draw[very thick,blue]
    (30:1cm |- x axis) -- node[below=2pt,fill=white] {$\cos \alpha$} (0,0);
  \path [name path=upward line] (1,0) -- (1,1);
  \path [name path=sloped line] (0,0) -- (30:1.5cm);
  \draw [name intersections={of=upward line and sloped line, by=t}]
    [very thick,orange] (1,0) -- node [right=1pt,fill=white]
    {$\displaystyle \tan \alpha \color{black}=
      \frac{{\color{red}\sin \alpha}}{\color{blue}\cos \alpha}$} (t);

  \draw (0,0) -- (t);

  \foreach \x/\xtext in {-1, -0.5/-\frac{1}{2}, 1}
    \draw (\x cm,1pt) -- (\x cm,-1pt) node[anchor=north,fill=white] {$\xtext$};
  \foreach \y/\ytext in {-1, -0.5/-\frac{1}{2}, 0.5/\frac{1}{2}, 1}
    \draw (1pt,\y cm) -- (-1pt,\y cm) node[anchor=east,fill=white] {$\ytext$};
\end{tikzpicture}
\end{codeexample}

You can also position labels on curves and, by adding the |sloped|
option, have them rotated such that they match the line's slope. Here
is an example:

\begin{codeexample}[]
\begin{tikzpicture}
  \draw (0,0) .. controls (6,1) and (9,1) ..
    node[near start,sloped,above] {near start}
    node {midway}
    node[very near end,sloped,below] {very near end} (12,0);
\end{tikzpicture}
\end{codeexample}

It remains to draw the explanatory text at the right of the
picture. The main difficulty here lies in limiting the width of the
text ``label,'' which is quite long, so that line breaking is
used. Fortunately, Karl can use the option |text width=6cm| to get the
desired effect. So, here is the full code:

\begin{codeexample}[code only]
\begin{tikzpicture}
  [scale=3,line cap=round,
  % Styles
  axes/.style=,
  important line/.style={very thick},
  information text/.style={rounded corners,fill=red!10,inner sep=1ex}]

  % Colors
  \colorlet{anglecolor}{green!50!black}
  \colorlet{sincolor}{red}
  \colorlet{tancolor}{orange!80!black}
  \colorlet{coscolor}{blue}

  % The graphic
  \draw[help lines,step=0.5cm] (-1.4,-1.4) grid (1.4,1.4);

  \draw (0,0) circle [radius=1cm];

  \begin{scope}[axes]
    \draw[->] (-1.5,0) -- (1.5,0) node[right] {$x$} coordinate(x axis);
    \draw[->] (0,-1.5) -- (0,1.5) node[above] {$y$} coordinate(y axis);

    \foreach \x/\xtext in {-1, -.5/-\frac{1}{2}, 1}
      \draw[xshift=\x cm] (0pt,1pt) -- (0pt,-1pt) node[below,fill=white] {$\xtext$};

    \foreach \y/\ytext in {-1, -.5/-\frac{1}{2}, .5/\frac{1}{2}, 1}
      \draw[yshift=\y cm] (1pt,0pt) -- (-1pt,0pt) node[left,fill=white] {$\ytext$};
  \end{scope}

  \filldraw[fill=green!20,draw=anglecolor] (0,0) -- (3mm,0pt)
    arc [start angle=0, end angle=30, radius=3mm];
  \draw (15:2mm) node[anglecolor] {$\alpha$};

  \draw[important line,sincolor]
    (30:1cm) -- node[left=1pt,fill=white] {$\sin \alpha$} (30:1cm |- x axis);

  \draw[important line,coscolor]
    (30:1cm |- x axis) -- node[below=2pt,fill=white] {$\cos \alpha$} (0,0);

  \path [name path=upward line] (1,0) -- (1,1);
  \path [name path=sloped line] (0,0) -- (30:1.5cm);
  \draw [name intersections={of=upward line and sloped line, by=t}]
    [very thick,orange] (1,0) -- node [right=1pt,fill=white]
    {$\displaystyle \tan \alpha \color{black}=
      \frac{{\color{red}\sin \alpha}}{\color{blue}\cos \alpha}$} (t);

  \draw (0,0) -- (t);

  \draw[xshift=1.85cm]
    node[right,text width=6cm,information text]
    {
      The {\color{anglecolor} angle $\alpha$} is $30^\circ$ in the
      example ($\pi/6$ in radians). The {\color{sincolor}sine of
        $\alpha$}, which is the height of the red line, is
      \[
      {\color{sincolor} \sin \alpha} = 1/2.
      \]
      By the Theorem of Pythagoras ...
    };
\end{tikzpicture}
\end{codeexample}



\subsection{Pics: The Angle Revisited}

Karl expects that the code of certain parts of the picture he created
might be so useful that he might wish to reuse them in the
future. A natural thing to do is to create \TeX\ macros that store
the code he wishes to reuse. However, \tikzname\ offers another way
that is integrated directly into its parser: pics!

A ``pic'' is ``not quite a full picture,'' hence the short name. The
idea is that a pic is simply some code that you can add to a picture
at different places using the |pic| command whose syntax is almost
identical to the |node| command. The main difference is that instead
of specifying some text in curly braces that should be shown, you
specify the name of a predefined picture that should be shown. 

Defining new pics is easy enough, see Section~\ref{section-pics}, but
right now we just want to use one such predefined pic: the |angle|
pic. As the name suggests, it is a small drawing of an angle
consisting of a little wedge and an arc together with some text (Karl
needs to load the |angle| library and the |quotes| for the following
examples). What makes this pic useful is the fact that the size of the
wedge will be computed automatically.

The |angle| pic draws an angle between the two lines $BA$ and $BC$,
where $A$, $B$, and $C$ are three coordinates. In our case, $B$ is the
origin, $A$ is somewhere on the $x$-axis and $C$ is somewhere on a
line at $30^\circ$. 

\begin{codeexample}[]
\begin{tikzpicture}[scale=3]
  \coordinate (A) at (1,0);
  \coordinate (B) at (0,0);
  \coordinate (C) at (30:1cm);

  \draw (A) -- (B) -- (C)
        pic [draw=green!50!black, fill=green!20, angle radius=9mm,
             "$\alpha$"] {angle = A--B--C};
\end{tikzpicture}  
\end{codeexample}

Let us see, what is happening here. First we have specified three
\emph{coordinates} using the |\coordinate| command. It allows us to
name a specific coordinate in the picture. Then comes something that
starts as a normal |\draw|, but then comes the |pic| command. This
command gets lots of options and, in curly braces, comes the most
important point: We specify that we want to add an |angle| pic and
this angle should be between the points we named |A|, |B|, and |C| (we
could use other names). Note that the text that we want to be shown in
the pic is specified in quotes inside the options of the |pic|, not
inside the curly braces.

To learn more about pics, please see Section~\ref{section-pics}.
% Copyright 2006 by Till Tantau
%
% This file may be distributed and/or modified
%
% 1. under the LaTeX Project Public License and/or
% 2. under the GNU Free Documentation License.
%
% See the file doc/generic/pgf/licenses/LICENSE for more details.

\section{Tutorial: A Petri-Net for Hagen}

In this second tutorial we explore the node mechanism of
\tikzname\ and \pgfname.

Hagen must give a talk tomorrow about his favorite formalism for
distributed systems: Petri nets! Hagen used to give his talks using a
blackboard and everyone seemed to be perfectly concent with
this. Unfortunately, his audience has been spoiled recently with fancy
projector-based presentations and there seems to be a certain amount
of peer pressure that this Petri nets should also be drawn using a
graphic program. One of the professors at his institutes recommends
\tikzname\ for this and Hagen decides to give it a try.


\subsection{Problem Statement}

For his talk, Hagen wishes to create a graphic that demonstrates how a
net with place capacities can be simulated by a net without
capacities. The graphic should look like this, ideally:

\begin{quote}
\begin{tikzpicture}
  [node distance=1.3cm,>=stealth',bend angle=45,auto,
   place/.style={circle,thick,draw=blue!75,fill=blue!20,minimum size=6mm},
   red place/.style={place,draw=red!75,fill=red!20},
   transition/.style={rectangle,thick,draw=black!75,fill=black!20,minimum size=4mm},
   every label/.style={red},on grid]

  \begin{scope}
    % First net
    \node [place,tokens=1] (w1)                                    {};
    \node [place] (c1) [below=of w1]                      {};
    \node [place] (s)  [below=of c1,label=above:$s\le 3$] {};
    \node [place] (c2) [below=of s]                       {};
    \node [place,tokens=1] (w2) [below=of c2]                      {};
    
    \node [transition] (e1) [left=of c1] {}
      edge [pre,bend left]                  (w1)
      edge [post,bend right]                (s)
      edge [post]                           (c1);

    \node [transition] (e2) [left=of c2] {}
      edge [pre,bend right]                 (w2)
      edge [post,bend left]                 (s)
      edge [post]                           (c2);
      
    \node [transition] (l1) [right=of c1] {}
      edge [pre]                            (c1)
      edge [pre,bend left]                  (s)
      edge [post,bend right] node[swap] {2} (w1);

    \node [transition] (l2) [right=of c2] {}
      edge [pre]                            (c2)
      edge [pre,bend right]                 (s)
      edge [post,bend left]  node {2}       (w2);
  \end{scope}
  
  \begin{scope}[xshift=6cm]
    % Second net
    \node [place,tokens=1]
                      (w1')                                                {};
    \node [place]     (c1') [below=of w1']                                 {};
    \node [red place] (s1') [below=of c1',xshift=-5mm,label=left:$s$]      {};
    \node [red place,tokens=3]
                      (s2') [below=of c1',xshift=5mm,label=right:$\bar s$] {};
    \node [place]     (c2') [below=of s1',xshift=5mm]                      {};
    \node [place,tokens=1]
                      (w2') [below=of c2']                                 {};
    
    \node [transition] (e1') [left=of c1'] {}
      edge [pre,bend left]                  (w1')
      edge [post]                           (s1')
      edge [pre]                            (s2')
      edge [post]                           (c1');

    \node [transition] (e2') [left=of c2'] {}
      edge [pre,bend right]                 (w2')
      edge [post]                           (s1')
      edge [pre]                            (s2')
      edge [post]                           (c2');
      
    \node [transition] (l1') [right=of c1'] {}
      edge [pre]                            (c1')
      edge [pre]                            (s1')
      edge [post]                           (s2')
      edge [post,bend right] node[swap] {2} (w1');

    \node [transition] (l2') [right=of c2'] {}
      edge [pre]                            (c2')
      edge [pre]                            (s1')
      edge [post]                           (s2')
      edge [post,bend left]  node {2}       (w2');
  \end{scope}

  \begin{pgfonlayer}{background}
    \node (r1) [fill=black!10,rounded corners,fit=(w1)(w2)(e1)(e2)(l1)(l2)] {};
    \node (r2) [fill=black!10,rounded corners,fit=(w1')(w2')(e1')(e2')(l1')(l2')] {};
  \end{pgfonlayer}

  \draw [shorten >=1mm,-to,thick,decorate,decoration={snake,amplitude=.4mm,segment
      length=2mm,pre=moveto,pre length=1mm,post length=2mm}]
    (r1) -- (r2)
    node [above=1mm,midway,text width=3cm,text centered]
      {replacement of the \textcolor{red}{capacity} by \textcolor{red}{two places}};

\end{tikzpicture}
\end{quote}


\subsection{Setting up the Environment}

For the picture Hagen will need to load the \tikzname\ package as did
Karl in the previous tutorial. However, Hagen will also need to load
some additional  \emph{library packages} that Karl did not need. These
library packages contain additional definitions like extra arrow tips
that are typically not needed in a picture and that need to be
loaded explicitly.

Hagen will need to load several libraries: The |arrows| library for the
special arrow tip used in the graphic, the |decoration.pathmorphing|
library for the ``snaking line'' in the middle, the background
library for the two rectangular areas that are behind the two main
parts of the picture, the |fit| library to easily compute the sizes of
these ractangles, and the |positioning| library for placing nodes
relative to other nodes.


\subsubsection{Setting up the Environment in \LaTeX}

When using \LaTeX\ use:

\begin{codeexample}[code only]
\documentclass{article} % say

\usepackage{tikz}
\usetikzlibrary{arrows,decorations.pathmorphing,backgrounds,positioning,fit}

\begin{document}
\begin{tikzpicture}
  \draw (0,0) -- (1,1);
\end{tikzpicture}
\end{document}
\end{codeexample}


\subsubsection{Setting up the Environment in Plain \TeX}

When using plain \TeX\ use:

\begin{codeexample}[code only]
%% Plain TeX file
\input tikz.tex
\usetikzlibrary{arrows,decorations.pathmorphing,backgrounds,positioning,fit}
\baselineskip=12pt
\hsize=6.3truein
\vsize=8.7truein
\tikzpicture
  \draw (0,0) -- (1,1);
\endtikzpicture
\bye
\end{codeexample}


\subsubsection{Setting up the Environment in Con\TeX t}

When using Con\TeX\ use:
\begin{codeexample}[code only]
%% ConTeXt file
\usemodule[tikz]
\usetikzlibrary[arrows,decorations.pathmorphing,backgrounds,positioning,fit]

\starttext
  \starttikzpicture
    \draw (0,0) -- (1,1);
  \stoptikzpicture
\startext
\end{codeexample}



\subsection{Introduction to Nodes}

In principle, we already know how to create the graphics that Hagen
desires (except perhaps for the snaked line, we will come to that): We
start with big light gray rectangle and then add lots of circles and
small rectangle, plus some arrows.

However, this approach has numerous disadvantages: First, it is hard
to change anything at a later stage. For example, if we decide to add
more places to the Petri nets (the circles are called places in Petri
net theory), all of the coordinates change and we need to recalculate
everything. Second, it is hard to read the code for the Petri net as
it just a long and complicated list of coordinates and drawing
commands -- the underlying structure of the Petri net is lost.

Fortunately, \tikzname\ offers a powerful mechanism for avoiding the
above problems: nodes. We already came across nodes in the previous
tutorial, where we used them to add labels to Karl's graphic. In the
present tutorial we will see that nodes are much more powerful.

A node is a small part of a picture. When a node is created, you
provide a position where the node should be drawn and a
\emph{shape}. A node of shape |circle| will be drawn as a circle, a
node of shape |rectangle| as a rectangle, and so on. A node may also
contain same text, which is why Karl used nodes to show text. Finally,
a node can get a \emph{name} for later reference.

In Hagen's picture we will use nodes for the places and for the
transitions of the Petri net (the places are the circles, the
transitions are the rectangles). Let us start with the upper half of
the left Petir net. In this upper half we have three places and two
transitions. Instead of drawing three circles and two rectangles, we
use three nodes of shape |circle| and two nodes of shape
|rectangle|.

\begin{codeexample}[]
\begin{tikzpicture}
  \path ( 0,2) node [shape=circle,draw] {}    
        ( 0,1) node [shape=circle,draw] {}    
        ( 0,0) node [shape=circle,draw] {}    
        ( 1,1) node [shape=rectangle,draw] {}
        (-1,1) node [shape=rectangle,draw] {};    
\end{tikzpicture}
\end{codeexample}

Hagen notes that this does not quite look like the final picture, but
it seems like a good first step.

Let us have a more detailed look at the code. The whole picture
consists of a single path. Ignoring the |node| operations there is not
much going on in this path: It is just a sequence of coordinates with
nothing ``happening'' between them. Indeed, even if something were to
happen like a line-to or a curve-to, the |\path| command would not
``do'' anything with the resulting path. So, all the magic must be in
the |node| commands.

In the previous tutorial we learned that a |node| will add a piece of
text at the last coordinate. Thus, each of the five nodes is added at
a different position. In the above code, this text is empty
(because of the empty |{}|). So, why do we see anything at all? The
answer is the |draw| option for the |node| operation: It causes the
``shape around the text'' to be drawn.

So, the code |(0,2) node [shape=circle,draw] {}| means the following:
``In the main path, add a move-to to the coordinate |(0,2)|. Then,
temporarily suspend the construction of the main path while the node
is build. This node will be a |circle| around an empty text. This
circle is to be |draw|n, but not filled or otherwise used. Once this
whole node is constructed, it is saved until after the 
main path is finished. Then, it is drawn.'' Then following
|(0,1) node [shape=circle,draw] {}| then has the following effect:
``Continue the main path with a move-to to |(0,1)|. Then construct a
node at this position also. This node is also shown after the main
path is finished.'' And so on.



\subsection{Placing Nodes Using the At Syntax}

Hagen now understands how the |node| operation adds nodes to the path,
but it seems a bit silly to create a path using the |\path| operation,
consisting of numerous superfluous move-to operations, only to place
nodes. He is pleased to learn that there are ways to add nodes in a
more sensible manner.

First, the |node| operation allows one to add
|at (|\meta{coordinate}|)| in order to directly specify where the node
should be placed, sidestepping the rule that nodes are placed on the
last coordinate. Hagen can then write the following:

\begin{codeexample}[]
\begin{tikzpicture}
  \path node at ( 0,2) [shape=circle,draw] {}    
        node at ( 0,1) [shape=circle,draw] {}    
        node at ( 0,0) [shape=circle,draw] {}    
        node at ( 1,1) [shape=rectangle,draw] {}
        node at (-1,1) [shape=rectangle,draw] {};    
\end{tikzpicture}
\end{codeexample}

Now Hagen is still left with a single empty path, but at least the
path no longer contains strange move-tos. It turns out that this can
be improved further: The |\node| command is an abbreviation for
|\path node|, which allows Hagen to write:

\begin{codeexample}[]
\begin{tikzpicture}
  \node at ( 0,2) [circle,draw] {};
  \node at ( 0,1) [circle,draw] {};   
  \node at ( 0,0) [circle,draw] {};   
  \node at ( 1,1) [rectangle,draw] {};
  \node at (-1,1) [rectangle,draw] {};    
\end{tikzpicture}
\end{codeexample}

Hagen likes this syntax much better than the previous one. Note that
Hagen has also omitted the |shape=| since, like |color=|, \tikzname\ 
allows you to omit the |shape=| if there is no confusion.



\subsection{Using Styles}

Feeling adventurous, Hagen tries to make the nodes look nicer. In the
final picture, the circles and rectangle should be filled with
different colors, resulting in the following code:

\begin{codeexample}[]
\begin{tikzpicture}[thick]
  \node at ( 0,2) [circle,draw=blue!50,fill=blue!20] {};
  \node at ( 0,1) [circle,draw=blue!50,fill=blue!20] {};   
  \node at ( 0,0) [circle,draw=blue!50,fill=blue!20] {};   
  \node at ( 1,1) [rectangle,draw=black!50,fill=black!20] {};
  \node at (-1,1) [rectangle,draw=black!50,fill=black!20] {};    
\end{tikzpicture}
\end{codeexample}

While this looks nicer in the picture, the code starts to get a bit
ugly. Ideally, we would like our code to transport the message ``there
are three places and two transitions'' and not so much which
filling colors should be used.

To solve this problem, Hagen uses styles. He defines a style for
places and another style for transitions:

\begin{codeexample}[]
\begin{tikzpicture}
  [place/.style={circle,draw=blue!50,fill=blue!20,thick},
   transition/.style={rectangle,draw=black!50,fill=black!20,thick}]
  \node at ( 0,2) [place] {};
  \node at ( 0,1) [place] {};   
  \node at ( 0,0) [place] {};   
  \node at ( 1,1) [transition] {};
  \node at (-1,1) [transition] {};    
\end{tikzpicture}
\end{codeexample}


\subsection{Node Size}

Before Hagen starts naming and connecting the nodes, let us first
make sure that the nodes get their final appearance. They are still
too small. Indeed, Hagen wonders why they have any size at all, after
all, the text is empty. The reason is than \tikzname\ automatically
adds some space around the text. The amount is set using the option
|inner sep|. So, to increase the size of the nodes, Hagen could write:

\begin{codeexample}[]
\begin{tikzpicture}
  [inner sep=2mm,
   place/.style={circle,draw=blue!50,fill=blue!20,thick},
   transition/.style={rectangle,draw=black!50,fill=black!20,thick}]
  \node at ( 0,2) [place] {};
  \node at ( 0,1) [place] {};   
  \node at ( 0,0) [place] {};   
  \node at ( 1,1) [transition] {};
  \node at (-1,1) [transition] {};    
\end{tikzpicture}
\end{codeexample}

However, this is not really the best way to achieve the desired
effect. It is much better to use the |minimum size| option
instead. This option allows Hagen to specify a minimum size that the
node should have. If the nodes actually needs to be bigger because of
a longer text, it will be larger, but if the text is empty, then the
node will have |minimum size|. This option is also useful to ensure
that several nodes containing different amounts of text have the same
size. The options |minimum height| and |minimum width| allow you to
specify the minimum height and width independently. 

So, what Hagen needs to do is to provide |minimum size| for the
nodes. To be on the safe side, he also sets |inner sep=0pt|. This
ensures that the nodes will really have size |minimum size| and not,
for very small minimum sizes, the minimal size necessary to encompass
the automatically added space.

\begin{codeexample}[]
\begin{tikzpicture}
  [place/.style={circle,draw=blue!50,fill=blue!20,thick,
                 inner sep=0pt,minimum size=6mm},
   transition/.style={rectangle,draw=black!50,fill=black!20,thick,
                      inner sep=0pt,minimum size=4mm}]  
  \node at ( 0,2) [place] {};
  \node at ( 0,1) [place] {};   
  \node at ( 0,0) [place] {};   
  \node at ( 1,1) [transition] {};
  \node at (-1,1) [transition] {};    
\end{tikzpicture}
\end{codeexample}




\subsection{Naming Nodes}

Hagen's next aim is to connect the nodes using arrows. This seems like
a tricky business since the arrows should not start in the middle of
the nodes, but somewhere on the border and Hagen would very much like
to avoid computing these positions by hand.

Fortunately, \pgfname\ will perform all the necessary calculations for
him. However, he first has to assign names to the nodes so that he can
reference them later on.

There are two ways to name a node. The first is the use the |name=|
option. The second method is to write the desired name in parentheses
after the |node| operation. Hagen thinks that this second method seems
strange, but he will soon change his opinion.

{
\tikzset{place/.style={circle,draw=blue!50,fill=blue!20,thick,
                   inner sep=0pt,minimum size=6mm},
transition/.style={rectangle,draw=black!50,fill=black!20,thick,
                        inner sep=0pt,minimum size=4mm}}
\begin{codeexample}[]
% ... setup styles
\begin{tikzpicture}
  \node (waiting 1)  at ( 0,2)     [place] {};
  \node (critical 1) at ( 0,1)     [place] {};   
  \node (semaphore)  at ( 0,0)     [place] {};   
  \node (leave critical) at ( 1,1) [transition] {};
  \node (enter critical) at (-1,1) [transition] {};    
\end{tikzpicture}
\end{codeexample}
}

Hagen is pleased to note that the names help in understanding the
code. Names for nodes can be pretty arbitrary, but they should not
contain commas, periods, parentheses, colons, and some other special
characters. However, they can contain underscores and hyphens. 

The syntax for the |node| operation is quite liberal with respect to
the order in which node names, the |at| specifier, and the options
must come. Indeed, you can even have multiple option blocks between
the |node| and the text in curly braces, they accumulate. You can
rearrange them arbitrarily and perhaps the following might be preferable:

{
\tikzset{place/.style={circle,draw=blue!50,fill=blue!20,thick,
                   inner sep=0pt,minimum size=6mm},
transition/.style={rectangle,draw=black!50,fill=black!20,thick,
                        inner sep=0pt,minimum size=4mm}}
\begin{codeexample}[]
\begin{tikzpicture}
  \node[place]      (waiting 1)      at ( 0,2) {};
  \node[place]      (critical 1)     at ( 0,1) {};   
  \node[place]      (semaphore)      at ( 0,0) {};   
  \node[transition] (leave critical) at ( 1,1) {};
  \node[transition] (enter critical) at (-1,1) {};    
\end{tikzpicture}
\end{codeexample}
}



\subsection{Placing Nodes Using Relative Placement}

Although Hagen still wishes to connect the nodes, he first wishes to
address another problem again: The placement of the nodes. Although he
likes the |at| syntax, in this particular case he would prefer placing
the nodes ``relative to each other.'' So, Hagen would like to say that
the |critical 1| node should be below the |waiting 1| node, wherever
the |waiting 1| node might be. There are different ways of achieving
this, but the nicest one in Hagen's case is the |below| option:

{
\tikzset{place/.style={circle,draw=blue!50,fill=blue!20,thick,
                   inner sep=0pt,minimum size=6mm},
transition/.style={rectangle,draw=black!50,fill=black!20,thick,
                        inner sep=0pt,minimum size=4mm}}
\begin{codeexample}[]
\begin{tikzpicture}
  \node[place]      (waiting)                            {};
  \node[place]      (critical)       [below=of waiting]  {};   
  \node[place]      (semaphore)      [below=of critical] {};   
  \node[transition] (leave critical) [right=of critical] {};
  \node[transition] (enter critical) [left=of critical]  {};    
\end{tikzpicture}
\end{codeexample}
}

With the |positioning| library loaded, when an option like |below| 
is followed by |of|, then the position of the node is shifted
such a manner that it is placed at the distance |node distance| in the
specified direction of the given direction. The |node distance| is
either the distance between the centers of the nodes (when the 
|on grid| option is set to true) or the distance between the borders
(when the |on grid| option is set to false, which is the default).

Even though the above code has the same effect the earlier code, Hagen
can pass it to his colleagues who will be able to just read and
understand it, perhaps without even having to see the picture.



\subsection{Adding Labels Next to Nodes}

Before we have a look at how Hagen can connect the nodes, let us add
the capacity ``$s \le 3$'' to the bottom node. For this, two
approaches are possible:
\begin{enumerate}
\item Hagen can just add a new node above the |north| anchor of the
  |semaphore| node.
{
\tikzset{place/.style={circle,draw=blue!50,fill=blue!20,thick,
                   inner sep=0pt,minimum size=6mm},
transition/.style={rectangle,draw=black!50,fill=black!20,thick,
                        inner sep=0pt,minimum size=4mm}}
\begin{codeexample}[]
\begin{tikzpicture}
  \node[place]      (waiting)                            {};
  \node[place]      (critical)       [below=of waiting]  {};   
  \node[place]      (semaphore)      [below=of critical] {};   
  \node[transition] (leave critical) [right=of critical] {};
  \node[transition] (enter critical) [left=of critical]  {};    

  \node [red,above] at (semaphore.north) {$s\le 3$};   
\end{tikzpicture}
\end{codeexample}
}
This is a general approach that will ``always work.''

\item Hagen can use the special |label| option. This option is given
  to a |node| and it causes \emph{another} node to be added next to
  the node where the option is given. Here is the idea: When we
  construct the |semaphore| node, we wish to indicate that we want
  another node with the capacity above it. For this, we use the option
  |label=above:$s\le 3$|. This option is interpreted as follows: We
  want a node above the |semaphore| node and this node should read
  ``$s \le 3$.'' Instead of |above| we could also use things like
  |below left| before the colon or a number like |60|. 
{
\tikzset{place/.style={circle,draw=blue!50,fill=blue!20,thick,
                   inner sep=0pt,minimum size=6mm},
transition/.style={rectangle,draw=black!50,fill=black!20,thick,
                        inner sep=0pt,minimum size=4mm}}
\begin{codeexample}[]
\begin{tikzpicture}
  \node[place]      (waiting)                            {};
  \node[place]      (critical)       [below=of waiting]  {};   
  \node[place]      (semaphore)      [below=of critical,
                                      label=above:$s\le3$] {};   
  \node[transition] (leave critical) [right=of critical] {};
  \node[transition] (enter critical) [left=of critical]  {};    
\end{tikzpicture}
\end{codeexample}
}
  It is also possible to give multiple |label| options, this causes
  multiple labels to be drawn.
\begin{codeexample}[]
\tikz
  \node [circle,draw,label=60:$60^\circ$,label=below:$-90^\circ$] {my circle};
\end{codeexample}
  Hagen is not fully satisfied with the |label| option since the label
  is not red. To achieve this, he has two options: First, he can
  redefine the |every label| style. Second, he can add options to the
  label's node. These options are given following the |label=|, so he
  would write |label=[red]above:$s\le3$|. However, this does not quite
  work since \TeX\ thinks that the |]| closes the whole option list of
  the |semaphore| node. So, Hagen has to add braces and writes
  |label={[red]above:$s\le3$}|. Since this looks a bit ugly, Hagen
  decides to redefine the |every label| style.
{
\tikzset{place/.style={circle,draw=blue!50,fill=blue!20,thick,
                   inner sep=0pt,minimum size=6mm},
transition/.style={rectangle,draw=black!50,fill=black!20,thick,
                        inner sep=0pt,minimum size=4mm}}
\begin{codeexample}[]
\begin{tikzpicture}[every label/.style={red}]    
  \node[place]      (waiting)                            {};
  \node[place]      (critical)       [below=of waiting]  {};   
  \node[place]      (semaphore)      [below=of critical,
                                      label=above:$s\le3$] {};   
  \node[transition] (leave critical) [right=of critical] {};
  \node[transition] (enter critical) [left=of critical]  {};    
\end{tikzpicture}
\end{codeexample}
}
\end{enumerate}



\subsection{Connecting Nodes}

It is now high time to connect the nodes. Let us start with something
simple, namely with the straight line from |enter critical| to
|critical|. We want this line to start at the right side of
|enter critical| and to end at the left side of |critical|. For
this, we can use the \emph{anchors} of the nodes. Every node defines a
whole bunch of anchors that lie on its border or inside it. For
example, the |center| anchor is at the center of the node, the |west|
anchor is on the left of the node, and so on. To access the coordinate
of a node, we use a coordinate that contains the node's name followed
by a dot, followed by the anchor's name:

{
\tikzset{place/.style={circle,draw=blue!50,fill=blue!20,thick,
                   inner sep=0pt,minimum size=6mm},
transition/.style={rectangle,draw=black!50,fill=black!20,thick,
                        inner sep=0pt,minimum size=4mm}}
\begin{codeexample}[]
\begin{tikzpicture}
  \node[place]      (waiting)                            {};
  \node[place]      (critical)       [below=of waiting]  {};   
  \node[place]      (semaphore)      [below=of critical] {};   
  \node[transition] (leave critical) [right=of critical] {};
  \node[transition] (enter critical) [left=of critical]  {};    
  \draw [->] (critical.west) -- (enter critical.east);
\end{tikzpicture}
\end{codeexample}
}

Next, let us tackle the curve from |waiting| to |enter critical|. This
can be specified using curves and controls:

{
\tikzset{place/.style={circle,draw=blue!50,fill=blue!20,thick,
                   inner sep=0pt,minimum size=6mm},
transition/.style={rectangle,draw=black!50,fill=black!20,thick,
                        inner sep=0pt,minimum size=4mm}}
\begin{codeexample}[]
\begin{tikzpicture}
  \node[place]      (waiting)                            {};
  \node[place]      (critical)       [below=of waiting]  {};   
  \node[place]      (semaphore)      [below=of critical] {};   
  \node[transition] (leave critical) [right=of critical] {};
  \node[transition] (enter critical) [left=of critical]  {};    
  \draw [->] (enter critical.east) -- (critical.west);
  \draw [->] (waiting.west) .. controls +(left:5mm) and +(up:5mm)
                            .. (enter critical.north);
\end{tikzpicture}
\end{codeexample}
}

Hagen sees how he can now add all his edges, but the whole process
seems a but awkward and not very flexible. Again, the code seems to
obscure the structure of the graphic rather than showing it.

So, let us start improving the code for the edges. First, Hagen can
leave out the anchors:

{
\tikzset{place/.style={circle,draw=blue!50,fill=blue!20,thick,
                   inner sep=0pt,minimum size=6mm},
transition/.style={rectangle,draw=black!50,fill=black!20,thick,
                        inner sep=0pt,minimum size=4mm}}
\begin{codeexample}[]
\begin{tikzpicture}
  \node[place]      (waiting)                            {};
  \node[place]      (critical)       [below=of waiting]  {};   
  \node[place]      (semaphore)      [below=of critical] {};   
  \node[transition] (leave critical) [right=of critical] {};
  \node[transition] (enter critical) [left=of critical]  {};    
  \draw [->] (enter critical) -- (critical);
  \draw [->] (waiting) .. controls +(left:8mm) and +(up:8mm)
                       .. (enter critical);
\end{tikzpicture}
\end{codeexample}
}

Hagen is a bit surprised that this works. After all, how did
\tikzname\ know that the line from |enter critical| to |critical|
should actually start on the borders? Whenever \tikzname\ encounters a
whole node name as a ``coordinate,'' it tries to ``be smart'' about
the anchor that it should choose for this node. Depending on what
happens next, \tikzname\ will choose an anchor that lies on the border
of the node on a line to the next coordinate or control point. The
exact rules are a bit complex, but the chosen point will usually be
correct -- and when it is not, Hagen can still specify the desired
anchor by hand.

Hagen would now like to simplify the curve operation somehow. It turns
out that this can be accomplished using a special path operation: the
|to| operation. This operation takes many options (you can even define
new ones yourself). One pair of option is useful for Hagen: The pair
|in| and |out|. These options take angles at which a curve should
leave or reach the start or target coordinates. Without these options,
a straight line is drawn:

{
\tikzset{place/.style={circle,draw=blue!50,fill=blue!20,thick,
                   inner sep=0pt,minimum size=6mm},
transition/.style={rectangle,draw=black!50,fill=black!20,thick,
                        inner sep=0pt,minimum size=4mm}}
\begin{codeexample}[]
\begin{tikzpicture}
  \node[place]      (waiting)                            {};
  \node[place]      (critical)       [below=of waiting]  {};   
  \node[place]      (semaphore)      [below=of critical] {};   
  \node[transition] (leave critical) [right=of critical] {};
  \node[transition] (enter critical) [left=of critical]  {};    
  \draw [->] (enter critical) to                 (critical);
  \draw [->] (waiting)        to [out=180,in=90] (enter critical);
\end{tikzpicture}
\end{codeexample}
}

There is another option for the |to| operation, that is even better
suited to Hagen's problem: The |bend right| option. This option also
takes an angle, but this angle only specifies the angle by which the
curve is bend to the right:

{
\tikzset{place/.style={circle,draw=blue!50,fill=blue!20,thick,
                   inner sep=0pt,minimum size=6mm},
transition/.style={rectangle,draw=black!50,fill=black!20,thick,
                        inner sep=0pt,minimum size=4mm}}
\begin{codeexample}[]
\begin{tikzpicture}
  \node[place]      (waiting)                            {};
  \node[place]      (critical)       [below=of waiting]  {};   
  \node[place]      (semaphore)      [below=of critical] {};   
  \node[transition] (leave critical) [right=of critical] {};
  \node[transition] (enter critical) [left=of critical]  {};    
  \draw [->] (enter critical) to                 (critical);
  \draw [->] (waiting)        to [bend right=45] (enter critical);
  \draw [->] (enter critical) to [bend right=45] (semaphore);
\end{tikzpicture}
\end{codeexample}
}

It is now time for Hagen to learn about yet another way of specifying
edges: Using the |edge| path operation. This operation is very similar
to the |to| operation, but there is one important difference: Like a
node the edge generated by the |edge| operation is not part of the
main path, but is added only later. This may not seem very important,
but is has some nice consequences. For example, every edge can have
its own arrow tips and its own color and so one and, still, all the
edges can be given on the same path. This allows Hagen to write the
following: 


{
\tikzset{place/.style={circle,draw=blue!50,fill=blue!20,thick,
                   inner sep=0pt,minimum size=6mm},
transition/.style={rectangle,draw=black!50,fill=black!20,thick,
                        inner sep=0pt,minimum size=4mm}}
\begin{codeexample}[]
\begin{tikzpicture}
  \node[place]      (waiting)                            {};
  \node[place]      (critical)       [below=of waiting]  {};   
  \node[place]      (semaphore)      [below=of critical] {};   
  \node[transition] (leave critical) [right=of critical] {};
  \node[transition] (enter critical) [left=of critical]  {}
    edge [->]               (critical)
    edge [<-,bend left=45]  (waiting)
    edge [->,bend right=45] (semaphore);
\end{tikzpicture}
\end{codeexample}
}

Each |edge| caused a new path to be constructed, consisting of a |to|
between the node |enter critical| and the node following the |edge|
command.

The finishing touch is to introduce two styles |pre| and |post| and to
use the |bend angle=45| option to set the bend angle once and for all:

{
\tikzset{place/.style={circle,draw=blue!50,fill=blue!20,thick,
                   inner sep=0pt,minimum size=6mm},
transition/.style={rectangle,draw=black!50,fill=black!20,thick,
                        inner sep=0pt,minimum size=4mm}}
\begin{codeexample}[]
% Styles place and transition as before
\begin{tikzpicture}
  [bend angle=45,
   pre/.style={<-,shorten <=1pt,>=stealth',semithick},
   post/.style={->,shorten >=1pt,>=stealth',semithick}]

  \node[place]      (waiting)                            {};
  \node[place]      (critical)       [below=of waiting]  {};   
  \node[place]      (semaphore)      [below=of critical] {};   

  \node[transition] (leave critical) [right=of critical] {}
    edge [pre]             (critical)
    edge [post,bend right] (waiting)
    edge [pre, bend left]  (semaphore);
  \node[transition] (enter critical) [left=of critical]  {}
    edge [post]            (critical)
    edge [pre, bend left]  (waiting)
    edge [post,bend right] (semaphore);
\end{tikzpicture}
\end{codeexample}
}




\subsection{Adding Labels Next to Lines}

The next thing that Hagen needs to add is the ``$2$'' at the arcs. For
this Hagen can use \tikzname's automatic node placement: By adding the
option |auto|, \tikzname\ will position nodes on curves and lines in
such a way that they are not on the curve but next to it. Adding
|swap| will mirror the label with respect to the line. Here is a
general example:

{
\begin{codeexample}[]
\begin{tikzpicture}[auto,bend right]
  \node (a) at (0:1) {$0^\circ$};
  \node (b) at (120:1) {$120^\circ$};
  \node (c) at (240:1) {$240^\circ$};

  \draw (a) to node {1} node [swap] {1'} (b)
        (b) to node {2} node [swap] {2'} (c)
        (c) to node {3} node [swap] {3'} (a);
\end{tikzpicture}
\end{codeexample}
}

What is happening here? The nodes are given somehow inside the |to|
operation! When this is done, the node is placed on the middle of the
curve or line created by the |to| operation. The |auto| option then
causes the node to be moved in such a way that it does not lie on the
curve, but next to it. In the example we provide even two nodes on
each |to| operation.

For Hagen that |auto| option is not really necessary since the two
``2'' labels could also easily be placed ``by hand.'' However, in a
complicated plot with numerous edges automatic placement can be a
blessing. 

{
\tikzset{place/.style={circle,draw=blue!50,fill=blue!20,thick,
                   inner sep=0pt,minimum size=6mm},
transition/.style={rectangle,draw=black!50,fill=black!20,thick,
                        inner sep=0pt,minimum size=4mm},
pre/.style={<-,shorten <=1pt,>=stealth',semithick},  
post/.style={->,shorten >=1pt,>=stealth',semithick}}  
\begin{codeexample}[]
% Styles as before
\begin{tikzpicture}[bend angle=45]
  \node[place]      (waiting)                            {};
  \node[place]      (critical)       [below=of waiting]  {};   
  \node[place]      (semaphore)      [below=of critical] {};   

  \node[transition] (leave critical) [right=of critical] {}
    edge [pre]                                 (critical)
    edge [post,bend right] node[auto,swap] {2} (waiting)
    edge [pre, bend left]                      (semaphore);
  \node[transition] (enter critical) [left=of critical]  {}
    edge [post]                                (critical)
    edge [pre, bend left]                      (waiting)
    edge [post,bend right]                     (semaphore);
\end{tikzpicture}
\end{codeexample}
}



\subsection{Adding the Snaked Line and Multi-Line Text}

With the node mechanism Hagen can now easily create the two Petri
nets. What he is unsure of is how he can create the snaked line
between the nets.

For this he can use a \emph{decoration}. 
To draw the snake, Hagen only needs to set the two options
|decoration=snake| and |decorate| on
the path. This causes all lines of the path to be replaced by
snakes. It is also possible to use snakes only in certain parts of a
path, but Hagen will not need this.

\begin{codeexample}[]
\begin{tikzpicture}
  \draw [->,decorate,decoration=snake] (0,0) -- (2,0);
\end{tikzpicture}
\end{codeexample}

Well, that does not look quite right, yet. The problem is that the
snake happens to end exactly at the position where the arrow
begins. Fortunately, there is an option that helps here. Also, the
snake should be a bit smaller, which can be influenced by even more
options. 

\begin{codeexample}[]
\begin{tikzpicture}
  \draw [->,decorate,
     decoration={snake,amplitude=.4mm,segment length=2mm,post length=1mm}]
    (0,0) -- (3,0);
\end{tikzpicture}
\end{codeexample}

Now Hagen needs to add the text above the snake. This text is a bit
challenging since it is a multi-line text. To typeset such text, Hagen
needs to specify a width for the text and he needs to specify that the
text should be centered.


\begin{codeexample}[]
\begin{tikzpicture}
  \draw [->,decorate,
      decoration={snake,amplitude=.4mm,segment length=2mm,post length=1mm}]
    (0,0) -- (3,0)
    node [above,text width=3cm,text centered,midway]
    {
      replacement of the \textcolor{red}{capacity} by
      \textcolor{red}{two places}
    };
\end{tikzpicture}
\end{codeexample}



\subsection{Using Layers: The Background Rectangles}

Hagen still needs to add the background rectangles. These are a bit
tricky: Hagen would like to draw the rectangles \emph{after} the Petri
nets are finished. The reason is that only then can he conveniently
refer to the coordinates that make up the corners of the
rectangle. If Hagen draws the rectangle first, then he needs to know
the exact size of the Petri net -- which he does not.

The solution is to use \emph{layers}. When the background library is
loaded, Hagen can put parts of his picture inside a |{pgfonlayer}|
environment. Then this part of the picture becomes part of the layer
that is given as an argument to this environment. When the
|{tikzpicture}| environment ends, the layers are put on top of each
other, starting with the background layer. This causes everything
drawn on the background layer to be behind the main text.

The next tricky question is, how big should the rectangle be?
Naturally, Hagen can compute the size ``by hand'' or using some clever
observations concerning the $x$- and $y$-coordinates of the nodes, but
it would be nicer to just have \tikzname\ compute a rectangle into
which all the nodes ``fit.'' For this, the |fit| library can be
used. It defines the |fit| options, which, when give to a node, causes
the node to be resized and shifted such that it exactly covers all the
nodes and coordinates given as parameters to the |fit| option.

{
\tikzset{place/.style={circle,draw=blue!50,fill=blue!20,thick,
                   inner sep=0pt,minimum size=6mm},
transition/.style={rectangle,draw=black!50,fill=black!20,thick,
                        inner sep=0pt,minimum size=4mm},
pre/.style={<-,shorten <=1pt,>=stealth',semithick}, 
post/.style={->,shorten >=1pt,>=stealth',semithick}}
\begin{codeexample}[]
% Styles as before
\begin{tikzpicture}[bend angle=45]
  \node[place]      (waiting)                            {};
  \node[place]      (critical)       [below=of waiting]  {};   
  \node[place]      (semaphore)      [below=of critical] {};   

  \node[transition] (leave critical) [right=of critical] {}
    edge [pre]                                 (critical)
    edge [post,bend right] node[auto,swap] {2} (waiting)
    edge [pre, bend left]                      (semaphore);
  \node[transition] (enter critical) [left=of critical]  {}
    edge [post]                                (critical)
    edge [pre, bend left]                      (waiting)
    edge [post,bend right]                     (semaphore);

  \begin{pgfonlayer}{background}
    \node [fill=black!30,fit=(waiting) (critical) (semaphore) 
             (leave critical) (enter critical)] {};
  \end{pgfonlayer}
\end{tikzpicture}
\end{codeexample}
}




\subsection{The Complete Code}

Hagen has now finally put everything together. Only then does he learn
that there is already a library for drawing Petri nets! It turns out
that this library mainly provides the same definitions as Hagen
did. For example, it defines a |place| style in a similar way as Hagen
did. Adjusting the code so that it uses the library shortens Hagen
code a bit, as shown in the following.

First, Hagen needs less style definitions, but he still needs to
specify the colors of places and transitions.

\begin{codeexample}[code only]
\begin{tikzpicture}
  [node distance=1.3cm,on grid,>=stealth',bend angle=45,auto,
   every place/.style=     {minimum size=6mm,thick,draw=blue!75,fill=blue!20},
   every transition/.style={thick,draw=black!75,fill=black!20},
   red place/.style=       {place,draw=red!75,fill=red!20},
   every label/.style=     {red}]
\end{codeexample}

Now comes the code for the nets:

{
\tikzset{%
  every place/.style={minimum size=6mm,thick,draw=blue!75,fill=blue!20},
  every transition/.style={thick,draw=black!75,fill=black!20},
  red place/.style={place,draw=red!75,fill=red!20},
  every label/.style={red},
  every picture/.style={on grid,node distance=1.3cm,>=stealth',bend angle=45,auto}}
\begin{codeexample}[pre=\begin{tikzpicture},post=\end{tikzpicture}]
   \node [place,tokens=1] (w1)                                    {};
   \node [place]          (c1) [below=of w1]                      {};
   \node [place]          (s)  [below=of c1,label=above:$s\le 3$] {};
   \node [place]          (c2) [below=of s]                       {};
   \node [place,tokens=1] (w2) [below=of c2]                      {};
  
   \node [transition] (e1) [left=of c1] {}
     edge [pre,bend left]                  (w1)
     edge [post,bend right]                (s)
     edge [post]                           (c1);
   \node [transition] (e2) [left=of c2] {}
     edge [pre,bend right]                 (w2)
     edge [post,bend left]                 (s)
     edge [post]                           (c2);
   \node [transition] (l1) [right=of c1] {}
     edge [pre]                            (c1)
     edge [pre,bend left]                  (s)
     edge [post,bend right] node[swap] {2} (w1);
   \node [transition] (l2) [right=of c2] {}
     edge [pre]                            (c2)
     edge [pre,bend right]                 (s)
     edge [post,bend left]  node {2}       (w2);
\end{codeexample}
}

{
\tikzset{
every place/.style=     {minimum size=6mm,thick,draw=blue!75,fill=blue!20},
every transition/.style={thick,draw=black!75,fill=black!20},
red place/.style=  {place,draw=red!75,fill=red!20},
every label/.style={red},
every picture/.style={on grid,node distance=1.3cm,>=stealth',bend angle=45,auto}}
\begin{codeexample}[pre=\begin{tikzpicture},post=\end{tikzpicture}]
  \begin{scope}[xshift=6cm]    
    \node [place,tokens=1]     (w1')                            {};
    \node [place]              (c1') [below=of w1']             {};
    \node [red place]          (s1') [below=of c1',xshift=-5mm]
            [label=left:$s$]                                    {};
    \node [red place,tokens=3] (s2') [below=of c1',xshift=5mm]
            [label=right:$\bar s$]                              {};
    \node [place]              (c2') [below=of s1',xshift=5mm]  {};
    \node [place,tokens=1]     (w2') [below=of c2']             {};
    
    \node [transition] (e1') [left=of c1'] {}
      edge [pre,bend left]                  (w1')
      edge [post]                           (s1')
      edge [pre]                            (s2')
      edge [post]                           (c1');
    \node [transition] (e2') [left=of c2'] {}
      edge [pre,bend right]                 (w2')
      edge [post]                           (s1')
      edge [pre]                            (s2')
      edge [post]                           (c2');
    \node [transition] (l1') [right=of c1'] {}
      edge [pre]                            (c1')
      edge [pre]                            (s1')
      edge [post]                           (s2')
      edge [post,bend right] node[swap] {2} (w1');
    \node [transition] (l2') [right=of c2'] {}
      edge [pre]                            (c2')
      edge [pre]                            (s1')
      edge [post]                           (s2')
      edge [post,bend left]  node {2}       (w2');
  \end{scope}
\end{codeexample}
}

The code for the background and the snake is the following:

\begin{codeexample}[code only]
  \begin{pgfonlayer}{background}
    \node (r1) [fill=black!10,rounded corners,fit=(w1)(w2)(e1)(e2)(l1)(l2)] {};
    \node (r2) [fill=black!10,rounded corners,fit=(w1')(w2')(e1')(e2')(l1')(l2')] {};
  \end{pgfonlayer}

  \draw [shorten >=1mm,-to,thick,decorate,
         decoration={snake,amplitude=.4mm,segment length=2mm,
                     pre=moveto,pre length=1mm,post length=2mm}]
    (r1) -- (r2) node [above=1mm,midway,text width=3cm,text centered]
      {replacement of the \textcolor{red}{capacity} by \textcolor{red}{two places}};
\end{tikzpicture}
\end{codeexample}

% Copyright 2006 by Till Tantau
%
% This file may be distributed and/or modified
%
% 1. under the LaTeX Project Public License and/or
% 2. under the GNU Free Documentation License.
%
% See the file doc/generic/pgf/licenses/LICENSE for more details.


\section{Tutorial: Euclid's Amber Version of the \emph{Elements}}

In this third tutorial we have a look at how \tikzname\ can be used to
draw geometric constructions.

Euclid is currently quite busy writing his new book series, whose
working title is ``Elements'' (Euclid is not quite sure whether this
title will convey the message of the series to future generations
correctly, but he intends to change the title before it goes to the
publisher). Up to know, he wrote down his text and graphics on
papyrus, but his publisher suddenly insists that he must submit in
electronic form. Euclid tries to argue with the publisher that 
electronics will only be discovered thousands of years later, but the
publisher informs him that the use of papyrus is no longer cutting edge
technology and Euclid will just have to keep up with modern tools.

Slightly disgruntled, Euclid starts converting his papyrus
entitled ``Book I, Proposition I'' to an amber version.  

\subsection{Book I, Proposition I}

The drawing on his papyrus looks like this:\footnote{The text is taken
from the wonderful interactive version of Euclid's Elements by David
E. Joyce, to be found on his website at Clark University.}

\bigskip
\noindent
\begin{tikzpicture}[thick,help lines/.style={thin,draw=black!50}]
  \def\A{\textcolor{input}{$A$}}
  \def\B{\textcolor{input}{$B$}}
  \def\C{\textcolor{output}{$C$}}
  \def\D{$D$}
  \def\E{$E$}
  
  \colorlet{input}{blue!80!black}
  \colorlet{output}{red!70!black}
  \colorlet{triangle}{orange}
  
  \coordinate [label=left:\A]
    (A) at ($ (0,0) + .1*(rand,rand) $);
  \coordinate [label=right:\B]
    (B) at ($ (1.25,0.25) + .1*(rand,rand) $);

  \draw [input] (A) -- (B);
  
  \node [name path=D,help lines,draw,label=left:\D] (D) at (A) [circle through=(B)] {};
  \node [name path=E,help lines,draw,label=right:\E] (E) at (B) [circle through=(A)] {};
  
  \path [name intersections={of=D and E,by={[label=above:\C]C}}];

  \draw [output] (A) -- (C);
  \draw [output] (B) -- (C);

  \foreach \point in {A,B,C}
    \fill [black,opacity=.5] (\point) circle (2pt);

  \begin{pgfonlayer}{background}
    \fill[triangle!80] (A) -- (C) -- (B) -- cycle;
  \end{pgfonlayer}
  
  \node [below right,text width=10cm,align=justify] at (4,3)
  {
    \small
    \textbf{Proposition I}\par
    \emph{To construct an \textcolor{triangle}{equilateral triangle}
      on a given \textcolor{input}{finite straight line}.}
    \par
    \vskip1em
    Let \A\B\ be the given \textcolor{input}{finite straight line}. It
    is required to construct an \textcolor{triangle}{equilateral
      triangle} on the \textcolor{input}{straight line}~\A\B. 

    Describe the circle \B\C\D\ with center~\A\ and radius \A\B. Again
    describe the circle \A\C\E\ with center~\B\ and radius \B\A. Join the
    \textcolor{output}{straight lines} \C\A\ and \C\B\ from the
    point~\C\ at which the circles cut one another to the points~\A\ and~\B.

    Now, since the point~\A\ is the center of the circle \C\D\B,
    therefore \A\C\ equals \A\B. Again, since the point \B\ is the
    center of the circle \C\A\E, therefore \B\C\ equals \B\A. But
    \A\C\ was proved equal to \A\B, therefore each of the straight
    lines \A\C\ and \B\C\ equals \A\B. And 
    things which equal the same thing also equal one another,
    therefore \A\C\ also equals \B\C. Therefore the three straight
    lines \A\C, \A\B, and \B\C\ equal one another. 
    Therefore the \textcolor{triangle}{triangle} \A\B\C\ is
    equilateral, and it has been  constructed on the given finite
    \textcolor{input}{straight line}~\A\B.  
  };
\end{tikzpicture}
\bigskip

Let us have a look at how Euclid can turn this into \tikzname\ code.

\subsubsection{Setting up the Environment}

As in the previous tutorials, Euclid needs to load \tikzname, together
with some libraries. These libraries are |calc|, |intersections|,
|through|, and |backgrounds|. Depending on which format he uses,
Euclid would use one of the following in the preamble: 

\begin{codeexample}[code only]
% For LaTeX:
\usepackage{tikz}
\usetikzlibrary{calc,intersections,through,backgrounds}
\end{codeexample}

\begin{codeexample}[code only]
% For plain TeX:
\input tikz.tex
\usetikzlibrary{calc,intersections,through,backgrounds}
\end{codeexample}

\begin{codeexample}[code only]
% For ConTeXt:
\usemodule[tikz]
\usetikzlibrary[calc,intersections,through,backgrounds]
\end{codeexample}


\subsubsection{The Line \emph{AB}}

The first part of the picture that Euclid wishes to draw is the line
$AB$. That is easy enough, something like |\draw (0,0) -- (2,1);|
might do. However, Euclid does not wish to reference the two points
$A$ and $B$ as $(0,0)$ and $(2,1)$ subsequently. Rather, he wishes to
just write |A| and |B|. Indeed, the whole point of his book is that
the points $A$ and $B$ can be arbitrary and all other points (like
$C$) are constructed in terms of their positions. It would not do
if Euclid were to write down the coordinates of $C$ explicitly.

So, Euclid starts with defining two coordinates using the
|\coordinate| command:
\begin{codeexample}[]
\begin{tikzpicture}
  \coordinate (A) at (0,0);
  \coordinate (B) at (1.25,0.25);

  \draw[blue] (A) -- (B);
\end{tikzpicture}
\end{codeexample}

That was easy enough. What is missing at this point are the labels for
the coordinates. Euclid does not want them \emph{on} the points, but
next to them. He decides to use the |label| option:
\begin{codeexample}[]
\begin{tikzpicture}
  \coordinate [label=left:\textcolor{blue}{$A$}]  (A) at (0,0);
  \coordinate [label=right:\textcolor{blue}{$B$}] (B) at (1.25,0.25);

  \draw[blue] (A) -- (B);
\end{tikzpicture}
\end{codeexample}

At this point, Euclid decides that it would be even nicer if the
points $A$ and $B$ were in some sense ``random.'' Then, neither Euclid
nor the reader can make the mistake of taking ``anything for granted''
concerning these position of these points. Euclid is pleased to learn
that there is a |rand| function in \tikzname\ that does exactly what
he needs: It produces a number between $-1$ and $1$. Since \tikzname\
can do a bit of math, Euclid can change the coordinates of the points
as follows:
\begin{codeexample}[code only]
\coordinate [...] (A) at (0+0.1*rand,0+0.1*rand);
\coordinate [...] (B) at (1.25+0.1*rand,0.25+0.1*rand);
\end{codeexample}

This works fine. However, Euclid is not quite satisfied since he would
prefer that the ``main coordinates'' $(0,0)$ and $(1.25,0.25)$ are
``kept separate'' from the perturbation
$0.1(\mathit{rand},\mathit{rand})$. This means, he would like to
specify that coordinate $A$ as ``The point that is at $(0,0)$ plus one
tenth of the vector  $(\mathit{rand},\mathit{rand})$.''

It turns out that the |calc| library allows him to do exactly this
kind of computation. When this library is loaded, you can use special
coordinates that start with |($| and end with |$)| rather than just
|(| and~|)|. Inside these special coordinates you can give a linear
combination of coordinates. (Note that the dollar signs are only
intended to signal that a ``computation'' is going on; no mathematical
typesetting is done.)

The new code for the coordinates is the following:

\begin{codeexample}[code only]
\coordinate [...] (A) at ($ (0,0) + .1*(rand,rand) $);
\coordinate [...] (B) at ($ (1.25,0.25) + .1*(rand,rand) $);
\end{codeexample}

Note that if a coordinate in such a computation has a factor (like
|.1|) you must place a |*| directly before the opening parenthesis of
the coordinate. You can nest such computations.



\subsubsection{The Circle Around \emph{A}}

The first tricky construction is the circle around~$A$. We will see
later how to do this in a very simple manner, but first let us do it
the ``hard'' way.

The idea is the following: We draw a circle around the point $A$ whose
radius is given by the length of the line $AB$. The difficulty lies in
computing the length of this line.

Two ideas ``nearly'' solve this problem: First, we can write
|($ (A) - (B) $)| for the vector that is the difference between $A$
and~$B$. All we need is the length of this vector. Second, given two
numbers $x$ and $y$, one can write |veclen(|$x$|,|$y$|)| inside a
mathematical expression. This gives the value $\sqrt{x^2+y^2}$, which
is exactly the desired length.

The only remaining problem is to access the $x$- and $y$-coordinate of
the vector~$AB$. For this, we need a new concept: the \emph{let
  operation}. A let operation can be given anywhere on a path where a
normal path operation like a line-to or a move-to is expected. The
effect of a let operation is to evaluate some coordinates and to
assign the results to special macros. These macros make it easy to
access the $x$- and $y$-coordinates of the coordinates.

Euclid would write the following:
\begin{codeexample}[]
\begin{tikzpicture}
  \coordinate [label=left:$A$]  (A) at (0,0);
  \coordinate [label=right:$B$] (B) at (1.25,0.25);
  \draw (A) -- (B);

  \draw (A) let
              \p1 = ($ (B) - (A) $)
            in
              circle ({veclen(\x1,\y1)});
\end{tikzpicture}
\end{codeexample}

Each assignment in a let operation starts with |\p|, usually followed
by a \meta{digit}. Then comes an equal sign and a coordinate. The
coordinate is evaluated and the result is stored internally. From
then on you can use the following expressions: 
\begin{enumerate}
\item |\x|\meta{digit} yields the $x$-coordinate of the resulting point.
\item |\y|\meta{digit} yields the $y$-coordinate of the resulting
  point.
\item |\p|\meta{digit} yields the same as |\x|\meta{digit}|,\y|\meta{digit}.
\end{enumerate}
You can have multiple assignments in a let operation, just separate
them with commas. In later assignments you can already use the results
of earlier assignments.

Note that |\p1| is not a coordinate in the usual sense. Rather, it
just expands to a string like |10pt,20pt|. So, you cannot write, for
instance, |(\p1.center)| since this would just expand to
|(10pt,20pt.center)|, which makes no sense.

Next, we want to draw both circles at the same time. Each time the
radius is |veclen(\x1,\y1)|. It seems natural to compute this radius
only once. For this, we can also use a let operation: Instead of
writing |\p1 = ...|, we write |\n2 = ...|. Here, ``n'' stands for
``number'' (while ``p'' stands for ``point''). The assignment of a
number should be followed by a number in curly braces.
\begin{codeexample}[]
\begin{tikzpicture}
  \coordinate [label=left:$A$]  (A) at (0,0);
  \coordinate [label=right:$B$] (B) at (1.25,0.25);
  \draw (A) -- (B);

  \draw let \p1 = ($ (B) - (A) $),
            \n2 = {veclen(\x1,\y1)}
        in
          (A) circle (\n2)
          (B) circle (\n2);
\end{tikzpicture}
\end{codeexample}
In the above example, you may wonder, what |\n1| would yield? The
answer is that it would be undefined -- the |\p|, |\x|, and |\y|
macros refer to the same logical point, while the |\n| macro has ``its
own namespace.'' We could even have replaced |\n2| in the example by
|\n1| and it would still work. Indeed, the digits following these
macros are just normal \TeX\ parameters. We could also use a longer
name, but then we have to use curly braces:
\begin{codeexample}[]
\begin{tikzpicture}
  \coordinate [label=left:$A$]  (A) at (0,0);
  \coordinate [label=right:$B$] (B) at (1.25,0.25);
  \draw (A) -- (B);

  \draw let \p1        = ($ (B) - (A) $),
            \n{radius} = {veclen(\x1,\y1)}
        in
          (A) circle (\n{radius})
          (B) circle (\n{radius});
\end{tikzpicture}
\end{codeexample}

At the beginning of this section it was promised that there is an
easier way to create the desired circle. The trick is to use the
|through| library. As the name suggests, it contains code for creating
shapes that go through a given point.

The option that we are looking for is |circle through|. This option is
given to a \emph{node} and has the following effects: First, it causes
the node's inner and outer separations to be set to zero. Then it sets
the shape of the node to |circle|. Finally, it sets the radius of the
node such that it goes through the parameter given to
|circle through|. This radius is computed in essentially the same way
as above.

\begin{codeexample}[]
\begin{tikzpicture}
  \coordinate [label=left:$A$]  (A) at (0,0);
  \coordinate [label=right:$B$] (B) at (1.25,0.25);
  \draw (A) -- (B);

  \node [draw,circle through=(B),label=left:$D$] at (A) {};
\end{tikzpicture}
\end{codeexample}


\subsubsection{The Intersection of the Circles}

Euclid can now draw the line and the circles. The final problem is to
compute the intersection of the two circles. This computation is a bit
involved if you want to do it ``by hand.'' Fortunately, the
intersection library allows us to compute the intersection of
arbitrary paths.

The idea is simple: First, you ``name'' two paths using the
|name path| option. Then, at some later point, you can use the option
|name intersections|, which creates coordinates called
|intersection-1|, |intersection-2|, and so on at all intersections of
the paths. Euclid assigns the names |D| and |E| to the paths of the
two circles (which happen to be the same names as the nodes
themselves, but nodes and their paths live in different
``namespaces''). 
\begin{codeexample}[]
\begin{tikzpicture}
  \coordinate [label=left:$A$]  (A) at (0,0);
  \coordinate [label=right:$B$] (B) at (1.25,0.25);
  \draw (A) -- (B);

  \node (D) [name path=D,draw,circle through=(B),label=left:$D$]  at (A) {};
  \node (E) [name path=E,draw,circle through=(A),label=right:$E$] at (B) {};

  % Name the coordinates, but do not draw anything:
  \path [name intersections={of=D and E}];
  
  \coordinate [label=above:$C$] (C) at (intersection-1);

  \draw [red] (A) -- (C);
  \draw [red] (B) -- (C);
\end{tikzpicture}
\end{codeexample}

It turns out that this can be further shortened: The
|name intersections| takes an optional argument |by|, which lets you
specify names for the coordinates and options for them. This creates
more compact code. Although Euclid does not need it for the current
picture, it is just a small step to computing the bisection of the line $AB$:

\begin{codeexample}[]
\begin{tikzpicture}
  \coordinate [label=left:$A$]  (A) at (0,0);
  \coordinate [label=right:$B$] (B) at (1.25,0.25);
  \draw [name path=A--B] (A) -- (B);

  \node (D) [name path=D,draw,circle through=(B),label=left:$D$]  at (A) {};
  \node (E) [name path=E,draw,circle through=(A),label=right:$E$] at (B) {};

  \path [name intersections={of=D and E, by={[label=above:$C$]C, [label=below:$C'$]C'}}];

  \draw [name path=C--C',red] (C) -- (C');

  \path [name intersections={of=A--B and C--C',by=F}];
  \node [fill=red,inner sep=1pt,label=-45:$F$] at (F) {};
\end{tikzpicture}
\end{codeexample}



\subsubsection{The Complete Code}

Back to Euclid's code. He introduces a few macros to make life
simpler, like a |\A| macro for typesetting a blue $A$. He also uses the
|background| layer for drawing the triangle behind everything at the
end. 

\begin{codeexample}[]
\begin{tikzpicture}[thick,help lines/.style={thin,draw=black!50}]
  \def\A{\textcolor{input}{$A$}}     \def\B{\textcolor{input}{$B$}}
  \def\C{\textcolor{output}{$C$}}    \def\D{$D$}
  \def\E{$E$}
  
  \colorlet{input}{blue!80!black}    \colorlet{output}{red!70!black}
  \colorlet{triangle}{orange}
  
  \coordinate [label=left:\A]  (A) at ($ (0,0) + .1*(rand,rand) $);
  \coordinate [label=right:\B] (B) at ($ (1.25,0.25) + .1*(rand,rand) $);

  \draw [input] (A) -- (B);
  
  \node [name path=D,help lines,draw,label=left:\D]   (D) at (A) [circle through=(B)] {};
  \node [name path=E,help lines,draw,label=right:\E]  (E) at (B) [circle through=(A)] {};
  
  \path [name intersections={of=D and E,by={[label=above:\C]C}}];

  \draw [output] (A) -- (C) -- (B);

  \foreach \point in {A,B,C}
    \fill [black,opacity=.5] (\point) circle (2pt);

  \begin{pgfonlayer}{background}
    \fill[triangle!80] (A) -- (C) -- (B) -- cycle;
  \end{pgfonlayer}
  
  \node [below right, text width=10cm,align=justify] at (4,3) {
    \small\textbf{Proposition I}\par
    \emph{To construct an \textcolor{triangle}{equilateral triangle}
      on a given \textcolor{input}{finite straight line}.}
    \par\vskip1em
    Let \A\B\ be the given \textcolor{input}{finite straight line}.  \dots
  };
\end{tikzpicture}
\end{codeexample}


\subsection{Book I, Proposition II}

The second proposition in the Elements is the following:

\bigskip\noindent
\begin{tikzpicture}[thick,help lines/.style={thin,draw=black!50}]
  \def\A{\textcolor{orange}{$A$}}   \def\B{\textcolor{input}{$B$}}
  \def\C{\textcolor{input}{$C$}}    \def\D{$D$}
  \def\E{$E$}                       \def\F{$F$}
  \def\G{$G$}                       \def\H{$H$}
  \def\K{$K$}                       \def\L{\textcolor{output}{$L$}}
  
  \colorlet{input}{blue!80!black}    \colorlet{output}{red!70!black}
  
  \coordinate [label=left:\A]  (A) at ($ (0,0) + .1*(rand,rand) $);
  \coordinate [label=right:\B] (B) at ($ (1,0.2) + .1*(rand,rand) $);
  \coordinate [label=above:\C] (C) at ($ (1,2) + .1*(rand,rand) $);

  \draw [input] (B) -- (C);
  \draw [help lines] (A) -- (B);

  \coordinate [label=above:\D] (D) at ($ (A)!.5!(B) ! {sin(60)*2} ! 90:(B) $);

  \draw [help lines] (D) -- ($ (D)!3.75!(A) $) coordinate [label=-135:\E] (E);
  \draw [help lines] (D) -- ($ (D)!3.75!(B) $) coordinate [label=-45:\F] (F);

  \node (H) at (B) [name path=H,help lines,circle through=(C),draw,label=135:\H] {};
  \path [name path=B--F] (B) -- (F);
  \path [name intersections={of=H and B--F}]
    coordinate [label=right:\G] (G) at (intersection-1);

  \node (K) at (D) [name path=K,help lines,circle through=(G),draw,label=135:\K] {};

  \path [name path=A to E line] (A) -- (E);
  \path [name intersections={of=K and A to E line}]
    coordinate [label=below:\L] (L) at (intersection-1);

  \draw [output] (A) -- (L);

  \foreach \point in {A,B,C,D,G,L}
    \fill [black,opacity=.5] (\point) circle (2pt);
  
  \node [below right, text width=9cm,align=justify] at (4,4) {
    \small\textbf{Proposition II}\par
    \emph{To place a \textcolor{output}{straight line} equal to a
      given \textcolor{input}{straight line} with 
      one end at a \textcolor{orange}{given point}.} 
    \par\vskip1em
    Let \A\ be the given point, and \B\C\ the given
    \textcolor{input}{straight line}. 
    It is required to place a \textcolor{output}{straight line} equal
    to the given \textcolor{input}{straight line} \B\C\ with one end
    at the point~\A.  

    Join the straight line \A\B\ from the point \A\ to the point \B, and
    construct the equilateral triangle \D\A\B\ on it.
    
    Produce the straight lines \A\E\ and \B\F\ in a straight line with
    \D\A\ and \D\B. Describe the circle \C\G\H\ with center \B\ and
    radius \B\C, and  again, describe the circle \G\K\L\ with center
    \D\ and radius \D\G. 	

    Since the point \B\ is the center of the circle \C\G\H, therefore
    \B\C\ equals \B\G. Again, since the point \D\ is the center of the
    circle \G\K\L, therefore \D\L\ equals \D\G. And in these \D\A\
    equals \D\B, therefore the remainder \A\L\ equals the remainder
    \B\G. But \B\C\ was also proved  equal to \B\G, therefore each of
    the straight lines \A\L\ and \B\C\ equals \B\G. And things which
    equal the same thing also equal one another, therefore \A\L\ also
    equals \B\C. 
    
    Therefore the \textcolor{output}{straight line} \A\L\ equal to the
    given \textcolor{input}{straight line} \B\C\  has been placed with
    one end at the \textcolor{orange}{given point}~\A.  
  };
\end{tikzpicture}




\subsubsection{Using Partway Calculations for the Construction of \emph{D}}

Euclid's construction starts with ``referencing'' Proposition~I for
the construction of the point~$D$. Now, while we could simply repeat the
construction, it seems a bit bothersome that one has to draw all these
circles and do all these complicated constructions.

For this reason, \tikzname\ supports some simplifications. First,
there is a simple syntax for computing a point that is ``partway'' on
a line from $p$ to~$q$: You place these two points in a coordinate
calculation -- remember, they start with |($| and end with |$)| -- and
then combine them using |!|\meta{part}|!|. A \meta{part} of |0| refers
to the \emph{first} coordinate, a \meta{part} of |1| refers to the
second coordinate, and a value in between refers to a point on the
line from $p$ to~$q$. Thus, the syntax is similar to the |xcolor|
syntax for mixing colors.

Here is the computation of the point in the middle of the line $AB$:
\begin{codeexample}[]
\begin{tikzpicture}
  \coordinate [label=left:$A$]  (A) at (0,0);
  \coordinate [label=right:$B$] (B) at (1.25,0.25);
  \draw (A) -- (B);
  \node [fill=red,inner sep=1pt,label=below:$X$] (X) at ($ (A)!.5!(B) $) {};
\end{tikzpicture}
\end{codeexample}

The computation of the point $D$ in Euclid's second proposition is a
bit more complicated. It can be expressed as follows: Consider the
line from $X$ to $B$. Suppose we 
rotate this line around $X$ for 90$^\circ$ and then stretch it by a
factor of $\sin(60^\circ)/2$. This yields the desired point~$D$. We
can do the stretching using the partway modifier above, for the
rotation we need a new modifier: the rotation modifier. The idea is
that the second coordinate in a partway computation can be prefixed by
an angle. Then the partway point is computed normally (as if no angle
were given), but the resulting point is rotated by this angle around
the first point.  

\begin{codeexample}[]
\begin{tikzpicture}
  \coordinate [label=left:$A$]  (A) at (0,0);
  \coordinate [label=right:$B$] (B) at (1.25,0.25);
  \draw (A) -- (B);
  \node [fill=red,inner sep=1pt,label=below:$X$] (X) at ($ (A)!.5!(B) $) {};
  \node [fill=red,inner sep=1pt,label=above:$D$] (D) at
    ($ (X) ! {sin(60)*2} ! 90:(B) $) {};
  \draw (A) -- (D) -- (B);
\end{tikzpicture}
\end{codeexample}

Finally, it is not necessary to explicitly name the point $X$. Rather,
again like in the |xcolor| package, it is possible to chain partway
modifiers:

\begin{codeexample}[]
\begin{tikzpicture}
  \coordinate [label=left:$A$]  (A) at (0,0);
  \coordinate [label=right:$B$] (B) at (1.25,0.25);
  \draw (A) -- (B);
  \node [fill=red,inner sep=1pt,label=above:$D$] (D) at
    ($ (A) ! .5 ! (B) ! {sin(60)*2} ! 90:(B) $) {};
  \draw (A) -- (D) -- (B);
\end{tikzpicture}
\end{codeexample}


\subsubsection{Intersecting a Line and a Circle}

The next step in the construction is to draw a circle around $B$
through $C$, which is easy enough to do using the |circle through|
option. Extending the lines $DA$ and $DB$ can be done using partway
calculations, but this time with a part value outside the range
$[0,1]$: 

\begin{codeexample}[]
\begin{tikzpicture}
  \coordinate [label=left:$A$]  (A) at (0,0);
  \coordinate [label=right:$B$] (B) at (0.75,0.25);
  \coordinate [label=above:$C$] (C) at (1,1.5);
  \draw (A) -- (B) -- (C);
  \coordinate [label=above:$D$] (D) at
    ($ (A) ! .5 ! (B) ! {sin(60)*2} ! 90:(B) $) {};
  \node (H) [label=135:$H$,draw,circle through=(C)] at (B) {};
  \draw (D) -- ($ (D) ! 3.5 ! (B) $) coordinate [label=below:$F$] (F);
  \draw (D) -- ($ (D) ! 2.5 ! (A) $) coordinate [label=below:$E$] (E);
\end{tikzpicture}
\end{codeexample}

We now face the problem of finding the point $G$, which is the
intersection of the line $BF$ and the circle $H$. One way is to use
yet another variant of the partway computation: Normally, a partway
computation has the form \meta{p}|!|\meta{factor}|!|\meta{q},
resulting in the point $(1-\meta{factor})\meta{p} +
\meta{factor}\meta{q}$. Alternatively, instead of \meta{factor} you
can also use a \meta{dimension} between the points. In this case, you
get the point that is \meta{dimension} removed from \meta{p} on the
straight line to \meta{q}.

We know that the point $G$ is on the way from $B$ to $F$. The distance
is given by the radius of the circle~$H$. Here is the code form
computing $H$:
\begin{codeexample}[pre={
\begin{tikzpicture}
  \coordinate [label=left:$A$]  (A) at (0,0);
  \coordinate [label=right:$B$] (B) at (0.75,0.25);
  \coordinate [label=above:$C$] (C) at (1,1.5);
  \draw (A) -- (B) -- (C);
  \coordinate [label=above:$D$] (D) at
    ($ (A) ! .5 ! (B) ! {sin(60)*2} ! 90:(B) $) {};
  \draw (D) -- ($ (D) ! 3.5 ! (B) $) coordinate [label=below:$F$] (F);
  \draw (D) -- ($ (D) ! 2.5 ! (A) $) coordinate [label=below:$E$] (E);
},post={\end{tikzpicture}}]
  \node (H) [label=135:$H$,draw,circle through=(C)] at (B) {};
  \path let \p1 = ($ (B) - (C) $) in
    coordinate [label=left:$G$] (G) at ($ (B) ! veclen(\x1,\y1) ! (F) $);
  \fill[red,opacity=.5] (G) circle (2pt);
\end{codeexample}

However, there is a simpler way: We can simply name the path of the
circle and of the line in question and then use |name intersections|
to compute the intersections.

\begin{codeexample}[pre={
\begin{tikzpicture}
  \coordinate [label=left:$A$]  (A) at (0,0);
  \coordinate [label=right:$B$] (B) at (0.75,0.25);
  \coordinate [label=above:$C$] (C) at (1,1.5);
  \draw (A) -- (B) -- (C);
  \coordinate [label=above:$D$] (D) at
    ($ (A) ! .5 ! (B) ! {sin(60)*2} ! 90:(B) $) {};
  \draw (D) -- ($ (D) ! 3.5 ! (B) $) coordinate [label=below:$F$] (F);
  \draw (D) -- ($ (D) ! 2.5 ! (A) $) coordinate [label=below:$E$] (E);
},post={\end{tikzpicture}}]
  \node (H) [name path=H,label=135:$H$,draw,circle through=(C)] at (B) {};
  \path [name path=B--F] (B) -- (F);
  \path [name intersections={of=H and B--F,by={[label=left:$G$]G}}];
  \fill[red,opacity=.5] (G) circle (2pt);
\end{codeexample}

\subsubsection{The Complete Code}

\begin{codeexample}[]
\begin{tikzpicture}[thick,help lines/.style={thin,draw=black!50}]
  \def\A{\textcolor{orange}{$A$}}   \def\B{\textcolor{input}{$B$}}
  \def\C{\textcolor{input}{$C$}}    \def\D{$D$}
  \def\E{$E$}                       \def\F{$F$}
  \def\G{$G$}                       \def\H{$H$}
  \def\K{$K$}                       \def\L{\textcolor{output}{$L$}}
  
  \colorlet{input}{blue!80!black}    \colorlet{output}{red!70!black}
  
  \coordinate [label=left:\A]  (A) at ($ (0,0) + .1*(rand,rand) $);
  \coordinate [label=right:\B] (B) at ($ (1,0.2) + .1*(rand,rand) $);
  \coordinate [label=above:\C] (C) at ($ (1,2) + .1*(rand,rand) $);

  \draw [input] (B) -- (C);
  \draw [help lines] (A) -- (B);

  \coordinate [label=above:\D] (D) at ($ (A)!.5!(B) ! {sin(60)*2} ! 90:(B) $);

  \draw [help lines] (D) -- ($ (D)!3.75!(A) $) coordinate [label=-135:\E] (E);
  \draw [help lines] (D) -- ($ (D)!3.75!(B) $) coordinate [label=-45:\F] (F);

  \node (H) at (B) [name path=H,help lines,circle through=(C),draw,label=135:\H] {};
  \path [name path=B--F] (B) -- (F);
  \path [name intersections={of=H and B--F,by={[label=right:\G]G}}];

  \node (K) at (D) [name path=K,help lines,circle through=(G),draw,label=135:\K] {};
  \path [name path=A--E] (A) -- (E);
  \path [name intersections={of=K and A--E,by={[label=below:\L]L}}];

  \draw [output] (A) -- (L);

  \foreach \point in {A,B,C,D,G,L}
    \fill [black,opacity=.5] (\point) circle (2pt);

  % \node ...
\end{tikzpicture}
\end{codeexample}
% Copyright 2006 by Till Tantau
%
% This file may be distributed and/or modified
%
% 1. under the LaTeX Project Public License and/or
% 2. under the GNU Free Documentation License.
%
% See the file doc/generic/pgf/licenses/LICENSE for more details.


\section{Tutorial: Putting a Diagram in Chains}

In this tutorial we have a look at how chains and matrices can be used
to typeset a diagram.

Ilka, who has just got tenured for her professorship of Old and
Lovable Programming Languages, has recently dug up an interesting
diagram in the dusty cellar of the library of her university. Having
been created in the good old times using a pen and a rules, it looks
like this:

\bigskip
\begin{tikzpicture}[
  >=stealth,thick,
  /pgf/every decoration/.style={/tikz/sharp corners},
  fuzzy/.style={decorate,decoration={post length=2pt,random steps,segment length=0.5mm,amplitude=0.15pt}},
  minimum size=6mm,line join=round,
  terminal/.style={rectangle,draw,fill=white,ultra thick,fuzzy,rounded corners=3mm},
  nonterminal/.style={rectangle,draw,fill=white,ultra thick,fuzzy},
  node distance=3mm]

  \ttfamily
  \begin{scope}[start chain,
                every node/.style={on chain},
                terminal/.append style={join=by {->,shorten >=1pt,fuzzy}},
                nonterminal/.append style={join=by {->,shorten >=1pt,fuzzy}},
                support/.style={coordinate,join=by fuzzy}]
    \node [support]             (start)        {};
    \node [nonterminal]                        {unsigned integer};
    \node [support]             (after ui)     {};
    \node [terminal]                           {.};
    \node [support]             (after dot)    {};
    \node [terminal]                           {digit};
    \node [support]             (after digit)  {};
    \node [support]             (skip)         {};    
    \node [support]             (before E)     {};
    \node [terminal]                           {E};
    \node [support]             (after E)      {};
    \node [support,xshift=5mm]  (between)      {};
    \node [support,xshift=5mm]  (before last)  {};
    \node [nonterminal]                        {unsigned integer};
    \node [support]             (after last)   {};
    \node [join=by ->]          (end)          {};
  \end{scope}
  \node (plus)  [terminal,above=of between] {$+$};
  \node (minus) [terminal,below=of between] {$-$};

  \begin{scope}[->,shorten >=1pt,rounded corners=2mm,every path/.style=fuzzy]
    \draw (after ui)    -- +(0,.7)  -| (skip);
    \draw (after digit) -- +(0,-.7) -| (after dot);
    \draw (before E)    -- +(0,-1.2) -| (after last);
    \draw (after E)     |- (plus);
    \draw (plus)        -| (before last);
    \draw (after E)     |- (minus);
    \draw (minus)       -| (before last);
  \end{scope}
\end{tikzpicture}
\bigskip

Ilka decides to redo this diagram in ``cleaner'' fashion.


% Copyright 2008 by Till Tantau
%
% This file may be distributed and/or modified
%
% 1. under the LaTeX Project Public License and/or
% 2. under the GNU Free Documentation License.
%
% See the file doc/generic/pgf/licenses/LICENSE for more details.


\section{Tutorial: A Lecture Map for Johannes}

In this tutorial we explore the tree and mind map mechanisms of
\tikzname.

Johannes is quite excited: For the first time he will be teaching a
course all by himself during the upcoming semester! Unfortunately, the
course is not on his favorite subject, which is of course Theoretical Immunology,
but on Complexity Theory, but as a young academic Johannes is not
likely to complain too loudly. In order to help the students get a
general overview of what is going to happen during the course as a
whole, he intends to draw some kind of tree or graph containing the
basic concepts. He got this idea from his old professor who seems to
be using these ``lecture maps'' with some success. Independently of
the success of these maps, Johannes thinks they look quite neat.



\subsection{Problem Statement}

Johannes wishes to create a lecture map with the following features:
\begin{enumerate}
\item It should contain a tree or graph depicting the main concepts.
\item It should somehow visualize the different lectures that will be
  taught. Note that the lectures are not necessarily the same as the
  concepts since the graph may contain more concepts than will be
  addressed in lectures and some concepts may be addressed during more
  than one lecture.
\item The map should also contain a calendar showing when the
  individual lectures will be given.
\item The aesthetical reasons, the whole map should have a visually
  nice and information-rich background.
\end{enumerate}

As always, Johannes will have to include the right libraries and
setup the environment. Since Johannes is going to use the
|mindmap| library and since he wishes to show a calendar, he will need
the |mindmap| and the |calendar| libraries. In order to put something
on a background layer, it seems like a good idea to also include the
|background| library.


\subsection{Introduction to Trees}

The first choice Johannes must make is whether he will organize the
concepts are a tree, with root concepts and concept branches and leaf
concepts, or as a general graph. The tree implicitly organizes the
concepts, while a graph is more flexible. Johannes decides to
compromise: Basically, the concepts will be organized as a
tree. However, he will selectively add connections between concepts
that are related, but which appear on different levels or branches of
the tree.

Johannes starts with a tree-like list of concepts that he feels are
important in Computational Complexity:

\begin{itemize}
\item Computational Problems
  \begin{itemize}\itemsep=0pt\parskip=0pt
  \item Problem Measures
  \item Problem Aspects
  \item Problem Domains
  \item Key Problems
  \end{itemize}
\item Computational Models
  \begin{itemize}\itemsep=0pt\parskip=0pt
  \item Turing Machines
  \item Random-Access Machines
  \item Circuits
  \item Binary Decision Diagrams
  \item Oracle Machines
  \item Programming in Logic
  \end{itemize}
\item Measuring Complexity
  \begin{itemize}\itemsep=0pt\parskip=0pt
  \item Complexity Measures
  \item Classifying Complexity
  \item Comparing Complexity
  \item Describing Complexity
  \end{itemize}
\item Solving Problems
  \begin{itemize}\itemsep=0pt\parskip=0pt
  \item Exact Algorithms
  \item Randomization
  \item Fixed-Parameter Algorithms
  \item Parallel Computation
  \item Partial Solutions
  \item Approximation
  \end{itemize}
\end{itemize}

Johannes will surely need to modify this list later on, but it looks
good as a first approximation. He will also need to add a number of
subtopics (like \emph{lots} of complexity classes under the topic
``classifying complexity''), but he will do this as he constructs the
map. 

Turning the list of topics into a \tikzname-tree is easy, in
principle. The basic idea is that a node can have \emph{children},
which in turn can have children of their own, and so on. To add a
child to a node, Johannes can simply write |child {|\meta{node}|}|
right after a node. The \meta{node} should, in turn, be the code for
creating a node. To add another node, Johannes can use |child| once
more, and so on. Johannes is eager to try out this construct and
writes down the following:

\begin{codeexample}[]
\tikz
  \node {Computational Complexity} % root
    child { node {Computational Problems}
      child { node {Problem Measures} }
      child { node {Problem Aspects} }
      child { node {Problem Domains} }
      child { node {Key Problems} }
    }
    child { node {Computational Models}
      child { node {Turing Machines} }
      child { node {Random-Access Machines} }
      child { node {Circuits} }
      child { node {Binary Decision Diagrams} }
      child { node {Oracle Machines} }
      child { node {Programming in Logic} }
    }
    child { node {Measuring Complexity} 
      child { node {Complexity Measures} }
      child { node {Classifying Complexity} }
      child { node {Comparing Complexity} }
      child { node {Describing Complexity} }
    }
    child { node {Solving Problems}
      child { node {Exact Algorithms} }
      child { node {Randomization} }
      child { node {Fixed-Parameter Algorithms} }
      child { node {Parallel Computation} }
      child { node {Partial Solutions} }
      child { node {Approximation} }
    };    
\end{codeexample}

Well, that did not quite work out as expected (although, what,
exactly, did one expect?). There are two problems:
\begin{enumerate}
\item The overlap of the nodes is due to the fact that \tikzname\ is
  not particularly smart when it comes to placing child nodes. Even
  though it is possible to configure \tikzname\ to use rather clever
  placement methods, \tikzname\ has no way of taking the actual size
  of the child nodes into account. This may seem strange but the
  reason is that the child nodes are rendered and placed one at a
  time, so the size of the last node is not known when the first node
  is being processed. In essence, you have to specify appropriate
  level and sibling node spacings ``by hand.''
\item The standard computer-science-top-down rendering of a tree is
  rather ill-suited to visualizing the concepts. It would be better to
  either rotate the map by ninety degrees or, even better, to use some
  sort of circular arrangement.
\end{enumerate}

Johannes redraws the tree, but this time with some more appropriate
options set, which he found more or less by trial-and-error:

\begin{codeexample}[render instead={
\tikz [font=\footnotesize,
       grow=right, level 1/.style={sibling distance=6em},
                   level 2/.style={sibling distance=1em}, level distance=5cm]
  \node {Computational Complexity} % root
    child { node {Computational Problems}
      child { node {Problem Measures} }           child { node {Problem Aspects} }
      child { node {Problem Domains} }            child { node {Key Problems} }
    }
    child { node {Computational Models}
      child { node {Turing Machines} }            child { node {Random-Access Machines} }
      child { node {Circuits} }                   child { node {Binary Decision Diagrams} }
      child { node {Oracle Machines} }            child { node {Programming in Logic} }
    }
    child { node {Measuring Complexity} 
      child { node {Complexity Measures} }        child { node {Classifying Complexity} }
      child { node {Comparing Complexity} }       child { node {Describing Complexity} }
    }
    child { node {Solving Problems}
      child { node {Exact Algorithms} }           child { node {Randomization} }
      child { node {Fixed-Parameter Algorithms} } child { node {Parallel Computation} }
      child { node {Partial Solutions} }          child { node {Approximation} }
    };    
    }]
\tikz [font=\footnotesize,
       grow=right, level 1/.style={sibling distance=6em},
                   level 2/.style={sibling distance=1em}, level distance=5cm]
  \node {Computational Complexity} % root
    child { node {Computational Problems}
      child { node {Problem Measures} }
      child { node {Problem Aspects} }
      ... % as before
\end{codeexample}

Still not quite what Johannes had in mind, but he is getting
somewhere.

For configuring the tree, two parameters are of particular importance:
The |level distance| tells \tikzname\ the distance between (the
centers of) the nodes on adjacent levels or layers of a tree. The
|sibling distance| is, as the name suggests, the distance between (the
centers of) siblings of the tree. 

You can globally set these parameters for a tree by simply setting
them somewhere before the tree starts, but you will
typically wish them to be different for different levels of the
tree. In this case, you should set styles like |level 1| or
|level 2|. For the first level of the tree, the |level 1| style is
used, for the second level the |level 2| style, and so on. You can
also set the sibling and level distances only for certain nodes by
passing these options to the |child| command as options. (Note that
the options of a |node| command are local to the node and have no
effect on the children. Also note that it is possible to specify
options that do have an effect on the children. Finally note that
specifying options for children ``at the right place'' is an arcane
art and you should peruse Section~\ref{section-tree-options} on 
a rainy Sunday afternoon, if you are really interested.)

The |grow| key is used to configure the direction in which a tree
grows. You can change growth direction ``in the middle of a tree''
simply by changing this key for a single child or a whole level. By
including the |tree| library you also get access to additional growth
strategies such as a ``circular'' growth:


\begin{codeexample}[render instead={
\tikz [text width=2.7cm, align=flush center,
       grow cyclic, 
       level 1/.style={level distance=2.5cm,sibling angle=90},
       level 2/.style={text width=2cm, font=\footnotesize, level distance=3cm,sibling angle=30}]
  \node[font=\bfseries] {Computational Complexity} % root
    child { node {Computational Problems}
      child { node {Problem Measures} }           child { node {Problem Aspects} }
      child { node {Problem Domains} }            child { node {Key Problems} }
    }
    child { node {Computational Models}
      child { node {Turing Machines} }            child { node {Random-Access Machines} }
      child { node {Circuits} }                   child { node {Binary Decision Diagrams} }
      child { node {Oracle Machines} }            child { node {Programming in Logic} }
    }
    child { node {Measuring Complexity} 
      child { node {Complexity Measures} }        child { node {Classifying Complexity} }
      child { node {Comparing Complexity} }       child { node {Describing Complexity} }
    }
    child { node {Solving Problems}
      child { node {Exact Algorithms} }           child { node {Randomization} }
      child { node {Fixed-Parameter Algorithms} } child { node {Parallel Computation} }
      child { node {Partial Solutions} }          child { node {Approximation} }
    };    
    }]
\tikz [text width=2.7cm, align=flush center,
       grow cyclic, 
       level 1/.style={level distance=2.5cm,sibling angle=90},
       level 2/.style={text width=2cm, font=\footnotesize, level distance=3cm,sibling angle=30}]
  \node[font=\bfseries] {Computational Complexity} % root
    child { node {Computational Problems}
      child { node {Problem Measures} }
      child { node {Problem Aspects} }
      ... % as before
\end{codeexample}


Johannes is pleased to learn that he can access and manipulate the
nodes of tree like any normal node. In particular, he can name them
using the |name=| option or the |(|\meta{name}|)| notation and he can
use any available shape or style for the trees nodes. He can connect
trees later on using the normal |\draw (some node) -- (another node);|
syntax. In essence, the |child| command just computes an appropriate
position for a node and adds a line from the child to the parent
node. 


\subsection{Creating the Lecture Map}

Johannes now has a first possible layout for his lecture map. The next
step is to make it ``look nicer.'' For this, the |mindmap| library is
helpful since it makes a number of styles available that will make a
tree look like a nice ``mind map'' or ``concept map.''

The first step is to include the |mindmap| library, which Johannes
already did. Next, he must add one of the following options to a scope
that will contain the lecture map: |mindmap| or |large mindmap| or
|huge mindmap|. These options all have the same effect, except that
for a |large mindmap| the predefined font size and node sizes are
somewhat larger than for a standard |mindmap| and for a |huge mindmap|
they are even larger. So, a |large mindmap| does not necessarily need
to have a lot of concepts, but it will need a lot of paper.

The second step is to add the |concept| option to every node that
will, indeed, be a concept of the mindmap. The idea is that some nodes
of a tree will be real concepts, while other nodes might just be
``simple children.'' Typically, this is not the case, so you might
consider saying |every node/.style=concept|.

The third step is to setup the sibling \emph{angle} (rather than a
sibling distance) to specify the angle between sibling concepts.

\begin{codeexample}[render instead={
\tikz [mindmap, every node/.style=concept, concept color=black!20,
       grow cyclic, 
       level 1/.append style={level distance=4.5cm,sibling angle=90},
       level 2/.append style={level distance=3cm,sibling angle=45}]
  \node [root concept] {Computational Complexity} % root
    child { node {\hbox to 2cm{Computational\hss} Problems}
      child { node {Problem Measures} }
      child { node {Problem Aspects} }
      child { node {Problem Domains} }
      child { node {Key Problems} }
    }
    child { node {\hbox to 2cm{Computational\hss} Models}
      child { node {Turing Machines} }
      child { node {Random-Access Machines} }
      child { node {Circuits} }
      child { node {Binary Decision Diagrams} }
      child { node {Oracle Machines} }
      child { node {\hbox to1.5cm{Programming\hss} in Logic} }
    }
    child { node {Measuring Complexity} 
      child { node {Complexity Measures} }
      child { node {Classifying Complexity} }
      child { node {Comparing Complexity} }
      child { node {Describing Complexity} }
    }
    child { node {Solving Problems}
      child { node {Exact Algorithms} }
      child { node {\hbox to 1.5cm{Randomization\hss}} }
      child { node {Fixed-Parameter Algorithms} }
      child { node {Parallel Computation} }
      child { node {Partial Solutions} }
      child { node {\hbox to1.5cm{Approximation\hss}} }
    };}]
\tikz [mindmap, every node/.style=concept, concept color=black!20,
       grow cyclic, 
       level 1/.append style={level distance=4.5cm,sibling angle=90},
       level 2/.append style={level distance=3cm,sibling angle=45}]
  \node [root concept] {Computational Complexity} % root
    child { node {Computational Problems}
      child { node {Problem Measures} }
      child { node {Problem Aspects} }
      ... % as before
\end{codeexample}

When Johannes typesets the above map, \TeX\ (rightfully) starts
complaining about several overfull boxes and, indeed, words like
``Randomization'' stretch out beyond the circle of the concept. This
seems a bit mysterious at first sight: Why does \TeX\ not hyphenate
the word? The reason is that \TeX\ will never hyphenate the first word
of a paragraph because it starts looking for ``hyphenatable'' letters
only after a so-called glue. In order to have \TeX\ hyphenate these
single words, Johannes must use a bit of evil trickery: He inserts a
|\hskip0pt| before the word. This has no effect except for inserting
an (invisible) glue before the word and, thereby, allowing \TeX\ to
hyphenate the first word also. Since Johannes does not want to add
|\hskip0pt| inside each node, he uses the |execute at begin node|
option to make \tikzname\ insert this text with every node.


\begin{codeexample}[render instead={
\begin{tikzpicture}
  [mindmap,
   every node/.style={concept, execute at begin node=\hskip0pt},
   concept color=black!20,
   grow cyclic, 
   level 1/.append style={level distance=4.5cm,sibling angle=90},
   level 2/.append style={level distance=3cm,sibling angle=45}]
  \clip (-1,2) rectangle ++ (-4,5);
  \node [root concept] {Computational Complexity} % root
    child { node {Computational Problems}
      child { node {Problem Measures} }
      child { node {Problem Aspects} }
      child { node {Problem Domains} }
      child { node {Key Problems} }
    }
    child { node {Computational Models}
      child { node {Turing Machines} }
      child { node {Random-Access Machines} }
      child { node {Circuits} }
      child { node {Binary Decision Diagrams} }
      child { node {Oracle Machines} }
      child { node {Programming in Logic} }
    }
    child { node {Measuring Complexity} 
      child { node {Complexity Measures} }
      child { node {Classifying Complexity} }
      child { node {Comparing Complexity} }
      child { node {Describing Complexity} }
    }
    child { node {Solving Problems}
      child { node {Exact Algorithms} }
      child { node {Randomization} }
      child { node {Fixed-Parameter Algorithms} }
      child { node {Parallel Computation} }
      child { node {Partial Solutions} }
      child { node {Approximation} }
    };
\end{tikzpicture}
}]
\begin{tikzpicture}
  [mindmap,
   every node/.style={concept, execute at begin node=\hskip0pt},
   concept color=black!20,
   grow cyclic, 
   level 1/.append style={level distance=4.5cm,sibling angle=90},
   level 2/.append style={level distance=3cm,sibling angle=45}]
  \clip (-1,2) rectangle ++ (-4,5);
  \node [root concept] {Computational Complexity} % root
    child { node {Computational Problems}
      child { node {Problem Measures} }
      child { node {Problem Aspects} }
      ... % as before
\end{tikzpicture}
\end{codeexample}


In the above example a clipping was used to show only part of the
lecture map, in order to save space. The same will be done in the
following examples, we return to the complete lecture map at the end of this
tutorial.

Johannes is now eager to colorize the map. The idea is to use
different colors for different parts of the map. He can then, during
his lectures, talk about the ``green'' or the ``red'' topics. This
will make it easier for his students to locate the topic he is talking
about on the map. Since ``computational problems'' somehow sounds
``problematic,'' Johannes chooses red for them, while he picks green
for the ``solving problems.'' The topics ``measuring complexity'' and
``computational models'' get more neutral colors; Johannes picks
orange and blue.

To set the colors, Johannes must use the |concept color| option,
rather than just, say, |node [fill=red]|. Setting just the fill color
to |red| would, indeed, make the node red, but it would \emph{just}
make the node red and not the bar connecting the concept to its parent
and also not its children. By comparison, the special |concept color|
option will not only set the color of the node and its children, but
it will also (magically) create appropriate shadings so that the color
of a parent concept smoothly changes to the color of a child concept.

For the root concept Johannes decides to do something special: He sets
the concept color to black, sets the line width to a large value, and
sets the fill color to white. The effect of this is that the root
concept will encircled with a thick black line and the children are
connected to the central concept via bars.

\begin{codeexample}[render instead={
\begin{tikzpicture}
  [mindmap,
   every node/.style={concept, execute at begin node=\hskip0pt},
   root concept/.append style={
     concept color=black,
     fill=white, line width=1ex,
     text=black},
   text=white,
   grow cyclic,
   level 1/.append style={level distance=4.5cm,sibling angle=90},
   level 2/.append style={level distance=3cm,sibling angle=45}]
  \clip (0,-1) rectangle ++(4,5);
  \node [root concept] {Computational Complexity} % root
    child [concept color=red] { node {Computational Problems}
      child { node {Problem Measures} }
      child { node {Problem Aspects} }
      child { node {Problem Domains} }
      child { node {Key Problems} }
    }
    child [concept color=blue] { node {Computational Models}
      child { node {Turing Machines} }
      child { node {Random-Access Machines} }
      child { node {Circuits} }
      child { node {Binary Decision Diagrams} }
      child { node {Oracle Machines} }
      child { node {Programming in Logic} }
    }
    child [concept color=orange] { node {Measuring Complexity} 
      child { node {Complexity Measures} }
      child { node {Classifying Complexity} }
      child { node {Comparing Complexity} }
      child { node {Describing Complexity} }
    }
    child [concept color=green!50!black] { node {Solving Problems}
      child { node {Exact Algorithms} }
      child { node {Randomization} }
      child { node {Fixed-Parameter Algorithms} }
      child { node {Parallel Computation} }
      child { node {Partial Solutions} }
      child { node {Approximation} }
    };
  \end{tikzpicture}}]
\begin{tikzpicture}
  [mindmap,
   every node/.style={concept, execute at begin node=\hskip0pt},
   root concept/.append style={
     concept color=black, fill=white, line width=1ex, text=black},
   text=white,
   grow cyclic,
   level 1/.append style={level distance=4.5cm,sibling angle=90},
   level 2/.append style={level distance=3cm,sibling angle=45}]
   \clip (0,-1) rectangle ++(4,5);
  \node [root concept] {Computational Complexity} % root
    child [concept color=red] { node {Computational Problems}
      child { node {Problem Measures} }
      ... % as before
    }
    child [concept color=blue] { node {Computational Models}
      child { node {Turing Machines} }
      ... % as before
    }
    child [concept color=orange] { node {Measuring Complexity} 
      child { node {Complexity Measures} }
      ... % as before
    }
    child [concept color=green!50!black] { node {Solving Problems}
      child { node {Exact Algorithms} }
      ... % as before
    };
\end{tikzpicture}
\end{codeexample}

Johannes adds three finishing touches: First, he changes the font
of the main concepts to small caps. Second, he decides that some
concepts should be ``faded,'' namely those that are important in
principle and belong on the map, but which he will not talk about in
his lecture. To achieve this, Johannes defines four styles, one for
each of the four main branches. These styles (a) setup the
correct concept color for the whole branch and (b) define the |faded|
style appropriately for this branch. Third, he adds a
|circular drop shadow|, defined in the |shadows| library, to the
concepts, just to make things look a bit more fancy. 

\begin{codeexample}[render instead={
\begin{tikzpicture}[mindmap]
  \begin{scope}[      
   every node/.style={concept, circular drop shadow,execute at begin node=\hskip0pt},
   root concept/.append style={
     concept color=black,
     fill=white, line width=1ex,
     text=black, font=\large\scshape},
   text=white,
   computational problems/.style={concept color=red,faded/.style={concept color=red!50}},
   computational models/.style={concept color=blue,faded/.style={concept color=blue!50}},
   measuring complexity/.style={concept color=orange,faded/.style={concept color=orange!50}},
   solving problems/.style={concept color=green!50!black,faded/.style={concept color=green!50!black!50}},
   grow cyclic,
   level 1/.append style={level distance=4.5cm,sibling angle=90,font=\scshape},
   level 2/.append style={level distance=3cm,sibling angle=45,font=\scriptsize}]
  \node [root concept] {Computational Complexity} % root
    child [computational problems] { node {Computational Problems}
      child         { node {Problem Measures} }
      child         { node {Problem Aspects} }
      child [faded] { node {Problem Domains} }
      child         { node {Key Problems} }
    }
    child [computational models] { node {Computational Models}
      child         { node {Turing Machines} }
      child [faded] { node {Random-Access Machines} }
      child         { node {Circuits} }
      child [faded] { node {Binary Decision Diagrams} }
      child         { node {Oracle Machines} }
      child         { node {Programming in Logic} }
    }
    child [measuring complexity] { node {Measuring Complexity} 
      child         { node {Complexity Measures} }
      child         { node {Classifying Complexity} }
      child         { node {Comparing Complexity} }
      child [faded] { node {Describing Complexity} }
    }
    child [solving problems] { node {Solving Problems}
      child         { node {Exact Algorithms} }
      child         { node {Randomization} }
      child         { node {Fixed-Parameter Algorithms} }
      child         { node {Parallel Computation} }
      child         { node {Partial Solutions} }
      child         { node {Approximation} }
    };
  \end{scope}
\end{tikzpicture}}]
\begin{tikzpicture}[mindmap]
  \begin{scope}[
    every node/.style={concept, circular drop shadow,execute at begin node=\hskip0pt},
    root concept/.append style={
      concept color=black, fill=white, line width=1ex, text=black, font=\large\scshape},
    text=white,
    computational problems/.style={concept color=red,faded/.style={concept color=red!50}},
    computational models/.style={concept color=blue,faded/.style={concept color=blue!50}},
    measuring complexity/.style={concept color=orange,faded/.style={concept color=orange!50}},
    solving problems/.style={concept color=green!50!black,faded/.style={concept color=green!50!black!50}},
    grow cyclic,
    level 1/.append style={level distance=4.5cm,sibling angle=90,font=\scshape},
    level 2/.append style={level distance=3cm,sibling angle=45,font=\scriptsize}]
    \node [root concept] {Computational Complexity} % root
      child [computational problems] { node {Computational Problems}
        child         { node {Problem Measures} }
        child         { node {Problem Aspects} }
        child [faded] { node {Problem Domains} }
        child         { node {Key Problems} }
      }
      child [computational models] { node {Computational Models}
        child         { node {Turing Machines} }
        child [faded] { node {Random-Access Machines} }
        ...
  \end{scope}
\end{tikzpicture}
\end{codeexample}


\subsection{Adding the Lecture Annotations}

Johannes will give about a dozen lectures during the course
``computational complexity.'' For each lecture he has compiled a
(short) list of learning targets that state what knowledge and
qualifications his students should acquire during this particular
lecture (note that learning targets are not the same as the contents
of a lecture). For each lecture he intends to put a little rectangle
on the map containing these learning targets and the name of the
lecture, each time somewhere near to the topic of the lecture. Such
``little rectangles'' are called ``annotations'' by the mindmap
library. 

In order to place the annotations next to the concepts, Johannes must
assign names to the nodes of the concepts. He could rely on
\tikzname's automatic naming of the nodes in a tree, where the
children of a node named |root| are named |root-1|, |root-2|,
|root-3|, and so on. However, since Johannes is not sure about the
final order of the concepts in the tree, it seems better to explicitly
name all concepts of the tree in the following manner:

\begin{codeexample}[code only]
\node [root concept] (Computational Complexity) {Computational Complexity} 
  child [computational problems] { node (Computational Problems) {Computational Problems}
    child         { node (Problem Measures) {Problem Measures} }
    child         { node (Problem Aspects) {Problem Aspects} }
    child [faded] { node (Problem Domains) {Problem Domains} }
    child         { node (Key Problems) {Key Problems} }
  }
...
\end{codeexample}

The |annotation| style of the mind map library mainly sets up a
rectangular shape of appropriate size. Johannes configures the style
by defining |every annotation| appropriately.

\begin{codeexample}[render instead={
\begin{tikzpicture}[mindmap]
  \clip (-5.25,-3) rectangle ++ (4,5);
  \begin{scope}[
    every node/.style={concept, circular drop shadow,execute at begin node=\hskip0pt},
    root concept/.append style={
      concept color=black,
      fill=white, line width=1ex,
      text=black, font=\large\scshape},
    text=white,
    computational problems/.style={concept color=red,faded/.style={concept color=red!50}},
    computational models/.style={concept color=blue,faded/.style={concept color=blue!50}},
    measuring complexity/.style={concept color=orange,faded/.style={concept color=orange!50}},
    solving problems/.style={concept color=green!50!black,faded/.style={concept color=green!50!black!50}},
    grow cyclic,
    level 1/.append style={level distance=4.5cm,sibling angle=90,font=\scshape},
    level 2/.append style={level distance=3cm,sibling angle=45,font=\scriptsize}]
    \node [root concept] (Computational Complexity) {Computational Complexity} % root
      child [computational problems] { node (Computational Problems) {Computational Problems}
        child         { node (Problem Measures) {Problem Measures} }
        child         { node (Problem Aspects) {Problem Aspects} }
        child [faded] { node (problem Domains) {Problem Domains} }
        child         { node (Key Problems) {Key Problems} }
      }
      child [computational models] { node (Computational Models) {Computational Models}
        child         { node (Turing Machines) {Turing Machines} }
        child [faded] { node (Random-Access Machines) {Random-Access Machines} }
        child         { node (Circuits) {Circuits} }
        child [faded] { node (Binary Decision Diagrams) {Binary Decision Diagrams} }
        child         { node (Oracle Machines) {Oracle Machines} }
        child         { node (Programming in Logic) {Programming in Logic} }
      }
      child [measuring complexity] { node (Measuring Complexity) {Measuring Complexity} 
        child         { node (Complexity Measures) {Complexity Measures} }
        child         { node (Classifying Complexity) {Classifying Complexity} }
        child         { node (Comparing Complexity) {Comparing Complexity} }
        child [faded] { node (Describing Complexity) {Describing Complexity} }
      }
      child [solving problems] { node (Solving Problems) {Solving Problems}
        child         { node (Exact Algorithms) {Exact Algorithms} }
        child         { node (Randomization) {Randomization} }
        child         { node (Fixed-Parameter Algorithms) {Fixed-Parameter Algorithms} }
        child         { node (Parallel Computation) {Parallel Computation} }
        child         { node (Partial Solutions) {Partial Solutions} }
        child         { node (Approximation) {Approximation} }
      };
  \end{scope}
  \begin{scope}[every annotation/.style={fill=black!40}]
    \node [annotation, above] at (Computational Problems.north) {
      Lecture 1: Computational Problems
      \begin{itemize}
      \item Knowledge of several key problems
      \item Knowledge of problem encondings
      \item Being able to formalize problems
      \end{itemize}
    };
  \end{scope}
\end{tikzpicture}}]
\begin{tikzpicture}[mindmap]
  \clip (-5,-5) rectangle ++ (4,5);
  \begin{scope}[
     every node/.style={concept, circular drop shadow, ...] % as before
    \node [root concept] (Computational Complexity)    ... % as before
  \end{scope}
  
  \begin{scope}[every annotation/.style={fill=black!40}]
    \node [annotation, above] at (Computational Problems.north) {
      Lecture 1: Computational Problems
      \begin{itemize}
      \item Knowledge of several key problems
      \item Knowledge of problem encondings
      \item Being able to formalize problems
      \end{itemize}
    };
  \end{scope}
\end{tikzpicture}        
\end{codeexample}

Well, that does not yet look quite perfect. The spacing or the
|{itemize}| is not really appropriate and the node is too
large. Johannes can configure these things ``by hand,'' but it seems
like a good idea to define a macro that will take care of these things
for him. The ``right'' way to do this is to define a |\lecture| macro
that takes a list of key-value pairs as argument and produces the
desired annotation. However, to keep things simple, Johannes'
|\lecture| macro simply takes a fixed number of arguments having the
following meaning: The first argument is the number of the lecture,
the second is the name of the lecture, the third are positioning
options like |above|, the fourth is the position where the node is
placed, the fifth is the list of items to be shown, and the sixth is a
date when the lecture will be held (this parameter is not yet needed,
we will, however, need it later on).

\begin{codeexample}[code only]
\def\lecture#1#2#3#4#5#6{
  \node [annotation, #3, scale=0.65, text width=4cm, inner sep=2mm] at (#4) {
    Lecture #1: \textcolor{orange}{\textbf{#2}}
    \list{--}{\topsep=2pt\itemsep=0pt\parsep=0pt
              \parskip=0pt\labelwidth=8pt\leftmargin=8pt
              \itemindent=0pt\labelsep=2pt}
    #5
    \endlist
  };
}
\end{codeexample}
\def\lecture#1#2#3#4#5#6{
  \node [annotation, #3, scale=0.65, text width=4cm, inner sep=2mm] at (#4) {
    Lecture #1: \textcolor{orange}{\textbf{#2}}
    \list{--}{\topsep=2pt\itemsep=0pt\parsep=0pt
              \parskip=0pt\labelwidth=8pt\leftmargin=8pt
              \itemindent=0pt\labelsep=2pt}
    #5
    \endlist
  };
}

\begin{codeexample}[render instead={
\begin{tikzpicture}[mindmap,every annotation/.style={fill=white}]
  \clip (-5.25,-3) rectangle ++ (4,5);
  \begin{scope}[
    every node/.style={concept, circular drop shadow,execute at begin node=\hskip0pt},
    root concept/.append style={
      concept color=black,
      fill=white, line width=1ex,
      text=black, font=\large\scshape},
    text=white,
    computational problems/.style={concept color=red,faded/.style={concept color=red!50}},
    computational models/.style={concept color=blue,faded/.style={concept color=blue!50}},
    measuring complexity/.style={concept color=orange,faded/.style={concept color=orange!50}},
    solving problems/.style={concept color=green!50!black,faded/.style={concept color=green!50!black!50}},
    grow cyclic,
    level 1/.append style={level distance=4.5cm,sibling angle=90,font=\scshape},
    level 2/.append style={level distance=3cm,sibling angle=45,font=\scriptsize}]
    \node [root concept] (Computational Complexity) {Computational Complexity} % root
      child [computational problems] { node (Computational Problems) {Computational Problems}
        child         { node (Problem Measures) {Problem Measures} }
        child         { node (Problem Aspects) {Problem Aspects} }
        child [faded] { node (problem Domains) {Problem Domains} }
        child         { node (Key Problems) {Key Problems} }
      }
      child [computational models] { node (Computational Models) {Computational Models}
        child         { node (Turing Machines) {Turing Machines} }
        child [faded] { node (Random-Access Machines) {Random-Access Machines} }
        child         { node (Circuits) {Circuits} }
        child [faded] { node (Binary Decision Diagrams) {Binary Decision Diagrams} }
        child         { node (Oracle Machines) {Oracle Machines} }
        child         { node (Programming in Logic) {Programming in Logic} }
      }
      child [measuring complexity] { node (Measuring Complexity) {Measuring Complexity} 
        child         { node (Complexity Measures) {Complexity Measures} }
        child         { node (Classifying Complexity) {Classifying Complexity} }
        child         { node (Comparing Complexity) {Comparing Complexity} }
        child [faded] { node (Describing Complexity) {Describing Complexity} }
      }
      child [solving problems] { node (Solving Problems) {Solving Problems}
        child         { node (Exact Algorithms) {Exact Algorithms} }
        child         { node (Randomization) {Randomization} }
        child         { node (Fixed-Parameter Algorithms) {Fixed-Parameter Algorithms} }
        child         { node (Parallel Computation) {Parallel Computation} }
        child         { node (Partial Solutions) {Partial Solutions} }
        child         { node (Approximation) {Approximation} }
      };
  \end{scope}
  \lecture{1}{Computational Problems}{above,xshift=-3mm}{Computational Problems.north}{
    \item Knowledge of several key problems
    \item Knowledge of problem encondings
    \item Being able to formalize problems
  }{2009-04-08}
\end{tikzpicture}}]
\begin{tikzpicture}[mindmap,every annotation/.style={fill=white}]
  \clip (-5,-5) rectangle ++ (4,5);
  \begin{scope}[
     every node/.style={concept, circular drop shadow, ... % as before
    \node [root concept] (Computational Complexity)    ... % as before
  \end{scope}
  
  \lecture{1}{Computational Problems}{above,xshift=-3mm}
  {Computational Problems.north}{
    \item Knowledge of several key problems
    \item Knowledge of problem encondings
    \item Being able to formalize problems
  }{2009-04-08}
\end{tikzpicture}        
\end{codeexample}

In the same fashion Johannes can now add the other lecture
annotations. Obviously, Johannes will have some trouble fitting
everything on a single A4-sized page, but by adjusting the spacing and
some experimentation he can quickly arrange all the annotations as needed.


\subsection{Adding the Background}

Johannes has already used colors to organize his lecture map into four
regions, each having a different color. In order to emphasize these
regions even more strongly, he wishes to add a background coloring to
each of these regions.

Adding these background colors turns out to be more tricky than
Johannes would have thought. At first sight, what he needs is some
sort of ``color wheel'' that is blue in the lower right direction and
then changes smoothly to orange in the upper right direction and then
to green in the upper left direction and so on. Unfortunately, there
is no easy way of creating a true such a color wheel shading (although
it can be done, in principle, but only at a very high cost, see
page~\pageref{shading-color-wheel} for an example).

Johannes decides to do something a bit more basic: He creates four
large rectangles, one for each of the four quadrants around the
central concept, each colored with a light version of the
quadrant. Then, in order to ``smooth'' the change between adjacent
rectangles, he puts four shadings on top of them.

Since these background rectangles should go ``behind'' everything
else, Johannes puts all his background stuff on the |background|
layer.

In the following code, only the central concept is shown to save some
space:
\begin{codeexample}[]
\begin{tikzpicture}[
  mindmap,
  concept color=black,
  root concept/.append style={
    concept,
    circular drop shadow,
    fill=white, line width=1ex,
    text=black, font=\large\scshape}
  ]

  \clip (-1.5,-5) rectangle ++(4,10);

  \node [root concept] (Computational Complexity) {Computational Complexity};
  
  \begin{pgfonlayer}{background}
    \clip (-1.5,-5) rectangle ++(4,10);
    
    \colorlet{upperleft}{green!50!black!25}
    \colorlet{upperright}{orange!25}
    \colorlet{lowerleft}{red!25}
    \colorlet{lowerright}{blue!25}

     % The large rectangles:
    \fill [upperleft]  (Computational Complexity) rectangle ++(-20,20);
    \fill [upperright] (Computational Complexity) rectangle ++(20,20);
    \fill [lowerleft]  (Computational Complexity) rectangle ++(-20,-20);
    \fill [lowerright] (Computational Complexity) rectangle ++(20,-20);

    % The shadings:
    \shade [left color=upperleft,right color=upperright]
      ([xshift=-1cm]Computational Complexity) rectangle ++(2,20);
    \shade [left color=lowerleft,right color=lowerright]
      ([xshift=-1cm]Computational Complexity) rectangle ++(2,-20);
    \shade [top color=upperleft,bottom color=lowerleft]
      ([yshift=-1cm]Computational Complexity) rectangle ++(-20,2);
    \shade [top color=upperright,bottom color=lowerright]
      ([yshift=-1cm]Computational Complexity) rectangle ++(20,2);
  \end{pgfonlayer}
\end{tikzpicture}
\end{codeexample}



\subsection{Adding the Calendar}

Johannes intends to plan his lecture rather carefully. In particular,
he already knows when each of his lectures will be held during the
course. Naturally, this does not mean that Johannes will slavishly
follow the plan and he might need longer for some subjects than he
anticipated, but nevertheless he has a detailed plan of when which
subject will be addressed.

Johannes intends to share this plan with his students by adding a
calendar to the lecture map. In addition to serving as a reference
on which particular day a certain  topic will be addressed, the
calendar is also useful so show the overall chronological order of the
course.

In order to add a calendar to a \tikzname\ graphic, the |calendar|
library is most useful. The library provides the |\calendar| command,
which takes a large number of options and which can be configured in
many ways to produce just about any kind of calendar imaginable. For
Johannes' purposes, a simple |day list downward| will be a nice option
since it produces a list of days that go ``downward''.

\begin{codeexample}[leave comments]
\tiny
\begin{tikzpicture}
  \calendar [day list downward,
             name=cal,
             dates=2009-04-01 to 2009-04-14]
    if (weekend)
      [black!25];
\end{tikzpicture}
\end{codeexample}

Using the |name| option, we gave a name to the calendar, which will
allow us to reference the nodes that make up the individual days of
the calendar later on. For instance, the rectangular node containing the
|1| that represents April 1st, 2009, can be referenced as
|(cal-2009-04-01)|. The |dates| option is used to specify an 
interval for which the calendar should be drawn. Johannes will need
several months in his calendar, but the above example only shows two
weeks to save some space.

Note the |if (weekend)| construct. The |\calendar| command is followed
by options and then by |if|-statements. These |if|-statements are
checked for each day of the calendar and when a date passes this test,
the options or the code following the |if|-statement is executed. In
the above example, we make weekend days (Saturdays and Sundays, to be
precise) lighter than normal days. (Use your favorite calendar to
check that, indeed, April 5th, 2009, is a Sunday.)

As mentioned above, Johannes can reference the nodes that are used to
typeset days. Recall that his |\lecture| macro already got passed a
date, which we did not use, yet. We can now use it to place the
lecture's title next to the date when the lecture will be held:


\begin{codeexample}[code only]
\def\lecture#1#2#3#4#5#6{
  % As before:    
  \node [annotation, #3, scale=0.65, text width=4cm, inner sep=2mm] at (#4) {
    Lecture #1: \textcolor{orange}{\textbf{#2}}
    \list{--}{\topsep=2pt\itemsep=0pt\parsep=0pt
              \parskip=0pt\labelwidth=8pt\leftmargin=8pt
              \itemindent=0pt\labelsep=2pt}
    #5
    \endlist
  };
  % New:
  \node [anchor=base west] at (cal-#6.base east) {\textcolor{orange}{\textbf{#2}}};
}
\end{codeexample}
\def\lecture#1#2#3#4#5#6{
  \node [anchor=base west] at (cal-#6.base east) {\textcolor{orange}{\textbf{#2}}};
}

Johannes can now use this new |\lecture| command as follows (in the
example, only the new part of the definition is used):

\begin{codeexample}[]
\tiny
\begin{tikzpicture}
  \calendar [day list downward,
             name=cal,
             dates=2009-04-01 to 2009-04-14]
    if (weekend)
      [black!25];

  % As before:
  \lecture{1}{Computational Problems}{above,xshift=-3mm}
  {Computational Problems.north}{
    \item Knowledge of several key problems
    \item Knowledge of problem encondings
    \item Being able to formalize problems
  }{2009-04-08}      
\end{tikzpicture}
\end{codeexample}


As a final step, Johannes needs to add a few more options to the
calendar command: He uses the |month text| option to configure how the
text of a month is rendered (see Section~\ref{section-calender} for
details) and then typesets the month text at a special position at the
beginning of each month.
    
\begin{codeexample}[leave comments]
\tiny
\begin{tikzpicture}
  \calendar [day list downward,
             month text=\%mt\ \%y0,
             month yshift=3.5em,
             name=cal,
             dates=2009-04-01 to 2009-05-01]
    if (weekend)
      [black!25]
    if (day of month=1) {
      \node at (0pt,1.5em) [anchor=base west] {\small\tikzmonthtext};
    };
    
  \lecture{1}{Computational Problems}{above,xshift=-3mm}
  {Computational Problems.north}{
    \item Knowledge of several key problems
    \item Knowledge of problem encondings
    \item Being able to formalize problems
  }{2009-04-08}      
    
  \lecture{2}{Computational Models}{above,xshift=-3mm}
  {Computational Models.north}{
    \item Knowledge of Turing machines
    \item Being able to compare the computational power of different
      models 
  }{2009-04-15}      
\end{tikzpicture}
\end{codeexample}



\subsection{The Complete Code}

Putting it all together, Johannes gets the following code:

First comes the definition of the |\lecture| command:

\begin{codeexample}[code only]
\def\lecture#1#2#3#4#5#6{
  % As before:    
  \node [annotation, #3, scale=0.65, text width=4cm, inner sep=2mm, fill=white] at (#4) {
    Lecture #1: \textcolor{orange}{\textbf{#2}}
    \list{--}{\topsep=2pt\itemsep=0pt\parsep=0pt
              \parskip=0pt\labelwidth=8pt\leftmargin=8pt
              \itemindent=0pt\labelsep=2pt}
    #5
    \endlist
  };
  % New:
  \node [anchor=base west] at (cal-#6.base east) {\textcolor{orange}{\textbf{#2}}};
}  
\end{codeexample}

This is followed by the main mindmap setup\dots

\begin{codeexample}[code only]
\noindent
\begin{tikzpicture}
  \begin{scope}[
    mindmap,
    every node/.style={concept, circular drop shadow,execute at begin node=\hskip0pt},
    root concept/.append style={
      concept color=black,
      fill=white, line width=1ex,
      text=black, font=\large\scshape},
    text=white,
    computational problems/.style={concept color=red,faded/.style={concept color=red!50}},
    computational models/.style={concept color=blue,faded/.style={concept color=blue!50}},
    measuring complexity/.style={concept color=orange,faded/.style={concept color=orange!50}},
    solving problems/.style={concept color=green!50!black,faded/.style={concept color=green!50!black!50}},
    grow cyclic,
    level 1/.append style={level distance=4.5cm,sibling angle=90,font=\scshape},
    level 2/.append style={level distance=3cm,sibling angle=45,font=\scriptsize}]
\end{codeexample}
\dots and contents:
\begin{codeexample}[code only]  
  \node [root concept] (Computational Complexity) {Computational Complexity} % root
      child [computational problems] { node [yshift=-1cm] (Computational Problems) {Computational Problems}
        child         { node (Problem Measures) {Problem Measures} }
        child         { node (Problem Aspects) {Problem Aspects} }
        child [faded] { node (problem Domains) {Problem Domains} }
        child         { node (Key Problems) {Key Problems} }
      }
      child [computational models] { node [yshift=-1cm]  (Computational Models) {Computational Models}
        child         { node (Turing Machines) {Turing Machines} }
        child [faded] { node (Random-Access Machines) {Random-Access Machines} }
        child         { node (Circuits) {Circuits} }
        child [faded] { node (Binary Decision Diagrams) {Binary Decision Diagrams} }
        child         { node (Oracle Machines) {Oracle Machines} }
        child         { node (Programming in Logic) {Programming in Logic} }
      }
      child [measuring complexity] { node [yshift=1cm] (Measuring Complexity) {Measuring Complexity} 
        child         { node (Complexity Measures) {Complexity Measures} }
        child         { node (Classifying Complexity) {Classifying Complexity} }
        child         { node (Comparing Complexity) {Comparing Complexity} }
        child [faded] { node (Describing Complexity) {Describing Complexity} }
      }
      child [solving problems] { node [yshift=1cm] (Solving Problems) {Solving Problems}
        child         { node (Exact Algorithms) {Exact Algorithms} }
        child         { node (Randomization) {Randomization} }
        child         { node (Fixed-Parameter Algorithms) {Fixed-Parameter Algorithms} }
        child         { node (Parallel Computation) {Parallel Computation} }
        child         { node (Partial Solutions) {Partial Solutions} }
        child         { node (Approximation) {Approximation} }
      };
  \end{scope}
\end{codeexample}
Now comes the calendar code:
\begin{codeexample}[code only]
  \tiny
  \calendar [day list downward,
             month text=\%mt\ \%y0,
             month yshift=3.5em,
             name=cal,
             at={(-.5\textwidth-5mm,.5\textheight-1cm)},
             dates=2009-04-01 to 2009-06-last]
    if (weekend)
      [black!25]
    if (day of month=1) {
      \node at (0pt,1.5em) [anchor=base west] {\small\tikzmonthtext};
    };
\end{codeexample}
The lecture annotations:
\begin{codeexample}[code only]
  \lecture{1}{Computational Problems}{above,xshift=-5mm,yshift=5mm}{Computational Problems.north}{
    \item Knowledge of several key problems
    \item Knowledge of problem encondings
    \item Being able to formalize problems
  }{2009-04-08}
    
  \lecture{2}{Computational Models}{above left}
  {Computational Models.west}{
    \item Knowledge of Turing machines
    \item Being able to compare the computational power of different
      models 
  }{2009-04-15}
\end{codeexample}
Finally, the background:
\begin{codeexample}[code only]
  \begin{pgfonlayer}{background}
    \clip[xshift=-1cm] (-.5\textwidth,-.5\textheight) rectangle ++(\textwidth,\textheight);
    
    \colorlet{upperleft}{green!50!black!25}
    \colorlet{upperright}{orange!25}
    \colorlet{lowerleft}{red!25}
    \colorlet{lowerright}{blue!25}

     % The large rectangles:
    \fill [upperleft]  (Computational Complexity) rectangle ++(-20,20);
    \fill [upperright] (Computational Complexity) rectangle ++(20,20);
    \fill [lowerleft]  (Computational Complexity) rectangle ++(-20,-20);
    \fill [lowerright] (Computational Complexity) rectangle ++(20,-20);

    % The shadings:
    \shade [left color=upperleft,right color=upperright]
      ([xshift=-1cm]Computational Complexity) rectangle ++(2,20);
    \shade [left color=lowerleft,right color=lowerright]
      ([xshift=-1cm]Computational Complexity) rectangle ++(2,-20);
    \shade [top color=upperleft,bottom color=lowerleft]
      ([yshift=-1cm]Computational Complexity) rectangle ++(-20,2);
    \shade [top color=upperright,bottom color=lowerright]
      ([yshift=-1cm]Computational Complexity) rectangle ++(20,2);
  \end{pgfonlayer}
\end{tikzpicture}
\end{codeexample}

The next page shows the resulting lecture map in all its glory (it
would be somewhat more glorious, if there were more lecture
annotations, but you should get the idea).

\def\lecture#1#2#3#4#5#6{
  % As before:    
  \node [annotation, #3, scale=0.65, text width=4cm, inner sep=2mm, fill=white] at (#4) {
    Lecture #1: \textcolor{orange}{\textbf{#2}}
    \list{--}{\topsep=2pt\itemsep=0pt\parsep=0pt
              \parskip=0pt\labelwidth=8pt\leftmargin=8pt
              \itemindent=0pt\labelsep=2pt}
    #5
    \endlist
  };
  % New:
  \node [anchor=base west] at (cal-#6.base east) {\textcolor{orange}{\textbf{#2}}};
}

\noindent
\begin{tikzpicture}
  \begin{scope}[
    mindmap,
    every node/.style={concept, circular drop shadow,execute at begin node=\hskip0pt},
    root concept/.append style={
      concept color=black,
      fill=white, line width=1ex,
      text=black, font=\large\scshape},
    text=white,
    computational problems/.style={concept color=red,faded/.style={concept color=red!50}},
    computational models/.style={concept color=blue,faded/.style={concept color=blue!50}},
    measuring complexity/.style={concept color=orange,faded/.style={concept color=orange!50}},
    solving problems/.style={concept color=green!50!black,faded/.style={concept color=green!50!black!50}},
    grow cyclic,
    level 1/.append style={level distance=4.5cm,sibling angle=90,font=\scshape},
    level 2/.append style={level distance=3cm,sibling angle=45,font=\scriptsize}]
    \node [root concept] (Computational Complexity) {Computational Complexity} % root
      child [computational problems] { node [yshift=-1cm] (Computational Problems) {Computational Problems}
        child         { node (Problem Measures) {Problem Measures} }
        child         { node (Problem Aspects) {Problem Aspects} }
        child [faded] { node (problem Domains) {Problem Domains} }
        child         { node (Key Problems) {Key Problems} }
      }
      child [computational models] { node [yshift=-1cm]  (Computational Models) {Computational Models}
        child         { node (Turing Machines) {Turing Machines} }
        child [faded] { node (Random-Access Machines) {Random-Access Machines} }
        child         { node (Circuits) {Circuits} }
        child [faded] { node (Binary Decision Diagrams) {Binary Decision Diagrams} }
        child         { node (Oracle Machines) {Oracle Machines} }
        child         { node (Programming in Logic) {Programming in Logic} }
      }
      child [measuring complexity] { node [yshift=1cm] (Measuring Complexity) {Measuring Complexity} 
        child         { node (Complexity Measures) {Complexity Measures} }
        child         { node (Classifying Complexity) {Classifying Complexity} }
        child         { node (Comparing Complexity) {Comparing Complexity} }
        child [faded] { node (Describing Complexity) {Describing Complexity} }
      }
      child [solving problems] { node [yshift=1cm] (Solving Problems) {Solving Problems}
        child         { node (Exact Algorithms) {Exact Algorithms} }
        child         { node (Randomization) {Randomization} }
        child         { node (Fixed-Parameter Algorithms) {Fixed-Parameter Algorithms} }
        child         { node (Parallel Computation) {Parallel Computation} }
        child         { node (Partial Solutions) {Partial Solutions} }
        child         { node (Approximation) {Approximation} }
      };
  \end{scope}
  
  \tiny
  \calendar [day list downward,
             month text=\%mt\ \%y0,
             month yshift=3.5em,
             name=cal,
             at={(-.5\textwidth-5mm,.5\textheight-1cm)},
             dates=2009-04-01 to 2009-06-last]
    if (weekend)
      [black!25]
    if (day of month=1) {
      \node at (0pt,1.5em) [anchor=base west] {\small\tikzmonthtext};
    };
 
  \lecture{1}{Computational Problems}{above,xshift=-5mm,yshift=5mm}{Computational Problems.north}{
    \item Knowledge of several key problems
    \item Knowledge of problem encondings
    \item Being able to formalize problems
  }{2009-04-08}
    
  \lecture{2}{Computational Models}{above left}
  {Computational Models.west}{
    \item Knowledge of Turing machines
    \item Being able to compare the computational power of different
      models 
  }{2009-04-15}
  
  \begin{pgfonlayer}{background}
    \clip[xshift=-1cm] (-.5\textwidth,-.5\textheight) rectangle ++(\textwidth,\textheight);
    
    \colorlet{upperleft}{green!50!black!25}
    \colorlet{upperright}{orange!25}
    \colorlet{lowerleft}{red!25}
    \colorlet{lowerright}{blue!25}

     % The large rectangles:
    \fill [upperleft]  (Computational Complexity) rectangle ++(-20,20);
    \fill [upperright] (Computational Complexity) rectangle ++(20,20);
    \fill [lowerleft]  (Computational Complexity) rectangle ++(-20,-20);
    \fill [lowerright] (Computational Complexity) rectangle ++(20,-20);

    % The shadings:
    \shade [left color=upperleft,right color=upperright]
      ([xshift=-1cm]Computational Complexity) rectangle ++(2,20);
    \shade [left color=lowerleft,right color=lowerright]
      ([xshift=-1cm]Computational Complexity) rectangle ++(2,-20);
    \shade [top color=upperleft,bottom color=lowerleft]
      ([yshift=-1cm]Computational Complexity) rectangle ++(-20,2);
    \shade [top color=upperright,bottom color=lowerright]
      ([yshift=-1cm]Computational Complexity) rectangle ++(20,2);
  \end{pgfonlayer}
\end{tikzpicture}

% $Header: /cvsroot/pgf/pgf/doc/pgf/text-en/pgfmanual-en-guidelines.tex,v 1.1 2005/09/02 16:05:42 tantau Exp $

% Copyright 2005 by Till Tantau <tantau@cs.tu-berlin.de>.
%
% This program can be redistributed and/or modified under the terms
% of the GNU Public License, version 2.



\section{Guidelines on Graphics}

The present section is not about \pgfname\ or \tikzname, but about
general guidelines and principles concerning the creation of
graphics for scientific presentations, papers, and books.

The guidelines in this section come from different sources. Many of
them are just what I would like to claim is ``common sense,'' some
reflect my personal experience (though, hopefully, not my personal
preferences), some come from books (the bibliography is still missing,
sorry) on graphic design and typography. 
The most influential source  are the brilliant books
by Edward Tufte. While I do not agree with everything written in these
books, many of Tufte's arguments are so convincing that I decided to
repeat them in the following guidelines. 




\subsection{Should You Follow Guidelines?}

The first thing you should ask yourself when someone presents a bunch of
guidelines is: Should I really follow these guidelines? This is an
important questions, because there are good reasons not to follow
general guidelines.
\begin{itemize}
\item
  The person who setup the guidelines may have had other
  objectives than you do. For example, a guideline might say ``use the
  color red for emphasis.'' While this guideline makes perfect sense
  for, say, a presentation using a projector, red ``color'' has the
  \emph{opposite} effect of ``emphasis'' when printed using a
  black-and-white printer.

  Guidelines were almost always setup to address a specific
  situation. If you are not in this situation, following a guideline
  can do more harm than good.
\item
  The basic rule of typography is: ``Every rule can be broken, as long
  as you are \emph{aware}  that you are breaking a rule.'' This rule
  also applies to graphics. Phrased differently, the basic rule
  states: ``The only mistakes in typography are things done is
  ignorance.''

  When you are aware of a rule and when you decide that breaking the
  rule has a desirable effect, break the rule.
\item
  Some guidelines are simply \emph{wrong}, but everyone follows them
  out of tradition or is forced to do so. My favorite example is a 
  guideline a software company I used to work for has set in a big
  project: All programmers had to declare the parameters of functions
  in \emph{increasing order of size}. So, one-byte
  parameters should come first, then two-byte parameters, and so on. 

  This guideline is total nonsense. An (arguably) sensible guideline
  is ``parameters must be declared alphabetically'' so that parameters
  are easier to find. Another (arguably) sensible guideline is
  ``parameters must be declared in decreasing order of size'' so that
  less byte-alignment cache misses occur when the stack is
  accessed. The guideline the company used maximized cache misses and
  resulted in a more or less random ordering so that programmers
  constantly had to look up the parameter ordering.
\end{itemize}

So, before you apply a guideline or choose not to apply it, ask
yourself these questions: 
\begin{enumerate}
\item
  Does this guideline really address my situation?
\item
  If you do the opposite a guideline says you should do, will the
  advantages outweigh the disadvantages this guideline was supposed to
  prevent?  
\end{enumerate}



\subsection{Planning the Time Needed for the Creation of Graphics}

When you create a paper with numerous graphics, the time needed to
create these graphics becomes an important factor. How much time
should you calculate for the creation of graphics?

As a general rule, assume that a graphic will need as much time to
create as would a text of the same length. For example, when I
write a paper, I need about one hour per page for
the first draft. Later, I need between two and four hours per page
for revisions. Thus, I expect to need about half an hour for the
creation of \emph{a first draft} of a half page graphic. Later on, I
expect another one to two hours before the final graphic is finished.

In many publications, even in good journals, the authors and editors
have obviously  invested a lot of time on the text, but seem to 
have spend about five minutes to create all of the
graphics. Graphics often seem to have been added as an
``afterthought'' or look like a screen shot of whatever the authors's
statistical software shows them. As will be argued later on, the
graphics that programs like \textsc{gnuplot} produce by default are of
poor quality.

Creating informative graphics that help the reader and that fit
together with the main text is a difficult, lengthy process. 
\begin{itemize}
\item
  Treat graphics as first-class citizens of your papers. They deserve
  as much time and energy as the text does.
\item
  Arguably, the creation of graphics deserves \emph{even more} time
  than the writing of the main text since more attention will  be paid
  to the graphics and they will be looked at first. 
\item
  Plan as much time for the creation and revision of a graphic as you
  would plan for text of the same size.
\item
  Difficult graphics with a high information density may require even
  more time.
\item
  Very simple graphics will require less time, but most likely you do
  not want to have ``very simple graphics'' in your paper, anyway;
  just as you would not like to have a ``very simple text'' of the
  same size.  
\end{itemize}



\subsection{Workflow for Creating a Graphic}

When you write a (scientific) paper, you will most likely follow the
following pattern: You have some results/ideas that you would
like to report about. The creation of the paper will typically start
with compiling a rough outline. Then, the different sections are
filled with text to create a first draft. This draft is then revised
repeatedly until, often after substantial revision, a final paper
results. In a good journal paper there is typically not be a single 
sentence that has survived unmodified from the first draft.

Creating a graphics follows the same pattern:
\begin{itemize}
\item
  Decide on what the graphic should communicate. Make this a conscious
  decision, that is, determine ``What is the graphic supposed to tell
  the reader?''
\item
  Create an ``outline,'' that is, the rough overall ``shape'' of the
  graphic, containing the most crucial elements. Often, it is
  useful to do this using pencil and paper.
\item
  Fill out the finer details of the graphic to create a first
  draft.
\item
  Revise the graphic repeatedly along with the rest of the paper.
\end{itemize}




\subsection{Linking Graphics With the Main Text}

Graphics can be placed at different places in a text. Either, they can
be inlined, meaning they are somewhere ``in the middle of the text''
or they can be placed in standalone ``figures.'' Since printers (the
people) like to have their pages ``filled,'' (both for aesthetic and
economic reasons) standalone figures may traditionally be placed on
pages in the document far removed from the main text that refers to
them. \LaTeX\ and \TeX\ tend to encourage this ``drifting away'' of
graphics for technical reasons. 

When a graphic is inlined, it will more or less automatically be
linked with the main text in the sense that the labels of the graphic
will be implicitly explained by the surrounding text. Also, the main
text will typically make it clear what the graphic is about and what
is shown.

Quite differently, a standalone figure will often be viewed at a time
when the main text that this graphic belongs to either has not yet
been read or has been read some time ago. For this reason, you should
follow the following guidelines when creating standalone figures:
\begin{itemize}
\item
  Standalone figures should have a caption than should make them
  ``understandable by themselves.''

  For example, suppose a graphic shows an example of the different
  stages of a quicksort algorithm. Then the figure's caption should,
  at the very least, inform the reader that ``The figure shows the
  different stages of the quicksort algorithm introduced on page
  xyz.'' and not just ``Quicksort algorithm.''
\item
  A good caption adds as much context information as possible. For
  example, you could say: ``The figure shows the different stages of
  the quicksort algorithm introduced on page xyz. In the first line,
  the pivot element 5 is chosen. This causes\dots'' While this
  information can also be given in the main text, putting it in the
  caption will ensure that the context is kept. Do not feel afraid of
  a 5-line caption. (Your editor may hate you for this. Consider
  hating them back.)
\item
  Reference the graphic in your main text as in ``For an example of
  quicksort `in action,' see Figure~2.1 on page xyz.''
\item
  Most books on style and typography recommend that you do not use
  abbreviations as in ``Fig.~2.1'' but write ``Figure 2.1.''

  The main argument against abbreviations is that ``a period is too
  valuable to waste it on an abbreviation.'' The idea is that a period
  will make the reader assume that the sentence ends after ``Fig'' and
  it takes a ``conscious backtracking'' to realize that the sentence
  did not end after all.

  The argument in favor of abbreviations is that they save space.
  
  Personally, I am not really convinced by either argument. On the one
  hand, I have not yet seen any hard evidence that abbreviations slow 
  readers down. On the other hand,  abbreviating all ``Figure'' by
  ``Fig.''\ is most unlikely to save even a single line in  most
  documents.  

  I avoid abbreviations.
\end{itemize}



\subsection{Consistency Between Graphics and Text}

Perhaps the most common ``mistake'' people do when creating graphics
(remember that a ``mistake'' in design is always just ``ignorance'')
is to have a mismatch between the way their graphics look and the way 
their text looks.

It is quite common that authors use several different programs for
creating the graphics of a paper. An author might produce some plots
using \textsc{gnuplot}, a diagram using \textsc{xfig}, and include an
|.eps| graphic a coauthor contributed using some unknown program. All
these graphics will, most likely, use different line widths, different
fonts, and have different sizes. In addition, authors often use
options like |[height=5cm]| when including graphics to scale them to
some ``nice size.''

If the same approach were taken to writing the main text, every
section would be written in a different font at a different size. In
some sections all theorems would be underlined, in another they would
be printed all in uppercase letters, and in another in red. In
addition, the margins would be different on each page.

Readers and editors would not tolerate a text if it were written in
this fashion, but with graphics they often have to.

To create consistency between graphics and text, stick to the
following guidelines:
\begin{itemize}
\item
  Do not scale graphics.

  This means that when generating graphics using an external program,
  create them ``at the right size.''
\item
  Use the same font(s) both in graphics and the body text.
\item
  Use the same line width in text and graphics.

  The  ``line width'' for normal text is the width of the stem of
  letters like T{}. For \TeX, this is usually
  $0.4\,\mathrm{pt}$. However, some journals will not accept graphics
  with a normal line width below $0.5\,\mathrm{pt}$.
\item
  When using colors, use a consistent color coding in the text and in  
  graphics. For example, if red is supposed to alert the reader to
  something in the main text, use red also in graphics for important
  parts of the graphic. If blue is used for structural elements like 
  headlines and section titles, use blue also for structural elements
  of your graphic.

  However, graphics may also use a logical intrinsic color
  coding. For example, no matter what colors you normally use, readers
  will generally assume, say, that the color green as ``positive, go,
  ok'' and red as ``alert, warning, action.''
\end{itemize}

Creating consistency when using different graphic programs is almost
impossible. For this reason, you should consider sticking to a single
graphic program.


\subsection{Labels in Graphics}

Almost all graphics will contain labels, that is, pieces of text that
explain parts of the graphics. When placing labels, stick to the
following guidelines:

\begin{itemize}
\item
  Follow the rule of consistency when placing labels. You should do
  so in two ways: First, be consistent with the main text, that is,
  use the same font as the main text also for labels. Second, be
  consistent between labels, that is, if you format some labels in
  some particular way, format all labels in this way.
\item
  In addition to using the same fonts in text and graphics, you should
  also use the same notation. For example, if you write $1/2$ in your
  main text, also use ``$1/2$'' as labels in graphics, not
  ``0.5''. A $\pi$ is a ``$\pi$'' and not ``$3.141$''. Finally,
  $\mathrm e^{-\mathrm i \pi}$ is ``$\mathrm e^{-\mathrm i \pi}$'',
  not ``$-1$'', let alone ``-1''. 
\item
  Labels should be legible. They should not only have a reasonably
  large size, they also should not be obscured by lines or other
  text. This also applies to of lines and text \emph{behind} the
  labels.
\item
  Labels should be ``in  place.'' Whenever there is enough space,
  labels should be placed next to the thing they label. Only if
  necessary, add a (subdued) line from the label to the labeled
  object. Try to avoid labels that only reference explanations in
  external legends. Reader have to jump back and forth between the
  explanation and the object that is described. 
\item
  Consider subduing ``unimportant'' labels using, for example, a gray
  color. This will keep the focus on the actual graphic.
\end{itemize}



\subsection{Plots and Charts}

One of the most frequent kind of graphics, especially in scientific
papers, are \emph{plots}. They come in a large variety, including
simple line plots, parametric plots, three dimensional plots, pie
charts, and many more.

Unfortunately, plots are notoriously hard to get right. Partly, the
default settings of programs like \textsc{gnuplot} or Excel are to
blame for this since these programs make it very convenient to create
bad plots.

The first question you should ask yourself when creating a plot is the
following:
\begin{itemize}
\item
  Are there enough data points to merit a plot?
\end{itemize}

If the answer is ``not really,'' use a table.

A typical situation where a plot is unnecessary is when people present
a few numbers in a bar diagram. Here is a real-life example: At the
end of a seminar a lecturer asked the participants for feedback. Of
the 50 participants, 30 returned the feedback form. According to the
feedback, three participants considered the seminar ``very good,''
nine considered it  ``good,'' ten ``ok,'' eight ``bad,'' and no one thought 
that the seminar was ``very bad.''

A simple way of summing up this information is the following table:

\medskip
\begin{tabular}{lp{3.75cm}r}
  \emph{Rating given} & \raggedright\emph{Participants (out of 50) who gave this rating} &
  \emph{Percentage} \\[1.75em]
  ``very good'' & \hfil\hphantom{0}3\hfil & \hphantom{0}6\% \\
  ``good'' & \hfil\hphantom{0}9\hfil & 18\% \\
  ``ok'' & \hfil10\hfil & 20\% \\
  ``bad'' & \hfil\hphantom{0}8\hfil & 16\% \\
  ``very bad'' & \hfil\hphantom{0}0\hfil & \hphantom{0}0\% \\[2mm]
  none & \hfil20\hfil & 40\% \\
\end{tabular}

\bigskip
What the lecturer did was to visualize the data using a 3D bar
diagram. It looked like this:

\bigskip
\par
\begin{tikzpicture}[y=0.03cm,z=3mm]
  \foreach \y in {0,20,40,60,80,100}
    \draw[dashed] (0,\y,0) node[left] {\y} -- (0,\y,1)  -- (6,\y,1);

  \draw (0,0,0) -- (0,100,0)  (0,0,1) -- (0,100,1);
  \draw (0,0,0) -- (6,0,0);

  \foreach \x/\xtext/\height in {1/very good/10,2/good/30,3/ok/33,4/bad/27,5/very bad/0}
  {
    \draw (\x,0) node[rotate=90,anchor=east] {\xtext};

    \begin{scope}[xshift=\x cm]
      
    \filldraw[fill=blue!50] (-.3,0,0) rectangle (.3,\height,0);
    \filldraw[fill=blue!30] (.3,0,0) -- (.3,0,1) -- (.3,\height,1) -- (.3,\height,0) --cycle;
    \filldraw[fill=blue!20] (-.3,\height,0) -- (.3,\height,0) --
    (.3,\height,1) -- (-.3,\height,1) --cycle;
    \end{scope}
  }
\end{tikzpicture}
\bigskip

Both the table and the ``plot'' have about the same size. If your first
thought is ``the graphic looks nicer than the table,'' try to answer
the following questions based on the information in the table or in
the graphic: 
\begin{enumerate}
\item
  How many participants where there?
\item
  How many participants returned the feedback form?
\item
  What percentage of the participants returned the feedback form?
\item
  How many participants checked ``very good''?
\item
  What percentage out of all participants checked ``very good''?
\item
  Did more than a quarter of the participants check ``bad'' or ``very bad''?
\item
  What percentage of the participants that returned the form checked ``very good''?
\end{enumerate}

Sadly, the graphic does not allow us to answer \emph{a single one of these
  questions}. The table answers all of them directly, except for the last
one. In essence, the information density of the graphic is very
nearly zero. The table has a much higher information density; despite
the fact that it uses quite a lot of white space to present a few numbers.

Here is the list of things that went wrong with the 3D-bar diagram:
\begin{itemize}
\item
  The whole graphic is dominated by irritating background lines.
\item
  It is not clear what the numbers at the left mean; presumably
  percentages, but it might also be the absolute number of
  participants.
\item
  The labels at the bottom are rotated, making them hard to read.

  (In the real presentation that I saw, the text was rendered at a very 
  low resolution with about 10 by 6 pixels per letter with wrong
  kerning, making the rotated text almost impossible to read.)
\item
  The third dimension adds complexity to the graphic without adding
  information.
\item
  The three dimensional setup makes it much harder to gauge the height
  of the bars correctly. Consider the ``bad'' bar. It the number this
  bar stands for more than 20 or less? While the front of the bar is
  below the 20 line, the back of the bar (which counts) is above.
\item
  It is impossible to tell which  numbers are represented by the
  bars. Thus, the bars needlessly hide the information these bars are
  all about.
\item
  What do the bar heights add up to? Is it 100\% or 60\%?
\item
  Does the bar for ``very bad'' represent 0 or~1?
\item
  Why are the bars blue?
\end{itemize}

You might argue that in the example the exact numbers are not
important for the graphic. The important things is the ``message,''
which is that there are more ``very good'' and ``good'' ratings than
``bad'' and ``very bad.'' However, to convey this message either use a
sentence that says so or use a graphic that conveys this message more
clearly:  

\medskip
\par
\begin{tikzpicture}
  \colorlet{good}{green!75!black}
  \colorlet{bad}{red}
  \colorlet{neutral}{black!60}
  \colorlet{none}{white}

  \node[text centered,text width=3cm]{Ratings given by 50~participants};

  \begin{scope}[line width=4mm,rotate=270]
    \draw[good]          (-123:2cm) arc (-123:-101:2cm);
    \draw[good!60!white] (-36:2cm) arc (-36:-101:2cm);
    \draw[neutral]       (-36:2cm) arc (-36:36:2cm);
    \draw[bad!60!white]  (36:2cm)  arc (36:93:2cm);

    \newcount\mycount
    \foreach \angle in {0,72,...,3599}
    {
      \mycount=\angle\relax
      \divide\mycount by 10\relax
      \draw[black!15,thick] (\the\mycount:18mm) -- (\the\mycount:22mm);
    }
    
    \draw (0:2.2cm) node[below] {``ok'': 10 (20\%)};
    \draw (165:2.2cm) node[above] {none: 20 (40\%)};
    \draw (-111:2.2cm) node[left] {``very good'': 3 (6\%)};
    \draw (-68:2.2cm) node[left] {``good'': 9 (18\%)};
    \draw (65:2.2cm) node[right] {``bad'': 8 (16\%)};
    \draw (93:2.2cm) node[right] {``very bad'': 0 (0\%)};
  \end{scope}  
  \draw[gray] (0,0) circle (2.2cm) circle (1.8cm);
\end{tikzpicture}

\bigskip
The above graphic has about the same information density as the table
(about the same size and the same numbers are shown). In addition, one
can directly ``see'' that there are more good or very good ratings
than bad ones. One can also ``see'' that the number of people who gave
no rating at all is not negligible, which is quite common for feedback
forms. 

Charts are not always a good idea. Let us look at an example
that I redrew from a pie chart in \emph{Die Zeit}, June 4th, 2005:

\bigskip
\par
\begin{tikzpicture}
  \begin{scope}[xscale=3.2,yscale=1.2]

    \sffamily
    \coordinate (right border) at (2.0cm,-1.7cm);
    \coordinate (left border)  at (-2.5cm,2.1cm);

    \fill[black!25] ([xshift=-2mm,yshift=1.1cm]left border) rectangle ([xshift=2mm,yshift=-.3cm]right border);

    \node[below right,text width=10cm,inner sep=0pt] at ([yshift=.9cm,xshift=-1mm]left border)
    { {\color{black!75} \Large Kohle ist am wichtigsten}\\
      Energiemix bei der deutschen Stromerzeugung 2004};

    \filldraw[draw=gray,fill=white] ([xshift=-1mm]left border) node[below right,black]
      {\footnotesize Gesamte Netto-Stromerzeugung in Prozent, in
        Milliarden Kilowattstunden (Mrd.\ kWh)}
      rectangle ([xshift=1mm]right border);
    
    % The 3D stuff
    \pgfdeclarehorizontalshading{zeit}{100bp}
    {color(0pt)=(black);
      color(25bp)=(black);
      color(37bp)=(white);
      color(50bp)=(black);
      color(62bp)=(white);
      color(75bp)=(black);
      color(100bp)=(black)}

    \shadedraw[very thin,shading=zeit,yshift=-1.5mm] (0,0) circle (1cm);

    \fill[green!20!gray]   (0,0) -- (90:1cm) arc (90:-5:1cm);
    \fill[white!20!gray]   (0,0) -- (-5:1cm) arc (-5:-105:1cm);
    \fill[orange!20!gray]  (0,0) -- (-105:1cm) arc (-105:-180:1cm);
    \fill[orange!60!white] (0,0) -- (180:1cm) arc (180:150:1cm);
    \fill[black!75!white]  (0,0) -- (150:1cm) arc (150:145:1cm);
    \fill[blue!90!white]   (0,0) -- (145:1cm) arc (145:135:1cm);
    \fill[blue!50!white]   (0,0) -- (135:1cm) arc (135:92:1cm);
    \fill[yellow!50!black] (0,0) -- (92:1cm) arc (92:90:1cm);

    \begin{scope}[very thin]
      \draw (0,0) -- (90:1cm);
      \draw (0,0) -- (-5:1cm);
      \draw (0,0) -- (-105:1cm);
      \draw (0,0) -- (-180:1cm);
      \draw (0,0) -- (150:1cm);
      \draw (0,0) -- (145:1cm);
      \draw (0,0) -- (135:1cm);
      \draw (0,0) -- (92:1cm);
      
      \draw(0,0) circle (1cm);
    \end{scope}

    \node (Regenerative)   at (115:.75cm)  {\bfseries 9,4\%};
    \node (Kernenergie)    at (30:.5cm)   {\bfseries 27,8\%};
    \node (Braunkohle)     at (-45:.6cm)  {\bfseries 25,6\%};
    \node (Steinkohle)     at (-135:.6cm) {\bfseries 22,3\%};
    \node (Erdgas)         at (168:.75cm) {\bfseries 10,4\%};
    \coordinate (Mineral)  at (147:.9cm);
    \coordinate (Sonstige) at (140:.9cm);

    \small
    \draw (Regenerative.north) |- ([yshift=.25cm]Regenerative.north -| right border) coordinate (Regenerative label);
    \draw (91:.9cm) |- (Regenerative label);
    \node[above left] at (Regenerative label) {Regenerative\
      {\footnotesize (53,7 kWh)/davon} Wind \textbf{4,4\%}  \footnotesize (25,0 kWh)};

    \draw (Kernenergie.base east) -- (Kernenergie.base east -| right border) coordinate (Kernenergie label);
    \node[above left] at (Kernenergie label) {Kernenergie};
    \node[below left] at (Kernenergie label) {\footnotesize (158,4 kWh)};

    \draw (Braunkohle.south) |- ([yshift=-.75cm]Braunkohle.south -| right border) coordinate (Braunkohle label);
    \node[above left] at (Braunkohle label) {Braunkohle\ \ \footnotesize (146,0 kWh)};

    \draw (Steinkohle.south) |- ([yshift=-.75cm]Steinkohle.south -| left border) coordinate (Steinkohle label);
    \node[above right] at (Steinkohle label) {Steinkohle\ \ \footnotesize (127,1 kWh)};

    \draw (Erdgas.base west) -- (Erdgas.base west -| left border) coordinate (Erdgas label);
    \node[above right] at (Erdgas label) {Erdgas\ \ \footnotesize (59,2 kWh)};

    \draw (Mineral) -- (Mineral -| left border) coordinate (Mineral label);
    \node[above right] at (Mineral label) {Mineral\"olprodukte\ \
      \footnotesize (9,2 kWh) \  \ \normalsize\textbf{1,6\%}};

    \draw (Sonstige) |- (Regenerative label -| left border) coordinate (Sonstige label);
    \node[above right] at (Sonstige label) {Sonstige\ \
      \footnotesize (16,5 kWh) \hskip1.5cm\
      \normalsize\textbf{2,9\%}};
  \end{scope}    
\end{tikzpicture}

This graphic has been redrawn in \tikzname, but the original looks very
similar.

At first sight, the graphic looks  ``nice and informative,'' but there
are a lot of things that went wrong:

\begin{itemize}
\item
  The chart is three dimensional. However, the shadings add
  nothing ``information-wise,'' at best, they distract.
\item
  In a 3D-pie-chart the relative sizes are very strongly
  distorted. For example, the area taken up by the gray color of ``Braunkohle''
  is larger than the area taken up by the green color of
  ``Kernenergie'' \emph{despite the fact that the percentage of
    Braunkohle is less than the percentage of Kernenergie}.
\item
  The 3D-distortion gets worse for small areas. The area of
  ``Regenerative'' somewhat larger  than the area of ``Erdgas.''  
  The area of ``Wind'' is slightly smaller than the area of
  ``Mineral\"olprodukte'' \emph{although the percentage of Wind is
    nearly three times larger than the percentage of
    Mineral\"olprodukte.}

  In the last case, the different sizes are only partly due to
  distortion. The designer(s) of the original graphic have also made
  the ``Wind'' slice too small, even taking distortion into
  account. (Just compare the size of ``Wind'' to ``Regenerative'' in
  general.) 
\item
  According to its caption, this chart is supposed to inform us that
  coal was the most important energy source in Germany in
  2004. Ignoring the strong distortions caused by the superfluous and
  misleading 3D-setup, it takes quite a while for this message to get
  across. 

  Coal as an energy source is split up into two slices: one for
  ``Steinkohle'' and one for ``Braunkohle'' (two different kinds of
  coal). When you add them up, you see that the whole lower half of
  the pie chart is taken up by coal.

  The two areas for the different kinds of coal are not visually
  linked at all. Rather, two different colors are used, the labels are
  on different sides of the graphic. By comparison, ``Regenerative''
  and ``Wind'' are very closely linked.
\item
  The color coding of the graphic follows no logical pattern at
  all. Why is nuclear energy green? Regenerative energy is light blue,
  ``other sources'' are blue. It seems more like a joke that the area
  for ``Braunkohle'' (which literally translates to ``brown coal'') is
  stone gray, while the area for ``Steinkohle'' (which literally
  translates to ``stone coal'') is brown.
\item
  The area with the lightest color is used for ``Erdgas.'' This area
  stands out most because of the brighter color. However, for this
  chart ``Erdgas'' is not really important at all.
\end{itemize}
Edward Tufte calls graphics like the above ``chart junk.'' 

Here are a few recommendations that may help you avoid producing chart junk:
\begin{itemize}
\item
  Do not use 3D pie charts. They are \emph{evil}.
\item
  Consider using a table instead of a pie chart.
\item
  Due not apply colors randomly; use them to direct the readers's 
  focus and to group things.
\item
  Do not use background patterns, like a crosshatch or diagonal
  lines, instead of colors. They distract. Background patterns in
  information graphics are \emph{evil}.
\end{itemize}



\subsection{Attention and Distraction}

Pick up your favorite fiction novel and have a look at a typical
page. You will notice that the page is very uniform. Nothing is there
to distract the reader while reading; no large headlines, no bold
text, no large white areas. Indeed, even when the author does wish to
emphasize something, this is done using italic letters. Such letters
blend nicely with the main text---at a distance you will not be able to
tell whether a page contains italic letters, but you would notice a
single bold word immediately. The reason novels are typeset this way
is the following paradigm: Avoid distractions.

Good typography (like good organization) is something you do
\emph{not} notice. The job of typography is to make reading the text,
that is, ``absorbing'' its information content, as effortless as
possible. For a novel, readers absorb the content by reading the text
line-by-line, as if they were listening to someone telling the
story. In this situation anything on the page that distracts the eye
from  going quickly and evenly from line to line will make the text
harder to read.

Now, pick up your favorite weekly magazine or newspaper and have a
look at a typical 
page. You will notice that there is quite a lot ``going on'' on the
page. Fonts are used at different sizes and in different arrangements,
the text is organized in narrow columns, typically interleaved with
pictures. The reason magazines are typeset in this way is another
paradigm: Steer attention.

Readers will not read a magazine like a novel. Instead of reading a
magazine line-by-line, we use headlines and short abstracts to check
whether we want to read a certain article or not. The job of
typography is to steer our attention to these abstracts and headlines,
first. Once we have decided that we want to read an article, however,
we no longer tolerate distractions, which is why the main text of
articles is typeset exactly the same way as a novel.

The two principles ``avoid distractions'' and ``steer attention'' also
apply to graphics. When you design a graphic, you should eliminate
everything that will ``distract the eye.'' At the same time, you
should try to actively help the reader ``through the graphic'' by
using fonts/colors/line widths to highlight different parts.

Here is a non-exhaustive list of things that can distract readers:
\begin{itemize}
\item
  Strong contrasts will always be registered first by the eye. For
  example, consider the following two grids:

  \medskip\par
  \begin{tikzpicture}[x=40pt,y=40pt]
    \draw[step=10pt,gray] (0,0) grid +(1,1);
    \draw[step=2pt]      (2,0) grid +(1,1);
  \end{tikzpicture}

  \medskip
  Even though the left grid comes first in our normal reading order,
  the right one is much more likely to be seen first: The
  white-to-black contrast is higher than the gray-to-white
  contrast. In addition, there are more ``places'' adding to the
  overall contrast in the right grid.

  Things like grids and, more generally, help lines usually should not
  grab the attention of the readers and, hence, should be typeset with
  a low contrast to the background. Also, a loosely-spaced grid is
  less distracting than a very closely-spaced grid.
\item
  Dashed lines create many points at which there is black-to-white
  contrast. Dashed or dotted lines can be very distracting and, hence,
  should be avoided in general.

  Do not use different dashing patterns to differentiate curves in
  plots. You loose data points this way and the eye is not
  particularly good at ``grouping things according to a dashing
  pattern.'' The eye is \emph{much} better at grouping things
  according to colors.
\item
  Background patterns filling an area using  diagonal lines or
  horizontal and vertical lines or just dots are almost always
  distracting and, usually, serve no real purpose.
\item
  Background images and shadings distract and only seldom add
  anything of importance to a graphic.
\item
  Cute little cliparts can easily draw attention away from the
  data.
\end{itemize}





\part{Installation and Configuration}

{\Large \emph{by Till Tantau}}


\bigskip
\noindent
This part explains how the system is installed. Typically, someone has
already done so for your system, so this part can be skipped; but if
this is not the case and you are the poor fellow who has to do the
installation, read the present part.


\vskip1cm

\begin{codeexample}[graphic=white]
\begin{tikzpicture}[->,>=stealth',shorten >=1pt,auto,node distance=2.8cm,on grid,semithick,
                    every state/.style={fill=red,draw=none,circular drop shadow,text=white}]

  \node[initial,state] (A)                    {$q_a$};
  \node[state]         (B) [above right=of A] {$q_b$};
  \node[state]         (D) [below right=of A] {$q_d$};
  \node[state]         (C) [below right=of B] {$q_c$};
  \node[state]         (E) [below=of D]       {$q_e$};

  \path (A) edge              node {0,1,L} (B)
            edge              node {1,1,R} (C)
        (B) edge [loop above] node {1,1,L} (B)
            edge              node {0,1,L} (C)
        (C) edge              node {0,1,L} (D)
            edge [bend left]  node {1,0,R} (E)
        (D) edge [loop below] node {1,1,R} (D)
            edge              node {0,1,R} (A)
        (E) edge [bend left]  node {1,0,R} (A);

   \node [right=1cm,text width=8cm] at (C)
   {
     The current candidate for the busy beaver for five states. It is
     presumed that this Turing machine writes a maximum number of
     $1$'s before halting among all Turing machines with five states
     and the tape alphabet $\{0, 1\}$. Proving this conjecture is an
     open research problem.
   };
\end{tikzpicture}
\end{codeexample}

% Copyright 2003 by Till Tantau <tantau@cs.tu-berlin.de>.
%
% This program can be redistributed and/or modified under the terms
% of the LaTeX Project Public License Distributed from CTAN
% archives in directory macros/latex/base/lppl.txt.


\section{Installation}

There are different ways of installing \pgfname, depending
on your system and needs, and you may need to install other
packages as well as, see below. Before installing, you may wish to
review the \textsc{gpl} license under which the package is
distributed, see Section~\ref{section-license}. 

Typically, the package will already be installed on your
system. Naturally, in this case you do not need to worry about the
installation process at all and you can skip the rest of this
section. 


\subsection{Package and Driver Versions}

This documentation is part of version \pgfversion\ of the \pgfname\
package. In order to run \pgfname, you need a reasonably recent 
\TeX\ installation. When using \LaTeX, you need the following packages
installed (newer versions should also work):
\begin{itemize}
\item
  |xcolor| version \xcolorversion.
\item
  |xkeyval| version \xkeyvalversion, if you wish to use \tikzname.
\end{itemize}
With plain \TeX, |xcolor| is not needed, but you obviously do not
get its (full) functionality. 

Currently, \pgfname\ supports the following backend drivers:
\begin{itemize}
\item
  |pdftex| version 0.14 or higher. Earlier versions do not work.
\item
  |dvips| version 5.94a or higher. Earlier versions may also work.
\item
  |dvipdfm| version 0.13.2c or higher. Earlier versions may also work.
\item
  |tex4ht| version 2003-05-05 or higher. Earlier versions may also work.
\item
  |vtex| version 8.46a or higher. Earlier versions may also work.
\end{itemize}

Currently, \pgfname\ supports the following formats:
\begin{itemize}
\item
  |latex| with complete functionality.
\item
  |plain| with complete functionality, except for graphics inclusion,
  which works only for pdf\TeX.
\item
  |context| should work as |plain|, but I have not tried it.
\end{itemize}

For more details, see Section~\ref{section-formats}.



\subsection{Installing Prebundled Packages}

I do not create or manage prebundled packages of \pgfname, but,
fortunately, nice other people do. I cannot give detailed instructions
on how to install these packages, since I do not manage them, but I
\emph{can} tell you were to find them. If you have a problem with
installing, you might wish to have a look at the Debian page or the
Mik\TeX\ page first.


\subsubsection{Debian}

The command ``|aptitude install pgf|'' should do the trick. Sit back
and relax. In detail, the following packages are installed:  
\begin{verbatim}
http://packages.debian.org/pgf
http://packages.debian.org/latex-xcolor
\end{verbatim}


\subsubsection{MiKTeX}

For MiK\TeX, use the update wizard to install the (latest versions of
the) packages called |pgf|, |xcolor|, and |xkeyval|. 




\subsection{Installation in a texmf Tree}

For a permanent installation, you place the files of the
the \textsc{pgf} package in an appropriate |texmf| tree. 

When you ask \TeX\ to use a certain class or package, it usually looks
for the necessary files in so-called |texmf| trees. These trees
are simply huge directories that contain these files. By default,
\TeX\ looks for files in three different |texmf| trees:
\begin{itemize}
\item
  The root |texmf| tree, which is usually located at
  |/usr/share/texmf/| or |c:\texmf\| or somewhere similar.
\item
  The local  |texmf| tree, which is usually located at
  |/usr/local/share/texmf/| or |c:\localtexmf\| or somewhere similar.
\item
  Your personal  |texmf| tree, which is usually located in your home
  directory at |~/texmf/| or |~/Library/texmf/|.   
\end{itemize}

You should install the packages either in the local tree or in
your personal tree, depending on whether you have write access to the
local tree. Installation in the root tree can cause problems, since an
update of the whole \TeX\ installation will replace this whole tree.


\subsubsection{Installation that Keeps Everything Together}

Once you have located the right texmf tree, you must decide whether
you want to install \pgfname\ in such a way that ``all its files are
kept in one place'' or whether you want to be
``\textsc{tds}-compliant,'' where \textsc{tds} means ``\TeX\ directory
structure.''

If you want to keep ``everything in one place,'' inside the |texmf|
tree that you have chosen create a sub-sub-directory called
|texmf/tex/generic/pgf| or
|texmf/tex/generic/pgf-|\texttt{\pgfversion}, if you prefer. Then
place all files of the |pgf| package in this directory. Finally,
rebuild \TeX's filename database. This is done by running the command
|texhash| or |mktexlsr| (they are the same). In Mik\TeX, there is a
menu option to do this. 


\subsubsection{Installation that is TDS-Compliant}

While the above installation process is the most ``natural'' one and
although I would like to recommend it since it makes updating and
managing the \pgfname\ package easy, it is not
\textsc{tds}-compliant. If you want to be \textsc{tds}-compliant,
proceed as follows: (If you do not know what \textsc{tds}-compliant
means, you probably do not want to be \textsc{tds}-compliant.)

The |.tar| file of the |pgf| package contains the following files and
directories at its root: |README|, |doc|,  |generic|, |plain|, and
|latex|. You should ``merge'' each of the four directories with the
following directories |texmf/doc|, |texmf/tex/generic|,
|texmf/tex/plain|, and |texmf/tex/latex|. For example, in the |.tar|
file the |doc| directory contains just the directory |pgf|, and this
directory has to be moved to |texmf/doc/pgf|. The root |README| file
can be ignored since it is reproduced in |doc/pgf/README|.

You may also consider keeping everything in one place and using
symbolic links to point from the \textsc{tds}-compliant directories to
the central installation.

\vskip1em
For a more detailed explanation of the standard installation process
of packages, you might wish to consult
\href{http://www.ctan.org/installationadvice/}{|http://www.ctan.org/installationadvice/|}.
However, note that the \pgfname\ package does not come with a
|.ins| file (simply skip that part).


\subsection{Updating the Installation}

To update your installation from a previous version, all you need to
do is to replace everything in the directory |texmf/tex/generic/pgf|
with the files of the new version (or in all the directories where
|pgf| was installed, if you chose a \textsc{tds}-compliant
installation). The easiest way to do this is to first delete the old
version and then proceed as described above. Sometimes, there are
changes in the syntax of certain command from version to version. If
things no longer work that used to work, you may wish to have a look
at the release notes and at the change log. 


% Copyright 2006 by Till Tantau
%
% This file may be distributed and/or modified
%
% 1. under the LaTeX Project Public License and/or
% 2. under the GNU Free Documentation License.
%
% See the file doc/generic/pgf/licenses/LICENSE for more details.


\section{Licenses and Copyright}
\label{section-license}


\subsection{Which License Applies?}

Different parts of the \pgfname\ package are distributed under
different licenses:

\begin{enumerate}
\item The \emph{code} of the package is dual-license. This means that
  you can decide which license you wish to use when using the
  \pgfname\ package. The two options are:
  \begin{enumerate}
  \item You can use the \textsc{gnu} Public License, version 2.
  \item You can use the \LaTeX\ Project Public License, version 1.3c.
  \end{enumerate}
\item The \emph{documentation} of the package is also dual-license. Again,
  you can choose between two options:
  \begin{enumerate}
  \item You can use the \textsc{gnu} Free Documentation License, version 1.2.
  \item You can use the \LaTeX\ Project Public License, version 1.3c.
  \end{enumerate}
\end{enumerate}

The ``documentation of the package'' refers to all files in the
subdirectory |doc| of the |pgf| package. A detailed listing can be
found in the file |doc/generic/pgf/licenses/manifest-documentation.txt|.
All files in other directories are part of the ``code of the
package.'' A detailed listing can be found in the file
|doc/generic/pgf/licenses/manifest-code.txt|.

In the resest of this section, the licenses are presented. The
following text is copyrighted, see the plain text versions of these
licenses in the directory |doc/generic/pgf/licenses| for details. 

The example picture used in this manual, the Brave \textsc{gnu} World
logo, is taken from the Brave \textsc{gnu} World homepage, where it is
copyrighted as follows: ``Copyright (C) 1999, 2000, 2001, 2002, 2003,
2004 Georg C.~F.\ Greve. Permission is granted to make and distribute
verbatim copies of this transcript as long as the copyright and this
permission notice appear.'' 


\subsection{The GNU Public License, Version 2}

\subsubsection{Preamble}

The licenses for most software are designed to take away your freedom to
share and change it.  By contrast, the \textsc{gnu} General Public License is
intended to guarantee your freedom to share and change free software---to
make sure the software is free for all its users.  This General Public
License applies to most of the Free Software Foundation's software and to
any other program whose authors commit to using it.  (Some other Free
Software Foundation software is covered by the \textsc{gnu} Library
General Public License instead.)  You can apply it to your programs, too.

When we speak of free software, we are referring to freedom, not price.
Our General Public Licenses are designed to make sure that you have the
freedom to distribute copies of free software (and charge for this service
if you wish), that you receive source code or can get it if you want it,
that you can change the software or use pieces of it in new free programs;
and that you know you can do these things.

To protect your rights, we need to make restrictions that forbid anyone to
deny you these rights or to ask you to surrender the rights.  These
restrictions translate to certain responsibilities for you if you
distribute copies of the software, or if you modify it.

For example, if you distribute copies of such a program, whether gratis or
for a fee, you must give the recipients all the rights that you have.  You
must make sure that they, too, receive or can get the source code.  And
you must show them these terms so they know their rights.

We protect your rights with two steps: (1) copyright the software, and (2)
offer you this license which gives you legal permission to copy,
distribute and/or modify the software.

Also, for each author's protection and ours, we want to make certain that
everyone understands that there is no warranty for this free software.  If
the software is modified by someone else and passed on, we want its
recipients to know that what they have is not the original, so that any
problems introduced by others will not reflect on the original authors'
reputations.

Finally, any free program is threatened constantly by software patents.
We wish to avoid the danger that redistributors of a free program will
individually obtain patent licenses, in effect making the program
proprietary.  To prevent this, we have made it clear that any patent must
be licensed for everyone's free use or not licensed at all.

The precise terms and conditions for copying, distribution and
modification follow.

\subsubsection{Terms and Conditions For Copying, Distribution and
  Modification}

\begin{enumerate}

\addtocounter{enumi}{-1}

\item 
This License applies to any program or other work which contains a notice
placed by the copyright holder saying it may be distributed under the
terms of this General Public License.  The ``Program'', below, refers to
any such program or work, and a ``work based on the Program'' means either
the Program or any derivative work under copyright law: that is to say, a
work containing the Program or a portion of it, either verbatim or with
modifications and/or translated into another language.  (Hereinafter,
translation is included without limitation in the term ``modification''.)
Each licensee is addressed as ``you''.

Activities other than copying, distribution and modification are not
covered by this License; they are outside its scope.  The act of
running the Program is not restricted, and the output from the Program
is covered only if its contents constitute a work based on the
Program (independent of having been made by running the Program).
Whether that is true depends on what the Program does.

\item You may copy and distribute verbatim copies of the Program's source
  code as you receive it, in any medium, provided that you conspicuously
  and appropriately publish on each copy an appropriate copyright notice
  and disclaimer of warranty; keep intact all the notices that refer to
  this License and to the absence of any warranty; and give any other
  recipients of the Program a copy of this License along with the Program.

You may charge a fee for the physical act of transferring a copy, and you
may at your option offer warranty protection in exchange for a fee.

\item
You may modify your copy or copies of the Program or any portion
of it, thus forming a work based on the Program, and copy and
distribute such modifications or work under the terms of Section 1
above, provided that you also meet all of these conditions:

\begin{enumerate}

\item 
You must cause the modified files to carry prominent notices stating that
you changed the files and the date of any change.

\item
You must cause any work that you distribute or publish, that in
whole or in part contains or is derived from the Program or any
part thereof, to be licensed as a whole at no charge to all third
parties under the terms of this License.

\item
If the modified program normally reads commands interactively
when run, you must cause it, when started running for such
interactive use in the most ordinary way, to print or display an
announcement including an appropriate copyright notice and a
notice that there is no warranty (or else, saying that you provide
a warranty) and that users may redistribute the program under
these conditions, and telling the user how to view a copy of this
License.  (Exception: if the Program itself is interactive but
does not normally print such an announcement, your work based on
the Program is not required to print an announcement.)

\end{enumerate}


These requirements apply to the modified work as a whole.  If
identifiable sections of that work are not derived from the Program,
and can be reasonably considered independent and separate works in
themselves, then this License, and its terms, do not apply to those
sections when you distribute them as separate works.  But when you
distribute the same sections as part of a whole which is a work based
on the Program, the distribution of the whole must be on the terms of
this License, whose permissions for other licensees extend to the
entire whole, and thus to each and every part regardless of who wrote it.

Thus, it is not the intent of this section to claim rights or contest
your rights to work written entirely by you; rather, the intent is to
exercise the right to control the distribution of derivative or
collective works based on the Program.

In addition, mere aggregation of another work not based on the Program
with the Program (or with a work based on the Program) on a volume of
a storage or distribution medium does not bring the other work under
the scope of this License.

\item
You may copy and distribute the Program (or a work based on it,
under Section 2) in object code or executable form under the terms of
Sections 1 and 2 above provided that you also do one of the following:

\begin{enumerate}

\item
Accompany it with the complete corresponding machine-readable
source code, which must be distributed under the terms of Sections
1 and 2 above on a medium customarily used for software interchange; or,

\item
Accompany it with a written offer, valid for at least three
years, to give any third party, for a charge no more than your
cost of physically performing source distribution, a complete
machine-readable copy of the corresponding source code, to be
distributed under the terms of Sections 1 and 2 above on a medium
customarily used for software interchange; or,

\item
Accompany it with the information you received as to the offer
to distribute corresponding source code.  (This alternative is
allowed only for noncommercial distribution and only if you
received the program in object code or executable form with such
an offer, in accord with Subsubsection b above.)

\end{enumerate}


The source code for a work means the preferred form of the work for
making modifications to it.  For an executable work, complete source
code means all the source code for all modules it contains, plus any
associated interface definition files, plus the scripts used to
control compilation and installation of the executable.  However, as a
special exception, the source code distributed need not include
anything that is normally distributed (in either source or binary
form) with the major components (compiler, kernel, and so on) of the
operating system on which the executable runs, unless that component
itself accompanies the executable.

If distribution of executable or object code is made by offering
access to copy from a designated place, then offering equivalent
access to copy the source code from the same place counts as
distribution of the source code, even though third parties are not
compelled to copy the source along with the object code.

\item
You may not copy, modify, sublicense, or distribute the Program
except as expressly provided under this License.  Any attempt
otherwise to copy, modify, sublicense or distribute the Program is
void, and will automatically terminate your rights under this License.
However, parties who have received copies, or rights, from you under
this License will not have their licenses terminated so long as such
parties remain in full compliance.

\item
You are not required to accept this License, since you have not
signed it.  However, nothing else grants you permission to modify or
distribute the Program or its derivative works.  These actions are
prohibited by law if you do not accept this License.  Therefore, by
modifying or distributing the Program (or any work based on the
Program), you indicate your acceptance of this License to do so, and
all its terms and conditions for copying, distributing or modifying
the Program or works based on it.

\item
Each time you redistribute the Program (or any work based on the
Program), the recipient automatically receives a license from the
original licensor to copy, distribute or modify the Program subject to
these terms and conditions.  You may not impose any further
restrictions on the recipients' exercise of the rights granted herein.
You are not responsible for enforcing compliance by third parties to
this License.

\item
If, as a consequence of a court judgment or allegation of patent
infringement or for any other reason (not limited to patent issues),
conditions are imposed on you (whether by court order, agreement or
otherwise) that contradict the conditions of this License, they do not
excuse you from the conditions of this License.  If you cannot
distribute so as to satisfy simultaneously your obligations under this
License and any other pertinent obligations, then as a consequence you
may not distribute the Program at all.  For example, if a patent
license would not permit royalty-free redistribution of the Program by
all those who receive copies directly or indirectly through you, then
the only way you could satisfy both it and this License would be to
refrain entirely from distribution of the Program.

If any portion of this section is held invalid or unenforceable under
any particular circumstance, the balance of the section is intended to
apply and the section as a whole is intended to apply in other
circumstances.

It is not the purpose of this section to induce you to infringe any
patents or other property right claims or to contest validity of any
such claims; this section has the sole purpose of protecting the
integrity of the free software distribution system, which is
implemented by public license practices.  Many people have made
generous contributions to the wide range of software distributed
through that system in reliance on consistent application of that
system; it is up to the author/donor to decide if he or she is willing
to distribute software through any other system and a licensee cannot
impose that choice.

This section is intended to make thoroughly clear what is believed to
be a consequence of the rest of this License.

\item
If the distribution and/or use of the Program is restricted in
certain countries either by patents or by copyrighted interfaces, the
original copyright holder who places the Program under this License
may add an explicit geographical distribution limitation excluding
those countries, so that distribution is permitted only in or among
countries not thus excluded.  In such case, this License incorporates
the limitation as if written in the body of this License.

\item
The Free Software Foundation may publish revised and/or new versions
of the General Public License from time to time.  Such new versions will
be similar in spirit to the present version, but may differ in detail to
address new problems or concerns.

Each version is given a distinguishing version number.  If the Program
specifies a version number of this License which applies to it and ``any
later version'', you have the option of following the terms and conditions
either of that version or of any later version published by the Free
Software Foundation.  If the Program does not specify a version number of
this License, you may choose any version ever published by the Free Software
Foundation.

\item
If you wish to incorporate parts of the Program into other free
programs whose distribution conditions are different, write to the author
to ask for permission.  For software which is copyrighted by the Free
Software Foundation, write to the Free Software Foundation; we sometimes
make exceptions for this.  Our decision will be guided by the two goals
of preserving the free status of all derivatives of our free software and
of promoting the sharing and reuse of software generally.

\end{enumerate}

\subsubsection{No Warranty}

\begin{enumerate}

\addtocounter{enumi}{9}

\item
Because the program is licensed free of charge, there is no warranty
for the program, to the extent permitted by applicable law.  Except when
otherwise stated in writing the copyright holders and/or other parties
provide the program ``as is'' without warranty of any kind, either expressed
or implied, including, but not limited to, the implied warranties of
merchantability and fitness for a particular purpose.  The entire risk as
to the quality and performance of the program is with you.  Should the
program prove defective, you assume the cost of all necessary servicing,
repair or correction.

\item
In no event unless required by applicable law or agreed to in writing
will any copyright holder, or any other party who may modify and/or
redistribute the program as permitted above, be liable to you for damages,
including any general, special, incidental or consequential damages arising
out of the use or inability to use the program (including but not limited
to loss of data or data being rendered inaccurate or losses sustained by
you or third parties or a failure of the program to operate with any other
programs), even if such holder or other party has been advised of the
possibility of such damages.
\end{enumerate}


\providecommand{\LPPLsection}{\subsection}
\providecommand{\LPPLsubsection}{\subsubsection}
\providecommand{\LPPLsubsubsection}{\subsubsection}
\providecommand{\LPPLparagraph}{\paragraph}


% The file lppl.tex, some minor typographic changes:

%
% $Id: pgfmanual-en-license.tex,v 1.3 2006/10/11 10:03:00 tantau Exp $
%
% Copyright 1999 2002-2006 LaTeX3 Project
%    Everyone is allowed to distribute verbatim copies of this
%    license document, but modification of it is not allowed.
%
%
% If you wish to load it as part of a ``doc'' source, you have to
% ensure that a) % is a comment character and b) that short verb
% characters are being turned off, i.e.,
%
%   \DeleteShortVerb{\'}   % or whatever was made a shorthand
%   \MakePercentComment
%   \input{lppl}
%   \MakePercentIgnore
%   \MakeShortVerb{\'}     % turn it on again if necessary
%
%
% By default the license is produced with \section* as the highest
% heading level. If this is not appropriate for the document in which
% it is included define the commands listed below before loading this
% document, e.g., for inclusion as a separate chapter define:
%
%  \providecommand{\LPPLsection}{\chapter*}
%  \providecommand{\LPPLsubsection}{\section*}
%  \providecommand{\LPPLsubsubsection}{\subsection*}
%  \providecommand{\LPPLparagraph}{\subsubsection*}
%
% 
% To allow cross-referencing the headings \label's have been attached
% to them, all starting with ``LPPL:''. As by default headings without
% numbers are produced, this will only allow page references.
% However, you can use the titleref package to produce textual
% references or you change the definitions of \LPPLsection, and
% friends to generated numbered headings.
%
%
% We want it to be possible that this file can be processed by
% (pdf)LaTeX on its own, or that this file can be included in another
% LaTeX document without any modification whatsoever.
% Hence the little test below.
%
%
\makeatletter
\ifx\@preamblecmds\@notprerr
  % In this case the preamble has already been processed so this file
  % is loaded as part of another document; just enclose everything in
  % a group
  \let\LPPLicense\bgroup
  \let\endLPPLicense\egroup
\else
  % In this case the preamble has not been processed yet so this file
  % is processed by itself.
  \documentclass{article}
  \let\LPPLicense\document
  \let\endLPPLicense\enddocument
\fi
\makeatother


\begin{LPPLicense}
  \providecommand{\LPPLsection}{\section*}
  \providecommand{\LPPLsubsection}{\subsection*}
  \providecommand{\LPPLsubsubsection}{\subsubsection*}
  \providecommand{\LPPLparagraph}{\paragraph*}
  \providecommand*{\LPPLfile}[1]{\texttt{#1}}
  \providecommand*{\LPPLdocfile}[1]{`\LPPLfile{#1.tex}'}
  \providecommand*{\LPPL}{\textsc{lppl}}

  \LPPLsection{The \LaTeX\ Project Public License, Version 1.3c 2006-05-20}
  \label{LPPL:LPPL}

%  \textbf{Copyright 1999, 2002--2006 \LaTeX3 Project}
%  \begin{quotation}
%    Everyone is allowed to distribute verbatim copies of this
%    license document, but modification of it is not allowed.
%  \end{quotation}

  \LPPLsubsection{Preamble}
  \label{LPPL:Preamble}
  
  The \LaTeX\ Project Public License (\LPPL) is the primary license
  under which the the \LaTeX\ kernel and the base \LaTeX\ packages are
  distributed.

  You may use this license for any work of which you hold the
  copyright and which you wish to distribute.  This license may be
  particularly suitable if your work is \TeX-related (such as a
  \LaTeX\ package), but it is written in such a way that you can use 
  it even if your work is unrelated to \TeX.

  The section `\textsc{wheter and how to distribute works under this
    license}', below, gives instructions, examples, and
  recommendations 
  for authors who are considering distributing their works under this
  license.

  This license gives conditions under which a work may be distributed
  and modified, as well as conditions under which modified versions of
  that work may be distributed.

  We, the \LaTeX3 Project, believe that the conditions below give you
  the freedom to make and distribute modified versions of your work
  that conform with whatever technical specifications you wish while
  maintaining the availability, integrity, and reliability of that
  work.  If you do not see how to achieve your goal while meeting
  these conditions, then read the document \LPPLdocfile{cfgguide} and
  \LPPLdocfile{modguide} in the base \LaTeX\ distribution for suggestions.


  \LPPLsubsection{Definitions}
  \label{LPPL:Definitions}

  In this license document the following terms are used:

  \begin{description}
  \item[Work] Any work being distributed under this License.

  \item[Derived Work] Any work that under any applicable law is
    derived from the Work.

  \item[Modification] Any procedure that produces a Derived Work under
    any applicable law -- for example, the production of a file
    containing an original file associated with the Work or a
    significant portion of such a file, either verbatim or with
    modifications and/or translated into another language.

  \item[Modify] To apply any procedure that produces a Derived Work
    under any applicable law.
    
  \item[Distribution] Making copies of the Work available from one
    person to another, in whole or in part.  Distribution includes
    (but is not limited to) making any electronic components of the
    Work accessible by file transfer protocols such as \textsc{ftp} or
    \textsc{http} or by shared file systems such as Sun's Network File
    System (\textsc{nfs}).

  \item[Compiled Work] A version of the Work that has been processed
    into a form where it is directly usable on a computer system.
    This processing may include using installation facilities provided
    by the Work, transformations of the Work, copying of components of
    the Work, or other activities.  Note that modification of any
    installation facilities provided by the Work constitutes
    modification of the Work.

  \item[Current Maintainer] A person or persons nominated as such
    within the Work.  If there is no such explicit nomination then it
    is the `Copyright Holder' under any applicable law.

  \item[Base Interpreter] A program or process that is normally needed
    for running or interpreting a part or the whole of the Work.
    
    A Base Interpreter may depend on external components but these are
    not considered part of the Base Interpreter provided that each
    external component clearly identifies itself whenever it is used
    interactively.  Unless explicitly specified when applying the
    license to the Work, the only applicable Base Interpreter is a
    `\LaTeX-Format' or in the case of files belonging to the
    `\LaTeX-format' a program implementing the `\TeX{} language'.
  \end{description}

  \LPPLsubsection{Conditions on Distribution and Modification}
  \label{LPPL:Conditions}

  \begin{enumerate}
  \item Activities other than distribution and/or modification of the
    Work are not covered by this license; they are outside its scope.
    In particular, the act of running the Work is not restricted and
    no requirements are made concerning any offers of support for the
    Work.

  \item\label{LPPL:item:distribute} You may distribute a complete, unmodified
    copy of the Work as you received it.  Distribution of only part of
    the Work is considered modification of the Work, and no right to
    distribute such a Derived Work may be assumed under the terms of
    this clause.

  \item You may distribute a Compiled Work that has been generated
    from a complete, unmodified copy of the Work as distributed under
    Clause~\ref{LPPL:item:distribute} above, as long as that Compiled Work is
    distributed in such a way that the recipients may install the
    Compiled Work on their system exactly as it would have been
    installed if they generated a Compiled Work directly from the
    Work.

  \item\label{LPPL:item:currmaint} If you are the Current Maintainer of the
    Work, you may, without restriction, modify the Work, thus creating
    a Derived Work.  You may also distribute the Derived Work without
    restriction, including Compiled Works generated from the Derived
    Work.  Derived Works distributed in this manner by the Current
    Maintainer are considered to be updated versions of the Work.

  \item If you are not the Current Maintainer of the Work, you may
    modify your copy of the Work, thus creating a Derived Work based
    on the Work, and compile this Derived Work, thus creating a
    Compiled Work based on the Derived Work.

  \item\label{LPPL:item:conditions} If you are not the Current Maintainer of the
    Work, you may distribute a Derived Work provided the following
    conditions are met for every component of the Work unless that
    component clearly states in the copyright notice that it is exempt
    from that condition.  Only the Current Maintainer is allowed to
    add such statements of exemption to a component of the Work.
    \begin{enumerate}
    \item If a component of this Derived Work can be a direct
      replacement for a component of the Work when that component is
      used with the Base Interpreter, then, wherever this component of
      the Work identifies itself to the user when used interactively
      with that Base Interpreter, the replacement component of this
      Derived Work clearly and unambiguously identifies itself as a
      modified version of this component to the user when used
      interactively with that Base Interpreter.
     
    \item Every component of the Derived Work contains prominent
      notices detailing the nature of the changes to that component,
      or a prominent reference to another file that is distributed as
      part of the Derived Work and that contains a complete and
      accurate log of the changes.
  
    \item No information in the Derived Work implies that any persons,
      including (but not limited to) the authors of the original
      version of the Work, provide any support, including (but not
      limited to) the reporting and handling of errors, to recipients
      of the Derived Work unless those persons have stated explicitly
      that they do provide such support for the Derived Work.

    \item You distribute at least one of the following with the Derived Work:
      \begin{enumerate}
      \item A complete, unmodified copy of the Work; if your
        distribution of a modified component is made by offering
        access to copy the modified component from a designated place,
        then offering equivalent access to copy the Work from the same
        or some similar place meets this condition, even though third
        parties are not compelled to copy the Work along with the
        modified component;

      \item Information that is sufficient to obtain a complete,
        unmodified copy of the Work.
      \end{enumerate}
    \end{enumerate}
  \item If you are not the Current Maintainer of the Work, you may
    distribute a Compiled Work generated from a Derived Work, as long
    as the Derived Work is distributed to all recipients of the
    Compiled Work, and as long as the conditions of
    Clause~\ref{LPPL:item:conditions}, above, are met with regard to the Derived
    Work.

  \item The conditions above are not intended to prohibit, and hence
    do not apply to, the modification, by any method, of any component
    so that it becomes identical to an updated version of that
    component of the Work as it is distributed by the Current
    Maintainer under Clause~\ref{LPPL:item:currmaint}, above.

  \item Distribution of the Work or any Derived Work in an alternative
    format, where the Work or that Derived Work (in whole or in part)
    is then produced by applying some process to that format, does not
    relax or nullify any sections of this license as they pertain to
    the results of applying that process.
     
  \item \null
    \begin{enumerate}
    \item A Derived Work may be distributed under a different license
      provided that license itself honors the conditions listed in
      Clause~\ref{LPPL:item:conditions} above, in regard to the Work, though it
      does not have to honor the rest of the conditions in this
      license.
      
    \item If a Derived Work is distributed under a different license,
      that Derived Work must provide sufficient documentation as part
      of itself to allow each recipient of that Derived Work to honor
      the restrictions in Clause~\ref{LPPL:item:conditions} above, concerning
      changes from the Work.
    \end{enumerate}
  \item This license places no restrictions on works that are
    unrelated to the Work, nor does this license place any
    restrictions on aggregating such works with the Work by any means.

  \item Nothing in this license is intended to, or may be used to,
    prevent complete compliance by all parties with all applicable
    laws.
  \end{enumerate}

  \LPPLsubsection{No Warranty}
  \label{LPPL:Warranty}

  There is no warranty for the Work.  Except when otherwise stated in
  writing, the Copyright Holder provides the Work `as is', without
  warranty of any kind, either expressed or implied, including, but
  not limited to, the implied warranties of merchantability and
  fitness for a particular purpose.  The entire risk as to the quality
  and performance of the Work is with you.  Should the Work prove
  defective, you assume the cost of all necessary servicing, repair,
  or correction.

  In no event unless required by applicable law or agreed to in
  writing will The Copyright Holder, or any author named in the
  components of the Work, or any other party who may distribute and/or
  modify the Work as permitted above, be liable to you for damages,
  including any general, special, incidental or consequential damages
  arising out of any use of the Work or out of inability to use the
  Work (including, but not limited to, loss of data, data being
  rendered inaccurate, or losses sustained by anyone as a result of
  any failure of the Work to operate with any other programs), even if
  the Copyright Holder or said author or said other party has been
  advised of the possibility of such damages.

  \LPPLsubsection{Maintenance of The Work}
  \label{LPPL:Maintenance}

  The Work has the status `author-maintained' if the Copyright Holder
  explicitly and prominently states near the primary copyright notice
  in the Work that the Work can only be maintained by the Copyright
  Holder or simply that it is `author-maintained'.

  The Work has the status `maintained' if there is a Current
  Maintainer who has indicated in the Work that they are willing to
  receive error reports for the Work (for example, by supplying a
  valid e-mail address). It is not required for the Current Maintainer
  to acknowledge or act upon these error reports.

  The Work changes from status `maintained' to `unmaintained' if there
  is no Current Maintainer, or the person stated to be Current
  Maintainer of the work cannot be reached through the indicated means
  of communication for a period of six months, and there are no other
  significant signs of active maintenance.

  You can become the Current Maintainer of the Work by agreement with
  any existing Current Maintainer to take over this role.

  If the Work is unmaintained, you can become the Current Maintainer
  of the Work through the following steps:
  \begin{enumerate}
  \item Make a reasonable attempt to trace the Current Maintainer (and
    the Copyright Holder, if the two differ) through the means of an
    Internet or similar search.
  \item If this search is successful, then enquire whether the Work is
    still maintained.
    \begin{enumerate}
    \item If it is being maintained, then ask the Current Maintainer
      to update their communication data within one month.
     
    \item\label{LPPL:item:intention} If the search is unsuccessful or
      no action to resume active maintenance is taken by the Current
      Maintainer, then announce within the pertinent community your
      intention to take over maintenance.  (If the Work is a \LaTeX{}
      work, this could be done, for example, by posting to
      \texttt{comp.text.tex}.)
    \end{enumerate}
  \item {}
    \begin{enumerate}
    \item If the Current Maintainer is reachable and agrees to pass
      maintenance of the Work to you, then this takes effect
      immediately upon announcement.
     
    \item\label{LPPL:item:announce} If the Current Maintainer is not
      reachable and the Copyright Holder agrees that maintenance of
      the Work be passed to you, then this takes effect immediately
      upon announcement.
    \end{enumerate}
  \item\label{LPPL:item:change} If you make an `intention
    announcement' as described in~\ref{LPPL:item:intention} above and
    after three months your intention is challenged neither by the
    Current Maintainer nor by the Copyright Holder nor by other
    people, then you may arrange for the Work to be changed so as to
    name you as the (new) Current Maintainer.
     
  \item If the previously unreachable Current Maintainer becomes
    reachable once more within three months of a change completed
    under the terms of~\ref{LPPL:item:announce}
    or~\ref{LPPL:item:change}, then that Current Maintainer must
    become or remain the Current Maintainer upon request provided they
    then update their communication data within one month.
  \end{enumerate}
  A change in the Current Maintainer does not, of itself, alter the
  fact that the Work is distributed under the \LPPL\ license.

  If you become the Current Maintainer of the Work, you should
  immediately provide, within the Work, a prominent and unambiguous
  statement of your status as Current Maintainer.  You should also
  announce your new status to the same pertinent community as
  in~\ref{LPPL:item:intention} above.

  \LPPLsubsection{Whether and How to Distribute Works under This License}
  \label{LPPL:Distribute}

  This section contains important instructions, examples, and
  recommendations for authors who are considering distributing their
  works under this license.  These authors are addressed as `you' in
  this section.

  \LPPLsubsubsection{Choosing This License or Another License}
  \label{LPPL:Choosing}

  If for any part of your work you want or need to use
  \emph{distribution} conditions that differ significantly from those
  in this license, then do not refer to this license anywhere in your
  work but, instead, distribute your work under a different license.
  You may use the text of this license as a model for your own
  license, but your license should not refer to the \LPPL\ or
  otherwise give the impression that your work is distributed under
  the \LPPL.

  The document \LPPLdocfile{modguide} in the base \LaTeX\ distribution
  explains the motivation behind the conditions of this license.  It
  explains, for example, why distributing \LaTeX\ under the
  \textsc{gnu} General Public License (\textsc{gpl}) was considered
  inappropriate.  Even if your work is unrelated to \LaTeX, the
  discussion in \LPPLdocfile{modguide} may still be relevant, and authors
  intending to distribute their works under any license are encouraged
  to read it.

  \LPPLsubsubsection{A Recommendation on Modification Without Distribution}
  \label{LPPL:WithoutDistribution}

  It is wise never to modify a component of the Work, even for your
  own personal use, without also meeting the above conditions for
  distributing the modified component.  While you might intend that
  such modifications will never be distributed, often this will happen
  by accident -- you may forget that you have modified that component;
  or it may not occur to you when allowing others to access the
  modified version that you are thus distributing it and violating the
  conditions of this license in ways that could have legal
  implications and, worse, cause problems for the community.  It is
  therefore usually in your best interest to keep your copy of the
  Work identical with the public one.  Many works provide ways to
  control the behavior of that work without altering any of its
  licensed components.

  \LPPLsubsubsection{How to Use This License}
  \label{LPPL:HowTo}

  To use this license, place in each of the components of your work
  both an explicit copyright notice including your name and the year
  the work was authored and/or last substantially modified.  Include
  also a statement that the distribution and/or modification of that
  component is constrained by the conditions in this license.

  Here is an example of such a notice and statement:
\begin{verbatim}
  %% pig.dtx
  %% Copyright 2005 M. Y. Name
  %
  % This work may be distributed and/or modified under the
  % conditions of the LaTeX Project Public License, either version 1.3
  % of this license or (at your option) any later version.
  % The latest version of this license is in
  %   http://www.latex-project.org/lppl.txt
  % and version 1.3 or later is part of all distributions of LaTeX
  % version 2005/12/01 or later.
  %
  % This work has the LPPL maintenance status `maintained'.
  % 
  % The Current Maintainer of this work is M. Y. Name.
  %
  % This work consists of the files pig.dtx and pig.ins
  % and the derived file pig.sty.
\end{verbatim}
  
  Given such a notice and statement in a file, the conditions given in
  this license document would apply, with the `Work' referring to the
  three files `\LPPLfile{pig.dtx}', `\LPPLfile{pig.ins}', and
  `\LPPLfile{pig.sty}' (the last being generated from
  `\LPPLfile{pig.dtx}' using `\LPPLfile{pig.ins}'), the `Base
  Interpreter' referring to any `\LaTeX-Format', and both `Copyright
  Holder' and `Current Maintainer' referring to the person `M. Y.
  Name'.

  If you do not want the Maintenance section of \LPPL\ to apply to
  your Work, change `maintained' above into `author-maintained'.
  However, we recommend that you use `maintained' as the Maintenance
  section was added in order to ensure that your Work remains useful
  to the community even when you can no longer maintain and support it
  yourself.

  \LPPLsubsubsection{Derived Works That Are Not Replacements}
  \label{LPPL:NotReplacements}

  Several clauses of the \LPPL\ specify means to provide reliability
  and stability for the user community. They therefore concern
  themselves with the case that a Derived Work is intended to be used
  as a (compatible or incompatible) replacement of the original
  Work. If this is not the case (e.g., if a few lines of code are
  reused for a completely different task), then clauses 6b and 6d
  shall not apply.

  \LPPLsubsubsection{Important Recommendations}
  \label{LPPL:Recommendations}

  \LPPLparagraph{Defining What Constitutes the Work}

  The \LPPL\ requires that distributions of the Work contain all the
  files of the Work.  It is therefore important that you provide a way
  for the licensee to determine which files constitute the Work.  This
  could, for example, be achieved by explicitly listing all the files
  of the Work near the copyright notice of each file or by using a
  line such as:
\begin{verbatim}
    % This work consists of all files listed in manifest.txt.
\end{verbatim}
  in that place.  In the absence of an unequivocal list it might be
  impossible for the licensee to determine what is considered by you
  to comprise the Work and, in such a case, the licensee would be
  entitled to make reasonable conjectures as to which files comprise
  the Work.

\end{LPPLicense}




\subsection{GNU Free Documentation License, Version 1.2, November 2002}
\label{label_fdl}


%  \textbf{Copyright  2000,2001,2002  Free Software Foundation, Inc.}\par
%  51 Franklin St, Fifth Floor, Boston, MA  02110-1301  USA
%  \begin{quotation}
%    Everyone is allowed to distribute verbatim copies of this
%    license document, but modification of it is not allowed.
%  \end{quotation}

\subsubsection{Preamble}

The purpose of this License is to make a manual, textbook, or other
functional and useful document ``free'' in the sense of freedom: to
assure everyone the effective freedom to copy and redistribute it,
with or without modifying it, either commercially or noncommercially.
Secondarily, this License preserves for the author and publisher a way
to get credit for their work, while not being considered responsible
for modifications made by others.

This License is a kind of ``copyleft'', which means that derivative
works of the document must themselves be free in the same sense.  It
complements the GNU General Public License, which is a copyleft
license designed for free software.

We have designed this License in order to use it for manuals for free
software, because free software needs free documentation: a free
program should come with manuals providing the same freedoms that the
software does.  But this License is not limited to software manuals;
it can be used for any textual work, regardless of subject matter or
whether it is published as a printed book.  We recommend this License
principally for works whose purpose is instruction or reference.

\subsubsection{Applicability and definitions}

This License applies to any manual or other work, in any medium, that
contains a notice placed by the copyright holder saying it can be
distributed under the terms of this License.  Such a notice grants a
world-wide, royalty-free license, unlimited in duration, to use that
work under the conditions stated herein.  The \textbf{``Document''}, below,
refers to any such manual or work.  Any member of the public is a
licensee, and is addressed as \textbf{``you''}.  You accept the license if you
copy, modify or distribute the work in a way requiring permission
under copyright law.

A \textbf{``Modified Version''} of the Document means any work containing the
Document or a portion of it, either copied verbatim, or with
modifications and/or translated into another language.

A \textbf{``Secondary Section''} is a named appendix or a front-matter section of
the Document that deals exclusively with the relationship of the
publishers or authors of the Document to the Document's overall subject
(or to related matters) and contains nothing that could fall directly
within that overall subject.  (Thus, if the Document is in part a
textbook of mathematics, a Secondary Section may not explain any
mathematics.)  The relationship could be a matter of historical
connection with the subject or with related matters, or of legal,
commercial, philosophical, ethical or political position regarding
them.

The \textbf{``Invariant Sections''} are certain Secondary Sections whose titles
are designated, as being those of Invariant Sections, in the notice
that says that the Document is released under this License.  If a
section does not fit the above definition of Secondary then it is not
allowed to be designated as Invariant.  The Document may contain zero
Invariant Sections.  If the Document does not identify any Invariant
Sections then there are none.

The \textbf{``Cover Texts''} are certain short passages of text that are listed,
as Front-Cover Texts or Back-Cover Texts, in the notice that says that
the Document is released under this License.  A Front-Cover Text may
be at most 5 words, and a Back-Cover Text may be at most 25 words.

A \textbf{``Transparent''} copy of the Document means a machine-readable copy,
represented in a format whose specification is available to the
general public, that is suitable for revising the document
straightforwardly with generic text editors or (for images composed of
pixels) generic paint programs or (for drawings) some widely available
drawing editor, and that is suitable for input to text formatters or
for automatic translation to a variety of formats suitable for input
to text formatters.  A copy made in an otherwise Transparent file
format whose markup, or absence of markup, has been arranged to thwart
or discourage subsequent modification by readers is not Transparent.
An image format is not Transparent if used for any substantial amount
of text.  A copy that is not ``Transparent'' is called \textbf{``Opaque''}.

Examples of suitable formats for Transparent copies include plain
ASCII without markup, Texinfo input format, LaTeX input format, SGML
or XML using a publicly available DTD, and standard-conforming simple
HTML, PostScript or PDF designed for human modification.  Examples of
transparent image formats include PNG, XCF and JPG.  Opaque formats
include proprietary formats that can be read and edited only by
proprietary word processors, SGML or XML for which the DTD and/or
processing tools are not generally available, and the
machine-generated HTML, PostScript or PDF produced by some word
processors for output purposes only.

The \textbf{``Title Page''} means, for a printed book, the title page itself,
plus such following pages as are needed to hold, legibly, the material
this License requires to appear in the title page.  For works in
formats which do not have any title page as such, ``Title Page'' means
the text near the most prominent appearance of the work's title,
preceding the beginning of the body of the text.

A section \textbf{``Entitled XYZ''} means a named subunit of the Document whose
title either is precisely XYZ or contains XYZ in parentheses following
text that translates XYZ in another language.  (Here XYZ stands for a
specific section name mentioned below, such as \textbf{``Acknowledgements''},
\textbf{``Dedications''}, \textbf{``Endorsements''}, or \textbf{``History''}.)  
To \textbf{``Preserve the Title''}
of such a section when you modify the Document means that it remains a
section ``Entitled XYZ'' according to this definition.

The Document may include Warranty Disclaimers next to the notice which
states that this License applies to the Document.  These Warranty
Disclaimers are considered to be included by reference in this
License, but only as regards disclaiming warranties: any other
implication that these Warranty Disclaimers may have is void and has
no effect on the meaning of this License.

\subsubsection{Verbatim Copying}

You may copy and distribute the Document in any medium, either
commercially or noncommercially, provided that this License, the
copyright notices, and the license notice saying this License applies
to the Document are reproduced in all copies, and that you add no other
conditions whatsoever to those of this License.  You may not use
technical measures to obstruct or control the reading or further
copying of the copies you make or distribute.  However, you may accept
compensation in exchange for copies.  If you distribute a large enough
number of copies you must also follow the conditions in section 3.

You may also lend copies, under the same conditions stated above, and
you may publicly display copies.

\subsubsection{Copying in Quantity}

If you publish printed copies (or copies in media that commonly have
printed covers) of the Document, numbering more than 100, and the
Document's license notice requires Cover Texts, you must enclose the
copies in covers that carry, clearly and legibly, all these Cover
Texts: Front-Cover Texts on the front cover, and Back-Cover Texts on
the back cover.  Both covers must also clearly and legibly identify
you as the publisher of these copies.  The front cover must present
the full title with all words of the title equally prominent and
visible.  You may add other material on the covers in addition.
Copying with changes limited to the covers, as long as they preserve
the title of the Document and satisfy these conditions, can be treated
as verbatim copying in other respects.

If the required texts for either cover are too voluminous to fit
legibly, you should put the first ones listed (as many as fit
reasonably) on the actual cover, and continue the rest onto adjacent
pages.

If you publish or distribute Opaque copies of the Document numbering
more than 100, you must either include a machine-readable Transparent
copy along with each Opaque copy, or state in or with each Opaque copy
a computer-network location from which the general network-using
public has access to download using public-standard network protocols
a complete Transparent copy of the Document, free of added material.
If you use the latter option, you must take reasonably prudent steps,
when you begin distribution of Opaque copies in quantity, to ensure
that this Transparent copy will remain thus accessible at the stated
location until at least one year after the last time you distribute an
Opaque copy (directly or through your agents or retailers) of that
edition to the public.

It is requested, but not required, that you contact the authors of the
Document well before redistributing any large number of copies, to give
them a chance to provide you with an updated version of the Document.

\subsubsection{Modifications}

You may copy and distribute a Modified Version of the Document under
the conditions of sections 2 and 3 above, provided that you release
the Modified Version under precisely this License, with the Modified
Version filling the role of the Document, thus licensing distribution
and modification of the Modified Version to whoever possesses a copy
of it.  In addition, you must do these things in the Modified Version:

\begin{itemize}
\item[A.] 
   Use in the Title Page (and on the covers, if any) a title distinct
   from that of the Document, and from those of previous versions
   (which should, if there were any, be listed in the History section
   of the Document).  You may use the same title as a previous version
   if the original publisher of that version gives permission.
   
\item[B.]
   List on the Title Page, as authors, one or more persons or entities
   responsible for authorship of the modifications in the Modified
   Version, together with at least five of the principal authors of the
   Document (all of its principal authors, if it has fewer than five),
   unless they release you from this requirement.
   
\item[C.]
   State on the Title page the name of the publisher of the
   Modified Version, as the publisher.
   
\item[D.]
   Preserve all the copyright notices of the Document.
   
\item[E.]
   Add an appropriate copyright notice for your modifications
   adjacent to the other copyright notices.
   
\item[F.]
   Include, immediately after the copyright notices, a license notice
   giving the public permission to use the Modified Version under the
   terms of this License, in the form shown in the Addendum below.
   
\item[G.]
   Preserve in that license notice the full lists of Invariant Sections
   and required Cover Texts given in the Document's license notice.
   
\item[H.]
   Include an unaltered copy of this License.
   
\item[I.]
   Preserve the section Entitled ``History'', Preserve its Title, and add
   to it an item stating at least the title, year, new authors, and
   publisher of the Modified Version as given on the Title Page.  If
   there is no section Entitled ``History'' in the Document, create one
   stating the title, year, authors, and publisher of the Document as
   given on its Title Page, then add an item describing the Modified
   Version as stated in the previous sentence.
   
\item[J.]
   Preserve the network location, if any, given in the Document for
   public access to a Transparent copy of the Document, and likewise
   the network locations given in the Document for previous versions
   it was based on.  These may be placed in the ``History'' section.
   You may omit a network location for a work that was published at
   least four years before the Document itself, or if the original
   publisher of the version it refers to gives permission.
   
\item[K.]
   For any section Entitled ``Acknowledgements'' or ``Dedications'',
   Preserve the Title of the section, and preserve in the section all
   the substance and tone of each of the contributor acknowledgements
   and/or dedications given therein.
   
\item[L.]
   Preserve all the Invariant Sections of the Document,
   unaltered in their text and in their titles.  Section numbers
   or the equivalent are not considered part of the section titles.
   
\item[M.]
   Delete any section Entitled ``Endorsements''.  Such a section
   may not be included in the Modified Version.
   
\item[N.]
   Do not retitle any existing section to be Entitled ``Endorsements''
   or to conflict in title with any Invariant Section.
   
\item[O.]
   Preserve any Warranty Disclaimers.
\end{itemize}

If the Modified Version includes new front-matter sections or
appendices that qualify as Secondary Sections and contain no material
copied from the Document, you may at your option designate some or all
of these sections as invariant.  To do this, add their titles to the
list of Invariant Sections in the Modified Version's license notice.
These titles must be distinct from any other section titles.

You may add a section Entitled ``Endorsements'', provided it contains
nothing but endorsements of your Modified Version by various
parties--for example, statements of peer review or that the text has
been approved by an organization as the authoritative definition of a
standard.

You may add a passage of up to five words as a Front-Cover Text, and a
passage of up to 25 words as a Back-Cover Text, to the end of the list
of Cover Texts in the Modified Version.  Only one passage of
Front-Cover Text and one of Back-Cover Text may be added by (or
through arrangements made by) any one entity.  If the Document already
includes a cover text for the same cover, previously added by you or
by arrangement made by the same entity you are acting on behalf of,
you may not add another; but you may replace the old one, on explicit
permission from the previous publisher that added the old one.

The author(s) and publisher(s) of the Document do not by this License
give permission to use their names for publicity for or to assert or
imply endorsement of any Modified Version.

\subsubsection{Combining Documents}

You may combine the Document with other documents released under this
License, under the terms defined in section 4 above for modified
versions, provided that you include in the combination all of the
Invariant Sections of all of the original documents, unmodified, and
list them all as Invariant Sections of your combined work in its
license notice, and that you preserve all their Warranty Disclaimers.

The combined work need only contain one copy of this License, and
multiple identical Invariant Sections may be replaced with a single
copy.  If there are multiple Invariant Sections with the same name but
different contents, make the title of each such section unique by
adding at the end of it, in parentheses, the name of the original
author or publisher of that section if known, or else a unique number.
Make the same adjustment to the section titles in the list of
Invariant Sections in the license notice of the combined work.

In the combination, you must combine any sections Entitled ``History''
in the various original documents, forming one section Entitled
``History''; likewise combine any sections Entitled ``Acknowledgements'',
and any sections Entitled ``Dedications''.  You must delete all sections
Entitled ``Endorsements''.


\subsubsection{Collection of Documents}

You may make a collection consisting of the Document and other documents
released under this License, and replace the individual copies of this
License in the various documents with a single copy that is included in
the collection, provided that you follow the rules of this License for
verbatim copying of each of the documents in all other respects.

You may extract a single document from such a collection, and distribute
it individually under this License, provided you insert a copy of this
License into the extracted document, and follow this License in all
other respects regarding verbatim copying of that document.


\subsubsection{Aggregating with independent Works}

A compilation of the Document or its derivatives with other separate
and independent documents or works, in or on a volume of a storage or
distribution medium, is called an ``aggregate'' if the copyright
resulting from the compilation is not used to limit the legal rights
of the compilation's users beyond what the individual works permit.
When the Document is included in an aggregate, this License does not
apply to the other works in the aggregate which are not themselves
derivative works of the Document.

If the Cover Text requirement of section 3 is applicable to these
copies of the Document, then if the Document is less than one half of
the entire aggregate, the Document's Cover Texts may be placed on
covers that bracket the Document within the aggregate, or the
electronic equivalent of covers if the Document is in electronic form.
Otherwise they must appear on printed covers that bracket the whole
aggregate.



\subsubsection{Translation}

Translation is considered a kind of modification, so you may
distribute translations of the Document under the terms of section 4.
Replacing Invariant Sections with translations requires special
permission from their copyright holders, but you may include
translations of some or all Invariant Sections in addition to the
original versions of these Invariant Sections.  You may include a
translation of this License, and all the license notices in the
Document, and any Warranty Disclaimers, provided that you also include
the original English version of this License and the original versions
of those notices and disclaimers.  In case of a disagreement between
the translation and the original version of this License or a notice
or disclaimer, the original version will prevail.

If a section in the Document is Entitled ``Acknowledgements'',
``Dedications'', or ``History'', the requirement (section 4) to Preserve
its Title (section 1) will typically require changing the actual
title.


\subsubsection{Termination}

You may not copy, modify, sublicense, or distribute the Document except
as expressly provided for under this License.  Any other attempt to
copy, modify, sublicense or distribute the Document is void, and will
automatically terminate your rights under this License.  However,
parties who have received copies, or rights, from you under this
License will not have their licenses terminated so long as such
parties remain in full compliance.


\subsubsection{Future Revisions of this License}

The Free Software Foundation may publish new, revised versions
of the GNU Free Documentation License from time to time.  Such new
versions will be similar in spirit to the present version, but may
differ in detail to address new problems or concerns.  See
http://www.gnu.org/copyleft/.

Each version of the License is given a distinguishing version number.
If the Document specifies that a particular numbered version of this
License ``or any later version'' applies to it, you have the option of
following the terms and conditions either of that specified version or
of any later version that has been published (not as a draft) by the
Free Software Foundation.  If the Document does not specify a version
number of this License, you may choose any version ever published (not
as a draft) by the Free Software Foundation.


\subsubsection{Addendum: How to use this License for your documents}

To use this License in a document you have written, include a copy of
the License in the document and put the following copyright and
license notices just after the title page:

\bigskip
\begin{quote}
    Copyright \copyright \textsc{year your name}.
    Permission is granted to copy, distribute and/or modify this document
    under the terms of the GNU Free Documentation License, Version 1.2
    or any later version published by the Free Software Foundation;
    with no Invariant Sections, no Front-Cover Texts, and no Back-Cover Texts.
    A copy of the license is included in the section entitled ``GNU
    Free Documentation License''.
\end{quote}
\bigskip
    
If you have Invariant Sections, Front-Cover Texts and Back-Cover Texts,
replace the ``with \dots\ Texts.'' line with this:

\bigskip
\begin{quote}
    with the Invariant Sections being \textsc{list their titles}, with the
    Front-Cover Texts being \textsc{list}, and with the Back-Cover
    Texts being \textsc{list}. 
\end{quote}
\bigskip
    
If you have Invariant Sections without Cover Texts, or some other
combination of the three, merge those two alternatives to suit the
situation.

If your document contains nontrivial examples of program code, we
recommend releasing these examples in parallel under your choice of
free software license, such as the GNU General Public License,
to permit their use in free software.



%%% Local Variables: 
%%% mode: latex
%%% TeX-master: "beameruserguide"
%%% End: 


% Copyright 2006 by Till Tantau
%
% This file may be distributed and/or modified
%
% 1. under the LaTeX Project Public License and/or
% 2. under the GNU Free Documentation License.
%
% See the file doc/generic/pgf/licenses/LICENSE for more details.


\section{Licenses and Copyright}
\label{section-license}


\subsection{Which License Applies?}

Different parts of the \pgfname\ package are distributed under
different licenses:

\begin{enumerate}
\item The \emph{code} of the package is dual-license. This means that
  you can decide which license you wish to use when using the
  \pgfname\ package. The two options are:
  \begin{enumerate}
  \item You can use the \textsc{gnu} Public License, version 2.
  \item You can use the \LaTeX\ Project Public License, version 1.3c.
  \end{enumerate}
\item The \emph{documentation} of the package is also dual-license. Again,
  you can choose between two options:
  \begin{enumerate}
  \item You can use the \textsc{gnu} Free Documentation License, version 1.2.
  \item You can use the \LaTeX\ Project Public License, version 1.3c.
  \end{enumerate}
\end{enumerate}

The ``documentation of the package'' refers to all files in the
subdirectory |doc| of the |pgf| package. A detailed listing can be
found in the file |doc/generic/pgf/licenses/manifest-documentation.txt|.
All files in other directories are part of the ``code of the
package.'' A detailed listing can be found in the file
|doc/generic/pgf/licenses/manifest-code.txt|.

In the resest of this section, the licenses are presented. The
following text is copyrighted, see the plain text versions of these
licenses in the directory |doc/generic/pgf/licenses| for details. 

The example picture used in this manual, the Brave \textsc{gnu} World
logo, is taken from the Brave \textsc{gnu} World homepage, where it is
copyrighted as follows: ``Copyright (C) 1999, 2000, 2001, 2002, 2003,
2004 Georg C.~F.\ Greve. Permission is granted to make and distribute
verbatim copies of this transcript as long as the copyright and this
permission notice appear.'' 


\subsection{The GNU Public License, Version 2}

\subsubsection{Preamble}

The licenses for most software are designed to take away your freedom to
share and change it.  By contrast, the \textsc{gnu} General Public License is
intended to guarantee your freedom to share and change free software---to
make sure the software is free for all its users.  This General Public
License applies to most of the Free Software Foundation's software and to
any other program whose authors commit to using it.  (Some other Free
Software Foundation software is covered by the \textsc{gnu} Library
General Public License instead.)  You can apply it to your programs, too.

When we speak of free software, we are referring to freedom, not price.
Our General Public Licenses are designed to make sure that you have the
freedom to distribute copies of free software (and charge for this service
if you wish), that you receive source code or can get it if you want it,
that you can change the software or use pieces of it in new free programs;
and that you know you can do these things.

To protect your rights, we need to make restrictions that forbid anyone to
deny you these rights or to ask you to surrender the rights.  These
restrictions translate to certain responsibilities for you if you
distribute copies of the software, or if you modify it.

For example, if you distribute copies of such a program, whether gratis or
for a fee, you must give the recipients all the rights that you have.  You
must make sure that they, too, receive or can get the source code.  And
you must show them these terms so they know their rights.

We protect your rights with two steps: (1) copyright the software, and (2)
offer you this license which gives you legal permission to copy,
distribute and/or modify the software.

Also, for each author's protection and ours, we want to make certain that
everyone understands that there is no warranty for this free software.  If
the software is modified by someone else and passed on, we want its
recipients to know that what they have is not the original, so that any
problems introduced by others will not reflect on the original authors'
reputations.

Finally, any free program is threatened constantly by software patents.
We wish to avoid the danger that redistributors of a free program will
individually obtain patent licenses, in effect making the program
proprietary.  To prevent this, we have made it clear that any patent must
be licensed for everyone's free use or not licensed at all.

The precise terms and conditions for copying, distribution and
modification follow.

\subsubsection{Terms and Conditions For Copying, Distribution and
  Modification}

\begin{enumerate}

\addtocounter{enumi}{-1}

\item 
This License applies to any program or other work which contains a notice
placed by the copyright holder saying it may be distributed under the
terms of this General Public License.  The ``Program'', below, refers to
any such program or work, and a ``work based on the Program'' means either
the Program or any derivative work under copyright law: that is to say, a
work containing the Program or a portion of it, either verbatim or with
modifications and/or translated into another language.  (Hereinafter,
translation is included without limitation in the term ``modification''.)
Each licensee is addressed as ``you''.

Activities other than copying, distribution and modification are not
covered by this License; they are outside its scope.  The act of
running the Program is not restricted, and the output from the Program
is covered only if its contents constitute a work based on the
Program (independent of having been made by running the Program).
Whether that is true depends on what the Program does.

\item You may copy and distribute verbatim copies of the Program's source
  code as you receive it, in any medium, provided that you conspicuously
  and appropriately publish on each copy an appropriate copyright notice
  and disclaimer of warranty; keep intact all the notices that refer to
  this License and to the absence of any warranty; and give any other
  recipients of the Program a copy of this License along with the Program.

You may charge a fee for the physical act of transferring a copy, and you
may at your option offer warranty protection in exchange for a fee.

\item
You may modify your copy or copies of the Program or any portion
of it, thus forming a work based on the Program, and copy and
distribute such modifications or work under the terms of Section 1
above, provided that you also meet all of these conditions:

\begin{enumerate}

\item 
You must cause the modified files to carry prominent notices stating that
you changed the files and the date of any change.

\item
You must cause any work that you distribute or publish, that in
whole or in part contains or is derived from the Program or any
part thereof, to be licensed as a whole at no charge to all third
parties under the terms of this License.

\item
If the modified program normally reads commands interactively
when run, you must cause it, when started running for such
interactive use in the most ordinary way, to print or display an
announcement including an appropriate copyright notice and a
notice that there is no warranty (or else, saying that you provide
a warranty) and that users may redistribute the program under
these conditions, and telling the user how to view a copy of this
License.  (Exception: if the Program itself is interactive but
does not normally print such an announcement, your work based on
the Program is not required to print an announcement.)

\end{enumerate}


These requirements apply to the modified work as a whole.  If
identifiable sections of that work are not derived from the Program,
and can be reasonably considered independent and separate works in
themselves, then this License, and its terms, do not apply to those
sections when you distribute them as separate works.  But when you
distribute the same sections as part of a whole which is a work based
on the Program, the distribution of the whole must be on the terms of
this License, whose permissions for other licensees extend to the
entire whole, and thus to each and every part regardless of who wrote it.

Thus, it is not the intent of this section to claim rights or contest
your rights to work written entirely by you; rather, the intent is to
exercise the right to control the distribution of derivative or
collective works based on the Program.

In addition, mere aggregation of another work not based on the Program
with the Program (or with a work based on the Program) on a volume of
a storage or distribution medium does not bring the other work under
the scope of this License.

\item
You may copy and distribute the Program (or a work based on it,
under Section 2) in object code or executable form under the terms of
Sections 1 and 2 above provided that you also do one of the following:

\begin{enumerate}

\item
Accompany it with the complete corresponding machine-readable
source code, which must be distributed under the terms of Sections
1 and 2 above on a medium customarily used for software interchange; or,

\item
Accompany it with a written offer, valid for at least three
years, to give any third party, for a charge no more than your
cost of physically performing source distribution, a complete
machine-readable copy of the corresponding source code, to be
distributed under the terms of Sections 1 and 2 above on a medium
customarily used for software interchange; or,

\item
Accompany it with the information you received as to the offer
to distribute corresponding source code.  (This alternative is
allowed only for noncommercial distribution and only if you
received the program in object code or executable form with such
an offer, in accord with Subsubsection b above.)

\end{enumerate}


The source code for a work means the preferred form of the work for
making modifications to it.  For an executable work, complete source
code means all the source code for all modules it contains, plus any
associated interface definition files, plus the scripts used to
control compilation and installation of the executable.  However, as a
special exception, the source code distributed need not include
anything that is normally distributed (in either source or binary
form) with the major components (compiler, kernel, and so on) of the
operating system on which the executable runs, unless that component
itself accompanies the executable.

If distribution of executable or object code is made by offering
access to copy from a designated place, then offering equivalent
access to copy the source code from the same place counts as
distribution of the source code, even though third parties are not
compelled to copy the source along with the object code.

\item
You may not copy, modify, sublicense, or distribute the Program
except as expressly provided under this License.  Any attempt
otherwise to copy, modify, sublicense or distribute the Program is
void, and will automatically terminate your rights under this License.
However, parties who have received copies, or rights, from you under
this License will not have their licenses terminated so long as such
parties remain in full compliance.

\item
You are not required to accept this License, since you have not
signed it.  However, nothing else grants you permission to modify or
distribute the Program or its derivative works.  These actions are
prohibited by law if you do not accept this License.  Therefore, by
modifying or distributing the Program (or any work based on the
Program), you indicate your acceptance of this License to do so, and
all its terms and conditions for copying, distributing or modifying
the Program or works based on it.

\item
Each time you redistribute the Program (or any work based on the
Program), the recipient automatically receives a license from the
original licensor to copy, distribute or modify the Program subject to
these terms and conditions.  You may not impose any further
restrictions on the recipients' exercise of the rights granted herein.
You are not responsible for enforcing compliance by third parties to
this License.

\item
If, as a consequence of a court judgment or allegation of patent
infringement or for any other reason (not limited to patent issues),
conditions are imposed on you (whether by court order, agreement or
otherwise) that contradict the conditions of this License, they do not
excuse you from the conditions of this License.  If you cannot
distribute so as to satisfy simultaneously your obligations under this
License and any other pertinent obligations, then as a consequence you
may not distribute the Program at all.  For example, if a patent
license would not permit royalty-free redistribution of the Program by
all those who receive copies directly or indirectly through you, then
the only way you could satisfy both it and this License would be to
refrain entirely from distribution of the Program.

If any portion of this section is held invalid or unenforceable under
any particular circumstance, the balance of the section is intended to
apply and the section as a whole is intended to apply in other
circumstances.

It is not the purpose of this section to induce you to infringe any
patents or other property right claims or to contest validity of any
such claims; this section has the sole purpose of protecting the
integrity of the free software distribution system, which is
implemented by public license practices.  Many people have made
generous contributions to the wide range of software distributed
through that system in reliance on consistent application of that
system; it is up to the author/donor to decide if he or she is willing
to distribute software through any other system and a licensee cannot
impose that choice.

This section is intended to make thoroughly clear what is believed to
be a consequence of the rest of this License.

\item
If the distribution and/or use of the Program is restricted in
certain countries either by patents or by copyrighted interfaces, the
original copyright holder who places the Program under this License
may add an explicit geographical distribution limitation excluding
those countries, so that distribution is permitted only in or among
countries not thus excluded.  In such case, this License incorporates
the limitation as if written in the body of this License.

\item
The Free Software Foundation may publish revised and/or new versions
of the General Public License from time to time.  Such new versions will
be similar in spirit to the present version, but may differ in detail to
address new problems or concerns.

Each version is given a distinguishing version number.  If the Program
specifies a version number of this License which applies to it and ``any
later version'', you have the option of following the terms and conditions
either of that version or of any later version published by the Free
Software Foundation.  If the Program does not specify a version number of
this License, you may choose any version ever published by the Free Software
Foundation.

\item
If you wish to incorporate parts of the Program into other free
programs whose distribution conditions are different, write to the author
to ask for permission.  For software which is copyrighted by the Free
Software Foundation, write to the Free Software Foundation; we sometimes
make exceptions for this.  Our decision will be guided by the two goals
of preserving the free status of all derivatives of our free software and
of promoting the sharing and reuse of software generally.

\end{enumerate}

\subsubsection{No Warranty}

\begin{enumerate}

\addtocounter{enumi}{9}

\item
Because the program is licensed free of charge, there is no warranty
for the program, to the extent permitted by applicable law.  Except when
otherwise stated in writing the copyright holders and/or other parties
provide the program ``as is'' without warranty of any kind, either expressed
or implied, including, but not limited to, the implied warranties of
merchantability and fitness for a particular purpose.  The entire risk as
to the quality and performance of the program is with you.  Should the
program prove defective, you assume the cost of all necessary servicing,
repair or correction.

\item
In no event unless required by applicable law or agreed to in writing
will any copyright holder, or any other party who may modify and/or
redistribute the program as permitted above, be liable to you for damages,
including any general, special, incidental or consequential damages arising
out of the use or inability to use the program (including but not limited
to loss of data or data being rendered inaccurate or losses sustained by
you or third parties or a failure of the program to operate with any other
programs), even if such holder or other party has been advised of the
possibility of such damages.
\end{enumerate}


\providecommand{\LPPLsection}{\subsection}
\providecommand{\LPPLsubsection}{\subsubsection}
\providecommand{\LPPLsubsubsection}{\subsubsection}
\providecommand{\LPPLparagraph}{\paragraph}


% The file lppl.tex, some minor typographic changes:

%
% $Id: pgfmanual-en-license.tex,v 1.3 2006/10/11 10:03:00 tantau Exp $
%
% Copyright 1999 2002-2006 LaTeX3 Project
%    Everyone is allowed to distribute verbatim copies of this
%    license document, but modification of it is not allowed.
%
%
% If you wish to load it as part of a ``doc'' source, you have to
% ensure that a) % is a comment character and b) that short verb
% characters are being turned off, i.e.,
%
%   \DeleteShortVerb{\'}   % or whatever was made a shorthand
%   \MakePercentComment
%   \input{lppl}
%   \MakePercentIgnore
%   \MakeShortVerb{\'}     % turn it on again if necessary
%
%
% By default the license is produced with \section* as the highest
% heading level. If this is not appropriate for the document in which
% it is included define the commands listed below before loading this
% document, e.g., for inclusion as a separate chapter define:
%
%  \providecommand{\LPPLsection}{\chapter*}
%  \providecommand{\LPPLsubsection}{\section*}
%  \providecommand{\LPPLsubsubsection}{\subsection*}
%  \providecommand{\LPPLparagraph}{\subsubsection*}
%
% 
% To allow cross-referencing the headings \label's have been attached
% to them, all starting with ``LPPL:''. As by default headings without
% numbers are produced, this will only allow page references.
% However, you can use the titleref package to produce textual
% references or you change the definitions of \LPPLsection, and
% friends to generated numbered headings.
%
%
% We want it to be possible that this file can be processed by
% (pdf)LaTeX on its own, or that this file can be included in another
% LaTeX document without any modification whatsoever.
% Hence the little test below.
%
%
\makeatletter
\ifx\@preamblecmds\@notprerr
  % In this case the preamble has already been processed so this file
  % is loaded as part of another document; just enclose everything in
  % a group
  \let\LPPLicense\bgroup
  \let\endLPPLicense\egroup
\else
  % In this case the preamble has not been processed yet so this file
  % is processed by itself.
  \documentclass{article}
  \let\LPPLicense\document
  \let\endLPPLicense\enddocument
\fi
\makeatother


\begin{LPPLicense}
  \providecommand{\LPPLsection}{\section*}
  \providecommand{\LPPLsubsection}{\subsection*}
  \providecommand{\LPPLsubsubsection}{\subsubsection*}
  \providecommand{\LPPLparagraph}{\paragraph*}
  \providecommand*{\LPPLfile}[1]{\texttt{#1}}
  \providecommand*{\LPPLdocfile}[1]{`\LPPLfile{#1.tex}'}
  \providecommand*{\LPPL}{\textsc{lppl}}

  \LPPLsection{The \LaTeX\ Project Public License, Version 1.3c 2006-05-20}
  \label{LPPL:LPPL}

%  \textbf{Copyright 1999, 2002--2006 \LaTeX3 Project}
%  \begin{quotation}
%    Everyone is allowed to distribute verbatim copies of this
%    license document, but modification of it is not allowed.
%  \end{quotation}

  \LPPLsubsection{Preamble}
  \label{LPPL:Preamble}
  
  The \LaTeX\ Project Public License (\LPPL) is the primary license
  under which the the \LaTeX\ kernel and the base \LaTeX\ packages are
  distributed.

  You may use this license for any work of which you hold the
  copyright and which you wish to distribute.  This license may be
  particularly suitable if your work is \TeX-related (such as a
  \LaTeX\ package), but it is written in such a way that you can use 
  it even if your work is unrelated to \TeX.

  The section `\textsc{wheter and how to distribute works under this
    license}', below, gives instructions, examples, and
  recommendations 
  for authors who are considering distributing their works under this
  license.

  This license gives conditions under which a work may be distributed
  and modified, as well as conditions under which modified versions of
  that work may be distributed.

  We, the \LaTeX3 Project, believe that the conditions below give you
  the freedom to make and distribute modified versions of your work
  that conform with whatever technical specifications you wish while
  maintaining the availability, integrity, and reliability of that
  work.  If you do not see how to achieve your goal while meeting
  these conditions, then read the document \LPPLdocfile{cfgguide} and
  \LPPLdocfile{modguide} in the base \LaTeX\ distribution for suggestions.


  \LPPLsubsection{Definitions}
  \label{LPPL:Definitions}

  In this license document the following terms are used:

  \begin{description}
  \item[Work] Any work being distributed under this License.

  \item[Derived Work] Any work that under any applicable law is
    derived from the Work.

  \item[Modification] Any procedure that produces a Derived Work under
    any applicable law -- for example, the production of a file
    containing an original file associated with the Work or a
    significant portion of such a file, either verbatim or with
    modifications and/or translated into another language.

  \item[Modify] To apply any procedure that produces a Derived Work
    under any applicable law.
    
  \item[Distribution] Making copies of the Work available from one
    person to another, in whole or in part.  Distribution includes
    (but is not limited to) making any electronic components of the
    Work accessible by file transfer protocols such as \textsc{ftp} or
    \textsc{http} or by shared file systems such as Sun's Network File
    System (\textsc{nfs}).

  \item[Compiled Work] A version of the Work that has been processed
    into a form where it is directly usable on a computer system.
    This processing may include using installation facilities provided
    by the Work, transformations of the Work, copying of components of
    the Work, or other activities.  Note that modification of any
    installation facilities provided by the Work constitutes
    modification of the Work.

  \item[Current Maintainer] A person or persons nominated as such
    within the Work.  If there is no such explicit nomination then it
    is the `Copyright Holder' under any applicable law.

  \item[Base Interpreter] A program or process that is normally needed
    for running or interpreting a part or the whole of the Work.
    
    A Base Interpreter may depend on external components but these are
    not considered part of the Base Interpreter provided that each
    external component clearly identifies itself whenever it is used
    interactively.  Unless explicitly specified when applying the
    license to the Work, the only applicable Base Interpreter is a
    `\LaTeX-Format' or in the case of files belonging to the
    `\LaTeX-format' a program implementing the `\TeX{} language'.
  \end{description}

  \LPPLsubsection{Conditions on Distribution and Modification}
  \label{LPPL:Conditions}

  \begin{enumerate}
  \item Activities other than distribution and/or modification of the
    Work are not covered by this license; they are outside its scope.
    In particular, the act of running the Work is not restricted and
    no requirements are made concerning any offers of support for the
    Work.

  \item\label{LPPL:item:distribute} You may distribute a complete, unmodified
    copy of the Work as you received it.  Distribution of only part of
    the Work is considered modification of the Work, and no right to
    distribute such a Derived Work may be assumed under the terms of
    this clause.

  \item You may distribute a Compiled Work that has been generated
    from a complete, unmodified copy of the Work as distributed under
    Clause~\ref{LPPL:item:distribute} above, as long as that Compiled Work is
    distributed in such a way that the recipients may install the
    Compiled Work on their system exactly as it would have been
    installed if they generated a Compiled Work directly from the
    Work.

  \item\label{LPPL:item:currmaint} If you are the Current Maintainer of the
    Work, you may, without restriction, modify the Work, thus creating
    a Derived Work.  You may also distribute the Derived Work without
    restriction, including Compiled Works generated from the Derived
    Work.  Derived Works distributed in this manner by the Current
    Maintainer are considered to be updated versions of the Work.

  \item If you are not the Current Maintainer of the Work, you may
    modify your copy of the Work, thus creating a Derived Work based
    on the Work, and compile this Derived Work, thus creating a
    Compiled Work based on the Derived Work.

  \item\label{LPPL:item:conditions} If you are not the Current Maintainer of the
    Work, you may distribute a Derived Work provided the following
    conditions are met for every component of the Work unless that
    component clearly states in the copyright notice that it is exempt
    from that condition.  Only the Current Maintainer is allowed to
    add such statements of exemption to a component of the Work.
    \begin{enumerate}
    \item If a component of this Derived Work can be a direct
      replacement for a component of the Work when that component is
      used with the Base Interpreter, then, wherever this component of
      the Work identifies itself to the user when used interactively
      with that Base Interpreter, the replacement component of this
      Derived Work clearly and unambiguously identifies itself as a
      modified version of this component to the user when used
      interactively with that Base Interpreter.
     
    \item Every component of the Derived Work contains prominent
      notices detailing the nature of the changes to that component,
      or a prominent reference to another file that is distributed as
      part of the Derived Work and that contains a complete and
      accurate log of the changes.
  
    \item No information in the Derived Work implies that any persons,
      including (but not limited to) the authors of the original
      version of the Work, provide any support, including (but not
      limited to) the reporting and handling of errors, to recipients
      of the Derived Work unless those persons have stated explicitly
      that they do provide such support for the Derived Work.

    \item You distribute at least one of the following with the Derived Work:
      \begin{enumerate}
      \item A complete, unmodified copy of the Work; if your
        distribution of a modified component is made by offering
        access to copy the modified component from a designated place,
        then offering equivalent access to copy the Work from the same
        or some similar place meets this condition, even though third
        parties are not compelled to copy the Work along with the
        modified component;

      \item Information that is sufficient to obtain a complete,
        unmodified copy of the Work.
      \end{enumerate}
    \end{enumerate}
  \item If you are not the Current Maintainer of the Work, you may
    distribute a Compiled Work generated from a Derived Work, as long
    as the Derived Work is distributed to all recipients of the
    Compiled Work, and as long as the conditions of
    Clause~\ref{LPPL:item:conditions}, above, are met with regard to the Derived
    Work.

  \item The conditions above are not intended to prohibit, and hence
    do not apply to, the modification, by any method, of any component
    so that it becomes identical to an updated version of that
    component of the Work as it is distributed by the Current
    Maintainer under Clause~\ref{LPPL:item:currmaint}, above.

  \item Distribution of the Work or any Derived Work in an alternative
    format, where the Work or that Derived Work (in whole or in part)
    is then produced by applying some process to that format, does not
    relax or nullify any sections of this license as they pertain to
    the results of applying that process.
     
  \item \null
    \begin{enumerate}
    \item A Derived Work may be distributed under a different license
      provided that license itself honors the conditions listed in
      Clause~\ref{LPPL:item:conditions} above, in regard to the Work, though it
      does not have to honor the rest of the conditions in this
      license.
      
    \item If a Derived Work is distributed under a different license,
      that Derived Work must provide sufficient documentation as part
      of itself to allow each recipient of that Derived Work to honor
      the restrictions in Clause~\ref{LPPL:item:conditions} above, concerning
      changes from the Work.
    \end{enumerate}
  \item This license places no restrictions on works that are
    unrelated to the Work, nor does this license place any
    restrictions on aggregating such works with the Work by any means.

  \item Nothing in this license is intended to, or may be used to,
    prevent complete compliance by all parties with all applicable
    laws.
  \end{enumerate}

  \LPPLsubsection{No Warranty}
  \label{LPPL:Warranty}

  There is no warranty for the Work.  Except when otherwise stated in
  writing, the Copyright Holder provides the Work `as is', without
  warranty of any kind, either expressed or implied, including, but
  not limited to, the implied warranties of merchantability and
  fitness for a particular purpose.  The entire risk as to the quality
  and performance of the Work is with you.  Should the Work prove
  defective, you assume the cost of all necessary servicing, repair,
  or correction.

  In no event unless required by applicable law or agreed to in
  writing will The Copyright Holder, or any author named in the
  components of the Work, or any other party who may distribute and/or
  modify the Work as permitted above, be liable to you for damages,
  including any general, special, incidental or consequential damages
  arising out of any use of the Work or out of inability to use the
  Work (including, but not limited to, loss of data, data being
  rendered inaccurate, or losses sustained by anyone as a result of
  any failure of the Work to operate with any other programs), even if
  the Copyright Holder or said author or said other party has been
  advised of the possibility of such damages.

  \LPPLsubsection{Maintenance of The Work}
  \label{LPPL:Maintenance}

  The Work has the status `author-maintained' if the Copyright Holder
  explicitly and prominently states near the primary copyright notice
  in the Work that the Work can only be maintained by the Copyright
  Holder or simply that it is `author-maintained'.

  The Work has the status `maintained' if there is a Current
  Maintainer who has indicated in the Work that they are willing to
  receive error reports for the Work (for example, by supplying a
  valid e-mail address). It is not required for the Current Maintainer
  to acknowledge or act upon these error reports.

  The Work changes from status `maintained' to `unmaintained' if there
  is no Current Maintainer, or the person stated to be Current
  Maintainer of the work cannot be reached through the indicated means
  of communication for a period of six months, and there are no other
  significant signs of active maintenance.

  You can become the Current Maintainer of the Work by agreement with
  any existing Current Maintainer to take over this role.

  If the Work is unmaintained, you can become the Current Maintainer
  of the Work through the following steps:
  \begin{enumerate}
  \item Make a reasonable attempt to trace the Current Maintainer (and
    the Copyright Holder, if the two differ) through the means of an
    Internet or similar search.
  \item If this search is successful, then enquire whether the Work is
    still maintained.
    \begin{enumerate}
    \item If it is being maintained, then ask the Current Maintainer
      to update their communication data within one month.
     
    \item\label{LPPL:item:intention} If the search is unsuccessful or
      no action to resume active maintenance is taken by the Current
      Maintainer, then announce within the pertinent community your
      intention to take over maintenance.  (If the Work is a \LaTeX{}
      work, this could be done, for example, by posting to
      \texttt{comp.text.tex}.)
    \end{enumerate}
  \item {}
    \begin{enumerate}
    \item If the Current Maintainer is reachable and agrees to pass
      maintenance of the Work to you, then this takes effect
      immediately upon announcement.
     
    \item\label{LPPL:item:announce} If the Current Maintainer is not
      reachable and the Copyright Holder agrees that maintenance of
      the Work be passed to you, then this takes effect immediately
      upon announcement.
    \end{enumerate}
  \item\label{LPPL:item:change} If you make an `intention
    announcement' as described in~\ref{LPPL:item:intention} above and
    after three months your intention is challenged neither by the
    Current Maintainer nor by the Copyright Holder nor by other
    people, then you may arrange for the Work to be changed so as to
    name you as the (new) Current Maintainer.
     
  \item If the previously unreachable Current Maintainer becomes
    reachable once more within three months of a change completed
    under the terms of~\ref{LPPL:item:announce}
    or~\ref{LPPL:item:change}, then that Current Maintainer must
    become or remain the Current Maintainer upon request provided they
    then update their communication data within one month.
  \end{enumerate}
  A change in the Current Maintainer does not, of itself, alter the
  fact that the Work is distributed under the \LPPL\ license.

  If you become the Current Maintainer of the Work, you should
  immediately provide, within the Work, a prominent and unambiguous
  statement of your status as Current Maintainer.  You should also
  announce your new status to the same pertinent community as
  in~\ref{LPPL:item:intention} above.

  \LPPLsubsection{Whether and How to Distribute Works under This License}
  \label{LPPL:Distribute}

  This section contains important instructions, examples, and
  recommendations for authors who are considering distributing their
  works under this license.  These authors are addressed as `you' in
  this section.

  \LPPLsubsubsection{Choosing This License or Another License}
  \label{LPPL:Choosing}

  If for any part of your work you want or need to use
  \emph{distribution} conditions that differ significantly from those
  in this license, then do not refer to this license anywhere in your
  work but, instead, distribute your work under a different license.
  You may use the text of this license as a model for your own
  license, but your license should not refer to the \LPPL\ or
  otherwise give the impression that your work is distributed under
  the \LPPL.

  The document \LPPLdocfile{modguide} in the base \LaTeX\ distribution
  explains the motivation behind the conditions of this license.  It
  explains, for example, why distributing \LaTeX\ under the
  \textsc{gnu} General Public License (\textsc{gpl}) was considered
  inappropriate.  Even if your work is unrelated to \LaTeX, the
  discussion in \LPPLdocfile{modguide} may still be relevant, and authors
  intending to distribute their works under any license are encouraged
  to read it.

  \LPPLsubsubsection{A Recommendation on Modification Without Distribution}
  \label{LPPL:WithoutDistribution}

  It is wise never to modify a component of the Work, even for your
  own personal use, without also meeting the above conditions for
  distributing the modified component.  While you might intend that
  such modifications will never be distributed, often this will happen
  by accident -- you may forget that you have modified that component;
  or it may not occur to you when allowing others to access the
  modified version that you are thus distributing it and violating the
  conditions of this license in ways that could have legal
  implications and, worse, cause problems for the community.  It is
  therefore usually in your best interest to keep your copy of the
  Work identical with the public one.  Many works provide ways to
  control the behavior of that work without altering any of its
  licensed components.

  \LPPLsubsubsection{How to Use This License}
  \label{LPPL:HowTo}

  To use this license, place in each of the components of your work
  both an explicit copyright notice including your name and the year
  the work was authored and/or last substantially modified.  Include
  also a statement that the distribution and/or modification of that
  component is constrained by the conditions in this license.

  Here is an example of such a notice and statement:
\begin{verbatim}
  %% pig.dtx
  %% Copyright 2005 M. Y. Name
  %
  % This work may be distributed and/or modified under the
  % conditions of the LaTeX Project Public License, either version 1.3
  % of this license or (at your option) any later version.
  % The latest version of this license is in
  %   http://www.latex-project.org/lppl.txt
  % and version 1.3 or later is part of all distributions of LaTeX
  % version 2005/12/01 or later.
  %
  % This work has the LPPL maintenance status `maintained'.
  % 
  % The Current Maintainer of this work is M. Y. Name.
  %
  % This work consists of the files pig.dtx and pig.ins
  % and the derived file pig.sty.
\end{verbatim}
  
  Given such a notice and statement in a file, the conditions given in
  this license document would apply, with the `Work' referring to the
  three files `\LPPLfile{pig.dtx}', `\LPPLfile{pig.ins}', and
  `\LPPLfile{pig.sty}' (the last being generated from
  `\LPPLfile{pig.dtx}' using `\LPPLfile{pig.ins}'), the `Base
  Interpreter' referring to any `\LaTeX-Format', and both `Copyright
  Holder' and `Current Maintainer' referring to the person `M. Y.
  Name'.

  If you do not want the Maintenance section of \LPPL\ to apply to
  your Work, change `maintained' above into `author-maintained'.
  However, we recommend that you use `maintained' as the Maintenance
  section was added in order to ensure that your Work remains useful
  to the community even when you can no longer maintain and support it
  yourself.

  \LPPLsubsubsection{Derived Works That Are Not Replacements}
  \label{LPPL:NotReplacements}

  Several clauses of the \LPPL\ specify means to provide reliability
  and stability for the user community. They therefore concern
  themselves with the case that a Derived Work is intended to be used
  as a (compatible or incompatible) replacement of the original
  Work. If this is not the case (e.g., if a few lines of code are
  reused for a completely different task), then clauses 6b and 6d
  shall not apply.

  \LPPLsubsubsection{Important Recommendations}
  \label{LPPL:Recommendations}

  \LPPLparagraph{Defining What Constitutes the Work}

  The \LPPL\ requires that distributions of the Work contain all the
  files of the Work.  It is therefore important that you provide a way
  for the licensee to determine which files constitute the Work.  This
  could, for example, be achieved by explicitly listing all the files
  of the Work near the copyright notice of each file or by using a
  line such as:
\begin{verbatim}
    % This work consists of all files listed in manifest.txt.
\end{verbatim}
  in that place.  In the absence of an unequivocal list it might be
  impossible for the licensee to determine what is considered by you
  to comprise the Work and, in such a case, the licensee would be
  entitled to make reasonable conjectures as to which files comprise
  the Work.

\end{LPPLicense}




\subsection{GNU Free Documentation License, Version 1.2, November 2002}
\label{label_fdl}


%  \textbf{Copyright  2000,2001,2002  Free Software Foundation, Inc.}\par
%  51 Franklin St, Fifth Floor, Boston, MA  02110-1301  USA
%  \begin{quotation}
%    Everyone is allowed to distribute verbatim copies of this
%    license document, but modification of it is not allowed.
%  \end{quotation}

\subsubsection{Preamble}

The purpose of this License is to make a manual, textbook, or other
functional and useful document ``free'' in the sense of freedom: to
assure everyone the effective freedom to copy and redistribute it,
with or without modifying it, either commercially or noncommercially.
Secondarily, this License preserves for the author and publisher a way
to get credit for their work, while not being considered responsible
for modifications made by others.

This License is a kind of ``copyleft'', which means that derivative
works of the document must themselves be free in the same sense.  It
complements the GNU General Public License, which is a copyleft
license designed for free software.

We have designed this License in order to use it for manuals for free
software, because free software needs free documentation: a free
program should come with manuals providing the same freedoms that the
software does.  But this License is not limited to software manuals;
it can be used for any textual work, regardless of subject matter or
whether it is published as a printed book.  We recommend this License
principally for works whose purpose is instruction or reference.

\subsubsection{Applicability and definitions}

This License applies to any manual or other work, in any medium, that
contains a notice placed by the copyright holder saying it can be
distributed under the terms of this License.  Such a notice grants a
world-wide, royalty-free license, unlimited in duration, to use that
work under the conditions stated herein.  The \textbf{``Document''}, below,
refers to any such manual or work.  Any member of the public is a
licensee, and is addressed as \textbf{``you''}.  You accept the license if you
copy, modify or distribute the work in a way requiring permission
under copyright law.

A \textbf{``Modified Version''} of the Document means any work containing the
Document or a portion of it, either copied verbatim, or with
modifications and/or translated into another language.

A \textbf{``Secondary Section''} is a named appendix or a front-matter section of
the Document that deals exclusively with the relationship of the
publishers or authors of the Document to the Document's overall subject
(or to related matters) and contains nothing that could fall directly
within that overall subject.  (Thus, if the Document is in part a
textbook of mathematics, a Secondary Section may not explain any
mathematics.)  The relationship could be a matter of historical
connection with the subject or with related matters, or of legal,
commercial, philosophical, ethical or political position regarding
them.

The \textbf{``Invariant Sections''} are certain Secondary Sections whose titles
are designated, as being those of Invariant Sections, in the notice
that says that the Document is released under this License.  If a
section does not fit the above definition of Secondary then it is not
allowed to be designated as Invariant.  The Document may contain zero
Invariant Sections.  If the Document does not identify any Invariant
Sections then there are none.

The \textbf{``Cover Texts''} are certain short passages of text that are listed,
as Front-Cover Texts or Back-Cover Texts, in the notice that says that
the Document is released under this License.  A Front-Cover Text may
be at most 5 words, and a Back-Cover Text may be at most 25 words.

A \textbf{``Transparent''} copy of the Document means a machine-readable copy,
represented in a format whose specification is available to the
general public, that is suitable for revising the document
straightforwardly with generic text editors or (for images composed of
pixels) generic paint programs or (for drawings) some widely available
drawing editor, and that is suitable for input to text formatters or
for automatic translation to a variety of formats suitable for input
to text formatters.  A copy made in an otherwise Transparent file
format whose markup, or absence of markup, has been arranged to thwart
or discourage subsequent modification by readers is not Transparent.
An image format is not Transparent if used for any substantial amount
of text.  A copy that is not ``Transparent'' is called \textbf{``Opaque''}.

Examples of suitable formats for Transparent copies include plain
ASCII without markup, Texinfo input format, LaTeX input format, SGML
or XML using a publicly available DTD, and standard-conforming simple
HTML, PostScript or PDF designed for human modification.  Examples of
transparent image formats include PNG, XCF and JPG.  Opaque formats
include proprietary formats that can be read and edited only by
proprietary word processors, SGML or XML for which the DTD and/or
processing tools are not generally available, and the
machine-generated HTML, PostScript or PDF produced by some word
processors for output purposes only.

The \textbf{``Title Page''} means, for a printed book, the title page itself,
plus such following pages as are needed to hold, legibly, the material
this License requires to appear in the title page.  For works in
formats which do not have any title page as such, ``Title Page'' means
the text near the most prominent appearance of the work's title,
preceding the beginning of the body of the text.

A section \textbf{``Entitled XYZ''} means a named subunit of the Document whose
title either is precisely XYZ or contains XYZ in parentheses following
text that translates XYZ in another language.  (Here XYZ stands for a
specific section name mentioned below, such as \textbf{``Acknowledgements''},
\textbf{``Dedications''}, \textbf{``Endorsements''}, or \textbf{``History''}.)  
To \textbf{``Preserve the Title''}
of such a section when you modify the Document means that it remains a
section ``Entitled XYZ'' according to this definition.

The Document may include Warranty Disclaimers next to the notice which
states that this License applies to the Document.  These Warranty
Disclaimers are considered to be included by reference in this
License, but only as regards disclaiming warranties: any other
implication that these Warranty Disclaimers may have is void and has
no effect on the meaning of this License.

\subsubsection{Verbatim Copying}

You may copy and distribute the Document in any medium, either
commercially or noncommercially, provided that this License, the
copyright notices, and the license notice saying this License applies
to the Document are reproduced in all copies, and that you add no other
conditions whatsoever to those of this License.  You may not use
technical measures to obstruct or control the reading or further
copying of the copies you make or distribute.  However, you may accept
compensation in exchange for copies.  If you distribute a large enough
number of copies you must also follow the conditions in section 3.

You may also lend copies, under the same conditions stated above, and
you may publicly display copies.

\subsubsection{Copying in Quantity}

If you publish printed copies (or copies in media that commonly have
printed covers) of the Document, numbering more than 100, and the
Document's license notice requires Cover Texts, you must enclose the
copies in covers that carry, clearly and legibly, all these Cover
Texts: Front-Cover Texts on the front cover, and Back-Cover Texts on
the back cover.  Both covers must also clearly and legibly identify
you as the publisher of these copies.  The front cover must present
the full title with all words of the title equally prominent and
visible.  You may add other material on the covers in addition.
Copying with changes limited to the covers, as long as they preserve
the title of the Document and satisfy these conditions, can be treated
as verbatim copying in other respects.

If the required texts for either cover are too voluminous to fit
legibly, you should put the first ones listed (as many as fit
reasonably) on the actual cover, and continue the rest onto adjacent
pages.

If you publish or distribute Opaque copies of the Document numbering
more than 100, you must either include a machine-readable Transparent
copy along with each Opaque copy, or state in or with each Opaque copy
a computer-network location from which the general network-using
public has access to download using public-standard network protocols
a complete Transparent copy of the Document, free of added material.
If you use the latter option, you must take reasonably prudent steps,
when you begin distribution of Opaque copies in quantity, to ensure
that this Transparent copy will remain thus accessible at the stated
location until at least one year after the last time you distribute an
Opaque copy (directly or through your agents or retailers) of that
edition to the public.

It is requested, but not required, that you contact the authors of the
Document well before redistributing any large number of copies, to give
them a chance to provide you with an updated version of the Document.

\subsubsection{Modifications}

You may copy and distribute a Modified Version of the Document under
the conditions of sections 2 and 3 above, provided that you release
the Modified Version under precisely this License, with the Modified
Version filling the role of the Document, thus licensing distribution
and modification of the Modified Version to whoever possesses a copy
of it.  In addition, you must do these things in the Modified Version:

\begin{itemize}
\item[A.] 
   Use in the Title Page (and on the covers, if any) a title distinct
   from that of the Document, and from those of previous versions
   (which should, if there were any, be listed in the History section
   of the Document).  You may use the same title as a previous version
   if the original publisher of that version gives permission.
   
\item[B.]
   List on the Title Page, as authors, one or more persons or entities
   responsible for authorship of the modifications in the Modified
   Version, together with at least five of the principal authors of the
   Document (all of its principal authors, if it has fewer than five),
   unless they release you from this requirement.
   
\item[C.]
   State on the Title page the name of the publisher of the
   Modified Version, as the publisher.
   
\item[D.]
   Preserve all the copyright notices of the Document.
   
\item[E.]
   Add an appropriate copyright notice for your modifications
   adjacent to the other copyright notices.
   
\item[F.]
   Include, immediately after the copyright notices, a license notice
   giving the public permission to use the Modified Version under the
   terms of this License, in the form shown in the Addendum below.
   
\item[G.]
   Preserve in that license notice the full lists of Invariant Sections
   and required Cover Texts given in the Document's license notice.
   
\item[H.]
   Include an unaltered copy of this License.
   
\item[I.]
   Preserve the section Entitled ``History'', Preserve its Title, and add
   to it an item stating at least the title, year, new authors, and
   publisher of the Modified Version as given on the Title Page.  If
   there is no section Entitled ``History'' in the Document, create one
   stating the title, year, authors, and publisher of the Document as
   given on its Title Page, then add an item describing the Modified
   Version as stated in the previous sentence.
   
\item[J.]
   Preserve the network location, if any, given in the Document for
   public access to a Transparent copy of the Document, and likewise
   the network locations given in the Document for previous versions
   it was based on.  These may be placed in the ``History'' section.
   You may omit a network location for a work that was published at
   least four years before the Document itself, or if the original
   publisher of the version it refers to gives permission.
   
\item[K.]
   For any section Entitled ``Acknowledgements'' or ``Dedications'',
   Preserve the Title of the section, and preserve in the section all
   the substance and tone of each of the contributor acknowledgements
   and/or dedications given therein.
   
\item[L.]
   Preserve all the Invariant Sections of the Document,
   unaltered in their text and in their titles.  Section numbers
   or the equivalent are not considered part of the section titles.
   
\item[M.]
   Delete any section Entitled ``Endorsements''.  Such a section
   may not be included in the Modified Version.
   
\item[N.]
   Do not retitle any existing section to be Entitled ``Endorsements''
   or to conflict in title with any Invariant Section.
   
\item[O.]
   Preserve any Warranty Disclaimers.
\end{itemize}

If the Modified Version includes new front-matter sections or
appendices that qualify as Secondary Sections and contain no material
copied from the Document, you may at your option designate some or all
of these sections as invariant.  To do this, add their titles to the
list of Invariant Sections in the Modified Version's license notice.
These titles must be distinct from any other section titles.

You may add a section Entitled ``Endorsements'', provided it contains
nothing but endorsements of your Modified Version by various
parties--for example, statements of peer review or that the text has
been approved by an organization as the authoritative definition of a
standard.

You may add a passage of up to five words as a Front-Cover Text, and a
passage of up to 25 words as a Back-Cover Text, to the end of the list
of Cover Texts in the Modified Version.  Only one passage of
Front-Cover Text and one of Back-Cover Text may be added by (or
through arrangements made by) any one entity.  If the Document already
includes a cover text for the same cover, previously added by you or
by arrangement made by the same entity you are acting on behalf of,
you may not add another; but you may replace the old one, on explicit
permission from the previous publisher that added the old one.

The author(s) and publisher(s) of the Document do not by this License
give permission to use their names for publicity for or to assert or
imply endorsement of any Modified Version.

\subsubsection{Combining Documents}

You may combine the Document with other documents released under this
License, under the terms defined in section 4 above for modified
versions, provided that you include in the combination all of the
Invariant Sections of all of the original documents, unmodified, and
list them all as Invariant Sections of your combined work in its
license notice, and that you preserve all their Warranty Disclaimers.

The combined work need only contain one copy of this License, and
multiple identical Invariant Sections may be replaced with a single
copy.  If there are multiple Invariant Sections with the same name but
different contents, make the title of each such section unique by
adding at the end of it, in parentheses, the name of the original
author or publisher of that section if known, or else a unique number.
Make the same adjustment to the section titles in the list of
Invariant Sections in the license notice of the combined work.

In the combination, you must combine any sections Entitled ``History''
in the various original documents, forming one section Entitled
``History''; likewise combine any sections Entitled ``Acknowledgements'',
and any sections Entitled ``Dedications''.  You must delete all sections
Entitled ``Endorsements''.


\subsubsection{Collection of Documents}

You may make a collection consisting of the Document and other documents
released under this License, and replace the individual copies of this
License in the various documents with a single copy that is included in
the collection, provided that you follow the rules of this License for
verbatim copying of each of the documents in all other respects.

You may extract a single document from such a collection, and distribute
it individually under this License, provided you insert a copy of this
License into the extracted document, and follow this License in all
other respects regarding verbatim copying of that document.


\subsubsection{Aggregating with independent Works}

A compilation of the Document or its derivatives with other separate
and independent documents or works, in or on a volume of a storage or
distribution medium, is called an ``aggregate'' if the copyright
resulting from the compilation is not used to limit the legal rights
of the compilation's users beyond what the individual works permit.
When the Document is included in an aggregate, this License does not
apply to the other works in the aggregate which are not themselves
derivative works of the Document.

If the Cover Text requirement of section 3 is applicable to these
copies of the Document, then if the Document is less than one half of
the entire aggregate, the Document's Cover Texts may be placed on
covers that bracket the Document within the aggregate, or the
electronic equivalent of covers if the Document is in electronic form.
Otherwise they must appear on printed covers that bracket the whole
aggregate.



\subsubsection{Translation}

Translation is considered a kind of modification, so you may
distribute translations of the Document under the terms of section 4.
Replacing Invariant Sections with translations requires special
permission from their copyright holders, but you may include
translations of some or all Invariant Sections in addition to the
original versions of these Invariant Sections.  You may include a
translation of this License, and all the license notices in the
Document, and any Warranty Disclaimers, provided that you also include
the original English version of this License and the original versions
of those notices and disclaimers.  In case of a disagreement between
the translation and the original version of this License or a notice
or disclaimer, the original version will prevail.

If a section in the Document is Entitled ``Acknowledgements'',
``Dedications'', or ``History'', the requirement (section 4) to Preserve
its Title (section 1) will typically require changing the actual
title.


\subsubsection{Termination}

You may not copy, modify, sublicense, or distribute the Document except
as expressly provided for under this License.  Any other attempt to
copy, modify, sublicense or distribute the Document is void, and will
automatically terminate your rights under this License.  However,
parties who have received copies, or rights, from you under this
License will not have their licenses terminated so long as such
parties remain in full compliance.


\subsubsection{Future Revisions of this License}

The Free Software Foundation may publish new, revised versions
of the GNU Free Documentation License from time to time.  Such new
versions will be similar in spirit to the present version, but may
differ in detail to address new problems or concerns.  See
http://www.gnu.org/copyleft/.

Each version of the License is given a distinguishing version number.
If the Document specifies that a particular numbered version of this
License ``or any later version'' applies to it, you have the option of
following the terms and conditions either of that specified version or
of any later version that has been published (not as a draft) by the
Free Software Foundation.  If the Document does not specify a version
number of this License, you may choose any version ever published (not
as a draft) by the Free Software Foundation.


\subsubsection{Addendum: How to use this License for your documents}

To use this License in a document you have written, include a copy of
the License in the document and put the following copyright and
license notices just after the title page:

\bigskip
\begin{quote}
    Copyright \copyright \textsc{year your name}.
    Permission is granted to copy, distribute and/or modify this document
    under the terms of the GNU Free Documentation License, Version 1.2
    or any later version published by the Free Software Foundation;
    with no Invariant Sections, no Front-Cover Texts, and no Back-Cover Texts.
    A copy of the license is included in the section entitled ``GNU
    Free Documentation License''.
\end{quote}
\bigskip
    
If you have Invariant Sections, Front-Cover Texts and Back-Cover Texts,
replace the ``with \dots\ Texts.'' line with this:

\bigskip
\begin{quote}
    with the Invariant Sections being \textsc{list their titles}, with the
    Front-Cover Texts being \textsc{list}, and with the Back-Cover
    Texts being \textsc{list}. 
\end{quote}
\bigskip
    
If you have Invariant Sections without Cover Texts, or some other
combination of the three, merge those two alternatives to suit the
situation.

If your document contains nontrivial examples of program code, we
recommend releasing these examples in parallel under your choice of
free software license, such as the GNU General Public License,
to permit their use in free software.



%%% Local Variables: 
%%% mode: latex
%%% TeX-master: "beameruserguide"
%%% End: 

% Copyright 2005 by Till Tantau <tantau@cs.tu-berlin.de>.
%
% This program can be redistributed and/or modified under the terms
% of the LaTeX Project Public License Distributed from CTAN
% archives in directory macros/latex/base/lppl.txt.


\section{Input and Output Formats}
\label{section-formats}


\TeX\ was designed to be a flexible system. This is true both for the
\emph{input} for \TeX\ as well as for the \emph{output}. The present
section explains which input formats there are and how they are
supported by \pgfname. It also explains which different output formats
can be produced.



\subsection{Supported Input Formats}

\TeX\ does not prescribe exactly how your input should be
formatted. While it is \emph{customary} that, say, an opening brace
starts a scope in \TeX, this is by no means necessary. Likewise, it is
\emph{customary} that environments start with |\begin|, but \TeX\
could not really care less about the exact command name.

Even though \TeX\ can be reconfigured, users can not. For this reason,
certain \emph{input formats} specify a set of commands and conventions
how input for \TeX\ should be formatted. There are currently three
``major'' formats: Donald Knuth's original |plain| \TeX\ format,
Leslie Lamport's popular \LaTeX\ format, and Hans Hangen's Con\TeX t
format.


\subsubsection{Using the  \LaTeX\ Format}

Using \pgfname\ and \tikzname\ with the \LaTeX\ format is easy: You
say |\usepackage{pgf}| or |\usepackage{tikz}|. Usually, that is all
you need to do, all configuration will be done automatically and
(hopefully) correctly.

The style files used for the \LaTeX\ format reside in the subdirectory
|latex/pgf/| of the \pgfname-system. Mainly, what these files do is to
include files in the directory |generic/pgf|. For example, here is the
content of the file |latex/pgf/frontends/tikz.sty|:

\begin{codeexample}[code only]
% Copyright 2005 by Till Tantau <tantau@users.sourceforge.net>.
%
% This program can be redistributed and/or modified under the terms
% of the GNU Public License, version 2.

\RequirePackage{pgf,calc,pgffor,pgflibraryplothandlers,xkeyval}

% Copyright 2006 by Till Tantau
%
% This file may be distributed and/or modified
%
% 1. under the LaTeX Project Public License and/or
% 2. under the GNU Public License.
%
% See the file doc/generic/pgf/licenses/LICENSE for more details.

\ProvidesPackageRCS[v\pgfversion] $Header: /cvsroot/pgf/pgf/generic/pgf/frontendlayer/tikz.code.tex,v 1.94 2007/07/31 08:21:18 tantau Exp $


% Always-present libraries:

\usepgflibrary{plothandlers}

% TikZ is a key family
\pgfkeys{/tikz/.is family}

\def\tikzset{\pgfqkeys{/tikz}}


\newdimen\tikz@lastx
\newdimen\tikz@lasty
\newdimen\tikz@lastxsaved
\newdimen\tikz@lastysaved

\newdimen\tikzleveldistance
\newdimen\tikzsiblingdistance

\newbox\tikz@figbox
\newbox\tikz@tempbox

\newcount\tikztreelevel
\newcount\tikznumberofchildren
\newcount\tikznumberofcurrentchild

\newcount\tikz@fig@count

\newif\iftikz@node@is@a@label
\newif\iftikz@snaked

\let\tikz@options=\pgfutil@empty
\def\tikz@addoption#1{\expandafter\def\expandafter\tikz@options\expandafter{\tikz@options#1}}
\def\tikz@addmode#1{\expandafter\def\expandafter\tikz@mode\expandafter{\tikz@mode#1}}
\def\tikz@addtransform#1{%
  \ifx\tikz@transform\relax%
    #1%
  \else%
    \expandafter\def\expandafter\tikz@transform\expandafter{\tikz@transform#1}%
  \fi%
}



% TikZ options:

% This command is supported for compatibility only:

\def\tikzoption#1{\pgfutil@ifnextchar[{\tikzoption@opt{#1}}{\tikzoption@noopt{#1}}}%}

\def\tikzoption@opt#1[#2]#3{\pgfkeysdef{/tikz/#1}{#3}\pgfkeyssetvalue{/tikz/#1/.@def}{#2}}
\def\tikzoption@noopt#1#2{\pgfkeysdef{/tikz/#1}{#2}\pgfkeyssetvalue{/tikz/#1/.@def}{\pgfkeysvaluerequired}}

% Baseline options
\tikzoption{baseline}[0pt]{\pgfutil@ifnextchar({\tikz@baseline@coordinate}{\tikz@baseline@simple}#1\@nil}%)
\def\tikz@baseline@simple#1\@nil{\pgfsetbaseline{#1}}
\def\tikz@baseline@coordinate#1\@nil{\pgfsetbaselinepointlater{\tikz@scan@one@point\@firstofone#1}}

% Draw options
\tikzoption{line width}{\tikz@semiaddlinewidth{#1}}%

\def\tikz@semiaddlinewidth#1{\tikz@addoption{\pgfsetlinewidth{#1}}\pgfmathsetlength\pgflinewidth{#1}}

\tikzoption{cap}{\tikz@addoption{\csname pgfset#1cap\endcsname}}
\tikzoption{join}{\tikz@addoption{\csname pgfset#1join\endcsname}}
\tikzoption{miter limit}{\tikz@addoption{\pgfsetmiterlimit{#1}}}

\tikzoption{dash pattern}{% syntax: on 2pt off 3pt on 4pt ...
  \def\tikz@temp{#1}%
  \ifx\tikz@temp\pgfutil@empty%
    \def\tikz@dashpattern{}%
    \tikz@addoption{\pgfsetdash{}{0pt}}%
  \else%
    \def\tikz@dashpattern{}%
    \expandafter\tikz@scandashon\pgfutil@gobble#1o\@nil%
    \edef\tikz@temp{{\tikz@dashpattern}{\noexpand\tikz@dashphase}}%
    \expandafter\tikz@addoption\expandafter{\expandafter\pgfsetdash\tikz@temp}%
  \fi}
\tikzoption{dash phase}{%
  \def\tikz@dashphase{#1}%
  \edef\tikz@temp{{\tikz@dashpattern}{\noexpand\tikz@dashphase}}%
  \expandafter\tikz@addoption\expandafter{\expandafter\pgfsetdash\tikz@temp}%
}%
\def\tikz@dashphase{0pt}

\def\tikz@scandashon n#1o{%
  \expandafter\def\expandafter\tikz@dashpattern\expandafter{\tikz@dashpattern{#1}}%
  \pgfutil@ifnextchar\@nil{\pgfutil@gobble}{\tikz@scandashoff}}
\def\tikz@scandashoff ff#1o{%
  \expandafter\def\expandafter\tikz@dashpattern\expandafter{\tikz@dashpattern{#1}}%
  \pgfutil@ifnextchar\@nil{\pgfutil@gobble}{\tikz@scandashon}}

\tikzoption{draw opacity}{\tikz@addoption{\pgfsetstrokeopacity{#1}}}

% Double draw options
\tikzoption{double}[]{%
  \def\tikz@temp{#1}%
  \ifx\tikz@temp\tikz@nonetext%
    \tikz@addmode{\tikz@mode@doublefalse}%
  \else%
    \ifx\tikz@temp\pgfutil@empty%
    \else%
      \def\tikz@double@color{#1}%
    \fi%
    \tikz@addmode{\tikz@mode@doubletrue}%
  \fi}
\tikzoption{double distance}{%
  \pgfmathsetlength{\pgf@x}{#1}%
  \edef\tikz@double@width@distance{\the\pgf@x}%
  \tikz@addmode{\tikz@mode@doubletrue}}

\def\tikz@double@width@distance{0.6pt}
\def\tikz@double@color{white}

% Fill options

\tikzoption{even odd rule}[]{\tikz@addoption{\pgfseteorule}}
\tikzoption{nonzero rule}[]{\tikz@addoption{\pgfsetnonzerorule}}

\tikzoption{fill opacity}{\tikz@addoption{\pgfsetfillopacity{#1}}}


% Joined fill/draw options

\tikzoption{opacity}{\tikz@addoption{\pgfsetstrokeopacity{#1}\pgfsetfillopacity{#1}}}


% Main color options
\tikzoption{color}{%
  \tikz@addoption{%
    \ifx\tikz@fillcolor\pgfutil@empty%
      \ifx\tikz@strokecolor\pgfutil@empty%
      \else%
        \pgfsys@color@reset@inorderfalse%
        \let\tikz@strokecolor\pgfutil@empty%
        \let\tikz@fillcolor\pgfutil@empty%
      \fi%
    \else%
      \pgfsys@color@reset@inorderfalse%
      \let\tikz@strokecolor\pgfutil@empty%
      \let\tikz@fillcolor\pgfutil@empty%
    \fi%
    \pgfutil@colorlet{tikz@color}{#1}%
    \pgfutil@colorlet{.}{tikz@color}%
    \pgfsetcolor{.}%
    \pgfsys@color@reset@inordertrue%
  }%
  \def\tikz@textcolor{#1}}



% Rounding options
\tikzoption{rounded corners}[4pt]{\pgfsetcornersarced{\pgfpoint{#1}{#1}}}
\tikzoption{sharp corners}[]{\pgfsetcornersarced{\pgfpointorigin}}



% Coordinate options
\tikzoption{x}{\tikz@handle@vec{\pgfsetxvec}{\tikz@handle@x}#1\relax}
\tikzoption{y}{\tikz@handle@vec{\pgfsetyvec}{\tikz@handle@y}#1\relax}
\tikzoption{z}{\tikz@handle@vec{\pgfsetzvec}{\tikz@handle@z}#1\relax}

\def\tikz@handle@vec#1#2{\pgfutil@ifnextchar({\tikz@handle@coordinate#1}{\tikz@handle@single#2}}
\def\tikz@handle@coordinate#1{\tikz@scan@one@point#1}
\def\tikz@handle@single#1#2\relax{#1{#2}}
\def\tikz@handle@x#1{\pgfsetxvec{\pgfpoint{#1}{0pt}}}
\def\tikz@handle@y#1{\pgfsetyvec{\pgfpoint{0pt}{#1}}}
\def\tikz@handle@z#1{\pgfsetzvec{\pgfpoint{#1}{#1}}}


% Transformation options
\tikzoption{scale}{\tikz@addtransform{\pgftransformscale{#1}}}
\tikzoption{xscale}{\tikz@addtransform{\pgftransformxscale{#1}}}
\tikzoption{xslant}{\tikz@addtransform{\pgftransformxslant{#1}}}
\tikzoption{yscale}{\tikz@addtransform{\pgftransformyscale{#1}}}
\tikzoption{yslant}{\tikz@addtransform{\pgftransformyslant{#1}}}
\tikzoption{rotate}{\tikz@addtransform{\pgftransformrotate{#1}}}
\tikzoption{rotate around}{\tikz@addtransform{\tikz@rotatearound{#1}}}
\def\tikz@rotatearound#1{%
  \edef\tikz@temp{#1}% get rid of active stuff
  \expandafter\tikz@rotateparseA\tikz@temp%
}%
\def\tikz@rotateparseA#1:{%
  \def\tikz@temp@rot{#1}%
  \tikz@scan@one@point\tikz@rotateparseB%
}
\def\tikz@rotateparseB#1{%
  \pgf@process{#1}%
  \pgf@xc=\pgf@x%
  \pgf@yc=\pgf@y%
  \pgftransformshift{\pgfqpoint{\pgf@xc}{\pgf@yc}}%
  \pgftransformrotate{\tikz@temp@rot}%
  \pgftransformshift{\pgfqpoint{-\pgf@xc}{-\pgf@yc}}%
}

\tikzoption{shift}{\tikz@addtransform{\tikz@scan@one@point\pgftransformshift#1\relax}}
\tikzoption{xshift}{\tikz@addtransform{\pgftransformxshift{#1}}}
\tikzoption{yshift}{\tikz@addtransform{\pgftransformyshift{#1}}}
\tikzoption{cm}{\tikz@addtransform{\tikz@parse@cm#1\relax}}
\tikzoption{reset cm}[]{\tikz@addtransform{\pgftransformreset}}
\tikzoption{shift only}[]{\tikz@addtransform{\pgftransformresetnontranslations}}

\def\tikz@parse@cm#1,#2,#3,#4,{%
  \def\tikz@p@cm{{#1}{#2}{#3}{#4}}%
  \tikz@scan@one@point\tikz@parse@cmA}
\def\tikz@parse@cmA#1{%
  \expandafter\pgftransformcm\tikz@p@cm{#1}%
}



% Grid options
\tikzoption{xstep}{\def\tikz@grid@x{#1}}
\tikzoption{ystep}{\def\tikz@grid@y{#1}}
\tikzoption{step}{\tikz@handle@vec{\tikz@step@point}{\tikz@step@single}#1\relax}
\def\tikz@step@single#1{\def\tikz@grid@x{#1}\def\tikz@grid@y{#1}}
\def\tikz@step@point#1{\pgf@process{#1}\edef\tikz@grid@x{\the\pgf@x}\edef\tikz@grid@y{\the\pgf@y}}

\def\tikz@grid@x{1cm}
\def\tikz@grid@y{1cm}


% Path usage options
\newif\iftikz@mode@double
\newif\iftikz@mode@fill
\newif\iftikz@mode@draw
\newif\iftikz@mode@clip
\newif\iftikz@mode@boundary
\newif\iftikz@mode@shade
\let\tikz@mode=\pgfutil@empty

\def\tikz@nonetext{none}

\tikzoption{path only}[]{\let\tikz@mode=\pgfutil@empty}
\tikzoption{shade}[]{\tikz@addmode{\tikz@mode@shadetrue}}
\tikzoption{fill}[]{%
  \def\tikz@temp{#1}%
  \ifx\tikz@temp\tikz@nonetext%
    \tikz@addmode{\tikz@mode@fillfalse}%
  \else%
    \ifx\tikz@temp\pgfutil@empty%
    \else%
      \tikz@addoption{\pgfsetfillcolor{#1}}%
      \def\tikz@fillcolor{#1}%
    \fi%
    \tikz@addmode{\tikz@mode@filltrue}%
  \fi%
}
\tikzoption{draw}[]{%
  \def\tikz@temp{#1}%
  \ifx\tikz@temp\tikz@nonetext%
    \tikz@addmode{\tikz@mode@drawfalse}%
  \else%
    \ifx\tikz@temp\pgfutil@empty%
    \else%
      \tikz@addoption{\pgfsetstrokecolor{#1}}%
      \def\tikz@strokecolor{#1}%
    \fi%
    \tikz@addmode{\tikz@mode@drawtrue}%
  \fi%
}
\tikzoption{clip}[]{\tikz@addmode{\tikz@mode@cliptrue}}
\tikzoption{use as bounding box}[]{\tikz@addmode{\tikz@mode@boundarytrue}}

\tikzoption{save path}{\tikz@addmode{\pgfsyssoftpath@getcurrentpath#1\global\let#1=#1}}

\let\tikz@fillcolor=\pgfutil@empty
\let\tikz@strokecolor=\pgfutil@empty


% Pattern options
\tikzoption{pattern color}{\def\tikz@pattern@color{#1}}
\tikzoption{pattern}[]{%
  \def\tikz@temp{#1}%
  \ifx\tikz@temp\tikz@nonetext%
    \tikz@addmode{\tikz@mode@fillfalse}%
  \else%
    \ifx\tikz@temp\pgfutil@empty%
    \else%
      \tikz@addoption{\pgfsetfillpattern{#1}{\tikz@pattern@color}}%
      \def\tikz@pattern{#1}%
    \fi%
    \tikz@addmode{\tikz@mode@filltrue}%
  \fi%
}
\def\tikz@pattern@color{black}
\def\tikz@pattern{dots}


% Shading options
\tikzoption{shading}{\def\tikz@shading{#1}\tikz@addmode{\tikz@mode@shadetrue}}
\tikzoption{shading angle}{\def\tikz@shade@angle{#1}\tikz@addmode{\tikz@mode@shadetrue}}
\tikzoption{top color}{%
  \pgfutil@colorlet{tikz@axis@top}{#1}%
  \pgfutil@colorlet{tikz@axis@middle}{tikz@axis@top!50!tikz@axis@bottom}%
  \def\tikz@shading{axis}\def\tikz@shade@angle{0}\tikz@addmode{\tikz@mode@shadetrue}}
\tikzoption{bottom color}{%
  \pgfutil@colorlet{tikz@axis@bottom}{#1}%
  \pgfutil@colorlet{tikz@axis@middle}{tikz@axis@top!50!tikz@axis@bottom}%
  \def\tikz@shading{axis}\def\tikz@shade@angle{0}\tikz@addmode{\tikz@mode@shadetrue}}
\tikzoption{middle color}{%
  \pgfutil@colorlet{tikz@axis@middle}{#1}%
  \def\tikz@shading{axis}\tikz@addmode{\tikz@mode@shadetrue}}
\tikzoption{left color}{%
  \pgfutil@colorlet{tikz@axis@top}{#1}%
  \pgfutil@colorlet{tikz@axis@middle}{tikz@axis@top!50!tikz@axis@bottom}%
  \def\tikz@shading{axis}\def\tikz@shade@angle{90}\tikz@addmode{\tikz@mode@shadetrue}}
\tikzoption{right color}{%
  \pgfutil@colorlet{tikz@axis@bottom}{#1}%
  \pgfutil@colorlet{tikz@axis@middle}{tikz@axis@top!50!tikz@axis@bottom}%
  \def\tikz@shading{axis}\def\tikz@shade@angle{90}\tikz@addmode{\tikz@mode@shadetrue}}
\tikzoption{ball color}{\pgfutil@colorlet{tikz@ball}{#1}\def\tikz@shading{ball}\tikz@addmode{\tikz@mode@shadetrue}}
\tikzoption{inner color}{\pgfutil@colorlet{tikz@radial@inner}{#1}\def\tikz@shading{radial}\tikz@addmode{\tikz@mode@shadetrue}}
\tikzoption{outer color}{\pgfutil@colorlet{tikz@radial@outer}{#1}\def\tikz@shading{radial}\tikz@addmode{\tikz@mode@shadetrue}}

\def\tikz@shading{axis}
\def\tikz@shade@angle{0}

\pgfdeclareverticalshading[tikz@axis@top,tikz@axis@middle,tikz@axis@bottom]{axis}{100bp}{%
  color(0bp)=(tikz@axis@bottom);
  color(25bp)=(tikz@axis@bottom);
  color(50bp)=(tikz@axis@middle);
  color(75bp)=(tikz@axis@top);
  color(100bp)=(tikz@axis@top)}

\pgfutil@colorlet{tikz@axis@top}{gray}
\pgfutil@colorlet{tikz@axis@middle}{gray!50!white}
\pgfutil@colorlet{tikz@axis@bottom}{white}

\pgfdeclareradialshading[tikz@ball]{ball}{\pgfqpoint{-10bp}{10bp}}{%
 color(0bp)=(tikz@ball!15!white);
 color(9bp)=(tikz@ball!75!white);
 color(18bp)=(tikz@ball!70!black);
 color(25bp)=(tikz@ball!50!black);
 color(50bp)=(black)}

\pgfutil@colorlet{tikz@ball}{blue}

\pgfdeclareradialshading[tikz@radial@inner,tikz@radial@outer]{radial}{\pgfpointorigin}{%
 color(0bp)=(tikz@radial@inner);
 color(25bp)=(tikz@radial@outer);
 color(50bp)=(tikz@radial@outer)}

\pgfutil@colorlet{tikz@radial@inner}{gray}
\pgfutil@colorlet{tikz@radial@outer}{white}


% Pin options
\tikzoption{pin}{\pgfutil@ifnextchar[{\tikz@parse@pin}{\tikz@parse@pin[]}#1\pgf@nil}
\tikzoption{pin distance}{\def\tikz@pin@distance{#1}}
\tikzoption{pin edge}{\def\tikz@pin@edge@style{#1}}

\tikzoption{tikz@pin@post}[]{%
  \tikz@compute@direction{\tikz@label@angle}{\tikz@pin@distance}%
  \global\let\tikz@pin@edge@style@smuggle=\tikz@pin@edge@style%
}
\tikzoption{tikz@pre@pin@edge}[]{\def\pgf@marshal{\tikzstyle{tikz@pin@options}=}
  \expandafter\pgf@marshal\expandafter[\tikz@pin@edge@style@smuggle]%
}

\def\tikz@pin@distance{3ex}
\def\tikz@pin@edge@style{}

\def\tikz@parse@pin[#1]#2:#3\pgf@nil{%
  \tikz@add@after@node@path{\bgroup
    \pgfextra{\let\tikz@save@last@node=\tikzlastnode}%
    node
    [every pin,tikz@label@angle=#2,#1,at=(\tikzlastnode.\tikz@label@angle),%
    after node path={(\tikz@save@last@node) edge[every pin edge,tikz@pre@pin@edge,tikz@pin@options] (\tikzlastnode)},
    tikz@pin@post]
    {#3} \egroup}
}


% Label and pin options

\tikzoption{label}{\pgfutil@ifnextchar[{\tikz@parse@label}{\tikz@parse@label[]}#1\pgf@nil}
\tikzoption{label distance}{\def\tikz@label@distance{#1}}

\tikzoption{tikz@label@angle}{\def\tikz@label@angle{#1}\csname tikz@label@angle@is@#1\endcsname}

\tikzoption{tikz@label@post}[]{\tikz@compute@direction{\tikz@label@angle}{\tikz@label@distance}}

\def\tikz@label@distance{0pt}

\def\tikz@parse@label[#1]#2:#3\pgf@nil{%
  \tikz@add@after@node@path{
    \bgroup
    \pgfextra{\let\tikz@save@last@fig@name=\tikz@last@fig@name}%
    node
    [every label,%
    tikz@label@angle=#2,%
    #1,%
    at=(\tikzlastnode.\tikz@label@angle),tikz@label@post]%
    {#3}%
    \pgfextra{\global\let\tikz@last@fig@name=\tikz@save@last@fig@name}%
    \egroup%
  }
}

\expandafter\def\csname tikz@label@angle@is@right\endcsname{\def\tikz@label@angle{0}}
\expandafter\def\csname tikz@label@angle@is@above right\endcsname{\def\tikz@label@angle{45}}
\expandafter\def\csname tikz@label@angle@is@above\endcsname{\def\tikz@label@angle{90}}
\expandafter\def\csname tikz@label@angle@is@above left\endcsname{\def\tikz@label@angle{135}}
\expandafter\def\csname tikz@label@angle@is@left\endcsname{\def\tikz@label@angle{180}}
\expandafter\def\csname tikz@label@angle@is@below left\endcsname{\def\tikz@label@angle{225}}
\expandafter\def\csname tikz@label@angle@is@below\endcsname{\def\tikz@label@angle{270}}
\expandafter\def\csname tikz@label@angle@is@below right\endcsname{\def\tikz@label@angle{315}}

\def\tikz@compute@direction#1#2{%
  \let\tikz@do@auto@anchor=\relax
  \c@pgf@counta=#1\relax%
  \ifnum\c@pgf@counta<0\relax
    \advance\c@pgf@counta by 360\relax%
  \fi%
  \ifnum\c@pgf@counta>359\relax
    \advance\c@pgf@counta by-360\relax%
  \fi%
  \ifnum\c@pgf@counta<4\relax%
    \def\tikz@anchor{west}%
  \else\ifnum\c@pgf@counta<87\relax%
    \def\tikz@anchor{south west}%
  \else\ifnum\c@pgf@counta<94\relax%
    \def\tikz@anchor{south}%
  \else\ifnum\c@pgf@counta<177\relax%
    \def\tikz@anchor{south east}%
  \else\ifnum\c@pgf@counta<184\relax%
    \def\tikz@anchor{east}%
  \else\ifnum\c@pgf@counta<267\relax%
    \def\tikz@anchor{north east}%
  \else\ifnum\c@pgf@counta<274\relax%
    \def\tikz@anchor{north}%
  \else\ifnum\c@pgf@counta<357\relax%
    \def\tikz@anchor{north west}%
  \else%
    \def\tikz@anchor{west}%
  \fi\fi\fi\fi\fi\fi\fi\fi%
  \tikz@addtransform{\pgftransformshift{\pgfpointpolar{#1}{#2}}}%  
}



% General shape options
\tikzoption{name}{\edef\tikz@fig@name{#1}}

\tikzoption{at}{\tikz@scan@one@point\tikz@set@at#1}
\def\tikz@set@at#1{\def\tikz@node@at{#1}}%

\tikzoption{shape}{\edef\tikz@shape{#1}}

\tikzoption{nodes}{\tikzstyle{every node}+=[#1]}


% These are /pgf options now:

%\tikzoption{inner sep}{\def\pgfshapeinnerxsep{#1}\def\pgfshapeinnerysep{#1}}
%\tikzoption{inner xsep}{\def\pgfshapeinnerxsep{#1}}
%\tikzoption{inner ysep}{\def\pgfshapeinnerysep{#1}}

%\tikzoption{outer sep}{\def\pgfshapeouterxsep{#1}\def\pgfshapeouterysep{#1}}
%\tikzoption{outer xsep}{\def\pgfshapeouterxsep{#1}}
%\tikzoption{outer ysep}{\def\pgfshapeouterysep{#1}}

%\tikzoption{minimum width}{\def\pgfshapeminwidth{#1}}
%\tikzoption{minimum height}{\def\pgfshapeminheight{#1}}
%\tikzoption{minimum size}{\def\pgfshapeminwidth{#1}\def\pgfshapeminheight{#1}}

\tikzoption{aspect}{\pgfsetshapeaspect{#1}}

\tikzoption{after node path}{\tikz@add@after@node@path{#1}}%
\def\tikz@add@after@node@path#1{\expandafter\def\expandafter\tikz@after@node\expandafter{\tikz@after@node#1}}

\def\tikzaddafternodepathoption#1{%
  #1%
  \expandafter\def\expandafter\tikz@afternodepathoptions\expandafter{\tikz@afternodepathoptions#1}}

\let\tikz@afternodepathoptions=\pgfutil@empty

\tikzoption{anchor}{\def\tikz@anchor{#1}\let\tikz@do@auto@anchor=\relax}

\tikzoption{left}[]{\def\tikz@anchor{east}\tikz@possibly@transform{x}{-}{#1}}
\tikzoption{right}[]{\def\tikz@anchor{west}\tikz@possibly@transform{x}{}{#1}}
\tikzoption{above}[]{\def\tikz@anchor{south}\tikz@possibly@transform{y}{}{#1}}
\tikzoption{below}[]{\def\tikz@anchor{north}\tikz@possibly@transform{y}{-}{#1}}
\tikzoption{above left}[]%
  {\def\tikz@anchor{south east}%
    \tikz@possibly@transform{x}{-}{#1}\tikz@possibly@transform{y}{}{#1}}
\tikzoption{above right}[]%
  {\def\tikz@anchor{south west}%
    \tikz@possibly@transform{x}{}{#1}\tikz@possibly@transform{y}{}{#1}}
\tikzoption{below left}[]%
  {\def\tikz@anchor{north east}%
    \tikz@possibly@transform{x}{-}{#1}\tikz@possibly@transform{y}{-}{#1}}
\tikzoption{below right}[]%
  {\def\tikz@anchor{north west}%
    \tikz@possibly@transform{x}{}{#1}\tikz@possibly@transform{y}{-}{#1}}

\tikzoption{node distance}{\def\tikz@node@distance{#1}}
\def\tikz@node@distance{1cm}

\tikzoption{above of}{\tikz@of{#1}{90}}%
\tikzoption{below of}{\tikz@of{#1}{-90}}%
\tikzoption{left of}{\tikz@of{#1}{180}}%
\tikzoption{right of}{\tikz@of{#1}{0}}%
\tikzoption{above left of}{\tikz@of{#1}{135}}%
\tikzoption{below left of}{\tikz@of{#1}{-135}}%
\tikzoption{above right of}{\tikz@of{#1}{45}}%
\tikzoption{below right of}{\tikz@of{#1}{-45}}%

\def\tikz@of#1#2{%
  \def\tikz@anchor{center}%
  \let\tikz@do@auto@anchor=\relax%
  \tikz@addtransform{\pgftransformshift{\pgfpointpolar{#2}{\tikz@node@distance}}}%
  \def\tikz@node@at{\pgfpointanchor{#1}{center}}}
  
\tikzoption{transform shape}[true]{%
  \csname tikz@fullytransformed#1\endcsname%
  \iftikz@fullytransformed%
    \pgfresetnontranslationattimefalse%
  \else%
    \pgfresetnontranslationattimetrue%
  \fi%
}

\newif\iftikz@fullytransformed
\pgfresetnontranslationattimetrue%

\def\tikz@anchor{center}%
\def\tikz@shape{rectangle}%

\def\tikz@possibly@transform#1#2#3{%
  \let\tikz@do@auto@anchor=\relax%
  \def\tikz@test{#3}%
  \ifx\tikz@test\pgfutil@empty%
  \else%
    \pgfmathsetlength{\pgf@x}{#3}%
    \pgf@x=#2\pgf@x\relax%
    \edef\tikz@marshal{\noexpand\tikz@addtransform{%
        \expandafter\noexpand\csname  pgftransform#1shift\endcsname{\the\pgf@x}}}% 
    \tikz@marshal%
  \fi%
}


% Inter-picture options
\tikzoption{remember picture}[true]{\csname pgfrememberpicturepositiononpage#1\endcsname}
\tikzoption{overlay}[]{\pgf@relevantforpicturesizefalse}



% Line/curve label placement options
\tikzoption{sloped}[true]{\csname pgfslopedattime#1\endcsname}
\tikzoption{allow upside down}[true]{\csname pgfallowupsidedownattime#1\endcsname}

\tikzoption{pos}{\edef\tikz@time{#1}}

\tikzoption{auto}[]{\csname tikz@install@auto@anchor@#1\endcsname}
\tikzoption{swap}[]{%
  \def\tikz@temp{left}%
  \ifx\tikz@auto@anchor@direction\tikz@temp%
    \def\tikz@auto@anchor@direction{right}%
  \else%
    \def\tikz@auto@anchor@direction{left}%
  \fi%
}

\def\tikz@time{.5}

\def\tikz@install@auto@anchor@{\let\tikz@do@auto@anchor=\tikz@auto@anchor@on}
\def\tikz@install@auto@anchor@false{\let\tikz@do@auto@anchor=\relax}
\def\tikz@install@auto@anchor@left{\let\tikz@do@auto@anchor=\tikz@auto@anchor@on\def\tikz@auto@anchor@direction{left}}
\def\tikz@install@auto@anchor@right{\let\tikz@do@auto@anchor=\tikz@auto@anchor@on\def\tikz@auto@anchor@direction{right}}

\let\tikz@do@auto@anchor=\relax%

\def\tikz@auto@anchor@on{\csname tikz@auto@anchor@\tikz@auto@anchor@direction\endcsname}

\def\tikz@auto@anchor@left{\tikz@auto@pre\tikz@auto@anchor\tikz@auto@post}
\def\tikz@auto@anchor@right{\tikz@auto@pre\tikz@auto@anchor@prime\tikz@auto@post}

\def\tikz@auto@anchor@direction{left}

% Text options
\tikzoption{text}{\def\tikz@textcolor{#1}}
\tikzoption{font}{\def\tikz@textfont{#1}}
\tikzoption{text opacity}{\def\tikz@textopacity{#1}}
\tikzoption{text width}{\def\tikz@text@width{#1}}
\tikzoption{text height}{\def\tikz@text@height{#1}}
\tikzoption{text depth}{\def\tikz@text@depth{#1}}
\tikzoption{text ragged}[]%
{\def\tikz@text@action{\raggedright\rightskip\z@ plus2em \spaceskip.3333em \xspaceskip.5em\relax}}
\tikzoption{text badly ragged}[]{\def\tikz@text@action{\raggedright\relax}}
\tikzoption{text ragged left}[]%
{\def\tikz@text@action{\raggedleft\leftskip\z@ plus2em \spaceskip.3333em \xspaceskip.5em\relax}}
\tikzoption{text badly ragged left}[]{\def\tikz@text@action{\raggedleft\relax}}
\tikzoption{text justified}[]{\def\tikz@text@action{\leftskip\z@\rightskip\z@\relax}}
\tikzoption{text centered}[]{\def\tikz@text@action{%
  \leftskip\z@ plus2em%
  \rightskip\z@ plus2em%
  \spaceskip.3333em \xspaceskip.5em%
  \parfillskip=0pt%
  \let\\=\@centercr% for latex
  \relax}}
\tikzoption{text badly centered}[]%
{\def\tikz@text@action{%
  \let\\=\@centercr% for latex
  \parfillskip=0pt%
  \rightskip\@flushglue%
  \leftskip\@flushglue\relax}}

\let\tikz@text@width=\pgfutil@empty
\let\tikz@text@height=\pgfutil@empty
\let\tikz@text@depth=\pgfutil@empty
\let\tikz@textcolor=\pgfutil@empty
\let\tikz@textfont=\pgfutil@empty
\let\tikz@textopacity=\pgfutil@empty

\def\tikz@text@action{\raggedright\rightskip\z@ plus2em \spaceskip.3333em \xspaceskip.5em\relax}





% Arrow options
\tikzoption{arrows}{\tikz@processarrows{#1}}

\tikzoption{>}{%
  \tikz@set@pointed{\csname pgf@arrows@invert#1\endcsname}{#1}%
  \expandafter\tikz@processarrows\expandafter{\tikz@current@arrows}%
}

\tikzoption{shorten <}{\pgfsetshortenstart{#1}}
\tikzoption{shorten >}{\pgfsetshortenend{#1}}

\def\tikz@set@pointed#1#2{%
  \pgfutil@ifundefined{pgf@arrow@code@tikze@>@#2}
  {%
    \pgfarrowsdeclarealias{tikzs@<@#2}{tikze@>@#2}{#1}{#2}%
    \pgfarrowsdeclarereversed{tikzs@>@#2}{tikze@<@#2}{#1}{#2}%
    \pgfarrowsdeclarecombine*{tikz@|<@#2}{tikz@>|@#2}{#1}{#2}{|}{|}%
    \pgfarrowsdeclaredouble[\pgflinewidth]{tikzs@<<@#2}{tikze@>>@#2}{#1}{#2}%
    \pgfarrowsdeclarereversed{tikzs@>>@#2}{tikze@<<@#2}{tikzs@<<@#2}{tikze@>>@#2}%
  }{}%
  \pgfutil@namedef{tikz@special@arrow@start<}{tikzs@<@#2}%
  \pgfutil@namedef{tikz@special@arrow@end>}{tikze@>@#2}%
  \pgfutil@namedef{tikz@special@arrow@start>}{tikzs@>@#2}%
  \pgfutil@namedef{tikz@special@arrow@end<}{tikze@<@#2}%
  \pgfutil@namedef{tikz@special@arrow@start|<}{tikz@|<@#2}%
  \pgfutil@namedef{tikz@special@arrow@end>|}{tikz@>|@#2}%
  \pgfutil@namedef{tikz@special@arrow@start<<}{tikzs@<<@#2}%
  \pgfutil@namedef{tikz@special@arrow@end>>}{tikze@>>@#2}%
  \pgfutil@namedef{tikz@special@arrow@start>>}{tikzs@<<@#2}%
  \pgfutil@namedef{tikz@special@arrow@end<<}{tikze@>>@#2}%
}

\def\tikz@processarrows#1{%
  \def\tikz@current@arrows{#1}%
  \def\tikz@temp{#1}%
  \ifx\tikz@temp\pgfutil@empty%
  \else%
    \tikz@@processarrows#1\@nil
  \fi%
}
\def\tikz@@processarrows#1-#2\@nil{%
  \expandafter\ifx\csname tikz@special@arrow@start#1\endcsname\relax%
    \pgfsetarrowsstart{#1}
  \else%
    \pgfsetarrowsstart{\csname tikz@special@arrow@start#1\endcsname}%
  \fi%
  \expandafter\ifx\csname tikz@special@arrow@end#2\endcsname\relax%
    \pgfsetarrowsend{#2}
  \else%
    \pgfsetarrowsend{\csname tikz@special@arrow@end#2\endcsname}%
  \fi%
}

\tikz@set@pointed{\pgf@arrows@invertto}{to}
\def\tikz@current@arrows{-}

% Parabola options
\tikzoption{bend}{\tikz@scan@one@point\tikz@set@parabola@bend#1\relax}%
\tikzoption{bend pos}{\def\tikz@parabola@bend@factor{#1}}
\tikzoption{parabola height}{%
  \def\tikz@parabola@bend@factor{.5}%
  \def\tikz@parabola@bend{\pgfpointadd{\pgfpoint{0pt}{#1}}{\tikz@last@position@saved}}}

\def\tikz@parabola@bend{\tikz@last@position@saved}
\def\tikz@parabola@bend@factor{0}

\def\tikz@set@parabola@bend#1{\def\tikz@parabola@bend{#1}}

% Axis options
\tikzoption{domain}{\def\tikz@plot@domain{#1}\expandafter\tikz@plot@samples@recalc\tikz@plot@domain\relax}
\tikzoption{range}{\def\tikz@plot@range{#1}}

% Plot options
\tikzoption{smooth}[]{\let\tikz@plot@handler=\pgfplothandlercurveto}
\tikzoption{smooth cycle}[]{\let\tikz@plot@handler=\pgfplothandlerclosedcurve}
\tikzoption{sharp plot}[]{\let\tikz@plot@handler\pgfplothandlerlineto}

\tikzoption{tension}{\pgfsetplottension{#1}}

\tikzoption{xcomb}[]{\let\tikz@plot@handler=\pgfplothandlerxcomb}
\tikzoption{ycomb}[]{\let\tikz@plot@handler=\pgfplothandlerycomb}
\tikzoption{polar comb}[]{\let\tikz@plot@handler=\pgfplothandlerpolarcomb}

\tikzoption{raw gnuplot}[true]{\csname tikz@plot@raw@gnuplot#1\endcsname}
\tikzoption{prefix}{\def\tikz@plot@prefix{#1}}
\tikzoption{id}{\def\tikz@plot@id{#1}}

\tikzoption{samples}{\def\tikz@plot@samples{#1}\expandafter\tikz@plot@samples@recalc\tikz@plot@domain\relax}
\tikzoption{samples at}{\def\tikz@plot@samplesat{#1}}
\tikzoption{parametric}[true]{\csname tikz@plot@parametric#1\endcsname}

\tikzoption{variable}{\def\tikz@plot@var{#1}}

\tikzoption{only marks}[]{\let\tikz@plot@handler\pgfplothandlerdiscard}

\tikzoption{mark}{\def\tikz@plot@mark{#1}}
\tikzoption{mark options}{\def\tikz@plot@mark@options{#1}}
\tikzoption{mark size}{\pgfsetplotmarksize{#1}}

\tikzoption{mark indices}{\def\tikz@mark@list{#1}}
\tikzoption{mark phase}{\pgfsetplotmarkphase{#1}}
\tikzoption{mark repeat}{\pgfsetplotmarkrepeat{#1}}

\let\tikz@mark@list=\pgfutil@empty

\let\tikz@plot@mark@options=\pgfutil@empty

\let\tikz@plot@handler=\pgfplothandlerlineto
\let\tikz@plot@mark=\pgfutil@empty

\def\tikz@plot@samples{25}
\def\tikz@plot@domain{-5:5}
\def\tikz@plot@var{\x}
\def\tikz@plot@samplesat{-5,-4.6,...,5}
\def\tikz@plot@samples@recalc#1:#2\relax{%
  \pgfmathparse{#1}%
  \let\tikz@temp@start=\pgfmathresult%
  \pgfmathparse{#2}%
  \let\tikz@temp@end=\pgfmathresult%
  \pgfmathparse{\tikz@temp@start+(\tikz@temp@end-\tikz@temp@start)/\tikz@plot@samples}%
  \edef\tikz@plot@samplesat{\tikz@temp@start,\pgfmathresult,...,\tikz@temp@end}%
}


\def\tikz@plot@prefix{\jobname.}
\def\tikz@plot@id{pgf-plot}

\newif\iftikz@plot@parametric
\newif\iftikz@plot@raw@gnuplot


% To options
\tikzoption{to path}{\def\tikz@to@path{#1}}

\def\tikz@to@path{-- (\tikztotarget) \tikztonodes}



% Tree options
\tikzoption{edge from parent path}{\def\tikz@edge@to@parent@path{#1}}

\tikzoption{parent anchor}{\def\tikzparentanchor{.#1}\ifx\tikzparentanchor\tikz@border@text\let\tikzparentanchor\pgfutil@empty\fi}
\tikzoption{child anchor}{\def\tikzchildanchor{.#1}\ifx\tikzchildanchor\tikz@border@text\let\tikzchildanchor\pgfutil@empty\fi}

\tikzoption{level distance}{\pgfmathsetlength\tikzleveldistance{#1}}
\tikzoption{sibling distance}{\pgfmathsetlength\tikzsiblingdistance{#1}}

\tikzoption{growth function}{\let\tikz@grow=#1}
\tikzoption{growth parent anchor}{\def\tikz@growth@anchor{#1}}
\tikzoption{grow}{\tikz@set@growth{#1}\edef\tikz@special@level{\the\tikztreelevel}}%
\tikzoption{grow'}{\tikz@set@growth{#1}\tikz@swap@growth\edef\tikz@special@level{\the\tikztreelevel}}%

\def\tikz@growth@anchor{center}

\def\tikz@special@level{-1}% never

\def\tikz@swap@growth{%
  % Swap left and right
  \let\tikz@temp=\tikz@angle@grow@right%
  \let\tikz@angle@grow@right=\tikz@angle@grow@left%
  \let\tikz@angle@grow@left=\tikz@temp%
}%

\def\tikz@set@growth#1{%
  \let\tikz@grow=\tikz@grow@direction%
  \expandafter\ifx\csname tikz@grow@direction@#1\endcsname\relax%
    \c@pgf@counta=#1\relax%
  \else%
    \c@pgf@counta=\csname tikz@grow@direction@#1\endcsname%
  \fi%
  \edef\tikz@angle@grow{\the\c@pgf@counta}%
  \advance\c@pgf@counta by-90\relax%
  \edef\tikz@angle@grow@left{\the\c@pgf@counta}%
  \advance\c@pgf@counta by180\relax%
  \edef\tikz@angle@grow@right{\the\c@pgf@counta}%
}

\def\tikz@border@text{.border}
\let\tikzparentanchor=\pgfutil@empty
\let\tikzchildanchor=\pgfutil@empty
\def\tikz@edge@to@parent@path{(\tikzparentnode\tikzparentanchor) -- (\tikzchildnode\tikzchildanchor)}

\tikzleveldistance=15mm
\tikzsiblingdistance=15mm

\def\tikz@grow@direction@down{-90}
\def\tikz@grow@direction@up{90}
\def\tikz@grow@direction@left{180}
\def\tikz@grow@direction@right{0}

\def\tikz@grow@direction@south{-90}
\def\tikz@grow@direction@north{90}
\def\tikz@grow@direction@west{180}
\def\tikz@grow@direction@east{0}

\expandafter\def\csname tikz@grow@direction@north east\endcsname{45}
\expandafter\def\csname tikz@grow@direction@north west\endcsname{135}
\expandafter\def\csname tikz@grow@direction@south east\endcsname{-45}
\expandafter\def\csname tikz@grow@direction@south west\endcsname{-135}

\def\tikz@grow@direction{%
  \pgftransformshift{\pgfpointpolar{\tikz@angle@grow}{\tikzleveldistance}}%
  \ifnum\tikztreelevel=\tikz@special@level%
  \else%
    \pgf@xc=.5\tikzsiblingdistance%
    \c@pgf@counta=\tikznumberofchildren%
    \advance\c@pgf@counta by1\relax%
    \pgfutil@tempdima=\c@pgf@counta\pgf@xc%
    \pgftransformshift{\pgfpointpolar{\tikz@angle@grow@left}{\pgfutil@tempdima}}%
    \pgftransformshift{\pgfpointpolar{\tikz@angle@grow@right}{\tikznumberofcurrentchild\tikzsiblingdistance}}%
  \fi%
}

\tikzset{grow=down}


% Snake options
\tikzoption{snake}[]{%
  \def\tikz@@snake{#1}%
  \ifx\tikz@@snake\pgfutil@empty%
    \tikz@snakedtrue%
  \else%
    \ifx\tikz@@snake\tikz@nonetext%
      \tikz@snakedfalse%
    \else%
      \tikz@snakedtrue%
      \let\tikz@snake=\tikz@@snake%
    \fi%
  \fi}

\tikzoption{segment amplitude}{\pgfmathsetlength{\pgfsnakesegmentamplitude}{#1}}
\tikzoption{segment length}{\pgfmathsetlength{\pgfsnakesegmentlength}{#1}}
\tikzoption{segment angle}{\pgfmathparse{#1}\let\pgfsnakesegmentangle=\pgfmathresult}
\tikzoption{segment aspect}{\pgfmathparse{#1}\let\pgfsnakesegmentaspect=\pgfmathresult}

\tikzoption{segment object length}{\pgfmathparse{#1}\edef\pgfsnakesegmentobjectlength{\pgfmathresult pt}}

\tikzoption{raise snake}{\def\pgf@snake@raise{\pgftransformyshift{#1}}}
\tikzoption{mirror snake}[true]{%
  \csname if#1\endcsname
    \def\pgf@snake@mirror{\pgftransformyscale{-1}}%
  \else%
    \let\pgf@snake@mirror=\pgfutil@empty%
  \fi
}

\tikzoption{gap before snake}{\def\tikz@presnake{{moveto}{#1}}}
\tikzoption{line before snake}{\def\tikz@presnake{{lineto}{#1}}}

\tikzoption{gap after snake}{\def\tikz@postsnake{{moveto}{#1}}\def\tikz@mainsnakelength{\pgfsnakeremainingdistance+-#1}}
\tikzoption{line after snake}{\def\tikz@postsnake{{lineto}{#1}}\def\tikz@mainsnakelength{\pgfsnakeremainingdistance+-#1}}

\tikzoption{gap around snake}{%
  \def\tikz@presnake{{moveto}{#1}}%
  \def\tikz@postsnake{{moveto}{#1}}%
  \def\tikz@mainsnakelength{\pgfsnakeremainingdistance+-#1}%
}
\tikzoption{line around snake}{%
  \def\tikz@presnake{{lineto}{#1}}%
  \def\tikz@postsnake{{lineto}{#1}}%
  \def\tikz@mainsnakelength{\pgfsnakeremainingdistance+-#1}%
}
\let\pgf@snake@mirror=\pgfutil@empty
\let\pgf@snake@raise=\pgfutil@empty

\pgfsetsnakesegmenttransformation{\pgf@snake@mirror\pgf@snake@raise}

\def\tikz@snake{zigzag}

\let\tikz@presnake=\pgfutil@empty
\let\tikz@postsnake=\pgfutil@empty
\def\tikz@mainsnakelength{\pgfsnakeremainingdistance}


% Matrix options
\tikzoption{matrix}[true]{\csname tikz@is@matrix#1\endcsname}

\tikzoption{matrix anchor}{\def\tikz@matrix@anchor{#1}}

\tikzoption{column sep}{\def\pgfmatrixcolumnsep{#1}}
\tikzoption{row sep}{\def\pgfmatrixrowsep{#1}}

\tikzoption{cells}{\tikzstyle{every cell}+=[#1]}

\tikzoption{ampersand replacement}{\def\tikz@ampersand@replacement{#1}}

\newif\iftikz@is@matrix
\let\tikz@matrix@anchor=\pgfutil@empty
\let\tikz@ampersand@replacement=\pgfutil@empty

% Execute option

\tikzoption{execute at begin picture}{\expandafter\def\expandafter\tikz@atbegin@picture\expandafter{\tikz@atbegin@picture#1}}
\tikzoption{execute at end picture}{\expandafter\def\expandafter\tikz@atend@picture\expandafter{\tikz@atend@picture#1}}
\tikzoption{execute at begin scope}{\expandafter\def\expandafter\tikz@atbegin@scope\expandafter{\tikz@atbegin@scope#1}}
\tikzoption{execute at end scope}{\expandafter\def\expandafter\tikz@atend@scope\expandafter{\tikz@atend@scope#1}}
\tikzoption{execute at begin to}{\expandafter\def\expandafter\tikz@atbegin@to\expandafter{\tikz@atbegin@to#1}}
\tikzoption{execute at end to}{\expandafter\def\expandafter\tikz@atend@to\expandafter{\tikz@atend@to#1}}
\tikzoption{execute at begin node}{\expandafter\def\expandafter\tikz@atbegin@node\expandafter{\tikz@atbegin@node#1}}
\tikzoption{execute at end node}{\expandafter\def\expandafter\tikz@atend@node\expandafter{\tikz@atend@node#1}}
\tikzoption{execute at begin cell}{\expandafter\def\expandafter\tikz@atbegin@cell\expandafter{\tikz@atbegin@cell#1}}
\tikzoption{execute at end cell}{\expandafter\def\expandafter\tikz@atend@cell\expandafter{\tikz@atend@cell#1}}
\tikzoption{execute at empty cell}{\expandafter\def\expandafter\tikz@at@emptycell\expandafter{\tikz@at@emptycell#1}}

\let\tikz@atbegin@picture=\pgfutil@empty
\let\tikz@atend@picture=\pgfutil@empty
\let\tikz@atbegin@scope=\pgfutil@empty
\let\tikz@atend@scope=\pgfutil@empty
\let\tikz@atbegin@to=\pgfutil@empty
\let\tikz@atend@to=\pgfutil@empty
\let\tikz@atbegin@node=\pgfutil@empty
\let\tikz@atend@node=\pgfutil@empty
\let\tikz@atbegin@cell=\pgfutil@empty
\let\tikz@atend@cell=\pgfutil@empty
\let\tikz@at@emptycell=\pgfutil@empty




% Styles
\tikzoption{set style}{\tikzstyle#1}

% Handled in a special way.
\def\tikzstyle{\pgfutil@ifnextchar\bgroup\tikz@style@parseA\tikz@style@parseB}
\def\tikz@style@parseB#1={\tikz@style@parseA{#1}=}
\def\tikz@style@parseA#1#2=#3[#4]{% check for an optional argument
  \pgfutil@in@[{#2}%]
  \ifpgfutil@in@%
    \tikz@style@parseC{#1}#2={#4}%
  \else%
    \tikz@style@parseD{#1}#2={#4}%
  \fi%
}%

\def\tikz@style@parseC#1[#2]#3=#4{%
  \pgfkeys{/tikz/#1/.default={#2}}%
  \pgfutil@in@+{#3}%
  \ifpgfutil@in@%
    \pgfkeys{/tikz/#1/.append style={#4}}%
  \else%
    \pgfkeys{/tikz/#1/.style={#4}}%
  \fi}
\def\tikz@style@parseD#1#2=#3{%
  \pgfutil@in@+{#2}%
  \ifpgfutil@in@%
    \pgfkeys{/tikz/#1/.append style={#3}}%
  \else%
    \pgfkeys{/tikz/#1/.style={#3}}%
  \fi}


%
%
% Predefined styles
%
%

\tikzstyle{help lines}=              [color=gray,line width=0.2pt]

\tikzstyle{every picture}=           []
\tikzstyle{every path}=              []
\tikzstyle{every scope}=             []
\tikzstyle{every plot}=              []
\tikzstyle{every node}=              []
\tikzstyle{every child}=             []
\tikzstyle{every child node}=        []
\tikzstyle{every to}=                []
\tikzstyle{every cell}=              []
\tikzstyle{every matrix}=            []
\tikzstyle{every edge}=              [draw]
\tikzstyle{every label}=             [draw=none,fill=none]
\tikzstyle{every pin}=               [draw=none,fill=none]
\tikzstyle{every pin edge}=          [help lines]

\tikzstyle{ultra thin}=              [line width=0.1pt]
\tikzstyle{very thin}=               [line width=0.2pt]
\tikzstyle{thin}=                    [line width=0.4pt]
\tikzstyle{semithick}=               [line width=0.6pt]
\tikzstyle{thick}=                   [line width=0.8pt]
\tikzstyle{very thick}=              [line width=1.2pt]
\tikzstyle{ultra thick}=             [line width=1.6pt]

\tikzstyle{solid}=                   [dash pattern=]
\tikzstyle{dotted}=                  [dash pattern=on \pgflinewidth off 2pt]
\tikzstyle{densely dotted}=          [dash pattern=on \pgflinewidth off 1pt]
\tikzstyle{loosely dotted}=          [dash pattern=on \pgflinewidth off 4pt]
\tikzstyle{dashed}=                  [dash pattern=on 3pt off 3pt]
\tikzstyle{densely dashed}=          [dash pattern=on 3pt off 2pt]
\tikzstyle{loosely dashed}=          [dash pattern=on 3pt off 6pt]

\tikzstyle{transparent}=             [opacity=0]
\tikzstyle{ultra nearly transparent}=[opacity=0.05]
\tikzstyle{very nearly transparent}= [opacity=0.1]
\tikzstyle{nearly transparent}=      [opacity=0.25]
\tikzstyle{semitransparent}=         [opacity=0.5]
\tikzstyle{nearly opaque}=           [opacity=0.75]
\tikzstyle{very nearly opaque}=      [opacity=0.9]
\tikzstyle{ultra nearly opaque}=     [opacity=0.95]
\tikzstyle{opaque}=                  [opacity=1]

\tikzstyle{at start}=                [pos=0]
\tikzstyle{very near start}=         [pos=0.125]
\tikzstyle{near start}=              [pos=0.25]
\tikzstyle{midway}=                  [pos=0.5]
\tikzstyle{near end}=                [pos=0.75]
\tikzstyle{very near end}=           [pos=0.875]
\tikzstyle{at end}=                  [pos=1]

\tikzstyle{bend at start}=           [bend pos=0,bend={+(0,0)}]
\tikzstyle{bend at end}=             [bend pos=1,bend={+(0,0)}]

\tikzstyle{edge from parent}=        [draw]

\tikzstyle{snake triangles 45}=      [snake=triangles,segment object length=2.41421356\pgfsnakesegmentamplitude]
\tikzstyle{snake triangles 60}=      [snake=triangles,segment object length=1.73205081\pgfsnakesegmentamplitude]
\tikzstyle{snake triangles 90}=      [snake=triangles,segment object length=\pgfsnakesegmentamplitude]


%
% Setting keys
%

\pgfkeys{/tikz/style/.style=#1}

\pgfkeys{/tikz/.unknown/.code=%
  % Is it a pgf key?
  \let\tikz@key\pgfkeyscurrentname% 
  \pgfkeys{/pgf/\tikz@key/.try=#1}%
  \ifpgfkeyssuccess%
  \else%
    \expandafter\pgfutil@in@\expandafter!\expandafter{\tikz@key}%
    \ifpgfutil@in@%
      % this is a color!
      \expandafter\tikz@addoption\expandafter{\expandafter\pgfutil@color\expandafter{\tikz@key}}%
      \edef\tikz@textcolor{\tikz@key}%
    \else%
      \pgfutil@doifcolorelse{\tikz@key}
      { %     
        \expandafter\tikz@addoption\expandafter{\expandafter\pgfutil@color\expandafter{\tikz@key}}%
        \edef\tikz@textcolor{\tikz@key}%
      }%
      {%
        % Ok, second chance: This might be an arrow specification:
        \expandafter\pgfutil@in@\expandafter-\expandafter{\tikz@key}
        \ifpgfutil@in@%
          % Ah, an arrow spec!
          \expandafter\tikz@processarrows\expandafter{\tikz@key}%
        \else%
          % Ok, third chance: A shape!
          \expandafter\ifx\csname pgf@sh@s@\tikz@key\endcsname\relax%
            \pgfkeys{/errors/unknown key={/tikz/\tikz@key}{#1}}%
          \else%
            \edef\tikz@shape{\tikz@key}%
          \fi%
        \fi%
      }%
    \fi%
  \fi%
}


%
% Main TikZ Environment
%

\def\tikzpicture{\pgfutil@ifnextchar[\tikz@picture{\tikz@picture[]}}%}
\def\tikz@picture[#1]{%
  \pgfpicture%
  \let\tikz@atbegin@picture=\pgfutil@empty%
  \let\tikz@atend@picture=\pgfutil@empty%
  \let\tikz@transform=\relax%
  \tikz@installcommands\scope[every picture,#1]%
  \tikz@atbegin@picture%
}
\def\endtikzpicture{%
    \tikz@atend@picture%
    \global\let\pgf@shift@baseline=\pgf@baseline%
    \global\let\pgf@remember@smuggle=\ifpgfrememberpicturepositiononpage%
    \endscope%
    \let\pgf@baseline=\pgf@shift@baseline%
    \let\ifpgfrememberpicturepositiononpage=\pgf@remember@smuggle%
  \endpgfpicture}

  

% Inlined picture
%
% #1 - some code to be put in a tikzpicture environment.
%
% If the command is not followed by braces, everything up to the next
% semicolon is used as argument.
%
% Example:
%
% The rectangle \tikz{\draw (0,0) rectangle (1em,1ex)} has width 1em and
% height 1ex.

\def\tikz{\pgfutil@ifnextchar[{\tikz@opt}{\tikz@opt[]}}
\def\tikz@opt[#1]{\tikzpicture[#1]\pgfutil@ifnextchar\bgroup{\tikz@}{\tikz@@}}
\def\tikz@#1{#1\endtikzpicture}
\def\tikz@@{%
  \let\tikz@next=\tikz@collectnormalsemicolon%
  \ifnum\the\catcode`\;=\active\relax%
    \let\tikz@next=\tikz@collectactivesemicolon%
  \fi%
  \tikz@next}
\def\tikz@collectnormalsemicolon#1;{#1;\endtikzpicture}
{
  \catcode`\;=\active
  \gdef\tikz@collectactivesemicolon#1;{#1;\endtikzpicture}
}



%
% Environment for scoping graphic state settings
%
\def\tikz@scope@env{\pgfutil@ifnextchar[\tikz@@scope@env{\tikz@@scope@env[]}}
\def\tikz@@scope@env[#1]{%
  \pgfscope%
  \begingroup%
  \let\tikz@atbegin@scope=\pgfutil@empty%
  \let\tikz@atend@scope=\pgfutil@empty%
  \let\tikz@options=\pgfutil@empty%
  \let\tikz@mode=\pgfutil@empty%
  \tikzset{every scope/.try,#1}%
  \tikz@options%
  \tikz@atbegin@scope%
}
\def\endtikz@scope@env{%
  \tikz@atend@scope%
  \endgroup%
  \endpgfscope%
}


%
% Install the abbreviated commands
%
\def\tikz@installcommands{%
  \ifnum\the\catcode`\;=\active\relax\expandafter\let\expandafter\tikz@origsemi\expandafter=\tikz@activesemicolon\fi%
  \ifnum\the\catcode`\:=\active\relax\expandafter\let\expandafter\tikz@origcolon\expandafter=\tikz@activecolon\fi%
  \ifnum\the\catcode`\|=\active\relax\expandafter\let\expandafter\tikz@origbar\expandafter=\tikz@activebar\fi%
  \let\tikz@origscope=\scope%
  \let\tikz@origendscope=\endscope%
  \let\tikz@origstartscope=\startscope%
  \let\tikz@origstopscope=\stopscope%
  \let\tikz@origpath=\path%
  \let\tikz@origagainpath=\againpath%
  \let\tikz@origdraw=\draw%
  \let\tikz@origpattern=\pattern%
  \let\tikz@origfill=\fill%
  \let\tikz@origfilldraw=\filldraw%
  \let\tikz@origshade=\shade%
  \let\tikz@origshadedraw=\shadedraw%
  \let\tikz@origclip=\clip%
  \let\tikz@origuseasboundingbox=\useasboundingbox%
  \let\tikz@orignode=\node%
  \let\tikz@origcoordinate=\coordinate%
  \let\tikz@origmatrix=\matrix%
  \let\tikz@origcalendar=\calendar%
  %
  \tikz@deactivatthings%
  %
  \let\scope=\tikz@scope@env%
  \let\endscope=\endtikz@scope@env%
  \let\startscope=\scope%
  \let\stopscope=\endscope%
  \let\path=\tikz@command@path%
  \let\againpath=\tikz@command@againpath%
  %
  \def\draw{\path[draw]}
  \def\pattern{\path[pattern]}
  \def\fill{\path[fill]}
  \def\filldraw{\path[fill,draw]}
  \def\shade{\path[shade]}
  \def\shadedraw{\path[shade,draw]}
  \def\clip{\path[clip]}
  \def\useasboundingbox{\path[use as bounding box]}
  \def\node{\tikz@path@overlay{node}}
  \def\coordinate{\tikz@path@overlay{coordinate}}
  \def\matrix{\tikz@path@overlay{node[matrix]}}
  \def\calendar{\tikz@lib@cal@calendar}%
}
\ifx\tikz@lib@cal@calendar\@undefined
\def\tikz@lib@cal@calendar{\PackageError{tikz}{You need to load the calendar library}{}}
\fi

\def\tikz@path@overlay#1{%
  \let\tikz@signal@path=\tikz@signal@path% for detection at begin of matrix cell
  \pgfutil@ifnextchar<{\tikz@path@overlayed{#1}}{\path #1}}
\def\tikz@path@overlayed#1<#2>{\path<#2> #1}

\def\tikz@uninstallcommands{%
  \ifnum\the\catcode`\;=\active\relax\expandafter\let\tikz@activesemicolon=\tikz@origsemi\fi%
  \ifnum\the\catcode`\:=\active\relax\expandafter\let\tikz@activecolon=\tikz@origcolon\fi%
  \ifnum\the\catcode`\|=\active\relax\expandafter\let\tikz@activebar=\tikz@origbar\fi%
  \let\scope=\tikz@origscope%
  \let\endscope=\tikz@origendscope%
  \let\startscope=\tikz@origstartscope%
  \let\stopscope=\tikz@origstopscope%
  \let\path=\tikz@origpath%
  \let\againpath=\tikz@origagainpath%
  \let\draw=\tikz@origdraw%
  \let\pattern=\tikz@origpattern%
  \let\fill=\tikz@origfill%
  \let\filldraw=\tikz@origfilldraw%
  \let\shade=\tikz@origshade%
  \let\shadedraw=\tikz@origshadedraw%
  \let\clip=\tikz@origclip%
  \let\useasboundingbox=\tikz@origuseasboundingbox%
  \let\node=\tikz@orignode%
  \let\coordinate=\tikz@origcoordinate%
  \let\matrix=\tikz@origmatrix%
  \let\calendar=\tikz@origcalendar%
}


{
  \catcode`\;=12
  \gdef\tikz@nonactivesemicolon{;}
  \catcode`\:=12
  \gdef\tikz@nonactivecolon{:}
  \catcode`\|=12
  \gdef\tikz@nonactivebar{|}
  \catcode`\;=\active
  \catcode`\:=\active
  \catcode`\|=\active
  \catcode`\"=\active
  \gdef\tikz@activesemicolon{;}%
  \gdef\tikz@activecolon{:}%
  \gdef\tikz@activebar{|}%
  \gdef\tikz@activequotes{"}%
  \gdef\tikz@deactivatthings{%
    \def;{\tikz@nonactivesemicolon}
    \def:{\tikz@nonactivecolon}
    \def|{\tikz@nonactivebar}
  }
}





% Constructs a path and draws/fills them according to the current
% settings.  

\def\tikz@command@path{%
  \let\tikz@signal@path=\tikz@signal@path% for detection at begin of matrix cell
  \pgfutil@ifnextchar[{\tikz@check@earg}%]
  {\pgfutil@ifnextchar<{\tikz@doopt}{\tikz@@command@path}}}
\def\tikz@signal@path{\tikz@signal@path}%
\def\tikz@check@earg[#1]{%
  \pgfutil@ifnextchar<{\tikz@swap@args[#1]}{\tikz@@command@path[#1]}}
\def\tikz@swap@args[#1]<#2>{\tikz@command@path<#2>[#1]}

\def\tikz@doopt{%
  \let\tikz@next=\tikz@eargnormalsemicolon%
  \ifnum\the\catcode`\;=\active\relax%
    \let\tikz@next=\tikz@eargactivesemicolon%
  \fi%
  \tikz@next}
\long\def\tikz@eargnormalsemicolon<#1>#2;{\only<#1>{\tikz@@command@path#2;}}
{
  \catcode`\;=\active
  \long\global\def\tikz@eargactivesemicolon<#1>#2;{\only<#1>{\tikz@@command@path#2;}}
}

\def\tikz@@command@path{%
  \edef\tikzscope@linewidth{\the\pgflinewidth}%
  \begingroup%
    \let\tikz@options=\pgfutil@empty%
    \let\tikz@mode=\pgfutil@empty%
    \let\tikz@moveto@waiting=\relax%
    \let\tikz@timer=\relax%
    \let\tikz@collected@onpath=\pgfutil@empty%
    \tikz@snakedfalse%
    \tikz@node@is@a@labelfalse%
    \tikz@expandcount=1000\relax%
    \tikz@lastx=0pt%
    \tikz@lasty=0pt%
    \tikz@lastxsaved=0pt%
    \tikz@lastysaved=0pt%
    \tikzset{every path/.try}%
    \tikz@scan@next@command%
}
\def\tikz@scan@next@command{%
  \ifx\tikz@collected@onpath\pgfutil@empty%
  \else%
    \tikz@invoke@collected@onpath%
  \fi%
  \afterassignment\tikz@handle\let\@let@token=%
}
\newcount\tikz@expandcount

% Central dispatcher for commands
\def\tikz@handle{%
  \let\@next=\tikz@expand%
    \ifx\@let@token(%)
      \let\@next=\tikz@movetoabs%
    \else%
      \ifx\@let@token+%
        \let\@next=\tikz@movetorel%
      \else%
        \ifx\@let@token-%
          \let\@next=\tikz@lineto%
        \else%
          \ifx\@let@token.%
            \let\@next=\tikz@dot%
          \else%
            \ifx\@let@token r%
              \let\@next=\tikz@rect%
            \else%
              \ifx\@let@token a%
                \let\@next=\tikz@arcA%
              \else%
                \ifx\@let@token[%]
                  \let\@next=\tikz@parse@options%
                \else%
                  \ifx\@let@token n%
                    \let\@next=\tikz@fig%
                  \else%
                    \ifx\@let@token\bgroup%
                      \let\@next=\tikz@beginscope%
                    \else%
                      \ifx\@let@token\egroup%
                        \let\@next=\tikz@endscope%
                      \else%
                        \ifx\@let@token;%
                          \let\@next=\tikz@finish%
                        \else%
                          \ifx\@let@token c%
                            \let\@next=\tikz@cchar%
                          \else%
                            \ifx\@let@token e%
                              \let\@next=\tikz@e@char%
                            \else%
                              \ifx\@let@token g%
                                \let\@next=\tikz@grid%
                              \else%
                                \ifx\@let@token s%
                                   \let\@next=\tikz@sine%
                                \else%
                                  \ifx\@let@token |%
                                     \let\@next=\tikz@vh@lineto%
                                  \else%
                                    \ifx\@let@token p%
                                      \let\@next=\tikz@pchar%
                                      \pgfsetmovetofirstplotpoint%
                                    \else%
                                      \ifx\@let@token t%
                                        \let\@next=\tikz@to%
                                      \else%
                                        \ifx\@let@token\pgfextra%
                                          \let\@next=\tikz@extra%
                                        \else%
                                          \ifx\@let@token\foreach%
                                            \let\@next=\tikz@foreach%
                                          \else%
                                            \ifx\@let@token\pgf@stop%
                                              \let\@next=\relax%
                                            \else%
                                              \ifx\@let@token\par%
                                                \let\@next=\tikz@scan@next@command%
                                              \fi%      
                                            \fi%      
                                          \fi%      
                                        \fi%      
                                      \fi%      
                                    \fi%  
                                  \fi%  
                                \fi%
                              \fi%
                            \fi%
                          \fi%
                        \fi%
                      \fi%  
                    \fi%  
                  \fi%  
                \fi%  
              \fi%
            \fi%
          \fi%
        \fi%
      \fi%
    \fi%
  \@next%
}

\def\tikz@pchar{\pgfutil@ifnextchar l{\tikz@plot}{\tikz@parabola}}
\def\tikz@cchar{%
  \pgfutil@ifnextchar i{\tikz@circle}%
  {\pgfutil@ifnextchar h{\tikz@children}{\tikz@cochar}}}%
\def\tikz@cochar o{%
  \pgfutil@ifnextchar o{\tikz@coordinate}{\tikz@cosine}}
\def\tikz@e@char{%
  \pgfutil@ifnextchar l{\tikz@ellipse}{\tikz@@e@char}}%
\def\tikz@@e@char dge{%
  \pgfutil@ifnextchar f{\tikz@edgetoparent}{\tikz@edge@plain}}%


\def\tikz@finish{%
  \tikz@mode@fillfalse%
  \tikz@mode@drawfalse%
  \tikz@mode@doublefalse%
  \tikz@mode@clipfalse%
  \tikz@mode@boundaryfalse%
  \edef\tikz@pathextend{%
    {\noexpand\pgfqpoint{\the\pgf@pathminx}{\the\pgf@pathminy}}%
    {\noexpand\pgfqpoint{\the\pgf@pathmaxx}{\the\pgf@pathmaxy}}%
  }%
  \tikz@mode% installs the mode settings
  % Rendering pipeline:  
  % 
  % Step 1: Setup options
  % 
  \ifx\tikz@options\pgfutil@empty%
  \else%
    \pgfsys@beginscope%
      \begingroup%
      \tikz@options%
  \fi%
  % 
  % Step 2: Do a fill if shade follows.
  %
  \iftikz@mode@fill%
    \iftikz@mode@shade%
      \pgfprocessround{\pgfsyssoftpath@currentpath}{\pgfsyssoftpath@currentpath}% change the current path
      \pgfsyssoftpath@invokecurrentpath%
      \pgfsys@fill%
      \tikz@mode@fillfalse% no more filling...
    \fi%
  \fi%
  % 
  % Step 3: Do a shade if necessary.
  %
  \iftikz@mode@shade%
    \pgfprocessround{\pgfsyssoftpath@currentpath}{\pgfsyssoftpath@currentpath}% change the current path
    \pgfshadepath{\tikz@shading}{\tikz@shade@angle}%
    \tikz@mode@shadefalse% no more shading...
  \fi%
  % 
  % Step 4: Double stroke, if necessary
  %
  \iftikz@mode@draw%
    \iftikz@mode@double%
      % Change line width
      \begingroup%
        \pgfsys@beginscope%
          \pgf@x=2\pgflinewidth%
          \advance\pgf@x by\tikz@double@width@distance%
          \pgflinewidth=\pgf@x%
          \pgfsetlinewidth{\the\pgflinewidth}%
    \fi%
  \fi%
  % 
  % Step 5: Do stroke/fill/clip as needed
  %
  \edef\tikz@temp{\noexpand\pgfusepath{%
    \iftikz@mode@fill fill,\fi%
    \iftikz@mode@draw draw,\fi%
    \iftikz@mode@clip clip,\fi%
    }}%
  \tikz@temp%
  \tikz@mode@fillfalse% no more filling
  % 
  % Step 6: Double stroke, if necessary
  %
  \iftikz@mode@draw%
    \iftikz@mode@double%
          \pgfsyssoftpath@setcurrentpath\pgf@last@used@path% reinstall
          \pgf@x=\tikz@double@width@distance%
          \pgfsetlinewidth{\the\pgf@x}%
          \pgfsetstrokecolor{\tikz@double@color}%
          \pgfsyssoftpath@flushcurrentpath%
          \pgfsys@stroke%
        \pgfsys@endscope%
        \pgf@add@arrows@as@needed
      \endgroup%
    \fi%
  \fi%
  \tikz@mode@drawfalse% no more stroking
  % 
  % Step 7: Add labels and nodes
  %
  \copy\tikz@figbox%
  \setbox\tikz@figbox=\box\voidb@x%
  %
  % Step 8: Close option brace
  %
  \ifx\tikz@options\pgfutil@empty%
  \else%
      \endgroup%
    \pgfsys@endscope%
    \iftikz@mode@clip%
      \PackageError{tikz}{Extra options not allowed for clipping path command.}{}%
    \fi%
  \fi%
  \iftikz@mode@clip%
    \aftergroup\pgf@relevantforpicturesizefalse%
  \fi%
  \iftikz@mode@boundary%
    \aftergroup\pgf@relevantforpicturesizefalse%
  \fi%
  \endgroup%
  \global\pgflinewidth=\tikzscope@linewidth%
}




\def\tikz@skip#1{\tikz@scan@next@command#1}
\def\tikz@expand{%
  \advance\tikz@expandcount by -1%
  \ifnum\tikz@expandcount<0\relax%
    \PackageError{tikz}{Giving up on this path. Did you forget a semicolon?}{}%
    \let\@next=\tikz@finish%
  \else%
    \let\@next=\tikz@@expand
  \fi%
  \@next}

\def\tikz@@expand{%
  \expandafter\tikz@scan@next@command\@let@token}



% Syntax for scopes: 
% {scoped path commands}

\def\tikz@beginscope{\begingroup\tikz@scan@next@command}
\def\tikz@endscope{%
  \global\setbox\tikz@tempbox=\copy\tikz@figbox%
  \endgroup%
  \setbox\tikz@figbox=\box\tikz@tempbox%
  \tikz@scan@next@command}


% Syntax for pgfextra: 
% \pgfextra {normal tex text}
% \pgfextra normal tex text \endpgfextra

\def\tikz@extra{\pgfutil@ifnextchar\bgroup\tikz@@extra\relax}
\long\def\tikz@@extra#1{#1\tikz@scan@next@command}
\let\endpgfextra=\tikz@scan@next@command

\def\pgfextra{pgfextra}


% Syntax for \foreach: 
% \foreach \var in {list} {path text}
%
% Example:
%
% \draw (0,0) \foreach \x in {1,2,3} {-- (\x,0) circle (1cm)} -- (5,5);

\def\tikz@foreach{%
  \def\pgffor@beginhook{\setbox\tikz@figbox=\box\tikz@tempbox\expandafter\tikz@scan@next@command\@firstofone}%
  \def\pgffor@endhook{\pgfextra{%
      \xdef\tikz@foreach@save@lastx{\the\tikz@lastx}%
      \xdef\tikz@foreach@save@lasty{\the\tikz@lasty}%
      \xdef\tikz@foreach@save@lastxsaved{\the\tikz@lastxsaved}%
      \xdef\tikz@foreach@save@lastysaved{\the\tikz@lastysaved}%
      \global\setbox\tikz@tempbox=\copy\tikz@figbox\pgfutil@gobble}}%
  \def\pgffor@afterhook{%
    \tikz@lastx=\tikz@foreach@save@lastx%
    \tikz@lasty=\tikz@foreach@save@lasty%
    \tikz@lastxsaved=\tikz@foreach@save@lastxsaved%
    \tikz@lastysaved=\tikz@foreach@save@lastysaved%
    \setbox\tikz@figbox=\box\tikz@tempbox\tikz@scan@next@command}%
  \global\setbox\tikz@tempbox=\copy\tikz@figbox%
  \foreach}

  
% Syntax for againpath: 
% \againpath \somepathname

\def\tikz@command@againpath#1{%
  \pgfextra{%
    \pgfsyssoftpath@getcurrentpath\tikz@temp%
    \expandafter\pgfutil@g@addto@macro\expandafter\tikz@temp\expandafter{#1}%
    \pgfsyssoftpath@setcurrentpath\tikz@temp%
  }
}




%
% When this if is set, a just-scanned point is a shape and its border
% position still needs to be determined, depending on subsequent
% commands. 
%

\newif\iftikz@shapeborder


% Syntax for moveto: 
% <point>
\def\tikz@movetoabs{\tikz@moveto(}
\def\tikz@movetorel{\tikz@moveto+}
\def\tikz@moveto{%
  \tikz@scan@one@point{\tikz@@moveto}}
\def\tikz@@moveto#1{%
  \tikz@make@last@position{#1}%
  \iftikz@shapeborder%
    % ok, the moveto will have to wait. flag that we have a moveto in
    % wainting:
    \edef\tikz@moveto@waiting{\tikz@shapeborder@name}%
  \else%
    \pgfpathmoveto{\tikz@last@position}%
    \let\tikz@moveto@waiting=\relax%
  \fi%
  \tikz@scan@next@command%
}

\let\tikz@moveto@waiting=\relax % normally, nothing is waiting...

\def\tikz@flush@moveto{%
  \ifx\tikz@moveto@waiting\relax%
  \else%
    \pgfpathmoveto{\tikz@last@position}%
  \fi%
  \let\tikz@moveto@waiting=\relax%
}


\def\tikz@flush@moveto@toward#1#2#3{%
  % #1 = a point towards which the last moveto should be corrected
  % #2 = a dimension to which the corrected x-coordinate should be stored
  % #3 = a dimension for the corrected y-coordinate
  \ifx\tikz@moveto@waiting\relax%
    % do nothing
  \else%
    \pgf@process{\pgfpointshapeborder{\tikz@moveto@waiting}{#1}}%
    #2=\pgf@x%
    #3=\pgf@y%
    \edef\tikz@timer@start{\noexpand\pgfqpoint{\the\pgf@x}{\the\pgf@y}}%
    \pgfpathmoveto{\pgfqpoint{\pgf@x}{\pgf@y}}%
  \fi%
  \let\tikz@moveto@waiting=\relax%
}


%
% Collecting labels on the path 
%

\def\tikz@collect@coordinate@onpath#1coordinate
\def\tikz@@collect@coordinate@opt#1[#2]{%
  \pgfutil@ifnextchar({\tikz@@collect@coordinate#1[#2]}
\def\tikz@@collect@coordinate#1[#2](#3){%
  \tikz@collect@label@onpath#1node[shape=coordinate,#2](#3){}}

\def\tikz@collect@label@onpath#1node{%
  \expandafter\def\expandafter\tikz@collected@onpath\expandafter{\tikz@collected@onpath node}%
  \tikz@collect@label@scan#1}

\def\tikz@collect@label@scan#1{%  
  \pgfutil@ifnextchar({\tikz@collect@paran#1}%
  {\pgfutil@ifnextchar[{\tikz@collect@options#1}%
    {\pgfutil@ifnextchar\bgroup{\tikz@collect@arg#1}%
      {#1}}}%
}%}}

\def\tikz@collect@paran#1(#2){%
  \expandafter\def\expandafter\tikz@collected@onpath\expandafter{\tikz@collected@onpath(#2)}%
  \tikz@collect@label@scan#1%
}
\def\tikz@collect@options#1[#2]{%
  \expandafter\def\expandafter\tikz@collected@onpath\expandafter{\tikz@collected@onpath[#2]}%
  \tikz@collect@label@scan#1%
}
\def\tikz@collect@arg#1#2{%
  \expandafter\def\expandafter\tikz@collected@onpath\expandafter{\tikz@collected@onpath{#2}}%
  #1%
}


\def\tikz@invoke@collected@onpath{%
  \tikz@node@is@a@labeltrue%
  \let\tikz@temp=\tikz@collected@onpath%
  \let\tikz@collected@onpath=\pgfutil@empty%
  \expandafter\tikz@scan@next@command\tikz@temp\pgf@stop%
  \tikz@node@is@a@labelfalse%
}




% Syntax for lineto: 
% -- <point>

\def\tikz@lineto{%
  \pgfutil@ifnextchar |%
  {\expandafter\tikz@hv@lineto\pgfutil@gobble}%
  {\expandafter\pgfutil@ifnextchar\tikz@activebar{\expandafter\tikz@hv@lineto\pgfutil@gobble}%
    {\expandafter\tikz@lineto@mid\pgfutil@gobble}}}
\def\tikz@lineto@mid{%
  \pgfutil@ifnextchar n{\tikz@collect@label@onpath\tikz@lineto@mid}%
  {%
    \pgfutil@ifnextchar c{\tikz@close}{%
      \pgfutil@ifnextchar p{\pgfsetlinetofirstplotpoint\expandafter\tikz@plot\pgfutil@gobble}%
        {\tikz@scan@one@point{\tikz@@lineto}}}}}
\def\tikz@@lineto#1{%
  % Record the starting point for later labels on the path:
  \edef\tikz@timer@start{\noexpand\pgfqpoint{\the\tikz@lastx}{\the\tikz@lasty}}
  \iftikz@shapeborder%
    % ok, target is a shape. recalculate end
    \pgf@process{\pgfpointshapeborder{\tikz@shapeborder@name}{\tikz@last@position}}%
    \tikz@make@last@position{\pgfqpoint{\pgf@x}{\pgf@y}}%
    \tikz@flush@moveto@toward{\tikz@last@position}\pgf@x\pgf@y%
    \tikz@path@lineto{\tikz@last@position}%
    \edef\tikz@timer@end{\noexpand\pgfqpoint{\the\tikz@lastx}{\the\tikz@lasty}}%
    \tikz@make@last@position{#1}%
    \edef\tikz@moveto@waiting{\tikz@shapeborder@name}%    
  \else%
    % target is a reasonable point...
    % Record the starting point for later labels on the path:
    \tikz@make@last@position{#1}%
    \tikz@flush@moveto@toward{\tikz@last@position}\pgf@x\pgf@y%
    \tikz@path@lineto{\tikz@last@position}%
    \edef\tikz@timer@end{\noexpand\pgfqpoint{\the\tikz@lastx}{\the\tikz@lasty}}%
  \fi%
  \let\tikz@timer=\tikz@timer@line%
  \tikz@scan@next@command%
}

% snake or lineto?
\def\tikz@path@lineto#1{%
  \iftikz@snaked%
    {
      \pgfsyssoftpathmovetorelevantfalse%
      \pgfpathsnakesto{\tikz@presnake,{\tikz@snake}{\tikz@mainsnakelength},\tikz@postsnake}{#1}%
    }
  \else%
    \pgfpathlineto{#1}%
  \fi%
}

% snake or lineto?
\def\tikz@path@close#1{%
  \iftikz@snaked%
    {%
      \pgftransformreset%
      \pgfpathsnakesto{\tikz@presnake,{\tikz@snake}{\tikz@mainsnakelength},\tikz@postsnake}{#1}%
    }%
    \pgfpathclose%
  \else%
    \pgfpathclose%
  \fi%
}


% Syntax for lineto horizontal/vertical: 
% -| <point>

\def\tikz@hv@lineto{%
  \pgfutil@ifnextchar n
  {\tikz@collect@label@onpath\tikz@hv@lineto}
  {\pgfutil@ifnextchar c{\tikz@collect@coordinate@onpath\tikz@hv@lineto}%
    {\tikz@scan@one@point{\tikz@@hv@lineto}}}}
\def\tikz@@hv@lineto#1{%
  \edef\tikz@timer@start{\noexpand\pgfqpoint{\the\tikz@lastx}{\the\tikz@lasty}}%
  \pgf@yc=\tikz@lasty%
  \tikz@make@last@position{#1}%
  \tikz@flush@moveto@toward{\pgfqpoint{\tikz@lastx}{\pgf@yc}}\pgf@x\pgf@yc%
  \iftikz@shapeborder%
    % ok, target is a shape. have to work now:
    {%
      \pgf@process{\pgfpointshapeborder{\tikz@shapeborder@name}{\pgfqpoint{\tikz@lastx}{\pgf@yc}}}%
      \tikz@make@last@position{\pgfqpoint{\pgf@x}{\pgf@y}}%
      \tikz@path@lineto{\pgfqpoint{\tikz@lastx}{\pgf@yc}}%
      \tikz@path@lineto{\tikz@last@position}%
      \xdef\tikz@timer@end@temp{\noexpand\pgfqpoint{\the\tikz@lastx}{\the\tikz@lasty}}% move out of group
    }%
    \let\tikz@timer@end=\tikz@timer@end@temp%
    \edef\tikz@moveto@waiting{\tikz@shapeborder@name}%    
  \else%
    \tikz@path@lineto{\pgfqpoint{\tikz@lastx}{\pgf@yc}}%
    \tikz@path@lineto{\tikz@last@position}%
    \edef\tikz@timer@end{\noexpand\pgfqpoint{\the\tikz@lastx}{\the\tikz@lasty}}% move out of group
  \fi%
  \let\tikz@timer=\tikz@timer@hvline%
  \tikz@scan@next@command%
}

% Syntax for lineto vertical/horizontal: 
% |- <point>

\def\tikz@vh@lineto-{\tikz@vh@lineto@next}
\def\tikz@vh@lineto@next{%
  \pgfutil@ifnextchar n
  {\tikz@collect@label@onpath\tikz@vh@lineto@next}
  {\pgfutil@ifnextchar c{\tikz@collect@coordinate@onpath\tikz@vh@lineto@next}%
    {\tikz@scan@one@point\tikz@@vh@lineto}}}
\def\tikz@@vh@lineto#1{%
  \edef\tikz@timer@start{\noexpand\pgfqpoint{\the\tikz@lastx}{\the\tikz@lasty}}%
  \pgf@xc=\tikz@lastx%
  \tikz@make@last@position{#1}%
  \tikz@flush@moveto@toward{\pgfqpoint{\pgf@xc}{\tikz@lasty}}\pgf@xc\pgf@y%
  \iftikz@shapeborder%
    % ok, target is a shape. have to work now:
    {%
      \pgf@process{\pgfpointshapeborder{\tikz@shapeborder@name}{\pgfqpoint{\pgf@xc}{\tikz@lasty}}}%
      \tikz@make@last@position{\pgfqpoint{\pgf@x}{\pgf@y}}%
      \tikz@path@lineto{\pgfqpoint{\pgf@xc}{\tikz@lasty}}%
      \tikz@path@lineto{\tikz@last@position}%
      \xdef\tikz@timer@end@temp{\noexpand\pgfqpoint{\the\tikz@lastx}{\the\tikz@lasty}}% move out of group
    }%
    \let\tikz@timer@end=\tikz@timer@end@temp%
    \edef\tikz@moveto@waiting{\tikz@shapeborder@name}%    
  \else%
    \tikz@path@lineto{\pgfqpoint{\pgf@xc}{\tikz@lasty}}%
    \tikz@path@lineto{\tikz@last@position}%
    \edef\tikz@timer@end{\noexpand\pgfqpoint{\the\tikz@lastx}{\the\tikz@lasty}}%
  \fi%
  \let\tikz@timer=\tikz@timer@vhline%
  \tikz@scan@next@command%
}

% Syntax for cycle: 
% -- cycle
\def\tikz@close c{%
  \pgfutil@ifnextchar o{\tikz@collect@coordinate@onpath\tikz@lineto@mid c}% oops, a coordinate
  {\tikz@@close c}}%
\def\tikz@@close cycle{%
  \tikz@flush@moveto%
  \tikz@path@close{\expandafter\pgfpoint\pgfsyssoftpath@lastmoveto}%
  \def\pgfstrokehook{}%
  \let\tikz@timer=\@undefined%
  \tikz@scan@next@command%
}


% Syntax for options: 
% [options]
\def\tikz@parse@options#1]{%
  \tikzset{#1}%
  \tikz@scan@next@command%
}

% Syntax for edges:
% edge [options] (coordinate)
% edge [options] node {node text} (coordinate)
\def\tikz@edge@plain{%
  \begingroup%
    \tikz@to@use@whom%
    \let\tikz@to@or@edge@function=\tikz@do@edge%
    \tikz@to@or@edge}

% Syntax for to paths:
% to [options] (coordinate)
% to [options] node {node text} (coordinate)
\def\tikz@to o{%
  \tikz@to@use@last@coordinate%
  \let\tikz@to@or@edge@function=\tikz@do@to%
  \tikz@to@or@edge}
  
\def\tikz@to@or@edge{\pgfutil@ifnextchar[\tikz@@to@or@edge{\tikz@@to@or@edge[]}}%}
\def\tikz@@to@or@edge[#1]{%
  \def\tikz@@to@local@options{[#1]}%
  \let\tikz@collected@onpath=\pgfutil@empty%
  \tikz@@to@collect%
}
\def\tikz@@to@collect{%
  \pgfutil@ifnextchar(\tikz@@to@or@edge@coordinate
  {\pgfutil@ifnextchar n{\tikz@collect@label@onpath\tikz@@to@collect}%
    {\pgfutil@ifnextchar c{\tikz@collect@coordinate@onpath\tikz@@to@collect}
      {\PackageError{tikz}{( expected}{}%}
        \tikz@@to@or@edge@coordinate()}}}%
}

\def\tikz@@to@or@edge@coordinate(#1){%
  \def\tikztotarget{#1}%
  \tikz@to@or@edge@function%
}

\def\tikz@do@edge{%
  \setbox\tikz@figbox=\hbox\bgroup%
    \unhbox\tikz@figbox%
    \hbox\bgroup
      \bgroup%
        \pgfinterruptpath%
          \pgfscope%
            \let\tikz@transform=\pgfutil@empty%
            \let\tikz@options=\pgfutil@empty%
            \let\tikz@tonodes=\tikz@collected@onpath%
            \def\tikztonodes{{\pgfextra{\tikz@node@is@a@labeltrue}\tikz@tonodes}}%
            \let\tikz@collected@onpath=\pgfutil@empty%
            \tikz@options%
            \tikz@transform%            
            % Typeset node:
            \tikz@atbegin@to%
            \path[style=every edge]\tikz@@to@local@options(\tikztostart)\tikz@to@path;%
            \tikz@atend@to%
          \endpgfscope%
        \endpgfinterruptpath%
      \egroup
    \egroup%
  \egroup%
    \global\setbox\tikz@tempbox=\copy\tikz@figbox%
  \endgroup%
  \setbox\tikz@figbox=\box\tikz@tempbox%  
  \tikz@scan@next@command%  
}

\def\tikz@do@to{%
  \let\tikz@tonodes=\tikz@collected@onpath%
  \def\tikztonodes{{\pgfextra{\tikz@node@is@a@labeltrue}\tikz@tonodes}}%
  \let\tikz@collected@onpath=\pgfutil@empty%
  \tikz@scan@next@command%
  \pgfextra{\tikz@atbegin@to}%
  [style=every to]\tikz@@to@local@options\tikz@to@path%
  \pgfextra{\tikz@atend@to}%
}


\def\tikz@to@use@last@coordinate{%
  \iftikz@shapeborder%
    \edef\tikztostart{\tikz@shapeborder@name}%
  \else%
    \edef\tikztostart{\the\tikz@lastx,\the\tikz@lasty}%
  \fi%
}
\def\tikz@to@use@last@fig@name{%
  \edef\tikztostart{\tikz@to@last@fig@name}%
}



% Syntax for edge from parent: 
% edge from parent [options]
\def\tikz@edgetoparent from parent{\pgfutil@ifnextchar[\tikz@@edgetoparent{\tikz@@edgetoparent[]}}%}
\def\tikz@@edgetoparent[#1]{%
  \let\tikz@edge@to@parent@needed=\pgfutil@empty%
  \tikz@node@is@a@labeltrue%
  \tikz@scan@next@command [style=edge from parent,#1] \tikz@edge@to@parent@path%
}


% Syntax for bezier curves
% .. controls(point) and (point) .. (target)
% .. controls(point) .. (target) 
% .. (target) % currently not supported

\def\tikz@dot.{\tikz@@dot}%
\def\tikz@@dot{%
  \pgfutil@ifnextchar n%
  {\tikz@collect@label@onpath\tikz@@dot}%
  {\pgfutil@ifnextchar c{\tikz@curveto@double}{\tikz@curveto@auto}}}

\def\tikz@curveto@double co{%
  \pgfutil@ifnextchar o{\tikz@collect@coordinate@onpath\tikz@@dot co}
  {\tikz@cureveto@@double}}
\def\tikz@cureveto@@double ntrols#1{%
  \tikz@scan@one@point\tikz@curveA#1%
}
\def\tikz@curveA#1{%
  \edef\tikz@timer@start{\noexpand\pgfqpoint{\the\tikz@lastx}{\the\tikz@lasty}}%
  {%
    \tikz@make@last@position{#1}%
    \xdef\tikz@curve@first{\noexpand\pgfqpoint{\the\tikz@lastx}{\the\tikz@lasty}}%
  }%
  \pgfutil@ifnextchar a
  {\tikz@curveBand}%
  {\let\tikz@curve@second\tikz@curve@first\tikz@curveCdots}%
}
\def\tikz@curveBand and{%
  \tikz@scan@one@point\tikz@curveB%
}
\def\tikz@curveB#1{%
  \def\tikz@curve@second{#1}%
  \tikz@curveCdots}
\def\tikz@curveCdots{%
  \afterassignment\tikz@curveCdot\let\@next=}
\def\tikz@curveCdot.{%
  \ifx\@next.%
  \else%
    \PackageError{tikz}{Dot expected}{}%
  \fi%
  \tikz@updatecurrenttrue%
  \tikz@curveCcheck%
}
\def\tikz@curveCcheck{%
  \pgfutil@ifnextchar n%
  {\tikz@collect@label@onpath\tikz@curveCcheck}
  {\pgfutil@ifnextchar c{\tikz@collect@coordinate@onpath\tikz@curveCcheck}
    {\tikz@scan@one@point\tikz@curveC}}%
}
\def\tikz@curveC#1{%
  \tikz@make@last@position{#1}%
  \edef\tikz@curve@third{\noexpand\pgfqpoint{\the\tikz@lastx}{\the\tikz@lasty}}%
  {%
    \tikz@lastxsaved=\tikz@lastx%
    \tikz@lastysaved=\tikz@lasty%
    \tikz@make@last@position{\tikz@curve@second}%
    \xdef\tikz@curve@second{\noexpand\pgfqpoint{\the\tikz@lastx}{\the\tikz@lasty}}%
  }%
  %
  % Start recalculating things in case start and end are shapes.
  %
  % First, the start:
  \ifx\tikz@moveto@waiting\relax%
  \else%
    \pgf@process{\pgfpointshapeborder{\tikz@moveto@waiting}{\tikz@curve@first}}%
    \edef\tikz@timer@start{\noexpand\pgfqpoint{\the\pgf@x}{\the\pgf@y}}%
    \pgfpathmoveto{\pgfqpoint{\pgf@x}{\pgf@y}}%
  \fi%
  \let\tikz@timer@cont@one=\tikz@curve@first%
  \let\tikz@timer@cont@two=\tikz@curve@second%    
  % Second, the end:
  \iftikz@shapeborder%
    % ok, target is a shape. recalculate third
    {%
      \pgf@process{\pgfpointshapeborder{\tikz@shapeborder@name}{\tikz@curve@second}}%
      \tikz@make@last@position{\pgfqpoint{\pgf@x}{\pgf@y}}%
      \edef\tikz@curve@third{\noexpand\pgfqpoint{\the\tikz@lastx}{\the\tikz@lasty}}%
      \pgfpathcurveto{\tikz@curve@first}{\tikz@curve@second}{\tikz@curve@third}%
      \global\let\tikz@timer@end@temp=\tikz@curve@third% move out of group
    }%
    \let\tikz@timer@end=\tikz@timer@end@temp%
    \edef\tikz@moveto@waiting{\tikz@shapeborder@name}%    
  \else%
    \pgfpathcurveto{\tikz@curve@first}{\tikz@curve@second}{\tikz@curve@third}%
    \let\tikz@timer@end=\tikz@curve@third
    \let\tikz@moveto@waiting=\relax%
  \fi%
  \let\tikz@timer=\tikz@timer@curve%  
  \tikz@scan@next@command%
}


% Syntax for rectangles: 
% rectangle <corner point> 
\def\tikz@rect ectangle{%
  \tikz@flush@moveto%
  \edef\tikz@timer@start{\noexpand\pgfqpoint{\the\tikz@lastx}{\the\tikz@lasty}}%
  \tikz@@rect}%
\def\tikz@@rect{%
  \pgfutil@ifnextchar n
  {\tikz@collect@label@onpath\tikz@@rect}
  {\pgfutil@ifnextchar c{\tikz@collect@coordinate@onpath\tikz@@rect}%
    {
      \pgf@xa=\tikz@lastx\relax%
      \pgf@ya=\tikz@lasty\relax%
      \tikz@scan@one@point\tikz@rectB}}}
\def\tikz@rectB#1{%
  \tikz@make@last@position{#1}%
  \edef\tikz@timer@end{\noexpand\pgfqpoint{\the\tikz@lastx}{\the\tikz@lasty}}%
  \let\tikz@timer=\tikz@timer@line%  
  \pgfpathmoveto{\pgfqpoint{\pgf@xa}{\pgf@ya}}%
  \tikz@path@lineto{\pgfqpoint{\pgf@xa}{\tikz@lasty}}%
  \tikz@path@lineto{\pgfqpoint{\tikz@lastx}{\tikz@lasty}}%
  \tikz@path@lineto{\pgfqpoint{\tikz@lastx}{\pgf@ya}}%
  \iftikz@snaked% 
    \tikz@path@lineto{\pgfqpoint{\pgf@xa}{\pgf@ya}}%
  \fi%
  \pgfpathclose%
  \pgfpathmoveto{\pgfqpoint{\tikz@lastx}{\tikz@lasty}}%
  \def\pgfstrokehook{}%
  \tikz@scan@next@command%
}



% Syntax for grids: 
% grid <corner point> 
\def\tikz@grid rid{%
  \tikz@flush@moveto%
  \pgf@xa=\tikz@lastx\relax%
  \pgf@ya=\tikz@lasty\relax%
  \pgfutil@ifnextchar[{\tikz@gridA}{\tikz@gridA[]}}%}
\def\tikz@gridA[#1]{%
  \def\tikz@grid@options{#1}%
  \tikz@scan@one@point\tikz@gridB}%
\def\tikz@gridB#1{%
  \tikz@make@last@position{#1}%
  {%
    \expandafter\tikzset\expandafter{\tikz@grid@options}
    \tikz@checkunit{\tikz@grid@x}%
    \iftikz@isdimension%
      \pgf@process{\pgfpoint{\tikz@grid@x}{0pt}}%
    \else%
      \pgf@process{\pgfpointxy{\tikz@grid@x}{0}}%
    \fi%
    \pgf@xb=\pgf@x%
    \pgf@yb=\pgf@y%
    \tikz@checkunit{\tikz@grid@y}%
    \iftikz@isdimension%
      \pgf@process{\pgfpoint{0pt}{\tikz@grid@y}}%
    \else%
      \pgf@process{\pgfpointxy{0}{\tikz@grid@y}}%
    \fi%
    \advance\pgf@xb by\pgf@x%
    \advance\pgf@yb by\pgf@y%
    \pgfpathgrid[stepx=\pgf@xb,stepy=\pgf@yb]%
      {\pgfqpoint{\pgf@xa}{\pgf@ya}}{\pgfqpoint{\tikz@lastx}{\tikz@lasty}}%
  }
  \tikz@scan@next@command%
}



% Syntax for plot: 
% plot [local options] ...    % starts with a moveto
% -- plot [local options] ... % starts with a lineto
\def\tikz@plot lot{%
  \tikz@flush@moveto%
  \pgfutil@ifnextchar[{\tikz@@plot}{\tikz@@plot[]}}%}
\def\tikz@@plot[#1]{%
  \begingroup%
    \let\tikz@options=\pgfutil@empty%
    \tikzset{every plot/.try}%
    \tikzset{#1}%
    \pgfutil@ifnextchar f{\tikz@plot@f}%
    {\pgfutil@ifnextchar c{\tikz@plot@scan@points}%
      {\pgfutil@ifnextchar ({\tikz@plot@expression}{%
      \PackageError{tikz}{Cannot parse this plotting data}{}%
       \endgroup}}}}
\def\tikz@plot@f f{\pgfutil@ifnextchar i{\tikz@plot@file}{\tikz@plot@function}}

\def\tikz@plot@file ile#1{\def\tikz@plot@data{\pgfplotxyfile{#1}}\tikz@@@plot}%
\def\tikz@plot@scan@points coordinates#1{%
  \pgfplothandlerrecord\tikz@plot@data%
  \pgfplotstreamstart%
  \pgfutil@ifnextchar\pgf@stop{\pgfplotstreamend\expandafter\tikz@@@plot\pgfutil@gobble}
  {\tikz@scan@one@point\tikz@plot@next@point}%
  #1\pgf@stop%
}
\def\tikz@plot@next@point#1{%
  \pgfplotstreampoint{#1}%
  \pgfutil@ifnextchar\pgf@stop{\pgfplotstreamend\expandafter\tikz@@@plot\pgfutil@gobble}%
  {\tikz@scan@one@point\tikz@plot@next@point}%
}  
\def\tikz@plot@function unction#1{%
  \def\tikz@plot@filename{\tikz@plot@prefix\tikz@plot@id}%  
  \iftikz@plot@raw@gnuplot%
    \def\tikz@plot@data{\pgfplotgnuplot[\tikz@plot@filename]{#1}}%
  \else%
    \iftikz@plot@parametric%   
      \def\tikz@plot@data{\pgfplotgnuplot[\tikz@plot@filename]{%
          set samples \tikz@plot@samples;
          set parametric;
          plot [t=\tikz@plot@domain] #1}}%
    \else%
      \def\tikz@plot@data{\pgfplotgnuplot[\tikz@plot@filename]{%
          set samples \tikz@plot@samples;
          plot [x=\tikz@plot@domain] #1}}%
    \fi%
  \fi%
  \tikz@@@plot%
}

\def\tikz@plot@no@resample{%
  \pgfutil@IfFileExists{\tikz@plot@filename.table}%
  {\def\tikz@plot@data{\pgfplotxyfile{\tikz@plot@filename.table}}}%
  {}%
}

\def\tikz@plot@expression(#1){%
  \edef\tikz@plot@data{\noexpand\pgfplotfunction{\expandafter\noexpand\tikz@plot@var}{\tikz@plot@samplesat}}%
  \expandafter\def\expandafter\tikz@plot@data\expandafter{\tikz@plot@data{\tikz@scan@one@point\pgfutil@firstofone(#1)}}%
  \tikz@@@plot%
}

\def\tikz@@@plot{%
    \def\pgfplotlastpoint{\pgfpointorigin}%
    \tikz@plot@handler%
    \tikz@plot@data%
    \global\let\tikz@@@temp=\pgfplotlastpoint%
    \ifx\tikz@plot@mark\pgfutil@empty%
    \else%
      % Marks are drawn after the path.
      \setbox\tikz@figbox=\hbox{%
        \unhbox\tikz@figbox%
        \hbox{{%
          \pgfinterruptpath%
            \pgfscope%
              \let\tikz@options=\pgfutil@empty%
              \let\tikz@transform=\pgfutil@empty%
              \expandafter\tikzset\expandafter{\tikz@plot@mark@options}%
              \tikz@options%
              \ifx\tikz@mark@list\pgfutil@empty%
                \pgfplothandlermark{\tikz@transform\pgfuseplotmark{\tikz@plot@mark}}%
              \else
                \pgfplothandlermarklisted{\tikz@transform\pgfuseplotmark{\tikz@plot@mark}}{\tikz@mark@list}%
              \fi
              \tikz@plot@data%
            \endpgfscope
          \endpgfinterruptpath%
        }}%
      }%
    \fi%
    \global\setbox\tikz@tempbox=\copy\tikz@figbox%
  \endgroup%
  \setbox\tikz@figbox=\box\tikz@tempbox%  
  \tikz@make@last@position{\tikz@@@temp}%  
  \tikz@scan@next@command%
}


\pgfdeclareplotmark{ball}
{%
  \def\tikz@shading{ball}%
  \shade (0,0) circle (\pgfplotmarksize);%
}




% Syntax for cosine curves:
% cos <end of quarter-period>
\def\tikz@cosine s{\tikz@scan@one@point\tikz@@cosine}
\def\tikz@@cosine#1{%
  \tikz@flush@moveto%
  \pgf@process{#1}%
  \pgf@xc=\pgf@x%
  \pgf@yc=\pgf@y%
  \advance\pgf@xc by-\tikz@lastx%
  \advance\pgf@yc by-\tikz@lasty%
  \advance\tikz@lastx by\pgf@xc%
  \advance\tikz@lasty by\pgf@yc%
  \tikz@lastxsaved=\tikz@lastx%
  \tikz@lastysaved=\tikz@lasty%
  \tikz@updatecurrenttrue%
  \pgfpathcosine{\pgfqpoint{\pgf@xc}{\pgf@yc}}%
  \tikz@scan@next@command%
}

% Syntax for sine curves:
% sin <end of quarter-period>
\def\tikz@sine in{\tikz@scan@one@point\tikz@@sine}
\def\tikz@@sine#1{%
  \tikz@flush@moveto%
  \pgf@process{#1}%
  \pgf@xc=\pgf@x%
  \pgf@yc=\pgf@y%
  \advance\pgf@xc by-\tikz@lastx%
  \advance\pgf@yc by-\tikz@lasty%
  \advance\tikz@lastx by\pgf@xc%
  \advance\tikz@lasty by\pgf@yc%
  \tikz@lastxsaved=\tikz@lastx%
  \tikz@lastysaved=\tikz@lasty%
  \tikz@updatecurrenttrue%
  \pgfpathsine{\pgfqpoint{\pgf@xc}{\pgf@yc}}%
  \tikz@scan@next@command%
}

% Syntax for parabolas: 
% parabola[options] bend <coordinate> <coordinate>
\def\tikz@parabola arabola

\def\tikz@parabola@options[#1]{%
  \def\tikz@parabola@option{#1}%
  \pgfutil@ifnextchar b{\tikz@parabola@scan@bend}{\tikz@scan@one@point\tikz@parabola@semifinal}}
\def\tikz@parabola@scan@bend bend{\tikz@scan@one@point\tikz@parabola@scan@bendB}
\def\tikz@parabola@scan@bendB#1{%
  \def\tikz@parabola@bend{#1}%
  \tikz@scan@one@point\tikz@parabola@semifinal%
}
\def\tikz@parabola@semifinal#1{%
  \tikz@flush@moveto%
  % Save original start:
  \pgf@xb=\tikz@lastx%
  \pgf@yb=\tikz@lasty%
  \tikz@make@last@position{#1}%
  \pgf@xc=\tikz@lastx%
  \pgf@yc=\tikz@lasty%
  \begingroup% now calculate bend:
    \expandafter\tikzset\expandafter{\tikz@parabola@option}%
    \tikz@lastxsaved=\tikz@parabola@bend@factor\tikz@lastx%
    \tikz@lastysaved=\tikz@parabola@bend@factor\tikz@lasty%
    \advance\tikz@lastxsaved by\pgf@xb%
    \advance\tikz@lastysaved by\pgf@yb%
    \advance\tikz@lastxsaved by-\tikz@parabola@bend@factor\pgf@xb%
    \advance\tikz@lastysaved by-\tikz@parabola@bend@factor\pgf@yb%
    \expandafter\tikz@make@last@position\expandafter{\tikz@parabola@bend}%
    % Calculate delta from bend
    \advance\pgf@xc by-\tikz@lastx%
    \advance\pgf@yc by-\tikz@lasty%
    % Ok, now calculate delta to bend
    \advance\tikz@lastx by-\pgf@xb%
    \advance\tikz@lasty by-\pgf@yb%
    \xdef\tikz@parabola@b{{\noexpand\pgfqpoint{\the\tikz@lastx}{\the\tikz@lasty}}{\noexpand\pgfqpoint{\the\pgf@xc}{\the\pgf@yc}}}%
  \endgroup%
  \expandafter\pgfpathparabola\tikz@parabola@b%
  \tikz@scan@next@command%
}


% Syntax for circles:
% circle (radius)
%
% Syntax for ellipses:
% ellipse (x-radius and y-radius)
%
% radii can be dimensionless, then they are in the xy-system
\def\tikz@circle ircle{\tikz@flush@moveto\tikz@@circle}
\def\tikz@ellipse llipse{\tikz@flush@moveto\tikz@@circle}
\def\tikz@@circle{%
  \pgfutil@ifnextchar(\tikz@@@circle{%)
    \advance\tikz@expandcount by -1%
    \ifnum\tikz@expandcount<0\relax%
      \let\@next=\tikz@@circle@scangiveup%
    \else%
      \let\@next=\tikz@@circle@scanexpand%
    \fi%
    \@next%
  }%
}
\def\tikz@@circle@scanexpand{\expandafter\tikz@@circle}
\def\tikz@@circle@scangiveup#1{\PackageError{tikz}{Cannot parse this radius}{}#1{\tikz@scan@next@command}}
\def\tikz@@@circle(#1){%
  \pgfutil@in@{ and }{#1}%
  \ifpgfutil@in@%
    \tikz@@ellipseB(#1)%
  \else%
    \tikz@@ellipseB(#1 and #1)%
  \fi%
  \tikz@scan@next@command%
}
\def\tikz@@ellipseB(#1 and #2){%
  \tikz@checkunit{#1}%
  \iftikz@isdimension%
    \pgfpathellipse{\tikz@last@position}{\pgfpoint{#1}{0pt}}{\pgfpoint{0pt}{#2}}%
  \else%
    \pgfpathellipse{\tikz@last@position}{\pgfpointxy{#1}{0}}{\pgfpointxy{0}{#2}}%
  \fi%
}

% Syntax 1 for arcs:
% arc (start angle:end angle:radius)
%
% Syntax 2 for arcs:
% arc (start angle:end angle:x-radius and y-radius)
%
% radius can be dimensionless, then the arc is in the xy-coordinate system
\def\tikz@arcA rc{%
  \tikz@flush@moveto%
  \pgfutil@ifnextchar({\tikz@@arcto}{\expandafter\tikz@arcA\expandafter r\expandafter c}}

\def\tikz@@arcto(#1){%
  \edef\tikz@temp{(#1)}%
   \expandafter\tikz@@@arcto@check@slashand\tikz@temp%
}

\def\tikz@@@arcto@check@slashand(#1:#2:#3){%
  \pgfutil@in@{ and }{#3}%
  \ifpgfutil@in@% 
    \tikz@parse@arc@and(#1:#2:#3)%
  \else%
    \tikz@parse@arc@and(#1:#2:#3 and #3)%
  \fi%
}

\def\tikz@parse@arc@and(#1:#2:#3 and #4){%
  \tikz@checkunit{#3}%
  \iftikz@isdimension%
    \tikz@@@arcfinal{\pgfpatharc{#1}{#2}{#3 and #4}}
    {\pgfpointpolar{#1}{#3 and #4}}
    {\pgfpointpolar{#2}{#3 and #4}}%
  \else%
    \tikz@@@arcfinal{\pgfpatharcaxes{#1}{#2}{\pgfpointxy{#3}{0}}{\pgfpointxy{0}{#4}}}
    {\pgfpointpolarxy{#1}{#3 and #4}}{\pgfpointpolarxy{#2}{#3 and #4}}%
  \fi%
}

\def\tikz@@@arcfinal#1#2#3{%
  #1%
  \pgf@process{#2}%
  \advance\tikz@lastx by-\pgf@x%
  \advance\tikz@lasty by-\pgf@y%
  \pgf@process{#3}%
  \advance\tikz@lastx by\pgf@x%
  \advance\tikz@lasty by\pgf@y%
  \tikz@lastxsaved=\tikz@lastx%
  \tikz@lastysaved=\tikz@lasty%
  \tikz@scan@next@command%
}


% Syntax for coordinates:
% coordinate[options] (coordinate name) at (point)
% where ``at (point)'' is optional
\def\tikz@coordinate ordinate{%
  \pgfutil@ifnextchar[{\tikz@@coordinate@opt}{\tikz@@coordinate@opt[]}}
\def\tikz@@coordinate@opt[#1]
\def\tikz@@coordinate[#1](#2){%
  \pgfutil@ifnextchar a{\tikz@@coordinate@at[#1](#2)}
  {\tikz@fig ode[shape=coordinate,#1](#2){}}}
\def\tikz@@coordinate@at[#1](#2)at#3(#4){%
  \tikz@fig ode[shape=coordinate,#1](#2)at(#4){}}
  


% Syntax for nodes:
% node[options] (node name) {label text}
%
% all of [options], (node name) and {label text} are optional. There
% can be multiple options before the label text as in
% node[draw] (a) [rotate=10] {text}
%
% A label text always ``ends'' the node.
\def\tikz@fig ode{%
  \edef\tikz@save@line@width{\the\pgflinewidth}%
  \begingroup%
  \let\tikz@fig@name=\pgfutil@empty%
    \begingroup%
      \tikz@is@matrixfalse%
      \let\nodepart=\tikz@nodepart%
      \let\tikz@options=\pgfutil@empty%
      \let\tikz@after@node=\pgfutil@empty%
      \let\tikz@afternodepathoptions=\pgfutil@empty%
      \let\tikz@transform=\pgfutil@empty%
      \let\tikz@mode=\pgfutil@empty%
      \def\tikz@node@at{\pgfqpoint{\the\tikz@lastx}{\the\tikz@lasty}}%
      \iftikz@node@is@a@label%
      \else%
        \let\tikz@time=\pgfutil@empty%
      \fi%
      \tikzset{every node/.try}%
      \tikz@@scan@fig}%
\def\tikz@@scan@fig{%
  \pgfutil@ifnextchar a{\tikz@fig@scan@at}
  {\pgfutil@ifnextchar({\tikz@fig@scan@name}
    {\pgfutil@ifnextchar[{\tikz@fig@scan@options}%
      {\pgfutil@ifnextchar\bgroup{\tikz@fig@main}%
      {\PackageError{tikz}{A node must have a (possibly empty) label text}{}%
       \tikz@fig@main{}}}}}}%}}
\def\tikz@fig@scan@at at{%
  \tikz@scan@one@point\tikz@@fig@scan@at}
\def\tikz@@fig@scan@at#1{%
  \def\tikz@node@at{#1}\tikz@@scan@fig}%
\def\tikz@fig@scan@name(#1){\edef\tikz@fig@name{#1}\tikz@@scan@fig}%
\def\tikz@fig@scan@options[#1]{\tikzset{#1}\def\test{#1}\tikz@@scan@fig}%
\def\tikz@fig@main{\afterassignment\tikz@@fig@main\let\next=}
\def\tikz@@fig@main{%
    \pgfutil@ifundefined{pgf@sh@s@\tikz@shape}%
    {\PackageError{tikz}%
      {Unknown shape ``\tikz@shape.'' Using ``rectangle'' instead}{}%
      \def\tikz@shape{rectangle}}%
    {}%
    \tikzset{every \tikz@shape\space node/.try}%
    \iftikz@is@matrix%
      \let\tikz@next=\tikz@do@matrix%
    \else%
      \let\tikz@next=\tikz@do@fig%
    \fi%
    \tikz@next%  
}
\def\tikz@do@fig{%  
    \setbox\pgfnodeparttextbox=\hbox%
      \bgroup%
        \tikzset{every text node part/.try}%
        \ifx\tikz@textopacity\pgfutil@empty%
        \else%
          \pgfsetfillopacity{\tikz@textopacity}%
          \pgfsetstrokeopacity{\tikz@textopacity}%
        \fi%
        \pgfinterruptpicture%
          \tikz@textfont%  
          \ifx\tikz@text@width\pgfutil@empty%
          \else%
            \begingroup%
              \pgfutil@minipage[t]{\tikz@text@width}%
                \tikz@text@action%
          \fi%
          \tikz@atbegin@node%
          \bgroup%
            \aftergroup\unskip%
            \ifx\tikz@textcolor\pgfutil@empty%
            \else%
              \pgfutil@colorlet{.}{\tikz@textcolor}%
            \fi%
            \pgfsetcolor{.}%
            \setbox\tikz@figbox=\box\voidb@x%
            \tikz@uninstallcommands%
            \aftergroup\tikz@fig@collectresetcolor%
            \ignorespaces%
}
\def\tikz@fig@collectresetcolor{%
  \pgfutil@ifnextchar\reset@color%
  {\reset@color\afterassignment\tikz@fig@collectresetcolor\let\tikz@temp=}%
  {\tikz@fig@boxdone}%
}
\def\tikz@fig@boxdone{%
            \tikz@atend@node%
          \ifx\tikz@text@width\pgfutil@empty%
          \else%
              \pgfutil@endminipage%
            \endgroup%
          \fi%
        \endpgfinterruptpicture%
      \egroup%
    \pgfutil@ifnextchar c{\tikz@fig@mustbenamed\tikz@fig@continue}%
    {\pgfutil@ifnextchar[{\tikz@fig@mustbenamed\tikz@fig@continue}%
      {\pgfutil@ifnextchar t{\tikz@fig@mustbenamed\tikz@fig@continue}
        {\pgfutil@ifnextchar e{\tikz@fig@mustbenamed\tikz@fig@continue}
          {\ifx\tikz@after@node\pgfutil@empty\expandafter\tikz@fig@continue\else\expandafter\tikz@fig@mustbenamed\expandafter\tikz@fig@continue\fi}}}}}%}

\def\tikz@do@matrix{%
    \tikzset{every matrix/.try}%
    \tikz@node@transformations%
    \tikz@fig@mustbenamed%
    \setbox\tikz@figbox=\hbox\bgroup%
      \setbox\pgfutil@tempboxa=\copy\tikz@figbox%
      \unhbox\pgfutil@tempboxa%
      \hbox\bgroup\bgroup%
          \pgfinterruptpath%
            \pgfscope%
              \tikz@options%
              \setbox\tikz@figbox=\box\voidb@x%
              \let\tikzmatrixname=\tikz@fig@name%
              \edef\tikz@m@anchor{\ifx\tikz@matrix@anchor\pgfutil@empty\tikz@anchor\else\tikz@matrix@anchor\fi}%
              \expandafter\pgfutil@in@\expandafter{\expandafter.\expandafter}\expandafter{\tikz@m@anchor}%
              \ifpgfutil@in@%
                \expandafter\tikz@matrix@split\tikz@m@anchor\relax%
              \else%
                \def\tikz@matrix@shift{\pgfpointorigin}%  
              \fi%
              \let\tikz@transform=\relax%
              \pgfmatrix%
              {\tikz@shape}%
              {\tikz@m@anchor}%
              {\tikz@fig@name}%
              {%
                \pgfutil@tempdima=\pgflinewidth%
                {\begingroup\tikz@finish}%
                \global\pgflinewidth=\pgfutil@tempdima%
              }%
              {\tikz@matrix@shift}%
              {%
                \tikz@matrix@make@active@ampersand%
                \def\pgfmatrixbegincode{%
                  \pgfsys@beginscope%
                  \tikz@common@matrix@code%
                  \tikz@atbegin@cell%
                }%
                \def\tikz@common@matrix@code{%
                  \let\tikz@options=\pgfutil@empty%
                  \let\tikz@mode=\pgfutil@empty%
                  \tikzset{every cell/.try={\the\pgfmatrixcurrentrow}{\the\pgfmatrixcurrentcolumn}}%
                  \tikzset{column \the\pgfmatrixcurrentcolumn/.try}%
                  \ifodd\pgfmatrixcurrentcolumn%
                    \tikzset{every odd column/.try}%
                  \else%
                    \tikzset{every even column/.try}%
                  \fi%
                  \tikzset{row \the\pgfmatrixcurrentrow/.try}%
                  \ifodd\pgfmatrixcurrentrow%
                    \tikzset{every odd row/.try}%
                  \else%
                    \tikzset{every even row/.try}%
                  \fi%
                  \tikzset{row \the\pgfmatrixcurrentrow\space column \the\pgfmatrixcurrentcolumn/.try}%
                  \tikz@options%
                }%
                \def\pgfmatrixendcode{%
                  \tikz@atend@cell%
                  \pgfsys@endscope%
                }%
                \def\pgfmatrixemptycode{%
                  \pgfsys@beginscope%
                  \tikz@common@matrix@code%
                  \tikz@at@emptycell%
                  \pgfsys@endscope%
                }%
                \aftergroup\tikz@do@matrix@cont}%
              \bgroup%
}
\def\tikz@do@matrix@cont{%            
            \endpgfscope
          \endpgfinterruptpath%
      \egroup\egroup%
    \egroup%
    %
    \tikz@node@finish%
}

{%
  \catcode`\&=13
  \gdef\tikz@matrix@make@active@ampersand{%
    \ifx\tikz@ampersand@replacement\pgfutil@empty%
      \catcode`\&=13%
      \let&=\pgfmatrixnextcell%
    \else%
      \expandafter\let\tikz@ampersand@replacement=\pgfmatrixnextcell%
    \fi%
  }%
}%


\def\tikz@matrix@split#1.#2\relax{%
  \def\tikz@m@anchor{text}%
  \def\tikz@matrix@shift{\pgfpointanchor{#1}{#2}}%
}
  
\def\tikz@fig@continue{%
    \ifx\tikz@text@width\pgfutil@empty%
    \else%
      \pgfmathsetlength{\pgf@x}{\tikz@text@width}%
      \wd\pgfnodeparttextbox=\pgf@x%
    \fi%
    \ifx\tikz@text@height\pgfutil@empty%
    \else%
      \pgfmathsetlength{\pgf@x}{\tikz@text@height}%
      \ht\pgfnodeparttextbox=\pgf@x%
    \fi%
    \ifx\tikz@text@depth\pgfutil@empty%
    \else%
      \pgfmathsetlength{\pgf@x}{\tikz@text@depth}%
      \dp\pgfnodeparttextbox=\pgf@x%
    \fi%
    %
    % Node transformation
    %
    \tikz@node@transformations
    %
    \setbox\tikz@figbox=\hbox{%
      \setbox\pgfutil@tempboxa=\copy\tikz@figbox%
      \unhbox\pgfutil@tempboxa%
      \hbox{{%
          \pgfinterruptpath%
            \pgfscope%
              \tikz@options%
              \setbox\tikz@figbox=\box\voidb@x%
              \pgfmultipartnode{\tikz@shape}{\tikz@anchor}{\tikz@fig@name}{%
                \pgfutil@tempdima=\pgflinewidth%
                {\begingroup\tikz@finish}%
                \global\pgflinewidth=\pgfutil@tempdima%
              }%
            \endpgfscope
          \endpgfinterruptpath%
      }}%
    }%
    %
    \tikz@node@finish%
}


\def\tikz@fig@mustbenamed{%
  \ifx\tikz@fig@name\pgfutil@empty%
    % Assign a dummy name
    \global\advance\tikz@fig@count by1\relax
    \edef\tikz@fig@name{tikz@f@\the\tikz@fig@count}%
  \fi%
}

\def\tikz@node@transformations{
  % 
  % Possibly, we are ``online''
  % 
  \ifx\tikz@time\pgfutil@empty%
    \pgftransformshift{\tikz@node@at}%
    \iftikz@fullytransformed%
    \else%
      \pgftransformresetnontranslations%
    \fi%
  \else%
    \tikz@do@auto@anchor%
    \tikz@timer%
  \fi%
  % Invoke local transformations
  \tikz@transform%
}

\def\tikz@node@finish{%  
    \global\let\tikz@last@fig@name=\tikz@fig@name%
    \global\let\tikz@after@node@smuggle=\tikz@after@node%
    \global\let\tikz@afternodepathoptions@smuggle=\tikz@afternodepathoptions%
    % shift box outside group
    \global\setbox\tikz@tempbox=\copy\tikz@figbox%
  \endgroup\endgroup%
  \setbox\tikz@figbox=\box\tikz@tempbox%
  \pgflinewidth=\tikz@save@line@width%
  \let\tikz@to@last@fig@name=\tikz@last@fig@name%
  \let\tikz@to@use@whom=\tikz@to@use@last@fig@name%
  \let\tikzlastnode=\tikz@last@fig@name%
  \ifx\tikz@after@node@smuggle\pgfutil@empty%
  \else%
    \tikz@scan@next@command{\pgfextra{\tikz@afternodepathoptions@smuggle}\tikz@after@node@smuggle}\pgf@stop%
  \fi%
  \tikz@scan@next@command%
}
\let\tikz@fig@continue@orig=\tikz@fig@continue



% Syntax for parts of  nodes:
% node ... {... \nodepart{name} ... \nodepart{name} ...}

\def\tikz@nodepart#1{%
  \tikz@atend@node%
  \unskip%
  \gdef\tikz@nodepart@name{#1}%
  \global\let\tikz@fig@continue=\tikz@nodepart@continue%
  \pgfutil@ifnextchar x{\egroup\relax}{\egroup\relax}% gobble spaces
}
\def\tikz@nodepart@continue{%
  \global\let\tikz@fig@continue=\tikz@fig@continue@orig%
  % Now start new box:
   \expandafter\setbox\csname pgfnodepart\tikz@nodepart@name box\endcsname=\hbox%
      \bgroup%
        \tikzset{every \tikz@nodepart@name\space node part/.try}%
        \pgfinterruptpicture%
          \tikz@textfont%  
          \ifx\tikz@text@width\pgfutil@empty%
          \else%
            \begingroup%
              \pgfutil@minipage[t]{\tikz@text@width}%
                \tikz@text@action%
          \fi%
          \bgroup%
            \aftergroup\unskip%
            \ifx\tikz@textcolor\pgfutil@empty%
            \else%
              \pgfutil@colorlet{.}{\tikz@textcolor}%
            \fi%
            \pgfsetcolor{.}%
            \setbox\tikz@figbox=\box\voidb@x%
            \tikz@uninstallcommands%
            \tikz@atbegin@node%
            \aftergroup\tikz@fig@collectresetcolor%
            \ignorespaces%
}


% Auto placement

\def\tikz@auto@pre{%
  \begingroup
    \pgfresetnontranslationattimefalse
    \pgfslopedattimetrue%
    \pgfallowupsidedownattimetrue%
    \tikz@timer%
    \pgf@x=\pgf@pt@aa pt% 
    \pgf@y=\pgf@pt@ab pt%
    \pgfpointnormalised{}%
}

\def\tikz@auto@post{%
    \global\let\tikz@anchor@smuggle=\tikz@anchor%
  \endgroup%
  \let\tikz@anchor=\tikz@anchor@smuggle%
}

\def\tikz@auto@anchor{%
    \ifdim\pgf@x>0.05pt%
      \ifdim\pgf@y>0.05pt%
        \def\tikz@anchor{south east}%
      \else\ifdim\pgf@y<-0.05pt%
        \def\tikz@anchor{south west}%
      \else
        \def\tikz@anchor{south}%
      \fi\fi%
    \else\ifdim\pgf@x<-0.05pt%
      \ifdim\pgf@y>0.05pt%
        \def\tikz@anchor{north east}%
      \else\ifdim\pgf@y<-0.05pt%
        \def\tikz@anchor{north west}%
      \else
        \def\tikz@anchor{north}%
      \fi\fi%
    \else%
      \ifdim\pgf@y>0pt%
        \def\tikz@anchor{east}%
      \else%
        \def\tikz@anchor{west}%
      \fi%
    \fi\fi%
}

\def\tikz@auto@anchor@prime{%
    \ifdim\pgf@x>0.05pt%
      \ifdim\pgf@y>0.05pt%
        \def\tikz@anchor{north west}%
      \else\ifdim\pgf@y<-0.05pt%
        \def\tikz@anchor{north east}%
      \else
        \def\tikz@anchor{north}%
      \fi\fi%
    \else\ifdim\pgf@x<-0.05pt%
      \ifdim\pgf@y>0.05pt%
        \def\tikz@anchor{south west}%
      \else\ifdim\pgf@y<-0.05pt%
        \def\tikz@anchor{south east}%
      \else
        \def\tikz@anchor{south}%
      \fi\fi%
    \else%
      \ifdim\pgf@y>0pt%
        \def\tikz@anchor{west}%
      \else%
        \def\tikz@anchor{east}%
      \fi%
    \fi\fi%
}




% Syntax for trees:
% node {...} child [options] {...} child [options] {...} ...
% node {...} child [options] foreach \var in {list} [options] {...} ...

\def\tikz@children{%
  % Start collecting the children:
  \let\tikz@children@list=\pgfutil@empty%
  \tikznumberofchildren=0\relax%
  \tikz@collect@children c}

\def\tikz@collect@children{\pgfutil@ifnextchar c{\tikz@collect@children@cchar}{\tikz@children@collected}}
\def\tikz@collect@children@cchar c{\pgfutil@ifnextchar h{\tikz@collect@child}{\tikz@children@collected c}}
\def\tikz@collect@child hild{\pgfutil@ifnextchar[{\tikz@collect@childA}{\tikz@collect@childA[]}}%}
\def\tikz@collect@childA[#1]{\pgfutil@ifnextchar f{\tikz@collect@children@foreach[#1]}{\tikz@collect@childB[#1]}}
\def\tikz@collect@childB[#1]{%
  \advance\tikznumberofchildren by1\relax
  \expandafter\def\expandafter\tikz@children@list\expandafter{\tikz@children@list \tikz@childnode[#1]}%
  \pgfutil@ifnextchar\bgroup{\tikz@collect@child@code}{\tikz@collect@child@code{}}}
\def\tikz@collect@child@code#1{%
  \expandafter\def\expandafter\tikz@children@list\expandafter{\tikz@children@list{#1}}%
  \tikz@collect@children%
}
\def\tikz@collect@children@foreach[#1]foreach#2in#3{%
  \pgfutil@ifnextchar\bgroup{\tikz@collect@children@foreachA{#1}{#2}{#3}}{\tikz@collect@children@foreachA{#1}{#2}{#3}{}}}
\def\tikz@collect@children@foreachA#1#2#3#4{%
  \expandafter\def\expandafter\tikz@children@list\expandafter
    {\tikz@children@list\tikz@childrennodes[#1]{#2}{#3}{#4}}%
  \c@pgf@counta=\tikznumberofchildren%
  \foreach#2in{#3}%
  {%
    \global\advance\c@pgf@counta by1\relax%
  }%
  \tikznumberofchildren=\c@pgf@counta%
  \tikz@collect@children%
}
\long\def\tikz@children@collected{%
  \begingroup%
    \advance\tikztreelevel by 1\relax%
    \let\tikz@options=\pgfutil@empty%
    \let\tikz@transform=\pgfutil@empty%
    \tikzset{level/.try=\the\tikztreelevel,level \the\tikztreelevel/.try}%
    \tikz@transform%            
    \let\tikzparentnode=\tikz@last@fig@name%
    % Transform to center of node
    \pgftransformshift{\pgfpointanchor{\tikzparentnode}{\tikz@growth@anchor}}%
    \tikznumberofcurrentchild=0\relax%
    \tikz@children@list%
    \global\setbox\tikz@tempbox=\copy\tikz@figbox%
  \endgroup%
  \setbox\tikz@figbox=\box\tikz@tempbox%  
  \tikz@scan@next@command%
}


% Syntax for children:
%
% child [all children options] foreach \var in {values} [child options] {...}
\def\tikz@childrennodes[#1]#2#3#4{%
  \c@pgf@counta=\tikznumberofcurrentchild\relax%
  \setbox\tikz@tempbox=\box\tikz@figbox%
  \foreach#2in{#3}{%
    \tikznumberofcurrentchild=\c@pgf@counta\relax%
    \setbox\tikz@figbox=\box\tikz@tempbox%
    \tikz@childnode[#1]{#4}%
    % we must now make the current child number and the figbox survive
    % the group
    \global\c@pgf@counta=\tikznumberofcurrentchild\relax%
    \global\setbox\tikz@tempbox=\box\tikz@figbox%
  }%
  \tikznumberofcurrentchild=\c@pgf@counta\relax%
  \setbox\tikz@figbox=\box\tikz@tempbox%
}


% Syntax for child:
%
% child
%
% child[options]
%
% child[options] {node (name) {child node text} ...
%   edge from parent[options] node {label text} node {label text}}

\def\tikz@childnode[#1]#2{%
  \advance\tikznumberofcurrentchild by1\relax%
  \setbox\tikz@figbox=\hbox\bgroup%
    \unhbox\tikz@figbox%
    \hbox\bgroup\bgroup%
        \pgfinterruptpath%
          \pgfscope%
            \let\tikz@transform=\pgfutil@empty%
            \tikzset{every child/.try,#1}%
            \tikz@options%
            \tikz@transform%            
            \tikz@grow%
            % Typeset node:
            \edef\tikz@parent@node@name{[name=\tikzparentnode-\the\tikznumberofcurrentchild,style=every child node]}%
            \def\tikz@child@node@text{[shape=coordinate]{}}
            \tikz@parse@child@node#2\pgf@stop%
            \expandafter\expandafter\expandafter\node
            \expandafter\tikz@parent@node@name
              \tikz@child@node@text
              \pgfextra{\global\let\tikz@childnode@name=\tikz@last@fig@name};%
            \let\tikzchildnode=\tikz@childnode@name%
            {%
              \def\tikz@edge@to@parent@needed{edge from parent}
              \ifx\tikz@child@node@rest\pgfutil@empty%
                \path edge from parent;%
              \else%
                \path (0,0) \tikz@child@node@rest \tikz@edge@to@parent@needed;%
              \fi%
            }%
        \endpgfscope%
      \endpgfinterruptpath%
    \egroup\egroup%
  \egroup%
}

\def\tikz@parse@child@node{%
  \pgfutil@ifnextchar n{\tikz@parse@child@node@n}%
  {\pgfutil@ifnextchar c{\tikz@parse@child@node@c}%
    {\tikz@parse@child@node@rest}}}
\def\tikz@parse@child@node@rest#1\pgf@stop{\def\tikz@child@node@rest{#1}}
\def\tikz@parse@child@node@c c{\pgfutil@ifnextchar o{\tikz@parse@child@node@co}{\tikz@parse@child@node@rest c}}
\def\tikz@parse@child@node@co o{\pgfutil@ifnextchar o{\tikz@parse@child@node@coordinate}{\tikz@parse@child@node@rest co}}
\def\tikz@parse@child@node@coordinate ordinate{%
  \pgfutil@ifnextchar ({\tikz@@parse@child@node@coordinate}{%
    \def\tikz@child@node@text{[shape=coordinate]{}}%
    \tikz@parse@child@node@rest}}%}
\def\tikz@@parse@child@node@coordinate(#1){%
  \pgfutil@ifnextchar a{\tikz@p@c@n@c@at(#1)}{%
    \def\tikz@child@node@text{[shape=coordinate,name=#1]{}}%
    \tikz@parse@child@node@rest}}
\def\tikz@p@c@n@c@at(#1)at#2(#3){%
  \def\tikz@child@node@text{[shape=coordinate,name=#1]at(#3){}}%
  \tikz@parse@child@node@rest}%
\def\tikz@parse@child@node@n node{%
  \let\tikz@child@node@text=\pgfutil@empty%
  \tikz@p@c@s}%
\def\tikz@p@c@s}
\def\tikz@p@c@s@at at#1(#2){%
  \expandafter\def\expandafter\tikz@child@node@text\expandafter{\tikz@child@node@text at(#2)}
  \tikz@p@c@s}
\def\tikz@p@c@s@paran(#1){%
  \expandafter\def\expandafter\tikz@child@node@text\expandafter{\tikz@child@node@text(#1)}
  \tikz@p@c@s}
\def\tikz@p@c@s@bra[#1]{%
  \expandafter\def\expandafter\tikz@child@node@text\expandafter{\tikz@child@node@text[#1]}
  \tikz@p@c@s}
\def\tikz@p@c@s@group#1{%
  \expandafter\def\expandafter\tikz@child@node@text\expandafter{\tikz@child@node@text{#1}}
  \tikz@parse@child@node@rest}


%
% Timers
% 

\def\tikz@timer@line{%
  \pgftransformlineattime{\tikz@time}{\tikz@timer@start}{\tikz@timer@end}%
}

\def\tikz@timer@vhline{%
  \ifdim\tikz@time pt<0.5pt% first half
    \pgf@process{\tikz@timer@start}%
    \pgf@xa=\pgf@x%
    \pgf@ya=\pgf@y%
    \pgf@process{\tikz@timer@end}%
    \pgf@xb=\tikz@time pt%
    \pgf@xb=2\pgf@xb%    
    \edef\tikz@marshal{\noexpand\pgftransformlineattime{\pgf@sys@tonumber{\pgf@xb}}{\noexpand\tikz@timer@start}{%
        \noexpand\pgfqpoint{\the\pgf@xa}{\the\pgf@y}}}%
    \tikz@marshal%
  \else% second half
    \pgf@process{\tikz@timer@start}%
    \pgf@xa=\pgf@x%
    \pgf@ya=\pgf@y%
    \pgf@process{\tikz@timer@end}%
    \pgf@xb=\tikz@time pt%
    \pgf@xb=2\pgf@xb%
    \advance\pgf@xb by-1pt%
    \edef\tikz@marshal{\noexpand\pgftransformlineattime{\pgf@sys@tonumber{\pgf@xb}}%
      {\noexpand\pgfqpoint{\the\pgf@xa}{\the\pgf@y}}{\noexpand\tikz@timer@end}}%
    \tikz@marshal%
  \fi%
}

\def\tikz@timer@hvline{%
  \ifdim\tikz@time pt<0.5pt% first half
    \pgf@process{\tikz@timer@start}%
    \pgf@xa=\pgf@x%
    \pgf@ya=\pgf@y%
    \pgf@process{\tikz@timer@end}%
    \pgf@xb=\tikz@time pt%
    \pgf@xb=2\pgf@xb%    
    \edef\tikz@marshal{\noexpand\pgftransformlineattime{\pgf@sys@tonumber{\pgf@xb}}{\noexpand\tikz@timer@start}{%
        \noexpand\pgfqpoint{\the\pgf@x}{\the\pgf@ya}}}%
    \tikz@marshal%
  \else% second half
    \pgf@process{\tikz@timer@start}%
    \pgf@xa=\pgf@x%
    \pgf@ya=\pgf@y%
    \pgf@process{\tikz@timer@end}%
    \pgf@xb=\tikz@time pt%
    \pgf@xb=2\pgf@xb%
    \advance\pgf@xb by-1pt%
    \edef\tikz@marshal{\noexpand\pgftransformlineattime{\pgf@sys@tonumber{\pgf@xb}}%
      {\noexpand\pgfqpoint{\the\pgf@x}{\the\pgf@ya}}{\noexpand\tikz@timer@end}}%
    \tikz@marshal%
  \fi%
}

\def\tikz@timer@curve{%
  \pgftransformcurveattime{\tikz@time}{\tikz@timer@start}{\tikz@timer@cont@one}{\tikz@timer@cont@two}{\tikz@timer@end}%
}



%
% Coordinate systems
% 

\def\tikzdeclarecoordinatesystem#1#2{%
  \expandafter\def\csname tikz@parse@cs@#1\endcsname(##1){%
    \pgf@process{%
      #2%
      % Smuggle outside:
      \iftikz@shapeborder%
        \global\let\tikz@smuggle@a=\tikz@shapebordertrue%
      \else%
        \global\let\tikz@smuggle@a=\tikz@shapeborderfalse%
      \fi%
      \global\let\tikz@smubble@b=\tikz@shapeborder@name%
    }%
    \tikz@smuggle@a%
    \let\tikz@shapeborder@name=\tikz@smubble@b%
    \edef\tikz@return@coordinate{\noexpand\pgfqpoint{\the\pgf@x}{\the\pgf@y}}}%
}
\def\tikzaliascoordinatesystem#1#2{%
  \edef\pgf@marshal{\noexpand\let\expandafter\noexpand\csname
    tikz@parse@cs@#1\endcsname=\expandafter\noexpand\csname
    tikz@parse@cs@#2\endcsname}%
  \pgf@marshal%
}


% Default coodinate systems:

\tikzdeclarecoordinatesystem{canvas}
{%
  \tikzset{cs/.cd,x=0pt,y=0pt,#1}%
  \pgfpoint{\tikz@cs@x}{\tikz@cs@y}%
}

\tikzdeclarecoordinatesystem{canvas polar}
{%
  \tikzset{cs/.cd,angle=0,radius=0cm,#1}%
  \pgfpointpolar{\tikz@cs@angle}{\tikz@cs@xradius/\tikz@cs@yradius}%
}

\tikzdeclarecoordinatesystem{xyz}
{%
  \tikzset{cs/.cd,x=0,y=0,z=0,#1}%
  \pgfpointxyz{\tikz@cs@x}{\tikz@cs@y}{\tikz@cs@z}%
}

\tikzdeclarecoordinatesystem{xyz polar}
{%
  \tikzset{cs/.cd,angle=0,radius=0,#1}%
  \pgfpointpolarxy{\tikz@cs@angle}{\tikz@cs@xradius and \tikz@cs@yradius}%
}
\tikzaliascoordinatesystem{xy polar}{xyz polar}


\tikzdeclarecoordinatesystem{node}
{%
  \tikzset{cs/.cd,name=,anchor=none,angle=none,#1}%
  \ifx\tikz@cs@anchor\tikz@nonetext%
    \ifx\tikz@cs@angle\tikz@nonetext%
      \expandafter\ifx\csname pgf@sh@ns@\tikz@cs@node\endcsname\tikz@coordinate@text%
      \else
        \tikz@shapebordertrue%
        \edef\tikz@shapeborder@name{\tikz@cs@node}%
      \fi%
      \pgfpointanchor{\tikz@cs@node}{center}%
    \else%
      \pgfpointanchor{\tikz@cs@node}{\tikz@cs@angle}%
    \fi%
  \else%
    \pgfpointanchor{\tikz@cs@node}{\tikz@cs@anchor}%
  \fi%
}

\tikzdeclarecoordinatesystem{intersection}
{%
  \tikzset{cs/.cd,#1}%
  \expandafter\tikz@@@scan@@absolute\expandafter\tikz@parse@intersection@a\tikz@cs@line@a@begin%
  \expandafter\tikz@@@scan@@absolute\expandafter\tikz@parse@intersection@b\tikz@cs@line@a@end%
  \expandafter\tikz@@@scan@@absolute\expandafter\tikz@parse@intersection@c\tikz@cs@line@b@begin%
  \expandafter\tikz@@@scan@@absolute\expandafter\tikz@parse@intersection@d\tikz@cs@line@b@end%
  \edef\pgf@marshal{%
    {\noexpand\pgfpointintersectionoflines%
      {\noexpand\pgfqpoint{\the\pgf@xa}{\the\pgf@ya}}%
      {\noexpand\pgfqpoint{\the\pgf@xb}{\the\pgf@yb}}%
      {\noexpand\pgfqpoint{\the\pgf@xc}{\the\pgf@yc}}%
      {\noexpand\pgfqpoint{\the\pgf@x}{\the\pgf@y}}}}%
  \pgf@marshal%
}

\tikzdeclarecoordinatesystem{perpendicular}
{%
  \tikzset{cs/.cd,#1}%
  \expandafter\tikz@@@scan@@absolute\expandafter\tikz@parse@intersection@a\tikz@cs@hori@line%
  \expandafter\tikz@@@scan@@absolute\expandafter\tikz@parse@intersection@b\tikz@cs@vert@line%
  \pgfqpoint{\the\pgf@xb}{\the\pgf@ya}
}

\tikzdeclarecoordinatesystem{barycentric}
{%
  {%
    \pgf@xa=0pt% point
    \pgf@ya=0pt%
    \pgf@xb=0pt% sum
    \tikz@bary@dolist#1,=,%
    \pgfmathparse{1/\the\pgf@xb}%
    \global\pgf@x=\pgfmathresult\pgf@xa%
    \global\pgf@y=\pgfmathresult\pgf@ya%
  }%
}

\def\tikz@bary@dolist#1=#2,{%
  \def\tikz@temp{#1}%
  \ifx\tikz@temp\pgfutil@empty%
  \else
    \pgf@process{\pgfpointanchor{#1}{center}}%
    \pgfmathparse{#2}%
    \advance\pgf@xa by\pgfmathresult\pgf@x%
    \advance\pgf@ya by\pgfmathresult\pgf@y%
    \advance\pgf@xb by\pgfmathresult pt%
    \expandafter\tikz@bary@dolist%
  \fi%
}

\tikzset{cs/x/.store in=\tikz@cs@x}
\tikzset{cs/y/.store in=\tikz@cs@y}
\tikzset{cs/z/.store in=\tikz@cs@z}
\tikzset{cs/angle/.store in=\tikz@cs@angle}
\tikzset{cs/x radius/.store in=\tikz@cs@xradius}
\tikzset{cs/y radius/.store in=\tikz@cs@yradius}
\tikzset{cs/radius/.style={/tikz/cs/x radius=#1,/tikz/cs/y radius=#1}}
\tikzset{cs/name/.store in=\tikz@cs@node}
\tikzset{cs/anchor/.store in=\tikz@cs@anchor}

\tikzset{cs/first line/.code args={(#1)--(#2)}{\def\tikz@cs@line@a@begin{(#1)}\def\tikz@cs@line@a@end{(#2)}}}
\tikzset{cs/second line/.code args={(#1)--(#2)}{\def\tikz@cs@line@b@begin{(#1)}\def\tikz@cs@line@b@end{(#2)}}}

\tikzset{cs/horizontal line through/.store in=\tikz@cs@hori@line}
\tikzset{cs/vertical line through/.store in=\tikz@cs@vert@line}




%
% Coordinate management
%


% Last position visited
\def\tikz@last@position{\pgfqpoint{\tikz@lastx}{\tikz@lasty}}
\def\tikz@last@position@saved{\pgfqpoint{\tikz@lastxsaved}{\tikz@lastysaved}}

% Make given point the last position visited
\def\tikz@make@last@position#1{%
  \pgf@process{#1}%
  \tikz@lastx=\pgf@x\relax%
  \tikz@lasty=\pgf@y\relax%
  \iftikz@updatecurrent%
    \tikz@lastxsaved=\pgf@x\relax%
    \tikz@lastysaved=\pgf@y\relax%
  \fi%
  \tikz@updatecurrenttrue%
}

\newif\iftikz@updatecurrent
\tikz@updatecurrenttrue



% Scanner: Scans a point or a relative point. 
% It then calls the first parameter with the argument set to an
% appropriate pgf command representing that point.

\def\tikz@scan@one@point#1{%
  \let\tikz@to@use@whom=\tikz@to@use@last@coordinate%
  \tikz@shapeborderfalse%
  \pgfutil@ifnextchar+{\tikz@scan@relative#1}{\tikz@scan@absolute#1}}
\def\tikz@scan@absolute#1{%
  \pgfutil@ifnextchar({\tikz@scan@@absolute#1}%)
  {%
    \advance\tikz@expandcount by -1%
    \ifnum\tikz@expandcount<0\relax%
      \let\@next=\tikz@@scangiveup%
    \else%
      \let\@next=\tikz@@scanexpand%
    \fi%
    \@next{#1}%
  }%
}
\def\tikz@@scanexpand#1{\expandafter\tikz@scan@one@point\expandafter#1}
\def\tikz@@scangiveup#1{\PackageError{tikz}{Cannot parse this coordinate}{}#1{\pgfpointorigin}}
\def\tikz@scan@@absolute#1(#2){%
  \edef\tikz@temp{(#2)}%
  \expandafter\tikz@@scan@@absolute\expandafter#1\tikz@temp%
}
\def\tikz@@scan@@absolute#1({%
  \pgfutil@ifnextchar[% uhoh... options!
  {\def\tikz@scan@point@recall{#1}\tikz@scan@options}%
  {\tikz@@@scan@@absolute#1(}%
}

\def\tikz@scan@options[#1]#2{%
  \def\tikz@scan@point@options{#1}%
  \tikz@@@scan@@absolute\tikz@scan@handle@options(#2%
}

\def\tikz@scan@handle@options#1{%
  {%
    % Ok, compute point with options set and zero transformation
    % matrix:
    \pgftransformreset%
    \let\tikz@transform=\pgfutil@empty%
    \expandafter\tikzset\expandafter{\tikz@scan@point@options}%
    \tikz@transform%
    \pgf@process{\pgfpointtransformed{#1}}%
    \xdef\tikz@marshal{\expandafter\noexpand\tikz@scan@point@recall{\noexpand\pgfqpoint{\the\pgf@x}{\the\pgf@y}}}%
  }%
  \tikz@marshal%  
}

\def\tikz@@@scan@@absolute#1(#2){%
  \pgfutil@in@{intersection of}{#2}%
  \ifpgfutil@in@%
    \let\@next\tikz@parse@intersection%
  \else%
    \pgfutil@in@|{#2}%
    \ifpgfutil@in@
      \pgfutil@in@{-|}{#2}%
      \ifpgfutil@in@
        \let\@next\tikz@parse@hv%
      \else%
        \let\@next\tikz@parse@vh%
      \fi%
    \else%
      \pgfutil@in@{cs:}{#2}%
      \ifpgfutil@in@%
        \let\@next\tikz@parse@coordinatesystem%
      \else%
        \pgfutil@in@:{#2}%
        \ifpgfutil@in@
          \let\@next\tikz@parse@polar%
        \else%
          \pgfutil@in@,{#2}%
          \ifpgfutil@in@%      
            \let\@next\tikz@parse@regular%
          \else%
            \let\@next\tikz@parse@node%
          \fi%
        \fi%
      \fi%
    \fi%
  \fi%
  \@next#1(#2)%
}

\def\tikz@parse@coordinatesystem#1(#2 cs:#3){%
  \let\tikz@return@coordinate=\pgfpointorigin%
  \pgfutil@ifundefined{tikz@parse@cs@#2}
  {\PackageError{tikz}{Unknown coordinate system '#2'}{}}
  {\csname tikz@parse@cs@#2\endcsname(#3)}%
  \expandafter#1\expandafter{\tikz@return@coordinate}%
}


\newif\iftikz@isdimension
\def\tikz@checkunit#1{%
  \pgfmathparse{#1}%
  \let\iftikz@isdimension=\ifpgfmathunitsdeclared%
}
\def\tikz@@checkunit{\pgfutil@ifnextchar\tikz@unique{\tikz@checkunit@number}{\tikz@checkunit@dimension}}
\def\tikz@checkunit@number\tikz@unique{\tikz@isdimensionfalse}
\def\tikz@checkunit@dimension#1\tikz@unique{\tikz@isdimensiontrue}

\def\tikz@parse@polar#1(#2:#3){%
  \pgfutil@ifundefined{tikz@polar@dir@#2}
  {\tikz@@parse@polar#1(#2:#3)}
  {\tikz@@parse@polar#1(\csname tikz@polar@dir@#2\endcsname:#3)}%
}
\def\tikz@@parse@polar#1(#2:#3){%
  \pgfutil@in@{ and }{#3}%
  \ifpgfutil@in@%
    \edef\tikz@args{(#2:#3)}%
  \else%
    \edef\tikz@args{(#2:#3 and #3)}%
  \fi%
  \expandafter\tikz@@@parse@polar\expandafter#1\tikz@args%
}
\def\tikz@@@parse@polar#1(#2:#3 and #4){%
  \tikz@checkunit{#3}%
  \iftikz@isdimension%
    \def\tikz@next{#1{\pgfpointpolar{#2}{#3 and #4}}}%
  \else%
    \def\tikz@next{#1{\pgfpointpolarxy{#2}{#3 and #4}}}%
  \fi%
  \tikz@next%
}
\def\tikz@polar@dir@up{90}
\def\tikz@polar@dir@down{-90}
\def\tikz@polar@dir@left{180}
\def\tikz@polar@dir@right{0}
\def\tikz@polar@dir@north{90}
\def\tikz@polar@dir@south{-90}
\def\tikz@polar@dir@east{0}
\def\tikz@polar@dir@west{180}
\expandafter\def\csname tikz@polar@dir@north east\endcsname{45}
\expandafter\def\csname tikz@polar@dir@north west\endcsname{135}
\expandafter\def\csname tikz@polar@dir@south east\endcsname{-45}
\expandafter\def\csname tikz@polar@dir@south west\endcsname{-135}

\def\tikz@parse@regular#1(#2,#3){%
  \pgfutil@in@,{#3}%
  \ifpgfutil@in@%  
    \tikz@parse@splitxyz{#1}{#2}#3,%
  \else%
    \tikz@checkunit{#2}%
    \iftikz@isdimension%
      \def\@next{#1{\pgfpoint{#2}{#3}}}%
    \else%
      \def\@next{#1{\pgfpointxy{#2}{#3}}}%
    \fi%
  \fi%
  \@next%
}

\def\tikz@parse@splitxyz#1#2#3,#4,{%
  \def\@next{#1{\pgfpointxyz{#2}{#3}{#4}}}%
}

\def\tikz@coordinate@text{coordinate}

\def\tikz@parse@node#1(#2){%
  \pgfutil@in@.{#2}% Ok, flag this
  \ifpgfutil@in@
    \tikz@calc@anchor#2\tikz@stop%
  \else%
    \tikz@calc@anchor#2.center\tikz@stop% to be on the save side, in
                                % case iftikz@shapeborder is ignored...
    \expandafter\ifx\csname pgf@sh@ns@#2\endcsname\tikz@coordinate@text%
    \else
      \tikz@shapebordertrue%
      \def\tikz@shapeborder@name{#2}%
    \fi%
  \fi%
  \edef\tikz@marshal{\noexpand#1{\noexpand\pgfqpoint{\the\pgf@x}{\the\pgf@y}}}%
  \tikz@marshal%
}

\def\tikz@calc@anchor#1.#2\tikz@stop{%
  \pgfpointanchor{#1}{#2}%
}


\def\tikz@parse@hv#1(#2){%
  \pgfutil@in@{ -| }{#2}%
  \ifpgfutil@in@%
    \let\tikz@next=\tikz@parse@hvboth%
  \else%
    \pgfutil@in@{ -|}{#2}%
    \ifpgfutil@in@%
      \let\tikz@next=\tikz@parse@hvleft%
    \else%
      \pgfutil@in@{-| }{#2}%
      \ifpgfutil@in@%
        \let\tikz@next=\tikz@parse@hvright%
      \else%
        \let\tikz@next=\tikz@parse@hvdone%
      \fi%
    \fi%
  \fi%
  \tikz@next#1(#2)}
\def\tikz@parse@hvboth#1(#2 -| #3){\tikz@parse@vhdone#1(#3|-#2)}
\def\tikz@parse@hvleft#1(#2 -|#3){\tikz@parse@vhdone#1(#3|-#2)}
\def\tikz@parse@hvright#1(#2-| #3){\tikz@parse@vhdone#1(#3|-#2)}
\def\tikz@parse@hvdone#1(#2-|#3){\tikz@parse@vhdone#1(#3|-#2)}

\def\tikz@parse@vh#1(#2){%
  \pgfutil@in@{ |- }{#2}%
  \ifpgfutil@in@%
    \let\tikz@next=\tikz@parse@vhboth%
  \else%
    \pgfutil@in@{ |-}{#2}%
    \ifpgfutil@in@%
      \let\tikz@next=\tikz@parse@vhleft%
    \else%
      \pgfutil@in@{|- }{#2}%
      \ifpgfutil@in@%
        \let\tikz@next=\tikz@parse@vhright%
      \else%
        \let\tikz@next=\tikz@parse@vhdone%
      \fi%
    \fi%
  \fi%
  \tikz@next#1(#2)}
\def\tikz@parse@vhboth#1(#2 |- #3){\tikz@parse@vhdone#1(#2|-#3)}
\def\tikz@parse@vhleft#1(#2 |-#3){\tikz@parse@vhdone#1(#2|-#3)}
\def\tikz@parse@vhright#1(#2|- #3){\tikz@parse@vhdone#1(#2|-#3)}
\def\tikz@parse@vhdone#1(#2|-#3){%
  {%
    \tikz@@@scan@@absolute\tikz@parse@vh@mid(#2)%
    \tikz@@@scan@@absolute\tikz@parse@vh@end(#3)%
    \xdef\tikz@marshal{\noexpand#1{\noexpand\pgfqpoint{\the\pgf@xa}{\the\pgf@ya}}}%
  }%
  \tikz@shapeborderfalse%
  \tikz@marshal%
}
\def\tikz@parse@vh@mid#1{\pgf@process{#1}\pgf@xa=\pgf@x}
\def\tikz@parse@vh@end#1{\pgf@process{#1}\pgf@ya=\pgf@y}

\def\tikz@parse@intersection#1(intersection of #2--#3 and #4--#5){%
  {%
    \tikz@@@scan@@absolute\tikz@parse@intersection@a(#2)%
    \tikz@@@scan@@absolute\tikz@parse@intersection@b(#3)%
    \tikz@@@scan@@absolute\tikz@parse@intersection@c(#4)%
    \tikz@@@scan@@absolute\tikz@parse@intersection@d(#5)%
    \xdef\tikz@marshal{\noexpand#1{\noexpand\pgfpointintersectionoflines%
        {\noexpand\pgfqpoint{\the\pgf@xa}{\the\pgf@ya}}%
        {\noexpand\pgfqpoint{\the\pgf@xb}{\the\pgf@yb}}%
        {\noexpand\pgfqpoint{\the\pgf@xc}{\the\pgf@yc}}%
        {\noexpand\pgfqpoint{\the\pgf@x}{\the\pgf@y}}}}%
  }%
  \tikz@shapeborderfalse%
  \tikz@marshal%  
}

\def\tikz@parse@intersection@a#1{\pgf@process{#1}\pgf@xa=\pgf@x\pgf@ya=\pgf@y}
\def\tikz@parse@intersection@b#1{\pgf@process{#1}\pgf@xb=\pgf@x\pgf@yb=\pgf@y}
\def\tikz@parse@intersection@c#1{\pgf@process{#1}\pgf@xc=\pgf@x\pgf@yc=\pgf@y}
\def\tikz@parse@intersection@d#1{\pgf@process{#1}}

\def\tikz@scan@relative#1+{%
  \pgfutil@ifnextchar+{\tikz@scan@plusplus#1}{\tikz@scan@oneplus#1}}

\def\tikz@scan@plusplus#1+{%
  \def\tikz@doafter{#1}%
  \tikz@scan@absolute\tikz@add%
}
\def\tikz@add#1{%
  \tikz@doafter{\pgfpointadd{#1}{\tikz@last@position@saved}}%
}
\def\tikz@scan@oneplus#1{%
  \def\tikz@doafter{#1}%
  \tikz@updatecurrentfalse%
  \tikz@scan@absolute\tikz@add%
} 



% Loading further libraries

% Include a library file.
%
% #1 = List of names of library file.
%  
% Description:
%
% This command includes a list of TikZ library files. For each file X in the
% list, the file pgflibrarytikzX.code.tex is included, provided this has
% not been done earlier. 
%
% For the convenience of Context users, both round and square brackets
% are possible for the argument.
%
% Example:
%
% \usetikzlibrary{arrows}
% \usetikzlibrary[patterns,topaths]

\def\usetikzlibrary{\pgfutil@ifnextchar[{\use@tikzlibrary}{\use@@tikzlibrary}}%}
\def\use@tikzlibrary[#1]{\use@@tikzlibrary{#1}}
\def\use@@tikzlibrary#1{%
  \edef\pgf@list{#1}%
  \pgfutil@for\pgf@temp:=\pgf@list\do{%
    \expandafter\ifx\csname tikz@library@\pgf@temp @loaded\endcsname\relax%
      \expandafter\global\expandafter\let\csname tikz@library@\pgf@temp @loaded\endcsname=\pgfutil@empty%
      \expandafter\edef\csname tikz@library@#1@atcode\endcsname{\the\catcode`\@}
      \expandafter\edef\csname tikz@library@#1@barcode\endcsname{\the\catcode`\|}
      \catcode`\@=11
      \catcode`\|=12
      \input pgflibrarytikz\pgf@temp.code.tex
      \catcode`\@=\csname tikz@library@#1@atcode\endcsname
      \catcode`\|=\csname tikz@library@#1@barcode\endcsname
    \fi%
  }%
}


% Always-present libraries:

\usetikzlibrary{topaths}




\endinput


\endinput
\end{codeexample}

The files in the |generic/pgf| directory do the actual work.



\subsubsection{Using the Plain \TeX\ Format}

When using the plain \TeX\ format, you say |% Copyright 2006 by Till Tantau
%
% This file may be distributed and/or modified
%
% 1. under the LaTeX Project Public License and/or
% 2. under the GNU Public License.
%
% See the file doc/generic/pgf/licenses/LICENSE for more details.


\edef\pgfatcode{\the\catcode`\@}
\catcode`\@=11


\input pgfrcs.tex
\ProvidesPackageRCS $Header: /cvsroot/pgf/pgf/plain/pgf/basiclayer/pgf.tex,v 1.9 2008/01/13 10:35:47 vibrovski Exp $

\input pgfcore.tex

\usepgfmodule{shapes,plot}

%\input pgfbasesnakes.tex
%\input pgfbasedecorations.tex
%\input pgfbasematrix.tex

\catcode`\@=\pgfatcode

\endinput
| or
|% Copyright 2006 by Till Tantau
%
% This file may be distributed and/or modified
%
% 1. under the LaTeX Project Public License and/or
% 2. under the GNU Public License.
%
% See the file doc/generic/pgf/licenses/LICENSE for more details.

% This file is tikz.tex

\edef\tikzatcode{\the\catcode`\@}
\catcode`\@=11

\input xkeyval.tex
\input pgf.tex
\input pgffor.tex
\input tikz.code.tex

\catcode`\@=\tikzatcode

\endinput
|. Instead of  |\begin{pgfpicture}| and
  |\end{pgfpicture}| you use  |\pgfpicture| and |\endpicture|. 

Unlike for the \LaTeX\ format, \pgfname\ is not as good at discerning
the appropriate configuration for the plain \TeX\ format. In
particular, it can only automatically determine the correct output
format if you use |pdftex| or |tex| plus |dvips|. For all other output
formats you need to set the macro |\pgfsysdriver| to the correct
value. See the description of using output formats later on. 

\pgfname\ was originally written for use with \LaTeX\ and this shows
in a number of places. Nevertheless, the plain \TeX\ support is
reasonably good.

Like the \LaTeX\ style files, the plain \TeX\ files like |tikz.tex|
also just include the correct |tikz.code.tex| file.



\subsubsection{Using the Con\TeX t Format}

Currently, there is no special support for the Con\TeX t
format. Rather, you have to use \pgfname\ and \tikzname\ as if you
were using the plain \TeX\ format when using Con\TeX t. This may
change in the future.





\subsection{Supported Output Formats}
\label{section-drivers}

An output format is a format in which \TeX\ outputs the text it has
typeset. Producing the output is (conceptually) a two-stage process:
\begin{enumerate}
\item
  \TeX\ typesets your text and graphics. The result of this
  typesetting is mainly a long list of letter--coordinate pairs, plus 
  (possibly) some ``special'' commands. This long list of pairs
  is written to something called a |.dvi|-file.
\item
  Some other program reads this |.dvi|-file and translates the
  letter--coordinate pairs into, say, PostScript commands for placing
  the given letter at the given coordinate.
\end{enumerate}

The classical example of this process is the combination of |latex|
and |dvips|. The |latex| program (which is just the |tex| program
called with the \LaTeX-macros preinstalled) produces a |.dvi|-file as
its output. The |dvips| program takes this output and produces a
|.ps|-file (a PostScript) file. Possibly, this file is further
converted using, say, |ps2pdf|, whose name is supposed to mean
``PostScript to PDF.'' Another example of programs using this
process is the combination of |tex| and |dvipdfm|. The |dvipdfm|
program takes a |.dvi|-file as 
input and translates the letter--coordinate pairs therein into
\pdf-commands, resulting in a |.pdf| file directly. Finally, the
|tex4ht| is also a program that takes a |.dvi|-file and produces an
output, this time it is a |.html| file. The programs |pdftex| and
|pdflatex| are special: They directly produce a |.pdf|-file without
the intermediate |.dvi|-stage. However, from the programmer's point of
view they behave exactly as if there where an intermediate stage.

Normally, \TeX\ only produces letter--coordinate pairs as its
``output.'' This obviously makes is difficult tho draw, say, a
curve. For this, ``special'' commands can be used. Unfortunately,
these special commands are not the same for the different programs
that process the |.dvi|-file. Indeed, every program that takes a
|.dvi|-file as input has a totally different syntax for the special
commands.

One of the main jobs of \pgfname\ is to ``abstract way'' the
difference in the syntax of the different programs. However, this
means that support for each program has to be ``programmed,'' which is
a time-consuming and complicated process. 


\subsubsection{Selecting the Backend Driver}

When \TeX\ typesets your document, it does not know which program
you are going to use to transform the |.dvi|-file. If your |.dvi|-file
does not contain any special commands, this would be fine; but these
days almost all |.dvi|-files contain lots of special commands. It is
thus necessary to tell \TeX\ which program you are going to use later
on.

Unfortunately, there is no ``standard'' way of telling this to
\TeX. For the \LaTeX\ format a sophisticated mechanism exists inside
the |graphics| package and \pgfname\ plugs into this mechanism. For
other formats and when this plugging does not work as expected, it is
necessary to tell \pgfname\ directly which program you are going to
use. This is done by redefining the macro |\pgfsysdriver| to an
appropriate value \emph{before} you load |pgf|. If you are going to
use the |dvips| program, you set this macro to the value
|pgfsys-dvips.def|; if you use |pdftex| or |pdflatex|, you set it to
|pgfsys-pdftex.def|; and so on. In the following, details of the
support of the different programs are discussed.


\subsubsection{Producing PDF Output}

\pgfname\ supports three programs that produce \pdf\ output (\pdf\ means
``portable document format'' and was invented by the Adobe company):
|dvipdfm|, |pdftex|, and |vtex|. The |pdflatex| program is the same as the
|pdftex| program: it uses a different input format, but the output is
exactly the same.

\begin{filedescription}{pgfsys-pdftex.def}
  This is the driver file for use with pdf\TeX, that is, with the
  |pdftex| or |pdflatex| command. It includes
  |pgfsys-common-pdf.def|.

  This driver has the ``complete'' functionality. This means,
  everything \pgfname\ ``can do at all'' is implemented in this
  driver. 
\end{filedescription}

\begin{filedescription}{pgfsys-dvipdfm.def}
  This is a driver file for use with (|la|)|tex| followed by |dvipdfm|. It
  includes |pgfsys-common-pdf.def|.

  This driver supports most of \pgfname's features, but there are some
  restrictions:
  \begin{enumerate}
  \item
    In \LaTeX\ mode it uses |graphicx| for the graphics
    inclusion and does not support masking.
  \item
    In plain \TeX\ mode it does not support image inclusion.
  \end{enumerate}
\end{filedescription}

\begin{filedescription}{pgfsys-vtex.def}
  This is the driver file for use with the commercial \textsc{vtex}
  program. Even though is will produce  \textsc{pdf} output, it
  includes |pgfsys-common-postscript.def|. Note that the
  \textsc{vtex} program can produce \emph{both} Postscript and
  \textsc{pdf} output, depending on the command line
  parameters. However, whether you produce Postscript or
  \textsc{pdf} output does not change anything with respect to the
  driver. 

  This driver supports most of \pgfname's features, except for
  the following restrictions:
  \begin{enumerate}
  \item
    In \LaTeX\ mode it uses |graphicx| for the graphics
    inclusion and does not support masking.
  \item
    In plain \TeX\ mode it does not support image inclusion.
  \item
    Shading is fully implemented, but yields the same quality as the
    implementation for |dvips|.
  \item
    Opacity is not implemented at all.
  \end{enumerate}
\end{filedescription}

It is also possible to produce a |.pdf|-file by first producing a
PostScript file (see below) and then using a PostScript-to-\pdf\
conversion program like |ps2pdf| or the Acrobat Distiller.


\subsubsection{Producing PostScript Output}

\begin{filedescription}{pgfsys-dvips.def}
  This is a driver file for use with (|la|)|tex| followed by
  |dvips|. It includes |pgfsys-common-postscript.def|.

  This driver also supports most of \pgfname's features, except for
  the following restrictions:
  \begin{enumerate}
  \item
    In \LaTeX\ mode it uses |graphicx| for the graphics
    inclusion and does not support masking.
  \item
    In plain \TeX\ mode it does not support image inclusion.
  \item
    Shading is fully implemented, but the results will not be 
    as good as with a driver producing |.pdf| as output. 
  \item
    Opacity works only in conjunction with newer versions of
    GhostScript. 
  \end{enumerate}
\end{filedescription}

You can also use the |vtex| program together with |pgfsys-vtex.def| to
produce Postscript output.



\subsubsection{Producing HTML / SVG Output}

The |tex4ht| program converts |.dvi|-files to |.html|-files. While the
\textsc{html}-format cannot be used to draw graphics, the
\textsc{svg}-format can. Using the following driver, you can ask
\pgfname\ to produce an \textsc{svg}-picture for each \pgfname\
graphic in your text.

\begin{filedescription}{pgfsys-tex4ht.def}
  This is a driver file for use with the |tex4ht| program. It includes
  |pgfsys-common-svg.def|.

  When using this driver you should be aware of the following
  restrictions: 
  \begin{enumerate}
  \item
    In \LaTeX\ mode it uses |graphicx| for the graphics
    inclusion.    
  \item
    In plain \TeX\ mode it does not support image inclusion.
  \item
    Text inside |pgfpicture|s is not supported very well. The reason
    is that the \textsc{svg} specification currently does not support
    text very well and it is also not possible to correctly ``escape
    back'' to \textsc{html}. All these problems will hopefully
    disappear in the future, but currently only two kinds of text work
    reasonably well: First, plain text without math mode, special
    characters or anything else special. Second, \emph{very} simple
    mathematical text that contains subscripts or superscripts. Even
    then, variables are not correctly set in italics and, in general,
    text simple does not look very nice.
  \item
    If you use text that contains anything special, even something as
    simple as |$\alpha$|, this may corrupt the graphic since |text4ht|
    does not always produce valid \textsc{xml} code. So, once more,
    \emph{stick to very simple node text inside graphics.} Sorry.
  \item
    Unlike for other output formats, the bounding box of a picture
    ``really crops'' the picture.
  \end{enumerate}

  The driver basically works as follows: When a |{pgfpicture}| is
  started, appropriate |\special| commands are used to directed the
  output of |tex4ht| to a new file called |\jobname-xxx.svg|, where
  |xxx| is a number that is increased for each graphic. Then, till the
  end of the picture, each (system layer) graphic command creates a
  specials that insert appropriate \textsc{svg} literal text into the
  output file. The exact details are a bit complicated since the
  imaging model and the processing model of PostScript/\pdf\ and
  \textsc{svg} are not quite the same; but they are ``close enough''
  for \pgfname's purposes.
\end{filedescription}





\part{Ti\emph{k}Z ist \emph{kein} Zeichenprogramm}
\label{part-tikz}

{\Large \emph{by Till Tantau}}


\bigskip
\noindent
\vskip3cm
\begin{codeexample}[graphic=white]
\begin{tikzpicture}
  \draw[fill=yellow] (0,0) -- (60:.75cm) arc (60:180:.75cm);
  \draw(120:0.4cm) node {$\alpha$};

  \draw[fill=green!30] (0,0) -- (right:.75cm) arc (0:60:.75cm);
  \draw(30:0.5cm) node {$\beta$};

  \begin{scope}[shift={(60:2cm)}]
    \draw[fill=green!30] (0,0) -- (180:.75cm) arc (180:240:.75cm);
    \draw (30:-0.5cm) node {$\gamma$};

    \draw[fill=yellow] (0,0) -- (240:.75cm) arc (240:360:.75cm);
    \draw (-60:0.4cm) node {$\delta$};
  \end{scope}

  \begin{scope}[thick]
    \draw  (60:-1cm) node[fill=white] {$E$} -- (60:3cm) node[fill=white] {$F$};
    \draw[red]                   (-2,0) node[left] {$A$} -- (3,0) node[right]{$B$};
    \draw[blue,shift={(60:2cm)}] (-3,0) node[left] {$C$} -- (2,0) node[right]{$D$};

    \draw[shift={(60:1cm)},xshift=4cm]
    node [right,text width=6cm,rounded corners,fill=red!20,inner sep=1ex]
    {
      When we assume that $\color{red}AB$ and $\color{blue}CD$ are
      parallel, i.\,e., ${\color{red}AB} \mathbin{\|} \color{blue}CD$,
      then $\alpha = \delta$ and $\beta = \gamma$.
    };
  \end{scope}
\end{tikzpicture}
\end{codeexample}



% Copyright 2006 by Till Tantau
%
% This file may be distributed and/or modified
%
% 1. under the LaTeX Project Public License and/or
% 2. under the GNU Free Documentation License.
%
% See the file doc/generic/pgf/licenses/LICENSE for more details.

\section{Design Principles}

This section describes the design principles behind the \tikzname\
frontend, where \tikzname\ means ``\tikzname\ ist \emph{kein}
Zeichenprogramm.'' To use \tikzname, as a \LaTeX\ user say
|\usepackage{tikz}| somewhere in the preamble, as a plain \TeX\ user
say |\input tikz.tex|. \tikzname's job is to make your life easier by
providing an easy-to-learn and easy-to-use syntax for describing
graphics.

The commands and syntax of \tikzname\ were influenced by several
sources. The basic command names and the notion of  path operations is
taken from \textsc{metafont}, the option mechanism comes from
\textsc{pstricks}, the notion of styles is reminiscent of
\textsc{svg}, the graph syntax is taken from \textsc{graphviz}. To make it
all work together, some compromises were necessary. I also added some
ideas of my own, like coordinate transformations.

The following basic design principles underlie \tikzname:
\begin{enumerate}
\item Special syntax for specifying points.
\item Special syntax for path specifications.
\item Actions on paths.
\item Key-value syntax for graphic parameters.
\item Special syntax for nodes.
\item Special syntax for trees.
\item Special syntax for graphs.
\item Grouping of graphic parameters.
\item Coordinate transformation system.
\end{enumerate}



\subsection{Special Syntax For Specifying Points}

\tikzname\ provides a special syntax for specifying points and
coordinates. In the simplest case, you provide two \TeX\ dimensions,
separated by commas, in round brackets as in |(1cm,2pt)|.

You can also specify a point in polar coordinates by using a colon
instead of a comma as in |(30:1cm)|, which means ``1cm in a 30
degrees direction.''

If you do not provide a unit, as in |(2,1)|, you specify a point in
\pgfname's $xy$-coordinate system. By default, the unit $x$-vector
goes 1cm to the right and the unit $y$-vector goes 1cm upward.

By specifying three numbers as in |(1,1,1)| you specify a point in
\pgfname's $xyz$-coordinate system.

It is also possible to use an anchor of a previously defined shape
as in |(first node.south)|.

You can add two plus signs before a coordinate as in
|++(1cm,0pt)|. This means ``1cm to the right of the last point
used.'' This allows you to easily specify relative movements. For
example, |(1,0) ++(1,0) ++(0,1)| specifies the three coordinates
|(1,0)|, then |(2,0)|, and |(2,1)|.

Finally, instead of two plus signs, you can also add a single
one. This also specifies a point in a relative manner, but it does
not ``change'' the current point used in subsequent relative
commands. For example, |(1,0) +(1,0) +(0,1)| specifies the three
coordinates |(1,0)|, then |(2,0)|, and |(1,1)|.

\subsection{Special Syntax For Path Specifications}

When creating a picture using \tikzname, your main job is the
specification of \emph{paths}. A path is a series of straight or curved
lines, which need not be connected. \tikzname\ makes it easy to
specify paths, partly using the syntax of \textsc{metapost}. For
example, to specify a triangular path you use
\begin{codeexample}[code only]
(5pt,0pt) -- (0pt,0pt) -- (0pt,5pt) -- cycle
\end{codeexample}
and you get \tikz \draw (5pt,0pt) -- (0pt,0pt) -- (0pt,5pt) -- cycle;
when you draw this path.

\subsection{Actions on Paths}

A path is just a series of straight and curved lines, but it is not
yet specified what should happen with it. One can \emph{draw} a
path, \emph{fill} a path, \emph{shade} it, \emph{clip} it, or do any
combination of these. Drawing (also known as \emph{stroking}) can be
thought of as taking a pen of a certain thickness and moving it
along the path, thereby drawing on the canvas. Filling means that
the interior of the path is filled with a uniform color. Obviously,
filling makes sense only for \emph{closed} paths and a path is
automatically closed prior to filling, if necessary.

Given a path as in |\path (0,0) rectangle (2ex,1ex);|, you can draw
it by adding the |draw| option as in
|\path[draw] (0,0) rectangle (2ex,1ex);|, which yields \tikz \path[draw]
(0,0) rectangle (2ex,1ex);. The |\draw| command is just an abbreviation for
|\path[draw]|. To fill a path, use the |fill| option or the |\fill|
command, which is an abbreviation for |\path[fill]|. The
|\filldraw| command is an abbreviation for
|\path[fill,draw]|. Shading is caused by the |shade| option (there
are |\shade| and |\shadedraw| abbreviations) and clipping by the
|clip| option. There is also a |\clip| command, which does the
same as |\path[clip]|, but not commands like |\drawclip|. Use, say,
|\draw[clip]| or |\path[draw,clip]| instead.

All of these commands can only be used inside |{tikzpicture}|
environments.

\tikzname\ allows you to use different colors for filling and
stroking.

\subsection{Key-Value Syntax for Graphic Parameters}

Whenever \tikzname\ draws or fills a path, a large number of graphic
parameters influences the rendering. Examples include the colors
used, the dashing pattern, the clipping area, the line width, and
many others. In \tikzname, all these options are specified as lists
of so called key-value pairs, as in |color=red|, that are
passed as optional parameters to the path drawing and filling
commands. This usage is similar to \textsc{pstricks}. For
example, the following will draw a thick, red triangle;
\begin{codeexample}[]
\tikz \draw[line width=2pt,color=red] (1,0) -- (0,0) -- (0,1) -- cycle;
\end{codeexample}

\subsection{Special Syntax for Specifying Nodes}
\tikzname\ introduces a special syntax for adding text or, more
generally, nodes to a graphic. When you specify a path, add nodes as
in the following example:
\begin{codeexample}[]
\tikz \draw (1,1) node {text} -- (2,2);
\end{codeexample}
Nodes are inserted at the current position of
the path, but either \emph{after} (the default) or \emph{before} the
complete path is rendered. When special options are given, as in
|\draw (1,1) node[circle,draw] {text};|, the text is not just put
at the current position. Rather, it is surrounded by a circle and
this circle is ``drawn.''

You can add a name to a node for later reference either by using the
option   |name=|\meta{node name} or by stating the node name in
parentheses outside the text as in |node[circle](name){text}|.

Predefined shapes include |rectangle|, |circle|, and |ellipse|, but
it is possible (though a bit challenging) to define new shapes.

\subsection{Special Syntax for Specifying Trees}

The ``node syntax'' can also be used to draw tress: A |node| can be
followed by any number of children, each introduced by the keyword
|child|. The children are nodes themselves, each of which may have
children in turn. 

\begin{codeexample}[]
\begin{tikzpicture}
  \node {root}
    child {node {left}}
    child {node {right}
      child {node {child}}
      child {node {child}}
    };
\end{tikzpicture}
\end{codeexample}

Since trees are made up from nodes, it is possible to use options to
modify the way trees are drawn. Here are two examples of the above tree,
redrawn with different options:

\begin{codeexample}[]
\begin{tikzpicture}
  [edge from parent fork down, sibling distance=15mm, level distance=15mm,
   every node/.style={fill=red!30,rounded corners},
   edge from parent/.style={red,-o,thick,draw}]
  \node {root}
      child {node {left}}
      child {node {right}
        child {node {child}}
        child {node {child}}
      };
\end{tikzpicture}
\end{codeexample}

\begin{codeexample}[]
\begin{tikzpicture}
  [parent anchor=east,child anchor=west,grow=east,
   sibling distance=15mm, level distance=15mm,
   every node/.style={ball color=red,circle,text=white},
   edge from parent/.style={draw,dashed,thick,red}]
  \node {root}
      child {node {left}}
      child {node {right}
        child {node {child}}
        child {node {child}}
      };
\end{tikzpicture}
\end{codeexample}



\subsection{Special Syntax for Graphs}

The |\node| command gives you fine control over where nodes should be
placed, what text they should use, and what they should look
like. However, when you draw a graph, you typically need to create
numerous fairly similar nodes that only differ with respect to the
name they show. In these cases, the |graph| syntax can be used, which
is another syntax layer build ``on top'' of the node syntax.

\begin{codeexample}[]
\tikz \graph [grow down, branch right] {
  root -> { left, right -> {child, child} }
};
\end{codeexample}
The syntax of the |graph| command extends the so-called
\textsc{dot}-notation used in the popular \textsc{graphviz} program.

Depending on the version of \TeX\ you use (it must allow you to call
Lua code, which is the case for Lua\TeX), you can also ask \tikzname\
to do automatically compute good positions for the nodes of a graph
using one of several integrated \emph{graph drawing algorithms}. 


\subsection{Grouping of Graphic Parameters}

Graphic parameters should often apply to several path drawing or
filling commands. For example, we may wish to draw numerous lines all
with the same line width of 1pt. For this, we put these commands
in a |{scope}| environment that takes the desired graphic options
as an optional parameter. Naturally, the specified graphic
parameters apply only to the drawing and filling commands inside the
environment. Furthermore, nested |{scope}| environments or
individual drawing commands can override the graphic parameters of
outer |{scope}| environments. In the following example, three red
lines, two green lines, and one blue line are drawn:

\begin{codeexample}[]
\begin{tikzpicture}
  \begin{scope}[color=red]
    \draw (0mm,10mm) -- (10mm,10mm);
    \draw (0mm, 8mm) -- (10mm, 8mm);
    \draw (0mm, 6mm) -- (10mm, 6mm);
  \end{scope}
  \begin{scope}[color=green]
    \draw             (0mm, 4mm) -- (10mm, 4mm);
    \draw             (0mm, 2mm) -- (10mm, 2mm);
    \draw[color=blue] (0mm, 0mm) -- (10mm, 0mm);
  \end{scope}
\end{tikzpicture}
\end{codeexample}

The |{tikzpicture}| environment itself also behaves like a
|{scope}| environment, that is, you can specify graphic parameters
using an optional argument. These optional apply to all commands in
the picture.


\subsection{Coordinate Transformation System}

\tikzname\ supports both \pgfname's \emph{coordinate} transformation
system to perform transformations as well as \emph{canvas}
transformations, a more low-level transformation system. (For
details on the difference between coordinate transformations and
canvas transformations see Section~\ref{section-design-transformations}.)

The syntax is set up in such a way that it is harder to use canvas
transformations than coordinate transformations. There are two reasons
for this: First, the canvas transformation must be used with great
care and often results in ``bad'' graphics with changing line width
and text in wrong sizes. Second, \pgfname\ loses track of where nodes
and shapes are positioned when canvas transformations are used.
So, in almost all circumstances, you should use coordinate
transformations rather than canvas transformations.

% Copyright 2003 by Till Tantau <tantau@cs.tu-berlin.de>.
%
% This program can be redistributed and/or modified under the terms
% of the LaTeX Project Public License Distributed from CTAN
% archives in directory macros/latex/base/lppl.txt.


\section[Hierarchical Structures: Package, Environments, Scopes, and Styles]
{Hierarchical Structures:\\
  Package, Environments, Scopes, and Styles}

The present section explains how your files should be structured when
you use \tikzname. On the top level, you need to include the |tikz|
package. In the main text, each graphic needs to be put in a
|{tikzpicture}| environment. Inside these environments, you can use
|{scope}| environments to create internal groups. Inside the scopes
you use |\path| commands to actually draw something. On all levels
(except for the package level), graphic options can be given that
apply to everything within the environment.



\subsection{Loading the Package and the Libraries}

\begin{package}{tikz}
  This package does not have any options.
  
  This will automatically load the \pgfname\ package and some other
  stuff that \tikzname\ needs (like the |xkeyval| package).

  \pgfname\ needs to know what \TeX\ driver you are intending to use. In
  most cases \pgfname\ is clever enough to determine the correct driver
  for you; this is true in particular if you \LaTeX. Currently, the only
  situation where \pgfname\ cannot know the driver ``by itself'' is when
  you use plain \TeX\ or Con\TeX t together with |dvipdfm|. In this case,
  you have to write |\def\pgfsysdriver{pgfsys-dvipdfm.def}|
  \emph{before} you input |tikz.tex|. 
\end{package}


\begin{command}{\usetikzlibrary\marg{list of libraries}}
  Once \tikzname\ has been loaded, you can use this command to load
  further libraries. The list of libraries should contain the names of
  libraries separated by commas. Instead of curly braces, you can also
  use square brackets, which is something Con\TeX t users will
  like. If you try to load a library a second time, nothing will
  happen. 

  \example |\usetikzlibrary{arrows}|

  The above command will load a whole bunch of extra arrow tip
  definitions.

  What this command does is to load the file
  |pgflibrarytikz|\meta{library}|.code.tex| for each \meta{library} in
  the \meta{list of libraries}. Thus, to write your own library file,
  all you need to do is to place a file of the appropriate name
  somewhere where \TeX\ can find it. \LaTeX, plain \TeX, and Con\TeX t
  users can then use your library.
\end{command}



\subsection{Creating a Picture}

\subsubsection{Creating a Picture Using an Environment}

The ``outermost'' scope of \tikzname\ is the |{tikzpicture}| 
environment. You may give drawing commands only inside this
environment, giving them outside (as is possible in many other
packages) will result in chaos.

In \tikzname, the way graphics are rendered is strongly influenced by
graphic options. For example, there is an option for setting the color used
for drawing, another for setting the color used for filling, and also
more obscure ones like the option  for setting the prefix used in the
filenames of temporary files written while plotting functions using an
external program. The graphic options are nearly always specified in a
so-called key-value style. (The ``nearly always'' refers to the name
of nodes, which can also be specified differently.) All graphic
options are local to the |{tikzpicture}| to which they apply.

\begin{environment}{{tikzpicture}\opt{\oarg{options}}}
  All \tikzname\ commands should be given inside this
  environment, except for the |\tikzstyle| command. Unlike other
  packages, it is not possible to use, say, |\pgfpathmoveto| outside
  this environment and doing so will result in chaos. For \tikzname,
  commands like |\path| are only defined inside this environment, so
  there is little chance that you will do something wrong here. 

  When this environment is encountered, the \meta{options} are
  parsed. All options given here will apply to the whole
  picture. 

  Next, the contents of the environment is processed and the graphic
  commands therein are put into a box. Non-graphic text is suppressed
  as well as possible, but non-\pgfname\ commands inside a
  |{tikzpicture}| environment should not produce any ``output'' since
  this may totally scramble the positioning system of the backend
  drivers. The suppressing of normal text, by the way, is done by
  temporarily switching the font to |\nullfont|. You can, however,
  ``escape back'' to normal \TeX\ typesetting. This happens, for
  example, when you specify a node.

  At the end of the environment, \pgfname\ tries to make a good guess
  at a good guess at the bounding box of the graphic and
  then resizes the box such that the box has this size. To ``make its
  guess,'' everytime \pgfname\ encounters a coordinate, it updates the
  bound box's size such that it encompasses all these
  coordinates. This will usually give a good 
  approximation at the bounding box, but will not always be
  accurate. First, the line thickness is not taken into
  account. Second, controls points of a curve often lie far
  ``outside'' the curve and make the bounding box too large. In this
  case, you should use the |[use as bounding box]| option.

  The following option influences the baseline of the resulting
  picture:
  \begin{itemize}
    \itemoption{baseline}\opt{|=|\meta{dimension or coordinate}}
    Normally, the lower end of the picture is put on the baseline of
    the surrounding text. For example, when you give the code
    |\tikz\draw(0,0)circle(.5ex);|, \pgfname\ will find out that the
    lower end of the picture is at $-.5\mathrm{ex}$ and that the upper
    end is at $.5\mathrm{ex}$. Then, the lower end will be put on the
    baseline, resulting in the following: \tikz\draw(0,0)circle(.5ex);.

    Using this option, you can specify that the picture should be
    raised or lowered such that the height \meta{dimension} is on the
    baseline. For example, |tikz[baseline=0pt]\draw(0,0)circle(.5ex);|
    yields \tikz[baseline=0pt]\draw(0,0)circle(.5ex); since, now, the
    baseline is on the height of the $x$-axis. If you omit the
    \meta{dimensions}, |0pt| is assumed as default.

    This options is often useful for ``inlined'' graphics as in
\begin{codeexample}[]
$A \mathbin{\tikz[baseline] \draw[->>] (0pt,.5ex) -- (3ex,.5ex);} B$
\end{codeexample}

    Instead of a \meta{dimension} you can also provide a coordinate in
    parantheses. Then the effect is to put the baseline on the
    $y$-coordinate that the give \meta{coordinate} has \emph{at the
      end of the picture}. This means that, at the end of the picture,
    the \meta{coordinate} is evaluated and then the baseline is set
    to the $y$-coordinate of the resulting point. This makes it easy
    to reference the $y$-coordinate of, say, the base line of nodes.
\begin{codeexample}[]
Hello
\tikz[baseline=(X.base)]
  \node [cross out,draw] (X) {world.};
\end{codeexample}

\begin{codeexample}[]
Top align:
\tikz[baseline=(current bounding box.north)]
  \draw (0,0) rectangle (1cm,1ex);
\end{codeexample}

    \itemoption{execute at begin picture}|=|\meta{code}
    This option can be used to install some code that will be executed
    at the beginning of the picture. This option must be
    given in the argument of the |{tikzpicture}| environment itself
    since this option will not have an effect otherwise. After all,
    the picture has already ``started'' later on.

    This option is mainly used in styles like the |every picture|
    style to execute certain code at the start  of a picture.

    \itemoption{execute at end picture}|=|\meta{code}
    This option installs some code that will be executed
    at the end of the picture. Using this option multiple times will
    cause the code to accumulate. This option must also be given in
    the optional argument of the |{tikzpicture}| environment.

\begin{codeexample}[]
\begin{tikzpicture}[execute at end picture=%
  {
    \begin{pgfonlayer}{background}
      \path[fill=yellow,rounded corners]
        (current bounding box.south west) rectangle
        (current bounding box.north east);
    \end{pgfonlayer}
  }]
  \node at (0,0) {X};
  \node at (2,1) {Y};
\end{tikzpicture}
\end{codeexample}
  \end{itemize}
  
  All options ``end'' at the end of the picture. To set an option
  ``globally'' you can use the following style:
  \begin{itemize}
    \itemstyle{every picture}
    This style is installed at the beginning of each picture.
\begin{codeexample}[code only]
\tikzstyle{every picture}=[semithick]
\end{codeexample}
  \end{itemize}
\end{environment}

In other \TeX\ format, you should use instead the following commands:

\begin{plainenvironment}{{tikzpicture}\opt{\oarg{options}}}
  This is the plain \TeX\ version of the environment.
\end{plainenvironment}

\begin{contextenvironment}{{tikzpicture}\opt{\oarg{options}}}
  This is the Con\TeX t version of the environment.
\end{contextenvironment}


\subsubsection{Creating a Picture Using a Command}

The following two commands are used for ``small'' graphics.

\begin{command}{\tikz\opt{\oarg{options}}\marg{commands}}
  This command places the \meta{commands} inside a
  |{tikzpicture}| environment and adds a semicolon at the end. This is
  just a convenience.

  The \meta{commands} may not contain a paragraph (an empty
  line). This is a precaution to ensure that users really use this
  command only for small graphics.

  \example |\tikz{\draw (0,0) rectangle (2ex,1ex)}| yields
  \tikz{\draw (0,0) rectangle (2ex,1ex);} 
\end{command}


\begin{command}{\tikz\opt{\oarg{options}}\meta{text}|;|}
  If the \meta{text} does not start with an opening brace, the end of
  the \meta{text} is the next semicolon that is encountered.

  \example |\tikz \draw (0,0) rectangle (2ex,1ex);| yields
  \tikz \draw (0,0) rectangle (2ex,1ex);
\end{command}



\subsubsection{Adding a Background}

By default, pictures do not have any background, that is, they are
``transparent'' on all parts on which you do not draw
anything. You may instead wish to have a colored background behind
your picture or a black frame around it or lines above and below it or
some other kind of decoration.

Since backgrounds are often not needed at all, the definition of
styles for adding backgrounds has been put in the library package
|pgflibrarytikzbackgrounds|. This package is documented in
Section~\ref{section-tikz-backgrounds}. 


\subsection{Using Scopes to Structure a Picture}

Inside a |{tikzpicture}| environment you can create scopes
using the |{scope}| environment. This environment is available only
inside the |{tikzpicture}| environment, so once more, there is little
chance of doing anything wrong.

\begin{environment}{{scope}\opt{\oarg{options}}}
  All \meta{options} are local to the \meta{environment
  contents}. Furthermore, the clipping path is also local to the
  environment, that is, any clipping done inside the environment
  ``ends'' at its end.

\begin{codeexample}[]
\begin{tikzpicture}
  \begin{scope}[red]
    \draw (0mm,0mm) -- (10mm,0mm);
    \draw (0mm,1mm) -- (10mm,1mm);
  \end{scope}
  \draw (0mm,2mm) -- (10mm,2mm);
  \begin{scope}[green]
    \draw (0mm,3mm) -- (10mm,3mm);
    \draw (0mm,4mm) -- (10mm,4mm);
    \draw[blue] (0mm,5mm) -- (10mm,5mm);
  \end{scope}
\end{tikzpicture}
\end{codeexample}
  
  The following style influences scopes:
  \begin{itemize}
    \itemstyle{every scope}
    This style is installed at the beginning of every scope. I do not
    know really know what this might be good for, but who knows?
  \end{itemize}

  The following options are useful for scopes:
  \begin{itemize}
    \itemoption{execute at begin scope}|=|\meta{code}
    This option install some code that will be executed
    at the beginning of the scope. This option must be
    given in the argument of the |{scope}| environment.

    The effect applies only to the current scope, not to subscopes.

    \itemoption{execute at end scope}|=|\meta{code}
    This option installs some code that will be executed
    at the end of the  current scope. Using this option multiple times
    will  cause the code to accumulate. This option must also be given
    in the optional argument of the |{scope}| environment. 

    Again, the effect applies only to the current scope, not to subscopes.
  \end{itemize}
\end{environment}

\begin{plainenvironment}{{scope}\opt{\oarg{options}}}
  Plain \TeX\ version of the environment.
\end{plainenvironment}

\begin{contextenvironment}{{scope}\opt{\oarg{options}}}
  Con\TeX t version of the environment.
\end{contextenvironment}



\subsection{Using Scopes Inside Paths}

The |\path| command, which is described in much more detail in later
sections, also takes graphic options. These options are local to the
path. Furthermore, it is possible to create local scopes within a
path simply by using curly braces as in
\begin{codeexample}[]
\tikz \draw (0,0) -- (1,1)
           {[rounded corners] -- (2,0) -- (3,1)}
           -- (3,0) -- (2,1);
\end{codeexample}

Note that many options apply only to the path as a whole and cannot be
scoped in this way. For example, it is not possible to scope the
|color| of the path. See the explanations in the section on paths for
more details.

Finally, certain elements that you specify in the argument to the
|\path| command also take local options. For example, a node
specification takes options. In this case, the options apply only to
the node, not to the surrounding path.



\subsection{Using Styles to Manage How Pictures Look}

There is a way of organizing sets of graphic options ``orthogonally''
to the normal scoping mechanism. For example, you might wish all your
``help lines'' to be drawn in a certain way like, say, gray and thin
(do \emph{not} dash them, that distracts). For this, you can use
\emph{styles}.

A style is simply a set of graphic options that is predefined at some
point. Once a style has been defined, it can be used anywhere using
the |style| option:

\begin{itemize}
  \itemoption{style}|=|\meta{style name}
  invokes all options that are currently set in the \meta{style
    name}. An example of a style is the predefined |help lines| style,
  which you should use for lines in the background like grid lines or
  construction lines. You can easily define new styles and modify
  existing ones.
\begin{codeexample}[]
\begin{tikzpicture}
  \draw                   (0,0) grid +(2,2);
  \draw[style=help lines] (2,0) grid +(2,2);
\end{tikzpicture}
\end{codeexample}
\end{itemize}


\begin{command}{\tikzstyle\meta{style name}\opt{|+|}|=[|\meta{options}|]|}
  This command defines the style \meta{style name}. Whenever it is
  used using the |style=|\meta{style name} command, the \meta{options}
  will be invoked. It is permissible that a style invokes another
  style using the |style=| command inside the \meta{options}, which
  allows you to build hierarchies of styles. Naturally, you should
  \emph{not} create cyclic dependencies.

  If the style already has a predefined meaning, it will
  unceremoniously be redefined without a warning.
\begin{codeexample}[]
\tikzstyle{help lines}=[blue!50,very thin]
\begin{tikzpicture}
  \draw                   (0,0) grid +(2,2);
  \draw[style=help lines] (2,0) grid +(2,2);
\end{tikzpicture}
\end{codeexample}

  If the optional |+| is given, the options are \emph{added} to the
  existing definition:
\begin{codeexample}[]
\tikzstyle{help lines}+=[dashed]% aaarghhh!!!
\begin{tikzpicture}
  \draw                   (0,0) grid +(2,2);
  \draw[style=help lines] (2,0) grid +(2,2);
\end{tikzpicture}
\end{codeexample}
\end{command}

It is also possible to set a style using an option:
\begin{itemize}
  \itemoption{set style}|={|\marg{style name}\opt{|+|}|=[|\meta{options}|]}|
  This option has the same effect as saying |\tikzstyle| before the
  argument of the option. 
\begin{codeexample}[]
\begin{tikzpicture}[set style={{help lines}+=[dashed]}]
  \draw                   (0,0) grid +(2,2);
  \draw[style=help lines] (2,0) grid +(2,2);
\end{tikzpicture}
\end{codeexample}
\end{itemize}



% Copyright 2006 by Till Tantau
%
% This file may be distributed and/or modified
%
% 1. under the LaTeX Project Public License and/or
% 2. under the GNU Free Documentation License.
%
% See the file doc/generic/pgf/licenses/LICENSE for more details.

\section{Specifying Coordinates}


\subsection{Overview}

A \emph{coordinate} is a position on the canvas on which your picture
is drawn. \tikzname\ uses a special syntax for specifying
coordinates. Coordinates are always put in round brackets. The general
syntax is 
\declare{|(|\opt{|[|\meta{options}|]|}\meta{coordinate  specification}|)|}. 

The \meta{coordinate specification} specified coordinates using one of
many different possible \emph{coordinate systems}. Examples are the
Cartesian coordinate system or polar coordinates or spherical
coordinates. No matter which coordinate system is used, in the end, a
specific point on the canvas is represented by the coordinate.

There are two ways of specifying which coordinate system should be used:
\begin{description}
\item[Explicitly] You can specify the coordinate system explicitly. To
  do so, you give the name of the coordinate system at the beginning,
  followed by |cs:|, which stands for ``coordinate system,'' followed
  by a specification of the coordinate using the key-value
  syntax. Thus, the general syntax for \meta{coordinate specification}
  in the explicit case is |(|\meta{coordinate system}| cs:|\meta{list
    of key-value pairs specific to the coordinate system}|)|.
\item[Implicitly] The explicit specification is often too verbose when
  numerous coordinates should be given. Because of this, for the
  coordinate systems that you are likely to use often a special syntax
  is provided. \tikzname\ will notice when you use a coordinate
  specified in a special syntax and will choose the correct coordinate
  system automatically.
\end{description}

Here is an example in which explicit the coordinate systems are
specified explicitly:
\begin{codeexample}[]
\begin{tikzpicture}
  \draw[help lines] (0,0) grid (3,2);
  \draw (canvas cs:x=0cm,y=2mm)
     -- (canvas polar cs:radius=2cm,angle=30);
\end{tikzpicture}
\end{codeexample}
In the next example, the coordinate systems are implicit:
\begin{codeexample}[]
\begin{tikzpicture}
  \draw[help lines] (0,0) grid (3,2);
  \draw (0cm,2mm) -- (30:2cm);
\end{tikzpicture}
\end{codeexample}

It is possible to give options that apply only to a single
coordinate, although this makes sense for transformation options
only. To give transformation options for a single coordinate, give
these options at the beginning in brackets:
\begin{codeexample}[]
\begin{tikzpicture}
  \draw[help lines] (0,0) grid (3,2);
  \draw      (0,0) -- (1,1);
  \draw[red] (0,0) -- ([xshift=3pt] 1,1);
  \draw      (1,0) -- +(30:2cm);
  \draw[red] (1,0) -- +([shift=(135:5pt)] 30:2cm);
\end{tikzpicture}
\end{codeexample}


\subsection{Coordinate Systems}

\subsubsection{Canvas, XYZ, and Polar Coordinate Systems}

Let us start with the basic coordinate systems.

\begin{coordinatesystem}{canvas}
  The simplest way of specifying a coordinate is to use the |canvas|
  coordinate system. You provide a dimension $d_x$ using the |x=|
  option and another dimension $d_y$ using the |y=| option. The position on
  the canvas is located at the position that is $d_x$ to the right and
  $d_y$ above the origin.

  \begin{key}{/tikz/cs/x=\meta{dimension} (initially 0pt)}
    Distance by which the coordinate
    is to the right of the origin. You can also write things like
    |1cm+2pt| since the mathematical engine is used to evaluate the
    \meta{dimension}.
  \end{key}

  \begin{key}{/tikz/cs/y=\meta{dimension} (initially 0pt)}
    Distance by which the coordinate
    is above the origin.
  \end{key}

\begin{codeexample}[]
\begin{tikzpicture}
  \draw[help lines] (0,0) grid (3,2);

  \fill (canvas cs:x=1cm,y=1.5cm)    circle (2pt);
  \fill (canvas cs:x=2cm,y=-5mm+2pt) circle (2pt);
\end{tikzpicture}
\end{codeexample}

  To specify a coordinate in the coordinate system implicitly, you use
  two dimensions that are separated by a comma as in |(0cm,3pt)| or
  |(2cm,\textheight)|. 
\begin{codeexample}[]
\begin{tikzpicture}
  \draw[help lines] (0,0) grid (3,2);

  \fill (1cm,1.5cm)    circle (2pt);
  \fill (2cm,-5mm+2pt) circle (2pt);
\end{tikzpicture}
\end{codeexample}
\end{coordinatesystem}


\begin{coordinatesystem}{xyz}
  The |xyz| coordinate system allows you to specify a point as a
  multiple of three vectors called the $x$-, $y$-, and
  $z$-vectors.  By default, the $x$-vector points 1cm to the right,
  the $y$-vector points 1cm upwards, but this can be changed
  arbitrarily as explained in Section~\ref{section-xyz}. The default
  $z$-vector points to $\bigl(-3.85\textrm{mm},-3.85\textrm{mm}\bigr)$.

  To specify the factors by which the vectors should be multiplied
  before being added, you use the following three options:  
  \begin{key}{/tikz/cs/x=\meta{factor} (initially 0)}
    Factor by which the $x$-vector is multiplied.
  \end{key}
  \begin{key}{/tikz/cs/y=\meta{factor} (initially 0)}
    Works like |x|.
  \end{key}
  \begin{key}{/tikz/cs/z=\meta{factor} (initially 0)}
    Works like |x|.
  \end{key}

\begin{codeexample}[]
\begin{tikzpicture}[->]
  \draw (0,0) -- (xyz cs:x=1);
  \draw (0,0) -- (xyz cs:y=1);
  \draw (0,0) -- (xyz cs:z=1);
\end{tikzpicture}
\end{codeexample}

  This coordinate system can also be selected implicitly. To do so,
  you just provide two or three comma-separated factors (not
  dimensions). 
\begin{codeexample}[]
\begin{tikzpicture}[->]
  \draw (0,0) -- (1,0);
  \draw (0,0) -- (0,1,0);
  \draw (0,0) -- (0,0,1);
\end{tikzpicture}
\end{codeexample}
\end{coordinatesystem}

\emph{Note:} It is possible to use coordinates like |(1,2cm)|, which
are neither |canvas| coordinates nor |xyz| coordinates. The rule is
the following: If a coordinate is of the implicit form
|(|\meta{x}|,|\meta{y}|)|, then \meta{x} and \meta{y} are checked,
independently, whether they have a dimension or whether they are
dimensionless. If both have a dimension, the |canvas| coordinate
system is used. If both lack a dimension, the |xyz| coordinate system
is used. If \meta{x} has a dimension and \meta{y} has not, then the
sum of two coordinate |(|\meta{x}|,0pt)| and |(0,|\meta{y}|)| is
used. If \meta{y} has a dimension and \meta{x} has not, then the sum
of two coordinate |(|\meta{x}|,0)| and |(0pt,|\meta{y}|)| is used.

\emph{Note furthermore:} An expression like |(2+3cm,0)| does
\emph{not} mean the same as |(2cm+3cm,0)|. Instead, if \meta{x} or
\meta{y} internally uses a mixture of dimensions and dimensionless
values, then all dimensionless values are ``upgraded'' to dimensions
by interpreting them as |pt|. So, |2+3cm| is the same dimension as
|2pt+3cm|. 

\begin{coordinatesystem}{canvas polar}
  The |canvas polar| coordinate system allows you to specify
  polar coordinates. You provide an angle using the |angle=| option
  and a radius using the |radius=| option. This yields the point on
  the canvas that is at the given radius distance from the origin at
  the given degree. A degree of zero points to the right, a degree of
  90 upward.
  \begin{key}{/tikz/cs/angle=\meta{degrees}}
    The angle of the coordinate.
    The angle must always be given in degrees and should be between
    $-360$ and $720$.
  \end{key}
  \begin{key}{/tikz/cs/radius=\meta{dimension}}
    The distance from the origin.
  \end{key}
  \begin{key}{/tikz/cs/x radius=\meta{dimension}}
    A polar coordinate is,
    after all, just a point on a circle of the given \meta{radius}. When
    you provide an $x$-radius and also a $y$-radius, you specify an
    ellipse instead of a circle. The |radius| option has the same effect
    as specifying identical |x radius| and |y radius| options.
  \end{key}
  \begin{key}{/tikz/cs/y radius=\meta{dimension}}
    Works like |x radius|.
  \end{key}
\begin{codeexample}[]
\tikz \draw (0,0) -- (canvas polar cs:angle=30,radius=1cm);
\end{codeexample}

  The implicit form for canvas polar coordinates is the following: 
  you specify the angle and the distance, separated by a colon as in
  |(30:1cm)|. 

\begin{codeexample}[]
\tikz \draw    (0cm,0cm) -- (30:1cm) -- (60:1cm) -- (90:1cm)
            -- (120:1cm) -- (150:1cm) -- (180:1cm);
\end{codeexample}

  Two different radii are specified by writing |(30:1cm and 2cm)|.

  For the implicit form, instead of an angle given as a number you can
  also use certain words. For example, |up| is the same as |90|, so
  that you can write |\tikz \draw (0,0) -- (2ex,0pt) -- +(up:1ex);|
  and get \tikz \draw (0,0) -- (2ex,0pt) -- +(up:1ex);. Apart from |up|
  you can use |down|, |left|, |right|, |north|, |south|, |west|, |east|,
  |north east|, |north west|, |south east|, |south west|, all of which
  have their natural meaning.
\end{coordinatesystem}

\begin{coordinatesystem}{xyz polar}
  This coordinate system work similarly to the |canvas polar|
  system. However, the radius and the angle are interpreted in the
  $xy$-coordinate system, not in the canvas system. More detailed,
  consider the circle or ellipse whose half axes are given by the
  current $x$-vector and the current $y$-vector. Then, consider the
  point that lies at a given angle on this ellipse, where an angle of
  zero is the same as the $x$-vector and an angle of 90 is the
  $y$-vector. Finally, multiply the resulting vector by the given
  radius factor. Voil�.
  \begin{key}{/tikz/cs/angle=\meta{degrees}}
    The angle of the coordinate
    interpreted in the ellipse whose axes are the $x$-vector and the
    $y$-vector.
  \end{key}
  \begin{key}{/tikz/cs/radius=\meta{factor}}
    A factor by which the $x$-vector
    and $y$-vector are multiplied prior to forming the ellipse.
  \end{key}
  \begin{key}{/tikz/cs/x radius=\meta{dimension}} A specific factor by
    which only the $x$-vector is multiplied.
  \end{key}
  \begin{key}{/tikz/cs/y radius=\meta{dimension}}
    Works like |x radius|.
  \end{key}
\begin{codeexample}[]
\begin{tikzpicture}[x=1.5cm,y=1cm]
  \draw[help lines] (0cm,0cm) grid (3cm,2cm);

  \draw (0,0) -- (xyz polar cs:angle=0,radius=1);
  \draw (0,0) -- (xyz polar cs:angle=30,radius=1);
  \draw (0,0) -- (xyz polar cs:angle=60,radius=1);
  \draw (0,0) -- (xyz polar cs:angle=90,radius=1);

  \draw (xyz polar cs:angle=0,radius=2)
     -- (xyz polar cs:angle=30,radius=2)
     -- (xyz polar cs:angle=60,radius=2)
     -- (xyz polar cs:angle=90,radius=2);
 \end{tikzpicture}
\end{codeexample}

  The implicit version of this option is the same as the implicit
  version of |canvas polar|, only you do not provide a unit.

\begin{codeexample}[]
\tikz[x={(0cm,1cm)},y={(-1cm,0cm)}]
  \draw  (0,0) -- (30:1) -- (60:1) -- (90:1)
             -- (120:1) -- (150:1) -- (180:1);
\end{codeexample}
\end{coordinatesystem}

\begin{coordinatesystem}{xy polar}
  This is just an alias for |xyz polar|, which some people might
  prefer as there is no z-coordinate involved in the |xyz polar|
  coordinates.   
\end{coordinatesystem}


\subsubsection{Barycentric Systems}
\label{section-barycentric-coordinates}

In the barycentric coordinate system a point is expressed as the
linear combination of multiple vectors. The idea is that you specify
vectors $v_1$, $v_2$, \dots, $v_n$ and numbers $\alpha_1$, $\alpha_2$,
\dots, $\alpha_n$. Then the barycentric coordinate specified by these
vectors and numbers is
\begin{align*}
  \frac{\alpha_1 v_1 + \alpha_2 v_2 + \cdots + \alpha_n v_n}{\alpha_1
    + \alpha_2 + \cdots + \alpha_n}
\end{align*}

The |barycentric cs| allows you to specify such coordinates easily.

\begin{coordinatesystem}{barycentric}
  For this coordinate system, the \meta{coordinate specification}
  should be a comma-separated list of expressions of the form
  \meta{node name}|=|\meta{number}. Note that (currently) the list
  should not contain any spaces before or after the \meta{node name}
  (unlike normal key-value pairs). 

  The specified coordinate is now computed as follows: Each pair
  provides one vector and a number. The vector is the |center| anchor
  of the \meta{node name}. The number is the \meta{number}. Note that
  (currently) you cannot specify a different anchor, so that in order
  to use, say, the |north| anchor of a node you first have to create a
  new coordinate at this north anchor. (Using for instance
  \texttt{\string\coordinate (mynorth) at (mynode.north);}.)

\begin{codeexample}[]
\begin{tikzpicture}
  \coordinate (content)   at (90:3cm);
  \coordinate (structure) at (210:3cm);
  \coordinate (form)      at (-30:3cm);
    
  \node [above]       at (content)   {content oriented};
  \node [below left]  at (structure) {structure oriented};
  \node [below right] at (form)      {form oriented};

  \draw [thick,gray] (content.south) -- (structure.north east) -- (form.north west) -- cycle;

  \small
  \node at (barycentric cs:content=0.5,structure=0.1 ,form=1)    {PostScript};
  \node at (barycentric cs:content=1  ,structure=0   ,form=0.4)  {DVI};
  \node at (barycentric cs:content=0.5,structure=0.5 ,form=1)    {PDF};
  \node at (barycentric cs:content=0  ,structure=0.25,form=1)    {CSS};
  \node at (barycentric cs:content=0.5,structure=1   ,form=0)    {XML};
  \node at (barycentric cs:content=0.5,structure=1   ,form=0.4)  {HTML};
  \node at (barycentric cs:content=1  ,structure=0.2 ,form=0.8)  {\TeX};
  \node at (barycentric cs:content=1  ,structure=0.6 ,form=0.8)  {\LaTeX};
  \node at (barycentric cs:content=0.8,structure=0.8 ,form=1)    {Word};
  \node at (barycentric cs:content=1  ,structure=0.05,form=0.05) {ASCII};
\end{tikzpicture}
\end{codeexample}
\end{coordinatesystem}

\subsubsection{Node Coordinate System}
\label{section-node-coordinates}

In \pgfname\ and in \tikzname\ it is quite easy to define a node that you
wish to reference at a later point. Once you have defined a node,
there are different ways of referencing points of the node. To do so,
you use the following coordinate system:

\begin{coordinatesystem}{node}
  This coordinate system is used to reference a specific point inside
  or on the border of a previously defined node. It can be used in
  different ways, so let us go over them one by one.

  You can use three options to specify which coordinate you mean:
  \begin{key}{/tikz/cs/name=\meta{node name}}
    Specifies the node in which you which to specify a coordinate. The
    \meta{node name} is 
    the name that was previously used to name the node using the
    |name=|\meta{node name} option or the special node name syntax.
  \end{key}
  \begin{key}{/tikz/anchor=\meta{anchor}}
    Specifies an anchor of the node. Here is an example: 
\begin{codeexample}[]
\begin{tikzpicture}
  \node (shape)   at (0,2)  [draw] {|class Shape|};
  \node (rect)    at (-2,0) [draw] {|class Rectangle|};
  \node (circle)  at (2,0)  [draw] {|class Circle|};
  \node (ellipse) at (6,0)  [draw] {|class Ellipse|};

  \draw (node cs:name=circle,anchor=north) |- (0,1);
  \draw (node cs:name=ellipse,anchor=north) |- (0,1);
  \draw[-open triangle 90] (node cs:name=rect,anchor=north)
        |- (0,1) -| (node cs:name=shape,anchor=south);
\end{tikzpicture}
\end{codeexample}
  \end{key}
  \begin{key}{/tikz/cs/angle=\meta{degrees}}
    It is also possible to provide an angle \emph{instead} of an
    anchor. This coordinate refers to a point of the node's
    border where a ray shot from the center
    in the given angle hits the border. Here is an example:
\begin{codeexample}[]
\begin{tikzpicture}
  \node (start) [draw,shape=ellipse] {start};
  \foreach \angle in {-90, -80, ..., 90}
    \draw (node cs:name=start,angle=\angle)
      .. controls +(\angle:1cm) and +(-1,0) .. (2.5,0);
  \end{tikzpicture}
\end{codeexample}
  \end{key}

  It is possible to provide \emph{neither} the |anchor=| option nor
  the |angle=| option. In this case, \tikzname\ will calculate an
  appropriate border position for you. Here is an example: 

\begin{codeexample}[]
\begin{tikzpicture}
  \path (0,0)  node(a) [ellipse,rotate=10,draw] {An ellipse}
        (3,-1) node(b) [circle,draw]            {A circle};
  \draw[thick] (node cs:name=a) -- (node cs:name=b);
\end{tikzpicture}
\end{codeexample}

  \tikzname\ will be reasonably clever at determining the border points that
  you ``mean,'' but, naturally, this may fail in some situations. If
  \tikzname\ fails to determine an appropriate border point, the center will
  be used instead.

  Automatic computation of anchors works only with the line-to operations
  |--|, the vertical/horizontal versions \verb!|-! and \verb!-|!, and
  with the curve-to operation |..|. For other path commands, such as
  |parabola| or |plot|, the center will be used. If this is not desired,
  you should give a named anchor or an angle anchor.
  
  Note that if you use an automatic coordinate for both the start and
  the end of a line-to, as in |--(node cs:name=b)--|, then \emph{two}
  border   coordinates are computed with a move-to between them. This
  is usually   exactly what you want.
  
  If you use relative coordinates together with automatic anchor
  coordinates, the relative coordinates are computed relative to
  the node's center, not relative to the border point. Here is an
  example:

\begin{codeexample}[]
\tikz \draw (0,0) node(x) [draw] {Text}
            rectangle (1,1)
            (node cs:name=x) -- +(1,1);
\end{codeexample}

Similarly, in the following examples both control points are $(1,1)$:

\begin{codeexample}[]
\tikz \draw (0,0) node(x) [draw] {X}
            (2,0) node(y) {Y}
            (node cs:name=x) .. controls +(1,1) and +(-1,1) ..
            (node cs:name=y);
\end{codeexample}

  The implicit way of specifying the node coordinate system is to
  simply use the name of the node in parentheses as in |(a)| or to
  specify a name together with an anchor or an angle separated by a
  dot as in |(a.north)| or |(a.10)|.

  Here is a more complete example:
\begin{codeexample}[]
\begin{tikzpicture}[fill=blue!20]
  \draw[help lines] (-1,-2) grid (6,3);
  \path (0,0)  node(a) [ellipse,rotate=10,draw,fill]    {An ellipse}
        (3,-1) node(b) [circle,draw,fill]               {A circle}
        (2,2)  node(c) [rectangle,rotate=20,draw,fill]  {A rectangle}
        (5,2)  node(d) [rectangle,rotate=-30,draw,fill] {Another rectangle};
  \draw[thick] (a.south) -- (b) -- (c) -- (d);
  \draw[thick,red,->] (a) |- +(1,3) -| (c) |- (b);       
  \draw[thick,blue,<->] (b) .. controls +(right:2cm) and +(down:1cm) .. (d);       
\end{tikzpicture}
\end{codeexample}
\end{coordinatesystem}


% Deprecated:
           
% \subsubsection{Intersection Coordinate Systems}

% Often you wish to specify a point that is on the
% intersection of two lines or shapes. For this, the following
% coordinate system is useful:

% \begin{coordinatesystem}{intersection}
%   First, you must specify two objects that should be
%   intersected. These ``objects'' can either be lines or the shapes of
%   nodes. There are two option to specify the first object:
%   \begin{key}{/tikz/cs/first line={\ttfamily\char`\{}|(|\meta{first
%           coordinate}|)--(|\meta{second coordinate}|)|{\ttfamily\char`\}}}
%     Specifies that the first object is a line that goes from
%     \meta{first coordinate} to meta{second coordinate}.
%   \end{key}
%   Note that you have to write |--| between the coordinate, but this
%   does not mean that anything is added to the path. This is simply a
%   special syntax.
%   \begin{key}{/tikz/cs/first node=\meta{node}}
%     Specifies that the first object is a previously defined node named
%     \meta{node}.
%   \end{key}
  
%   To specify the second object, you use one of the following keys:
%   \begin{key}{/tikz/cs/second line={\ttfamily\char`\{}|(|\meta{first
%           coordinate}|)--(|\meta{second coordinate}|)|{\ttfamily\char`\}}}
%     As above.
%   \end{key}
%   \begin{key}{/tikz/cs/second node=\meta{node}}
%     Specifies that the second object is a previously defined node
%     named \meta{node}.
%   \end{key}

%   Since it is possible that two objects have multiple intersections,
%   you may need to specify which solution you want:
%   \begin{key}{/tikz/cs/solution=\meta{number} (initially 1)}
%     Specifies which solution should be used. Numbering starts with 1.
%   \end{key}
%   The coordinate specified in this way is the \meta{number}th
%   intersection of the two objects.  If the objects do not intersect,
%   an error may occur.

% \begin{codeexample}[]
% \begin{tikzpicture}
%   \draw[help lines] (0,0) grid (3,2);
%   \draw (0,0) coordinate (A) -- (3,2) coordinate (B)
%         (1,2)                -- (3,0);

%   \fill[red] (intersection cs:
%     first line={(A)--(B)},
%     second line={(1,2)--(3,0)}) circle (2pt);
% \end{tikzpicture}
% \end{codeexample}

%   The implicit way of specifying this coordinate system is to write
%   \declare{|(intersection |\opt{\meta{number}}| of |\meta{first
%       object}%
%     | and |\meta{second object}|)|}. Here, \meta{first obejct} either
%   has the form \meta{$p_1$}|--|\meta{$p_2$} or it is just a node
%   name. Likewise for \meta{second object}. Note that there are \emph{no}
%   parentheses around the $p_i$. Thus, you would write
%   |(intersection of A--B and 1,2--3,0)|  for the intersection of the
%   line through the coordinates |A| and |B| and the line through the
%   points $(1,2)$ and $(3,0)$. You would write 
%   |(intersection 2 of c_1 and c_2)| for the second
%   intersection of the node named |c_1| and the node named
%   |c_2|.

%   \tikzname\ needs an explicit algorithm for computing the
%   intersection of two shapes and such an algorithm is available only
%   for few shapes. Currently, the following intersection will be
%   computed correctly:
%   \begin{itemize}
%   \item a line and a line
%   \item a |circle| node and a line (in any order)
%   \item a |circle| and a |circle|
%   \end{itemize}
% \begin{codeexample}[]
% \begin{tikzpicture}[scale=.25]
%   \coordinate [label=-135:$a$] (a) at ($ (0,0)   + (rand,rand) $);
%   \coordinate [label=45:$b$]   (b) at ($ (3,2) + (rand,rand) $);

%   \coordinate [label=-135:$u$] (u) at (-1,1);
%   \coordinate [label=45:$v$]   (v) at (6,0);

%   \draw (a) -- (b)
%         (u) -- (v);

%   \node (c1) at (a) [draw,circle through=(b)] {};
%   \node (c2) at (b) [draw,circle through=(a)] {};

%   \coordinate [label=135:$c$] (c) at (intersection 2 of c1 and c2);
%   \coordinate [label=-45:$d$] (d) at (intersection of u--v and c2);
%   \coordinate [label=135:$e$] (e) at (intersection of u--v and a--b);

%   \foreach \p in {a,b,c,d,e,u,v}
%     \fill [opacity=.5] (\p) circle (8pt);
% \end{tikzpicture}
% \end{codeexample}
% \end{coordinatesystem}

           
\subsubsection{Tangent Coordinate Systems}

\begin{coordinatesystem}{tangent}
  This coordinate system, which is available only when the \tikzname\
  library |calc| is loaded, allows you to compute the point that lies
  tangent to a shape. In detail, consider a \meta{node} and a
  \meta{point}. Now, draw a straight line from the \meta{point} so
  that it ``touches'' the \meta{node} (more formally, so that it is
  \emph{tangent} to this \meta{node}). The point where the line
  touches the shape is the point referred to by the |tangent|
  coordinate system.

  The following options may be given:
  \begin{key}{/tikz/cs/node=\meta{node}}
    This key specifies the node on whose border the tangent should
    lie. 
  \end{key}
  \begin{key}{/tikz/cs/point=\meta{point}}
    This key specifes the point through which the tangent should go.
  \end{key}
  \begin{key}{/tikz/cs/solution=\meta{number}}
    Specifies which solution should be used if there are more than one.
  \end{key}

  A special algorithm is needed in order to compute the tangent for a
  given shape. Currently, tangents can be computed for nodes whose
  shape is one of the following:
  \begin{itemize}
  \item |coordinate|
  \item |circle|
  \end{itemize}

\begin{codeexample}[]
\begin{tikzpicture}
  \draw[help lines] (0,0) grid (3,2);

  \coordinate (a) at (3,2);

  \node [circle,draw] (c) at (1,1) [minimum size=40pt] {$c$};
  
  \draw[red] (a)  -- (tangent cs:node=c,point={(a)},solution=1) --
       (c.center) -- (tangent cs:node=c,point={(a)},solution=2) -- cycle;
\end{tikzpicture}
\end{codeexample}

  There is no implicit syntax for this coordinate system.
\end{coordinatesystem}



\subsubsection{Defining New Coordinate Systems}

While the set of coordinate systems that \tikzname\ can parse via
their special syntax is fixed, it is possible and quite easy to define
new explicitly named coordinate systems. For this, the following
commands are used:

\begin{command}{\tikzdeclarecoordinatesystem\marg{name}\marg{code}}
  This command declares a new coordinate system named \meta{name} that
  can later on be used by writing
  |(|\meta{name}| cs:|\meta{arguments}|)|. When \tikzname\ encounters a coordinate
  specified in this way, the \meta{arguments} are passed to
  \meta{code} as argument |#1|.

  It is now the job of \meta{code} to make sense of the
  \meta{arguments}. At the end of \meta{code}, the two \TeX\ dimensions
  |\pgf@x| and |\pgf@y| should be have the $x$- and $y$-canvas
  coordinate of the coordinate.

  It is not necessary, but customary, to parse \meta{arguments} using
  the key-value syntax. However, you can also parse it in any way you
  like.

  In the following example, a coordinate system |cylindrical| is
  defined.
\begin{codeexample}[]
\makeatletter
\define@key{cylindricalkeys}{angle}{\def\myangle{#1}}    
\define@key{cylindricalkeys}{radius}{\def\myradius{#1}}    
\define@key{cylindricalkeys}{z}{\def\myz{#1}}
\tikzdeclarecoordinatesystem{cylindrical}%
{%
  \setkeys{cylindricalkeys}{#1}%
  \pgfpointadd{\pgfpointxyz{0}{0}{\myz}}{\pgfpointpolarxy{\myangle}{\myradius}}
}
\begin{tikzpicture}[z=0.2pt]
  \draw [->] (0,0,0) -- (0,0,350);
  \foreach \num in {0,10,...,350}
    \fill (cylindrical cs:angle=\num,radius=1,z=\num) circle (1pt);
\end{tikzpicture}
\end{codeexample}
\end{command}

\begin{command}{\tikzaliascoordinatesystem\marg{new name}\marg{old name}}
  Creates an alias of \meta{old name}.  
\end{command}



\subsection{Coordinates at Intersections}
\label{section-intersection-coordinates}

You will wish to compute the intersection of two paths. For the
special and frequent case of two perpendicular lines, a special
coordinate system called |perpendicular| is available. For more
general cases, the |intersection| library can be used.


\subsubsection{Intersections of Perpendicular Lines}

A frequent special case of path intersections is the intersection of a 
vertical line going through a point $p$ and a horizontal line going
through some other point $q$. For this situation there is a useful 
coordinate system.

\begin{coordinatesystem}{perpendicular}
  You can specify the two lines using the following keys:

  \begin{key}{/tikz/cs/horizontal line through={\ttfamily\char`\{}|(|\meta{coordinate}|)|{\ttfamily\char`\}}}
    Specifies that one line is a horizontal line that goes through the
    given coordinate.
  \end{key}
  \begin{key}{/tikz/cs/vertical line through={\ttfamily\char`\{}|(|\meta{coordinate}|)|{\ttfamily\char`\}}}
    Specifies that the other line is vertical and goes through the
    given coordinate.  
  \end{key}

  However, in almost all cases you should, instead, use the implicit
  syntax. Here, you write \declare{|(|\meta{p}\verb! |- !\meta{q}|)|} or
  \declare{|(|\meta{q}\verb! -| !\meta{p}|)|}.

  For example, \verb!(2,1 |- 3,4)! and  \verb!(3,4 -| 2,1)! both yield
  the same as \verb!(2,4)! (provided the $xy$-coordinate system has not
  been modified). 

  The most useful application of the syntax is to draw a line up to some
  point on a vertical or horizontal line. Here is an example:

\begin{codeexample}[]
\begin{tikzpicture}
  \path (30:1cm) node(p1) {$p_1$}   (75:1cm) node(p2) {$p_2$};

  \draw (-0.2,0) -- (1.2,0) node(xline)[right] {$q_1$};
  \draw (2,-0.2) -- (2,1.2) node(yline)[above] {$q_2$};

  \draw[->] (p1) -- (p1 |- xline);
  \draw[->] (p2) -- (p2 |- xline);
  \draw[->] (p1) -- (p1 -| yline);
  \draw[->] (p2) -- (p2 -| yline);
\end{tikzpicture}
\end{codeexample}
\end{coordinatesystem}


\subsubsection{Intersections of Arbitrary Paths}

\begin{tikzlibrary}{intersections}
  This library enables the calculation of intersections of
  two arbitrary paths. However, due to the low accuracy of
  \TeX, the paths should not be ``too complicated''.
  In particular, you should not try to intersect paths consisting 
  lots of very small segments such as plots or decorated paths.
\end{tikzlibrary}

To find the intersections of two paths in \tikzname, they must be
``named''. A ``named path'' is, quite simply, a path that has been 
named using the following key:
  
\begin{keylist}{%
	/tikz/name path=\meta{name},
	/tikz/name path global=\meta{name}}
  The effect of this key is that, after the path has been constructed, 
  just before it is used, it is associated with \meta{name}. For |name path|,
  this association survives beyond the final semi-colon of the path 
  but not the end of the surrounding scope. For |name path global|, the association
  will survive beyond any scope as well. Handle with care.
  
  Any paths created by nodes on the (main) path are ignored, unless
  this key is explicitly used. If the same \meta{name} is used for the
  main path and the node path(s), then the paths will be added
  together and then associated with \meta{name}.
\end{keylist}

To find the intersection of named paths, the following key is used:

\begin{key}{/tikz/name intersections=\marg{options}}
  This key changes the key path to |/tikz/intersection| and processes
  \meta{options}. These options determine, among other things,
  which paths to use for the intersection. Having processed the 
  options, any intersections are then found. A coordinate is created 
  at each intersection, which by default, will be named 
  |intersection-1|, |intersection-2|, and so on. 
  Optionally, the prefix |intersection| can be changed, and the 
  total number of intersections stored in a \TeX-macro. 

\begin{codeexample}[]
\begin{tikzpicture}[every node/.style={opacity=1, black, above left}]
  \draw [help lines] grid (3,2);
  \draw [name path=ellipse] (2,0.5) ellipse (0.75cm and 1cm);
  \draw [name path=rectangle, rotate=10] (0.5,0.5) rectangle +(2,1);
  \fill [red, opacity=0.5, name intersections={of=ellipse and rectangle}]
    (intersection-1) circle (2pt) node {1}
    (intersection-2) circle (2pt) node {2};
\end{tikzpicture}
\end{codeexample}

The following keys can be used in \meta{options}:
  
\begin{key}{/tikz/intersection/of=\meta{name path 1}| and |\meta{name path 2}}
  This key is used to specify the names of the paths to use for
  the intersection.
\end{key}

\begin{key}{/tikz/intersection/name=\meta{prefix} (initially intersection)}
  This key specifies the prefix name for the coodinate nodes placed
  at each intersection.
\end{key}

\begin{key}{/tikz/intersection/total=\meta{macro}}
  This key will mean than the total number of intersections found
  will be stored in \meta{macro}.
\end{key}

\begin{codeexample}[]
\begin{tikzpicture}
  \clip (-2,-2) rectangle (2,2);
  \draw [name path=curve 1] (-2,-1) .. controls (8,-1) and (-8,1) .. (2,1);
  \draw [name path=curve 2] (-1,-2) .. controls (-1,8) and (1,-8) .. (1,2);
  
  \fill [name intersections={of=curve 1 and curve 2, name=i, total=\t}]
        [red, opacity=0.5, every node/.style={above left, black, opacity=1}] 
        \foreach \s in {1,...,\t}{(i-\s) circle (2pt) node {\footnotesize\s}};
\end{tikzpicture}
\end{codeexample}

  
  \begin{key}{/tikz/intersection/by=\meta{comma-separated list}}
    This key allows you to specify a list of names for the intersection
    coordinates. The intersection coordinates will still be named
    \meta{prefix}|-|\meta{number}, but additionally the first
    coordinate will also be named by the first element of the
    \meta{comma-separated list}. What happens is that the
    \meta{comma-separated list} is passed to the |\foreach| statement
    and for \meta{list member} a coordinate is created at the
    already-named intersection.
\begin{codeexample}[]
\begin{tikzpicture}
  \clip (-2,-2) rectangle (2,2);
  \draw [name path=curve 1] (-2,-1) .. controls (8,-1) and (-8,1) .. (2,1);
  \draw [name path=curve 2] (-1,-2) .. controls (-1,8) and (1,-8) .. (1,2);
  
  \fill [name intersections={of=curve 1 and curve 2, by={a,b}}]
        (a) circle (2pt)
        (b) circle (2pt);
\end{tikzpicture}
\end{codeexample}    

    You can also use the |...| notation of the |\foreach| statement
    inside the \meta{comma-separated list}.

    In case an element of the \meta{comma-separated list} starts with
    options in square brackets, these options are used when the
    coordinate is created. A coordinate name can still, but need not,
    follow the  options. This
    makes it easy to add labels to intersections: 
\begin{codeexample}[]
\begin{tikzpicture}
  \clip (-2,-2) rectangle (2,2);
  \draw [name path=curve 1] (-2,-1) .. controls (8,-1) and (-8,1) .. (2,1);
  \draw [name path=curve 2] (-1,-2) .. controls (-1,8) and (1,-8) .. (1,2);
  
  \fill [name intersections={
          of=curve 1 and curve 2,
          by={[label=center:a],[label=center:...],[label=center:i]}}];
\end{tikzpicture}
\end{codeexample}
  \end{key}

  \begin{key}{/tikz/intersection/sort by=\meta{path name}}
By default, the intersections are simply returned in the order that 
the intersection algorithm finds them. Unfortunately, this is not 
necessarily a ``helpful'' ordering. This key can be used to sort
the intersections along the path specified by \meta{path name},
which should be one of the paths mentioned in the 
|/tikz/intersection/of| key.

\begin{codeexample}[]
\begin{tikzpicture}
\clip (-0.5,-0.75) rectangle (3.25,2.25);
\foreach \pathname/\shift in {line/0cm, curve/2cm}{
  \tikzset{xshift=\shift}
  \draw [->, name path=curve] (1,1.5) .. controls (-1,1) and (2,0.5) .. (0,0);
  \draw [->, name path=line]  (0,-.5) -- (1,2) ;
  \fill [name intersections={of=line and curve,sort by=\pathname, name=i}]
    [red, opacity=0.5, every node/.style={left=.25cm, black, opacity=1}]
    \foreach \s in {1,2,3}{(i-\s) circle (2pt) node {\footnotesize\s}};
}
\end{tikzpicture}
\end{codeexample}

  \end{key}
\end{key}




\subsection{Relative and Incremental Coordinates}


\subsubsection{Specifying Relative Coordinates}

You can prefix coordinates by |++| to make them ``relative.'' A
coordinate such as |++(1cm,0pt)| means ``1cm to the right of the
previous position.'' Relative coordinates are often useful in
``local'' contexts:

\begin{codeexample}[]
\begin{tikzpicture}
  \draw (0,0)     -- ++(1,0) -- ++(0,1) -- ++(-1,0) -- cycle;
  \draw (2,0)     -- ++(1,0) -- ++(0,1) -- ++(-1,0) -- cycle;
  \draw (1.5,1.5) -- ++(1,0) -- ++(0,1) -- ++(-1,0) -- cycle;
\end{tikzpicture}
\end{codeexample}

Instead of |++| you can also use a single |+|. This also specifies a
relative coordinate, but it does not ``update'' the current point for
subsequent usages of relative coordinates. Thus, you can use this
notation to specify numerous points, all relative to the same
``initial'' point:

\begin{codeexample}[]
\begin{tikzpicture}
  \draw (0,0)     -- +(1,0) -- +(1,1) -- +(0,1) -- cycle;
  \draw (2,0)     -- +(1,0) -- +(1,1) -- +(0,1) -- cycle;
  \draw (1.5,1.5) -- +(1,0) -- +(1,1) -- +(0,1) -- cycle;
\end{tikzpicture}
\end{codeexample}

There is a special situation, where relative coordinates are
interpreted differently. If you use a relative coordinate as a control
point of a B�zier curve, the following rule applies: First, a relative
first control point is taken relative to the beginning of the
curve. Second, a relative second control point is taken relative to
the end of the curve. Third, a relative end point of a curve is taken
relative to the start of the curve.

This special behavior makes it easy to specify that a curve should
``leave or arrives from a certain direction'' at the start or end. In
the following example, the curve ``leaves'' at $30^\circ$ and
``arrives'' at $60^\circ$: 

\begin{codeexample}[]
\begin{tikzpicture}
  \draw (1,0) .. controls +(30:1cm) and +(60:1cm) .. (3,-1);
  \draw[gray,->] (1,0) -- +(30:1cm);
  \draw[gray,<-] (3,-1) -- +(60:1cm);
\end{tikzpicture}
\end{codeexample}


\subsubsection{Relative Coordinates and Scopes}
\label{section-scopes-relative}
An interesting question is, how do relative coordinates behave in the
presence of scopes? That is, suppose we use curly braces in a path to
make part of it ``local,'' how does that affect the current position?
On the one hand, the current position certainly changes since the
scope only affects options, not the path itself. On the other hand, it
may be useful to ``temporarily escape'' from the updating of the
current point.

Since both interpretations of how the current point and scopes should
``interact'' are useful, there is a (local!) option that allows you to
decide which you need.

\begin{key}{/tikz/current point is local=\opt{\meta{boolean}} (initially
    false)}
  Normally, the scope path operation has no effect on the current
  point. That is, curly braces on a path have no effect on the current
  position:
\begin{codeexample}[]
\begin{tikzpicture}
  \draw      (0,0) -- ++(1,0)   -- ++(0,1)   -- ++(-1,0);
  \draw[red] (2,0) -- ++(1,0) { -- ++(0,1) } -- ++(-1,0);
\end{tikzpicture}
\end{codeexample}
  If you set this key to |true|, this behaviour changes. In this case,
  at the end of a group created on a path, the last current position
  reverts to whatever value it had at the beginning of the scope. More
  precisely, when \tikzname\ encounters |}| on a path, it checks
  whether at this particular moment the key is set to |true|. If so,
  the current position reverts to the value is had when the matching
  |{| was read.
\begin{codeexample}[]
\begin{tikzpicture}
  \draw      (0,0) -- ++(1,0)   -- ++(0,1)   -- ++(-1,0);
  \draw[red] (2,0) -- ++(1,0)
     { [current point is local] -- ++(0,1) } -- ++(-1,0);
\end{tikzpicture}
\end{codeexample}  
  In the above example, we could also have given the option outside
  the scope, for instance as a parameter to the whole scope.
\end{key}


\subsection{Coordinate Calculations}

\begin{tikzlibrary}{calc}
  You need to load this library in order to use the coordinate
  calculation functions described in the present section.
\end{tikzlibrary}


It is possible to do some basic calculations that involve
coordinates. In essence, you can add and subtract coordinates, scale
them, compute midpoints, and do projections. For instance,
|($(a) + 1/3*(1cm,0)$)| is the coordinate that is $1/3$cm to the right
of the point |a|:
\begin{codeexample}[]
\begin{tikzpicture}
  \draw [help lines] (0,0) grid (3,2);

  \node (a) at (1,1) {A};
  \fill [red] ($(a) + 1/3*(1cm,0)$) circle (2pt);
\end{tikzpicture}
\end{codeexample}



\subsubsection{The General Syntax}

The general syntax is the following:

\begin{quote}
  \declare{|(|\opt{|[|\meta{options}|]|}|$|\meta{coordinate computation}|$)|}. 
\end{quote}

As you can see, the syntax uses the \TeX\ math symbol |$| to %$
indicate that a ``mathematical computation'' is involved. However, the |$| %$
has no other effect, in particular, no mathematical text is typeset.

The \meta{coordinate computation} has the following structure:
\begin{enumerate}
\item
  It starts with
  \begin{quote}
    \opt{\meta{factor}|*|}\meta{coordinate}\opt{\meta{modifiers}} 
  \end{quote}
\item
  This is optionally followed by |+| or |-| and then another
  \begin{quote}
    \opt{\meta{factor}|*|}\meta{coordinate}\opt{\meta{modifiers}} 
  \end{quote}
\item
  This is once more followed by |+| or |-| and another of the above
  modified coordinate; and so on.
\end{enumerate}

In the following, the syntax of factors and of the different modifiers
is explained in detail.


\subsubsection{The Syntax of Factors}

The \meta{factor}s are optional and detected
by checking whether the \meta{coordinate computation} starts with a
|(|. Also, after each $\pm$ a \meta{factor} is present if, and only
if, the |+| or |-| sign is not directly followed by~|(|.

If a \meta{factor} is present, it is evaluated using the
|\pgfmathparse| macro. This means that you can use pretty complicated
computations inside a factor. A \meta{factor} may even contain opening
parentheses, which creates a complication: How does \tikzname\ know
where a \meta{factor} ends and where a coordinate starts? For
instance, if the beginning of a \meta{coordinate computation} is
|2*(3+4|\dots, it is not clear whether |3+4| is part of a
\meta{coordinate} or part of a \meta{factor}. Because of this, the
following rule is used: Once it has been determined, that a
\meta{factor} is present, in principle, the \meta{factor} contains
everything up to the next occurrence of |*(|. Note that there is no
space between the asterisk and the parenthesis.

It is permissible to put the \meta{factor} is curly braces. This can
be used whenever it is unclear where the \meta{factor} would end. 

Here are some examples of coordinate specifications that consist of
exactly one \meta{factor} and one \meta{coordinate}:
\begin{codeexample}[]
\begin{tikzpicture}
  \draw [help lines] (0,0) grid (3,2);

  \fill [red] ($2*(1,1)$) circle (2pt);
  \fill [green] (${1+1}*(1,.5)$) circle (2pt);
  \fill [blue] ($cos(0)*sin(90)*(1,1)$) circle (2pt);
  \fill [black] (${3*(4-3)}*(1,0.5)$) circle (2pt);
\end{tikzpicture}
\end{codeexample}



\subsubsection{The Syntax of Partway Modifiers}

A \meta{coordinate} can be followed by different \meta{modifiers}. The
first kind of modifier is the \emph{partway modifier}. The syntax
(which is loosely inspired by Uwe Kern's |xcolor| package) is the
following:
\begin{quote}
  \meta{coordinate}\declare{|!|\meta{number}|!|\opt{\meta{angle}|:|}\meta{second coordinate}}
\end{quote}
One could write for instance
\begin{codeexample}[code only]
(1,2)!.75!(3,4)
\end{codeexample}
The meaning of this is: ``Use the coordinate that is three quarters on
the way from |(1,2)| to |(3,4)|.'' In general, \meta{coordinate
  x}|!|\meta{number}|!|\meta{coordinate y} yields the coordinate
$(1-\meta{number})\meta{coordinate x} + \meta{number} \meta{coordinate
  y}$. Note that this is a bit different from the way the
\meta{number} is interpreted in the |xcolor| package: First, you use a
factor between $0$ and $1$, not a percentage, and, second, as the
\meta{number} approaches $1$, we approach the second coordinate, not
the first. It is permissible to use \meta{numbers} that are smaller
than $0$ or larger than $1$. The \meta{number} is evaluated using the
|\pgfmathparse| command and, thus, it can involve complicated
computations. 

\begin{codeexample}[]
\begin{tikzpicture}
  \draw [help lines] (0,0) grid (3,2);

  \draw (1,0) -- (3,2);
  
  \foreach \i in {0,0.2,0.5,0.9,1}
    \node at ($(1,0)!\i!(3,2)$) {\i};
\end{tikzpicture}
\end{codeexample}

The \meta{second coordinate} may be prefixed by an \meta{angle},
separated with a colon, as in |(1,1)!.5!60:(2,2)|. The general meaning
of \meta{a}|!|\meta{factor}|!|\meta{angle}|:|\meta{b} is ``First,
consider the line from \meta{a} to \meta{b}. Then rotate this line by
\meta{angle} \emph{around the point \meta{a}}. Then the two endpoints
of this line will be \meta{a} and some point \meta{c}. Use this point
\meta{c} for the subsequent computation, namely the partway
computation.''

Here are two examples:
\begin{codeexample}[]
\begin{tikzpicture}
  \draw [help lines] (0,0) grid (3,3);

  \coordinate (a) at (1,0);
  \coordinate (b) at (3,2);

  \draw[->] (a) -- (b);

  \coordinate (c) at ($ (a)!1! 10:(b) $);

  \draw[->,red] (a) -- (c);

  \fill ($ (a)!.5! 10:(b) $) circle (2pt);
\end{tikzpicture}
\end{codeexample}


\begin{codeexample}[]
\begin{tikzpicture}
  \draw [help lines] (0,0) grid (4,4);

  \foreach \i in {0,0.1,...,2}
    \fill ($(2,2) !\i! \i*180:(3,2)$) circle (2pt);
\end{tikzpicture}
\end{codeexample}


You can repeatedly apply modifiers. That is, after any modifier
you can add another (possibly different) modifier.

\begin{codeexample}[]
\begin{tikzpicture}
  \draw [help lines] (0,0) grid (3,2);

  \draw (0,0) -- (3,2);
  \draw[red] ($(0,0)!.3!(3,2)$) -- (3,0);
  \fill[red] ($(0,0)!.3!(3,2)!.7!(3,0)$) circle (2pt);
\end{tikzpicture}
\end{codeexample}


\subsubsection{The Syntax of Distance Modifiers}

A \emph{distance modifier} has nearly the same syntax as a partway
modifier, only you use a \meta{dimension} (something like |1cm|)
instead of a \meta{factor} (something like |0.5|):
\begin{quote}
  \meta{coordinate}\declare{|!|\meta{dimension}|!|\opt{\meta{angle}|:|}\meta{second coordinate}}
\end{quote}

When you write \meta{a}|!|\meta{dimension}|!|\meta{b}, this means the
following: Use the point that is distanced \meta{dimension} from
\meta{a} on the straight line from \meta{a} to \meta{b}. Here is an example:
\begin{codeexample}[]
\begin{tikzpicture}
  \draw [help lines] (0,0) grid (3,2);

  \draw (1,0) -- (3,2);
  
  \foreach \i in {0cm,1cm,15mm}
    \node at ($(1,0)!\i!(3,2)$) {\i};
\end{tikzpicture}
\end{codeexample}

As before, if you use a \meta{angle}, the \meta{second coordinate} is
rotated by this much around the \meta{coordinate} before it is used.

The combination of an \meta{angle} of |90| degrees with a distance can
be used to ``offset'' a point relative to a line. Suppose, for
instance, that you have computed a point |(c)| that lies somewhere on
a line from |(a)| to~|(b)| and you now wish to offset this point by
|1cm| so that the distance from this offset point to the line is
|1cm|. This can be achieved as follows:
\begin{codeexample}[]
\begin{tikzpicture}
  \draw [help lines] (0,0) grid (3,2);

  \coordinate (a) at (1,0);
  \coordinate (b) at (3,1);

  \draw (a) -- (b);

  \coordinate (c) at ($ (a)!.25!(b) $);
  \coordinate (d) at ($ (c)!1cm!90:(b) $);

  \draw [<->] (c) -- (d) node [sloped,midway,above] {1cm};
\end{tikzpicture}
\end{codeexample}



\subsubsection{The Syntax of Projection Modifiers}

The projection modifier is also similar to the above modifiers: It also
gives a point on a line from the \meta{coordinate} to the \meta{second
  coordinate}. However, the \meta{number} or \meta{dimension} is replaced by a
\meta{projection coordinate}:
\begin{quote}
  \meta{coordinate}\declare{|!|\meta{projection coordinate}|!|\opt{\meta{angle}|:|}\meta{second coordinate}}
\end{quote}

Here is an example:
\begin{codeexample}[code only]
(1,2)!(0,5)!(3,4)
\end{codeexample}

The effect is the following: We project the \meta{projection
  coordinate} orthogonally onto to the line from \meta{coordinate} to
\meta{second coordinate}. This makes it easy to compute projected
points: 
\begin{codeexample}[]
\begin{tikzpicture}
  \draw [help lines] (0,0) grid (3,2);

  \coordinate (a) at (0,1);
  \coordinate (b) at (3,2);
  \coordinate (c) at (2.5,0);

  \draw (a) -- (b) -- (c) -- cycle;

  \draw[red]    (a) -- ($(b)!(a)!(c)$);
  \draw[orange] (b) -- ($(a)!(b)!(c)$);
  \draw[blue]   (c) -- ($(a)!(c)!(b)$);
\end{tikzpicture}
\end{codeexample}

% Copyright 2005 by Till Tantau <tantau@cs.tu-berlin.de>.
%
% This program can be redistributed and/or modified under the terms
% of the LaTeX Project Public License Distributed from CTAN
% archives in directory macros/latex/base/lppl.txt.


\section{Syntax for Path Specifications}

A \emph{path} is a series of straight and curved line segments. It is
specified following a |\path| command and the specification must
follow a special syntax, which is described in the subsections of the
present section.


\begin{command}{\path\meta{specification}|;|}
  This command is available only inside a |{tikzpicture}| environment.

  The \meta{specification} is a long stream of \emph{path
  operations}. Most of these path operations tell \tikzname\ how the path
  is build. For example, when you write |--(0,0)|, you use a
  \emph{line-to operation} and it means ``continue the path from
  wherever you are to the origin.''

  At any point where \tikzname\ expects a path operation, you can also
  give some graphic options, which is a list of options in brackets,
  such as |[rounded corners]|. These options can have different
  effects:
  \begin{enumerate}
  \item
    Some options take ``immediate'' effect and apply to all subsequent
    path operations on the path. For example, the |rounded corners|
    option will round all following corners, but not the corners
    ``before'' and if the |sharp corners| is given later on the path
    (in a new set of brackets), the rounding effect will end.

\begin{codeexample}[]
\tikz \draw (0,0) -- (1,1)
           [rounded corners] -- (2,0) -- (3,1)
           [sharp corners] -- (3,0) -- (2,1);
\end{codeexample}
    Another example are the transformation options, which also apply
    only to subsequent coordinates.
  \item
    The options that have immediate effect can be ``scoped'' by
    putting part of a path in curly braces. For example, the above
    example could also be written as follows:

\begin{codeexample}[]
\tikz \draw (0,0) -- (1,1)
           {[rounded corners] -- (2,0) -- (3,1)}
           -- (3,0) -- (2,1);
\end{codeexample}
  \item
    Some options only apply to the path as a whole. For example, the
    |color=| option for determining the color used for, say, drawing
    the path always applies to all parts of the path. If several
    different colors are given for different parts of the path, only
    the last one (on the outermost scope) ``wins'':
 
\begin{codeexample}[]
\tikz \draw (0,0) -- (1,1)
           [color=red] -- (2,0) -- (3,1)
           [color=blue] -- (3,0) -- (2,1);
\end{codeexample}

    Most options are of this type. In the above example, we would have
    had to ``split up'' the path into several |\path| commands:
\begin{codeexample}[]
\tikz{\draw (0,0) -- (1,1);
      \draw [color=red] (2,0) -- (3,1);
      \draw [color=blue] (3,0) -- (2,1);}
\end{codeexample}
  \end{enumerate}

  By default, the |\path| command does ``nothing'' with the
  path, it just ``throws it away.'' Thus, if you write
  |\path(0,0)--(1,1);|, nothing is drawn 
  in your picture. The only effect is that the area occupied by the
  picture is (possibly) enlarged so that the path fits inside the
  area. To actually ``do'' something with the path, an option like
  |draw| or |fill| must be given somewhere on the path. Commands like
  |\draw| do this implicitly.
  
  Finally, it is also possible to give \emph{node specifications} on a
  path. Such specifications can come at different locations, but they
  are always allowed when a normal path operation could follow. A node
  specification starts with |node|. Basically, the effect is to
  typeset the node's text as normal \TeX\ text and to place
  it at the ``current location'' on the path. The details are explained
  in Section~\ref{section-nodes}.

  Note, however, that the nodes are \emph{not} part of the path in any
  way. Rather, after everything has been done with the path what is
  specified by the path options (like filling and drawing the path due
  to a |fill| and a |draw| option somewhere in the
  \meta{specification}), the nodes are added in a post-processing
  step.   
  
  The following style influences scopes:
  \begin{itemize}
    \itemstyle{every path}
    This style is installed at the beginning of every path. This can
    be useful for (temporarily) adding, say, the |draw| option to
    everything in a scope.
\begin{codeexample}[]
\begin{tikzpicture}[fill=examplefill] % only sets the color
  \tikzstyle{every path}=[draw]           % all paths are drawn
  \fill  (0,0) rectangle +(1,1);
  \shade (2,0) rectangle +(1,1);
\end{tikzpicture}
\end{codeexample}
  \end{itemize}
\end{command}




\subsection{The Move-To Operation}

The perhaps simplest operation is the move-to operation, which is
specified by just giving a coordinate where a path operation is
expected.

\begin{pathoperation}[noindex]{}{\meta{coordinate}}
  \index{empty@\protect\meta{empty} path operation}%
  \index{Path operations!empty@\protect\texttt{\meta{empty}}}%
  The move-to operation normally starts a path at a certain
  point. This does not cause a line segment to be created, but it  
  specifies the starting point of the next segment. If a path is
  already under construction, that is, if several segments have
  already been created, a move-to operation will start a new part of the
  path that is not connected to any of the previous segments.

\begin{codeexample}[]
\begin{tikzpicture}
  \draw (0,0) --(2,0) (0,1) --(2,1);
\end{tikzpicture}
\end{codeexample}

  In the specification |(0,0) --(2,0) (0,1) --(2,1)| two move-to
  operations are specified: |(0,0)| and |(0,1)|. The other two
  operations, namely |--(2,0)| and |--(2,1)| are line-to operations,
  described next.
\end{pathoperation}


\subsection{The Line-To Operation}


\subsubsection{Straight Lines}

\begin{pathoperation}{--}{\meta{coordinate}}
  The line-to operation extends the current path from the current
  point in a straight line to the given coordinate. The ``current
  point'' is the endpoint of the previous drawing operation or the point
  specified by a prior move-to operation.

  You use two minus signs followed by a coordinate in round
  brackets. You can add spaces before and after the~|--|.

  When a line-to operation is used and some path segment has just been
  constructed, for example by another line-to operation, the two line
  segments become joined. This means that if they are drawn, the point
  where they meet is ``joined'' smoothly. To appreciate the difference,
  consider the following two examples: In the left example, the path
  consists of two path segments that are not joined, but that happen to
  share a point, while in the right example a smooth join is shown.

\begin{codeexample}[]
\begin{tikzpicture}[line width=10pt]
  \draw (0,0) --(1,1)  (1,1) --(2,0);
  \draw (3,0) -- (4,1) -- (5,0);
  \useasboundingbox (0,1.5); % make bounding box higher
\end{tikzpicture}
\end{codeexample}

\end{pathoperation}


\subsubsection{Horizontal and Vertical Lines}

Sometimes you want to connect two points via straight lines that are
only horizontal and vertical. For this, you can use two path
construction operations.

{\catcode`\|=12
\begin{pathoperation}[noindex]{-|}{\meta{coordinate}}
  \index{--1@\protect\texttt{-\protect\pgfmanualbar} path operation}%
  \index{Path operations!--1@\protect\texttt{-\protect\pgfmanualbar}}%
  This operation means ``first horizontal, then vertical.''

  \begin{codeexample}[]
\begin{tikzpicture}
  \draw (0,0) node(a) [draw] {A}  (1,1) node(b) [draw] {B};
  \draw (a.north) |- (b.west);
  \draw[color=red] (a.east) -| (2,1.5) -| (b.north);
\end{tikzpicture}
\end{codeexample}
\end{pathoperation}
\begin{pathoperation}[noindex]{|-}{\meta{coordinate}}
  \index{--2@\protect\texttt{\protect\pgfmanualbar-} path operation}%
  \index{Path operations!--2@\protect\texttt{\protect\pgfmanualbar-}}%
  This operations means  ``first vertical, then horizontal.''
\end{pathoperation}
}


\subsubsection{Snaked Lines}
\label{section-tikz-snakes}

The line-to operation can not only be used to append straight lines to
the path, but also ``snaked'' lines (called thus because they look a
little bit like snakes seen from above).

\tikzname\ and \pgfname\ use a concept that I termed \emph{snakes}
for appending such ``squiggly'' lines. A snake specifies a way of
extending a path between two points in a ``fancy manner.''

Normally, a snake will just connect the start point to the end point
without starting new subpaths. Thus, a path containing a snaked line
can, nevetheless, still be used for filling. However, this is not
always the case. Some snakes consist of numerous unconnected
segments. ``Lines'' consisting of such snakes cannot be used as the
borders of enclosed areas.

Here are some examples of snakes in action:

\begin{codeexample}[]
\begin{tikzpicture}[thick]
  \draw                                        (0,3)   -- (3,3);
  \draw[snake=zigzag]                          (0,2.5) -- (3,2.5);
  \draw[snake=brace]                           (0,2)   -- (3,2);
  \draw[snake=triangles]                       (0,1.5) -- (3,1.5);
  \draw[snake=coil,segment length=4pt]         (0,1)   -- (3,1);
  \draw[snake=coil,segment aspect=0]           (0,.5)  -- (3,.5);
  \draw[snake=expanding waves,segment angle=7] (0,0)   -- (3,0);
\end{tikzpicture}
\end{codeexample}

\begin{codeexample}[]
\begin{tikzpicture}
  \filldraw[fill=red!20,snake=bumps] (0,0) rectangle (3,2);
\end{tikzpicture}
\end{codeexample}

\begin{codeexample}[]
\begin{tikzpicture}
  \filldraw[fill=blue!20]              (0,3)
  [snake=saw]                       -- (3,3)
  [snake=coil,segment aspect=0]     -- (2,1)
  [snake=bumps]                     -| (0,3);
\end{tikzpicture}
\end{codeexample}

No special path operation is needed to use a snake. Instead, you use
the following option to ``switch on'' snaking:

\begin{itemize}
  \itemoption{snake}\opt{|=|\meta{snake name}}
  This option causes the snake \meta{snake name} to be used for
  subsequent line-to operations. So, whenever you use the |--| syntax
  to specify that a straight line should be added to the path, a snake
  to this path will be added instead. Snakes will also be used when
  you use the \verb!-|! and \verb!|-! syntax and also when you use the
  |rectangle| operation. Snakes will \emph{not} be used when you use
  the curve-to operation nor when any other ``curved'' line is added
  to the path.

  This option has to be given anew for each path. However, you can
  also leave out the \meta{snake name}. In this case, the enclosing
  scope's \meta{snake name} is used. Thus, you can specify a
  ``standard'' snake name for scope and then just say |\draw[snake]|
  every time this snake should actually be used.

  The \meta{snake name} |none| is special. It can be used to switch
  off snaking after it has been switched on on a path.

  A bit strangely, no valid \meta{snake names} are defined by
  \tikzname\ by default. Instead, you have to include the library
  package |pgflibrarysnakes|. This package defines numerous snakes,
  see Section~\ref{section-library-snakes} for the complete list.
\end{itemize}

Most snakes can be configured. For example, for a snake that looks
like a sine curve, you might wish to change the amplitude or the
frequency. There are numerous options that influence these
parameters. Not all options apply to all snakes, see
Section~\ref{section-library-snakes} once more for details.

\begin{itemize}
  \itemoption{gap before snakes}|=|\meta{dimension}
  This option allows you to add a certain ``gap'' to the snake at its
  beginning. The snake will not start at the current point; instead
  the start point of the snake is move be \meta{dimension} in the
  direction of the target.
\begin{codeexample}[]
\begin{tikzpicture}
  \draw[help lines] (0,0) grid (3,2);
  \draw[snake=zigzag]                      (0,1) -- ++(3,1);
  \draw[snake=zigzag,gap before snake=1cm] (0,0) -- ++(3,1);
\end{tikzpicture}
\end{codeexample}
  \itemoption{gap after snake}|=|\meta{dimension}
  This option has the same effect as |gap before snake|, only it
  affects the end of the snake, which will ``end early.''
  \itemoption{gap around snake}|=|\meta{dimension}
  This option sets the gap before and after the gap to
  \meta{dimension}. 
\begin{codeexample}[]
\begin{tikzpicture}
  \draw[help lines] (0,0) grid (3,2);
  \draw[snake=brace]                      (0,1) -- ++(3,1);
  \draw[snake=brace,gap around snake=5mm] (0,0) -- ++(3,1);
\end{tikzpicture}
\end{codeexample}
  \itemoption{line before snake}|=|\meta{dimension}
  This option works like |gap before snake|, only it will connect the
  current point with a straight line to the start of the snake.
\begin{codeexample}[]
\begin{tikzpicture}
  \draw[help lines] (0,0) grid (3,2);
  \draw[snake=zigzag]                       (0,1) -- ++(3,1);
  \draw[snake=zigzag,line before snake=1cm] (0,0) -- ++(3,1);
\end{tikzpicture}
\end{codeexample}
  \itemoption{line after snake}|=|\meta{dimension}
  Works line |gap after snake|, only it adds a straight line.
  \itemoption{line around snake}|=|\meta{dimension}
  Works line |gap around snake|, only it adds straight lines.
  \itemoption{raise snake}|=|\meta{dimension}
  This option can be used with all snakes. It will offset the snake by
  ``raising'' it by \meta{dimension}. A negative \meta{dimension} will
  lower the snake. Raising and lowering is always relative to the line
  along which the snake is drawn. Here is an example:
\begin{codeexample}[]
\begin{tikzpicture}
  \node (a) {A};
  \node (b) at (2,1) {B};
  \draw                                  (a) -- (b);
  \draw[snake=brace]                     (a) -- (b);
  \draw[snake=brace,raise snake=5pt,red] (a) -- (b);
\end{tikzpicture}
\end{codeexample}
  \itemoption{mirror snake}
  This option causes the snake to be ``reflected along the path.''
  This is best understood by looking at an example:
\begin{codeexample}[]
\begin{tikzpicture}
  \node (a) {A};
  \node (b) at (2,1) {B};
  \draw                                     (a) -- (b);
  \draw[snake=brace]                        (a) -- (b);
  \draw[snake=brace,mirror snake,red,thick] (a) -- (b);
\end{tikzpicture}
\end{codeexample}
  This option can be used with every snake and can be combined with
  the |raise snake| option.
  \itemoption{segment amplitude}|=|\meta{dimension}
  This option sets the ``amplitude'' of the snake. For a snake that is
  a sine wave this would be the amplitude of this line. For other
  snakes this value typically describes how far the snakes ``rises
  above'' or ``falls below'' the path. For some snakes, this value is
  ignored. 
\begin{codeexample}[]
\begin{tikzpicture}
  \node (a) {A}   node (b) at (2,1) {B}  node (c) at (2,-1) {C};
  \draw[snake=zigzag]                                 (a) -- (b);
  \draw[snake=zigzag,segment amplitude=5pt,red,thick] (a) -- (c);
\end{tikzpicture}
\end{codeexample}
  \itemoption{segment length}|=|\meta{dimension}
  This option sets the length of each ``segment'' of a snake. For a
  sine wave this would be the wave length, for other snakes it is the
  length of each ``repetitive part'' of the snake.
\begin{codeexample}[]
\begin{tikzpicture}
  \node (a) {A}   node (b) at (2,1) {B}  node (c) at (2,-1) {C};
  \draw[snake=zigzag]                               (a) -- (b);
  \draw[snake=zigzag,segment length=20pt,red,thick] (a) -- (c);
\end{tikzpicture}
\end{codeexample}
\begin{codeexample}[]
\begin{tikzpicture}
  \node (a) {A}   node (b) at (2,1) {B}  node (c) at (2,-1) {C};
  \draw[snake=bumps]                               (a) -- (b);
  \draw[snake=bumps,segment length=20pt,red,thick] (a) -- (c);
\end{tikzpicture}
\end{codeexample}
  \itemoption{segment object length}|=|\meta{dimension}
  This option sets the length of the objects inside each segment of a
  snake. This option is only used for snakes in which each segment
  contains an object like a triangle or a star. 
\begin{codeexample}[]
\begin{tikzpicture}
  \node (a) {A}   node (b) at (2,1) {B}  node (c) at (2,-1) {C};
  \draw[snake=triangles]                                     (a) -- (b);
  \draw[snake=triangles,segment object length=8pt,red,thick] (a) -- (c);
\end{tikzpicture}
\end{codeexample}
  \itemoption{segment angle}|=|\meta{degrees}
  This option sets an angle that is interpreted in a snake-specific
  way. For example, the |waves| and |expanding waves| snakes interpret
  this as (half the) opening angle of the wave. The |border| snake
  uses this value for the angle of the little ticks.
\begin{codeexample}[]
\begin{tikzpicture}[segment amplitude=10pt]
  \node (a) {A}   node (b) at (2,0) {B};
  \draw[snake=border]                            (a) -- (b);
  \draw[snake=border,segment angle=20,red,thick] (a) -- (b);
\end{tikzpicture}
\end{codeexample}
\begin{codeexample}[]
\begin{tikzpicture}[segment amplitude=10pt]
  \node (a)            {A}   node (b)  at (2,0)  {B};
  \node (a') at (0,-1) {A}   node (b') at (2,-1) {B};
  \draw[snake=expanding waves]                            (a)  -- (b);
  \draw[snake=expanding waves,segment angle=20,red,thick] (a') -- (b');
\end{tikzpicture}
\end{codeexample}
  \itemoption{segment aspect}|=|\meta{ratio}
  This option sets an aspect ratio that is interpreted in a
  snake-specific way. For example, for the |coils| snake this
  describes the ``direction'' from which the coil is viewed.
\begin{codeexample}[]
\begin{tikzpicture}[segment amplitude=5pt,segment length=5pt]
  \node (a) {A}   node (b) at (2,1) {B}  node (c) at (2,-1) {C};
  \draw[snake=coil]                            (a) -- (b);
  \draw[snake=coil,segment aspect=0,red,thick] (a) -- (c);
\end{tikzpicture}
\end{codeexample}
\end{itemize}

It is possible to define new snakes, but this cannot be done inside
\tikzname. You need to use the command |\pgfdeclaresnake| from the
basic level directly, see Section~\ref{section-base-snakes}.

The following styles define combinations of segment settings that may
be useful:
\begin{itemize}
  \itemstyle{snake triangles 45}
  Installs a snake the consists of little triangles with an opening
  angle of $45^\circ$.
  \itemstyle{snake triangles 60}
  Installs a snake the consists of little triangles with an opening
  angle of $60^\circ$.
  \itemstyle{snake triangles 90}
  Installs a snake the consists of little triangles with an opening
  angle of $90^\circ$.
\end{itemize}



\subsection{The Curve-To Operation}

The curve-to operation allows you to extend a path using a B�zier
curve.

\begin{pathoperation}{..}{\declare{|controls|}\meta{c}\opt{|and|\meta{d}}\declare{|..|\meta{y}}}
  This operation extends the current path from the current
  point, let us call it $x$, via a curve to a the current point~$y$.
  The curve is a cubic B�zier curve. For such a curve, 
  apart from $y$, you also specify two control points $c$ and $d$. The
  idea is that the curve starts at $x$, ``heading'' in the direction
  of~$c$. Mathematically spoken, the tangent of the curve at $x$ goes
  through $c$. Similarly, the curve ends at $y$, ``coming from'' the
  other control point,~$d$. The larger the distance between $x$ and~$c$
  and between $d$ and~$y$, the larger the curve will be.

  If the ``|and|\meta{d}'' part is not given, $d$ is assumed to be
  equal to $c$.

\begin{codeexample}[]
\begin{tikzpicture}
  \draw[line width=10pt] (0,0) .. controls (1,1) .. (4,0)
                               .. controls (5,0) and (5,1) .. (4,1);
  \draw[color=gray] (0,0) -- (1,1) -- (4,0) -- (5,0) -- (5,1) -- (4,1);
\end{tikzpicture}
\end{codeexample}

  As with the line-to operation, it makes a difference whether two curves
  are joined because they resulted from consecutive curve-to or line-to
  operations, or whether they just happen to have the same ending:

\begin{codeexample}[]
\begin{tikzpicture}[line width=10pt]
  \draw (0,0) -- (1,1) (1,1) .. controls (1,0) and (2,0) .. (2,0);
  \draw (3,0) -- (4,1) .. controls (4,0) and (5,0) .. (5,0);
  \useasboundingbox (0,1.5); % make bounding box higher
\end{tikzpicture}
\end{codeexample}
\end{pathoperation}


\subsection{The Cycle Operation}

\begin{pathoperation}{--cycle}{}
  This operation adds a straight line from the current
  point to the last point specified by a move-to operation. Note that
  this need not be the beginning of the path. Furthermore, a smooth join
  is created between the first segment created after the last move-to
  operation and the straight line appended by the cycle operation.

  Consider the following example. In the left example, two triangles are
  created using three straight lines, but they are not joined at the
  ends. In the second example cycle operations are used.

\begin{codeexample}[]
\begin{tikzpicture}[line width=10pt]
  \draw (0,0) -- (1,1) -- (1,0) -- (0,0) (2,0) -- (3,1) -- (3,0) -- (2,0);
  \draw (5,0) -- (6,1) -- (6,0) -- cycle (7,0) -- (8,1) -- (8,0) -- cycle;
  \useasboundingbox (0,1.5); % make bounding box higher
\end{tikzpicture}
\end{codeexample}
\end{pathoperation}



\subsection{The Rectangle Operation}

A rectangle can obviously be created using four straight lines and a
cycle operation. However, since rectangles are needed so often, a
special syntax is available for them.

\begin{pathoperation}{rectangle}{\meta{corner}}
  When this operation is used, one corner will be the current point,
  another corner is given by \meta{corner}, which becomes the new
  current point.

\begin{codeexample}[]
\begin{tikzpicture}
  \draw (0,0) rectangle (1,1);
  \draw (.5,1) rectangle (2,0.5) (3,0) rectangle (3.5,1.5) -- (2,0);
\end{tikzpicture}
\end{codeexample}
\end{pathoperation}


\subsection{Rounding Corners}

All of the path construction operations mentioned up to now are
influenced by the following option:
\begin{itemize}
  \itemoption{rounded corners}\opt{|=|\meta{inset}}
  When this option is in force, all corners (places where a line is
  continued either via line-to or a curve-to operation) are replaced by
  little arcs so that the corner becomes smooth. 

\begin{codeexample}[]
\tikz \draw [rounded corners] (0,0) -- (1,1)
           -- (2,0) .. controls (3,1) .. (4,0);
\end{codeexample}

  The \meta{inset} describes how big the corner is. Note that the
  \meta{inset} is \emph{not} scaled along if you use a scaling option
  like |scale=2|. 

\begin{codeexample}[]
\begin{tikzpicture}
  \draw[color=gray,very thin] (10pt,15pt) circle (10pt);
  \draw[rounded corners=10pt] (0,0) -- (0pt,25pt) -- (40pt,25pt);
\end{tikzpicture}
\end{codeexample}

  You can switch the rounded corners on and off ``in the middle of
  path'' and different corners in the same path can have different
  corner radii:

\begin{codeexample}[]
\begin{tikzpicture}
  \draw (0,0) [rounded corners=10pt] -- (1,1) -- (2,1)
                     [sharp corners] -- (2,0)
               [rounded corners=5pt] -- cycle;
\end{tikzpicture}
\end{codeexample}

Here is a rectangle with rounded corners:
\begin{codeexample}[]
\tikz \draw[rounded corners=1ex] (0,0) rectangle (20pt,2ex);
\end{codeexample}

  You should be aware, that there are several pitfalls when using this
  option. First, the rounded corner will only be an arc (part of a
  circle) if the angle is $90^\circ$. In other cases, the rounded
  corner will still be round, but ``not as nice.''

  Second, if there are very short line segments in a path, the
  ``rounding'' may cause inadverted effects. In such case it may be
  necessary to temporarily switch off the rounding using
  |sharp corners|. 

  \itemoption{sharp corners}
  This options switches off any rounding on subsequent corners of the
  path.   
\end{itemize}



\subsection{The Circle and Ellipse Operations}

A circle can be approximated well using four B�zier curves. However,
it is difficult to do so correctly. For this reason, a special syntax
is available for adding such an approximation of a circle to the
current path.

\begin{pathoperation}{circle}{|(|\meta{radius}|)|}
  The center of the circle is given by the current point. The new
  current point of the path will remain to be the center of the
  circle.  
\end{pathoperation}

\begin{pathoperation}{ellipse}{|(|\meta{half width}| and |\meta{half height}|)|}
  Note that you can add spaces after |ellipse|, but you have to place
  spaces around |and|.

\begin{codeexample}[]
\begin{tikzpicture}
  \draw (1,0) circle (.5cm);
  \draw (3,0) ellipse (1cm and .5cm) -- ++(3,0) circle (.5cm)
    -- ++(2,-.5) circle (.25cm);
\end{tikzpicture}
\end{codeexample}
\end{pathoperation}


\subsection{The Arc Operation}

The \emph{arc operation} allows you to add an arc to the current
path.
\begin{pathoperation}{arc}{|(|\meta{start angle}|:|\meta{end
    angle}|:|\meta{radius}\opt{| and |\meta{half height}}|)|}
  The arc operation adds a part of a circle of the given radius
  between the given angles. The arc will start at the current point
  and will end at the end of the arc.

  \begin{codeexample}[]
\begin{tikzpicture}
  \draw (0,0) arc (180:90:1cm) -- (2,.5) arc (90:0:1cm);
  \draw (4,0) -- +(30:1cm) arc (30:60:1cm) -- cycle;
  \draw (8,0) arc (0:270:1cm and .5cm) -- cycle;
\end{tikzpicture}
\end{codeexample}

\begin{codeexample}[]
\begin{tikzpicture}
  \draw (-1,0) -- +(3.5,0);
  \draw (1,0) ++(210:2cm) -- +(30:4cm);
  \draw (1,0) +(0:1cm) arc (0:30:1cm);      
  \draw (1,0) +(180:1cm) arc (180:210:1cm);
  \path (1,0) ++(15:.75cm) node{$\alpha$};
  \path (1,0) ++(15:-.75cm) node{$\beta$};
\end{tikzpicture}
\end{codeexample}
\end{pathoperation}


\subsection{The Grid Operation}

You can add a grid to the current path using the |grid| path
operation. 

\begin{pathoperation}{grid}{\opt{\oarg{options}}\meta{corner}}
  This operations adss a grid filling a rectangle whose two corners
  are given by \meta{corner} and by the previous coordinate. Thus, the
  typical way in which a grid is drawn is |\draw (1,1) grid (3,3);|,
  which yields a grid filling the rectangle whose corners are at
  $(1,1)$ and $(3,3)$. All coordinate transformations apply to the grid.

\begin{codeexample}[]
\tikz[rotate=30] \draw[step=1mm] (0,0) grid (2,2);
\end{codeexample}

  The stepping of the grid is governed by the following options:

\begin{itemize}
  \itemoption{step}|=|\meta{dimension} sets the stepping in both the
  $x$ and $y$-direction.
  \itemoption{xstep}|=|\meta{dimension} sets the stepping in the
  $x$-direction. 
  \itemoption{ystep}|=|\meta{dimension} sets the stepping in the
  $y$-direction. 
\end{itemize}

  It is important to note that the grid is always ``phased'' such that
  it contains the point $(0,0)$ if that point happens to be inside the
  rectangle. Thus, the grid does \emph{not} always have an intersection
  at the corner points; this occurs only if the corner points are
  multiples of the stepping. Note that due to rounding errors, the
  ``last'' lines of a grid may be omitted. In this case, you have to
  add an epsilon to the corner points.

  The following style is useful for drawing grids:
\begin{itemize}
  \itemstyle{help lines}
  This style makes lines ``subdued'' by using thin gray lines for
  them. However, this style is not installed automatically and you
  have to say for example:
\begin{codeexample}[]
\tikz \draw[style=help lines] (0,0) grid (3,3);
\end{codeexample}
\end{itemize}
\end{pathoperation}



\subsection{The Parabola Operation}

The |parabola| path operation continues the current path with a
parabola. A parabola is a (shifted and scaled) curve defined by the
equation $f(x) = x^2$ and looks like this: \tikz \draw (-1ex,1.5ex)
parabola[parabola height=-1.5ex] +(2ex,0ex);.

\begin{pathoperation}{parabola}{\opt{\oarg{options}|bend|\meta{bend
        coordinate}}\meta{coordinate}}
  This operation adds a parabola through the current point and the
  given \meta{coordinate}. If the |bend| is given, it specifies where
  the bend should go; the \meta{options} can also be used to specify
  where the bend is. By default, the bend is at the old current point. 

\begin{codeexample}[]
\begin{tikzpicture}
  \draw               (0,0) rectangle                (1,1.5)
                      (0,0) parabola                 (1,1.5);
  \draw[xshift=1.5cm] (0,0) rectangle                (1,1.5)
                      (0,0) parabola[bend at end]    (1,1.5);
  \draw[xshift=3cm]   (0,0) rectangle                (1,1.5)
                      (0,0) parabola bend (.75,1.75) (1,1.5);
\end{tikzpicture}
\end{codeexample}

  The following options influence parabolas:
\begin{itemize}
  \itemoption{bend}|=|\meta{coordinate}
  Has the same effect as saying |bend|\meta{coordinate} outside the
  \meta{options}. The option specifies that the bend of the parabola
  should be at the given \meta{coordinate}. You have to take care
  yourself that the bend position is a ``valid'' position; which means
  that if there is no parabola of the form $f(x) = a x^2 + b x + c$
  that goes through the old current point, the given bend, and the new
  current point, the result will not be a parabola.

  There is one special property of the \meta{coordinate}: When a
  relative coordinate is given like |+(0,0)|, the position relative
  to which this coordinate is ``flexible.'' More precisely, this
  position lies somewhere on a line from the old current point to the
  new current point. The exact position depends on the next
  option.

  \itemoption{bend pos}|=|\meta{fraction}
  Specifies where the ``previous'' point is relative to which the bend
  is calculated. The previous point will be at the \meta{fraction}th
  part of the line from the old current point to the new current
  point.

  The idea is the following: If you say |bend pos=0| and
  |bend +(0,0)|, the bend will be at the old current point. If you say
  |bend pos=1| and |bend +(0,0)|, the bend will be at the new current
  point. If you say |bend pos=0.5| and |bend +(0,2cm)| the bend will
  be 2cm above the middle of the line between the start and end
  point. This is most useful in situations such as the following:
\begin{codeexample}[]
\begin{tikzpicture}
  \draw[help lines] (0,0) grid (3,2);
  \draw (-1,0) parabola[bend pos=0.5] bend +(0,2) +(3,0);
\end{tikzpicture}
\end{codeexample}

  In the above example, the |bend +(0,2)| essentially means ``a
  parabola that is 2cm high'' and |+(3,0)| means ``and 3cm wide.''
  Since this situation arises often, there is a special shortcut
  option:
  \itemoption{parabola height}|=|\meta{dimension} This option has the
  same effect as if you had written the following instead:
  |[bend pos=0.5,bend={+(0pt,|\meta{dimension}|)}]|. 
\begin{codeexample}[]
\begin{tikzpicture}
  \draw[help lines] (0,0) grid (3,2);
  \draw (-1,0) parabola[parabola height=2cm] +(3,0);
\end{tikzpicture}
\end{codeexample}
\end{itemize}

The following styles are useful shortcuts:
\begin{itemize}
  \itemstyle{bend at start} This places the bend at the start of a
  parabola. It is a shortcut for the following options:
  |bend pos=0,bend={+(0,0)}|. 
  \itemstyle{bend at end} This places the bend at the end of a
  parabola.
\end{itemize}
\end{pathoperation}


\subsection{The Sine and Cosine Operation}

The |sin| and |cos| operations are similar to the |parabola|
operation. They, too, can be used to draw (parts of) a sine or cosine
curve.

\begin{pathoperation}{sin}{\meta{coordinate}}
  The effect of |sin| is to draw a scaled and shifted version of a sine
  curve in the interval $[0,\pi/2]$. The scaling and shifting is done in
  such a way that the start of the sine curve in the interval is at the
  old current point and that the end of the curve in the interval is at
  \meta{coordinate}. Here is an example that should clarify this:

\begin{codeexample}[]
\tikz \draw (0,0) rectangle (1,1)     (0,0) sin (1,1)
            (2,0) rectangle +(1.57,1) (2,0) sin +(1.57,1);
\end{codeexample}
\end{pathoperation}

\begin{pathoperation}{cos}{\meta{coordinate}}
  This operation works similarly, only a cosine in the interval
  $[0,\pi/2]$ is drawn. By correctly alternating |sin| and |cos|
  operations, you can create a complete sine or cosine curve:

\begin{codeexample}[]
\begin{tikzpicture}[xscale=1.57]
  \draw (0,0) sin (1,1) cos (2,0) sin (3,-1) cos (4,0) sin (5,1);
  \draw[color=red] (0,1.5) cos (1,0) sin (2,-1.5) cos (3,0) sin (4,1.5) cos (5,0);
\end{tikzpicture}
\end{codeexample}
\end{pathoperation}

Note that there is no way to (conveniently) draw an interval on a sine
or cosine curve whose end points are not multiples of $\pi/2$.



\subsection{The Plot Operation}

The |plot| operation can be used to append a line or curve to the path
that goes through a large number of coordinates. These coordinates are
either given in a simple list of coordinates or they are read from
some file.

The syntax of the |plot| comes in different versions.

\begin{pathoperation}{--plot}{\meta{further arguments}}
  This operation plots the curve through the coordinates specified in
  the \meta{further arguments}. The current (sub)path is simply
  continued, that is, a line-to operation to the first point of the
  curve is implicitly added. The details of the \meta{further
    arguments}  will be explained in a moment.
\end{pathoperation}

\begin{pathoperation}{plot}{\meta{further arguments}}
  This operation plots the curve through the coordinates specified in
  the \meta{further arguments} by first ``moving'' to the first
  coordinate of the curve.
\end{pathoperation}

The \meta{further arguments} are used in three different ways to
specifying the coordinates of the points to be plotted:

\begin{enumerate}
\item
  \opt{|--|}|plot|\oarg{local options}\declare{|coordinates{|\meta{coordinate
    1}\meta{coordinate 2}\dots\meta{coordinate $n$}|}|}
\item
  \opt{|--|}|plot|\oarg{local options}\declare{|file{|\meta{filename}|}|}
\item
  \opt{|--|}|plot|\oarg{local options}\declare{|function{|\meta{gnuplot formula}|}|}
\end{enumerate}

These different ways are explained in the following.


\subsubsection{Plotting Points Given Inline}

In the first two cases, the points are given directly in the \TeX-file
as in the following example:

\begin{codeexample}[]
\tikz \draw plot coordinates {(0,0) (1,1) (2,0) (3,1) (2,1) (10:2cm)};
\end{codeexample}

Here is an example showing the difference between |plot| and |--plot|:

\begin{codeexample}[]
\begin{tikzpicture}
  \draw (0,0) -- (1,1) plot coordinates {(2,0)  (4,0)};
  \draw[color=red,xshift=5cm]
        (0,0) -- (1,1) -- plot coordinates {(2,0)  (4,0)};
\end{tikzpicture}
\end{codeexample}


\subsubsection{Plotting Points Read From an External File}

The second way of specifying points is to put them in an external
file named \meta{filename}. Currently, the only file format that
\tikzname\ allows is the following: Each line of the \meta{filename}
should contain one line starting with two numbers, separated by a
space. Anything following the two numbers on the line is
ignored. Also, lines starting with a |%| or a |#| are ignored as well
as empty lines. (This is exactly the format that \textsc{gnuplot}
produces when you say |set terminal table|.) If necessary, more
formats will be supported in the future, but it is usually easy to
produce a file containing data in this form.

\begin{codeexample}[]
\tikz \draw plot[mark=x,smooth] file {plots/pgfmanual-sine.table};
\end{codeexample}

The file |plots/pgfmanual-sine.table| reads:
\begin{codeexample}[code only]
#Curve 0, 20 points
#x y type
0.00000 0.00000  i
0.52632 0.50235  i
1.05263 0.86873  i
1.57895 0.99997  i
...
9.47368 -0.04889  i
10.00000 -0.54402  i
\end{codeexample}
It was produced from the following source, using |gnuplot|:
\begin{codeexample}[code only]
set terminal table
set output "../plots/pgfmanual-sine.table"
set format "%.5f"
set samples 20
plot [x=0:10] sin(x)
\end{codeexample}

The \meta{local options} of the |plot| operation are local to each
plot and do not affect other plots ``on the same path.'' For example,
|plot[yshift=1cm]| will locally shift the plot 1cm upward. Remember,
however, that most options can only be applied to paths as a
whole. For example, |plot[red]| does not have the effect of making the
plot red. After all, you are trying to ``locally'' make part of the
path red, which is not possible.

\subsubsection{Plotting a Function}
\label{section-tikz-gnuplot}

Often, you will want to plot points that are given via a function like
$f(x) = x \sin x$. Unfortunately, \TeX\ does not really have enough
computational power to generate the points on such a function
efficiently (it is a text processing program, after all). However,
if you allow it, \TeX\ can try to call external programs that can
easily produce the necessary points. Currently, \tikzname\ knows how to
call \textsc{gnuplot}.

When \tikzname\ encounters your operation
|plot[id=|\meta{id}|] function{x*sin(x)}| for 
the first time, it will create a file called
\meta{prefix}\meta{id}|.gnuplot|, where \meta{prefix} is |\jobname.| by
default, that is, the name of you main |.tex| file. If no \meta{id} is
given, it will be empty, which is alright, but it is better when each
plot has a unique \meta{id} for reasons explained in a moment. Next,
\tikzname\ writes some initialization code into this file followed by
|plot x*sin(x)|. The initialization code sets up things 
such that the |plot| operation will write the coordinates into another
file called \meta{prefix}\meta{id}|.table|. Finally, this table file
is read as if you had said |plot file{|\meta{prefix}\meta{id}|.table}|. 

For the plotting mechanism to work, two conditions must be met:
\begin{enumerate}
\item
  You must have allowed \TeX\ to call external programs. This is often
  switched off by default since this is a security risk (you might,
  without knowing, run a \TeX\ file that calls all sorts of ``bad''
  commands). To enable this ``calling external programs'' a command
  line option must be given to the \TeX\ program. Usually, it is
  called something like |shell-escape| or |enable-write18|. For
  example, for my |pdflatex| the option |--shell-escape| can be
  given.
\item
  You must have installed the |gnuplot| program and \TeX\ must find it
  when compiling your file.
\end{enumerate}

Unfortunately, these conditions will not always be met. Especially if
you pass some source to a coauthor and the coauthor does not have
\textsc{gnuplot} installed, he or she will have trouble compiling your
files.

For this reason, \tikzname\ behaves differently when you compile your
graphic for the second time: If upon reaching
|plot[id=|\meta{id}|] function{...}| the file \meta{prefix}\meta{id}|.table|
already exists \emph{and} if the \meta{prefix}\meta{id}|.gnuplot| file
contains what \tikzname\ thinks that it ``should'' contain, the |.table|
file is immediately read without trying to call a |gnuplot|
program. This approach has the following advantages: 
\begin{enumerate}
\item
  If you pass a bundle of your |.tex| file and all |.gnuplot| and
  |.table| files to someone else, that person can \TeX\ the |.tex|
  file without having to have |gnuplot| installed.
\item
  If the |\write18| feature is switched off for security reasons (a
  good idea), then, upon the first compilation of the |.tex| file, the
  |.gnuplot| will still be generated, but not the |.table|
  file. You can then simply call |gnuplot| ``by hand'' for each
  |.gnuplot| file, which will produce all necessary |.table| files.
\item
  If you change the function that you wish to plot or its
  domain, \tikzname\ will automatically try to regenerate the |.table|
  file.
\item
  If, out of laziness, you do not provide an |id|, the same |.gnuplot|
  will be used for different plots, but this is not a problem since
  the |.table| will automatically be regenerated for each plot
  on-the-fly. \emph{Note: If you intend to share your files with
  someone else, always use an id, so that the file can by typeset
  without having \textsc{gnuplot} installed.} Also, having unique ids
  for each plot will improve compilation speed since no external
  programs need to be called, unless it is really necessary.
\end{enumerate}

When you use |plot function{|\meta{gnuplot formula}|}|, the \meta{gnuplot
  formula} must be given in the |gnuplot| syntax, whose details are
beyond the scope of this manual. Here is the ultra-condensed
essence: Use |x| as the variable and use the C-syntax for normal
plots, use |t| as the variable for parametric plots. Here are some examples:

\begin{codeexample}[]
\begin{tikzpicture}[domain=0:4]
  \draw[very thin,color=gray] (-0.1,-1.1) grid (3.9,3.9);
  
  \draw[->] (-0.2,0) -- (4.2,0) node[right] {$x$};
  \draw[->] (0,-1.2) -- (0,4.2) node[above] {$f(x)$};
  
  \draw[color=red]    plot[id=x]   function{x}           node[right] {$f(x) =x$};
  \draw[color=blue]   plot[id=sin] function{sin(x)}      node[right] {$f(x) = \sin x$};
  \draw[color=orange] plot[id=exp] function{0.05*exp(x)} node[right] {$f(x) = \frac{1}{20} \mathrm e^x$};
\end{tikzpicture}
\end{codeexample}


The following options influence the plot:

\begin{itemize}
  \itemoption{samples}|=|\meta{number}
  sets the number of samples used in the plot. The default is 25.
  \itemoption{domain}|=|\meta{start}|:|\meta{end}
  sets the domain between which the samples are taken. The default is
  |-5:5|. 
  \itemoption{parametric}\opt{|=|\meta{true or false}}
  sets whether the plot is a parametric plot. If true, then |t| must
  be used instead of |x| as the parameter and two comma-separated
  functions must be given in the \meta{gnuplot formula}. An example is
  the following:
\begin{codeexample}[]
\tikz \draw[scale=0.5,domain=-3.141:3.141,smooth]
  plot[parametric,id=parametric-example] function{t*sin(t),t*cos(t)};
\end{codeexample}
  
  \itemoption{id}|=|\meta{id}
  sets the identifier of the current plot. This should be a unique
  identifier for each plot (though things will also work if it is not,
  but not as well, see the explanations above). The \meta{id} will be
  part of a filename, so it should not contain anything fancy like |*|
  or |$|.%$
  \itemoption{prefix}|=|\meta{prefix}
  is put before each plot file name. The default is |\jobname.|, but
  if you have many plots, it might be better to use, say |plots/| and
  have all plots placed in a directory. You have to create the
  directory yourself.
  \itemoption{raw gnuplot}
  causes the \meta{gnuplot formula} to be passed on to
  \textsc{gnuplot} without setting up the samples or the |plot|
  operation. Thus, you could write
\begin{codeexample}[code only]
plot[raw gnuplot,id=raw-example] function{set samples 25; plot sin(x)}
\end{codeexample}
  This can be 
  useful for complicated things that need to be passed to
  \textsc{gnuplot}. However, for really complicated situations you
  should create a special external generating \textsc{gnuplot} file
  and use the |file|-syntax to include the table ``by hand.''
\end{itemize}

The following styles influence the plot:
\begin{itemize}
  \itemstyle{every plot}
  This style is installed in each plot, that is, as if you always said
\begin{codeexample}[code only]
  plot[style=every plot,...]
\end{codeexample}
 This is most useful for globally setting a prefix for all plots by saying:
\begin{codeexample}[code only]
\tikzstyle{every plot}=[prefix=plots/]
\end{codeexample}
\end{itemize}



\subsubsection{Placing Marks on the Plot}

As we saw already, it is possible to add \emph{marks} to a plot using
the |mark| option. When this option is used, a copy of the plot
mark is placed on each point of the plot. Note that the marks are
placed \emph{after} the whole path has been drawn/filled/shaded. In
this respect, they are handled like text nodes. 

In detail, the following options govern how marks are drawn:
\begin{itemize}
  \itemoption{mark}|=|\meta{mark mnemonic}
  Sets the mark to a mnemonic that has previously been defined using
  the |\pgfdeclareplotmark|. By default, |*|, |+|, and |x| are available,
  which draw a filled circle, a plus, and a cross as marks. Many more
  marks become available when the library |pgflibraryplotmarks| is
  loaded. Section~\ref{section-plot-marks} lists the available plot
  marks.

  One plot mark is special: the |ball| plot mark is available only
  it \tikzname. The |ball color| determines the balls's color. Do not use
  this option with large number of marks since it will take very long
  to render in PostScript.
  
  \begin{tabular}{lc}
    Option & Effect \\\hline \vrule height14pt width0pt
    \plotmarkentrytikz{ball}
  \end{tabular}
  
  \itemoption{mark size}|=|\meta{dimension}
  Sets the size of the plot marks. For circular plot marks,
  \meta{dimension} is the radius, for other plot marks
  \meta{dimension} should be about half the width and height.

  This option is not really necessary, since you achieve the same
  effect by specifying |scale=|\meta{factor} as a local option, where
  \meta{factor} is the quotient of the desired size and the default
  size. However, using |mark size| is a bit faster and more natural. 

  \itemoption{mark options}|=|\meta{options}
  These options are applied to marks when they are drawn. For example,
  you can scale (or otherwise transform) the plot mark or set its
  color. 
\begin{codeexample}[]
\tikz \fill[fill=blue!20]
  plot[mark=triangle*,mark options={color=blue,rotate=180}]
    file{plots/pgfmanual-sine.table} |- (0,0);
\end{codeexample}
\end{itemize}



\subsubsection{Smooth Plots, Sharp Plots, and Comb Plots}

There are different things the |plot| operation can do with the points
it reads from a file or from the inlined list of points. By default,
it will connect these points by straight lines. However, you can also
use options to change the behavior of |plot|.

\begin{itemize}
  \itemoption{sharp plot}
  This is the default and causes the points to be connected by
  straight lines. This option is included only so that you can
  ``switch back'' if you ``globally'' install, say, |smooth|.
  
  \itemoption{smooth}
  This option causes the points on the path to be connected using a
  smooth curve:

\begin{codeexample}[]
\tikz\draw plot[smooth] file{plots/pgfmanual-sine.table};
\end{codeexample}

  Note that the smoothing algorithm is not very intelligent. You will
  get the best results if the bending angles are small, that is, less
  than about $30^\circ$ and, even more importantly, if the distances
  between points are about the same all over the plotting path.

  \itemoption{tension}|=|\meta{value}
  This option influences how ``tight'' the smoothing is. A lower value
  will result in sharper corners, a higher value in more ``round''
  curves. A value of $1$ results in a circle if four points at
  quarter-positions on a circle are given. The default is $0.55$. The
  ``correct'' value depends on the details of plot.
  
\begin{codeexample}[]
\begin{tikzpicture}[smooth cycle]
  \draw                 plot[tension=0.2]
    coordinates{(0,0) (1,1) (2,0) (1,-1)};
  \draw[yshift=-2.25cm] plot[tension=0.5]
    coordinates{(0,0) (1,1) (2,0) (1,-1)};
  \draw[yshift=-4.5cm]  plot[tension=1]
    coordinates{(0,0) (1,1) (2,0) (1,-1)};
\end{tikzpicture}
\end{codeexample}
  
  \itemoption{smooth cycle}
  This option causes the points on the path to be connected using a
  closed smooth curve. 

\begin{codeexample}[]
\tikz[scale=0.5]
  \draw plot[smooth cycle] coordinates{(0,0) (1,0) (2,1) (1,2)}
        plot               coordinates{(0,0) (1,0) (2,1) (1,2)} -- cycle;
\end{codeexample}

  \itemoption{ycomb}
  This option causes the |plot| operation to interpret the plotting
  points differently. Instead of connecting them, for each point of
  the plot a straight line is added to the path from the $x$-axis to the point,
  resulting in a sort of ``comb'' or ``bar diagram.''

\begin{codeexample}[]
\tikz\draw[ultra thick] plot[ycomb,thin,mark=*] file{plots/pgfmanual-sine.table};
\end{codeexample}

\begin{codeexample}[]
\begin{tikzpicture}[ycomb]
  \draw[color=red,line width=6pt]
    plot coordinates{(0,1) (.5,1.2) (1,.6) (1.5,.7) (2,.9)};
  \draw[color=red!50,line width=4pt,xshift=3pt]
    plot coordinates{(0,1.2) (.5,1.3) (1,.5) (1.5,.2) (2,.5)};
\end{tikzpicture}
\end{codeexample}

  \itemoption{xcomb}
  This option works like |ycomb| except that the bars are horizontal. 

\begin{codeexample}[]
\tikz \draw plot[xcomb,mark=x] coordinates{(1,0) (0.8,0.2) (0.6,0.4) (0.2,1)};
\end{codeexample}

  \itemoption{polar comb}
  This option causes a line from the origin to the point to be added
  to the path for each plot point.

\begin{codeexample}[]
\tikz \draw plot[polar comb,
     mark=pentagon*,mark options={fill=white,draw=red},mark size=4pt]
   coordinates {(0:1cm) (30:1.5cm) (160:.5cm) (250:2cm) (-60:.8cm)};
\end{codeexample}


  \itemoption{only marks}
  This option causes only marks to be shown; no path segments are
  added to the actual path. This can be useful for quickly adding some
  marks to a path.

\begin{codeexample}[]
\tikz \draw (0,0) sin (1,1) cos (2,0)
  plot[only marks,mark=x] coordinates{(0,0) (1,1) (2,0) (3,-1)};
\end{codeexample}
\end{itemize}



  

\subsection{The Scoping Operation}

When \tikzname\ encounters and opening or a closing brace (|{| or~|}|) at
some point where a path operation should come, it will open or close a
scope. All options that can be applied ``locally'' will be scoped
inside the scope. For example, if you apply a transformation like
|[xshift=1cm]| inside the scoped area, the shifting only applies to
the scope. On the other hand, an option like |color=red| does not have
any effect inside a scope since it can only be applied to the path as
a whole. 


\subsection{The Node Operation}

You can add nodes to a path using the |node| operation. Since this
operation is quite complex and since the nodes are not really part of
the path itself, there is a separate section dealing with nodes, see
Section~\ref{section-nodes}. 

% Copyright 2003 by Till Tantau <tantau@cs.tu-berlin.de>.
%
% This program can be redistributed and/or modified under the terms
% of the LaTeX Project Public License Distributed from CTAN
% archives in directory macros/latex/base/lppl.txt.


\section{Actions on Paths}

Once a path has been constructed, different things can be done with
it. It can be drawn (or stroked) with a ``pen,'' it can be filled with
a color or shading, it can be used for clipping subsequent drawing, it
can be used to specify the extend of the picture---or  any
combination of these actions at the same time.

To decide what is to be done with a path, two methods can be
used. First, you can use a special-purpose command like |\draw| to
indicate that the path should be drawn. However, commands like |\draw|
and |\fill| are just abbreviations for special cases of the more
general method: Here, the |\path| command is used to specify the
path. Then, options encountered on the path indicate what should be
done with the path.

For example, |\path (0,0) circle (1cm);| means ``This is a path
consisting of a circle around the origin. Do not do anything with it
(throw it away).'' However, if the option |draw| is encountered
anywhere on the path, the circle will be drawn. ``Anywhere'' is any
point on the path where an option can be given, which is everywhere
where a path command like |circle (1cm)| or |rectangle (1,1)| or even
just |(0,0)| would also be allowed. Thus, the following commands all
draw the same circle:
\begin{codeexample}[code only]
\path [draw] (0,0) circle (1cm);
\path (0,0) [draw] circle (1cm);
\path (0,0) circle (1cm) [draw];
\end{codeexample}
Finally, |\draw (0,0) circle (1cm);| also draws a path, because
|\draw| is an abbreviation for |\path [draw]| and thus the command
expands to the first line of the above example.

Similarly, |\fill| is an abbreviation for |\path[fill]| and
|\filldraw| is an abbreviation for the command
|\path[fill,draw]|. Since options accumulate, the following commands
all have the same effect: 
\begin{codeexample}[code only]
\path [draw,fill]   (0,0) circle (1cm);
\path [draw] [fill] (0,0) circle (1cm);
\path [fill] (0,0) circle (1cm) [draw];
\draw [fill] (0,0) circle (1cm);
\fill (0,0) [draw] circle (1cm);
\filldraw (0,0) circle (1cm);
\end{codeexample}

In the following subsection the different actions are explained that
can be performed on a path. The following commands are abbreviations for
certain sets of actions, but for many useful combinations there are no
abbreviations:

\begin{command}{\draw}
  Inside |{tikzpicture}| this is an abbreviation for |\path[draw]|.
\end{command}

\begin{command}{\fill}
  Inside |{tikzpicture}| this is an abbreviation for |\path[fill]|.
\end{command}

\begin{command}{\filldraw}
  Inside |{tikzpicture}| this is an abbreviation for |\path[fill,draw]|.
\end{command}

\begin{command}{\shade}
  Inside |{tikzpicture}| this is an abbreviation for |\path[shade]|.
\end{command}

\begin{command}{\shadedraw}
  Inside |{tikzpicture}| this is an abbreviation for |\path[shade,draw]|.
\end{command}

\begin{command}{\clip}
  Inside |{tikzpicture}| this is an abbreviation for |\path[clip]|.
\end{command}

\begin{command}{\useasboundingbox}
  Inside |{tikzpicture}| this is an abbreviation for |\path[use as bounding box]|.
\end{command}

\begin{command}{\node}
  Inside |{tikzpicture}| this is an abbreviation for |\path node|. Note
  that, for once, |node| is not an option but a path operation.
\end{command}

\begin{command}{\coordinate}
  Inside |{tikzpicture}| this is an abbreviation for |\path coordinate|.
\end{command}



\subsection{Specifying a Color}

The most unspecific option for setting colors is the following:

\begin{itemize}
  \itemoption{color}|=|\meta{color name}%
  \indexoption{color option}%
  This option sets the color that is used for fill, drawing, and text
  inside the current scope. Any special settings for filling colors or
  drawing colors are immediately ``overruled'' by this option.

  The \meta{color name} is the name of a previously defined color. For
  \LaTeX\ users, this is just a normal ``\LaTeX-color'' and the
  |xcolor| extensions are allows. Here is an example:

\begin{codeexample}[]
\tikz \fill[color=red!20] (0,0) circle (1ex);
\end{codeexample}

  It is possible to ``leave out'' the |color=| part and you can also
  write:
\begin{codeexample}[]
\tikz \fill[red!20] (0,0) circle (1ex);
\end{codeexample}
  What happens is that every option that \tikzname\ does not know, like
  |red!20|, gets a ``second chance'' as a color name.

  For plain \TeX\ users, it is not so easy to specify colors since
  plain \TeX\ has no ``standardized'' color naming
  mechanism. Because of this, \pgfname\ emulates the |xcolor| package,
  though the emulation is \emph{extremely basic} (more precisely, what
  I could hack together in two hours or so). The emulation allows you
  to do the following:
  \begin{itemize}
  \item Specify a new color using |\definecolor|. Only the two color
    models |gray| and |rgb| are supported.
    \example |\definecolor{orange}{rgb}{1,0.5,0}|
  \item Use |\colorlet| to define a new color based on an old
    one. Here, the |!| mechanism is supported, though only ``once''
    (use multiple |\colorlet| for more fancy colors).
    \example |\colorlet{lightgray}{black!25}|
  \item Use |\color|\marg{color name} to set the color in the current
    \TeX\ group. |\aftergroup|-hackery is used to restore the color
    after the group.
  \end{itemize}
\end{itemize}

As pointed out above, the |color=| option applies to ``everything''
(except to shadings), which is not always what you want. Because of
this, there are several more specialized color options. For example,
the |draw=| option sets the color used for drawing, but does not
modify the color used for filling. These color options are documented
where the path action they influence is described.


\subsection{Drawing a Path}

You can draw a path using the following option:
\begin{itemize}
  \itemoption{draw}\opt{|=|\meta{color}}
  Causes the path to be drawn. ``Drawing'' (also known as
  ``stroking'') can be thought of as picking up a pen and moving it
  along the path, thereby leaving ``ink'' on the canvas.

  There are numerous parameters that influence how a line is drawn,
  like the thickness or the dash pattern. These options are explained
  below.

  If the optional \meta{color} argument is given, drawing is done
  using the given \meta{color}. This color can be different from the
  current filling color, which allows you to draw and fill a path with
  different colors. If no \meta{color} argument is given, the last
  usage of the |color=| option is used.

  If the special color name |none| is given, this option causes
  drawing to be ``switched off.'' This is useful if a style has
  previously switched on drawing and you locally wish to undo this
  effect. 

  Although this option is normally used on paths to indicate that the
  path should be drawn, it also makes sense to use the option with a
  |{scope}| or |{tikzpicture}| environment. However, this will
  \emph{not} cause all path to drawn. Instead, this just sets the
  \meta{color} to be used for drawing paths inside the environment.

\begin{codeexample}[]
\begin{tikzpicture}
  \path[draw=red] (0,0) -- (1,1) -- (2,1) circle (10pt);
\end{tikzpicture}
\end{codeexample}
\end{itemize}

The following subsections list the different options that influence
how a path is drawn. All of these options only have an effect if the
|draw| options is given (directly or indirectly).

\subsubsection{Graphic Parameters: Line Width, Line Cap, and Line Join}

\label{section-cap-joins}

\begin{itemize}
  \itemoption{line width}|=|\meta{dimension}
  Specifies the line width. Note the space. Default: |0.4pt|.

\begin{codeexample}[]
  \tikz \draw[line width=5pt] (0,0) -- (1cm,1.5ex);
\end{codeexample}
\end{itemize}

There are a number of predefined styles that provide more ``natural''
ways of setting the line width. You can also redefine these
styles. Remember that you can leave out the |style=| when setting a
style.

\begin{itemize}
  \itemstyle{ultra thin}
  Sets the line width to 0.1pt.
\begin{codeexample}[]
  \tikz \draw[ultra thin] (0,0) -- (1cm,1.5ex);
\end{codeexample}

  \itemstyle{very thin}
  Sets the line width to 0.2pt.
\begin{codeexample}[]
  \tikz \draw[very thin] (0,0) -- (1cm,1.5ex);
\end{codeexample}

  \itemstyle{thin}
  Sets the line width to 0.4pt.
\begin{codeexample}[]
  \tikz \draw[thin] (0,0) -- (1cm,1.5ex);
\end{codeexample}

  \itemstyle{semithick}
  Sets the line width to 0.6pt.
\begin{codeexample}[]
  \tikz \draw[semithick] (0,0) -- (1cm,1.5ex);
\end{codeexample}

  \itemstyle{thick}
  Sets the line width to 0.8pt.
\begin{codeexample}[]
  \tikz \draw[thick] (0,0) -- (1cm,1.5ex);
\end{codeexample}

  \itemstyle{very thick}
  Sets the line width to 1.2pt.
\begin{codeexample}[]
  \tikz \draw[very thick] (0,0) -- (1cm,1.5ex);
\end{codeexample}

  \itemstyle{ultra thick}
  Sets the line width to 1.6pt.
\begin{codeexample}[]
  \tikz \draw[ultra thick] (0,0) -- (1cm,1.5ex);
\end{codeexample}
\end{itemize}

\begin{itemize}
  \itemoption{cap}|=|\meta{type}
  Specifies how lines ``end.'' Permissible \meta{type} are |round|,
  |rect|, and |butt| (default). They have the following effects:

\begin{codeexample}[]
\begin{tikzpicture}
  \begin{scope}[line width=10pt]
    \draw[cap=rect]  (0,0 ) -- (1,0);
    \draw[cap=butt]  (0,.5) -- (1,.5);
    \draw[cap=round] (0,1 ) -- (1,1);
  \end{scope}
  \draw[white,line width=1pt]
    (0,0 ) -- (1,0) (0,.5) -- (1,.5) (0,1 ) -- (1,1);
\end{tikzpicture}
\end{codeexample}

  \itemoption{join}|=|\meta{type}
  Specifies how lines ``join.'' Permissible \meta{type} are |round|,
  |bevel|, and |miter| (default). They have the following effects:

\begin{codeexample}[]
\begin{tikzpicture}[line width=10pt]
  \draw[join=round] (0,0) -- ++(.5,1) -- ++(.5,-1);
  \draw[join=bevel] (1.25,0) -- ++(.5,1) -- ++(.5,-1); 
  \draw[join=miter] (2.5,0) -- ++(.5,1) -- ++(.5,-1); 
  \useasboundingbox (0,1.5); % make bounding box bigger
\end{tikzpicture}
\end{codeexample}

  \itemoption{miter limit}|=|\meta{factor}
  When you use the miter join and there is a very sharp corner (a
  small angle), the miter join may protrude very far over the actual
  joining point. In this case, if it were to protrude by 
  more than \meta{factor} times the line width, the miter join is
  replaced by a bevel join. Default value is |10|.

\begin{codeexample}[]
\begin{tikzpicture}[line width=5pt]
  \draw                 (0,0) -- ++(5,.5) -- ++(-5,.5);
  \draw[miter limit=25] (6,0) -- ++(5,.5) -- ++(-5,.5);
  \useasboundingbox (14,0); % make bounding box bigger
\end{tikzpicture}
\end{codeexample}
\end{itemize}

\subsubsection{Graphic Parameters: Dash Pattern}

\begin{itemize}
  \itemoption{dash pattern}|=|\meta{dash pattern}
  Sets the dashing pattern. The syntax is the same as in
  \textsc{metafont}. For example |on 2pt off 3pt on 4pt off 4pt| means ``draw
  2pt, then leave out 3pt, then draw 4pt once more, then leave out 4pt
  again, repeat''. 

\begin{codeexample}[]
\begin{tikzpicture}[dash pattern=on 2pt off 3pt on 4pt off 4pt]
  \draw (0pt,0pt) -- (3.5cm,0pt);
\end{tikzpicture}
\end{codeexample}

  \itemoption{dash phase}|=|\meta{dash phase}
  Shifts the start of the dash pattern by \meta{phase}.

\begin{codeexample}[]
\begin{tikzpicture}[dash pattern=on 20pt off 10pt]
  \draw[dash phase=0pt] (0pt,3pt) -- (3.5cm,3pt);
  \draw[dash phase=10pt] (0pt,0pt) -- (3.5cm,0pt);
\end{tikzpicture}
\end{codeexample}
\end{itemize}

As for the line thickness, some predefined styles allow you to set the
dashing conveniently.

\begin{itemize}
\itemstyle{solid}
  Shorthand for setting a solid line as ``dash pattern.'' This is the default.

\begin{codeexample}[]
\tikz \draw[solid] (0pt,0pt) -- (50pt,0pt);
\end{codeexample}

  \itemstyle{dotted}
  Shorthand for setting a dotted dash pattern.

\begin{codeexample}[]
\tikz \draw[dotted] (0pt,0pt) -- (50pt,0pt);
\end{codeexample}

  \itemstyle{densely dotted}
  Shorthand for setting a densely dotted dash pattern.

\begin{codeexample}[]
\tikz \draw[densely dotted] (0pt,0pt) -- (50pt,0pt);
\end{codeexample}

  \itemstyle{loosely dotted}
  Shorthand for setting a loosely dotted dash pattern.

\begin{codeexample}[]
\tikz \draw[loosely dotted] (0pt,0pt) -- (50pt,0pt);
\end{codeexample}

  \itemstyle{dashed}
  Shorthand for setting a dashed dash pattern.

\begin{codeexample}[]
\tikz \draw[dashed] (0pt,0pt) -- (50pt,0pt);
\end{codeexample}

  \itemstyle{densely dashed}
  Shorthand for setting a densely dashed dash pattern.

\begin{codeexample}[]
\tikz \draw[densely dashed] (0pt,0pt) -- (50pt,0pt);
\end{codeexample}

  \itemstyle{loosely dashed}
  Shorthand for setting a loosely dashed dash pattern.

\begin{codeexample}[]
\tikz \draw[loosely dashed] (0pt,0pt) -- (50pt,0pt);
\end{codeexample}
\end{itemize}




\subsubsection{Graphic Parameters: Arrow Tips}

When you draw a line, you can add arrow tips at the ends. Currently, it is
only possible to add one arrow tip at the start and one at the end. Thus,
even if the path consists of several segments, only the first and last
segments get arrow tips. In general, it is a good idea to add arrow tips only
to paths that consist of a single, unbroken line. The behavior for
paths that consist of several segments is not specified and may change
in the future.

\begin{itemize}
\itemoption{arrows}\opt{|=|\meta{start arrow kind}|-|\meta{end arrow kind}}
  This option sets the start and end arrow tips (an empty value as in |->|
  indicates that no arrow tip should be drawn at the start).%
  \indexoption{arrows}

  \emph{Note: Since the arrow option is so often used, you can leave
    out the text |arrows=|.} What happens is that every option that
  contains a |-| is interpreted as an arrow specification.

\begin{codeexample}[]
\begin{tikzpicture}
  \draw[->]        (0,0)   -- (1,0);
  \draw[o-stealth] (0,0.3) -- (1,0.3);
\end{tikzpicture}
\end{codeexample}

  The permissible values are all predefined arrow tips, though
  you can also define new arrow tip kinds as explained in
  Section~\ref{section-arrows}. This is often necessary to obtain
  ``double'' arrow tips and arrow tips that have a fixed size. Since
  |pgflibraryarrows| is loaded by default, all arrow tips described in
  Section~\ref{section-library-arrows} are available.

  One arrow tip kind is special: |>| (and all arrow tip kinds containing the
  arrow tip kind such as |<<| or \verb!>|!). This arrow tip type is not  
  fixed. Rather, you can redefine it using the |>=| option, see
  below. 

  \example You can also combine arrow tip types as in
\begin{codeexample}[]
\begin{tikzpicture}[thick]
  \draw[to reversed-to]   (0,0) .. controls +(.5,0) and +(-.5,-.5) .. +(1.5,1);
  \draw[[-latex reversed] (1,0) .. controls +(.5,0) and +(-.5,-.5) .. +(1.5,1);
  \draw[latex-)]          (2,0) .. controls +(.5,0) and +(-.5,-.5) .. +(1.5,1);
  \useasboundingbox (-.1,-.1) rectangle (3.1,1.1); % make bounding box bigger
\end{tikzpicture}
\end{codeexample}

  \itemoption{>}|=|\meta{end arrow kind}
  This option can be used to redefine the ``standard'' arrow tip |>|. The
  idea is that different people have different ideas what arrow tip kind
  should normally be used. I prefer the arrow tip of \TeX's |\to| command
  (which is used in things like $f\colon A \to B$). Other people will
  prefer \LaTeX's standard arrow tip, which looks like this: \tikz
  \draw[-latex] (0,0) -- (10pt,1ex);. Since the arrow tip kind |>| is
  certainly the most ``natural'' one to use, it is kept free of any
  predefined meaning. Instead, you can change it by saying |>=to| to
  set the ``standard'' arrow tip kind to \TeX's arrow tip, whereas |>=latex|
  will set it to \LaTeX's arrow tip and |>=stealth| will use a
  \textsc{pstricks}-like arrow tip.

  Apart from redefining the arrow tip kind |>| (and |<| for the start),
  this option also redefines the following arrow tip kinds: |>| and |<| as
  the swapped version of \meta{end arrow kind}, |<<| and |>>| as
  doubled versions, |>>| and |<<| as swapped doubled versions, %>>
  and \verb!|<! and \verb!>|! as arrow tips ending with a vertical bar.

\begin{codeexample}[]
\begin{tikzpicture}[scale=2]
  \begin{scope}[>=latex]
    \draw[->]    (0pt,6ex) -- (1cm,6ex);
    \draw[>->>]  (0pt,5ex) -- (1cm,5ex);
    \draw[|<->|] (0pt,4ex) -- (1cm,4ex);
  \end{scope}
  \begin{scope}[>=diamond]
    \draw[->]    (0pt,2ex) -- (1cm,2ex);
    \draw[>->>]  (0pt,1ex) -- (1cm,1ex);
    \draw[|<->|] (0pt,0ex) -- (1cm,0ex);
  \end{scope} 
\end{tikzpicture}
\end{codeexample}

  \itemoption{shorten >}|=|\meta{dimension}
  This option will shorten the end of lines by the given
  \meta{dimension}. If you specify an arrow tip, lines are already
  shortened a bit such that the arrow tip touches the specified endpoint
  and does not ``protrude over'' this point. Here is an example:

\begin{codeexample}[]
\begin{tikzpicture}[line width=20pt]
  \useasboundingbox (0,-1.5) rectangle (3.5,1.5);
  \draw[red]        (0,0) -- (3,0);
  \draw[gray,->]    (0,0) -- (3,0);
\end{tikzpicture}
\end{codeexample}

  The |shorten >| option allows you to shorten the end on the line
  \emph{additionally} by the given distance. This option can also be
  useful if you have not specified an arrow tip at all.

\begin{codeexample}[]
\begin{tikzpicture}[line width=20pt]
  \useasboundingbox (0,-1.5) rectangle (3.5,1.5);
  \draw[red]                     (0,0) -- (3,0);
  \draw[-to,shorten >=10pt,gray] (0,0) -- (3,0);
\end{tikzpicture}
\end{codeexample}

  \itemoption{shorten <}|=|\meta{dimension} works like |shorten >|,
  but for the start.
\end{itemize}



\subsubsection{Graphic Parameters: Double Lines and Bordered Lines}

\begin{itemize}
  \itemoption{double}\opt{|=|\meta{core color}}
  This option causes ``two'' lines to be drawn instead of a single
  one. However, this is not what really happens. In reality, the path
  is drawn twice. First, with the normal drawing color, secondly with
  the \meta{core color}, which is normally |white|. Upon the second
  drawing, the line width is reduced. The net effect is that it
  appears as if two lines had been drawn and this works well even with
  complicated, curved paths:

\begin{codeexample}[]
\tikz \draw[double]
  plot[smooth cycle] coordinates{(0,0) (1,1) (1,0) (0,1)};
\end{codeexample}

  You can also use the doubling option to create an effect in which a
  line seems to have a certain ``border'':

\begin{codeexample}[]
\begin{tikzpicture}
  \draw (0,0) -- (1,1);
  \draw[draw=white,double=red,very thick] (0,1) -- (1,0);
\end{tikzpicture}
\end{codeexample}

  \itemoption{double distance}|=|\meta{dimension}
  Sets the distance the ``two'' lines are spaced apart (default is
  0.6pt). In reality, this is the thickness of the line that is used
  to draw the path for the second time. The thickness of the
  \emph{first} time the path is drawn is twice the normal line width
  plus the given \meta{dimension}. As a side-effect, this option
  ``selects'' the |double| option.

\begin{codeexample}[]
\begin{tikzpicture}
  \draw[very thick,double]              (0,0) arc (180:90:1cm);
  \draw[very thick,double distance=2pt] (1,0) arc (180:90:1cm);
  \draw[thin,double distance=2pt]       (2,0) arc (180:90:1cm);
\end{tikzpicture}
\end{codeexample}
\end{itemize}


  




\subsection{Filling a Path}
\label{section-rules}
To fill a path, use the following option:
\begin{itemize}
  \itemoption{fill}\opt{|=|\meta{color}}
  This option causes the path to be filled. All unclosed parts of the
  path are first closed, if necessary. Then, the area enclosed by the
  path is filled with the current filling color, which is either the
  last color set using the general |color=| option or the optional
  color \meta{color}. For self-intersection paths and for paths
  consisting of several closed areas, the ``enclosed area'' is
  somewhat complicated to define and two different definitions exist,
  namely the nonzero winding number rule and the even odd rule, see
  the explanation of these options, below.

  Just as for the |draw| option, setting \meta{color} to |none|
  disables filling locally.

\begin{codeexample}[]
\begin{tikzpicture}
  \fill (0,0) -- (1,1) -- (2,1);
  \fill (4,0) circle (.5cm)  (4.5,0) circle (.5cm);
  \fill[even odd rule] (6,0) circle (.5cm)  (6.5,0) circle (.5cm);
  \fill (8,0) -- (9,1) -- (10,0) circle (.5cm);
\end{tikzpicture}
\end{codeexample}

  If the |fill| option is used together with the |draw| option (either
  because both are given as options or because a |\filldraw| command
  is used), the path is filled \emph{first}, then the path is drawn
  \emph{second}. This is especially useful if different colors are
  selected for drawing and for filling. Even if the same color is
  used, there is a difference between this command and a plain 
  |fill|: A ``filldrawn'' area will be slightly larger than a filled
  area because of the thickness of the ``pen.''

\begin{codeexample}[]
\begin{tikzpicture}[fill=examplefill,line width=5pt]
  \filldraw (0,0) -- (1,1) -- (2,1);
  \filldraw (4,0) circle (.5cm)  (4.5,0) circle (.5cm);
  \filldraw[even odd rule] (6,0) circle (.5cm)  (6.5,0) circle (.5cm);
  \filldraw (8,0) -- (9,1) -- (10,0) circle (.5cm);
\end{tikzpicture}
\end{codeexample}
\end{itemize}

The following two options can be used to decide how interior points
should be determined:
\begin{itemize}
  \itemoption{nonzero rule}
  If this rule is used (which is the default), the following method is
  used to determine whether a given point is ``inside'' the path: From
  the point, shoot a ray in some direction towards infinity (the
  direction is chosen such that no strange borderline cases
  occur). Then the ray may hit the path. Whenever it hits the path, we
  increase or decrease a counter, which is initially zero. If the ray
  hits the path as the path goes ``from left to right'' (relative to
  the ray), the counter is increased, otherwise it is decreased. Then,
  at the end, we check whether the counter is nonzero (hence the
  name). If so, the point is deemed to lie ``inside,'' otherwise it is
  ``outside.'' Sounds complicated? It is.

\begin{codeexample}[]
\begin{tikzpicture}
  \filldraw[fill=examplefill]
  % Clockwise rectangle
  (0,0) -- (0,1) -- (1,1) -- (1,0) -- cycle
  % Counter-clockwise rectangle
  (0.25,0.25) -- (0.75,0.25) -- (0.75,0.75) -- (0.25,0.75) -- cycle;

  \draw[->] (0,1) (.4,1);
  \draw[->] (0.75,0.75) (0.3,.75);

  \draw[->] (0.5,0.5) -- +(0,1) node[above] {crossings: $-1+1 = 0$};

  \begin{scope}[yshift=-3cm]
    \filldraw[fill=examplefill]
    % Clockwise rectangle
    (0,0) -- (0,1) -- (1,1) -- (1,0) -- cycle
    % Clockwise rectangle
    (0.25,0.25) -- (0.25,0.75) -- (0.75,0.75) -- (0.75,0.25) -- cycle;

    \draw[->] (0,1) (.4,1);
    \draw[->] (0.25,0.75) (0.4,.75);
      
    \draw[->] (0.5,0.5) -- +(0,1) node[above] {crossings: $1+1 = 2$};
  \end{scope}
\end{tikzpicture}
\end{codeexample}

\itemoption{even odd rule}
  This option causes a different method to be used for determining the
  inside and outside of paths. While it is less flexible, it turns out
  to be more intuitive.

  With this method, we also shoot rays from the point for which we
  wish to determine whether it is inside or outside the filling
  area. However, this time we only count how often we ``hit'' the path
  and declare the point to be ``inside'' if the number of hits is odd.

  Using the even-odd rule, it is easy to ``drill holes'' into a path.
  
\begin{codeexample}[]
\begin{tikzpicture}
  \filldraw[fill=examplefill,even odd rule]
    (0,0) rectangle (1,1) (0.5,0.5) circle (0.4cm);
  \draw[->] (0.5,0.5) -- +(0,1) [above] node{crossings: $1+1 = 2$};
\end{tikzpicture}
\end{codeexample}
\end{itemize}




\subsection{Shading a Path}

You can shade a path using the |shade| option. A shading is like a
filling, only the shading changes its color smoothly from one color to
another.

\begin{itemize}
  \itemoption{shade}
  Causes the path to be shaded using the currently selected shading
  (more on this later). If this option is used together with the
  |draw| option, then the path is first shaded, then drawn.

  It is not an error to use this option together with the |fill|
  option, but it makes no sense.

\begin{codeexample}[]
\tikz \shade (0,0) circle (1ex);
\end{codeexample}

\begin{codeexample}[]
\tikz \shadedraw (0,0) circle (1ex);
\end{codeexample}
\end{itemize}

For some shadings it is not really clear how they can ``fill'' the
path. For example, the |ball| shading normally looks like this: \tikz
\shade[shading=ball] (0,0) circle (0.75ex);. How is this supposed to
shade a rectangle? Or a triangle?

To solve this problem, the predefined shadings like |ball| or |axis|
fill a large rectangle completely in a sensible way. Then, when the
shading is used to ``shade'' a path, what actually happens is that the
path is temporarily used for clipping and then the rectangular shading
is drawn, scaled and shifted such that all parts of the path are
filled.


\subsubsection{Choosing a Shading Type}

The default shading is a smooth transition from gray
to white and from above to bottom. However, other shadings are also
possible, for example a shading that will sweep a color from the
center to the corners outward. To choose the shading, you can use the
|shading=| option, which will also automatically invoke the |shade|
option. Note that this does \emph{not} change the shading color, only
the way the colors sweep. For changing the colors, other options are
needed, which are explained below.

\begin{itemize}
  \itemoption{shading}|=|\meta{name}
  This selects a shading named \meta{name}. The following shadings are
  predefined:
  \begin{itemize}
  \item \declare{|axis|}
    This is the default shading in which the color changes gradually
    between three horizontal lines. The top line is at the top
    (uppermost) point of the path, the middle is in the middle, the
    bottom line is at the bottom of the path.

\begin{codeexample}[]
\tikz \shadedraw [shading=axis] (0,0) rectangle (1,1);
\end{codeexample}

    The default top color is gray, the default bottom color is white,
    the default middle is the ``middle'' of these two.
  \item \declare{|radial|}
    This shading fills the path with a gradual sweep from a certain
    color in the middle to another color at the border. If the path is
    a circle, the outer color will be reached exactly at the
    border. If the shading is not a circle, the outer color will
    continue a bit towards the corners. The default inner color is
    gray, the default outer color is white.

\begin{codeexample}[]
\tikz \shadedraw [shading=radial] (0,0) rectangle (1,1);
\end{codeexample}
  \item \declare{|ball|}
    This shading fills the path with a shading that ``looks like a
    ball.'' The default ``color'' of the ball is blue (for no
    particular reason).

\begin{codeexample}[]
\tikz \shadedraw [shading=ball] (0,0) rectangle (1,1);
\end{codeexample}

\begin{codeexample}[]
\tikz \shadedraw [shading=ball] (0,0) circle (.5cm);
\end{codeexample}
  \end{itemize}
  \itemoption{shading angle}|=|\meta{degrees}
  This option rotates the shading (not the path!) by the given
  angle. For example, we can turn a top-to-bottom axis shading into a
  left-to-right shading by rotating it by $90^\circ$.

\begin{codeexample}[]
\tikz \shadedraw [shading=axis,shading angle=90] (0,0) rectangle (1,1);
\end{codeexample}
\end{itemize}


You can also define new shading types yourself. However, for this, you
need to use the basic layer directly, which is, well, more basic and
harder to use. Details on how to create a shading appropriate for
filling paths are given in Section~\ref{section-shading-a-path}.



\subsubsection{Choosing a Shading Color}

The following options can be used to change the colors used for
shadings. When one of these options is given, the |shade| option is
automatically selected and also the ``right'' shading.

\begin{itemize}
  \itemoption{top color}|=|\meta{color}
  This option prescribes the color to be used at the top in an |axis|
  shading. When this option is given, several things happen:
  \begin{enumerate}
  \item
    The |shade| option is selected.
  \item
    The |shading=axis| option is selected.
  \item
    The middle color of the axis shading is set to the average of the
    given top color \meta{color} and of whatever color is currently
    selected for the bottom.
  \item
    The rotation angle of the shading is set to 0.
  \end{enumerate}

\begin{codeexample}[]
\tikz \draw[top color=red] (0,0) rectangle (2,1);
\end{codeexample}
  
  \itemoption{bottom color}|=|\meta{color}
  This option works like |top color|, only for the bottom color.
  
  \itemoption{middle color}|=|\meta{color}
  This option specifies the color for the middle of an axis
  shading. It also sets the |shade| and |shading=axis| options, but it
  does not change the rotation angle.

  \emph{Note:} Since both |top color| and |bottom color| change the
  middle color, this option should be given \emph{last} if all of
  these options need to be given:

\begin{codeexample}[]
\tikz \draw[top color=white,bottom color=black,middle color=red]
  (0,0) rectangle (2,1);
\end{codeexample}  

  \itemoption{left color}|=|\meta{color}
  This option does exactly the same as |top color|, except that the
  shading angle is set to $90^\circ$.

  \itemoption{right color}|=|\meta{color}
  Works like |left color|.

  \itemoption{inner color}|=|\meta{color}
  This option sets the color used at the center of a |radial|
  shading. When this option is used, the |shade| and |shading=radial|
  options are set.
  
\begin{codeexample}[]
\tikz \draw[inner color=red] (0,0) rectangle (2,1);
\end{codeexample}

  \itemoption{outer color}|=|\meta{color}
  This option sets the color used at the border and outside of a
  |radial| shading.
  
\begin{codeexample}[]
\tikz \draw[outer color=red,inner color=white]
  (0,0) rectangle (2,1);
\end{codeexample}

  \itemoption{ball color}|=|\meta{color}
  This option sets the color used for the ball shading. It sets the
  |shade| and |shading=ball| options. Note that the ball will never
  ``completely'' have the color \meta{color}. At its ``highlight'' spot
  a certain amount of white is mixed in, at the border a certain
  amount of black. Because of this, it also makes sense to say
  |ball color=white| or |ball color=black|

\begin{codeexample}[]
\begin{tikzpicture}
  \shade[ball color=white] (0,0) circle (2ex);
  \shade[ball color=red] (1,0) circle (2ex);
  \shade[ball color=black] (2,0) circle (2ex);
\end{tikzpicture}
\end{codeexample}
\end{itemize}




\subsection{Establishing a Bounding Box}

\pgfname\ is reasonably good at keeping track of the size of your picture
and reserving just the right amount of space for it in the main
document. However, in some cases you may want to say things like
``do not count this for the picture size'' or ``the picture is
actually a little large.'' For this you can use the option
|use as bounding box| or the command |\useasboundingbox|, which is just
a shorthand for |\path[use as bounding box]|.

\begin{itemize}
  \itemoption{use as bounding box}
  Normally, when this option is given on a path, the bounding box of
  the present path is used to determine the size of the picture and
  the size of all \emph{subsequent} paths are
  ignored. However, if there were previous path operations that have
  already established a larger bounding box, it will not be made
  smaller by this operation.

  In a sense, |use as bounding box| has the same effect as clipping
  all subsequent drawing against the current path---without actually
  doing the clipping, only making \pgfname\ treat everything as if it
  were clipped.

  The first application of this option is to have a |{tikzpicture}|
  overlap with the main text:

\begin{codeexample}[]
Left of picture\begin{tikzpicture}
  \draw[use as bounding box] (2,0) rectangle (3,1);
  \draw (1,0) -- (4,.75);
\end{tikzpicture}right of picture.
\end{codeexample}

  In a second application this option can be used to get better
  control over the white space around the picture:
  
\begin{codeexample}[]
Left of picture
\begin{tikzpicture}
  \useasboundingbox (0,0) rectangle (3,1);
  \fill (.75,.25) circle (.5cm);
\end{tikzpicture}
right of picture.
\end{codeexample}

  Note: If this option is used on a path inside a \TeX\ group (scope),
  the effect ``lasts'' only till the end of the scope. Again, this
  behavior is the same as for clipping.
\end{itemize}

There is a node that allows you to get the size of the current
bounding box. The |current bounding box| node has the |rectangle|
shape |rectangle| shape and its size is always the size of the current 
bounding box.


\begin{codeexample}[]
\begin{tikzpicture}
  \draw[red] (0,0) circle (2pt);
  \draw[red] (2,1) circle (3pt);

  \draw (current bounding box.south west) rectangle
        (current bounding box.north east);

  \draw[red] (3,-1) circle (4pt);

  \draw[thick] (current bounding box.south west) rectangle
               (current bounding box.north east);
\end{tikzpicture}
\end{codeexample}





\subsection{Using a Path For Clipping}

To use a path for clipping, use the |clip| option. 

\begin{itemize}
  \itemoption{clip}
  This option causes all subsequent drawings to be clipped against the
  current path and the size of subsequent paths will not be important
  for the picture size.  If you clip against a self-intersecting path,
  the even-odd rule or  the nonzero winding number rule is used to
  determine whether a point is inside or outside the clipping region.

  The clipping path is a graphic state parameter, so it will be reset
  at the end of the current scope. Multiple clippings accumulate, that
  is, clipping is always done against the intersection of all clipping
  areas that have been specified inside the current scopes. The only
  way of enlarging the clipping area is to end a |{scope}|.

\begin{codeexample}[]
\begin{tikzpicture}
  \draw[clip] (0,0) circle (1cm);
  \fill[red] (1,0) circle (1cm);
\end{tikzpicture}
\end{codeexample}

  It  is usually a \emph{very} good idea to apply the |clip| option only
  to the first path command in a scope. 

  If you ``only wish to clip'' and do not wish to draw anything, you can
  use the |\clip| command, which is a shorthand for |\path[clip]|.

\begin{codeexample}[]
\begin{tikzpicture}
  \clip (0,0) circle (1cm);
  \fill[red] (1,0) circle (1cm);
\end{tikzpicture}
\end{codeexample}

  To keep clipping local, use |{scope}| environments as in the
  following example:

\begin{codeexample}[]
\begin{tikzpicture}
  \draw (0,0) -- ( 0:1cm);
  \draw (0,0) -- (10:1cm);
  \draw (0,0) -- (20:1cm);
  \draw (0,0) -- (30:1cm);
  \begin{scope}[fill=red]
    \fill[clip] (0.2,0.2) rectangle (0.5,0.5);
    
    \draw (0,0) -- (40:1cm);
    \draw (0,0) -- (50:1cm);
    \draw (0,0) -- (60:1cm);
  \end{scope}
  \draw (0,0) -- (70:1cm);
  \draw (0,0) -- (80:1cm);
  \draw (0,0) -- (90:1cm);
\end{tikzpicture}
\end{codeexample}

  There is a slightly annoying catch: You cannot specify certain graphic
  options for the command used for clipping. For example, in the above
  code we could not have moved the |fill=red| to the |\fill|
  command. The reasons for this have to do with the internals of the
  \pdf\ specification. You do not want to know the details. It is best
  simply not to specify any options for these 
  commands. 
\end{itemize}

% Copyright 2006 by Till Tantau
%
% This file may be distributed and/or modified
%
% 1. under the LaTeX Project Public License and/or
% 2. under the GNU Free Documentation License.
%
% See the file doc/generic/pgf/licenses/LICENSE for more details.

\section{Nodes and Edges}

\label{section-nodes}

\subsection{Overview}

In the present section, the usage of \emph{nodes} in
\tikzname\ is explained. A node is typically a rectangle or circle or
another simple shape with some text on it. 

Nodes are added to paths using the special path
operation |node|. Nodes \emph{are not part of the path
  itself}. Rather, they are added to the picture after the path has
been drawn. 

In Section~\ref{section-nodes-basic} the basic syntax of the node
operation is explained, followed in Section~\ref{section-nodes-multi}
by the syntax for multi-part nodes, which are nodes that contain
several different text parts. After this, the different options for
the text in nodes are explained. In
Section~\ref{section-nodes-anchors} the concept of \emph{anchors} is
introduced along with their usage. In
Section~\ref{section-nodes-transformations} the different ways
transformations affect nodes are
studied. Sections~\ref{section-nodes-placing-1}
and~\ref{section-nodes-placing-2} are about placing nodes on or next
to straight lines and curves. In
Section~\ref{section-nodes-connecting} it is explained how a node can
be used as a ``pseudo-coordinate.'' Section~\ref{section-nodes-edges}
introduces the |edge| operation, which
works similar to the |to| operation and also similar to the |node|
operation. Section~\ref{section-nodes-predefined} lists the predefined
shapes. Finally, Section~\ref{section-nodes-executing} explains the
special |after node path| options.


\subsection{Nodes and Their Shapes}

\label{section-nodes-basic}

In the simplest case, a node is just some text that is
placed at some coordinate. However, a node can also have a border
drawn around it or have a more complex background and
foreground. Indeed, some nodes do not have a text at all, but consist
solely of the background. You can name nodes so that you can reference
their coordinates later in the same picture or, if certain precautions
are taken as explained in Section~\ref{section-cross-picture-tikz},
also in different pictures.

There are no special \TeX\ commands for adding a node to a picture; rather,
there is path operation called |node| for this. Nodes are created
whenever \tikzname\ encounters |node| or |coordinate| at a point on a
path where it would expect a normal path operation (like |-- (1,1)| or
|sin (1,1)|). It is also possible to give node specifications
\emph{inside} certain path operations as explained later.

The node operation is typically followed by some options, which apply
only to the node. Then, you can optionally \emph{name} the node by
providing a name in round braces. Lastly, for the |node| operation you
must provide some label text for the node in curly braces, while for
the |coordinate| operation you may not. The node is placed at the
current position of the path \emph{after the path has been
  drawn}. Thus, all nodes are drawn ``on top'' of the path and
retained until the path is complete. If there are several nodes on a
path, they are drawn on top of the path in the order they are
encountered. 

\begin{codeexample}[]
\tikz \fill[fill=examplefill]
     (0,0) node {first node}
  -- (1,1) node {second node}
  -- (0,2) node {third node};
\end{codeexample}

The syntax for specifying nodes is the following:
\begin{pathoperation}{node}{\opt{|[|\meta{options}|]|}\opt{|(|\meta{name}|)|}%
    \opt{|at(|\meta{coordinate}|)|}\opt{\marg{text}}}
  The effect of |at| is to place the node at the coordinate given
  after |at| and not, as would normally be the case, at the last
  position. The |at| syntax is not available when a node is given
  inside a path operation (it would not make any sense, there).
  
  The |(|\meta{name}|)| is a name for later reference and it is
  optional. You may also add the option |name=|\meta{name} to the
  \meta{option} list; it has the same effect.

  \begin{itemize}
    \itemoption{name}|=|\meta{node name}
    assigns a name to the node for later reference. Since this is a
    ``high-level'' name (drivers never know of it), you can use spaces,
    number, letters, or whatever you like when naming a node. Thus, you
    can name a node just |1| or perhaps |start of chart| or even
    |y_1|. Your node name should \emph{not} contain any punctuation like
    a dot, a comma, or a colon since these are used to detect what kind
    of coordinate you mean when you reference a node. 

    \itemoption{at}|=|\meta{coordinate}
    is another way of specifying ath |at| coordinate.
  \end{itemize}

  The \meta{options} is an optional list of options that \emph{apply
    only to the node} and have no effect outside. The other way round,
  most ``outside'' options also apply to the node, but not all. For
  example, the ``outside'' rotation does not apply to nodes (unless some
  special options are used, sigh). Also, the outside path action, like
  |draw| or |fill|, never applies to the node and must be given in the
  node (unless some special other options are used, deep sigh).

  As mentioned before, we can add a border and even a background to a
  node:  
\begin{codeexample}[]
\tikz \fill[fill=examplefill]
      (0,0) node {first node}
   -- (1,1) node[draw] {second node}
   -- (0,2) node[fill=red!20,draw,double,rounded corners] {third node};
\end{codeexample}

  The ``border'' is actually just a special case of a much more general
  mechanism. Each node has a certain \emph{shape} which, by default, is
  a rectangle. However, we can also ask \tikzname\ to use a circle shape
  instead or an ellipse shape (you have to include |pgflibraryshapes| for
  the latter shape): 

\begin{codeexample}[]
\tikz \fill[fill=examplefill]
      (0,0) node{first node}
   -- (1,1) node[ellipse,draw] {second node}
   -- (0,2) node[circle,fill=red!20] {third node};
\end{codeexample}

  In the future, there might be much more complicated shapes available
  such as, say, a shape for a resistor or a shape for a \textsc{uml}
  class. Unfortunately, creating new shapes is a bit tricky and makes
  it necessary to use the basic layer directly. Life is hard.

  To select the shape of a node, the following option is used:
  \begin{itemize}
    \itemoption{shape}|=|\meta{shape name}
    select the shape either of the current node or, when this option is
    not given inside a node but somewhere outside, the shape of all
    nodes in the current scope.%
    \indexoption{\meta{shape name}}

    Since this option is used often, you can leave out the
    |shape=|. When \tikzname\ encounters an option like |circle|
    that it does not know, it will, after everything else has failed,
    check whether this option is the name of some shape. If so, that
    shape is selected as if you had said |shape=|\meta{shape name}.

    By default, the following shapes are available: |rectangle|,
    |circle|, |coordinate|, and, when the package |pgflibraryshapes| is
    loaded, also |ellipse|. Details of these shapes, like their anchors
    and size options, are discussed in Section~\ref{section-the-shapes}.
  \end{itemize}
  
  The following styles influences how nodes are rendered:
  \begin{itemize}
    \itemstyle{every node}
    This style is installed at the beginning of every node. 
\begin{codeexample}[]
\begin{tikzpicture}
  \tikzstyle{every node}=[draw] 
  \draw (0,0) node {A} -- (1,1) node {B};
\end{tikzpicture}
\end{codeexample}

    \itemstyle{every \meta{shape} node}
    These styles are installed at the beginning of a node of a given
    \meta{shape}. For example, |every rectangle node| is used for
    rectangle nodes, and so on.
\begin{codeexample}[]
\begin{tikzpicture}
  \tikzstyle{every rectangle node}=[draw] 
  \tikzstyle{every circle node}=   [draw,double] 
  \draw (0,0) node[rectangle] {A} -- (1,1) node[circle] {B};
\end{tikzpicture}
\end{codeexample}
  \end{itemize}
\end{pathoperation}

The is a special syntax for specifying ``light-weighed'' nodes:

\begin{pathoperation}{coordinate}{\opt{|[|\meta{options}|]|}|(|\meta{name}|)|\opt{|at(|\meta{coordinate}|)|}}
  This has the same effect as

  |node[shape=coordinate][|\meta{options}|](|\meta{name}|)at(|\meta{coordinate}|){}|,
  
  where the |at| part might be missing.
\end{pathoperation}



\subsection{Multi-Part Nodes}

\label{section-nodes-multi}

Most nodes just have a single simple text label. However, nodes of a
more complicated shapes might be made up from several \emph{node
  parts}. For example, in automata theory a so-called Moore state has
a state name, drawn in the upper part of the state circle, and an
output text, drawn in the lower part of the state circle. These two
parts are quite independent. Similarly, a \textsc{uml} class shape
would have a name part, a method part, and an attributes
part. Different molecule shape might use parts for the different atoms
to be drawn at the different positions, and so on.

Both \pgfname\ and \tikzname\ support such multipart nodes. On the
lower level, \pgfname\ provides a system for specifying that a shape
consists of several parts. On the \tikzname\ level, you specify the
different node parts by using the following command:

\begin{command}{\nodepart\marg{part name}}
  This command can only be used inside the \meta{text} argument of a
  |node| path operation. It works a little bit like a |\part| command
  in \LaTeX. It will stop the typesetting of whatever node part was
  typeset until now and then start putting all following text into the
  node part named \meta{part name}---until another |\partname| is
  encountered or until the node \meta{text} ends.

\begin{codeexample}[]
\begin{tikzpicture}
  \node [circle split,draw,double,fill=red!20]
  {
    % No \nodepart has been used, yet. So, the following is put in the
    % ``text'' node part by default.
    $q_1$ 
    \nodepart{lower} % Ok, end ``text'' part, start ``output'' part
    $00$
  }; % output part ended.
\end{tikzpicture}
\end{codeexample}

  You will have to lookup which parts are defined by a shape.

  The following styles influences node parts:
  \begin{itemize}
    \itemstyle{every \meta{part name} node part}
    This style is installed at the beginning of every node part named
    \meta{part name}. 
\begin{codeexample}[]
\tikzstyle{every lower node part}=[red] 
\tikz \node [circle split,draw] {$q_1$ \nodepart{lower} $00$};
\end{codeexample}
  \end{itemize}
\end{command}



\subsection{Options for the Text in  Nodes}

\label{section-nodes-options}

The simplest option for the text in nodes is its color. Normally, this
color is just the last color installed using |color=|, possibly
inherited from another scope. However, it is possible to specificly
set the color used for text using the following option:

\begin{itemize}
  \itemoption{text}|=|\meta{color}
  Sets the color to be used for text labels. A |color=| option
  will immediately override this option.
\begin{codeexample}[]
\begin{tikzpicture}
  \draw[red]       (0,0) -- +(1,1) node[above]     {red};
  \draw[text=red]  (1,0) -- +(1,1) node[above]     {red};
  \draw            (2,0) -- +(1,1) node[above,red] {red};
\end{tikzpicture}
\end{codeexample}
\end{itemize}

Just like the color itself, you may also wish to set the opacity of
the text only. For this, use the following option:
\begin{itemize}
  \itemoption{text opacity}|=|\meta{value}
  Sets the opacity of text labels. 
\begin{codeexample}[]
\begin{tikzpicture}
  \draw[line width=2mm,blue!50,cap=round] (0,0) grid (3,2);
  \tikzstyle{every node}=[fill,draw]

  \node[opacity=0.5] at (1.5,2) {Upper node};
  \node[draw opacity=0.8,fill opacity=0.2,text opacity=1]
    at (1.5,0) {Lower node};
\end{tikzpicture}
\end{codeexample}
\end{itemize}

Next, you may wish to adjust the font used for the text. Use the
following option for this:
\begin{itemize}
  \itemoption{font}|=|\meta{font commands}
  Sets the font used for text labels. 
\begin{codeexample}[]
\begin{tikzpicture}
  \draw[font=\itshape] (1,0) -- +(1,1) node[above] {italic};
\end{tikzpicture}
\end{codeexample}
  A perhaps more useful example is the following:

\begin{codeexample}[]
\tikzstyle{every text node part}=[font=\itshape] 
\tikzstyle{every lower node part}=[font=\footnotesize]
\tikz \node [circle split,draw] {state \nodepart{lower} output};
\end{codeexample}
\end{itemize}


Normally, when a node is typeset, all the text you give in the braces
is but in one long line (in an |\hbox|, to be precise) and the node
will become as wide as necessary.

You can change this behaviour using the following options. They allow
you to limit the width of a node (naturally, at the expense of its
height).

\begin{itemize}
  \itemoption{text width}|=|\meta{dimension}
  This option will put the text of a node in a box of the given width
  (more precisely, in a |{minipage}| of this width; for plain \TeX\ a
  rudimentary ``minipage emulation'' is used).

  If the node text is not as wide as \meta{dimension}, it will
  nevertheless be put in a box of this width. If it is larger, line
  breaking will be done.

  By default, when this option is given, a ragged right border will be
  used. This is sensible since, typically, these boxes are narrow and
  justifying the text looks ugly.
\begin{codeexample}[]
\tikz \draw (0,0) node[fill=examplefill,text width=3cm]
  {This is a demonstration text for showing how line breaking works.};  
\end{codeexample}
  \itemoption{text justified}
  causes the text to be justified instead of (right)ragged. Use this
  only with pretty broad nodes.
{%
\hbadness=10000
\begin{codeexample}[]
\tikz \draw (0,0) node[fill=examplefill,text width=3cm,text justified]
  {This is a demonstration text for showing how line breaking works.};  
\end{codeexample}
}
  In the above example, \TeX\ complains (rightfully) about three very
  badly typeset lines. (For this manual I asked \TeX\ to stop
  complaining by using |\hbadness=10000|, but this is a foul deed,
  indeed.) 
  \itemoption{text ragged}
  causes the text to be typeset with a ragged right. This uses the
  original plain \TeX\ definition of a ragged right border, in which
  \TeX\ will try to balance the right border as well as possible. This
  is the default.
\begin{codeexample}[]
\tikz \draw (0,0) node[fill=examplefill,text width=3cm,text ragged]
  {This is a demonstration text for showing how line breaking works.};  
\end{codeexample}
  \itemoption{text badly ragged}
  causes the right border to be ragged in the \LaTeX-style, in which
  no balancing occurs. This looks ugly, but it may be useful for very
  narrow boxes and when you wish to avoid hyphenations.
\begin{codeexample}[]
\tikz \draw (0,0) node[fill=examplefill,text width=3cm,text badly ragged]
  {This is a demonstration text for showing how line breaking works.};  
\end{codeexample}
  \itemoption{text centered}
  centers the text, but tries to balance the lines.
\begin{codeexample}[]
\tikz \draw (0,0) node[fill=examplefill,text width=3cm,text centered]
  {This is a demonstration text for showing how line breaking works.};  
\end{codeexample}
  \itemoption{text badly centered}
  centers the text, without balancing the lines.
\begin{codeexample}[]
\tikz \draw (0,0) node[fill=examplefill,text width=3cm,text badly centered]
  {This is a demonstration text for showing how line breaking works.};  
\end{codeexample}
\end{itemize}

In addition to changing the width of nodes, you can also change the
height of nodes. This can be done in two ways: First, you can use the
option |minimum height|, which ensures that the height of the whole
node is at least the given height (this option is described in more
detail later). Second, you can use the option |text height|, which
sets the height of the text itself, more precisely, of the \TeX\ text
box of the text. Note that the |text height| typically is not the
height of the shape's box: In addition to the |text height|, an
internal |inner sep| is added as extra space and the text depth is
also taken into account.

I recommend using |minimum size| instead of |text height| except for
special situations.

\begin{itemize}
  \itemoption{text height}|=|\meta{dimension}
  Sets the height of the text boxes in shapes. Thus, when you write
  something like |node {text}|, the |text| is first typeset, resulting
  in some box of a certain height. This height is then replaced by the
  height |text height|. The resulting box is then used to determine
  the size of the shape, which will typically be larger. When you
  write |text height=| without specifying anything, the ``natural''
  size of the text box remains unchanged.
\begin{codeexample}[]
\tikz \node[draw]                  {y};    
\tikz \node[draw,text height=10pt] {y};       
\end{codeexample}
  \itemoption{text depth}|=|\meta{dimension}
  This option works like |text height|, only for the depth of the text
  box. This option is mostly useful when you need to ensure a uniform
  depth of text boxes that need to be aligned. 
\end{itemize}




\subsection{Placing Nodes Using Anchors}

\label{section-nodes-anchors}

When you place a node at some coordinate, the node is centered on this
coordinate by default. This is often undesirable and it would be
better to have the node to the right or above the actual coordinate.

\pgfname\ uses a so-called anchoring mechanism to give you a very fine
control over the placement. The idea is simple: Imaging a node of
rectangular shape of a certain size. \pgfname\ defines numerous anchor
positions in the shape. For example to upper right corner is called,
well, not ``upper right anchor,'' but the |north east| anchor of the
shape. The center of the shape has an anchor called |center| on top of
it, and so on. Here are some examples (a complete list is given in
Section~\ref{section-the-shapes}).

\medskip\noindent
\begin{tikzpicture}
  \path node[minimum height=2cm,minimum width=5cm,fill=blue!25](x) {Big node};
  \fill (x.north)      circle (2pt) node[above] {|north|}
        (x.north east) circle (2pt) node[above] {|north east|}
        (x.north west) circle (2pt) node[above] {|north west|}
        (x.west) circle (2pt)       node[left]  {|west|}
        (x.east) circle (2pt)       node[right] {|east|}
        (x.base) circle (2pt)       node[below] {|base|};
\end{tikzpicture}

Now, when you place a node at a certain coordinate, you can ask \tikzname\
to place the node shifted around in such a way that a certain
anchor is at the coordinate. In the following example, we ask \tikzname\
to shift the first node such that its  |north east| anchor is at
coordinate |(0,0)| and that the |west| anchor of the second node is at
coordinate |(1,1)|.

\begin{codeexample}[]
\tikz \draw           (0,0) node[anchor=north east] {first node}
            rectangle (1,1) node[anchor=west] {second node};
\end{codeexample}

Since the default anchor is |center|, the default behaviour is to
shift the node in such a way that it is centered on the current
position.

\begin{itemize}
  \itemoption{anchor}|=|\meta{anchor name}
  causes the node to be shifted such that it's anchor \meta{anchor
  name} lies on the current coordinate.

  The only anchor that is present in all shapes is |center|. However,
  most shapes will at least define anchors in all ``compass
  directions.'' Furthermore, the standard shapes also define a |base|
  anchor, as well as |base west| and |base east|, for placing things on
  the baseline of the text.
  
  The standard shapes also define a |mid| anchor (and |mid west| and
  |mid east|). This anchor is half the height of the character ``x''
  above the base line. This anchor is useful for vertically centering
  multiple nodes that have different heights and depth. Here is an
  example:
\begin{codeexample}[]
\begin{tikzpicture}[scale=3,transform shape]
  % First, center alignment -> wobbles
  \draw[anchor=center] (0,1)  node{x} -- (0.5,1)  node{y} -- (1,1)  node{t};
  % Second, base alignment -> no wobble, but too high
  \draw[anchor=base]   (0,.5) node{x} -- (0.5,.5) node{y} -- (1,.5) node{t};
  % Third, mid alignment
  \draw[anchor=mid]    (0,0)  node{x} -- (0.5,0)  node{y} -- (1,0)  node{t};
\end{tikzpicture}
\end{codeexample}
\end{itemize}

Unfortunately, while perfectly logical, it is often rather
counter-intuitive that in order to place a node \emph{above} a given
point, you need to specify the |south| anchor. For this reason, there
are some useful options that allow you to select the standard anchors
more intuitively:
\begin{itemize}
  \itemoption{above}\opt{|=|\meta{offset}}
  does the same as |anchor=south|. If the \meta{offset} is specified,
  the node is additionally shifted upwards by the given
  \meta{offset}. 
\begin{codeexample}[]
\tikz \fill (0,0) circle (2pt) node[above] {above};
\end{codeexample}
\begin{codeexample}[]
\tikz \fill (0,0) circle (2pt) node[above=2pt] {above};
\end{codeexample}
  \itemoption{above left}\opt{|=|\meta{offset}}
  does the same as |anchor=south east|. If the \meta{offset} is
  specified, the node is additionally shifted upwards and right by
  \meta{offset}. 
\begin{codeexample}[]
\tikz \fill (0,0) circle (2pt) node[above left] {above left};
\end{codeexample}
\begin{codeexample}[]
\tikz \fill (0,0) circle (2pt) node[above left=2pt] {above left};
\end{codeexample}
  \itemoption{above right}\opt{|=|\meta{offset}}
  does the same as |anchor=south west|.
\begin{codeexample}[]
\tikz \fill (0,0) circle (2pt) node[above right] {above right};
\end{codeexample}
  \itemoption{left}\opt{|=|\meta{offset}}
  does the same as |anchor=east|.
\begin{codeexample}[]
\tikz \fill (0,0) circle (2pt) node[left] {left};
\end{codeexample}
  \itemoption{right}\opt{|=|\meta{offset}}
  does the same as |anchor=west|.
  \itemoption{below}\opt{|=|\meta{offset}}
  does the same as |anchor=north|.
  \itemoption{below left}\opt{|=|\meta{offset}}
  does the same as |anchor=north east|.
  \itemoption{below right}\opt{|=|\meta{offset}}
  does the same as |anchor=north west|.
\end{itemize}

A second set of options behaves similarly, namely the |above of|,
|below of|, and so on options. They cause the same anchors to be set
as the options without |of|, however, their parameter is different:
You must provide the name of another node. The current node will then
be placed, say, above this specified node at a distance given by the
option |node distance|. 
\begin{itemize}
  \itemoption{above of}\opt{|=|\meta{node}}
  This option causes the node to be placed at the distance
  |node distance| above of \meta{node}. The anchor is |center|.
\begin{codeexample}[]
\begin{tikzpicture}[node distance=1cm]
  \draw[help lines] (0,0) grid (3,2);
  \node (a)                    {a};
  \node (b) [above of=a]       {b};
  \node (c) [above of=b]       {c};
  \node (d) [right of=c]       {d};
  \node (e) [below right of=d] {e};
\end{tikzpicture}
\end{codeexample}
  \itemoption{above left of}\opt{|=|\meta{node}}
  Works like |above of|, only the node is now put above and left. The
  |node distance| is the Euclidean distance between the two nodes, not
  the $L_1$-distance.
  \itemoption{above right of}\opt{|=|\meta{node}} works similarly.
  \itemoption{left of}\opt{|=|\meta{node}} works similarly.
  \itemoption{right of}\opt{|=|\meta{node}} works similarly.
  \itemoption{below of}\opt{|=|\meta{node}} works similarly.
  \itemoption{below left of}\opt{|=|\meta{node}} works similarly.
  \itemoption{below right of}\opt{|=|\meta{node}} works similarly.
  \itemoption{node distance}\opt{|=|\meta{dimension}} sets the
  distance between nodes that are placed using the |... of|
  options. Note that this distance is the distance between the
  centers of the nodes, not the distance between their borders.
\end{itemize}


\subsection{Transformations}

\label{section-nodes-transformations}


It is possible to transform nodes, but, by default, transformations do
not apply to nodes. The reason is that you usually do \emph{not} want
your text to be scaled or rotated even if the main graphic is
transformed. Scaling text is evil, rotating slightly less so.

However, sometimes you \emph{do} wish to transform a node, for
example, it certainly sometimes makes sense to rotate a node by
90 degrees. There are two ways in which you can achieve this:

\begin{enumerate}
\item
  You can use the following option:
  \begin{itemize}
    \itemoption{transform shape}
    causes the current ``external'' transformation matrix to be
    applied to the shape. For example, if you said
    |\tikz[scale=3]| and then say |node[transform shape] {X}|, you
    will get a ``huge'' X in your graphic.
  \end{itemize}
\item
  You can give transformation option \emph{inside} the option list of
  the node. \emph{These} transformations always apply to the node.
\begin{codeexample}[]
\begin{tikzpicture}
  \tikzstyle{every node}=[draw]    
  \draw[style=help lines] (0,0) grid (3,2);
  \draw            (1,0) node{A}
                   (2,0) node[rotate=90,scale=1.5] {B};
  \draw[rotate=30] (1,0) node{A}
                   (2,0) node[rotate=90,scale=1.5] {B};
  \draw[rotate=60] (1,0) node[transform shape] {A}
                   (2,0) node[transform shape,rotate=90,scale=1.5] {B};
\end{tikzpicture}
\end{codeexample}
\end{enumerate}




\subsection{Placing Nodes on a Line or Curve Explicitly}

\label{section-nodes-placing-1}

Until now, we always placed node on a coordinate that is mentioned in
the path. Often, however, we wish to place nodes on ``the middle'' of
a line and we do not wish to compute these coordinates ``by hand.''
To facilitate such placements, \tikzname\ allows you to specify that a
certain node should be somewhere ``on'' a line. There are two ways of
specifying this: Either explicitly by using the |pos| option or
implicitly by placing the node ``inside'' a path operation. These two
ways are described in the following.

\label{section-pos-option}

\begin{itemize}
  \itemoption{pos}|=|\meta{fraction}
  When this option is given, the node is not anchored on the last
  coordinate. Rather, it is anchored on some point on the line from
  the previous coordinate to the current point. The \meta{fraction}
  dictates how ``far'' on the line the point should be. A
  \meta{fraction} or 0 is the previous coordinate, 1 is the current
  one, everything else is in between. In particular, 0.5 is the
  middle.  

  Now, what is ``the previous line''? This depends on the previous
  path construction operation.

  In the simplest case, the previous path operation was a ``line-to''
  operation, that is, a  |--|\meta{coordinate} operation:
\begin{codeexample}[]
\tikz \draw (0,0) -- (3,1)
    node[pos=0]{0} node[pos=0.5]{1/2} node[pos=0.9]{9/10};
\end{codeexample}

  The next case is the curve-to operation (the |..| operation). In this
  case, the ``middle'' of the curve, that is, the position |0.5| is
  not necessarily the point at the exact half distance on the
  line. Rather, it is some point at ``time'' 0.5 of a point traveling
  from the start of the curve, where it is at time 0, to the end of
  the curve, which it reaches at time 0.5. The ``speed'' of the point
  depends on the length of the support vectors (the vectors that
  connect the start and end points to the control points). The exact
  math is a bit complicated (depending on your point of view, of
  course); you may wish to consult a good book on computer graphics
  and B�zier curves if you are intrigued. 
\begin{codeexample}[]
  \tikz \draw (0,0) .. controls +(right:3.5cm) and +(right:3.5cm) .. (0,3)
    \foreach \p in {0,0.125,...,1} {node[pos=\p]{\p}};
\end{codeexample}

  Another interesting case are the horizontal/vertical line-to operations
  \verb!|-! and \verb!-|!. For them, the position (or time) |0.5| is
  exactly the corner point.

\begin{codeexample}[]
\tikz \draw (0,0) |- (3,1)
  node[pos=0]{0} node[pos=0.5]{1/2} node[pos=0.9]{9/10};
\end{codeexample}

\begin{codeexample}[]
\tikz \draw (0,0) -| (3,1)
  node[pos=0]{0} node[pos=0.5]{1/2} node[pos=0.9]{9/10};
\end{codeexample}

  For all other path construction operations, \emph{the position
  placement does not work}, currently. This will hopefully change in
  the future (especially for the arc operation).  

  \itemoption{auto}\opt{|=|\meta{direction}}
  This option causes an anchor positions to be calculated
  automatically according to the following rule. Consider a line
  between to points. If the \meta{direction} is |left|, then the
  anchor is chosen such that the node is to the left of this line. If
  the \meta{direction} is |right|, then the node is to the right of
  this line. Leaving out \meta{direction} causes automatic placement
  to be enabled with the last value of |left| or |right| used. A
  \meta{direction} of |false| disables automatic placement. This
  happens also  whenever an anchor is given explicitly by the
  |anchor| option or by one of the |above|, |below|, etc.\ options.

  This option only has an effect for nodes that are placed on lines or
  curves. 

\begin{codeexample}[]
\begin{tikzpicture}[scale=.8,auto=left]
  \tikzstyle{every node}=[circle,fill=blue!20]
  \node (a) at (-1,-2) {a};
  \node (b) at ( 1,-2) {b};
  \node (c) at ( 2,-1) {c};
  \node (d) at ( 2, 1) {d};
  \node (e) at ( 1, 2) {e};
  \node (f) at (-1, 2) {f};
  \node (g) at (-2, 1) {g};
  \node (h) at (-2,-1) {h};

  \tikzstyle{every node}=[fill=red!20]
  \foreach \from/\to in {a/b,b/c,c/d,d/e,e/f,f/g,g/h,h/a}
    \draw [->] (\from) -- (\to) node[midway] {\from--\to};
\end{tikzpicture}
\end{codeexample}
  \itemoption{swap}
  This option exchanges the roles of |left| and |right| in automatic
  placement. That is, if |left| is the current |auto| placement,
  |right| is set instead and the other way round.
\begin{codeexample}[]
\begin{tikzpicture}[auto]
  \draw[help lines,use as bounding box] (0,-.5) grid (4,5);  

  \draw (0.5,0) .. controls (9,6) and (-5,6) .. (3.5,0)
    \foreach \pos in {0,0.1,0.2,0.3,0.4,0.5,0.6,0.7,0.8,0.9,1}
      {node [pos=\pos,swap,fill=red!20] {\pos}}
    \foreach \pos in {0.025,0.2,0.4,0.6,0.8,0.975}
      {node [pos=\pos,fill=blue!20] {\pos}};
\end{tikzpicture}
\end{codeexample}
\begin{codeexample}[]
\begin{tikzpicture}[shorten >=1pt,node distance=2cm,auto]
  \draw[help lines] (0,0) grid (3,2);

  \node[state] (q_0)                      {$q_0$};
  \node[state] (q_1) [above right of=q_0] {$q_1$};
  \node[state] (q_2) [below right of=q_0] {$q_2$};
  \node[state] (q_3) [below right of=q_1] {$q_3$};

  \path[->] (q_0) edge              node        {0} (q_1)
                  edge              node [swap] {1} (q_2)
            (q_1) edge              node        {1} (q_3)
                  edge [loop above] node        {0} ()
            (q_2) edge              node [swap] {0} (q_3)
                  edge [loop below] node        {1} ();
\end{tikzpicture}
\end{codeexample}

  \itemoption{sloped}
  This option causes the node to be rotated such that a horizontal
  line becomes a tangent to the curve. The rotation is normally 
  done in such a way that text is never ``upside down.'' To get
  upside-down text, use can use |[rotate=180]| or
  |[allow upside down]|, see below.
\begin{codeexample}[]
\tikz \draw (0,0) .. controls +(up:2cm) and +(left:2cm) .. (1,3)
    \foreach \p in {0,0.25,...,1} {node[sloped,above,pos=\p]{\p}};
\end{codeexample}
\begin{codeexample}[]
\begin{tikzpicture}[->]
  \draw (0,0)   -- (2,0.5) node[midway,sloped,above] {$x$};
  \draw (2,-.5) -- (0,0)   node[midway,sloped,below] {$y$};
\end{tikzpicture}
\end{codeexample}
  \itemoption{allow upside down}\opt{|=|\meta{true or false}}
  If set to |true|, \tikzname\ will not ``righten'' upside down text. 
\begin{codeexample}[]
\tikz [allow upside down]
  \draw (0,0) .. controls +(up:2cm) and +(left:2cm) .. (1,3)
    \foreach \p in {0,0.25,...,1} {node[sloped,above,pos=\p]{\p}};
\end{codeexample}
\begin{codeexample}[]
\begin{tikzpicture}[->,allow upside down]
  \draw (0,0)   -- (2,0.5) node[midway,sloped,above] {$x$};
  \draw (2,-.5) -- (0,0)   node[midway,sloped,below] {$y$};
\end{tikzpicture}
\end{codeexample}
\end{itemize}


There exist styles for specifying positions a bit less ``technically'':
\begin{itemize}
  \itemstyle{midway}
  is set to |pos=0.5|.
\begin{codeexample}[]
\tikz \draw (0,0) .. controls +(up:2cm) and +(left:3cm) .. (1,5)
       node[at end]          {|at end|}
       node[very near end]   {|very near end|}
       node[near end]        {|near end|}
       node[midway]          {|midway|}
       node[near start]      {|near start|}
       node[very near start] {|very near start|}
       node[at start]        {|at start|};
\end{codeexample}
  \itemstyle{near start}
  is set to |pos=0.25|.
  \itemstyle{near end}
  is set to |pos=0.75|.
  \itemstyle{very near start}
  is set to |pos=0.125|.
  \itemstyle{very near end}
  is set to |pos=0.875|.
  \itemstyle{at start}
  is set to |pos=0|.
  \itemstyle{at end}
  is set to |pos=1|.
\end{itemize}


\subsection{Placing Nodes on a Line or Curve Implicitly}

\label{section-nodes-placing-2}

When you wish to place a node on the line |(0,0) -- (1,1)|,
it is natural to specify the node not following the |(1,1)|, but
``somewhere in the middle.'' This is, indeed, possible and you can
write |(0,0) -- node{a} (1,1)| to place a node midway between |(0,0)| and
|(1,1)|.

What happens is the following: The syntax of the line-to path
operation is actually |--|
\opt{|node|\meta{node specification}}\meta{coordinate}. (It is even
possible to give multiple nodes in this way.) When the optional
|node| is encountered, that is, 
when the |--| is directly followed by |node|, then the
specification(s) are read and ``stored away.'' Then, after the
\meta{coordinate} has finally been reached, they are inserted again,
but with the |pos| option set.

There are two things to note about this: When a node specification is
``stored,'' its catcodes become fixed. This means that you cannot use
overly complicated verbatim text in them. If you really need, say, a
verbatim text, you will have to put it in a normal node following the
coordinate and add the |pos| option.

Second, which |pos| is chosen for the node? The position is inherited
from the surrounding scope. However, this holds only for nodes
specified in this implicit way. Thus, if you add the option
|[near end]| to a scope, this does not mean that \emph{all} nodes given
in this scope will be put on near the end of lines. Only the nodes
for which an implicit |pos| is added will be placed near the
end. Typically, this is what you want. Here are some examples that
should make this clearer:

\begin{codeexample}[]
\begin{tikzpicture}[near end]
  \draw (0cm,4em) -- (3cm,4em) node{A};    
  \draw (0cm,3em) --           node{B}          (3cm,3em);
  \draw (0cm,2em) --           node[midway] {C} (3cm,2em);
  \draw (0cm,1em) -- (3cm,1em) node[midway] {D} ;
\end{tikzpicture}
\end{codeexample}

Like the line-to operation, the curve-to operation |..| also allows you to
specify nodes ``inside'' the operation. After both the first |..| and
also after the second |..| you can place node specifications. Like for
the |--| operation, these will be collected and then reinserted after
the operation with the |pos| option set.


\subsection{The Label and Pin Options}

In addition to the |node| path operation, nodes can also be added
using the |label| and the |pin| option. This is mostly useful
for simple nodes. 

\begin{itemize}
  \itemoption{label}|=|\opt{|[|\meta{options}|]|}\meta{angle}|:|\meta{text}
  When this option is given to a |node| operation, it causes
  \emph{another} node to be added to the path after the current node
  has been finished. This extra node will have the text
  \meta{text}. It is placed according to the following rule: Suppose
  the |node| currently under construction is called |main node| and let us
  call the label node |label node|. Then the anchor of |label node| is
  placed at |main node.|\meta{angle}. The anchor that is chosen
  depends on the \meta{angle}. If the \meta{angle} lies between
  $-3^\circ$ and $+3^\circ$, then the anchor |west| is chosen, which
  causes |label node| to be placed right of the right end
  |main node|. If \meta{angle} lies between $4^\circ$ and $86^\circ$,
  the anchor |south west| is chosen, causing the |label node| to be
  placed above and right of the |main node|; and so on.
\begin{codeexample}[]
\tikz
  \node [circle,draw,label=60:$60^\circ$,label=below:$-90^\circ$] {my circle};
\end{codeexample}

  As can be seen in the above example, instead of specifying
  \meta{angle} as a number, it is also possible to use |left|,
  |right|, |above|, |above left|, and so on.

  You can pass \meta{options} to the node |label node|. For this, you
  provide the options in square brackets before the \meta{angle}. If you
  do so, you need to add braces around the whole argument of the
  |label| option and this is also the case if you have brackets or
  commas or semicolons or anything special in the \meta{text}.
\begin{codeexample}[]
\tikz \node [circle,draw,label={[red]above:X}] {my circle};
\end{codeexample}

\begin{codeexample}[]
\begin{tikzpicture}
  \node [circle,draw,label={[name=label node]above left:$a,b$}] {};
  \draw (label node) -- +(1,1);
\end{tikzpicture}
\end{codeexample}

  If you provide multiple |label| options, then multiple extra label
  nodes are added in the order they are given.

  The following styles influence how labels are drawn:
  \begin{itemize}
    \itemoption{label distance}|=|\meta{distance}
    The \meta{distance} is additionally inserted between the main node
    and the label node. The default is |0pt|.
\begin{codeexample}[]
\tikz[label distance=5mm]
  \node [circle,draw,label=right:X,
                     label=above right:Y,
                     label=above:Z]       {my circle};
\end{codeexample}
  \itemstyle{every label}
    This style is used in every node created by the |label|
    option. The default is |draw=none,fill=none|.
  \end{itemize}
  \itemoption{pin}|=|\opt{|[|\meta{options}|]|}\meta{angle}|:|\meta{text}
  This is option is quite similar to the |label| option, but there is
  one difference: In addition to adding a extra node to the picture, 
  it also adds an edge from this node to the main node. This causes
  the node to look like a pin that has been added to the main node: 
\begin{codeexample}[]
\tikz \node [circle,fill=blue!50,minimum size=1cm,pin=60:$q_0$] {};
\end{codeexample}

  The meaning of the \meta{options} and the \meta{angle} and the
  \meta{text} is exactly the same as for the |node| option. Only, the
  options and styles the influence the way pins look are different:
  \begin{itemize}
    \itemoption{pin distance}|=|\meta{distance}
    This \meta{distance} is used instead of the |label distance| for
    the distance between the main node and the label node. The default
    is |3ex|. 
\begin{codeexample}[]
\tikz[pin distance=1cm]
  \node [circle,draw,pin=right:X,
                     pin=above right:Y,
                     pin=above:Z]       {my circle};
\end{codeexample}
  \itemstyle{every pin}
    This style is used in every node created by the |pin|
    option. The default is |draw=none,fill=none|.
  \itemstyle{every pin edge}
    This style is used in every edge created by the |pin| optins. The
    default is |help lines|.
\begin{codeexample}[]
\tikzstyle{every pin edge}=[<-,shorten <=1pt,snake=snake,line before snake=4pt]
\tikz[pin distance=15mm]
  \node [circle,draw,pin=right:X,
                     pin=above right:Y,
                     pin=above:Z]       {my circle};
\end{codeexample}
  \itemoption{pin edge}|=|\meta{options}
    This option can be used to set the options that are to be used
    in the edge created by the |pin| option. The default is empty.
\begin{codeexample}[]
\tikz[pin distance=10mm]
  \node [circle,draw,pin={[pin edge={blue,thick}]right:X},
                     pin=above:Z]       {my circle};
\end{codeexample}
\begin{codeexample}[]
\tikzstyle{every pin edge}=[]
\tikzstyle{initial}=[pin={[pin distance=5mm,
                           pin edge={<-,shorten <=1pt}]left:start}]
\tikz \node [circle,draw,initial] {my circle};
\end{codeexample}
  \end{itemize}
\end{itemize}


\subsection{Connecting Nodes: Using Nodes as Coordinates}

\label{section-nodes-connecting}

Once you have defined a node and given it a name, you can use this
name to reference it. This can be done in two ways, see also
Section~\ref{section-node-coordinates}. Suppose you have said
|\path(0,0) node(x) {Hello World!};| in order to define a node named |x|. 
\begin{enumerate}
\item
  Once the node |x| has been defined, you can use
  |(x.|\meta{anchor}|)| wherever you would normally use a normal
  coordinate. This will yield the position at which the given
  \meta{anchor} is in the picture. Note that transformations do not
  apply to this coordinate, that is, |(x.north)| will be the northern
  anchor of |x| even if you have said |scale=3| or |xshift=4cm|. This
  is usually what you would expect.
\item
  You can also just use |(x)| as a coordinate. In most cases, this
  gives the same coordinate as |(x.center)|. Indeed, if the |shape| of
  |x| is |coordinate|, then |(x)| and |(x.center)| have exactly the
  same effect.

  However, for most other shapes, some path construction operations like
  |--| try to be ``clever'' when this they are asked to draw a line
  from such a coordinate or to such a coordinate. When you say
  |(x)--(1,1)|, the |--| path operation will not draw a line from the center
  of |x|, but \emph{from the border} of |x| in the direction going
  towards |(1,1)|. Likewise, |(1,1)--(x)| will also have the line
  end on the border in the direction coming from |(1,1)|.

  In addition to |--|, the curve-to path operation |..| and the path
  operations \verb!-|! and \verb!|-! will also handle nodes without
  anchors correctly. Here is an example, see also
  Section~\ref{section-node-coordinates}:
\begin{codeexample}[]
\begin{tikzpicture}
  \path (0,0) node             (x) {Hello World!}
        (3,1) node[circle,draw](y) {$\int_1^2 x \mathrm d x$};

  \draw[->,blue]   (x) -- (y);
  \draw[->,red]    (x) -| node[near start,below] {label} (y);
  \draw[->,orange] (x) .. controls +(up:1cm) and +(left:1cm) .. node[above,sloped] {label} (y);
\end{tikzpicture}
\end{codeexample}
\end{enumerate}




\subsection{Connecting Nodes: Using the Edge Operation}

\label{section-nodes-edges}

The |edge| operation works like a |to| operation that is added after
the main path has been drawn, much like a node is added after the main
path has been drawn. This allows you to have each |edge| to have a
different appearance. As the |node| operation, an |edge| temporarily
suspends the construction of the current path and a new path $p$ is 
constructed. This new path $p$ will be drawn after the main path has
been drawn. Note that $p$ can be totally different from the main
path with respect to its options. Also note that if there are
several |to| and/or |node| operations in the main path, each
creates its own path(s) and they are drawn in the order that they
are encountered on the path.

\begin{pathoperation}{edge}{\opt{|[|\meta{options}|]|}
    \opt{\meta{nodes}} |(|\meta{coordinate}|)|}
  The effect of the |edge| operation is that after the main path the
  following path is added to the picture:
  \begin{quote}
    |\path[every edge,|\meta{options}|] (\tikztostart) |\meta{path}|;|
  \end{quote}
  Here, \meta{path} is the |to path|. Note that, unlike the path added
  by the |to| operation, the |(\tikztostart)| is added before the
  \meta{path} (which is unnecessary for the |to| operation, since this
  coordinate is already part of the main path).

  The |\tikztostart| is the last coordinate on the path just before
  the |edge| operation, just as for the |node| or |to| operations. 
  However, there is one exception to this rule: If the |edge|
  operation is directly preceded by a |node| operation, then this
  just-declared node is the start coordinate (and not, as would
  normally be the case, the coordinate where this just-declared node
  is placed -- a small, but subtle difference). In this regard, |edge|
  differs from both |node| and |to|.
  
  If there are several |edge| operations in a row, the start coordinate
  is the same for all of them as their target coordiantes are not,
  after all, part of the main path. The start coordinate is, thus, the
  coordinate preceding the first |edge| operation. This is
  similar to nodes insofar as the |edge| operation does not modify the
  current path at all. In particular, it does not change the last
  coordinate visited, see the following example:
  
\begin{codeexample}[]
\begin{tikzpicture}
  \node (a) at   (0:1) {$a$};
  \node (b) at  (90:1) {$b$} edge [->]     (a);
  \node (c) at (180:1) {$c$} edge [->]     (a)
                             edge [<-]     (b);
  \node (d) at (270:1) {$d$} edge [->]     (a)
                             edge [dotted] (b)
                             edge [<-]     (c);  
\end{tikzpicture}
\end{codeexample}

  A different way of specifying the above graph using the |edge|
  operation is the following:  

\begin{codeexample}[]
\begin{tikzpicture}
  \foreach \name/\angle in {a/0,b/90,c/180,d/270}
    \node (\name) at (\angle:1) {$\name$};

  \path[->] (b) edge (a)
                edge (c)
                edge [-,dotted] (d)
            (c) edge (a)
                edge (d)
            (d) edge (a);
\end{tikzpicture}
\end{codeexample}

  As can be seen, the path of the |edge| operation inherits the
  options from the main path, but you can locally overrule them.

\begin{codeexample}[]
\begin{tikzpicture}
  \foreach \name/\angle in {a/0,b/90,c/180,d/270}
    \node (\name) at (\angle:1.5) {$\name$};

  \path[->] (b) edge            node[above right]  {$5$}     (a)
                edge                                         (c)
                edge [-,dotted] node[below,sloped] {missing} (d)
            (c) edge                                         (a)
                edge                                         (d)
            (d) edge [red]      node[above,sloped] {very}  
                                node[below,sloped] {bad}     (a);
\end{tikzpicture}
\end{codeexample}

  Instead of |every to|, the style |every edge| is installed at the
  beginning of the main path.
  \begin{itemize}
  \itemstyle{every edge} this is is |draw| by default.
\begin{codeexample}[]
\begin{tikzpicture}
  \tikzstyle{every to}=[draw,dashed]
  \path (0,0) to (3,2);
\end{tikzpicture}
\end{codeexample}
  \end{itemize}
\end{pathoperation}


\subsection{Referencing Nodes Outside the Current Pictures}

\label{section-cross-picture-tikz}

\subsubsection{Referencing a Node in a Different Picture}

It is possible (but not quite trivial) to reference nodes in pictures
other than the current one. This means that you can create a picture
and a node therein and, later, you can draw a line from some other
position to this node. 

To reference nodes in different pictures, proceed as follows:
\begin{enumerate}
\item You need to add the |remember picture| option to all pictures
  that contain nodes that you wish to reference and also to all
  pictures from which you wish to reference a node in another
  picture.
\item You need to add the |overlay| option to paths or to whole
  pictures that contain references to nodes in different
  pictures. (This option switches the computation of the
  bounding box off.)
\item You need to use a driver that supports picture remembering
  (currently, this is only pdf\TeX). With the pdf\TeX\ driver you also
  need to run \TeX\ twice.
\end{enumerate}
(For more details on what is going on behind the scenes, see
Section~\ref{section-cross-pictures-pgf}.)

Let us have a look at the effect of these options.
\begin{itemize}
  \itemoption{remember picture}\opt{|=|\meta{true or false}} This option
  tells \tikzname\ that it should attempt to remember the position of
  the current picture on the page. This attempt may fail depending on
  which backend driver is used. Also, even if remembering works, the
  position may only be available on a second run of \TeX.

  Provided that remebering works, you may consider saying
\begin{codeexample}[code only]
\tikzsytle{every picture}+=[remember picture]
\end{codeexample}
  to make \tikzname\ remember all pictures. This will add one line in
  the |.aux| file for each picture in your document -- which typically
  is not very much. Then, you do not have to worry about remembered
  pictures at all.
  \itemoption{overlay}
  This option is mainly intended for use when nodes in other pictures
  are referenced, but you can also use it in other situations. The
  effect of this option is that everything within the current scope is
  not taken into consideration when the bounding box of the current
  picture is computed.

  You need to specify this option on all paths (or at least on all
  parts of paths) that contain a reference to a node in another
  picture. The reason is that, otherwise, \tikzname\ will attempt to
  make the current picture large enough to encompass \emph{the node in
    the other picture}. However, on a second run of \TeX\ this will
  create an even bigger picture, leading to larger and larger
  pictures. Unless you know what you are doing, I suggest specifying
  the |overlay| option with all pictures that contain references to
  other pictures.
\end{itemize}

Let us now have a look at a few examples. These examples work only if
this document is processed with a driver that supports picture
remembering.

\noindent\begin{minipage}{\textwidth}
Inside the current text we place two pictures, containing nodes named
|n1| and |n2|, using
\begin{codeexample}[code only]
\tikz[remember picture] \node[circle,fill=red!50] (n1) {};
\end{codeexample}
which yields \tikz[remember picture] \node[circle,fill=red!50] (n1)
{};, and
\begin{codeexample}[code only]
\tikz[remember picture] \node[fill=blue!50] (n2) {};
\end{codeexample}
yielding the node \tikz[remember picture] \node[fill=blue!50] (n2)
{};. To connect these nodes, we create another picture using the
|overlay| option and also the |remember picture| option.
\begin{codeexample}[]
\begin{tikzpicture}[remember picture,overlay]
  \draw[->,very thick] (n1) -- (n2);
\end{tikzpicture}
\end{codeexample}
Note that the last picture is seemingly empty. What happens is that it
has zero size and contains an arrow that lies well outside its bounds.
As a last example, we connect a node in another picture to the first
two nodes. Here, we provide the |overlay| option only with the line
that we do not wish to count as part of the picture.
\begin{codeexample}[]
\begin{tikzpicture}[remember picture]
  \node (c) [circle,draw] {Big circle};
  
  \draw [overlay,->,very thick,red,opacity=.5]
    (c) to[bend left] (n1) (n1) -| (n2);
\end{tikzpicture}
\end{codeexample}
\end{minipage}


\subsubsection{Referencing the Current Page Node -- Absolute Positioning}

There is a special node called |current page| that can be used to
access the current page. It is a node of shape rectangle whose
|south west| anchor is the lower left corner of the page and whose
|north east| anchor is the upper right corner of the page. While this
node is handled in a special way internally, you can reference it as
if it were defined in some remembered picture other than the current
one. Thus, by giving the |remembered picture| and the |overlay|
options to a picture, you can position nodes \emph{absolutely} on a
page.

The first example places some text in the lower left corner of the
current page:
\begin{codeexample}[]
\begin{tikzpicture}[remember picture,overlay]
  \node [xshift=1cm,yshift=1cm] at (current page.south west)
        [text width=7cm,fill=red!20,rounded corners,above right]
  {
    This is an absolutely positioned text in the
    lower left corner. No shipout-hackery is used.
  };
\end{tikzpicture}
\end{codeexample}

The next example adds a circle in the middle of the page.
\begin{codeexample}[]
\begin{tikzpicture}[remember picture,overlay]
  \draw [line width=1mm,opacity=.25]
    (current page.center) circle (3cm);
\end{tikzpicture}
\end{codeexample}

The final example overlays some text over the page (depending on where
this example is found on the page, the text may also be behind the
page).
\begin{codeexample}[]
\begin{tikzpicture}[remember picture,overlay]
  \node [rotate=60,scale=10,text opacity=0.2]
    at (current page.center) {Example};
\end{tikzpicture}
\end{codeexample}



\subsection{Predefined Shapes}

\label{section-nodes-predefined}

\label{section-the-shapes}

\pgfname\ and \tikzname\ define three shapes, by default:
\begin{itemize}
\item
  |rectangle|,
\item
  |circle|, and
\item
  |coordinate|.
\end{itemize}
By loading library packages, you can define more shapes like ellipses
or diamonds; see the library section for the complete list of shapes.

The exact behaviour of these shapes differs, shapes defined for more
special purposes (like a, say, transistor shape) will have even more
custom behaviors. However, there are some options that apply to most
shapes:
\begin{itemize}
  \itemoption{inner sep}|=|\meta{dimension}
  An additional (invisible) separation space of \meta{dimension} will
  be added inside the shape, between the text and the shape's
  background path. The effect is as if you had added appropriate
  horizontal and vertical skips at the beginning and end of the text
  to make it a bit ``larger.'' The default |inner sep| is the size of
  a normal space. 

\begin{codeexample}[]
\begin{tikzpicture}
  \draw (0,0)     node[inner sep=0pt,draw] {tight}
        (0cm,2em) node[inner sep=5pt,draw] {loose}
        (0cm,4em) node[fill=examplefill]   {default};
\end{tikzpicture}
\end{codeexample}
  \itemoption{inner xsep}|=|\meta{dimension}
  Specifies the inner separation in the $x$-direction, only.
  \itemoption{inner ysep}|=|\meta{dimension}
  Specifies the inner separation in the $y$-direction, only.
  
  \itemoption{outer sep}|=|\meta{dimension}
  This option adds an additional (invisible) separation space of
  \meta{dimension} outside the background path. The main effect of
  this option is that all anchors will move a little ``to the
  outside.'' 

  The default for this option is half the line width. When the default
  is used and when the background path is draw, the anchors will lie
  exactly on the ``outside border'' of the path (not on the path
  itself). When the shape is filled, but not drawn, this may not be
  desirable. In this case, the |outer sep| should be set to zero
  point. 
\begin{codeexample}[]
\begin{tikzpicture}
  \draw[line width=5pt]
    (0,0) node[outer sep=0pt,fill=examplefill]     (f) {filled}
    (2,0) node[inner sep=.5\pgflinewidth+2pt,draw] (d) {drawn};

  \draw[->] (1,-1) -- (f);
  \draw[->] (1,-1) -- (d);  
\end{tikzpicture}
\end{codeexample}
  \itemoption{outer xsep}|=|\meta{dimension}
  Specifies the outer separation in the $x$-direction, only.
  \itemoption{outer ysep}|=|\meta{dimension}
  Specifies the outer separation in the $y$-direction, only.

  \itemoption{minimum height}|=|\meta{dimension}
  This option ensures that the height of the shape (including the
  inner, but ignoring the outer separation) will be at least
  \meta{dimension}. Thus, if the text plus the inner separation is not
  at least as large as \meta{dimension}, the shape will be enlarged 
  appropriately. However, if the text is already larger than
  \meta{dimension}, the shape will not be shrunk.
\begin{codeexample}[]
\begin{tikzpicture}
  \draw (0,0) node[minimum height=1cm,draw] {1cm}
        (2,0) node[minimum height=0cm,draw] {0cm};
\end{tikzpicture}
\end{codeexample}

  \itemoption{minimum width}|=|\meta{dimension}
  same as |minimum height|, only for the width.
\begin{codeexample}[]
\begin{tikzpicture}
  \draw (0,0) node[minimum height=2cm,minimum width=3cm,draw] {$3 \times 2$};
\end{tikzpicture}
\end{codeexample}
  \itemoption{minimum size}|=|\meta{dimension}
  sets both the minimum height and width at the same time.
\begin{codeexample}[]
\begin{tikzpicture}
  \draw (0,0)  node[minimum size=2cm,draw] {square};
  \draw (0,-2) node[minimum size=2cm,draw,circle] {circle};
\end{tikzpicture}
\end{codeexample}

  \itemoption{aspect}|=|\meta{aspect ratio}
  sets a desired aspect ratio for the shape. For the |diamond| shape,
  this option sets the ratio between width and height of the shape.
\begin{codeexample}[]
\begin{tikzpicture}
  \draw (0,0)  node[aspect=1,diamond,draw] {aspect 1};
  \draw (0,-2) node[aspect=2,diamond,draw] {aspect 2};
\end{tikzpicture}
\end{codeexample}
\end{itemize}

\label{section-tikz-coordinate-shape}
The |coordinate| shape is handled in a special way by \tikzname. When
a node |x| whose shape is |coordinate| is used as a coordinate |(x)|,
this has the same effect as if you had said |(x.center)|. None  of the
special ``line shortening rules'' apply in this case. This can be
useful since, normally, the line shortening causes paths to be
segmented and they cannot be used for filling. Here is an example that
demonstrates the difference: 
\begin{codeexample}[]
\begin{tikzpicture}
  \tikzstyle{every node}=[draw]
  \path[yshift=1.5cm,shape=rectangle]
    (0,0) node(a1){} (1,0) node(a2){}
    (1,1) node(a3){} (0,1) node(a4){};
  \filldraw[fill=examplefill] (a1) -- (a2) -- (a3) -- (a4);
  
  \path[shape=coordinate]
    (0,0) coordinate(b1) (1,0) coordinate(b2)
    (1,1) coordinate(b3) (0,1) coordinate(b4);
  \filldraw[fill=examplefill] (b1) -- (b2) -- (b3) -- (b4);
\end{tikzpicture}
\end{codeexample}


\subsection{Executing Code After Nodes}

\label{section-nodes-executing}

It is possible to add a path right after a node using the option
|after node path|. The idea is that a style might use this option to
add some additional stuff to the node that has just been typeset.

\begin{itemize}
  \itemoption{after node path}|=|\meta{path}
  The \meta{path} is added to the main path right after the node, as
  if you had given the path thereafter. This option can only be given
  inside the option list of a node and multiple calls of this option
  accumulate. 

  Inside the \meta{path} you have access to the node that has just
  been created via the macro \declare{|\tikzlastnode|}.
\begin{codeexample}[]
\tikz
  \draw node [draw,after node path={(\tikzlastnode) circle (2cm)}]
    {hello};
\end{codeexample}

  Note that in the above example, if we had written |\path| instead of
  |\draw|, the circle would not have been drawn since the circle is
  part of the main path, not part of the node itself. 
\end{itemize}

\begin{command}{\tikzaddafternodepathoption\marg{code}}
  This command allows you to specify that the \meta{code} should be
  executed at the beginning of the |after node path| of the current
  node. The code will also be executed immediately, but also again at
  the beginning of an |after node path|.
\end{command}



%%% Local Variables: 
%%% mode: latex
%%% TeX-master: "pgfmanual"
%%% End: 

% Copyright 2010 by Till Tantau
% Copyright 2011 by Jannis Pohlmann
%
% This file may be distributed and/or modified
%
% 1. under the LaTeX Project Public License and/or
% 2. under the GNU Free Documentation License.
%
% See the file doc/generic/pgf/licenses/LICENSE for more details.

\section{Specifying Graphs}
\label{section-library-graphs}


\subsection{Overview}

\tikzname\ offers a powerful path command for specifying how the nodes
in a graph are connected by edges and arcs: The |graph| path
command, which becomes available when you load the |graphs| library.

\begin{tikzlibrary}{graphs}
  The package must be loaded to use the |graph| path command.
\end{tikzlibrary}

In this section, by \emph{graph} we refer to a set of nodes together
with some edges (sometimes also called arcs, in case they are
directed) such as the following:

\begin{codeexample}[]
\tikz \graph { a -> {b, c} -> d };  
\end{codeexample}

\begin{codeexample}[]
\tikz \graph {
  subgraph I_nm [V={a, b, c}, W={1,...,4}];

  a -> { 1, 2, 3 };
  b -> { 1, 4 };
  c -> { 2 [>green!75!black], 3, 4 [>red]}
};
\end{codeexample}

\begin{codeexample}[]
\tikz
  \graph [nodes={draw, circle}, clockwise, radius=.5cm, empty nodes, n=5] {
    subgraph I_n [name=inner] --[complete bipartite]
    subgraph I_n [name=outer]
  };
\end{codeexample}

\begin{codeexample}[]
\tikz
  \graph [nodes={draw, circle}, clockwise, radius=.75cm, empty nodes, n=8] {
    subgraph C_n [name=inner] <->[shorten <=1pt, shorten >=1pt]
    subgraph C_n [name=outer]
  };
\end{codeexample}

\begin{codeexample}[width=6.6cm]
\tikz [>={To[sep]}, rotate=90, xscale=-1,
       mark/.style={fill=black!50}, mark/.default=]
  \graph [trie, simple, 
          nodes={circle,draw},
          edges={nodes={
              inner sep=1pt, anchor=mid,
              fill=graphicbackground}}, % yellowish background
          put node text on incoming edges]
    {
      root[mark] -> {
        a -> n -> {
          g [mark],
          f -> a -> n -> g [mark]
        },
        f -> a -> n -> g [mark],
        g[mark],
        n -> {
          g[mark],
          f -> a -> n -> g[mark]
        }
      },
      { [edges=red] % highlight one path
        root -> f -> a -> n
      }    
    };
\end{codeexample}

The nodes of a graph are normal \tikzname\ nodes, the edges are
normal lines drawn between nodes. There is nothing in the |graph|
library that you cannot do using the normal |\node| and the |edge|
commands. Rather, its purpose is to offer a concise and powerful way of
\emph{specifying} which nodes are present 
and how they are connected. The |graph| library only offers simple
methods for specifying \emph{where} the nodes should be shown, its
main strength is in specifying which nodes and edges are present in 
principle. The problem of finding ``good positions on the canvas'' for
the nodes of a graph is left to \emph{graph drawing algorithms}, which
are covered in Part~\ref{part-gd} of this manual and which
are not part of the |graphs| library; indeed, these algorithms can be
used also with graphs specified using |node| and |edge|
commands. \ifluatex
As an example, consider the above drawing of a trie, which is drawn
without using the graph drawing libraries. Its layout can be 
somewhat improved by loading the |layered| graph drawing library,
saying |\tikz[layered layout,...|, and then using Lua\TeX, resulting
in the following drawing of the same graph:
\medskip

\tikz [layered layout, >={To[sep]}, rotate=90, xscale=-1,
       mark/.style={fill=black!50}, mark/.default=]
  \graph [trie, simple, sibling distance=8mm,
          nodes={circle,draw},
          edges={nodes={
              inner sep=1pt, anchor=mid, fill=white}},
          put node text on incoming edges]
    {
      root[mark] -> {
        a -> n -> {
          g [mark],
          f -> a -> n -> g [mark]
        },
        f -> a -> n -> g [mark],
        g[mark],
        n -> {
          g[mark],
          f -> a -> n -> g[mark]
        }
      },
      { [edges=red] % highlight one path
        root -> f -> a -> n
      }    
    };
\medskip
\fi

The |graph| library uses a syntax that is quite different from the
normal \tikzname\ syntax for specifying nodes. The reason for this is
that for many medium-sized graphs it can become quite cumbersome to
specify all the nodes using |\node| repeatedly and then using a great
number of |edge| command; possibly with complicated |\foreach|
statements. Instead, the syntax of the |graph| library is loosely
inspired by the \textsc{dot} format, which is quite useful for
specifying medium-sized graphs, with some extensions on top.



\subsection{Concepts}

The present section aims at giving a quick overview of the main
concepts behind the |graph| command. The exact syntax is explained in
more detail in later sections.


\subsubsection{Concept: Node Chains}

The basic way of specifying a graph is to write down a \emph{node
  chain} as in the following example: 

\begin{codeexample}[]
\tikz [every node/.style = draw]
  \graph { foo -> bar -> blub };  
\end{codeexample}

As can be seen, the text |foo -> bar -> my node| creates three nodes,
one with the text |foo|, one with |bar| and one with the text
|blub|. These nodes are connected by arrows, which are caused by
the |->| between the node texts. Such a sequence of node texts and
arrows between them is called a \emph{chain} in the following. 

Inside a graph there can be more than one chain:

\begin{codeexample}[]
\tikz \graph {
  a -> b -> c;
  d -> e -> f;
  g -> f;
};  
\end{codeexample}

Multiple chains are separated by a semicolon or a comma (both have
exactly the same effect). As the example shows, when a node text is
seen for the second time, instead of creating a new node, a connection
is created to the already existing node.

When a node like |f| is created, both the node name and the node text
are identical by default. This is not always desirable and can be
changed by using the |as| key or by providing another text after
a slash:

\begin{codeexample}[]
\tikz \graph {
  x1/$x_1$ -> x2 [as=$x_2$, red] -> x34/{$x_3,x_4$};
  x1 -> [bend left] x34;
};  
\end{codeexample}

When you wish to use a node name that contains special symbols like
commas or dashes, you must surround the node name by quotes. This
allows you to use quite arbitrary text as a ``node name'':
\begin{codeexample}[]
\tikz \graph {
  "$x_1$" -> "$x_2$"[red] -> "$x_3,x_4$";
  "$x_1$" ->[bend left] "$x_3,x_4$";
};  
\end{codeexample}


\subsubsection{Concept: Chain Groups}

Multiple chains that are separated by a semicolon or a comma and that
are surrounded by curly braces form what will be called a \emph{chain
  group} or just a \emph{group}. A group in itself has no special
effect. However, things get interesting when you write down a node or
even a whole group and connect it to another group. In this case, the
``exit points'' of the first node or group get connected to the
``entry points'' of the second node or group:

\begin{codeexample}[]
\tikz \graph {
  a -> {
    b -> c,
    d -> e
  } -> f
};  
\end{codeexample}

Chain groups make it easy to create tree structures:

\begin{codeexample}[width=10cm]
\tikz
  \graph [grow down,
          branch right=2.5cm] {
  root -> {
    child 1,
    child 2 -> {
      grand child 1,
      grand child 2
    },
    child 3 -> {
      grand child 3
    }
  }
};
\end{codeexample}

As can be seen, the placement is not particularly nice by default, use
the algorithms from the graph drawing libraries to get a better
layout. For instance, adding |tree layout| to the above code results in the
following somewhat more pleasing rendering:
\ifluatex
\medskip

\tikz \graph [grow down, branch right=2.5cm, tree layout] {
  root -> {
    child 1,
    child 2 -> {
      grand child 1,
      grand child 2
    },
    child 3 -> {
      grand child 3
    }
  }
};
\else
(You need to use Lua\TeX\ to typeset this graphic.)
\fi

\subsubsection{Concept: Edge Labels and Styles}

When connectors like |->| or |--| are used to connect nodes or whole
chain groups, one or more edges will typically be created. These edges
can be styles easily by providing options in square brackets directly
after these connectors:

\begin{codeexample}[]
\tikz \graph {
  a ->[red] b --[thick] {c, d};
};
\end{codeexample}

Using the quotes syntax, see Section~\ref{section-label-quotes},
you can even add labels to the edges easily by putting the labels in
quotes: 

\begin{codeexample}[]
\tikz \graph {
  a ->[red, "foo"] b --[thick, "bar"] {c, d};
};
\end{codeexample}

For the first edge, the effect is as desired, however 
between |b| and the group |{c,d}| two edges are inserted and the
options |thick| and the label option |"bar"| is applied to both of
them. While this is the correct and consistent behaviour, we typically
might wish to specify different labels for the edge going from |b| to
|c| and the edge going from |b| to |d|. To achieve this effect, we can
no longer specify the label as part of the options of |--|. Rather, we
must pass the desired label to the nodes |c| and |d|, but we must
somehow also indicate that these options actually ``belong'' to the
edge ``leading to'' to nodes. This is achieved by preceding the
options with a greater-than sign:

\begin{codeexample}[]
\tikz \graph {
  a -> b -- {c [> "foo"], d [> "bar"']};
};
\end{codeexample}

Symmetrically, preceding the options by |<| causes the options and
labels to apply to the ``outgoing'' edges of the node:

\begin{codeexample}[]
\tikz \graph {
  a [< red] -> b -- {c [> blue], d [> "bar"']};
};
\end{codeexample}

This syntax allows you to easily create trees with special edge
labels as in the following example of a treap:

\begin{codeexample}[]
\tikz 
  \graph [edge quotes={fill=white,inner sep=1pt},
          grow down, branch right, nodes={circle,draw}] {
    "" -> h [>"9"] -> {
      c [>"4"] -> {
        a [>"2"],
        e [>"0"]
      },
      j [>"7"]
    }
  };
\end{codeexample}



\subsubsection{Concept: Node Sets}

When you write down some node text inside a |graph| command, a new
node is created by default unless this node has already been created
inside the same |graph| command. In particular, if a node has
already been declared outside of the current |graph| command, a new
node of the same name gets created.

This is not always the desired behaviour. Often, you may wish to make
nodes part of a graph than have already been defined prior to the use
of the |graph| command. For this, simply surround a node name by
parentheses. This will cause a reference to be created to an already
existing node:

\begin{codeexample}[]
\tikz {
  \node (a) at (0,0) {A};
  \node (b) at (1,0) {B};
  \node (c) at (2,0) {C};
  
  \graph { (a) -> (b) -> (c) };
}
\end{codeexample}

You can even go a step further: A whole collection of nodes can all be
flagged to belong to a \emph{node set} by adding the option
|set=|\meta{node set name}. Then, inside a |graph| command, you can
collectively refer to these nodes by surrounding the node set name in
parentheses: 

\begin{codeexample}[]
\tikz [new set=my nodes] {
  \node [set=my nodes, circle,    draw] at (1,1)   {A};
  \node [set=my nodes, rectangle, draw] at (1.5,0) {B};
  \node [set=my nodes, diamond,   draw] at (1,-1)  {C};
  \node (d)           [star,      draw] at (3,0)   {D};

  \graph { X -> (my nodes) -> (d) };
}
\end{codeexample}


\subsubsection{Concept: Graph Macros}

Often, a graph will consist -- at least in parts -- of standard
parts. For instance, a graph might contain a cycle of certain size or
a path or a clique. To facilitate specifying such graphs, you can
define a \emph{graph macro}. Once a graph macro has been defined, you
can use the name of the graph to make a copy of the graph part of the
graph currently being specified:

\begin{codeexample}[]
\tikz \graph { subgraph K_n [n=6, clockwise] };
\end{codeexample}

\begin{codeexample}[]
\tikz \graph { subgraph C_n [n=5, clockwise] -> mid };
\end{codeexample}

The library |graphs.standard| defines a number of such graphs,
including the complete clique $K_n$ on $n$ nodes, the complete
bipartite graph $K_{n,m}$ with shores sized $n$ and $m$, the cycle
$C_n$ on $n$ nodes, the path $P_n$ on $n$ nodes, and the independent
set $I_n$ on $n$ nodes.


\subsubsection{Concept: Graph Expressions and Color Classes}

When a graph is being constructed using the |graph| command, it is
constructed recursively by uniting smaller graphs to larger
graphs. During this recursive union process the nodes
of the graph get implicitly \emph{colored} (conceptually) and you can
also explicitly assign colors to individual nodes and even change the
colors as the graph is being specified. All nodes having the same
color form what is called a \emph{color class}.

The power of color class is that special \emph{connector operators}
allow you to add edges between nodes having certain colors. For instance,
saying |clique=red| at the beginning of a group will
cause all nodes that have been flagged as being (conceptually) ``red''
to be connected as a clique. Similarly, saying
|complete bipartite={red}{green}| will cause edges to be added
between all red and all green nodes. More advanced connectors, like
the |butterfly| connector, allow you to add edges between color
classes in a fancy manner.

\begin{codeexample}[]
\tikz [x=8mm, y=6mm, circle]
  \graph [nodes={fill=blue!70}, empty nodes, n=8] {
    subgraph I_n [name=A] --[butterfly={level=4}]
    subgraph I_n [name=B] --[butterfly={level=2}]
    subgraph I_n [name=C] --[butterfly]
    subgraph I_n [name=D] -- 
    subgraph I_n [name=E]  
  };
\end{codeexample}



\subsection{Syntax of the Graph Path Command}

\subsubsection{The Graph Command}

In order to construct a graph, you should use the |graph| path
command, which can be used anywhere on a path at any place where
you could also use a command like, say, |plot| or |--|.

\begin{command}{\graph}
  Inside a |{tikzpicture}| this is an abbreviation for |\path graph|.
\end{command}

\begin{pathoperation}{graph}{\opt{\oarg{options}}\meta{group specification}}
  When this command is encountered on a path, the construction of the
  current path is suspended (similarly to an |edge| command or a
  |node| command). In a local scope, the \meta{options} are first
  executed with the key path |/tikz/graphs| using the following
  command:
  \begin{command}{\tikzgraphsset\marg{options}}
    Executes the \meta{options} with the path prefix |/tikz/graphs|.    
  \end{command}
  Apart from the keys explained in the following, further permissible
  keys will be listed during the course of the rest of this section.

  \begin{stylekey}{/tikz/graphs/every graph}
    This style is executed at the beginning of every |graph| path
    command prior to the \meta{options}.
  \end{stylekey}

  Once the scope has been set up and once the \meta{options} have been
  executed, a parser starts to parse the \meta{group
    specification}. The exact syntax of such a group specification
  in explained in detail in
  Section~\ref{section-library-graphs-group-spec}. Basically, a group
  specification is a list of chain specifications, separated by commas
  or semicolons.

  Depending on the content of the \meta{group specification}, two
  things will happen:
  \begin{enumerate}
  \item A number of new nodes may be created. These will be inserted
    into the picture in the same order as if they had been created
    using multiple |node| path commands at the place where the |graph|
    path command was used. In other words, all nodes created in a
    |graph| path command will be painted on top of any nodes created
    earlier in the path and behind any nodes created later in the
    path. Like normal nodes, the newly created nodes always lie on top
    of the path that is currently being created (which is often
    empty, for instance when the |\graph| command is used).
  \item Edges between the nodes may be added. They are added in the
    same order as if the |edge| command had been used at the position
    where the |graph| command is being used.
  \end{enumerate}

  Let us now have a look at some common keys that may be used inside
  the \meta{options}:
  \begin{key}{/tikz/graphs/nodes=\meta{options}}
    This option causes the \meta{options} to be applied to each newly
    created node inside the \meta{group specification}.
    \begin{codeexample}[]
\tikz \graph [nodes=red] { a -> b -> c };      
    \end{codeexample}
    Multiple uses of this key accumulate.
  \end{key}
  \begin{key}{/tikz/graphs/edges=\meta{options}}
    This option causes the \meta{options} to be applied to each newly
    created edge inside the \meta{group specification}.
    \begin{codeexample}[]
\tikz \graph [edges={red,thick}] { a -> b -> c };      
    \end{codeexample}
    Again, multiple uses of this key accumulate.
  \end{key}
  \begin{key}{/tikz/graphs/edge=\meta{options}}
    This is an alias for |edges|.
  \end{key}
  
  \begin{key}{/tikz/graphs/edge node=\meta{node specification}}
    This key specifies that the \meta{node specification} should be
    added to each newly created edge as an implicitly placed node. 
    \begin{codeexample}[]
\tikz \graph [edge node={node [red, near end] {X}}] { a -> b -> c };      
    \end{codeexample}
    Again, multiple uses of this key accumulate.
    \begin{codeexample}[]
\tikz \graph [edge node={node [near end] {X}},
              edge node={node [near start] {Y}}] { a -> b -> c };      
    \end{codeexample}
  \end{key}
  
  \begin{key}{/tikz/graphs/edge label=\meta{text}}
    This key is an abbreviation for
    |edge node=node[auto]{|\meta{text}|}|. The net effect is that the
    |text| is placed next to the newly created edges.
    \begin{codeexample}[]
\tikz \graph [edge label=x] { a -> b -> {c,d} };      
    \end{codeexample}
  \end{key}
  
  \begin{key}{/tikz/graphs/edge label'=\meta{text}}
    This key is an abbreviation for
    |edge node=node[auto,swap]{|\meta{text}|}|. 
    \begin{codeexample}[]
\tikz \graph [edge label=out, edge label'=in]
  { subgraph C_n [clockwise, n=5] };      
    \end{codeexample}
  \end{key}  
\end{pathoperation}


\subsubsection{Syntax of Group Specifications}
\label{section-library-graphs-group-spec}

A \meta{group specification} inside a |graph| path command has the
following syntax:
\begin{quote}
  |{|\opt{\oarg{options}}\meta{list of chain specifications}|}|
\end{quote}
The \meta{chain specifications} must contain chain specifications,
whose syntax is detailed in the next section, separated by either
commas or semicolons; you can freely mix them.
It is permissible to use empty lines (which are mapped to |\par|
commands internally) to structure the chains visually, they are simply
ignored by the parser. 

In the following example, the group specification consists of three
chain specifications, namely of |a -> b|, then |c| alone, and finally
|d -> e -> f|:
\begin{codeexample}[]
\tikz \graph {
  a -> b,
  c;

  d -> e -> f
};
\end{codeexample}
The above has the same effect as the more compact group specification
|{a->b,c,d->e->f}|.

Commas are used to detect where chain specifications end. However, you
will often wish to use a comma also inside the options of a single
node like in the following example:

\begin{codeexample}[]
\tikz \graph {
  a [red, draw] -> b [blue, draw],
  c [brown, draw, circle]
};
\end{codeexample}

Note that the above example works as expected: The first comma inside
the option list of |a| is \emph{not} interpreted as the end of the
chain specification ``|a [red|''. Rather, commas inside square
brackets are ``protected'' against being interpreted as separators of
group specifications.

The \meta{options} that can be given at the beginning of a group
specification are local to the group. They are executed with the path
prefix |/tikz/graphs|. Note that for the outermost group specification
of a graph it makes no difference whether the options are passed to
the |graph| command or whether they are given at the beginning of this
group. However, for groups nested inside other groups, it does make a
difference:

\begin{codeexample}[]
\tikz \graph {
  a -> { [nodes=red] % the option is local to these nodes:
    b, c
  } ->
  d
};
\end{codeexample}

\medskip
\textbf{Using foreach.}
There is special support for the |\foreach| statement inside groups:
You may use the statement inside a group
specification at any place where a \meta{chain specification} would
normally go. In this case, the |\foreach| statement is executed and
for each iteration the content of the statement's body is treated and
parsed as a new chain specification.

\begin{codeexample}[]
\tikz \graph [math nodes, branch down=5mm] {
  a -> { 
    \foreach \i in {1,2,3} {
      a_\i -> { x_\i, y_\i }
    },
    b
  }
};
\end{codeexample}

\medskip
\textbf{Using macros.}
In some cases you may wish to use macros and \TeX\ code to compute
which nodes and edges are present in a group. You cannot use macros in
the normal way inside a graph specification since the parser does not
expand macros as it scans for the start and end of groups and node
names. Rather, only after commas, semicolons, and hyphens have already
been detected and only after all other parsing decisions have been
made will macros be expanded. At this point, when a macro expands to,
say |a,b|, this will not result in two nodes to be created since the
parsing is already done. For these reasons, a special key is needed to
make it possible to ``compute'' which nodes should be present in a
group.

\begin{key}{/tikz/graph/parse=\meta{text}}
  This key can only be used inside the \meta{options} of a \meta{group
    specification}. Its effect is that the \meta{text} is inserted at
  the beginning of the current group as if you had entered it there.
  Naturally, it makes little sense to just write down some static
  \meta{text} since you could just as well directly place it at the
  beginning of the group. The real power of this command stems from
  the fact that the keys mechanism allows you to say, for instance,
  |parse/.expand once| to insert the text stored in some macro into
  the group.
\begin{codeexample}[]
\def\mychain{ a -> b -> c; }  
\tikz \graph { [parse/.expand once=\mychain] d -> e };
\end{codeexample}
  In the following, more fancy example we use a loop to create a chain
  of dynamic length.
\begin{codeexample}[]
\def\mychain#1{
  \def\mytext{1}
  \foreach \i in {2,...,#1} {
    \xdef\mytext{\mytext -> \i}
  }
}
\tikzgraphsset{my chain/.style={
    /utils/exec=\mychain{#1},
    parse/.expand once=\mytext}
}
\tikz \graph { [my chain=4] };
\end{codeexample}
  Multiple uses of this key accumulate, that is, all the \text{text}s
  given in the different uses is inserted in the order it is given.
\end{key}


\subsubsection{Syntax of Chain Specifications}

A \meta{chain specification} has the following syntax: It consists of
a sequence of \meta{node specifications}, where subsequent node 
specifications are separated by \meta{edge specifications}. Node
specifications, which typically consist of some text, are discussed in
the next section in more detail. They normally represent a single node
that is either newly created or exists already, but they may also
specify a whole set of nodes.

An \meta{edge specification} specifies \emph{which} of the node(s) to
the left of the edge specification should be connected to which
node(s) to the right of it and it also specifies in which direction
the connections go. In the following, we only discuss how the
direction is chosen, the powerful mechanism behind choosing which
nodes should be connect is detailed in 
Section~\ref{section-library-graphs-color-classes}.

The syntax of an edge specification is always one of the following
five possibilities: 

\begin{quote}
  |->| \opt{\oarg{options}}\\
  |--| \opt{\oarg{options}}\\
  |<-| \opt{\oarg{options}}\\
  |<->| \opt{\oarg{options}}\\
  |-!-| \opt{\oarg{options}}
\end{quote}

The first four correspond to a directed edge, an undirected edge, a
``backward'' directed edge, and a bidirected edge, respectively. The
fifth edge specification means that there should be no edge (this
specification can be used together with the |simple| option to remove
edges that have previously been added, see
Section~\ref{section-library-graphs-simple}). 

Suppose the nodes \meta{left nodes} are to the left of the \meta{edge
  specification} and \meta{right nodes} are to the right and suppose
we have written |->| between them. Then the following happens:
\begin{enumerate}
\item The \meta{options} are executed (inside a local scope) with the
  path |/tikz/graphs|.  These options may setup the connector algorithm
  (see below) and may also use keys like |edge| or |edge label| to
  specify how the edge should look like. As a convenience, whenever an
  unknown key is encountered for the path |/tikz/graphs|, the key is
  passed to the |edge| key. This means that you can directly use
  options like |thick| or |red| inside the \meta{options} and they
  will apply to the edge as expected.
\item The chosen connector algorithm, see 
  Section~\ref{section-library-graphs-color-classes}, is used to
  compute from which of the \meta{left nodes} an edge should lead to
  which of the \meta{right nodes}. Suppose that $(l_1,r_1)$, \dots,
  $(l_n,r_n)$ is the list of node pairs that result (so there should
  be an edge between $l_1$ and $r_1$ and another edge between $l_2$
  and $r_2$ and so on).
\item For each pair $(l_i,r_i)$ an edge is created. This is done by
  calling the following key (for the edge specification |->|, other
  keys are executed for the other kinds of specifications):
  \begin{key}{/tikz/graphs/new ->=\marg{left node}\marg{right node}\marg{edge options}\marg{edge nodes}}
    This key will be called for a |->| edge specification with the
    following four parameters: 
    \begin{enumerate}
    \item \meta{left node} is the name of the ``left'' node, that is,
      the name of $l_i$.
    \item \meta{right node} is the name of the right node.
    \item \meta{edge options} are the accumulated options from all
      calls of |/tikz/graph/edges| in groups that surround the edge
      specification.
    \item \meta{edge nodes} is text like |node {A} node {B}| that
      specifies some nodes that should be put as labels on the edge
      using \tikzname's implicit positioning mechanism.
    \end{enumerate}
    By default, the key executes the following code:
    \begin{quote}
      |\path [->,every new ->]|\\
      \hbox{}\quad|(|\meta{left node}|\tikzgraphleftanchor) edge [|%
      \meta{edge options}|]| \meta{edge nodes}||\\
      \hbox{}\quad|(|\meta{right node}|\tikzgraphrightanchor);|
    \end{quote}
    You are welcome to change the code underlying the key.
    \begin{stylekey}{/tikz/every new ->}
      This key gets executed by default for a |new ->|.
    \end{stylekey}
  \end{key}
  \begin{key}{/tikz/graphs/left anchor=\meta{anchor}}
    This anchor is used for the node that is to the left of an edge
    specification. Setting this anchor to the empty string means that
    no special anchor is used (which is the default). The
    \meta{anchor} is stored in the macro |\tikzgraphleftanchor| with a
    leading dot.
    \begin{codeexample}[]
\tikz \graph {
  {a,b,c} -> [complete bipartite] {e,f,g}
};
    \end{codeexample}
    \begin{codeexample}[]
\tikz \graph [left anchor=east, right anchor=west] {
  {a,b,c} -- [complete bipartite] {e,f,g}
};
    \end{codeexample}
  \end{key}
  \begin{key}{/tikz/graphs/right anchor=\meta{anchor}}
    Works like |left anchor|, only for |\tikzgraphrightanchor|.
  \end{key}
  For the other three kinds of edge specifications, the following keys
  will be called:
  \begin{key}{/tikz/graphs/new --=\marg{left node}\marg{right node}\marg{edge options}\marg{edge nodes}}
    This key is called for |--| with the same parameters as above. The
    only difference in the definition is that in the |\path| command
    the |->| gets replaced by |-|.
    \begin{stylekey}{/tikz/every new --}
    \end{stylekey}
  \end{key}
  \begin{key}{/tikz/graphs/new <->=\marg{left node}\marg{right node}\marg{edge options}\marg{edge nodes}}
    Called for |<->| with the same parameters as above. The |->| is
    replaced by |<-|
    \begin{stylekey}{/tikz/every new <->}
    \end{stylekey}
  \end{key}
  \begin{key}{/tikz/graphs/new <-=\marg{left node}\marg{right node}\marg{edge options}\marg{edge nodes}}
    Called for |<-| with the same parameters as above.%
    \footnote{You might
      wonder why this key is needed: It seems more logical at first
      sight to just call |new edge directed| with swapped first
      parameters. However, a positioning
      algorithm might wish to take the fact into account that an edge is
      ``backward'' rather than ``forward''  in order to
      improve the layout. Also, different arrow heads might be used.}
    \begin{stylekey}{/tikz/every new <-}
    \end{stylekey}
  \end{key}
  \begin{key}{/tikz/graphs/new -\protect\exclamationmarktext-=\marg{left node}\marg{right node}\marg{edge options}\marg{edge nodes}}
    Called for |-!-| with the same parameters as above. Does nothing
    by default.
  \end{key}
\end{enumerate}

Here is an example that shows the default rendering of the different
edge specifications:

\begin{codeexample}[]
\tikz \graph [branch down=5mm] {
  a -> b;
  c -- d;
  e <- f;
  g <-> h;
  i -!- j;
};  
\end{codeexample}



\subsubsection{Syntax of Node Specifications}

\label{section-library-graphs-node-spec}

Node specifications are the basic building blocks of a graph
specification. There are three different possible kinds of node
specifications, each of which has a different syntax: 

\begin{description}
\item[Direct Node Specification]
  \ \\
  \opt{|"|}\meta{node name}\opt{|"|}\opt{|/|\opt{|"|}\meta{text}\opt{|"|}} \opt{\oarg{options}}\\
  (note that the quotation marks are optional and only needed when the
  \meta{node name} contains special symbols)
\item[Reference Node Specification]
  \ \\
  |(|\meta{node name or node set name}|)|
\item[Group Node Specification]
  \ \\
  \meta{group specification}
\end{description}

The rule for determining which of the possible kinds is meant is
as follows: If the node specification starts with an opening
parenthesis, a reference node specification is meant; if it starts
with an opening curly brace, a group specification is meant; and in 
all other cases a direct node specification is meant.

\medskip
\textbf{Direct Node Specifications.} If after reading the first symbol
of a node specification is has been detected to be \emph{direct},
\tikzname\ will collect all text up to the next edge
specification and store it as the \meta{node name}; however, square
brackets are used to indicate options and a slash ends the \meta{node
  name} and start a special \meta{text} that is used as a 
rendering text instead of the original \meta{node name}.

Due to the way the parsing works and due to the restrictions on node
names, most special characters are forbidding inside the \meta{node
  name}, including commas, semicolons, hyphens, braces, dots,
parentheses, slashes, dashes, and more (but spaces, single
underscores, and the hat character \emph{are} allowed). To use special
characters in the name of a node, you can optionally surround the
\meta{node name} and/or the \meta{text} by quotation marks. In this
case, you can use all of the special symbols once more. The details
of what happens, exactly, when the \meta{node name} is surrounded by
quotation marks is explained later; surrounding the \meta{text} by
quotation marks has essentially the same effect as surrounding it by
curly braces.

Once the node name has been determined, it is checked whether the same
node name was already used inside the current graph. If this is the
case, then we say that the already existing node is \emph{referenced};
otherwise we say that the node is \emph{fresh}.

\begin{codeexample}[]
\tikz \graph {
  a -> b; % both are fresh
  c -> a; % only c is fresh, a is referenced
};
\end{codeexample}

This behaviour of deciding whether a node is fresh or referenced can,
however, be modified by using the following keys:
\begin{key}{/tikz/graphs/use existing node=\opt{\meta{true or
        false}} (default true)}
  When this key is set to |true|, all nodes will be considered to the
  referenced, no node will be fresh. This option is useful if you have
  already created all the nodes of a graph prior to using the |graph|
  command and you now only wish to connect the nodes.
\end{key}
\begin{key}{/tikz/graphs/fresh nodes=\opt{\meta{true or
        false}} (default true)}
  When this key is set to |true|, all nodes will be considered to be
  fresh. This option is useful when you create for instance a tree
  with many identical nodes.

  When a node name is encountered that was already used previously,
  a new name is chosen is follows: An apostrophe (|'|) is appended
  repeatedly until a node name is found that has not yet been
  used:
\begin{codeexample}[]
\tikz \graph [branch down=5mm] {
  { [fresh nodes]
    a -> {
      b -> {c, c},
      b -> {c, c},
      b -> {c, c},
    }
  },  
  b' -- b''
};
\end{codeexample}
\end{key}

\begin{key}{/tikz/graphs/number nodes=\opt{\meta{start number}} (default 1)}
  When this key is used in a scope, each encountered node name will
  get appended a new number, starting with \meta{start}. Typically,
  this ensures that all node names are different. Between the original
  node name and the appended number, the setting of the following will
  be inserted:
  \begin{key}{/tikz/graphs/number nodes sep=\meta{text}} (initially
    \normalfont space)}
  \end{key}
\begin{codeexample}[]
\tikz \graph [branch down=5mm] {
  { [number nodes]
    a -> {
      b -> {c, c},
      b -> {c, c},
      b -> {c, c},
    }
  },  
  b 2 -- b 5
};
\end{codeexample}
\end{key}

When a fresh node has been detected, a new node is created in the
inside a protecting scope. For this, the current
placement strategy is asked to compute a default position for the
node, see Section~\ref{section-library-graphs-placement} for
details. Then, the command
\begin{quote}
  |\node (|\meta{full node name}|) [|\meta{node options}|] {|\meta{text}|};|
\end{quote}
is called. The different parameters are as follows:
\begin{itemize}
\item
  The \meta{full node name} is normally the \meta{node name} that has
  been determined as described before. However, there are two exceptions:

  First, if the \meta{node name} is empty (which happens when there
  is no \meta{node name} before the slash), then a fresh internal node
  name is created and used as 
  \meta{full node name}. This name is guaranteed to be different from all
  node names used in this or any other graph. Thus, a direct node
  starting with a slash represents an anonymous fresh node. 

  Second, you can use the following key to prefix the \meta{node name}
  inside the \meta{full node name}:

  \begin{key}{/tikz/graphs/name=\meta{text}}
    This key prepends the \meta{text}, followed by a separating symbol
    (a space by default), to all
    \meta{node name}s inside a \meta{full node name}. Repeated calls
    of this key accumulate, leading to ever-longer ``name paths'':
\begin{codeexample}[]
\begin{tikzpicture}
  \graph {
    { [name=first]  1, 2, 3} --
    { [name=second] 1, 2, 3}
  };
  \draw [red] (second 1) circle [radius=3mm];
\end{tikzpicture}
\end{codeexample}
    Note that, indeed, in the above example six nodes are created even
    though the first and second set of nodes have the same \meta{node
      name}. The reason is that the full names of the six nodes are
    all different. Also note that only the \meta{node name} is used as
    the node text, not the full name. This can be changed as described
    later on.

    This key can be used repeatedly, leading to ever longer node names.
  \end{key}

  \begin{key}{/tikz/graphs/name separator=\meta{symbols} (initially \string\space)}
    Changes the symbol that is used to separate the \meta{text} from
    the \meta{node name}. The default is |\space|, resulting in a
    space.
\begin{codeexample}[]
\begin{tikzpicture}
  \graph [name separator=] { % no separator
    { [name=first]  1, 2, 3} --
    { [name=second] 1, 2, 3}
  };
  \draw [red] (second1) circle [radius=3mm];
\end{tikzpicture}
\end{codeexample}
\begin{codeexample}[]
\begin{tikzpicture}
  \graph [name separator=-] {
    { [name=first]  1, 2, 3} --
    { [name=second] 1, 2, 3}
  };
  \draw [red] (second-1) circle [radius=3mm];
\end{tikzpicture}
\end{codeexample}
  \end{key}
\item 
  The \meta{node options} are
  \begin{enumerate}
  \item The options that have accumulated in calls to |nodes| from
    the surrounding scopes.
  \item The local \meta{options}.
  \end{enumerate}
  The options are executed with the path prefix |/tikz/graphs|, but
  any unknown key is executed with the prefix |/tikz|. This means, in
  essence, that some esoteric keys are more difficult to use inside
  the options and that any key with the prefix |/tikz/graphs| will
  take precedence over a key with the prefix |/tikz|.
\item The \meta{text} that is passed to the |\node| command is
  computed as follows: First, you can use the following key to
  directly set the \meta{text}: 
  \begin{key}{/tikz/graphs/as=\meta{text}}
    The \meta{text} is used as the text of the node. This allows you
    to provide a text for the node that differs arbitrarily from the
    name of the node.
    \begin{codeexample}[]
\tikz \graph { a [as=$x$] -- b [as=$y_5$] -> c [red, as={a--b}] };
    \end{codeexample}
    This key always takes precedence over all of the mechanisms
    described below.
  \end{key}
  In case the |as| key is not used, a default text
  is chosen as follows: First, when a direct node specification
  contains a slash (or, for historical reasons, a double underscore),
  the text to the right of the slash (or double underscore) is stored
  in the macro |\tikzgraphnodetext|; if 
  there is no slash, the \meta{node name} is stored in
  |\tikzgraphnodetext|, instead. Then, the current value of the
  following key is used as \meta{text}:
  \begin{key}{/tikz/graphs/typeset=\meta{code}}
    The macro or code stored in this key is used as the
    \meta{text} if the node. Inside the \meta{code}, the following
    macros are available:
    \begin{command}{\tikzgraphnodetext}
      This macro expands to the \meta{text} to the right of the double
      underscore or slash in a direct node specification or, if there
      is no slash, to the \meta{node name}.
    \end{command}
    \begin{command}{\tikzgraphnodename}
      This macro expands to the name of the current node with the
      path. 
    \end{command}
    \begin{command}{\tikzgraphnodepath}
      This macro expands to the current path of the node. These
      paths result from the use of the |name| key as described above.
    \end{command}
    \begin{command}{\tikzgraphnodefullname}
      This macro contains the concatenation of the above two.
    \end{command}
  \end{key}
  By default, the typesetter is just set to |\tikzgraphnodetext|,
  which means that the default text of a node is its name. However,
  it may be useful to change this: For instance, you might wish that
  the text of all graph nodes is, say, surrounded by parentheses:
  \begin{codeexample}[]
\tikz \graph [typeset=(\tikzgraphnodetext)]
  { a -> b -> c };
  \end{codeexample}
  A more advanced macro might take apart the node text and render it
  differently: 
  \begin{codeexample}[]
\def\mytypesetter{\expandafter\myparser\tikzgraphnodetext\relax}
\def\myparser#1 #2 #3\relax{%
  $#1_{#2,\dots,#3}$
}
\tikz \graph [typeset=\mytypesetter, grow down]
  { a 1 n -> b 2 m -> c 4 nm };
  \end{codeexample}
  The following styles install useful predefined typesetting macros:
  \begin{key}{/tikz/graphs/empty nodes}
    Just sets |typeset| to nothing, which causes all nodes to have an
    empty text (unless, of course, the |as| option is used):
    \begin{codeexample}[]
\tikz \graph [empty nodes, nodes={circle, draw}] { a -> {b, c} };  
    \end{codeexample}
  \end{key}
  \begin{key}{/tikz/graphs/math nodes}
    Sets |typeset| to |$\tikzgraphnodetext$|, which causes all nodes
    names to be typeset in math mode:
    \begin{codeexample}[]
\tikz \graph [math nodes, nodes={circle, draw}] { a_1 -> {b^2, c_3^n} };  
    \end{codeexample}
  \end{key}
\end{itemize}

If a node is referenced instead of fresh, then this node becomes the
node that will be connected by the preceding or following edge
specification to other 
nodes. The \meta{options} are executed even for a referenced node, but
they cannot be used to change the appearance of the node (because the
node exists already). Rather, the \meta{options} can only be used to
change the logical coloring of the node, see
Section~\ref{section-library-graphs-color-classes} for details.

\medskip
\textbf{Quoted Node Names.} When the \meta{node name} and/or the
\meta{text} of a node is surrounded by quotation marks, you can use
all sorts of special symbols as part of the text that are
normally forbidden:
\begin{codeexample}[]
\begin{tikzpicture}
  \graph [grow right=2cm] {
    "Hi, World!"       -> "It's \emph{important}!"[red,rotate=-45];
    "name"/actual text -> "It's \emph{important}!";
  };
  \draw (name) circle [radius=3pt];
\end{tikzpicture}
\end{codeexample}

In detail, for the following happens when qutation marks are
encountered at the beginning of a node name or its text:
\begin{itemize}
\item Everything following the quotation mark up to the next single
  quotation mark is collected into a macro \meta{collected}. All sorts
  of special characters, including commas, square brackets, dashes,
  and even backslashes are allowed here. Basically, the only
  restriction is that braces must be balanced.
\item A double quotation mark (|""|) does not count as the ``next
  single quotation mark.'' Rather, it is replaced by a single
  quotation mark. For instance, |"He said, ""Hello world."""| would be
  stored inside \meta{collected} as |He said, "Hello world."|
  However, this rule applies only on the outer-most level of
  braces. Thus, in
\begin{codeexample}[code only]
"He {said, ""Hello world.""}"
\end{codeexample}
  we would get |He {said, ""Hello world.""}| as \meta{collected}.
\item ``The next single quotation mark'' refers to the next
  quotation mark on the current level of braces, so in
  |"hello {"} world"|, the next quotation mark would be the one
  following |world|.
\end{itemize}

Now, once the \meta{collected} text has been gather, it is used as
follows: When used as \meta{text} (what is actually displayed), it is
just used ``as is''. When it is used as \meta{node name}, however, the
following happens: Every ``special character'' in \meta{collected} is
replaced by its Unicode name, surrounded by |@|-signs. For instance,
if \meta{collected} is |Hello, world!|, the \meta{node name} is the
somewhat longer text |Hello@COMMA@ world@EXCLAMATION MARK@|. Admittedly,
referencing such a node from outside the graph is 
cumbersome, but when you use exactly the same \meta{collected} text
once more, the same \meta{node name} will result. The
following characters are considered ``special'':
\begin{quote}
  \texttt{\char`\|}|$&^~_[](){}/.-,+*'`!":;<=>?@#%\{}|%$
\end{quote}
These are exactly the Unicode character with a decimal code number
between 33 and 126 that are neither digits nor letters. 


\medskip
\textbf{Reference Node Specifications.} A reference node specification
is a node specification that starts with an opening parenthesis. In
this case, parentheses must surround a \meta{name} as in |(foo)|,
where |foo| is the \meta{name}. The following will now happen:

\begin{enumerate}
\item It is tested whether \meta{name} is the name of a currently
  active \emph{node set}. This case will be discussed in a moment.
\item Otherwise, the \meta{name} is interpreted and treated as a
  referenced node, but independently of whether the node has already
  been fresh in the current graph or not. In other words, the node
  must have been defined either already inside the graph (in which
  case the parenthesis are more or less superfluous) or it must have
  been defined outside the current picture.

  The way the referenced node is handled is the same way as for a
  direct node that is a referenced node.

  If the node does not already exist, an error message is printed.
\end{enumerate}

Let us now have a look at node sets. Inside a |{tikzpicture}| you can
locally define a \emph{node set} by using the following key:
\begin{key}{/tikz/new set=\meta{set name}}
  This will setup a node set named \meta{set name} within the current
  scope. Inside the scope, you can add nodes to the node set using the
  |set| key. If a node set of the same name already exists in the
  current scope, it will be reset and made empty for the current
  scope.

  Note that this command has the path |/tikz| and is normally used
  \emph{outside} the |graph| command.
\end{key}
\begin{key}{/tikz/set=\meta{set name}}
  This key can be used as an option with a |node| command. The
  \meta{set name} must be the name of a node set that has previously
  been created inside some enclosing scope via the |new set| key. The
  effect is that the current node is added to the node set.
\end{key}

When you use a |graph| command inside a scope where some node set
called \meta{set name} is defined, then inside this |graph| command
you use |(|\meta{set name}|)| to reference \emph{all} of the nodes in
the node set. The effect is the same as if instead of the reference to
the set name you had created a group specification containing a list
of references to all the nodes that are part of the node set.

\begin{codeexample}[]
\begin{tikzpicture}[new set=red, new set=green, shorten >=2pt]
  \foreach \i in {1,2,3} {
    \node [draw, red!80,         set=red]   (r\i) at (\i,1) {$r_\i$};
    \node [draw, green!50!black, set=green] (g\i) at (\i,2) {$g_\i$};
  }
  \graph {
    root [xshift=2cm] ->
    (red)             -> [complete bipartite, right anchor=south]
    (green)
  };
\end{tikzpicture}
\end{codeexample}

There is an interesting caveat with referencing node sets: Suppose
that at the beginning of a graph you just say |(foo);| where |foo| is
a set name. Unless you have specified special options, this will cause
the following to happen: A group is created whose members are all the
nodes of the node set |foo|. These nodes become referenced nodes, but
otherwise nothing happens since, by default, the nodes of a group are
not connected automatically. However, the referenced nodes have now
been referenced inside the graph, you can thus subsequently access
them as if they had been defined inside the graph. Here is an example
showing how you can create nodes outside a |graph| command and then
connect them inside as if they had been declared inside:

\begin{codeexample}[]
\begin{tikzpicture}[new set=import nodes]
  \begin{scope}[nodes={set=import nodes}] % make all nodes part of this set
    \node [red] (a) at (0,1) {$a$};
    \node [red] (b) at (1,1) {$b$};
    \node [red] (d) at (2,1) {$d$};
  \end{scope}

  \graph {
    (import nodes);         % "import" the nodes

    a -> b -> c -> d -> e;  % only c and e are new
  };
\end{tikzpicture}
\end{codeexample}


\medskip
\textbf{Group Node Specifications.}
At a place where a node specification should go, you can also instead
provide a group specification. Since nodes specifications are part of
chain specifications, which in turn are part of group specifications,
this is a recursive definition.

\begin{codeexample}[]
\tikz \graph { a -> {b,c,d} -> {e -> {f,g}, h} };
\end{codeexample}

As can be seen in the above example, when two groups of nodes are
connected via an edge specification, it is not immediately obvious
which connecting edges are added. This is detailed in
Section~\ref{section-library-graphs-color-classes}. 



\subsubsection{Specifying Tries}

In computer science, a \emph{trie} is a special kind of tree, where
for each node and each symbol of an alphabet, there is at most one
child of the node labeled with this symbol.

The |trie| key is useful for drawing tries, but it can also be used in
other situations. What it does, essentially, is to prepend the node
names of all nodes \emph{before} the current node of the current chain
to the node's name. This will often make it easier or more natural to
specify graphs in which several nodes have the same label. 

\begin{key}{/tikz/graphs/trie=\opt{\meta{true or false}} (default true, initially false)}
  If this key is set to |true|, after a node has been created on a
  chain, the |name| key is executed with the node's \meta{node
    name}. Thus, all nodes later on this chain have the ``path'' of
  nodes leading to this node as their name. This means, in particular,
  that
  \begin{enumerate}
  \item two nodes of the same name but in different parts of a chain
    will be different,
  \item while if another chain starts with the same nodes, no new
    nodes get created.
  \end{enumerate}
  In total, this is exactly the behaviour you would expect of a trie:
\begin{codeexample}[]
\tikz \graph [trie] {
  a -> {
    a,
    c -> {a, b},
    b
  }
};
\end{codeexample}
  You can even ``reiterate'' over a path in conjunction with the
  |simple| option. However, in this case, the default placement
  strategies will not work and you will need options like
  |layered layout| from the graph drawing libraries, which need
  Lua\TeX. 
\ifluatex  
\begin{codeexample}[]
\tikz \graph [trie, simple, layered layout] {
  a -> b -> a,
  a -> b -> c,
  a -> {d,a}
};
\end{codeexample}
  In the following example, we setup the |typeset| key so that it
  shows the complete names of the nodes:
\begin{codeexample}[]
\tikz \graph [trie, simple, layered layout,
              typeset=\tikzgraphnodefullname] {
  a -> b -> a,
  a -> b -> c,
  a -> {d,a}
};
\end{codeexample}
\fi
  You can also use the |trie| key locally and later reference nodes
  using their full name:
\begin{codeexample}[]
\tikz \graph {
  { [trie, simple]
    a -> {
      b,
      c -> a
    }
  },
  a b ->[red] a c a
};
\end{codeexample}
\end{key}


\subsection{Quick Graphs}

\label{section-library-graphs-quick}

The graph syntax is powerful, but this power comes at a price: parsing
the graph syntax, which is done by \TeX, can take some time. Normally,
the parsing is fast enough that you will not notice it, but it can be
bothersome when you have graphs with hundreds of nodes as happens
frequently when nodes are generated algorithmically by some other
program. Fortunately, when another program generated a graph
specification, we typically do not need the full power of the graph
syntax. Rather, a small subset of the graph syntax would suffice that
allows to specify nodes and edges. For these reasons, the is a special
``quick'' version of the graph syntax.

Note, however, that using this syntax will usually at most halve the
time needed to parse a graph. Thus, it really mostly makes sense in
conjunction with large, algorithmically generated graphs.

\begin{key}{/tikz/graphs/quick}
  When you provide this key with a graph, the syntax of graph
  specifications gets restricted. You are no longer allowed to use
  certain features of the graph syntax; but all features that are
  still allowed are also allowed in the same way when you do not
  provide the |quick| option. Thus, leaving out the |quick| option
  will never hurt.

  Since the syntax is so severely restricted, it is easier to explain
  which aspects of the graph syntax \emph{will} still work:
  
  \begin{enumerate}
  \item 
    A quick graph consists of a sequence of either nodes, edges sequences, or
    groups. These are separated by commas or semicolons.
  \item  
    Every node is of the form
    
    \begin{quote}
      |"|\meta{node name}|"|\opt{|/"|\meta{node text}|"[|\meta{options}|]|}
    \end{quote}

    The quotation marks are mandatory. The part |/"|\meta{node
      text}|"| may  be missing, in which case the node name is used as
    the node text. The \meta{options} may also be missing. The
    \meta{node name} may not contain any ``funny'' characters (unlike
    in the normal graph command).
  \item
    Every chain is of the form
    
    \begin{quote}
      \meta{node spec} \meta{connector} \meta{node spec}
      \meta{connector} \dots \meta{connector} \meta{node spec}|;|
    \end{quote}

    Here, the \meta{node spec} are node specifications as described
    above, the \meta{connector} is one of the four connectors |->|,
    |<-|, |--|, and |<->| (the connector |-!-| is not allowed since
    the |simple| option is also not allowed). Each connector may be
    followed by options in square brackets. The semicolon may be
    replaced by a comma. 
  \item
    Every group is of the form
    
    \begin{quote}
      |{ [|\meta{options}|]| \meta{chains and groups} |};|
    \end{quote}
    The \meta{options} are compulsory. The semicolon can, again, be 
    replaced by a comma.
  \item
    The |number nodes| option will work as expected.
  \end{enumerate}

  Here is a typical way this syntax might be used:
\begin{codeexample}[]
\tikz \graph [quick] { "a" --["foo"] "b"[x=1] };
\end{codeexample}

\begin{codeexample}[]
\tikz \graph [quick] {
  "a"/"$a$" -- "b"[x=1] --[red] "c"[x=2];
  { [nodes=blue] "a" -- "d"[y=1]; };
};
\end{codeexample}

  Let us now have a look at the most important things that will
  \emph{not} work when the |quick| option is used:

  \begin{itemize}
  \item Connecting a node and a group as in |a->{b,c}|.
  \item Node names without quotation marks as in |a--b|.
  \item Everything described in subsequent subsections, which includes
    subgraphs (graph macros), graph sets, graph color classes,
    anonymous nodes, the |fresh nodes| option, sublayouts, simple
    graphs, edge annotations.
  \item Placement strategies -- you either have to define all node
    positions explicitly using |at=| or |x=| and |y=| or you must use
    a graph drawing algorithm like |layered layout|.
  \end{itemize}  
\end{key}


\subsection{Simple Versus Multi-Graphs}

\label{section-library-graphs-simple}

The |graph| library allows you to construct both simple graphs and
multi-graphs. In a simple graph there can be at most one edge between
any two vertices, while in a multi-graph there can be multiple edges
(hence the name). The two keys |multi| and |simple| allow you to
switch (even locally inside on of the graph's scopes) between which
kind of graph is being constructed. By default, the |graph| command
produces a multi-graph since these are faster to construct.

\begin{key}{/tikz/graphs/multi}
  When this edge is set for a whole graph (which is the default) or
  just for a group (which is useful if the whole graph is simple in
  general, but a part is a multi-graph), then when you specify an edge
  between two nodes several times, several such edges get created:

\begin{codeexample}[]
\tikz \graph [multi] { % "multi" is not really necessary here
  a ->[bend left,  red]  b;
  a ->[bend right, blue] b;
};
\end{codeexample}
  In case |multi| is used for a scope inside a larger scope where the
  |simple| option is specified, then inside the local |multi| scope
  edges are immediately created and they are completely ignored when
  it comes to deciding which kind of edges should be present in the
  surrounding simple graph. From the surrounding scope's point of view
  it is as if the local |multi| graph contained no edges at all.

  This means, in particular, that you can use the |multi| option with
  a single edge to ``enforce'' this edge to be present in a simple
  graph. 
\end{key}

\begin{key}{/tikz/graphs/simple}
  In contrast a multi-graph, in a simple graph, at most one edge gets
  created for every pair of vertices:
\begin{codeexample}[]
\tikz \graph [simple]{
  a ->[bend left,  red]  b;
  a ->[bend right, blue] b;
};
\end{codeexample}
  As can be seen, the second edge ``wins'' over the first edge. The
  general rule is as follows: In a simple graph, whenever an edge
  between two vertices is specified multiple times, only the very last
  specification and its options will actually be executed.
  
  The real power of the |simple| option lies in the fact that you can
  first create a complicated graph and then later redirect and otherwise
  modify edges easily: 

\begin{codeexample}[]
\tikz \graph [simple, grow right=2cm] {
  {a,b,c,d} ->[complete bipartite] {e,f,g,h};

  { [edges={red,thick}] a -> e -> d -> g -> a };
};
\end{codeexample}

  One particularly interesting kind of edge specification for a simple
  graph is |-!-|. Recall that this is used to indicate that ``no
  edge'' should be added between certain nodes. In a multi-graph, this
  key usually has no effect (unless the key |new -!-| has been
  redefined) and is pretty superfluous. In a simple graph, however, it
  counts as an edge kind and you can thus use it to remove an edge
  that been added previously:

\begin{codeexample}[]
\tikz \graph [simple] {
  subgraph K_n [n=8, clockwise];
  % Get rid of the following edges:
  1 -!- 2;
  3 -!- 4;
  6 -!- 8;
  % And make one edge red:
  1 --[red] 3;
};
\end{codeexample}

  Creating a graph such as the above in other fashions is pretty
  awkward.

  For every unordered pair $\{u,v\}$ of vertices at most one edge will
  be created in a simple graph. In particular, when you say |a -> b|
  and later also |a <- b|, then only the edge |a <- b| will be
  created. Similarly, when you say |a -> b| and later |b -> a|, then
  only the edge |b -> a| will be created. 

  The power of the |simple| command comes at a certain cost: As the
  graph is being constructed, a (sparse) array is created that keeps
  track for each edge of the last edge being specified. Then, at the
  end of the scope containing the |simple| command, for every pair of
  vertices the edge is created. This is implemented by two nested
  loops iterating over all possible pairs of vertices -- which may
  take quite a while in a graph of, say, 1000 vertices.
  Internally, the |simple| command is implemented as an operator that
  adds the edges when it is called, but
  this should be unimportant in normal situations.
\end{key}




\subsection{Graph Edges: Labeling and Styling}

When the |graph| library creates an edge between two nodes in a graph,
the appearance (called ``styling'' in \tikzname) can be specified in
different ways. Sometimes you will simply wish to say ``the edges
between these two groups of node should be red,'' but sometimes you
may wish to say ``this particular edge going into this node should be
red.'' In the following, different ways of specifying such styling
requirements are discussed. Note that adding labels to edges is, from
\tikzname's point of view, almost the same as styling edges, since
they are also specified using options.


\subsubsection{Options For All Edges Between Two Groups}

When you write |... ->[options] ...| somewhere inside your graph
specification, this typically cause one or more edges to be created
between the nodes in the chain group before the |->| and the nodes in
the chain group following it. The |options| are applied to all of
them. In particular, if you use the |quotes| library and you write
some text in quotes inside the |options|, this text will be added as a
label to each edge:

\begin{codeexample}[]
\tikz 
  \graph [edge quotes=near start] {
    { a, b } -> [red, "x", complete bipartite] { c, d };
  };
\end{codeexample}

As documented in the |quotes| library in more detail, you can easily
modify the appearance of edge labels created using the quotes syntax
by adding options after the closing quotes:

\begin{codeexample}[]
\tikz \graph {
  a ->["x"] b ->["y"'] c ->["z" red] d;
};
\end{codeexample}

The following options make it easy to setup the styling of nodes
created in this way:

\begin{key}{/tikz/graphs/edge quotes=\opt{\meta{options}}}
  A shorthand for setting the style |every edge quotes| to \meta{options}.  
\begin{codeexample}[]
  \tikz \graph [edge quotes={blue,auto}] {
  a ->["x"] b ->["y"'] c ->["b" red] d;
};
\end{codeexample}
\end{key}

\begin{key}{/tikz/graphs/edge quotes center}
  A shorthand for |edge quotes| to |anchor=center|.
\begin{codeexample}[]
\tikz \graph [edge quotes center] {
  a ->["x"] b ->["y"] c ->["z" red] d;
};
\end{codeexample}
\end{key}

\begin{key}{/tikz/graphs/edge quotes mid}
  A shorthand for |edge quotes| to |anchor=mid|.
\begin{codeexample}[]
\tikz \graph [edge quotes mid] {
  a ->["x"] b ->["y"] c ->["z" red] d;
};
\end{codeexample}
\end{key}



\subsubsection{Changing Options For Certain Edges}


Consider the following tree-like graph:

\begin{codeexample}[]
\tikz \graph { a -> {b,c} };
\end{codeexample}

Suppose we wish to specify that the edge from |a| to |b| should be
red, while the edge from |a| to |c| should be blue. The difficulty
lies in the fact that \emph{both} edges are created by the single |->|
operator and we can only add one of these option |red| or |blue| to
the operator.

There are several ways to solve this problem. First, we can simply
split up the specification and specify the two edges separately:

\begin{codeexample}[]
\tikz \graph {
  a -> [red]  b;
  a -> [blue] c;
};  
\end{codeexample}
While this works quite well, we can no longer use the nice chain group
syntax of the |graphs| library. For the rather simple graph |a->{b,c}|
this is not a big problem, but if you specify a tree with, say, 30
nodes it is really worthwhile being able to specify the tree ``in its
natural form in the \TeX\ code'' rather than having to list all of the
edges explicitly. Also, as can be seen in the above example, the
node placement is changed, which is not always desirable. 

One can sidestep this problem using the |simple| option: This option
allows you to first specify a graph and then, later on, replace edges
by other edges and, thereby, provide new options:

\begin{codeexample}[]
\tikz \graph [simple] {
  a -> {b,c};
  a -> [red]  b;
  a -> [blue] c;
};  
\end{codeexample}

The first line is the original specification of the tree, while the
following two lines replace some edges of the tree (in this case, all
of them) by edges with special options. While this method is slower
and in the above example creates even longer code, it is very useful
if you wish to, say, highlight a path in a larger tree: First specify
the tree normally and, then, ``respecify'' the path or paths with some
other edge options in force. In the following example, we use this to
highlight a whole subtree of a larger tree:

\begin{codeexample}[]
\tikz \graph [simple] {
  % The larger tree, no special options in force
  a -> {
    b -> {c,d},
    e -> {f,g},
    h
  },  
  { [edges=red] % Now highlight a part of the tree
    a -> e -> {f,g}
  }
};  
\end{codeexample}



\subsubsection{Options For Incoming and Outgoing Edges}

When you use the syntax |... ->[options] ...| to specify options, you
specify options for the ``connections between two sets of nodes''. In
many cases, however, it will be more natural to specify options ``for
the edges lead to or coming from a certain node'' and you will want to
specify these options ``at the node''. Returning to the example of the
graph |a->{b,c}| where we want a red edge between |a| and |b| and a
blue edge between |a| and |c|, this could also be phrased as follows:
``Make the edge leading to |b| red and make the edge leading to |c|
blue.''

For this situation, the |graph| library offers a number of special
keys, which are documented in the following. However, most of the time
you will not use these keys directly, but, rather, use a special
syntax explained in Section~\ref{section-syntax-outgoing-incoming}.

\begin{key}{/tikz/graphs/target edge style=\meta{options}}
  This key can (only) be used with a \emph{node} inside a graph 
  specification. When used, the \meta{options} will be added to every
  edge that is created by a connector like |->| in which the node is a
  \emph{target}. Consider the following example:
\begin{codeexample}[]
\tikz \graph {
  { a, b } ->
  { c [target edge style=red], d } ->
  { e, f }
};
\end{codeexample}
  In the example, only when the edge from |a| to |c| is created, |c|
  is the ``target'' of the edge. Thus, only this edge becomes red.
  
  When an edge already has options set directly, the \meta{options}
  are executed after these direct options, thus, they ``overrule''
  them:
\begin{codeexample}[]
\tikz \graph {
  { a, b } -> [blue, thick]
  { c [target edge style=red], d } ->
  { e, f }
};
\end{codeexample}
  
  The \meta{options} set in this way will stay attached to the node,
  so also for edges created later on that lead to the node will have
  these options set:
\begin{codeexample}[]
\tikz \graph {
  { a, b } ->
  { c [target edge style=red], d } ->
  { e, f },
  b -> c
};
\end{codeexample}

  Multiple uses of this key accumulate. However, you may sometimes
  also wish to ``clear'' these options for a key since at some later
  point you no longer wish the \meta{options} to be added when some
  further edges are added. This can be achieved using the following
  key:
  \begin{key}{/tikz/graphs/target edge clear}
    Clears all \meta{options} for edges with the node as a target and
    also edge labels (see below) for this node.    
  \end{key}
\begin{codeexample}[]
\tikz \graph {
  { a, b } ->
  { c [target edge style=red], d },
  b -> c[target edge clear]
};
\end{codeexample}
\end{key}

\begin{key}{/tikz/graphs/target edge node=\meta{node specification}}
  This key works like |target edge style|, only the \meta{node
    specification} will not be added as options to any newly created
  edges with the current node as their target, but rather it will be
  added as a node specification.  
\begin{codeexample}[]
\tikz \graph {
  { a, b } ->
  { c [target edge node=node{X}], d } ->
  { e, f }
};
\end{codeexample}
  As for |target edge style| multiple uses of this key accumulate and
  the key |target edge clear| will (also) clear all target edge nodes
  that have been set for a node earlier on.
\end{key}

\begin{key}{/tikz/graphs/source edge style=\meta{options}}
  Works exactly like |target edge style|, only now the \meta{options}
  are only added when the node is a source of a newly created edge:
\begin{codeexample}[]
\tikz \graph {
  { a, b } ->
  { c [source edge style=red], d } ->
  { e, f }
};
\end{codeexample}
  If both for the source and also for the target of an edge
  \meta{options} have been specified, the options are applied in the
  following order:
  \begin{enumerate}
  \item First come the options from the edge itself.
  \item Then come the options contributed by the source node using
    this key.
  \item Then come the options contributed by the target node using
    |target node style|.    
  \end{enumerate}
\begin{codeexample}[]
\tikz \graph {
  a [source edge style=red] ->[green]
  b [target edge style=blue]  % blue wins
};
\end{codeexample}
\end{key}

\begin{key}{/tikz/graphs/source edge node=\meta{node specification}}
  Works like |source edge style| and |target edge node|.  
\end{key}

\begin{key}{/tikz/graphs/source edge clear=\meta{node specification}}
  Works like |target edge clear|.  
\end{key}



\subsubsection{Special Syntax for Options For Incoming and Outgoing Edges}
\label{section-syntax-outgoing-incoming}

The keys |target node style| and its friends are powerful, but a bit
cumbersome to write down. For this reason, the |graphs| library
introduces a special syntax that is based on what I call the
``first-char syntax'' of keys. Inside the options of a node inside a
graph, the following special rules apply:
\begin{enumerate}
\item Whenever an option starts with |>|, the rest of the options are
  passed to |target edge style|. For instance, when you write
  |a[>red]|, then this has the same effect as if you had written
\begin{codeexample}[code only]
a[target edge style={red}]    
\end{codeexample}
\item Whenever an options starts with |<|, the rest of the options are
  passed to |source edge style|.
\item In both of the above case, in case the options following the |>|
  or |<| sign start with a quote, the created edge label is passed to
  |source edge node| or |target edge node|, respectively.

  This is exactly what you want to happen.
\end{enumerate}
Additionally, the following styles provide shorthands for ``clearing''
the target and source options:
\begin{key}{/tikz/graphs/clear >}
  A more easy-to-remember shorthand for |target edge clear|.  
\end{key}
\begin{key}{/tikz/graphs/clear <}
  A more easy-to-remember shorthand for |source edge clear|.  
\end{key}

These mechanisms make it especially easy to create trees in which the
edges are labeled in some special way:
\begin{codeexample}[]
\tikz 
  \graph [edge quotes={fill=white,inner sep=1pt},
          grow down, branch right] {
    / -> h [>"9"] -> {
      c [>"4" text=red,] -> {
        a [>"2", >thick],
        e [>"0"]
      },
      j [>"7"]
    }
  };
\end{codeexample}


\subsubsection{Placing Node Texts on Incoming Edges}

Normally, the text of a node is shown (only) inside the node. In some
case, for instance when drawing certain kind of trees, the nodes
themselves should not get any text, but rather the edge leading to the
node should be labeled as in the following example:
\begin{codeexample}[]
\tikz \graph [empty nodes]
{
  root -> {
    a [>"a"],
    b [>"b"] -> {
      c [>"c"],
      d [>"d"] 
    }
  }
};
\end{codeexample}
As the example shows, it is a bit cumbersome that we have to label the
nodes and then specify the same text once more using the incoming edge
syntax.

For these cases, it would be better if the text of the node where not
used with the node but, rather, be passed directly to the incoming or
the outgoing edge. The following styles do exactly this:

\begin{key}{/tikz/graphs/put node text on incoming edges=\opt{\meta{options}}}
  When this key is used with a node or a group, the following happens:
  \begin{enumerate}
  \item The command |target edge node={node[|\meta{options}|]{\tikzgraphnodetext}}|
    is executed. This means that all incoming edges of the node get a
    label with the text that would usually be displayed in the
    node. You can use keys like |math nodes| normally.
  \item The command |as={}| is executed. This means that the node
    itself will display nothing.
  \end{enumerate}
  Here is an example that show how this command is used.
\begin{codeexample}[]
\tikz \graph [put node text on incoming edges,
              math nodes, nodes={circle,draw}]
  { a -> b -> {c, d} };
\end{codeexample}
\end{key}

\begin{key}{/tikz/graphs/put node text on outgoing
    edges=\opt{\meta{options}}}
  Works like the previous key, only with |target| replaced by |source|.  
\end{key}



\subsection{Graph Operators, Color Classes, and Graph Expressions}
\label{section-library-graphs-color-classes}

\tikzname's |graph| command employs a powerful mechanism for
adding edges between nodes and sets of nodes. To a graph theorist,
this mechanism may be known as a \emph{graph expression}: A graph is
specified by starting with small graphs and then applying
\emph{operators} to them that form larger graphs and that connect and
recolor colored subsets of the graph's node in different ways. 


\subsubsection{Color Classes}

\label{section-library-graph-coloring}

\tikzname\ keeps track of a \emph{(multi)coloring} of the graph as it
is being constructed. This does not mean that the 
actual color of the nodes on the page will be different, rather, in
the following we refer to ``logical'' colors in the way graph
theoreticians do. These ``logical'' colors are only important while
the graph is being constructed and they are ``thrown away'' at the end
of the construction. The actual (``physical'') colors of the nodes are
set independently of these logical colors.

As a graph is being constructed, each node can be part of one or more
overlapping \emph{color classes}. So, unlike what is sometimes called a
\emph{legal coloring}, the logical colorings that \tikzname\ keeps
track of may assign multiple colors to the same node and two nodes
connected by an edge may well have the same color.

Color classes must be declared prior to use. This is done using the
following key:
\begin{key}{/tikz/graphs/color class=\meta{color class name}}
  This sets up a new color class called \meta{color class name}. Nodes
  and whole groups of nodes can now be colored with \meta{color class
    name}. This is done using the following keys, which become
  available inside the current scope: 
  \begin{key}{/tikz/graphs/\meta{color class name}}
    This key internally uses the |operator| command to setup an
    operator that will cause all nodes of the current group to get the
    ``logical color'' \meta{color class name}. Nodes retain this color
    in all encompassing scopes, unless it is explicitly changed (see
    below) or unset (again, see below).
\begin{codeexample}[]
\tikz \graph [color class=red] {
  [cycle=red]  % causes all "logically" red nodes to be connected in
               % a cycle
  a,
  b [red],
  { [red] c ->[bend right] d },
  e
};
\end{codeexample}
\begin{codeexample}[]
\tikz \graph [color class=red, color class=green,
              math nodes, clockwise, n=5] {
  [complete bipartite={red}{green}]
  { [red]   r_1, r_2 },
  { [green] g_1, g_2, g_3 }
};
\end{codeexample}
  \end{key}
  \begin{key}{/tikz/graphs/not \meta{color class name}}
    Sets up an operator for the current scope so that all nodes in it
    loose the color \meta{color class name}. You can also use
    |!|\meta{color class name} as an alias for this key.
\begin{codeexample}[]
\tikz \graph [color class=red, color class=green,
              math nodes, clockwise, n=5] {
  [complete bipartite={red}{green}]
  { [red]   r_1, r_2 },
  { [green] g_1, g_2, g_3 },
  g_2 [recolor green by=red]
};
\end{codeexample}
  \end{key}
  \begin{key}{/tikz/graphs/recolor \meta{color class name} by=\meta{new color}}
    Causes all keys having color \meta{color class name} to get
    \meta{new color} instead. They loose having color \meta{color
      class name}, but other colors are not affected.
\begin{codeexample}[]
\tikz \graph [color class=red, color class=green,
              math nodes, clockwise, n=5] {
  [complete bipartite={red}{green}]
  { [red]   r_1, r_2 },
  { [green] g_1, g_2, g_3 },
  g_2 [not green]
};
\end{codeexample}
  \end{key}
\end{key}

The following color classes are available by default:
\begin{itemize}
\item 
  Color class |all|. Every node is part of this class by default. This
  is useful to access all nodes of a (sub)graph, since you can simply
  access all nodes of this color class. 
\item
  Color classes |source| and |target|. These classes are used to
  identify nodes that lead ``into'' a group of nodes and nodes from
  which paths should ``leave'' the group. Details on how these colors
  are assigned are explained in
  Section~\ref{section-library-graphs-join}. By saying |not source| or
  |not target| with a node, you can influence how it is connected:
  \begin{codeexample}[]
\tikz \graph { a -> { b, c, d } -> e };    
  \end{codeexample}
  \begin{codeexample}[]
\tikz \graph { a -> { b[not source], c, d[not target] } -> e };    
  \end{codeexample}
\item
  Color classes |source'| and |target'|. These are temporary colors
  that are also explained in
  Section~\ref{section-library-graphs-join}.
\end{itemize}



\subsubsection{Graph Operators on Groups of Nodes}

Recall that the |graph| command constructs graphs recursively from
nested \meta{group specifications}. Each such \meta{group
  specification} describes a subset of the nodes of the final graph. A
\emph{graph operator} is an algorithm that gets the nodes of a group
as input and (typically) adds edges between these nodes in some
sensible way. For instance, the |clique| operator will simply add
edges between all nodes of the group.

\begin{key}{/tikz/graphs/operator=\meta{code}}
  This key has an effect in three places:
  \begin{enumerate}
  \item It can be used in the \meta{options} of a \meta{direct node specification}.
  \item It can be used in the \meta{options} of a \meta{group
      specification}.
  \item It can be used in the \meta{options} of an \meta{edge specification}.
  \end{enumerate}
  The first case is a special case of the second, since it is treated
  like a group specification containing a single node. The last case
  is more complicated and discussed in the next section.
  So, let us focus on the second case.

  Even though the \meta{options} of a group are given at the beginning
  of the \meta{group specification}, the \meta{code} is only executed
  when the group has been parsed completely and all its nodes have
  been identified. If you use the |operator| multiple times in the
  \meta{options}, the effect accumulates, that is, all code passed to
  the different calls of |operator| gets executed in the order it is
  encountered. 

  The \meta{code} can do ``whatever it wants,'' but it will typically
  add edges between certain nodes. You can configure what kind of
  edges (directed, undirected, etc.) are created by using the
  following keys:
  \begin{key}{/tikz/graphs/default edge kind=\meta{value} (initially --)}
    This key stores one of the five edge kinds |--|, |<-|, |->|,
    |<->|, and |-!-|. When an operator wishes to create a new edge, it
    should typically set
\begin{codeexample}[code only]
\tikzgraphsset{new \pfkeysvalueof{/tikz/graphs/default edge kind}=...}      
\end{codeexample}
    While this key can be set explicitly, it may be more convenient to
    use the abbreviating keys listed below. Also, this key is
    automatically set to the current value of \meta{edge
      specification} when a joining operator is called, see the
    discussion of joining operators in
    Section~\ref{section-library-graphs-join}.
  \end{key}
  \begin{key}{/tikz/graphs/--}
    Sets the |default edge kind| to |--|.
\begin{codeexample}[]
\tikz \graph { subgraph K_n [--, n=5, clockwise, radius=6mm] };      
\end{codeexample}
  \end{key}
  \begin{key}{/tikz/graphs/->}
    Sets the |default edge kind| to |->|.
\begin{codeexample}[]
\tikz \graph { subgraph K_n [->, n=5, clockwise, radius=6mm] };      
\end{codeexample}
  \end{key}
  \begin{key}{/tikz/graphs/<-}
    Sets the |default edge kind| to |<-|.
\begin{codeexample}[]
\tikz \graph { subgraph K_n [<-, n=5, clockwise, radius=6mm] };      
\end{codeexample}
  \end{key}
  \begin{key}{/tikz/graphs/<->}
    Sets the |default edge kind| to |<->|.
\begin{codeexample}[]
\tikz \graph { subgraph K_n [<->, n=5, clockwise, radius=6mm] };      
\end{codeexample}
  \end{key}
  \begin{key}{/tikz/graphs/-\protect\exclamationmarktext-}
    Sets the |default edge kind| to |-!-|.
  \end{key}
  
  When the \meta{code} of an operator is executed, the following
  commands can be used to find the nodes that should be connected:
  \begin{command}{\tikzgraphforeachcolorednode\marg{color name}\marg{macro}}
    When this command is called inside \meta{code}, the following will
    happen: \tikzname\ will iterate over all nodes inside the
    just-specified group that have the color \meta{color name}. The
    order in which they are iterated over is the order in which they
    appear inside the group specification (if a node is encountered
    several times inside the specification, only the first occurrence
    counts). Then, for each node the \meta{macro} is executed with the
    node's name as the only argument.

    In the following example we use an operator to connect every
    node colored |all| inside the subgroup to he node |root|. 
    \begin{codeexample}[]
\def\myconnect#1{\tikzset{graphs/new ->={root}{#1}{}{}}}      

\begin{tikzpicture}
  \node (root) at (-1,-1) {root};
  
  \graph {
    x,
    {
      [operator=\tikzgraphforeachcolorednode{all}{\myconnect}]
      a, b, c
    }
  };
\end{tikzpicture}
    \end{codeexample}
  \end{command}
  
  \begin{command}{\tikzgraphpreparecolor\marg{color
        name}\marg{counter}\marg{prefix}}
    This command is used to ``prepare'' the nodes of a certain color
    for random access. The effect is the following: It is counted how
    many nodes there are having color \meta{color name} in the current
    group and the result is stored in \meta{counter}. Next, macros
    named \meta{prefix}|1|, \meta{prefix}|2|, and so on are defined,
    that store the names of the first, second, third, and so on node
    having the color \meta{color name}.

    The net effect is that after you have prepared a color, you can
    quickly iterate over them. This is especially useful when you
    iterate over several color at the same time.

    As an example, let us create an operator then adds a zig-zag path
    between two color classes:
    \begin{codeexample}[]
\newcount\leftshorecount   \newcount\rightshorecount
\newcount\mycount          \newcount\myothercount
\def\zigzag{
  \tikzgraphpreparecolor{left shore}\leftshorecount{left shore prefix}
  \tikzgraphpreparecolor{right shore}\rightshorecount{right shore prefix}
  \mycount=0\relax
  \loop
    \advance\mycount by 1\relax%
    % Add the "forward" edge
    \tikzgraphsset{new ->=
      {\csname left shore prefix\the\mycount\endcsname}
      {\csname right shore prefix\the\mycount\endcsname}{}{}}
    \myothercount=\mycount\relax%
    \advance\myothercount by1\relax%
    \tikzgraphsset{new <-=
      {\csname left shore prefix\the\myothercount\endcsname}
      {\csname right shore prefix\the\mycount\endcsname}{}{}}    
  \ifnum\myothercount<\leftshorecount\relax
  \repeat
}
\begin{tikzpicture}
  \graph [color class=left shore, color class=right shore]
  { [operator=\zigzag]
    { [left shore, Cartesian placement]                      a, b, c },
    { [right shore, Cartesian placement, nodes={xshift=1cm}] d, e, f }
  };
\end{tikzpicture}
\end{codeexample}
    Naturally, in order to turn the above code into a usable operator,
    some more code would be needed (like default values and taking
    care of shores of different sizes).
  \end{command}
\end{key}

There are a number of predefined operators, like |clique| or |cycle|,
see the reference Section~\ref{section-library-graphs-reference} for a
complete list.



\subsubsection{Graph Operators for Joining Groups}
\label{section-library-graphs-join}

When you join two nodes |foo| and |bar| by the edge specification
|->|, it is fairly obvious, what should happen: An edge from |(foo)|
to |(bar)| should be created. However, suppose we use an edge
specification between two node sets like |{a,b,c}| and |{d,e,f}|. In
this case, it is not so clear which edges should be created. One might
argue that all possible edges from any node in the first set to any node
in the second set should be added. On the other hand, one might also
argue that only a matching between these two sets should be
created. Things get even more muddy when a longer chain of node sets
are joined.

Instead of fixing how edges are created between two node sets,
\tikzname\ takes a somewhat more general, but also more complicated
approach, which can be broken into two parts. In the following, assume
that the following chain specification is given:
\begin{quote}
  \meta{spec$_1$} \meta{edge specification} \meta{spec$_2$}
\end{quote}
An example might be |{a,b,c} -> {d, e->f}|.

\medskip
\textbf{The source and target vertices.} Let us start with the
question of which vertices of the first node set should be connected to
vertices in the second node set.

There are two predefined special color classes that are used for this:
|source| and |target|. For every group specification, some vertices
are colored as |source| vertices and some vertices are |target|
vertices (a node can both be a target and a source). Initially, every
vertex is both a source and a target, but that can change as we will
see in a moment.

The intuition behind source and target vertices is that, in some
sense, edges ``from the outside'' lead into the group via the source
vertices and lead out of the group via the target vertices. To be more
precise, the following happens:
\begin{enumerate}
\item The target vertices of the first group are connected to
  the source vertices of the second group.
\item In the group resulting from the union of the nodes from
  \meta{spec$_1$} and \meta{spec$_2$}, the source vertices are only
  those from the first group, and the target vertices are only those
  from the second group.
\end{enumerate}

Let us go over the effect of these rules for the example
|{a,b,c} -> {d, e->f}|. First, each individual node is initially both
a |source| and a |target| vertex. Then, in |{a,b,c}| all nodes are
still both source and target vertices since just grouping vertices
does not change their colors. Now, in |e->f| something interesting
happens for the first time: the target vertices of the ``group'' |e|
(which is just the node |e|) are connected to the source vertices of
the ``group'' |f|. This means, that an edge is added from |e| to
|f|. Then, in the resulting group |e->f| the only source vertex is |e|
and the only target vertex is |f|. This implies that in the group
|{d,e->f}| the sources are |d| and |e| and the targets are |d| and~|f|.

Now, in |{a,b,c} -> {d,e->f}| the targets  of |{a,b,c}| (which are all
three of them) are connected to the sources of |{d,e->f}| (which are
just |d| and~|e|). Finally, in the whole graph only |a|, |b|, and |c|
are sources while only  |d| and |f| are targets.

\begin{codeexample}[]
\def\hilightsource#1{\fill [green, opacity=.25] (#1) circle [radius=2mm]; }  
\def\hilighttarget#1{\fill [red,   opacity=.25] (#1) circle [radius=2mm]; }  
\tikz \graph
  [operator=\tikzgraphforeachcolorednode{source}{\hilightsource},
   operator=\tikzgraphforeachcolorednode{target}{\hilighttarget}]
  { {a,b,c} -> {d, e->f} };
\end{codeexample}

The next objective is to make more precise what it means that ``the
targets of the first graph'' and the ``sources of the second graph''
should be connected. We know already of a general way of connecting
nodes of a graph: operators! Thus, we use an operator for this job.
For instance, the |complete bipartite| operator adds an edge from every node
having a certain color to every node have a certain other color. This
is exactly what we need here: The first color is ``the color |target|
restricted to the nodes of the first graph'' and the second color is
``the color |source| restricted to the nodes of the second graph.''

However, we cannot really specify that only nodes from a certain
subgraph are meant -- the |operator| machinery only operates on all
nodes of the current graph. For this reason, what really happens is
the following: When the |graph| command encounters \meta{spec$_1$}
\meta{edge specification} \meta{spec$_2$}, it first computes and
colors the nodes of the first and the second specification
independently. Then, the |target| nodes of the first graph are
recolored to |target'| and the |source| nodes of the second graph are
recolored to |source'|. Then, the two graphs are united into one
graph and a \emph{joining operator} is executed, which should add
edges between |target'| and |source'|. Once this is done, 
the colors |target'| and |source'| get erased. Note that in the
resulting graph only the |source| nodes from the first graph are still
|source| nodes and likewise for the |target| nodes of the second graph.


\medskip
\textbf{The joining operators.} The job of a joining operator is
to add edges between nodes colored |target'| and |source'|. The
following rule is used to determine which operator should be chosen
for performing this job:
\begin{enumerate}
\item If the \meta{edge specification} explicitly sets the |operator|
  key to something non-empty (and also not to |\relax|), then the
  \meta{code} of this |operator| call is used.
\item Otherwise, the current value of the following key is used: 
  \begin{key}{/tikz/graphs/default edge operator=\meta{key}}
    This key stores the name of a \meta{key} that is executed for every
    \meta{edge specification} whose \meta{options} do not contain the
    |operator| key.
\begin{codeexample}[]
\tikz \graph [default edge operator=matching] {
  {a, b}    ->[matching and star]
  {c, d, e} --[complete bipartite]
  {f, g, h} -- 
  {i, j, k}
};    
\end{codeexample}
  \end{key}
\end{enumerate}

A typical joining operator is |complete bipartite|. It takes the names of two
color classes as input and adds edges from all vertices of the first
class to all vertices of the second class. Now, the trick is that the
default value for the |complete bipartite| key is |{target'}{source'}|. Thus,
if you just write |->[complete bipartite]|, the same happens as if you had
written
\begin{quote}
|->[complete bipartite={target'}{source'}]|    
\end{quote}
This is exactly what we want to happen. The same default values are
also set for other joining operators like |matching| or |butterfly|.

Even though an operator like |complete bipartite| is typically used
together with an edge specification, it can also be used as a normal
operator together with a group specification. In this case, however,
the color classes must be named explicitly:

\begin{codeexample}[]
\begin{tikzpicture}
  \graph [color class=red, color class=green, math nodes]
  { [complete bipartite={red}{green}]
    { [red,   Cartesian placement]                      r_1, r_2, r_3 },
    { [green, Cartesian placement, nodes={xshift=1cm}]  g_1, g_2, g_3 }
  };
\end{tikzpicture}
\end{codeexample}

A list of predefined joining operators can be found in the reference
Section~\ref{section-library-graphs-reference}.

The fact that joining operators can also be used as normal operators
leads to a subtle problem: A normal operator will typically use the
current value of |default edge kind| to decide which kind of edges
should be put between the identified vertices, while a joining
operator should, naturally, use the kind of edge specified by the
\meta{edge specification}. This problem is solved as follows: Like a
normal operator, a joining operator should also use the current value
of |default edge kind| for the edges it produces. The trick is that
this will automatically be set to the current \meta{edge
  specification} when the operator explicitly in the \meta{options} of
the edge specification or implicitly in the |default edge operator|.




\subsection{Graph Macros}
\label{section-library-graphs-macros}

A \emph{graph macro} is a small graph that is inserted at some point
into the graph that is currently being constructed. There is special
support for such graph macros in \tikzname. You might wonder why this
is necessary -- can't one use \TeX's normal macro mechanism? The
answer is ``no'': one cannot insert new nodes into a graph using
normal macros because the chains, groups, and nodes are determined
prior to macro expansion. Thus, any macro encountered where some node
text should go will only be expanded when this node is being named and 
typeset.

A graph macro is declared using the following key:

\begin{key}{/tikz/graphs/declare=\marg{graph name}\marg{specification}}
  This key declares that \meta{graph name} can subsequently be used as
  a replacement for a \meta{node name}. Whenever the \meta{graph name}
  is used in the following, a graph group will be inserted instead
  whose content is exactly \meta{specification}. In case \meta{graph
    name} is used together with some \meta{options}, they are executed
  prior to inserting the \meta{specification}.
\begin{codeexample}[]
\tikz \graph [branch down=4mm, declare={claw}{1 -- {2,3,4}}] {
  a;
  claw;
  b;
};
\end{codeexample}
  In the next example, we use a key to configure a subgraph:
\begin{codeexample}[]
\tikz \graph [ n/.code=\def\n{#1}, branch down=4mm,
               declare={star}{root -- { \foreach \i in {1,...,\n} {\i} }}]
{ star [n=5]; };
\end{codeexample}
  Actually, the |n| key is already defined internally for a similar
  purpose.

  As a last example, let us define a somewhat more complicated graph
  macro. 
\begin{codeexample}[]
\newcount\mycount  
\tikzgraphsset{
  levels/.store in=\tikzgraphlevel,
  levels=1,
  declare={bintree}{%
    [/utils/exec={%
      \ifnum\tikzgraphlevel=1\relax%
        \def\childtrees{ / }%
      \else%
        \mycount=\tikzgraphlevel%
        \advance\mycount by-1\relax%
        \edef\childtrees{
          / -> {
            bintree[levels=\the\mycount],
            bintree[levels=\the\mycount]
          }}
      \fi%
    },
    parse/.expand once=\childtrees
    ]
    % Everything is inside the \childtrees...
  }
}
\tikz \graph [grow down=5mm, branch right=5mm] { bintree [levels=5] };
\end{codeexample}
\end{key}

Note that when you use a graph macro several time inside the same
graph, you will typically have to use the |name| option so that
different copies of the subgraph are created:
\begin{codeexample}[]
\tikz \graph [branch down=4mm, declare={claw}{1 -- {2,3,4}}] {
  claw [name=left],
  claw [name=right]
};
\end{codeexample}

You will find a list of useful graph macros in the reference section,
Section~\ref{section-library-graphs-reference-macros}. 


\subsection{Online Placement Strategies}
\label{section-library-graphs-placement}

The main job of the |graph| library is to make it easy to specify
which nodes are present in a graph and how they are connected. In
contrast, it is \emph{not} the primary job of the library to compute
good positions for nodes in a graph -- use for instance a |\matrix|,
specify good positions ``by hand'' or use the graph drawing
facilities. Nevertheless, some basic support for automatic node
placement is provided for simple cases. The graph library will provide
you with information about the position of nodes inside their groups
and chains. 

As a graph is being constructed, a \emph{placement strategy} is used
to determine a (reasonably good) position for the nodes as they are
created. These placement strategies get some information about what
\tikzname\ has already seen concerning the already constructed nodes,
but it gets no information concerning the upcoming nodes. Because of
this lack of information concerning the future, the strategies need to
be what is called an \emph{online strategy} in computer science. (The
opposite are \emph{offline strategies}, which get information about
the whole graph and all the sizes of the nodes in it. The graph
drawing libraries employ such offline strategies.)

Strategies are selected using keys like |no placement| or
|Cartesian placement|. It is permissible to use different strategies inside
different parts of a graph, even though the different strategies do
not always work together in perfect harmony.


\subsubsection{Manual Placement}
\label{section-graphs-xy}

\begin{key}{/tikz/graphs/no placement}
  This strategy simply ``switches off'' the whole placement
  mechanism, causing all nodes to be placed at the origin by
  default. You need to use this strategy if you position nodes ``by
  hand''. For this, you can use the |at| key, the |shift| keys:
\begin{codeexample}[]
\tikz \graph [no placement]
{
  a[at={(0:0)}] -> b[at={(1,0)}] -> c[yshift=1cm];
};
\end{codeexample}
  Since the syntax and the many braces and parentheses are a bit
  cumbersome, the following two keys might also be useful: 
  \begin{key}{/tikz/graphs/x=\meta{x dimension}}
    When you use this key, it will have the same effect as if you had written
    |at={(|\meta{x dimension}|,|\meta{y dimension}|)}|, where \meta{y
      dimension} is a value set using the |y| key:
\begin{codeexample}[]
\tikz \graph [no placement]
{
  a[x=0,y=0] -> b[x=1,y=0] -> c[x=0,y=1];
};
\end{codeexample}
    Note that you can specify an |x| or a |y| key for a whole scope
    and then vary only the other key:
\begin{codeexample}[]
\tikz \graph [no placement]
{
  a ->
  { [x=1] % group option
    b [y=0] -> c[y=1]
  };
};
\end{codeexample}
    Note that these keys have the path |/tikz/graphs/|, so they will
    be available inside |graph|s and will not clash with the usual |x|
    and |y| keys of \tikzname, which are used to specify the basic
    lengths of vectors.
  \end{key}
  \begin{key}{/tikz/graphs/y=\meta{y dimension}}
    See above.    
  \end{key}
\end{key}




\subsubsection{Placement on a Grid}

\begin{key}{/tikz/graphs/Cartesian placement}
  This strategy is the default strategy. It works, roughly, as
  follows: For each new node on a chain, advance a ``logical width''
  counter and for each new node in a group, advance a ``logical
  depth'' counter. When a chain contains a whole group, then the
  ``logical width'' taken up by the group is the maximum over the
  logical widths taken up by the chains inside the group; and
  symmetrically the logical depth of a chain is the maximum of the
  depths of the groups inside it.
  
  This slightly confusing explanation is perhaps best exemplified. In
  the below example, the two numbers indicate the two logical width
  and depth of each node as computed by the |graph| library. Just
  ignore the arcane code that is used to print these numbers.
\begin{codeexample}[]
\tikz
  \graph [nodes={align=center, inner sep=1pt}, grow right=7mm,
          typeset={\tikzgraphnodetext\\[-4pt]
                   \tiny\mywidth\\[-6pt]\tiny\mydepth},
          placement/compute position/.append code=
            \pgfkeysgetvalue{/tikz/graphs/placement/width}{\mywidth}
            \pgfkeysgetvalue{/tikz/graphs/placement/depth}{\mydepth}]
{
  a,
  b,
  c -> d -> {
    e -> f -> g,
    h -> i
  } -> j,
  k -> l
};
\end{codeexample}  
  You will find a detailed description of how these logical units are
  computed, exactly, in Section~\ref{section-library-graphs-new-online}.
  
  Now, even though we talk about ``widths'' and ``depths'' and even
  though by default a graph ``grows'' to the right and down, this is
  by no means fixed. Instead, you can use the following keys to change
  how widths and heights are interpreted:
  \begin{key}{/tikz/graphs/chain shift=\meta{coordinate} (initially {(1,0)})}
    Under the regime of the |Cartesian placement| strategy, each node is
    shifted by the current logical width times this \meta{coordinate}.
\begin{codeexample}[]
\tikz \graph [chain shift=(45:1)] {
  a -> b -> c;
  d -> e;
  f -> g -> h;
};
\end{codeexample}
  \end{key}  
  \begin{key}{/tikz/graphs/group shift=\meta{coordinate} (initially {(0,-1)})}
    Like for |chain shift|, each node is shifted by the current
    logical depth times this \meta{coordinate}. 
\begin{codeexample}[]
\tikz \graph [chain shift=(45:7mm), group shift=(-45:7mm)] {
  a -> b -> c;
  d -> e;
  f -> g -> h;
};
\end{codeexample}
  \end{key}  
\end{key}


\begin{key}{/tikz/graphs/grow up=\meta{distance} (default 1)}
  Sets the |chain shift| to |(|\meta{distance}|,0)|, so that chains
  ``grow upward.'' The distance by which the center of each new
  element is removed from the center of the previous one is
  \meta{distance}. 
\begin{codeexample}[]
\tikz \graph [grow up=7mm] { a -> b -> c};      
\end{codeexample}
\end{key}
\begin{key}{/tikz/graphs/grow down=\meta{distance} (default 1)}
  Like |grow up|.
\begin{codeexample}[]
\tikz \graph [grow down=7mm] { a -> b -> c};      
\end{codeexample}
\end{key}
\begin{key}{/tikz/graphs/grow left=\meta{distance} (default 1)}
  Like |grow up|.
\begin{codeexample}[]
\tikz \graph [grow left=7mm] { a -> b -> c};      
\end{codeexample}
\end{key}
\begin{key}{/tikz/graphs/grow right=\meta{distance} (default 1)}
  Like |grow up|.
\begin{codeexample}[]
\tikz \graph [grow right=7mm] { a -> b -> c};      
\end{codeexample}
\end{key}
\begin{key}{/tikz/graphs/branch up=\meta{distance} (default 1)}
  Sets the |group shift| so that groups ``branch upward.''  The
  distance by which the center of each new element is removed from
  the center of the previous one is \meta{distance}.
\begin{codeexample}[]
\tikz \graph [branch up=7mm] { a -> b -> {c, d, e} };      
\end{codeexample}
  Note that when you draw a tree, the |branch ...| keys specify how
  siblings (or adjacent branches) are arranged, while the |grow ...|
  keys specify in which direction the branches ``grow''.
\end{key}
\begin{key}{/tikz/graphs/branch down=\meta{distance} (default 1)}
\begin{codeexample}[]
\tikz \graph [branch down=7mm] { a -> b -> {c, d, e}};      
\end{codeexample}
\end{key}
\begin{key}{/tikz/graphs/branch left=\meta{distance} (default 1)}
\begin{codeexample}[]
\tikz \graph [branch left=7mm, grow down=7mm] { a -> b -> {c, d, e}};      
\end{codeexample}
\end{key}
\begin{key}{/tikz/graphs/branch right=\meta{distance} (default 1)}
\begin{codeexample}[]
\tikz \graph [branch right=7mm, grow down=7mm] { a -> b -> {c, d, e}};      
\end{codeexample}
\end{key}

The following keys place nodes in a $N\times M$ grid. 
\begin{key}{/tikz/graphs/grid placement}
  This key works similar to |Cartesian placement|. As for that placement
  strategy, a node has logical width and depth 1. However, the computed
  total width and depth are mapped to a $N\times M$ grid.
  The values of $N$ and $M$ depend on the size of the graph and the
  value of |wrap after|. The number of columns $M$ is either set to 
  |wrap after| explicitly or computed automatically as 
  $\sqrt{\verb!|V|!}$. $N$ is the number of rows needed to lay out the 
  graph in a grid with $M$ columns. 
\begin{codeexample}[]
% An example with 6 nodes, 3 columns and therefor 2 rows
\tikz \graph [grid placement] { subgraph I_n[n=6, wrap after=3] };
\end{codeexample}
\begin{codeexample}[]
% An example with 9 nodes with columns and rows computed automatically
\tikz \graph [grid placement] { subgraph Grid_n [n=9] };
\end{codeexample}
\begin{codeexample}[]
% Directions can be changed
\tikz \graph [grid placement, branch up, grow left] { subgraph Grid_n [n=9] };
\end{codeexample}
  In case a user-defined graph instead of a pre-defined
  |subgraph| is to be layed out using |grid placement|, |n| has to be
  specified explicitly:
\begin{codeexample}[]
\tikz \graph [grid placement] { 
  [n=6, wrap after=3] 
  a -- b -- c -- d -- e -- f 
};
\end{codeexample}
\end{key}



\subsubsection{Placement Taking Node Sizes Into Account}

Options like |grow up| or |branch right| do not take the sizes of the
to-be-positioned nodes into account -- all nodes are placed quite
``dumbly'' at grid positions. It turns out that the
|Cartesian placement| can also be used to place notes in such a way
that their height and/or width is taken into account. Note, however,
that while the following options may yield an adequate placement in
many situations, when you need advanced alignments you should use a
|matrix| or advanced offline strategies to place the nodes.


\begin{key}{/tikz/graphs/grow right sep=\meta{distance} (default 1em)}
  This key has several effects, but let us start with the bottom line:
  Nodes along a chain are placed in such a way that the left end of a
  new node is \meta{distance} from the right end of the previous node: 
\begin{codeexample}[]
\tikz \graph [grow right sep, left anchor=east, right anchor=west] {
  start -- {
    long text -- {short, very long text} -- more text,
    long -- longer -- longest
  } -- end
};      
\end{codeexample}
  What happens internally is the following: First, the |anchor| of the
  nodes is set to |west| (or |north west| or |south west|, see
  below). Second, the logical width of a node is no 
  longer |1|, but set to the actual width of the node (which we define
  as the horizontal difference between the |west| anchor and the
  |east| anchor) in points. Third, the |chain shift| is set to
  |(1pt,0pt)|.
\end{key}
\begin{key}{/tikz/graphs/grow left sep=\meta{distance} (default 1em)}
\begin{codeexample}[]
\tikz \graph [grow left sep] { long -- longer -- longest };      
\end{codeexample}
\end{key}
\begin{key}{/tikz/graphs/grow up sep=\meta{distance} (default 1em)}
\begin{codeexample}[]
\tikz \graph [grow up sep] {
  a / $a=x$ --
  b / {$b=\displaystyle \int_0^1 x dx$} --
  c [draw, circle, inner sep=7mm]
};      
\end{codeexample}
\end{key}
\begin{key}{/tikz/graphs/grow down sep=\meta{distance} (default 1em)}
  As above.
\end{key}

\begin{key}{/tikz/graphs/branch right sep=\meta{distance} (default 1em)}
  This key works like |grow right sep|, only it affects groups rather
  than chains.
\begin{codeexample}[]
\tikz \graph [grow down, branch right sep] {
  start -- {
    an even longer text -- {short, very long text} -- more text,
    long -- longer -- longest,
    some text -- a -- b
  } -- end
};      
\end{codeexample}
  When both this key and, say, |grow down sep| are set, instead of the
  |west| anchor, the |north west| anchor will be selected
  automatically. 
\end{key}

\begin{key}{/tikz/graphs/branch left sep=\meta{distance} (default 1em)}
\begin{codeexample}[]
\tikz \graph [grow down sep, branch left sep] {
  start -- {
    an even longer text -- {short, very long text} -- more text,
    long -- longer,
    some text -- a -- b
  } -- end
};      
\end{codeexample}
\end{key}

\begin{key}{/tikz/graphs/branch up sep=\meta{distance} (default 1em)}
\begin{codeexample}[]
\tikz \graph [branch up sep] { a, b, c[draw, circle, inner sep=7mm] };      
\end{codeexample}
\end{key}

\begin{key}{/tikz/graphs/branch down sep=\meta{distance} (default 1em)}
\end{key}



\subsubsection{Placement On a Circle}

The following keys place nodes on circles. Note that, typically, you
do not use |circular placement| directly, but rather use one of the
two keys |clockwise| or |counterclockwise|.

\begin{key}{/tikz/graphs/circular placement}
  This key works quite similar to |Cartesian placement|. As for that
  placement strategy, a node has logical width and depth |1|. However,
  the computed total width and depth are mapped to polar coordinates
  rather than Cartesian coordinates.
  
  \begin{key}{/tikz/graphs/chain polar shift=|(|\meta{angle}|:|\meta{radius}|)| (initially {(0:1)})}
    Under the regime of the |circular placement| strategy, each node
    on a chain is shifted by |(|\meta{logical
      width}\meta{angle}|:|\meta{logical width}\meta{angle}|)|.
\begin{codeexample}[]
\tikz \graph [circular placement] {
  a -> b -> c;
  d -> e;
  f ->  g -> h;
};
\end{codeexample}
  \end{key}  
  \begin{key}{/tikz/graphs/group polar shift=|(|\meta{angle}|:|\meta{radius}|)| (initially {(45:0)})}
    Like for |group shift|, each node
    on a chain is shifted by |(|\meta{logical
      depth}\meta{angle}|:|\meta{logical depth}\meta{angle}|)|.
\begin{codeexample}[]
\tikz \graph [circular placement, group polar shift=(30:0)] {
  a -> b -> c;
  d -> e;
  f -> g -> h;
};
\end{codeexample}
\begin{codeexample}[]
\tikz \graph [circular placement,
              chain polar shift=(30:0),
              group polar shift=(0:1cm)] {
  a -- b -- c;
  d -- e;
  f -- g -- h;
};
\end{codeexample}
  \end{key}
  \begin{key}{/tikz/graphs/radius=\meta{dimension} (initially 1cm)}
    This is an initial value that is added to the total computed
    radius when the polar shift of a node has been
    calculated. Essentially, this key allows you to set the
    \meta{radius} of the innermost circle.
\begin{codeexample}[]
\tikz \graph [circular placement, radius=5mm] { a, b, c, d };
\end{codeexample}
\begin{codeexample}[]
\tikz \graph [circular placement, radius=1cm] { a, b, c, d };
\end{codeexample}
  \end{key}
  \begin{key}{/tikz/graphs/phase=\meta{angle} (initially 90)}
    This is an initial value that is added to the total computed
    angle when the polar shift of a node has been
    calculated. 
\begin{codeexample}[]
\tikz \graph [circular placement] { a, b, c, d };
\end{codeexample}
\begin{codeexample}[]
\tikz \graph [circular placement, phase=0] { a, b, c, d };
\end{codeexample}
  \end{key}
\end{key}

\label{key-graphs-clockwise}%
\begin{key}{/tikz/graphs/clockwise=\meta{number} (default \string\tikzgraphVnum)}
  This key sets the |group shift| so that if there are exactly
  \meta{number} many nodes in a group, they will form a complete
  circle. If you do not provide a \meta{number}, the current value of
  |\tikzgraphVnum| is used, which is exactly what you want when you
  use predefined graph macros like |subgraph K_n|.
\begin{codeexample}[]
\tikz \graph [clockwise=4] { a, b, c, d };
\end{codeexample}
\begin{codeexample}[]
\tikz \graph [clockwise] { subgraph K_n [n=5] };
\end{codeexample}
\end{key}

\label{key-graphs-counterclockwise}%
\begin{key}{/tikz/graphs/counterclockwise=\meta{number} (default \string\tikzgraphVnum)}
  Works like |clockwise|, only the direction is inverted.
\end{key}


\subsubsection{Levels and Level Styles}

As a graph is being parsed, the |graph| command keeps track of a
parameter called the \emph{level} of a node. Provided that the graph
is actually constructed in a tree-like manner, the level is exactly
equal to the level of the node inside this tree.

\begin{key}{/tikz/graphs/placement/level}
  This key stores a number that is increased for each element on a
  chain, but gets reset at the end of a group:
\begin{codeexample}[]
\tikz \graph [ branch down=5mm, typeset=
    \tikzgraphnodetext:\pgfkeysvalueof{/tikz/graphs/placement/level}]
{  
  a -> {
    b,
    c -> {
      d,
      e -> {f,g},
      h
    },
    j
  }
};
\end{codeexample}
  Unlike the parameters |depth| and |width| described in the next
  section, the key |level| is always available.
\end{key}

In addition to keeping track of the value of the |level| key, the
|graph| command also executes the following keys whenever it creates a
node:

\begin{stylekey}{/tikz/graph/level=\meta{level}}
  This key gets executed for each newly created node with \meta{level}
  set to the current level of the node. You can use this key to, say,
  reconfigure the node distance or the node color.
\end{stylekey}

\begin{stylekey}{/tikz/graph/level \meta{level}}
  This key also gets executed for each newly created node with
  \meta{level} set to the current level of the node.
\begin{codeexample}[]
\tikz \graph [
  branch down=5mm,
  level 1/.style={nodes=red},
  level 2/.style={nodes=green!50!black},
  level 3/.style={nodes=blue}]
{  
  a -> {
    b,
    c -> {
      d,
      e -> {f,g},
      h
    },
    j
  }
};
\end{codeexample}
\begin{codeexample}[]
\tikz \graph [
  branch down=5mm,
  level 1/.style={grow right=2cm},
  level 2/.style={grow right=1cm},
  level 3/.style={grow right=5mm}]
{  
  a -> {
    b,
    c -> {
      d,
      e -> {f,g},
      h
    },
    j
  }
};
\end{codeexample}
\end{stylekey}



\subsubsection{Defining New Online Placement Strategies}

\label{section-library-graphs-new-online}

In the following the details of how to define a new placement strategy
are explained. Most readers may wish to skip this section.

As a graph specification is being parsed, the |graph| library will keep
track of different numbers that identify the positions of the
nodes. Let us start with what happens on a chain. First, the following 
counter is increased for each element of the chain:
\begin{key}{/tikz/graphs/placement/element count}
  This key stores a number that tells us the position of the node on
  the current chain. However, you only have access to this value
  inside the code passed to the macro |compute position|, explained
  later on.
\begin{codeexample}[]
\tikz \graph [
  grow right sep, typeset=\tikzgraphnodetext:\mynum,
  placement/compute position/.append code=
    \pgfkeysgetvalue{/tikz/graphs/placement/element count}{\mynum}]
{
  a -> b -> c,
  d -> {e, f->h} -> j
};
\end{codeexample}
  As can be seen, each group resets the element counter.
\end{key}

The second value that is computed is more complicated to explain, but
it also gives more interesting information:
\begin{key}{/tikz/graphs/placement/width}
  This key stores the ``logical width'' of the nodes parsed up to now
  in the current group or chain (more precisely, parsed since the last
  call of |place| in an enclosing group). This is not necessarily the
  ``total physical width'' of the nodes, but rather a number
  representing how ``big'' the elements prior to the current element
  were. This \emph{may} be their width, but it may also be their
  height or even their number (which, incidentally, is the default). 
  You can use the |width| to perform shifts or rotations of
  to-be-created nodes (to be explained later).
  
  The logical width is defined recursively as follows. First, the
  width of a single node is computed by calling the following key:
  \begin{key}{/tikz/graphs/placement/logical node width=\meta{full
        node name}}
    This key is called to compute a physical or logical width of the
    node \meta{full node name}. You can change the code of this
    key. The code should return the computed value in the macro
    |\pgfmathresult|. By default, this key returns |1|.
  \end{key}
  The width of a chain is the sum of the widths of its elements. The 
  width of a group is the maximum of the widths of its elements.

  To get a feeling what the above rules imply in practice, let us
  first have a look at an example where each node has logical width
  and height |1| (which is the default). The arcane options at the
  beginning of the code just setup things so that the computed width
  and depth of each node is displayed at the bottom of each node.
\begin{codeexample}[]
\tikz
  \graph [nodes={align=center, inner sep=1pt}, grow right=7mm,
          typeset={\tikzgraphnodetext\\[-4pt]
                   \tiny\mywidth\\[-6pt]\tiny\mydepth},
          placement/compute position/.append code=
            \pgfkeysgetvalue{/tikz/graphs/placement/width}{\mywidth}
            \pgfkeysgetvalue{/tikz/graphs/placement/depth}{\mydepth}]
{
  a,
  b,
  c -> d -> {
    e -> f -> g,
    h -> i
  } -> j,
  k -> l
};
\end{codeexample}
  In the next example the ``logical'' width and depth actually match
  the ``physical'' width and height. This is caused by the
  |grow right sep| option, which internally sets the
  |logical node width| key so that it returns the width of its
  parameter in points.
\begin{codeexample}[]
\tikz
  \graph [grow right sep, branch down sep, nodes={align=left, inner sep=1pt},
          typeset={\tikzgraphnodetext\\[-4pt] \tiny Width: \mywidth\\[-6pt] \tiny Depth: \mydepth},
          placement/compute position/.append code=
            \pgfkeysgetvalue{/tikz/graphs/placement/width}{\mywidth}
            \pgfkeysgetvalue{/tikz/graphs/placement/depth}{\mydepth}]
{
  a,
  b,
  c -> d -> {
    e -> f -> g,
    h -> i
  } -> j,
  k -> l
};
\end{codeexample}  
\end{key}

Symmetrically to chains, as a group is being constructed, counters are
available for the number of chains encountered so far in the current
group and for the logical depth of the current group:
\begin{key}{/tikz/graphs/placement/element count}
  This key stores a number that tells us the sequence number of the
  chain in the current group.
\begin{codeexample}[]
\tikz \graph [
  grow right sep, branch down=5mm, typeset=\tikzgraphnodetext:\mynum,
  placement/compute position/.append code=
    \pgfkeysgetvalue{/tikz/graphs/placement/chain count}{\mynum}]
{
  a -> b -> {c,d,e},
  f,
  g -> h
};
\end{codeexample}
\end{key}

\begin{key}{/tikz/graphs/placement/depth}
  Similarly to the |width| key, this key stores the ``logical depth''
  of the nodes parsed up to now in the current group or chain and, also
  similarly, this key may or may not be related to the actual
  depth/height of the current node. 
  As for the |width|, the exact definition is as follows: For a single
  node, the depth is computed by the following key: 
  \begin{key}{/tikz/graphs/placement/logical node depth=\meta{full
        node name}}
    The code behind this key should return the ``logical height'' of
    the node   \meta{full node name} in the macro |\pgfmathresult|. 
  \end{key}
  Second, the depth of a group is the sum of the depths of its
  elements. Third, the depth of a chain is the maximum of the depth of
  its elements.  
\end{key}

The |width|, |depth|, |element count|, and |chain count| keys get
updated automatically, but do not have an effect by themselves. This
is to the following two keys:

\begin{key}{/tikz/graphs/placement/compute position=\meta{code}}
  The \meta{code} is called by the |graph| command just prior to
  creating a new node (the exact moment when this key is called is
  detailed in the description of the |place| key). When the
  \meta{code} is called, all of the keys described above will hold
  numbers computed in the way described above.

  The job of the \meta{code} is to setup node options appropriately so
  that the to-be-created node will be placed correctly. Thus, the
  \meta{code} should typically set the key
  |nodes={shift=|\meta{coordinate}|}| where \meta{coordinate} is the
  computed position for the node. The \meta{code} could
  also set other options like, say, the color of a node depending on
  its depth.

  The following example appends some code to the standard code of
  |compute position| so that ``deeper'' nodes of a tree are
  lighter. (Naturally, the same effect could be achieved much more
  easily using the |level| key.)
\begin{codeexample}[]
\newcount\mycount  
\def\lightendeepernodes{
  \pgfmathsetcount{\mycount}{
    100-20*\pgfkeysvalueof{/tikz/graphs/placement/width}
  }
  \edef\mydepth{\the\mycount}
  \tikzset{nodes={fill=red!\mydepth,circle,text=white}}
}
\tikz
  \graph [placement/compute position/.append code=\lightendeepernodes]
   {
     a -> {
       b -> c -> d,
       e -> {
         f,
         g
       },
       h
     }
   };
\end{codeexample}
\end{key}

\begin{key}{/tikz/graphs/placement/place}
  Executing this key has two effects: First, the key
  |compute position| is called to compute a good 
  position for future nodes (usually, these ``future nodes'' are just
  a single node that is created immediately). Second, all of the above
  counters like |depth| or |width| are reset (but not |level|).

  There are two places where this key is sensibly called: First, just
  prior to creating a node, which happens automatically. Second, when
  you change the online strategy. In this case, the computed width and
  depth values from one strategy typically make no sense in the other
  strategy, which is why the new strategy should proceed ``from a
  fresh start.'' In this case, the implicit call of |compute position|
  ensures that the new strategy gets the last place the old strategy
  would have used as its starting point, while the computation of its
  positions is now relative to this new starting point.

  For these reasons, when an online strategy like
  |Cartesian placement| is called, this key gets called
  implicitly. You will rarely need to call this key directly, except
  when you define a new online strategy.
\end{key}



\subsection{Reference: Predefined Elements}

\label{section-library-graphs-reference}


\subsubsection{Graph Macros}
\label{section-library-graphs-reference-macros}

\begin{tikzlibrary}{graph.standard}
  This library defines a number of graph macros that are often used in
  the literature. When new graphs are added to this collection, they
  will follow the definitions in the Mathematica program, see
  |mathworld.wolfram.com/topics/SimpleGraphs.html|. 
\end{tikzlibrary}


\begin{graph}{subgraph I\_n}
  This graph consists just of $n$ unconnected vertices. The following
  key is used to specify the set of these vertices:
  \begin{key}{/tikz/graphs/V=\marg{list of vertices}}
    Sets a list of vertex names for use with graphs like
    |subgraph I_n| and also other graphs. This list is available in
    the macro |\tikzgraphV|. The number of elements of this list is
    available in |\tikzgraphVnum|.
  \end{key}
  \begin{key}{/tikz/graphs/n=\meta{number}}
    This is an abbreviation for
    |V={1,...,|\meta{number}|}, name shore V={name=V}|.
  \end{key}
\begin{codeexample}[]
\tikz \graph [branch right, nodes={draw, circle}]
  { subgraph I_n [V={a,b,c}] };    
\end{codeexample}
  This graph is not particularly exciting by itself. However, it is
  often used to introduce nodes into a graph that are then connected
  as in the following example:
\begin{codeexample}[]
\tikz \graph [clockwise, clique] { subgraph I_n [n=4] };    
\end{codeexample}
\end{graph}


\begin{graph}{subgraph I\_nm}
  This graph consists of two sets of once $n$ unconnected vertices and
  then $m$ unconnected vertices. The first set consists of the
  vertices set by the key |V|, the other set consists of the vertices
  set by the key |W|.
\begin{codeexample}[]
\tikz \graph { subgraph I_nm [V={1,2,3}, W={a,b,c}] };    
\end{codeexample}
  In order to set the graph path name of the two
  sets, the following keys get executed:
  \begin{stylekey}{/tikz/graphs/name shore V (initially \normalfont empty)}
    Set this style to, say, |name=my V set| in order to set a
    name for the |V| set.    
  \end{stylekey}
  \begin{stylekey}{/tikz/graphs/name shore W (initially \normalfont empty)}
    Same as for |name shore V|.
  \end{stylekey}
  \begin{key}{/tikz/graphs/W=\marg{list of vertices}}
    Sets the list of vertices for the |W| set. The elements and
    their number are available in the macros |\tikzgraphW| and
    |\tikzgraphWnum|, respectively.
  \end{key}
  \begin{key}{/tikz/graphs/m=\meta{number}}
    This is an abbreviation for
    |W={1,...,|\meta{number}|}, name shore W={name=W}|.
  \end{key}
  The main purpose of this subgraph is to setup the nodes in a
  bipartite graph:
\begin{codeexample}[]
\tikz \graph {
  subgraph I_nm [n=3, m=4];

  V 1 -- { W 2, W 3 };
  V 2 -- { W 1, W 3 };
  V 3 -- { W 1, W 4 };
};    
\end{codeexample}
\end{graph}

\begin{graph}{subgraph K\_n}
  This graph is the complete clique on the vertices from the |V| key. 
\begin{codeexample}[]
\tikz \graph [clockwise] { subgraph K_n [n=7] };    
\end{codeexample}
\end{graph}


\begin{graph}{subgraph K\_nm}
  This graph is the complete bipartite graph with the two shores |V|
  and |W| as in |subgraph I_nm|.
\begin{codeexample}[]
\tikz \graph [branch right, grow down]
  { subgraph K_nm [V={6,...,9}, W={b,...,e}] };    
\end{codeexample}
\begin{codeexample}[]
\tikz \graph [simple, branch right, grow down]
{
  subgraph K_nm [V={1,2,3}, W={a,b,c,d}, ->];
  subgraph K_nm [V={2,3},   W={b,c},     <-];
};    
\end{codeexample}
\end{graph}

\begin{graph}{subgraph P\_n}
  This graph is the path on the vertices in |V|.
\begin{codeexample}[]
\tikz \graph [branch right] { subgraph P_n [n=3] };    
\end{codeexample}
\end{graph}


\begin{graph}{subgraph C\_n}
  This graph is the cycle on the vertices in |V|.
\begin{codeexample}[]
\tikz \graph [clockwise] { subgraph C_n [n=7, ->] };    
\end{codeexample}
\end{graph}


\begin{graph}{subgraph Grid\_n}
  This graph is a grid of the vertices in |V|.
  \begin{key}{/tikz/graphs/wrap after=\meta{number}}
    Defines the number of nodes placed in a single row of the grid. This
    value implicitly defines the number of grid columns as well.
    In the following example a |grid placement| is used to visualize the
    edges created between the nodes of a |Grid_n| |subgraph| using
    different values for |wrap after|.
    \begin{codeexample}[]
\tikz \graph [grid placement] { subgraph Grid_n [n=3,wrap after=1] };
\tikz \graph [grid placement] { subgraph Grid_n [n=3,wrap after=3] };
    \end{codeexample}
    \begin{codeexample}[]
\tikz \graph [grid placement] { subgraph Grid_n [n=4,wrap after=2] };
\tikz \graph [grid placement] { subgraph Grid_n [n=4] };
    \end{codeexample}
  \end{key}
\end{graph}


% TODO: Implement the Grid_nm subgraph described here:
%
%\begin{graph}{subgraph Grid\_nm}
%  This graph is a grid built from the cartesian product of the two node
%  sets |V| and |W| which are either defined using the keys
%  |/tikz/graphs/V| and |/tikz/graphs/W| or |/tikz/graphs/n| and
%  |/tikz/graphs/m| or a mixture of both. 
%  
%  The resulting |Grid_nm| subgraph has $n$ ``rows'' and $m$ ``columns'' and
%  the nodes are named |V i W j| with $1\le i\le n$ and $1\le j\le n$.
%  The names of the two shores |V| and |W| can be changed as described in
%  the documentation of the keys |/tikz/graphs/name shore V| and
%  |/tikz/graphs/name shore W|.
%  \begin{codeexample}[]
%\tikz \graph [grid placement] { subgraph Grid_nm [V={1,2,3}, W={4, 5, 6}] };
%  \end{codeexample}
%\end{graph}



\subsubsection{Group Operators}

The following keys use the |operator| key to setup operators that
connect the vertices of the current group having a certain color in a
specific way.

\begin{key}{/tikz/graphs/clique=\meta{color} (default all)}
  Adds an edge between all vertices of the current group having the
  (logical) color \meta{color}. Since, by default, this color is set
  to |all|, which is a color that all nodes get by default, when you
  do not specify anything, all nodes will be connected.
\begin{codeexample}[]
\tikz \graph [clockwise, n=5] {
  a,
  b,
  {
    [clique]
    c, d, e
  }
};    
\end{codeexample}
\begin{codeexample}[]
\tikz \graph [color class=red, clockwise, n=5] {
  [clique=red, ->]
  a, b[red], c[red], d, e[red]
};    
\end{codeexample}
\end{key}

\begin{key}{/tikz/graphs/induced independent set=\meta{color} (default all)}
  This key is the ``opposite'' of a |clique|: It removes all edges in
  the current group having belonging to color class \meta{color}. More
  precisely, an edge of kind |-!-| is added for each pair of
  vertices. This means that edge only get removed if you specify the
  |simple| option.
\begin{codeexample}[]
\tikz \graph [simple] {
  subgraph K_n [<->, n=7, clockwise]; % create lots of edges
  
  { [induced independent set] 1, 3, 4, 5, 6 }
};    
\end{codeexample}
\end{key}


\begin{key}{/tikz/graphs/cycle=\meta{color} (default all)}
  Connects the nodes colored \meta{color} is a cyclic fashion. The
  ordering is the ordering in which they appear in the whole graph
  specification.
\begin{codeexample}[]
\tikz \graph [clockwise, n=6, phase=60] {
  { [cycle, ->] a, b, c },
  { [cycle, <-] d, e, f }
};    
\end{codeexample}
\end{key}

\begin{key}{/tikz/graphs/induced cycle=\meta{color} (default all)}
  While the |cycle| command will only add edges, this key will also
  remove all other edges between the nodes of the cycle, provided we
  are constructing a |simple| graph.
\begin{codeexample}[]
\tikz \graph [simple] {
  subgraph K_n [n=7, clockwise]; % create lots of edges
  
  { [induced cycle, ->, edge=red] 2, 3, 4, 6, 7 },
};    
\end{codeexample}
\end{key}
  
\begin{key}{/tikz/graphs/path=\meta{color} (default all)}
  Works like |cycle|, only there is no edge from the last to the first
  vertex. 
\begin{codeexample}[]
\tikz \graph [clockwise, n=6] {
  { [path, ->] a, b, c },
  { [path, <-] d, e, f }
};    
\end{codeexample}
\end{key}
\begin{key}{/tikz/graphs/induced path=\meta{color} (default all)}
  Works like |induced cycle|, only there is no edge from the last to the first
  vertex. 
\begin{codeexample}[]
\tikz \graph [simple] {
  subgraph K_n [n=7, clockwise]; % create lots of edges
  
  { [induced path, ->, edges=red] 2, 3, 4, 6, 7 },
};    
\end{codeexample}
\end{key}


\subsubsection{Joining Operators}

The following keys are typically used as options of an \meta{edge
  specification}, but can also be called in a group specification
(however, then, the colors need to be set explicitly).

\begin{key}{/tikz/graphs/complete bipartite=\meta{from color}\meta{to
      color} (default \char`\{source'\char`\}\char`\{target'\char`\})}
  Adds all possible edges from every node having color \meta{from color}
  to every node having color \meta{to color}: 
\begin{codeexample}[]
\tikz \graph { {a, b}       ->[complete bipartite]
               {c, d, e}    --[complete bipartite]
               {g, h, i, j} --[complete bipartite]
               k };    
\end{codeexample}
\begin{codeexample}[]
\tikz \graph [color class=red, color class=green, clockwise, n=6] {
  [complete bipartite={red}{green}, ->]
  a [red], b[red], c[red], d[green], e[green], f[green]
};
\end{codeexample}
\end{key}
\begin{key}{/tikz/graphs/induced complete bipartite}
  Works like the |complete bipartite| operator, but in a |simple|
  graph any edges between the vertices in either shore are removed
  (more precisely, they get replaced by |-!-| edges).
\begin{codeexample}[]
\tikz \graph [simple] {
  subgraph K_n [n=5, clockwise];  % Lots of edges
  
  {2, 3} ->[induced complete bipartite] {4, 5}
};
\end{codeexample}  
\end{key}
\begin{key}{/tikz/graphs/matching=\meta{from color}\meta{to
      color} (default \char`\{source'\char`\}\char`\{target'\char`\})}
  This joining operator forms a maximum
  \emph{matching} between the nodes of the two sets of nodes having
  colors \meta{from color} and \meta{to color}, respectively. The first node of
  the from set is connected to the first node of to set, the second
  node of the from set is connected to the second node of the to set,
  and so on. If the sets have the same 
  size, what results is what graph theoreticians call a \emph{perfect
    matching}, otherwise only a maximum, but not perfect matching
  results. 
\begin{codeexample}[]
\tikz \graph {
  {a, b, c} ->[matching]
  {d, e, f} --[matching]
  {g, h}    --[matching]
  {i, j, k}
};    
\end{codeexample}
\end{key}

\begin{key}{/tikz/graphs/matching and star=\meta{from color}\meta{to
      color} (default \char`\{source'\char`\}\char`\{target'\char`\})}
  The |matching and star| connector works like the |matching|
  connector, only it behaves differently when the two to-be-connected
  sets have different size. In this case, all the surplus nodes get
  connected to the last node of the other set, resulting in what is known
  as a \emph{star} in graph theory. This simple rule allows
  for some powerful effects (since this connector is the default,
  there is no need to add it here):
  \begin{codeexample}[]
\tikz \graph { a -> {b, c} -> {d, e} -- f};    
  \end{codeexample}
  The |matching and star| connector also makes it easy to create trees and
  series-parallel graphs.
\end{key}

\begin{key}{/tikz/graphs/butterfly=\opt{\meta{options}}}
  The |butterfly| connector is used to create the kind of connections
  present between layers of a so-called \emph{butterfly network}.
  As for other connectors, two sets of nodes are connected, which are
  the nodes having color |target'| and |source'| by default. In a
  \emph{level $l$} connection, the first $l$ nodes of the first set
  are connected to the second $l$ nodes of the second set, while the
  second $l$ nodes of the first set get connected to the first $l$
  nodes of the second set. Then, for next $2l$ nodes of both sets a
  similar kind of connection is installed. Additionally, each node
  gets connected to the corresponding node in the other set with the
  same index (as in a |matching|):
\begin{codeexample}[]
\tikz \graph [left anchor=east, right anchor=west,
              branch down=4mm, grow right=15mm] {
  subgraph I_n [n=12, name=A] --[butterfly={level=3}]
  subgraph I_n [n=12, name=B] --[butterfly={level=2}]
  subgraph I_n [n=12, name=C]
};
\end{codeexample}
  Unlike most joining operators, the colors of the nodes in the first
  and the second set are not passed as parameters to the |butterfly|
  key. Rather, they can be set using the \meta{options}, which are
  executed with the path prefix |/tikz/graphs/butterfly|.
  \begin{key}{/tikz/graphs/butterfly/level=\meta{level} (initially 1)}
    Sets the level $l$ for the connections.
  \end{key}
  \begin{key}{/tikz/graphs/butterfly/from=\meta{color} (initially target')}
    Sets the color class of the from nodes.
  \end{key}
  \begin{key}{/tikz/graphs/butterfly/to=\meta{color} (initially source')}
    Sets the color class of the to nodes.
  \end{key}
\end{key}



%%% Local Variables: 
%%% mode: latex
%%% TeX-master: "pgfmanual-pdftex-version"
%%% End: 

% Copyright 2006 by Till Tantau
%
% This file may be distributed and/or modified
%
% 1. under the LaTeX Project Public License and/or
% 2. under the GNU Free Documentation License.
%
% See the file doc/generic/pgf/licenses/LICENSE for more details.

\section{Matrices and Alignment}

\label{section-matrices}

\subsection{Overview}

When creating pictures, one often faces the problem of correctly
aligning parts of the picture. For example, you might wish that the
base lines of certain nodes should be on the same line and some
further nodes should be below these nodes with, say, their centers on
a vertical lines. There are different ways of solving such
problems. For example, by making clever use of anchors, nearly all
such alignment problems can be solved. However, this often leads to
complicated code. An often simpler way is to use \emph{matrices},
the use of which is explaied in the current section.

A \tikzname\ matrix is similar to \LaTeX's |{tabular}| or
|{array}| environment, only instead of text each cell contains a
little picture or a node. The sizes of the cells are automatically
adjusted such that they are large enough to contain all the cell
contents.

Matrices are a powerful tool and they need to handled with some care.
For impatient readers who skip the rest of this section: you
\emph{must} end \emph{every} row with |\\|. In particular, the last
row \emph{must} be ended with |\\|.

Many of the ideas implemented in \tikzname's matrix support are due to
Mark Wibrow -- many thanks to Mark at this point!



\subsection{Matrices are Nodes}

Matrices are special in many ways, but for most purposes matrices are
treated like nodes. This means, that you use the |node| path command
to create a matrix and you only use a special option, namely the
|matrix| option, to signal that the node will contain a
matrix. Instead of the usual \TeX-box that makes up the |text| part of
the node's shape, the matrix is used. Thus, in particular, a matrix
can have a shape, this shape can be drawn or filled, it can be used in
a tree, and so on. Also, you can refer to the different anchors of a
matrix. 

\begin{itemize}
  \itemoption{matrix}\opt{|=|\meta{true or false}} This option can be
  passed to a |node| path command. It signals that the node will contain
  a matrix. The default parameter is |true| and should usually be
  omitted.
\begin{codeexample}[]
\begin{tikzpicture}
  \draw[help lines] (0,0) grid (4,2);
  \node [matrix,fill=red!20,draw=blue,very thick] (my matrix) at (2,1)
  {
    \draw (0,0)   circle (4mm); & \node[rotate=10] {Hello};        \\
    \draw (0.2,0) circle (2mm); & \fill[red]   (0,0) circle (3mm); \\
  };

  \draw [very thick,->] (0,0) |- (my matrix.west);
\end{tikzpicture}
\end{codeexample}
  The exact syntax of the matrix is explained in the course of this
  section.
  \itemstyle{every matrix}
  This style is used in every matrix. It is empty by default.
\end{itemize}

Even more so than nodes, matrices will often be the only object on a
path. Because of this, there is a special abbreviation for creating matrices:

\begin{command}{\matrix}
  Inside |{tikzpicture}| this is an abbreviation for |\path node[matrix]|.
\end{command}

Even though matrices are nodes, some options do not have the same
effect as for normal nodes:
\begin{enumerate}
\item Rotations and scaling have no effect on a matrix as a whole
  (however, you can still transform the contents of the cells
  normally). Before the matrix is typeset, the rotational and scaling
  part of the transformation matrix is reset.
\item For multi-part shapes you can only set the |text| part of the
  node. 
\item All options starting with |text| such as |text width| have no
  effect.
\end{enumerate}



\subsection{Cell Pictures}

A matrix consists of rows of \emph{cells}. Each row (including the
last one!) is ended by the command |\\|. The character |&| is used
to separate cells. Inside each cell, you must place commands for
drawing a picture, called the \emph{cell picture} in the
following. (However, cell pictures are not enclosed in a complete
|{pgfpicture}| environment, they are a bit more light-weight. The main
difference is that cell pictures cannot have layers.) It is not
necessary to specify beforehand how many rows or columns there are
going to be and if a row contains less cell pictures than another
line, empty cells are automatically added as needed.


\subsubsection{Alignment of Cell Pictures}

For each cell picture a bounding box is computed. These bounding boxes
and the origins of the cell pictures determine how the cells are
aligned. Let us start with the rows: Consider the cell pictures on the first
row. Each has a bounding box and somewhere inside this bounding box
the origin of the cell picture can be found (the origin might even lie
outside the bounding box, but let us ignore this problem for the
moment). The cell pictures are then shifted around such that all
origins lie on the same horizontal line. This may make it necessary to
shift some cell pictures upwards and other downwards, but it can be
done and this yields the vertical alignment of the cell pictures this
row. The top of the row is then given by the top of the ``highest''
cell picture in the row, the bottom of the row is given by the bottom
of the lowest cell picture. (To be more precise, the height of the row
is the maximum $y$-value of any of the bounding boxes and the depth of
the row is the negated minimum $y$-value of the bounding boxes).

\begin{codeexample}[]
\begin{tikzpicture}
  [every node/.style={draw=black,anchor=base,font=\huge}]

  \matrix [draw=red]
  {
    \node {a}; \fill[blue] (0,0) circle (2pt); &
    \node {X}; \fill[blue] (0,0) circle (2pt); &
    \node {g}; \fill[blue] (0,0) circle (2pt); \\
  };
\end{tikzpicture}
\end{codeexample}

Each row is aligned in this fashion: For each row the cell pictures
are vertically aligned such that the origins lie on the same
line. Then the second row is placed below the first row such that the
bottom of the first row touches the top of the second row (unless a
|row sep| is used to add a bit of space). Then the bottom of the
second row touches the top of the third row, and so on. Typically,
each row will have an individual height and depth.

\begin{codeexample}[]
\begin{tikzpicture}
  [every node/.style={draw=black,anchor=base}]

  \matrix [draw=red]
  {
    \node {a}; & \node {X}; & \node {g}; \\
    \node {a}; & \node {X}; & \node {g}; \\
  };

  \matrix [row sep=3mm,draw=red] at (0,-2)
  {
    \node {a}; & \node {X}; & \node {g}; \\
    \node {a}; & \node {X}; & \node {g}; \\
  };
\end{tikzpicture}
\end{codeexample}

Let us now have a look at the columns. The rules for how the pictures
on any given column are aligned are very similar to the row
alignment: Consider all cell pictures in the first column. Each is
shifted horizontally such that the origins lie on the same vertical
line. Then, the left end of the column is at the left end of the
bounding box that protrudes furthest to the left. The right end of the
column is at the right end of the bounding box that protrudes furthest
to the left. This fixes the horizontal alignment of the cell pictures
in the first column and the same happens the cell pictures in the
other columns. Then, the right end of the first column touches the
left end of the second column (unless |column sep| is used). The right
end of the second column touches the left end of the third column, and
so on. (Internally, two columns are actually used to achieve the
desired horizontal alignment, but that is only and implementation
detail.) 

\begin{codeexample}[]
\begin{tikzpicture}[every node/.style={draw}]
  \matrix [draw=red]
  {
    \node[left]  {Hallo}; \fill[blue] (0,0) circle (2pt); \\
    \node        {X};     \fill[blue] (0,0) circle (2pt); \\
    \node[right] {g};     \fill[blue] (0,0) circle (2pt); \\
  };
\end{tikzpicture}
\end{codeexample}

\begin{codeexample}[]
\begin{tikzpicture}[every node/.style={draw}]
  \matrix [draw=red,column sep=1cm]
  {
    \node {8}; & \node{1}; & \node {6}; \\
    \node {3}; & \node{5}; & \node {7}; \\
    \node {4}; & \node{9}; & \node {2}; \\
  };
\end{tikzpicture}
\end{codeexample}



\subsubsection{Setting and Adjusting Column and Row Spacing}

There are different ways of setting and adjusting the spacing between
columns and rows. First, you can use the options |column sep| and
|row sep| to set a default spacing for all rows and all
columns. Second, you can add options to the |&| character and the |\\|
command to adjust the spacing between two specific columns or
rows. Additionally, you can specify whether the space between two
columns or rows should be considered between the origins of cells in
the column or row or between their borders. 

\begin{itemize}
  \itemoption{column sep}|=|\meta{spacing list}
  This option sets a default space that is added between every two
  columns. This space can be positive or negative and is zero by
  default. The \meta{spacing list} normally contains a single
  dimension like |2pt|.
\begin{codeexample}[]
\begin{tikzpicture}
  \matrix [draw,column sep=1cm,nodes=draw]
  {
    \node(a) {123}; & \node (b) {1};   & \node {1}; \\
    \node    {12};  & \node     {12};  & \node {1}; \\
    \node(c) {1};   & \node (d) {123}; & \node {1}; \\
  };
  \draw [red,thick]  (a.east) -- (a.east |- c)
                     (d.west) -- (d.west |- b);
  \draw [<->,red,thick] (a.east) -- (d.west |- b)
    node [above,midway] {1cm};
\end{tikzpicture}
\end{codeexample}
  More generally, the \meta{spacing list} may contain a whole list of
  numbers, separated by commas, and occurrences of the two key words
  |between origins| and |between borders|. The effect of specifying
  such a list is the following: First, all numbers occurring in the
  list are simply added to compute the final spacing. Second,
  concerning the two keywords, the last occurrence of one of the keywords is
  important. If the last occurrence is |between borders| or if neither
  occurs, then the space is inserted between the two columns
  normally. However, if the last occurs is |between origins|, then the 
  following happens: The distance between the columns is adjusted such
  that the difference between the origins of all the cells in the
  first column (remember that they all lie on straight line) and the
  origins of all the cells in the second column is exactly the given
  distance.

  \emph{The |between origins| option can only be used for columns
    mentioned in the first row, that is, you cannot specify this
    option for columns introduced only in later rows.}
  
\begin{codeexample}[]
\begin{tikzpicture}
  \matrix [draw,column sep={1cm,between origins},nodes=draw]
  {
    \node(a) {123}; & \node (b) {1};   & \node {1}; \\
    \node    {12};  & \node     {12};  & \node {1}; \\
    \node    {1};   & \node     {123}; & \node {1}; \\
  };
  \draw [<->,red,thick] (a.center) -- (b.center) node [above,midway] {1cm};
\end{tikzpicture}
\end{codeexample}
  \itemoption{row sep}|=|\meta{spacing list}
  This option works like |column sep|, only for rows. Here, too, you
  can specify whether the space is added between the lower end of the
  first row and the upper end of the second row, or whether the space
  is computed between the origins of the two rows.
\begin{codeexample}[]
\begin{tikzpicture}
  \matrix [draw,row sep=1cm,nodes=draw]
  {
    \node (a) {123}; & \node {1};   & \node {1}; \\
    \node (b) {12};  & \node {12};  & \node {1}; \\
    \node     {1};   & \node {123}; & \node {1}; \\
  };
  \draw [<->,red,thick] (a.south) -- (b.north) node [right,midway] {1cm};
\end{tikzpicture}
\end{codeexample}
\begin{codeexample}[]
\begin{tikzpicture}
  \matrix [draw,row sep={1cm,between origins},nodes=draw]
  {
    \node (a) {123}; & \node {1};   & \node {1}; \\
    \node (b) {12};  & \node {12};  & \node {1}; \\
    \node     {1};   & \node {123}; & \node {1}; \\
  };
  \draw [<->,red,thick] (a.center) -- (b.center) node [right,midway] {1cm};
\end{tikzpicture}
\end{codeexample}
\end{itemize}

The row-end command |\\| allows you to provide an optional
argument, which must be a dimension. This dimension will be added to
the list in |row sep|. This means that, firstly, any numbers you list
in this argument will be added as an extra row separation between the
line being ended and the next line and, secondly, you can use the
keywords |between origins| and |between borders| to locally overrule
the standard setting for this line pair.
\begin{codeexample}[]
\begin{tikzpicture}
  \matrix [row sep=1mm]
  {
    \draw (0,0) circle (2mm); & \draw (0,0) circle (2mm); \\
    \draw (0,0) circle (2mm); & \draw (0,0) circle (2mm); \\[-1mm]
    \draw (0,0) coordinate (a) circle (2mm); &
    \draw (0,0) circle (2mm); \\[1cm,between origins]
    \draw (0,0) coordinate (b) circle (2mm); &
    \draw (0,0) circle (2mm); \\
  };
  \draw [<->,red,thick] (a.center) -- (b.center) node [right,midway] {1cm};
\end{tikzpicture}
\end{codeexample}

The cell separation character |&| also takes an optional
argument, which must also be a spacing list. This spacing list is
added to the |column sep| having a similar effect as the option for
the |\\| command for rows.

This optional spacing list can only be given the first time
a new column is started (usually in the first row), subsequent usages
of this option in later rows have no effect. 
\begin{codeexample}[]
\begin{tikzpicture}
  \matrix [draw,nodes=draw,column sep=1mm]
  {
    \node {8}; &[2mm] \node{1}; &[-1mm] \node {6}; \\
    \node {3}; &      \node{5}; &       \node {7}; \\
    \node {4}; &      \node{9}; &       \node {2}; \\
  };
\end{tikzpicture}
\end{codeexample}
\begin{codeexample}[]
\begin{tikzpicture}
  \matrix [draw,nodes=draw,column sep=1mm]
  {
    \node {8}; &[2mm] \node(a){1}; &[1cm,between origins] \node(b){6}; \\
    \node {3}; &      \node   {5}; &                      \node   {7}; \\
    \node {4}; &      \node   {9}; &                      \node   {2}; \\
  };
  \draw [<->,red,thick] (a.center) -- (b.center) node [above,midway] {11mm};
\end{tikzpicture}
\end{codeexample}
\begin{codeexample}[]
\begin{tikzpicture}
  \matrix [draw,nodes=draw,column sep={1cm,between origins}]
  {
    \node (a) {8}; & \node (b) {1}; &[between borders] \node (c) {6}; \\
    \node     {3}; & \node     {5}; &                  \node     {7}; \\
    \node     {4}; & \node     {9}; &                  \node     {2}; \\
  };
  \draw [<->,red,thick] (a.center) -- (b.center) node [above,midway] {10mm};
  \draw [<->,red,thick] (b.east) -- (c.west) node [above,midway] {10mm};
\end{tikzpicture}
\end{codeexample}




\subsubsection{Cell Styles and Options}

For following style and option are useful for changing the appearance
of the all cell pictures:

\begin{stylekey}{/tikz/every cell=\marg{row}\marg{column} (initially \normalfont empty)}
  This style is installed at the beginning of each cell picture with
  the two parameters being the current \meta{row} and \meta{column} of
  the cell. Note that setting this style to |draw| will \emph{not}
  cause all nodes to be drawn since the |draw| option has to be passed
  to each node individually.

  Inside this style (and inside all cells), the current \meta{row} and
  \meta{column} number are also accessible via the counters
  |\pgfmatrixcurrentrow| and |\pgfmatrixcurrentcolumn|.   
\end{stylekey}

\begin{itemize}
  \itemoption{cells}|=|\meta{options} This option adds the
  \meta{options} to the style |every cell|. It is just a shorthand for
  |every cell/.append style=|\meta{options}.
  \itemoption{nodes}|=|\meta{options} This option adds the
  \meta{options} to the style |every node|. It is just a shorthand for
  |every node/.append style=|\meta{options}.

  The main use of this option is the install some options for the
  nodes \emph{inside} the matrix that should not apply to the matrix
  \emph{itself}. 
\end{itemize}

\begin{codeexample}[]
\begin{tikzpicture}
  \matrix [nodes={fill=blue!20,minimum size=5mm}]
  {
    \node {8}; & \node{1}; & \node {6}; \\
    \node {3}; & \node{5}; & \node {7}; \\
    \node {4}; & \node{9}; & \node {2}; \\
  };
\end{tikzpicture}
\end{codeexample}

The next set of styles can be used to change the appearance of certain
rows, columns, or cells. If more than one of these styles is defined,
they are executed in the below order (the |every cell| style is
executed before all of the below).
\begin{itemize}
  \itemstyle{column \meta{number}}
  This style is used for every cell in column \meta{number}.
  \itemstyle{every odd column}
  This style is used for every cell in an odd column.
  \itemstyle{every even column}
  This style is used for every cell in an even column.
  \itemstyle{row \meta{number}}
  This style is used for every cell in row \meta{number}.
  \itemstyle{every odd row}
  This style is used for every cell in an odd row.
  \itemstyle{every even row}
  This style is used for every cell in an even row.
  \itemstyle{row \meta{row number} column \meta{column number}}
  This style is used for the cell in row \meta{row number} and column
  \meta{column number}.
\end{itemize}


\begin{codeexample}[]
\begin{tikzpicture}
  [row 1/.style={red},
   column 2/.style={green!50!black},
   row 3 column 3/.style={blue}]
    
  \matrix
  {
    \node {8}; & \node{1}; & \node {6}; \\
    \node {3}; & \node{5}; & \node {7}; \\
    \node {4}; & \node{9}; & \node {2}; \\
  };
\end{tikzpicture}
\end{codeexample}

You can use the |column |\meta{number} option to change the alignment
for different columns.

\begin{codeexample}[]
\begin{tikzpicture}
  [column 1/.style={anchor=base west},
   column 2/.style={anchor=base east},
   column 3/.style={anchor=base}]
  \matrix
  {
    \node {123}; & \node{456}; & \node {789}; \\
    \node {12}; & \node{45}; & \node {78}; \\
    \node {1}; & \node{4}; & \node {7}; \\
  };
\end{tikzpicture}
\end{codeexample}


In many matrices all cell pictures have nearly the same code. For
example, cells typically start with |\node{| and end |};|. The
following options allow you to execute such code in all cells:

\begin{itemize}
  \itemoption{execute at begin cell}|=|\meta{code}
  The code will be executed at the beginning of each nonempty cell.
  \itemoption{execute at end cell}|=|\meta{code}
  The code will be executed at the end of each nonempty cell.
  \itemoption{execute at empty cell}|=|\meta{code}
  The code will be executed inside each empty cell.

\begin{codeexample}[]
\begin{tikzpicture}
  [matrix of nodes/.style={
     execute at begin cell=\node\bgroup,
     execute at end cell=\egroup;%
   }]
  \matrix [matrix of nodes]
  {
    8 & 1 & 6 \\
    3 & 5 & 7 \\
    4 & 9 & 2 \\
  };
\end{tikzpicture}
\end{codeexample}
\begin{codeexample}[]
\begin{tikzpicture}
  [matrix of nodes/.style={
     execute at begin cell=\node\bgroup,
     execute at end cell=\egroup;,%
     execute at empty cell=\node{--};%
   }]
  \matrix [matrix of nodes]
  {
    8 & 1 &   \\
    3 &   & 7 \\
      &   & 2 \\
  };
\end{tikzpicture}
\end{codeexample}
\end{itemize}

The |matrix| library defines a number of styles that make use of the
above options.





\subsection{Anchoring a Matrix}

Since matrices are nodes, they can be anchored in the usual fashion
using the |anchor| option. However, there are two ways to influence
this placement further. First, the following option is often useful:

\begin{itemize}
  \itemoption{matrix anchor}|=|\meta{anchor}
  This option has the same effect as |anchor|, but the option applies
  only to the matrix itself, not to the cells inside. If you just say
  |anchor=north| as an option to the matrix node, all nodes inside
  matrix will also have this anchor, unless it is explicitly set
  differently for each node. By comparison, |matrix anchor| sets the
  anchor for the matrix, but for the nodes inside the value of
  |anchor| remain unchanged.

\begin{codeexample}[]
\begin{tikzpicture}
  \matrix [matrix anchor=west] at (0,0)
  {
    \node {123}; \\ % still center anchor
    \node {12}; \\
    \node {1}; \\
  };
  \matrix [anchor=west] at (0,-2)
  {
    \node {123}; \\ % inherited west anchor
    \node {12}; \\
    \node {1}; \\
  };
\end{tikzpicture}
\end{codeexample}
\end{itemize}

The second way to anchor a matrix is to use \emph{an anchor of a node
  inside the matrix}. For this, the |anchor| option has a special
effect when given as an argument to a matrix:

\begin{itemize}
  \itemoption{anchor}|=|\meta{anchor or node.anchor}
  Normally, the argument of this option refers to anchor of the matrix
  node, which is the node than includes all of the stuff of the
  matrix. However, you can also provide an argument of the form
  \meta{node}|.|\meta{anchor} where \meta{node} must be node defined
  inside the matrix and \meta{anchor} is an anchor of this node. In
  this case, the whole matrix is shifted around in such a way that
  this particular anchor of this particular node lies at the |at|
  position of the matrix. The same is true for |matrix anchor|.
\end{itemize}

\begin{codeexample}[]
\begin{tikzpicture}
  \draw[help lines] (0,0) grid (3,2);
  \matrix[matrix anchor=inner node.south,anchor=base,row sep=3mm] at (1,1)
  {
    \node {a}; & \node             {b}; & \node {c}; & \node {d}; \\
    \node {a}; & \node(inner node) {b}; & \node {c}; & \node {d}; \\
    \node {a}; & \node             {b}; & \node {c}; & \node {d}; \\
  };
  \draw (inner node.south) circle (1pt);
\end{tikzpicture}
\end{codeexample}



\subsection{Considerations Concerning Active Characters}

Even though \tikzname\ seems to use |&| to separate cells, \pgfname\ actually
uses a different command to separate cells, namely the command
|\pgfmatrixnextcell| and using a normal |&| character will normally
fail. What happens is that, \tikzname\ makes |&| an active character
and then defines this character to be equal to
|\pgfmatrixnextcell|. In most situations this will work 
nicely, but sometimes |&| cannot be made active; for
instance because the matrix is used in an argument of some macro or
the matrix contains nodes that contain normal |{tabular}|
environments. In this case you can use the following option to avoid
having to type |\pgfmatrixnextcell| each time:

\begin{itemize}
  \itemoption{ampersand replacement}|=|\meta{macro name or empty}
  If a macro name is provided, this macro will be defined to be equal
  to |\pgfmatrixnextcell| inside matrices and |&| will not be made
  active. For instance, you could say |ampersand replacement=\&| and
  then use \& to separate columns as in the following example:
\begin{codeexample}[]
\tikz
  \matrix [ampersand replacement=\&]
  {
    \draw (0,0)   circle (4mm); \& \node[rotate=10] {Hello};        \\
    \draw (0.2,0) circle (2mm); \& \fill[red]   (0,0) circle (3mm); \\
  };
\end{codeexample}
\end{itemize}


\subsection{Examples}

The following examples are adapted from code by Mark Wibrow. The first
two redraw pictures from Timothy van Zandt's PSTricks documentation: 

{\catcode`\|=12
\begin{codeexample}[]
\begin{tikzpicture} 
  \matrix [matrix of math nodes,row sep=1cm]
  { 
    |(U)| U &[2mm]                       &[8mm]    \\ 
            &      |(XZY)| X \times_Z Y  &      |(X)| X \\ 
            &      |(Y)|   Y             &      |(Z)| Z \\
  }; 
  \begin{scope}[every node/.style={midway,auto,font=\scriptsize}] 
    \draw [double, dashed] (U)   -- node {$x$} (X); 
    \draw                  (X)   -- node {$p$} (X -| XZY.east)
                           (X)   -- node {$f$} (Z)
                                 -- node {$g$} (Y)
                                 -- node {$q$} (XZY)
                                 -- node {$y$} (U);
   \end{scope}
\end{tikzpicture} 
\end{codeexample}

\begin{codeexample}[]
\begin{tikzpicture}[>=stealth,->,shorten >=2pt,looseness=.5,auto] 
  \matrix [matrix of math nodes,
           column sep={2cm,between origins},
           row sep={3cm,between origins},
           nodes={circle, draw, minimum size=7.5mm}]
  { 
            & |(A)| A &         \\ 
    |(B)| B & |(E)| E & |(C)| C \\ 
            & |(D)| D           \\
  }; 
  \begin{scope}[every node/.style={font=\small\itshape}]
    \draw (A) to [bend left] (B) node [midway]   {g}; 
    \draw (B) to [bend left] (A) node [midway]   {f}; 
    \draw (D) --             (B) node [midway]   {c}; 
    \draw (E) --             (B) node [midway]   {b}; 
    \draw (E) --             (C) node [near end] {a}; 
    \draw [-,line width=8pt,draw=graphicbackground] 
          (D) to [bend right, looseness=1] (A); 
    \draw (D) to [bend right, looseness=1] (A)
            node [near start] {b} node [near end]   {e}; 
  \end{scope}
\end{tikzpicture}
\end{codeexample}

\begin{codeexample}[]
\begin{tikzpicture} 
  \matrix (network)
    [matrix of nodes,%
     nodes in empty cells,
     nodes={outer sep=0pt,circle,minimum size=4pt,draw},
     column sep={1cm,between origins},
     row sep={1cm,between origins}]
  {
                  &                &                 & \\ 
                  &                &                 & \\ 
    |[draw=none]| & |[xshift=1mm]| & |[xshift=-1mm]|   \\
  }; 
  \foreach \a in {1,...,4}{ 
    \draw (network-3-2) -- (network-2-\a); 
    \draw (network-3-3) -- (network-2-\a); 
    \draw [-stealth] ([yshift=5mm]network-1-\a.north) -- (network-1-\a); 
    \foreach \b in {1,...,4} 
      \draw (network-1-\a) -- (network-2-\b); 
  } 
  \draw [stealth-] ([yshift=-5mm]network-3-2.south) -- (network-3-2); 
  \draw [stealth-] ([yshift=-5mm]network-3-3.south) -- (network-3-3); 
\end{tikzpicture} 
\end{codeexample}

The following example is adapted from code written by Kjell Magne
Fauske, which is based on the following paper: K.~Bossley, M.~Brown, 
and C.~Harris, Neurofuzzy identification of an autonomous underwater
vehicle, \emph{International Journal of Systems Science}, 1999, 30, 901--913.

\begin{codeexample}[]
\begin{tikzpicture}
  [auto, 
   decision/.style={diamond, draw=blue, thick, fill=blue!20, 
                    text width=4.5em, text badly centered,
                    inner sep=1pt}, 
   block/.style   ={rectangle, draw=blue, thick, fill=blue!20, 
                    text width=5em, text centered, rounded corners,
                    minimum height=4em},
   line/.style    ={draw, thick, -latex',shorten >=2pt},
   cloud/.style   ={draw=red, thick, ellipse,fill=red!20,
                    minimum height=2em}]
  
  \matrix [column sep=5mm,row sep=7mm]
  {
    % row 1
      \node [cloud] (expert)   {expert}; & 
      \node [block] (init)     {initialize model}; & 
      \node [cloud] (system)   {system}; \\
    % row 2
      & \node [block] (identify) {identify candidate model}; & \\ 
    % row 3
      \node [block] (update)   {update model};  & 
      \node [block] (evaluate) {evaluate candidate models}; & \\ 
    % row 4
      & \node [decision] (decide) {is best candidate}; & \\ 
    % row 5
      & \node [block] (stop)      {stop}; & \\
  }; 
  \begin{scope}[every path/.style=line]
    \path          (init)     -- (identify); 
    \path          (identify) -- (evaluate); 
    \path          (evaluate) -- (decide); 
    \path          (update)   |- (identify); 
    \path          (decide)   -| node [near start] {yes} (update); 
    \path          (decide)   -- node [midway] {no} (stop); 
    \path [dashed] (expert)   -- (init); 
    \path [dashed] (system)   -- (init); 
    \path [dashed] (system)   |- (evaluate); 
  \end{scope}
\end{tikzpicture} 
\end{codeexample}
}

%%% Local Variables: 
%%% mode: latex
%%% TeX-master: "pgfmanual"
%%% End: 

% Copyright 2005 by Till Tantau <tantau@cs.tu-berlin.de>.
%
% This program can be redistributed and/or modified under the terms
% of the LaTeX Project Public License Distributed from CTAN
% archives in directory macros/latex/base/lppl.txt.


\section{Making Trees Grow}

\label{section-trees}


\subsection{Introduction to the  Child Operation}

\emph{Trees} are a common way of visualizing hierarchical
structures. A simple tree looks like this:
\begin{codeexample}[]
\begin{tikzpicture}
  \node {root}
    child {node {left}}
    child {node {right}
      child {node {child}}
      child {node {child}}
    };
\end{tikzpicture}
\end{codeexample}

Admittedly, in reality trees are more likely to grow \emph{upward} and
not downward as above. You can tell whether the author of a paper is a
mathematician or a computer scientist by looking at the direction
their trees grow. A computer scientist's trees will grow downward
while a mathematician's tree will grow upward. Naturally, the
\emph{correct} way is the mathematician's way, which can be specify as
follows: 
\begin{codeexample}[]
\begin{tikzpicture}
  \node {root} [grow'=up]
    child {node {left}}
    child {node {right}
      child {node {child}}
      child {node {child}}
    };
\end{tikzpicture}
\end{codeexample}

In \tikzname, trees are specified by adding \emph{children} to a
node on a path. The syntax for the child operation is the following:

\begin{pathoperation}{child}{\opt{\oarg{options}}\opt{\marg{child path}}}
  This operation should directly follow a completed |node| operation
  or another |child| operation, although it is permissible that the
  first |child| operation is preceded by options (we will come to
  that).

  When a |node| operation like |node {X}| is followed by |child|,
  \tikzname\ starts counting the number of child nodes that follow the
  original |node {X}|. For this, it scans the input and stores away each
  |child| and its arguments until it reaches a path operation that is
  not a |child|. Note that this will fix the character codes of all
  text inside the child arguments, which means, in essence, that you
  cannot use verbatim text inside the nodes inside a |child|. Sorry. 

  Once the children have been collected and counted, \tikzname\ starts
  generating the child nodes. For each child of a parent node
  \tikzname\ computes an appropriate position where the child is
  placed. For each child, the coordinate system is transformed so that
  the origin is at this position. Then the \meta{child path} is
  drawn. Typically, the child path just consists of a |node|
  specification, which results in a node being drawn at the child's
  position. Finally, an edge is drawn from the first node in the
  \meta{child path} to the parent node.
  
  The details and options for this operation are described in the rest
  of this present section.
\end{pathoperation}


\subsection{Child Paths and the Child Nodes}

For each |child| of a root node, its \meta{child path} is inserted at
a specific location in the picture (the placement rules are discussed
in Section~\ref{section-tree-placement}). The first node in the
\meta{child path}, if it exists, is special and called the \emph{child
  node}. If there is no first node in the \meta{child path}, that is,
if the \meta{child path} is missing (including the curly braces) or if
it does not start with |node| or with |coordinate|, then an empty
child node of shape |coordinate| is automatically added.

Consider the example |\node {x} child {node {y}} child;|. For the
first child, the \meta{child path} has the child node |node {y}|. For
the second child, no child node is specified and, thus, it is just
|coordinate|.

As for any normal node, you can give the child node a name, shift it 
around, or use options to influence how it is rendered.
\begin{codeexample}[]
\begin{tikzpicture}
  \node[rectangle,draw] {root}
    child {node[circle,draw] (left node) {left}}
    child {node[ellipse,draw] (right node) {right}};
  \draw[dashed,->] (left node) -- (right node);
\end{tikzpicture}
\end{codeexample}

In many cases, the \meta{child path} will just consist of a
specification of a child node and, possibly, children of this child
node. However, the node specification may be followed by arbitrary
other material that will be added to the picture, transformed to the
child's coordinate system. For your convenience, a move-to |(0,0)|
operation is inserted automatically at the beginning of the path. Here
is an example: 

\begin{codeexample}[]
\begin{tikzpicture}
  \node {root}
    child {[fill] circle (2pt)}
    child {[fill] circle (2pt)};
\end{tikzpicture}    
\end{codeexample}


At the end of the \meta{child path} you may add a special path
operation called |edge from parent|. If this operation is not given by
yourself somewhere on the path, it will be automatically added at the
end. This option causes a connecting edge from the parent node to the
child node to be added to the path. By giving options to this
operation you can influence how the edge is rendered. Also, nodes
following the |edge from parent| operation will be placed on this
edge, see Section~\ref{section-edge-from-parent} for details.

To sum up:
\begin{enumerate}
\item
  The child path starts with a node specification. If it is not there,
  it is added automatically.
\item
  The child path ends with a |edge from parent| operation, possibly
  followed by nodes to be put on this edge. If the operation is not
  given at the end, it is added automatically.
\end{enumerate}



\subsection{Naming Child Nodes}

Child nodes can be named like any other node using either the |name|
option or the special syntax in which the name of the node is placed
in round parentheses between the |node| operation and the node's
text.

If you do not assign a name to a child node, \tikzname\ will
automatically assign a name as follows: Assume that the name of the
parent node is, say, |parent|. (If you did not assign a
name to the parent, \tikzname\ will do so itself, but that name will
not be user-accessible.) The first child
of |parent| will be named |parent-1|, the second child is named
|parent-2|, and so on.

This naming convention works recursively. If the second child
|parent-2| has children, then the first of these children will be
called |parent-2-1| and the second |parent-2-2| and so on.

If you assign a name to a child node yourself, no name is generated
automatically (the node does not have two names). However, ``counting
continues,'' which means that the third child of |parent| is called
|parent-3| independently of whether you have assigned names to the
first and/or second child of |parent|.

Here is an example:

\begin{codeexample}[]
\begin{tikzpicture}
  \node (root) {root}
    child 
    child {
      child {coordinate (special)}
      child
    };
  \node at (root-1) {root-1};
  \node at (root-2) {root-2};
  \node at (special) {special};
  \node at (root-2-2) {root-2-2};
\end{tikzpicture}
\end{codeexample}

\subsection{Specifying Options for Trees and Children}

Each |child| may have its own \meta{options}, which apply to ``the
whole child,'' including all of its grandchildren. Here is an
example:

\begin{codeexample}[]
\begin{tikzpicture}[thick]
  \tikzstyle{level 2}=[sibling distance=10mm]
  \coordinate
    child[red]   {child child}
    child[green] {child child[blue]};
\end{tikzpicture}
\end{codeexample}

The options of the root node have no effect on the children since
the options of a node are always ``local'' to that node. Because of
this, the edges in the following tree are black, not red.
  
\begin{codeexample}[]
\begin{tikzpicture}[thick]
  \node [red] {root}
    child
    child;
\end{tikzpicture}
\end{codeexample}
  This raises the problem of how to set options for \emph{all}
  children. Naturally, you could always set options for the whole path
  as in |\path [red] node {root} child child;| but this is bothersome
  in some situations. Instead, it is easier to give the options
  \emph{before the first child} as follows:
\begin{codeexample}[]
\begin{tikzpicture}[thick]
  \node [red] {root}
    [green] % option applies to all children
    child
    child;
\end{tikzpicture}
\end{codeexample}

Here is the set of rules:
\begin{enumerate}
\item
  Options for the whole tree are given before the root node.
\item
  Options for the root node are given directly to the |node| operation
  of the root.
\item
  Options for all children can be given between the root node and the
  first child.
\item
  Options applying to a specific child path are given as options to
  the |child| operation.
\item
  Options applying to the node of a child, but not to the whole child
  path, are given as options to the |node| command inside the
  \meta{child path}.
\end{enumerate}

\begin{codeexample}[code only]
\begin{tikzpicture}
  \path
    [...]             % Options apply to the whole tree
    node[...] {root}  % Options apply to the root node only
      [...]           % Options apply to all children
      child[...]      % Options apply to this child and all its children
      {
        node[...] {}  % Options apply to the child node only
        ...
      }
      child[...]      % Options apply to this child and all its children
    ;
\end{tikzpicture}
\end{codeexample}

There are additional styles that influence how children are rendered:
\begin{itemize}
  \itemstyle{every child}
  This style is used at the beginning of each child, as if you had
  given the options to the |child| operation.
  \itemstyle{every child node}
  This style is used at the beginning of each child node in addition
  to the |every node| style.
  \itemstyle{level \meta{number}}
  This style is used at the beginning of each set of children, where
  \meta{number} is the current level in the current tree. For example,
  when you say |\node {x} child child;|, then the style |level 1| is
  used before the first |child|. If this first |child| has children
  itself, then |level 2| would be used for them.

\begin{codeexample}[]
\begin{tikzpicture}
  \tikzstyle{level 1}=[sibling distance=20mm]
  \tikzstyle{level 2}=[sibling distance=5mm]
  \node {root}
    child { child child }
    child { child child child };
\end{tikzpicture}
\end{codeexample}
\end{itemize}




\subsection{Placing Child Nodes}

\label{section-tree-placement}

Perhaps the most difficult part in drawing a tree is the correct
layout of the children. Typically, the children have different sizes
and it is not easy to arrange them in such a manner that not too much
space is wasted, the children do not overlap, and they are either 
evenly spaced or their centers are evenly distributed. Calculating
good positions is especially difficult since a good position for the
first child may depend on the size of the last child.

In \tikzname, a comparatively simple approach is taken to placing the
children. In order to compute a child's position, all that is taken
into account is the number of the current child in the list of
children and the number of children in this list. Thus, if a node has
five children, then there is a fixed position for the first child, a
position for the second child, and so on. These positions \emph{do not
  depend on the size of the children} and, hence, children can easily
overlap. However, since you can use options to shift individual
children a bit, this is not as great a problem as it may seem.

Although the placement of the children only depends on their number in
the list of children and the total number of children, everything else
about the placement is highly configurable. You can change the
distance between children (appropriately called the
|sibling distance|) and the distance between levels of the tree. These
distances may change from level to level. The direction in which the
tree grows can be changed globally and for parts of the tree. You can
even specify your own ``growth function'' to arrange children on a
circle or along special lines or curves. 

The default growth function works as follows: Assume that we are given
a node and five children. These children will be placed on a line with
their centers (or, more generally, with their anchors) spaced apart by
the current |sibling distance|. The line is 
orthogonal to the current \emph{direction of growth}, which is set
with the |grow| and |grow'| option (the latter option reverses the
ordering of the children). The distance from the line to the parent node
is given by the |level distance|.

{\catcode`\|=12
\begin{codeexample}[]
\begin{tikzpicture}
  \path [help lines]
    node (root) {root}
    [grow=-10]
    child {node {1}}
    child {node {2}}
    child {node {3}}
    child {node {4}};

  \draw[|<->|,thick] (root-1.center)
    -- node[above,sloped] {sibling distance} (root-2.center);

  \draw[|<->|,thick] (root.center) 
    -- node[above,sloped] {level distance} +(-10:\tikzleveldistance);
\end{tikzpicture}
\end{codeexample}
}

Here is a detailed description of the options:
\begin{itemize}
  \itemoption{level distance}|=|\meta{distance}
  This option allows you to change the distance between different
  levels of the tree, more precisely, between the parent and the line
  on which its children are arranged. When given to a single child,
  this will set the distance for this child only.
 
\begin{codeexample}[]
\begin{tikzpicture}
  \node {root}
    [level distance=20mm]
    child
    child {
      [level distance=5mm]
      child
      child
      child
    }
    child[level distance=10mm];  
\end{tikzpicture}
\end{codeexample}
 
\begin{codeexample}[]
\begin{tikzpicture}
  \tikzstyle{level 1}=[level distance=10mm]    
  \tikzstyle{level 2}=[level distance=5mm]    
  \node {root}
    child
    child {
      child
      child[level distance=10mm]
      child
    }
    child;
\end{tikzpicture}
\end{codeexample}

  \itemoption{sibling distance}|=|\meta{distance}
  This option specifies the distance between the anchors of the
  children of a parent node.   

\begin{codeexample}[]
\begin{tikzpicture}[level distance=4mm]
  \tikzstyle{level 1}=[sibling distance=8mm]
  \tikzstyle{level 2}=[sibling distance=4mm]
  \tikzstyle{level 3}=[sibling distance=2mm]
  \coordinate
     child {
       child {child child}
       child {child child}
     }
     child {
       child {child child}
       child {child child}
     };
\end{tikzpicture}
\end{codeexample}

\begin{codeexample}[]
\begin{tikzpicture}[level distance=10mm]
  \tikzstyle{every node}=[fill=red!60,circle,inner sep=1pt]
  \tikzstyle{level 1}=[sibling distance=20mm,
    set style={{every node}+=[fill=red!45]}]
  \tikzstyle{level 2}=[sibling distance=10mm,
    set style={{every node}+=[fill=red!30]}]
  \tikzstyle{level 3}=[sibling distance=5mm,
    set style={{every node}+=[fill=red!15]}]
  \node {31}
     child {node {30}
       child {node {20}
         child {node {5}}
         child {node {4}}
       }
       child {node {10}
         child {node {9}}
         child {node {1}}
       }
     }
     child {node {20}
       child {node {19}
         child {node {1}}
         child[fill=none] {edge from parent[draw=none]}
       }
       child {node {18}}
     };
\end{tikzpicture}
\end{codeexample}
  
  \itemoption{grow}|=|\meta{direction}
  This option is used to define the \meta{direction} in which the tree
  will grow. The \meta{direction} can either be an angle in degrees or
  one of the following special text strings: |down|, |up|, |left|,
  |right|, |north|, |south|, |east|, |west|, |north east|,
  |north west|, |south east|, and |south west|. All of these have
  ``their obvious meaning,'' so, say, |south west| is the same as the
  angle $-135^\circ$.

  As a side effect, this option installs the default growth function.

  In addition to setting the direction, this option also has a
  seemingly strange effect: It sets the sibling distance for the
  current level to |0pt|, but leaves the sibling distance for later
  levels unchanged.

  This somewhat strange behaviour has a highly desirable effect: If
  you give this option before the list of children of a node starts,
  the ``current level'' is still the parent level. Each child will be
  on a later level and, hence, the sibling distance will be as
  specified originally. This will cause the children to be neatly
  aligned in a line orthogonal to the given \meta{direction}. However,
  if you give this option locally to a single child, then ``current
  level'' will be the same as the child's level. The zero sibling
  distance will then cause the child to be placed exactly at a point
  at distance |level distance| in the direction
  \meta{direction}. However, the children of the child will be placed
  ``normally'' on a line orthogonal to the \meta{direction}.

  These placement effects are best demonstrated by some examples:
\begin{codeexample}[]
\tikz \node {root} [grow=right] child child;
\end{codeexample}

\begin{codeexample}[]
\tikz \node {root} [grow=south west] child child;
\end{codeexample}

\begin{codeexample}[]
\begin{tikzpicture}[level distance=10mm,sibling distance=5mm]
  \node {root}
    [grow=down]
    child
    child
    child[grow=right] {
      child child child
    };  
\end{tikzpicture}
\end{codeexample}

\begin{codeexample}[]
\begin{tikzpicture}[level distance=2em]
  \node {C}
    child[grow=up]    {node {H}}
    child[grow=left]  {node {H}}
    child[grow=down]  {node {H}}
    child[grow=right] {node {C}
        child[grow=up]    {node {H}}
        child[grow=right] {node {H}}
        child[grow=down]  {node {H}}
      edge from parent[double]
        coordinate (wrong)
    };
  \draw[<-,red] ([yshift=-2mm]wrong) -- +(0,-1)
    node[below]{This is wrong!};  
\end{tikzpicture}
\end{codeexample}

\begin{codeexample}[]
\begin{tikzpicture}
  \node[rectangle,draw] (a) at (0,0) {start node};
  \node[rectangle,draw] (b) at (2,1) {end};

  \draw (a) -- (b)
    node[coordinate,midway] {}
      child[grow=100,<-] {node[above] {the middle is here}};
\end{tikzpicture}
\end{codeexample}

  \itemoption{grow'}|=|\meta{direction}
  This option has the same effect as |grow|, only the children are
  arranged in the opposite order.
  \itemoption{growth function}|=|\meta{macro name}
  This rather low-level option allows you to set a new growth
  function. The \meta{macro name} must be the name of a macro without
  parameters. This macro will be called for each child of a node.

  The effect of executing the macro should be the following: It should
  transform the coordinate system in such a way that the origin
  becomes the place where the current child should be anchored. When
  the macro is called, the current coordinate system will be setup
  such that the anchor of the parent node is in the origin. Thus, in
  each call, the \meta{macro name} must essentially do a shift to the
  child's origin. When the macro is called, the \TeX\ counter
  |\tikznumberofchildren| will be set to the total number of children
  of the parent node and the counter |\tikznumberofcurrentchild| will
  be set to the number of the current child.

  The macro may, in addition to shifting the coordinate system, also
  transform the coordinate system further. For example, it could be
  rotated or scaled.

  Additional growth functions are defined in the library, see 
  Section~\ref{section-tree-library}.
\end{itemize}



\subsection{Edges From the Parent Node}

\label{section-edge-from-parent}

Every child node is connected to its parent node via a special kind of
edge called the |edge from parent|. This edge is added to the
\meta{child path} when the following path operation is encountered:

\begin{pathoperation}{edge from parent}{\opt{\oarg{options}}}
  This path operation can only be used inside \meta{child paths} and
  should be given at the end, possibly followed by node specifications
  (we will come to that). If a \meta{child path} does not contain this
  operation, it will be added at the end of the \meta{child path}
  automatically.

  This operation has several effects. The most important is that it
  inserts the current ``edge from parent path'' into the child
  path. The edge from parent path can be set using the following
  option:
  \begin{itemize}
    \itemoption{edge from parent path}|=|\meta{path}
    This options allows you to set the edge from parent path to a new
    path. The default for this path is the following:
    \begin{codeexample}[code only]
(\tikzparentnode\tikzparentanchor) -- (\tikzchildnode\tikzchildanchor)      
    \end{codeexample}
    The |\tikzparentnode| is a macro that will expand to the name of
    the parent node. This works even when you have not assigned a name
    to the parent node, in this case an internal name is automatically
    generated. The |\tikzchildnode| is a macro that expands to the
    name of the child node. The two |...anchor| macros are empty by
    default. So, what is essentially inserted is just the path segment
    |(\tikzparentnode) -- (\tikzchildnode)|; which is exactly an edge
    from the parent to the child.

    You can modify this edge from parent path to achieve all sorts of
    effects. For example, we could replace the straight line by a
    curve as follows:
\begin{codeexample}[]
\begin{tikzpicture}[edge from parent path=
  {(\tikzparentnode.south) .. controls +(0,-1) and +(0,1)
                           .. (\tikzchildnode.north)}]
  \node {root}
    child {node {left}}
    child {node {right}
      child {node {child}}
      child {node {child}}
    };
\end{tikzpicture}
\end{codeexample}

    Further useful edge from parent paths are defined in the tree
    library, see Section~\ref{section-tree-library}.

    As said before, the anchors in the default edge from parent path
    are empty. However, you can set them using the following options:
    \begin{itemize}
      \itemoption{child anchor}|=|\meta{anchor}
      Specifies the anchor where the edge from parent meets the child
      node by setting the macro |\tikzchildanchor| to
      |.|\meta{anchor}.

      If you specify |border| as the \meta{anchor}, then the macro
      |\tikzchildanchor| is set to the empty string. The effect of
      this is that the edge from the parent will meet the child on the
      border at an automatically calculated position.
\begin{codeexample}[]
\begin{tikzpicture}
  \node {root}
    [child anchor=north]
    child {node {left} edge from parent[dashed]}
    child {node {right}
      child {node {child}}
      child {node {child} edge from parent[draw=none]}
    };
\end{tikzpicture}
\end{codeexample}
      \itemoption{parent anchor}|=|\meta{anchor}
      This option works the same way as the |child anchor|, only for
      the parent.
    \end{itemize}
  \end{itemize}

  Besides inserting the edge from parent path, the |edge from parent|
  operation has another effect: The \meta{options} are inserted
  directly before the edge from parent path and the following style is
  also installed prior to inserting the path:
  \begin{itemize}
    \itemstyle{edge from parent}
    This style is inserted right before the edge from parent path and
    before the \meta{options} are inserted. By default, it just draws
    the edge from parent, but you can use it to make the edge look
    different. 
\begin{codeexample}[]
\begin{tikzpicture}
  \tikzstyle{edge from parent}=[draw,red,thick]    
  \node {root}
    child {node {left} edge from parent[dashed]}
    child {node {right}
      child {node {child}}
      child {node {child} edge from parent[draw=none]}
    };
\end{tikzpicture}
\end{codeexample}
  \end{itemize}

  Note: The \meta{options} inserted before the edge from parent path
  is added \emph{apply to the whole child path}. Thus, it is not
  possible to, say, draw a circle in red as part of the child path and
  then have an edge to parent in blue. However, as always, the child
  node is a node and can be drawn in a totally different way.

  Finally, the |edge from parent| operation has one more effect: It
  causes all nodes \emph{following} the operation to be placed on the
  edge. This is the same effect as if you had added the |pos| option
  to all these nodes, see also Section~\ref{section-pos-option}.

  As an example, consider the following code:
\begin{codeexample}[code only]
\node (root) {} child {node (child) {} edge to parent node {label}};    
\end{codeexample}
  The |edge to parent| operation and the following |node| operation
  will, together, have the same effect as if we had said:
\begin{codeexample}[code only]
(root) -- (child) node [pos=0.5] {label}
\end{codeexample}

  Here is a more complicated example:
\begin{codeexample}[]
\begin{tikzpicture}
  \node {root}
    child {
      node {left}
      edge from parent
        node[left] {a}
        node[right] {b}
    }
    child {
      node {right}
        child {
          node {child}
          edge from parent
            node[left] {c}
        }
        child {node {child}}
      edge from parent
        node[near end] {x}
    };
\end{tikzpicture}
\end{codeexample}

\end{pathoperation}



%%% Local Variables: 
%%% mode: latex
%%% TeX-master: "pgfmanual-pdftex-version"
%%% End: 

% Copyright 2007 by Till Tantau
%
% This file may be distributed and/or modified
%
% 1. under the LaTeX Project Public License and/or
% 2. under the GNU Free Documentation License.
%
% See the file doc/generic/pgf/licenses/LICENSE for more details.


\section{Plots of Functions}

\label{section-tikz-plots}

\subsection{When Should One Use \tikzname\ for Generating Plots? }

\label{section-why-pgname-for-plots}

There exist many powerful programs that produce plots, examples are
\textsc{gnuplot} or \textsc{mathematica}. These programs can produce
two different kinds of output: First, they can output a complete plot
picture in a certain format (like \pdf) that includes all low-level
commands necessary for drawing the complete plot (including axes and
labels). Second, they can usually also produce ``just plain data'' in
the form of a long list of coordinates. Most of the powerful programs
consider it a to be ``a bit boring'' to just output tabled data and
very much prefer to produce fancy pictures. Nevertheless, when coaxed,
they can also provide the plain data.

\emph{Note that is often not necessary to use \tikzname\ for plots.}
Programs like \textsc{gnuplot} can produce very sophisticated plots
and it is usually much easier to simply include these plots as a
finished \textsc{pdf} or PostScript graphics.

However, there are a number of reasons why you may wish to invest time
and energy into mastering the \pgfname\ commands for creating plots:

\begin{itemize}
\item
  Virtually all plots produced by ``external programs'' use different
  fonts from the one used in your document.
\item
  Even worse, formulas will look totally different, if they can be
  rendered at all.
\item
  Line width will usually be too large or too small.
\item
  Scaling effects upon inclusion can create a mismatch between sizes
  in the plot and sizes in the text.
\item
  The automatic grid generated by most programs is mostly
  distracting. 
\item
  The automatic ticks generated by most programs are cryptic
  numerics. (Try adding a tick reading ``$\pi$'' at the right point.)
\item
  Most programs make it very easy to create ``chart junk'' in a most
  convenient fashion.  All show, no content.
\item
  Arrows and plot marks will almost never match the arrows used in the
  rest of the document.
\end{itemize}

The above list is not exhaustive, unfortunately.


\subsection{The Plot Path Operation}

The |plot| path operation can be used to append a line or curve to the path
that goes through a large number of coordinates. These coordinates are
either given in a simple list of coordinates, read from some file, or
they are computed on the fly.

The syntax of the |plot| comes in different versions.

\begin{pathoperation}{--plot}{\meta{further arguments}}
  This operation plots the curve through the coordinates specified in
  the \meta{further arguments}. The current (sub)path is simply
  continued, that is, a line-to operation to the first point of the
  curve is implicitly added. The details of the \meta{further
    arguments}  will be explained in a moment.
\end{pathoperation}

\begin{pathoperation}{plot}{\meta{further arguments}}
  This operation plots the curve through the coordinates specified in
  the \meta{further arguments} by first ``moving'' to the first
  coordinate of the curve.
\end{pathoperation}

The \meta{further arguments} are used in three different ways to
specifying the coordinates of the points to be plotted:

\begin{enumerate}
\item
  \opt{|--|}|plot|\oarg{local options}\declare{|coordinates{|\meta{coordinate
    1}\meta{coordinate 2}\dots\meta{coordinate $n$}|}|}
\item
  \opt{|--|}|plot|\oarg{local options}\declare{|file{|\meta{filename}|}|}
\item
  \opt{|--|}|plot|\oarg{local options}\declare{\meta{coordinate expression}}
\item
  \opt{|--|}|plot|\oarg{local options}\declare{|function{|\meta{gnuplot formula}|}|}
\end{enumerate}

These different ways are explained in the following.


\subsection{Plotting Points Given Inline}

In the first two cases, the points are given directly in the \TeX-file
as in the following example:

\begin{codeexample}[]
\tikz \draw plot coordinates {(0,0) (1,1) (2,0) (3,1) (2,1) (10:2cm)};
\end{codeexample}

Here is an example showing the difference between |plot| and |--plot|:

\begin{codeexample}[]
\begin{tikzpicture}
  \draw (0,0) -- (1,1) plot coordinates {(2,0)  (4,0)};
  \draw[color=red,xshift=5cm]
        (0,0) -- (1,1) -- plot coordinates {(2,0)  (4,0)};
\end{tikzpicture}
\end{codeexample}


\subsection{Plotting Points Read From an External File}

The second way of specifying points is to put them in an external
file named \meta{filename}. Currently, the only file format that
\tikzname\ allows is the following: Each line of the \meta{filename}
should contain one line starting with two numbers, separated by a
space. Anything following the two numbers on the line is
ignored. Also, lines starting with a |%| or a |#| are ignored as well
as empty lines. (This is exactly the format that \textsc{gnuplot}
produces when you say |set terminal table|.) If necessary, more
formats will be supported in the future, but it is usually easy to
produce a file containing data in this form.

\begin{codeexample}[]
\tikz \draw plot[mark=x,smooth] file {plots/pgfmanual-sine.table};
\end{codeexample}

The file |plots/pgfmanual-sine.table| reads:
\begin{codeexample}[code only]
#Curve 0, 20 points
#x y type
0.00000 0.00000  i
0.52632 0.50235  i
1.05263 0.86873  i
1.57895 0.99997  i
...
9.47368 -0.04889  i
10.00000 -0.54402  i
\end{codeexample}
It was produced from the following source, using |gnuplot|:
\begin{codeexample}[code only]
set terminal table
set output "../plots/pgfmanual-sine.table"
set format "%.5f"
set samples 20
plot [x=0:10] sin(x)
\end{codeexample}

The \meta{local options} of the |plot| operation are local to each
plot and do not affect other plots ``on the same path.'' For example,
|plot[yshift=1cm]| will locally shift the plot 1cm upward. Remember,
however, that most options can only be applied to paths as a
whole. For example, |plot[red]| does not have the effect of making the
plot red. After all, you are trying to ``locally'' make part of the
path red, which is not possible.

\subsection{Plotting a Function}
\label{section-tikz-plot}

When you plot a function, the coordinates of the plot data can be
computed by evaluating a mathematical expression. Since \pgfname\
comes with a mathematical engine, you can specify this expression and
then have \tikzname\ produce the desired coordinates for you,
automatically.

Since this case is quite common when plotting a function, the syntax
is easy: Following the |plot| command and its local options, you
directly provide a \meta{coordinate expression}. It looks like a
normal coordinate, but inside you may use a special macro, which is
|\x| by default, but this can be changed using the |variable|
option. The \meta{coordinate expression} is then evaluated for
different values for |\x| and the resulting coordinates are plotted.

Note that you will often have to put the $x$- or $y$-coordinate inside
braces, namely whenever you use an expression involving a paranthesis.

The following options influence how the \meta{coordinate expression}
is evaluated:
\begin{itemize}
  \itemoption{variable}|=|\meta{macro}
  sets the macro whose value is set to the different values when
  \meta{coordinate expression} is evaluated.
  \itemoption{samples}|=|\meta{number}
  sets the number of samples used in the plot. The default is 25.
  \itemoption{domain}|=|\meta{start}|:|\meta{end}
  sets the domain between which the samples are taken. The default is
  |-5:5|.
  \itemoption{samples at}|=|\meta{sample list}
  This option specifies a list of positions for which the
  variable should be evaluated. For instance, you can say
  |samples at={1,2,8,9,10}| to have the variable evaluated exactly for
  values $1$, $2$, $8$, $9$, and $10$. You can use the |\foreach|
  syntax, so you can use |...| inside the \meta{sample list}.

  When this option is used, the |samples| and |domain| option are
  overruled. The other ways round, setting either |samples| or
  |domain| will overrule this option.
\end{itemize}

\begin{codeexample}[]
\begin{tikzpicture}[domain=0:4]
  \draw[very thin,color=gray] (-0.1,-1.1) grid (3.9,3.9);
  
  \draw[->] (-0.2,0) -- (4.2,0) node[right] {$x$};
  \draw[->] (0,-1.2) -- (0,4.2) node[above] {$f(x)$};
  
  \draw[color=red]    plot (\x,\x)             node[right] {$f(x) =x$};
  \draw[color=blue]   plot (\x,{sin(\x r)})    node[right] {$f(x) = \sin x$};
  \draw[color=orange] plot (\x,{0.05*exp(\x)}) node[right] {$f(x) = \frac{1}{20} \mathrm e^x$};
\end{tikzpicture}
\end{codeexample}

\begin{codeexample}[]
\tikz \draw[scale=0.5,domain=-3.141:3.141,smooth,variable=\t]
  plot ({\t*sin(\t r)},{\t*cos(\t r)});
\end{codeexample}

\begin{codeexample}[]
\tikz \draw[domain=0:360,smooth,variable=\t]
  plot ({sin(\t)},\t/360,{cos(\t)});
\end{codeexample}


\subsection{Plotting a Function Using Gnuplot}
\label{section-tikz-gnuplot}

Often, you will want to plot points that are given via a function like
$f(x) = x \sin x$. Unfortunately, \TeX\ does not really have enough
computational power to generate the points on such a function
efficiently (it is a text processing program, after all). However,
if you allow it, \TeX\ can try to call external programs that can
easily produce the necessary points. Currently, \tikzname\ knows how to
call \textsc{gnuplot}.

When \tikzname\ encounters your operation
|plot[id=|\meta{id}|] function{x*sin(x)}| for 
the first time, it will create a file called
\meta{prefix}\meta{id}|.gnuplot|, where \meta{prefix} is |\jobname.| by
default, that is, the name of you main |.tex| file. If no \meta{id} is
given, it will be empty, which is alright, but it is better when each
plot has a unique \meta{id} for reasons explained in a moment. Next,
\tikzname\ writes some initialization code into this file followed by
|plot x*sin(x)|. The initialization code sets up things 
such that the |plot| operation will write the coordinates into another
file called \meta{prefix}\meta{id}|.table|. Finally, this table file
is read as if you had said |plot file{|\meta{prefix}\meta{id}|.table}|. 

For the plotting mechanism to work, two conditions must be met:
\begin{enumerate}
\item
  You must have allowed \TeX\ to call external programs. This is often
  switched off by default since this is a security risk (you might,
  without knowing, run a \TeX\ file that calls all sorts of ``bad''
  commands). To enable this ``calling external programs'' a command
  line option must be given to the \TeX\ program. Usually, it is
  called something like |shell-escape| or |enable-write18|. For
  example, for my |pdflatex| the option |--shell-escape| can be
  given.
\item
  You must have installed the |gnuplot| program and \TeX\ must find it
  when compiling your file.
\end{enumerate}

Unfortunately, these conditions will not always be met. Especially if
you pass some source to a coauthor and the coauthor does not have
\textsc{gnuplot} installed, he or she will have trouble compiling your
files.

For this reason, \tikzname\ behaves differently when you compile your
graphic for the second time: If upon reaching
|plot[id=|\meta{id}|] function{...}| the file \meta{prefix}\meta{id}|.table|
already exists \emph{and} if the \meta{prefix}\meta{id}|.gnuplot| file
contains what \tikzname\ thinks that it ``should'' contain, the |.table|
file is immediately read without trying to call a |gnuplot|
program. This approach has the following advantages: 
\begin{enumerate}
\item
  If you pass a bundle of your |.tex| file and all |.gnuplot| and
  |.table| files to someone else, that person can \TeX\ the |.tex|
  file without having to have |gnuplot| installed.
\item
  If the |\write18| feature is switched off for security reasons (a
  good idea), then, upon the first compilation of the |.tex| file, the
  |.gnuplot| will still be generated, but not the |.table|
  file. You can then simply call |gnuplot| ``by hand'' for each
  |.gnuplot| file, which will produce all necessary |.table| files.
\item
  If you change the function that you wish to plot or its
  domain, \tikzname\ will automatically try to regenerate the |.table|
  file.
\item
  If, out of laziness, you do not provide an |id|, the same |.gnuplot|
  will be used for different plots, but this is not a problem since
  the |.table| will automatically be regenerated for each plot
  on-the-fly. \emph{Note: If you intend to share your files with
  someone else, always use an id, so that the file can by typeset
  without having \textsc{gnuplot} installed.} Also, having unique ids
  for each plot will improve compilation speed since no external
  programs need to be called, unless it is really necessary.
\end{enumerate}

When you use |plot function{|\meta{gnuplot formula}|}|, the \meta{gnuplot
  formula} must be given in the |gnuplot| syntax, whose details are
beyond the scope of this manual. Here is the ultra-condensed
essence: Use |x| as the variable and use the C-syntax for normal
plots, use |t| as the variable for parametric plots. Here are some examples:

\begin{codeexample}[]
\begin{tikzpicture}[domain=0:4]
  \draw[very thin,color=gray] (-0.1,-1.1) grid (3.9,3.9);
  
  \draw[->] (-0.2,0) -- (4.2,0) node[right] {$x$};
  \draw[->] (0,-1.2) -- (0,4.2) node[above] {$f(x)$};
  
  \draw[color=red]    plot[id=x]   function{x}           node[right] {$f(x) =x$};
  \draw[color=blue]   plot[id=sin] function{sin(x)}      node[right] {$f(x) = \sin x$};
  \draw[color=orange] plot[id=exp] function{0.05*exp(x)} node[right] {$f(x) = \frac{1}{20} \mathrm e^x$};
\end{tikzpicture}
\end{codeexample}


The following options influence the plot:

\begin{itemize}
  \itemoption{samples}|=|\meta{number}
  sets the number of samples used in the plot. The default is 25.
  \itemoption{domain}|=|\meta{start}|:|\meta{end}
  sets the domain between which the samples are taken. The default is
  |-5:5|. 
  \itemoption{parametric}\opt{|=|\meta{true or false}}
  sets whether the plot is a parametric plot. If true, then |t| must
  be used instead of |x| as the parameter and two comma-separated
  functions must be given in the \meta{gnuplot formula}. An example is
  the following:
\begin{codeexample}[]
\tikz \draw[scale=0.5,domain=-3.141:3.141,smooth]
  plot[parametric,id=parametric-example] function{t*sin(t),t*cos(t)};
\end{codeexample}
  
  \itemoption{id}|=|\meta{id}
  sets the identifier of the current plot. This should be a unique
  identifier for each plot (though things will also work if it is not,
  but not as well, see the explanations above). The \meta{id} will be
  part of a filename, so it should not contain anything fancy like |*|
  or |$|.%$
  \itemoption{prefix}|=|\meta{prefix}
  is put before each plot file name. The default is |\jobname.|, but
  if you have many plots, it might be better to use, say |plots/| and
  have all plots placed in a directory. You have to create the
  directory yourself.
  \itemoption{raw gnuplot}
  causes the \meta{gnuplot formula} to be passed on to
  \textsc{gnuplot} without setting up the samples or the |plot|
  operation. Thus, you could write
\begin{codeexample}[code only]
plot[raw gnuplot,id=raw-example] function{set samples 25; plot sin(x)}
\end{codeexample}
  This can be 
  useful for complicated things that need to be passed to
  \textsc{gnuplot}. However, for really complicated situations you
  should create a special external generating \textsc{gnuplot} file
  and use the |file|-syntax to include the table ``by hand.''
\end{itemize}

The following styles influence the plot:
\begin{itemize}
  \itemstyle{every plot}
  This style is installed in each plot, that is, as if you always said
\begin{codeexample}[code only]
  plot[every plot,...]
\end{codeexample}
 This is most useful for globally setting a prefix for all plots by saying:
\begin{codeexample}[code only]
\tikzstyle{every plot}=[prefix=plots/]
\end{codeexample}
\end{itemize}



\subsection{Placing Marks on the Plot}

As we saw already, it is possible to add \emph{marks} to a plot using
the |mark| option. When this option is used, a copy of the plot
mark is placed on each point of the plot. Note that the marks are
placed \emph{after} the whole path has been drawn/filled/shaded. In
this respect, they are handled like text nodes. 

In detail, the following options govern how marks are drawn:
\begin{itemize}
  \itemoption{mark}|=|\meta{mark mnemonic}
  Sets the mark to a mnemonic that has previously been defined using
  the |\pgfdeclareplotmark|. By default, |*|, |+|, and |x| are available,
  which draw a filled circle, a plus, and a cross as marks. Many more
  marks become available when the library |pgflibraryplotmarks| is
  loaded. Section~\ref{section-plot-marks} lists the available plot
  marks.

  One plot mark is special: the |ball| plot mark is available only
  it \tikzname. The |ball color| determines the balls's color. Do not use
  this option with a large number of marks since it will take very long
  to render in PostScript.
  
  \begin{tabular}{lc}
    Option & Effect \\\hline \vrule height14pt width0pt
    \plotmarkentrytikz{ball}
  \end{tabular}

  \itemoption{mark repeat}|=|\meta{r}
  This option tells \tikzname\ that only every $r$th mark should be
  drawn.
  
\begin{codeexample}[]
\tikz \draw plot[mark=x,mark repeat=3,smooth] file {plots/pgfmanual-sine.table};
\end{codeexample}

  \itemoption{mark phase}|=|\meta{p}
  This option tells \tikzname\ that the first mark to be draw should
  be the $p$th, followed by the $(p+r)$th, then the $(p+2r)$th, and so
  on.
  
\begin{codeexample}[]
\tikz \draw plot[mark=x,mark repeat=3,mark phase=6,smooth] file {plots/pgfmanual-sine.table};
\end{codeexample}

  \itemoption{mark indices}|=|\meta{list}
  This option allows you to specify explicitly the indices at which a
  mark should be placed. Counting starts with 1. You can use the
  |\foreach| syntax, that is, |...| can be used.
    
\begin{codeexample}[]
\tikz \draw plot[mark=x,mark indices={1,4,...,10,11,12,...,16,20},smooth]
  file {plots/pgfmanual-sine.table};
\end{codeexample}
  
  \itemoption{mark size}|=|\meta{dimension}
  Sets the size of the plot marks. For circular plot marks,
  \meta{dimension} is the radius, for other plot marks
  \meta{dimension} should be about half the width and height.

  This option is not really necessary, since you achieve the same
  effect by specifying |scale=|\meta{factor} as a local option, where
  \meta{factor} is the quotient of the desired size and the default
  size. However, using |mark size| is a bit faster and more natural. 

  \itemoption{mark options}|=|\meta{options}
  These options are applied to marks when they are drawn. For example,
  you can scale (or otherwise transform) the plot mark or set its
  color. 
\begin{codeexample}[]
\tikz \fill[fill=blue!20]
  plot[mark=triangle*,mark options={color=blue,rotate=180}]
    file{plots/pgfmanual-sine.table} |- (0,0);
\end{codeexample}
\end{itemize}



\subsection{Smooth Plots, Sharp Plots, and Comb Plots}

There are different things the |plot| operation can do with the points
it reads from a file or from the inlined list of points. By default,
it will connect these points by straight lines. However, you can also
use options to change the behavior of |plot|.

\begin{itemize}
  \itemoption{sharp plot}
  This is the default and causes the points to be connected by
  straight lines. This option is included only so that you can
  ``switch back'' if you ``globally'' install, say, |smooth|.
  
  \itemoption{smooth}
  This option causes the points on the path to be connected using a
  smooth curve:

\begin{codeexample}[]
\tikz\draw plot[smooth] file{plots/pgfmanual-sine.table};
\end{codeexample}

  Note that the smoothing algorithm is not very intelligent. You will
  get the best results if the bending angles are small, that is, less
  than about $30^\circ$ and, even more importantly, if the distances
  between points are about the same all over the plotting path.

  \itemoption{tension}|=|\meta{value}
  This option influences how ``tight'' the smoothing is. A lower value
  will result in sharper corners, a higher value in more ``round''
  curves. A value of $1$ results in a circle if four points at
  quarter-positions on a circle are given. The default is $0.55$. The
  ``correct'' value depends on the details of plot.
  
\begin{codeexample}[]
\begin{tikzpicture}[smooth cycle]
  \draw                 plot[tension=0.2]
    coordinates{(0,0) (1,1) (2,0) (1,-1)};
  \draw[yshift=-2.25cm] plot[tension=0.5]
    coordinates{(0,0) (1,1) (2,0) (1,-1)};
  \draw[yshift=-4.5cm]  plot[tension=1]
    coordinates{(0,0) (1,1) (2,0) (1,-1)};
\end{tikzpicture}
\end{codeexample}
  
  \itemoption{smooth cycle}
  This option causes the points on the path to be connected using a
  closed smooth curve. 

\begin{codeexample}[]
\tikz[scale=0.5]
  \draw plot[smooth cycle] coordinates{(0,0) (1,0) (2,1) (1,2)}
        plot               coordinates{(0,0) (1,0) (2,1) (1,2)} -- cycle;
\end{codeexample}

  \itemoption{ycomb}
  This option causes the |plot| operation to interpret the plotting
  points differently. Instead of connecting them, for each point of
  the plot a straight line is added to the path from the $x$-axis to the point,
  resulting in a sort of ``comb'' or ``bar diagram.''

\begin{codeexample}[]
\tikz\draw[ultra thick] plot[ycomb,thin,mark=*] file{plots/pgfmanual-sine.table};
\end{codeexample}

\begin{codeexample}[]
\begin{tikzpicture}[ycomb]
  \draw[color=red,line width=6pt]
    plot coordinates{(0,1) (.5,1.2) (1,.6) (1.5,.7) (2,.9)};
  \draw[color=red!50,line width=4pt,xshift=3pt]
    plot coordinates{(0,1.2) (.5,1.3) (1,.5) (1.5,.2) (2,.5)};
\end{tikzpicture}
\end{codeexample}

  \itemoption{xcomb}
  This option works like |ycomb| except that the bars are horizontal. 

\begin{codeexample}[]
\tikz \draw plot[xcomb,mark=x] coordinates{(1,0) (0.8,0.2) (0.6,0.4) (0.2,1)};
\end{codeexample}

  \itemoption{polar comb}
  This option causes a line from the origin to the point to be added
  to the path for each plot point.

\begin{codeexample}[]
\tikz \draw plot[polar comb,
     mark=pentagon*,mark options={fill=white,draw=red},mark size=4pt]
   coordinates {(0:1cm) (30:1.5cm) (160:.5cm) (250:2cm) (-60:.8cm)};
\end{codeexample}


  \itemoption{only marks}
  This option causes only marks to be shown; no path segments are
  added to the actual path. This can be useful for quickly adding some
  marks to a path.

\begin{codeexample}[]
\tikz \draw (0,0) sin (1,1) cos (2,0)
  plot[only marks,mark=x] coordinates{(0,0) (1,1) (2,0) (3,-1)};
\end{codeexample}
\end{itemize}



% Copyright 2006 by Till Tantau
%
% This file may be distributed and/or modified
%
% 1. under the LaTeX Project Public License and/or
% 2. under the GNU Free Documentation License.
%
% See the file doc/generic/pgf/licenses/LICENSE for more details.


\section{Transparency}

\label{section-tikz-transparency}


\subsection{Overview}

Normally, when you paint something using any of \tikzname's commands
(this includes stroking, filling, shading, patterns, and images), the
newly painted objects totally obscure whatever was painted earlier in
the same area.

You can change this behaviour by using something that can be thought
of as ``(semi)transparent colors.'' Such colors do not completely
obscure the background, rather they blend the background with the new
color. At first sight, using such semitransparent colors might seem quite
straightforward, but the math going on in the background is quite
involved and the correct handling of transparency fills some 64 pages
in the PDF specification.

In the present section, we start with the different ways of specifying
``how transparent'' newly drawn objects should be. The simplest way is
to just specify a percentage like ``60\% transparent.'' A much more
general way is to use something that I call a \emph{fading,} also
known as a soft mask or a mask.

At the end of the section we address the problem of creating so-called
\emph{transparency groups}. This problem arises when you paint over a
position several times with a semitransparent color. Sometimes you
want the effect to accumulate, sometimes you do not.

\emph{Note:} Transparency is best supported by the pdf\TeX\
driver. The \textsc{svg} driver also has some support. For PostScript
output, opacity is rendered correctly only with the most recent
versions of Ghostscript. Printers and other programs will typically
ignore the opacity setting.



\subsection{Specifying a Uniform Opacity}

Specifying a stroke and/or fill opacity is quite easy using the
following options.


\begin{key}{/tikz/draw opacity=\meta{value}}
  This option sets ``how transparent'' lines should be. A value of |1|
  means ``fully opaque'' or ``not transparent at all,'' a value of |0|
  means ``fully transparent'' or ``invisible.'' A value of |0.5|
  yields lines that are semitransparent.

  Note that when you use PostScript as your output format,
  this option works only with recent versions of Ghostscript.

\begin{codeexample}[]
\begin{tikzpicture}[line width=1ex]
  \draw (0,0) -- (3,1);
  \filldraw [fill=yellow!80!black,draw opacity=0.5] (1,0) rectangle (2,1);
\end{tikzpicture}
\end{codeexample}
\end{key}

Note that the |draw opacity| options only sets the opacity of drawn
lines. The opacity of fillings is set using the option
|fill opacity| (documented in Section~\ref{section-fill-opacity}. The
option |opacity| sets both at the same time.

\begin{key}{/tikz/opacity=\meta{value}}
  Sets both the drawing and filling opacity to \meta{value}.

  The following predefined styles make it easier to use this option:
  \begin{stylekey}{/tikz/transparent}
    Makes everything totally transparent and, hence, invisible.

\begin{codeexample}[]
\tikz{\fill[red]             (0,0)   rectangle (1,0.5);
      \fill[transparent,red] (0.5,0) rectangle (1.5,0.25); }
\end{codeexample}
  \end{stylekey}

  \begin{stylekey}{/tikz/ultra nearly transparent}
    Makes everything, well, ultra nearly transparent.

\begin{codeexample}[]
\tikz{\fill[red]                      (0,0)   rectangle (1,0.5);
      \fill[ultra nearly transparent] (0.5,0) rectangle (1.5,0.25); }
\end{codeexample}
  \end{stylekey}

  \begin{stylekey}{/tikz/very nearly transparent}
\begin{codeexample}[]
\tikz{\fill[red]                     (0,0)   rectangle (1,0.5);
      \fill[very nearly transparent] (0.5,0) rectangle (1.5,0.25); }
\end{codeexample}
  \end{stylekey}

  \begin{stylekey}{/tikz/nearly transparent}
\begin{codeexample}[]
\tikz{\fill[red]                (0,0)   rectangle (1,0.5);
      \fill[nearly transparent] (0.5,0) rectangle (1.5,0.25); }
\end{codeexample}
  \end{stylekey}

  \begin{stylekey}{/tikz/semitransparent}
\begin{codeexample}[]
\tikz{\fill[red]             (0,0)   rectangle (1,0.5);
      \fill[semitransparent] (0.5,0) rectangle (1.5,0.25); }
\end{codeexample}
  \end{stylekey}

  \begin{stylekey}{/tikz/nearly opaque}
\begin{codeexample}[]
\tikz{\fill[red]           (0,0)   rectangle (1,0.5);
      \fill[nearly opaque] (0.5,0) rectangle (1.5,0.25); }
\end{codeexample}
  \end{stylekey}

  \begin{stylekey}{/tikz/very nearly opaque}
\begin{codeexample}[]
\tikz{\fill[red]                (0,0)   rectangle (1,0.5);
      \fill[very nearly opaque] (0.5,0) rectangle (1.5,0.25); }
\end{codeexample}
  \end{stylekey}

  \begin{stylekey}{/tikz/ultra nearly opaque}
\begin{codeexample}[]
\tikz{\fill[red]                 (0,0)   rectangle (1,0.5);
      \fill[ultra nearly opaque] (0.5,0) rectangle (1.5,0.25); }
\end{codeexample}
  \end{stylekey}

  \begin{stylekey}{/tikz/opaque}
    This yields completely opaque drawings, which is the default.
\begin{codeexample}[]
\tikz{\fill[red]    (0,0)   rectangle (1,0.5);
      \fill[opaque] (0.5,0) rectangle (1.5,0.25); }
\end{codeexample}
  \end{stylekey}
\end{key}


\begin{key}{/tikz/fill opacity=\meta{value}}
  This option sets the opacity of fillings. In addition to filling
  operations, this opacity also applies to text and images.

  Note, again, that when you use PostScript as your output format,
  this option works only with recent versions of Ghostscript.

\begin{codeexample}[]
\begin{tikzpicture}[thick,fill opacity=0.5]
  \filldraw[fill=red]   (0:1cm)    circle (12mm);
  \filldraw[fill=green] (120:1cm)  circle (12mm);
  \filldraw[fill=blue]  (-120:1cm) circle (12mm);
\end{tikzpicture}
\end{codeexample}

\begin{codeexample}[]
\begin{tikzpicture}
  \fill[red] (0,0) rectangle (3,2);

  \node                   at (0,0) {\huge A};
  \node[fill opacity=0.5] at (3,2) {\huge B};
\end{tikzpicture}
\end{codeexample}
\end{key}

\begin{key}{/tikz/text opacity=\meta{value}}
  Sets the opacity of text labels, overriding the |fill opacity| setting.
\begin{codeexample}[]
\begin{tikzpicture}[every node/.style={fill,draw}]
  \draw[line width=2mm,blue!50,line cap=round] (0,0) grid (3,2);

  \node[opacity=0.5] at (1.5,2) {Upper node};
  \node[draw opacity=0.8,fill opacity=0.2,text opacity=1]
    at (1.5,0) {Lower node};
\end{tikzpicture}
\end{codeexample}
\end{key}


Note the following effect: If you set up a certain opacity for stroking
or filling and you stroke or fill the same area twice, the effect
accumulates:

\begin{codeexample}[]
\begin{tikzpicture}[fill opacity=0.5]
  \fill[red] (0,0) circle (1);
  \fill[red] (1,0) circle (1);
\end{tikzpicture}
\end{codeexample}

Often, this is exactly what you intend, but not always. You can use
transparency groups, see the end of this section, to change this.



\subsection{Blend Modes}
\label{section-blend-modes}

A \emph{blend mode} specifies how colors mix when you paint on a
canvas. Normally, if you paint a red box on a green circle, the red
color will completely replace the green circle. However, in some
situations you might also wish the red color to somehow ``mix'' or
``blend'' with the green circle. We already saw that, using transparency,
we can draw something without completely obscuring the
background. \emph{Blending} is a similar operation, only here we mix
colors in more complicated ways.

\emph{Note:} Blending is a rather ``advanced'' feature of
\textsc{pdf}. Most renderers, let alone printers, will have trouble
rendering blending correctly.

\begin{key}{/tikz/blend mode=\meta{mode}}
  Sets the current blend mode to \meta{mode}. Here \meta{mode} must be
  one of the modes described in Section 7.2.4 of the \textsc{pdf}
  Specification, version 1.7.

  In the following example, the blend mode is only used and set inside
  a transparency group (see also
  Section~\ref{section-transparency-groups}). This is because most
  renderers (viewing 
  programs) have trouble rendering blending correctly otherwise. For
  instance, at the time of writing, the versions of Adobe's Reader and
  Apple's Preview render the following drawing very differently, if
  the transparency group is not used in the following example.

\begin{codeexample}[]
\tikz {
  \begin{scope}[transparency group]
    \begin{scope}[blend mode=Screen] 
      \fill[red!90!black]   ( 90:.6) circle (1);
      \fill[green!80!black] (210:.6) circle (1);
      \fill[blue!90!black]  (330:.6) circle (1);
    \end{scope}
  \end{scope}
}
\end{codeexample}

  Because of the trouble with rendering blending correctly outside
  transparency groups, there is a special key that establishes a
  transparency group and sets a blend mode simultaneously:
  
  \begin{key}{/tikz/blend group=\meta{mode}}
    This key can only be used with a scope (like
    |transparency group|). It will cause the current scope to become a
    transparency group and, inside this group, the blend mode will be
    set to \meta{mode}.

\begin{codeexample}[]
\tikz [blend group=Screen] {
  \fill[red!90!black]   ( 90:.6) circle (1);
  \fill[green!80!black] (210:.6) circle (1);
  \fill[blue!90!black]  (330:.6) circle (1);
}
\end{codeexample}
  \end{key}

  Here is an overview of the effects of the different available blend
  modes. In the examples, we always have three circles drawn on
  top of each other (as in the example code earlier): We start with a
  triple of pure red, green, and blue. Below it, we have a triple of
  light versions of these three colors (|red!50|, |green!50|, and
  |blue!50|). Next comes the triple  yellow, cyan, and magenta; again
  with a triple of light versions below it. The large example consists
  of three balls (produced using |ball color|) having the colors red,
  green, and blue, are drawn on top of each other just like the
  circles.  
  
  \definecolor{rg}{rgb}{1,1,0}
  \definecolor{gb}{rgb}{0,1,1}
  \definecolor{br}{rgb}{1,0,1}
  
  \def\makeline#1#2#3{\leavevmode
    \hbox to 4cm{#1\hss}\ \hbox to
    2cm{#2\hss}\ \begin{minipage}[t]{9cm}\raggedright#3\end{minipage}\par
    \textcolor{black!25}{\hrule height1pt}
  }

  \def\showmode#1#2{
    \makeline{
    \tikz [blend mode=#1,baseline=-.5ex] {
      \fill[red]      ( 90:.5em) circle (.75em);
      \fill[green]    (210:.5em) circle (.75em);
      \fill[blue]     (330:.5em) circle (.75em);
      \scoped[yshift=-2.5em]{
        \fill[red!50]   ( 90:.5em) circle (.75em);
        \fill[green!50] (210:.5em) circle (.75em);
        \fill[blue!50]  (330:.5em) circle (.75em);
      }
    }
    \tikz [blend mode=#1,baseline=-.5ex] {
      \fill[rg]   ( 90:.5em) circle (.75em);
      \fill[gb]  (210:.5em) circle (.75em);
      \fill[br]    (330:.5em) circle (.75em);
      \scoped[yshift=-2.5em]{
        \fill[rg!50]  ( 90:.5em) circle (.75em);
        \fill[gb!50]  (210:.5em) circle (.75em);
        \fill[br!50]  (330:.5em) circle (.75em);
      }
    }
    \tikz [blend mode=#1,baseline=-.5ex+1.25em] {
      \shade[ball color=red]      ( 90:1em) circle (1.5em);
      \shade[ball color=green]    (210:1em) circle (1.5em);
      \shade[ball color=blue]     (330:1em) circle (1.5em);
    }}{|#1|}{#2}}

  \medskip
  \makeline{\emph{Example}}{\emph{Mode}}{\emph{Explanations quoted from Table 7.2 of the
      \textsc{pdf} Specification, Version 1.7}}
  \showmode{Normal}{When painting a pixel with a some color (called
    the ``source color''), the background color
      (called the ``backdrop'') is completely ignored.}
    \showmode{Multiply}{Multiplies the backdrop and source color
      values. The result color is always at least as dark as
      either of the two constituent colors. Multiplying any color with
      black produces black; multiplying with white leaves the original
      color unchanged. Painting successive overlapping objects with a
      color other than black or white produces progressively darker
      colors.}
    \showmode{Screen}{Multiplies the complements of the backdrop and
      source color       values, then complements the result. The
      result color is always 
      at least as light as either of the two constituent
      colors. Screening any color with white produces white; screening
      with black leaves the original color unchanged. The effect is
      similar to projecting multiple photographic slides
      simultaneously onto a single screen.}
    \showmode{Overlay}{Multiplies or screens the colors, depending on
      the backdrop color value. Source colors overlay the backdrop
      while preserving its highlights and shadows. The backdrop color
      is not replaced but is mixed with the source color to reflect
      the lightness or darkness of the backdrop.}
    \showmode{Darken}{Selects the darker of the backdrop and source
      colors. The backdrop is replaced with the source where the
      source is darker; otherwise, it is left unchanged.}
    \showmode{Lighten}{Selects the lighter of the backdrop and source
      colors. The backdrop is replaced with the source where the
      source is lighter; otherwise, it is left unchanged.}
    \showmode{ColorDodge}{Brightens the backdrop color to reflect the
      source color. Painting with black produces no changes.}
    \showmode{ColorBurn}{Darkens the backdrop color to reflect the
      source color. Painting with white produces no change.}
    \showmode{HardLight}{Multiplies or screens the colors, depending
      on the source color value. The effect is similar to shining a
      harsh spotlight on the backdrop.}
    \showmode{SoftLight}{Darkens or lightens the colors, depending on
      the source color value. The effect is similar to shining a
      diffused spotlight on the backdrop.}
    \showmode{Difference}{Subtracts the darker of the two constituent
      colors from the lighter color. Painting with white inverts the
      backdrop color; painting with black produces no change.}
    \showmode{Exclusion}{Produces an effect similar to that of the
      Difference mode but lower in contrast. Painting with white
      inverts the backdrop color; painting with black produces no
      change.}
    \showmode{Hue}{Creates a color with the hue of the source color
      and the saturation and luminosity of the backdrop color.} 
    \showmode{Saturation}{Creates a color with the saturation of the
      source color and the hue and luminosity of the backdrop
      color. Painting with this mode in an area of the backdrop that
      is a pure gray (no saturation) produces no change.}
    \showmode{Color}{Creates a color with the hue and saturation of
      the source color and the luminosity of the backdrop color. This
      preserves the gray levels of the backdrop and is useful for
      coloring monochrome images or tinting color images.}
    \showmode{Luminosity}{Creates a color with the luminosity of the
      source color and the hue and saturation of the backdrop
      color. This produces an inverse effect to that of the Color
      mode.}
  
\end{key}



\subsection{Fadings}

For complicated graphics, uniform transparency settings are not always
sufficient. Suppose, for instance, that while you paint a picture, you
want the transparency to vary smoothly from completely opaque to
completely transparent. This is a ``shading-like'' transparency. For
such a form of transparency I will use the term \emph{fading} (as a
noun). They are also known as \emph{soft masks}, \emph{opacity masks},
\emph{masks}, or \emph{soft clips}.


\subsubsection{Creating Fadings}

How do we specify a fading? This is a bit of an art since the
underlying mechanism is quite powerful, but a bit difficult to use.

Let us start with a bit of terminology. A \emph{fading} specifies for
each point of an area the transparency of that point. This transparency
can by any number between 0 and 1. A \emph{fading picture} is a normal
graphic that, in a way to be described in a moment, determines the
transparency of points inside the fading. Each fading has an
underlying fading picture.

The fading picture is a normal graphic drawn using any of the normal
graphic drawing commands. A fading and its fading picture are related
as follows: Given any point of the fading, the transparency of this
point is determined by the luminosity of the fading picture at the
same position. The luminosity of a point determines ``how bright'' the
point is. The brighter the point in the fading picture, the more
opaque is the point in the fading. In particular, a white point of the
fading picture is completely opaque in the fading and a black point of
the fading picture is completely transparent in the fading. (The
background of the fading picture is always transparent in the fading
as if the background were black.)

It is rather counter-intuitive that a \emph{white} pixel of the fading
picture will be \emph{opaque} in the fading and a \emph{black} pixel
will be \emph{transparent}. For this reason, \tikzname\ defines a
color called |transparent| that is the same as |black|. The nice thing
about this definition is that the color
|transparent!|\meta{percentage} in the fading picture yields a
pixel that is \meta{percentage} percent transparent in the fading.

Turning a fading picture into a normal picture is achieved using the
following commands, which are \emph{only defined in the library},
namely the library |fadings|. So, to use them, you have to say
|\usetikzlibrary{fadings}| first.

\begin{environment}{{tikzfadingfrompicture}\oarg{options}}
  This command works like a |{tikzpicture}|, only the picture is not
  shown, but instead a fading is defined based on this picture. To set
  the name of the picture, use the |name| option (which is normally
  used to set the name of a node).
  \begin{key}{/tikz/name=\marg{name}}
    Use this option with the |{tikzfadingfrompicture}| environment to
    set the name of the fading. You \emph{must} provide this option.
  \end{key}

  The following shading is 2cm by 2cm and gets more and more
  transparent from left to right, but is 50\% transparent for a large
  circle in the middle.
{\tikzexternaldisable
\begin{codeexample}[]
\begin{tikzfadingfrompicture}[name=fade right]
  \shade[left color=transparent!0,
         right color=transparent!100] (0,0) rectangle (2,2);
  \fill[transparent!50] (1,1) circle (0.7);
\end{tikzfadingfrompicture}

% Now we use the fading in another picture:
\begin{tikzpicture}
  % Background
  \fill [black!20] (-1.2,-1.2) rectangle (1.2,1.2);
  \pattern [pattern=checkerboard,pattern color=black!30]
                   (-1.2,-1.2) rectangle (1.2,1.2);

  \fill [path fading=fade right,red] (-1,-1) rectangle (1,1);
\end{tikzpicture}
\end{codeexample}
  In the next example we create a fading picture that contains some
  text. When the fading is used, we only see the shading ``through
  it.''
\begin{codeexample}[]
\begin{tikzfadingfrompicture}[name=tikz]
  \node [text=transparent!20]
    {\fontfamily{ptm}\fontsize{45}{45}\bfseries\selectfont Ti\emph{k}Z};
\end{tikzfadingfrompicture}

% Now we use the fading in another picture:
\begin{tikzpicture}
  \fill [black!20] (-2,-1) rectangle (2,1);
  \pattern [pattern=checkerboard,pattern color=black!30]
                   (-2,-1) rectangle (2,1);

  \shade[path fading=tikz,fit fading=false,
         left color=blue,right color=black]
    (-2,-1) rectangle (2,1);
\end{tikzpicture}
\end{codeexample}
}%

  The same effect can also be achieved using knockout groups, see
  Section~\ref{section-transparency-groups}.
\end{environment}

\begin{plainenvironment}{{tikzfadingfrompicture}\oarg{options}}
  The plain\TeX\ version of the environment.
\end{plainenvironment}

\begin{contextenvironment}{{tikzfadingfrompicture}\oarg{options}}
  The Con\TeX t version of the environment.
\end{contextenvironment}

\begin{command}{\tikzfading\oarg{options}}
  This command is used to define a fading similarly to the way a
  shading is defined. In the \meta{options} you should
  \begin{enumerate}
  \item use the |name=|\meta{name} option to set a name for the fading,
  \item use the |shading| option to set the name of the shading that
    you wish to use,
  \item extra options for setting the colors of the shading (typically
    you will set them to the color |transparent!|\meta{percentage}).
  \end{enumerate}
  Then, a new fading named \meta{name} will be created based on the
  shading.

\begin{codeexample}[]
\tikzfading[name=fade right,
            left color=transparent!0,
            right color=transparent!100]

% Now we use the fading in another picture:
\begin{tikzpicture}
  % Background
  \fill [black!20] (-1.2,-1.2) rectangle (1.2,1.2);
  \path [pattern=checkerboard,pattern color=black!30]
                   (-1.2,-1.2) rectangle (1.2,1.2);

  \fill [red,path fading=fade right] (-1,-1) rectangle (1,1);
\end{tikzpicture}
\end{codeexample}

\begin{codeexample}[]
\tikzfading[name=fade out,
            inner color=transparent!0,
            outer color=transparent!100]

% Now we use the fading in another picture:
\begin{tikzpicture}
  % Background
  \fill [black!20] (-1.2,-1.2) rectangle (1.2,1.2);
  \path [pattern=checkerboard,pattern color=black!30]
                   (-1.2,-1.2) rectangle (1.2,1.2);

  \fill [blue,path fading=fade out] (-1,-1) rectangle (1,1);
\end{tikzpicture}
\end{codeexample}
\end{command}



\subsubsection{Fading a Path}

A fading specifies for each pixel of a certain area how transparent
this pixel will be. The following options are used to install such a
fading for the current scope or path.

\pgfdeclarefading{fade down}{%
  \tikzset{top color=pgftransparent!0,bottom color=pgftransparent!100}
  \pgfuseshading{axis}
}
\pgfdeclarefading{fade inside}{%
  \tikzset{inner color=pgftransparent!90,outer color=pgftransparent!30}
  \pgfuseshading{radial}
}

\begin{key}{/tikz/path fading=\meta{name} (default \normalfont scope's setting)}
  This option tells \tikzname\ that the current path should be faded
  with the fading \meta{name}. If no \meta{name} is given, the
  \meta{name} set for the whole scope is used. Similarly to options
  like |draw| or |fill|, this option is reset for each path, so you
  have to add it to each path that should be faded. You can also
  specify |none| as \meta{name}, in which case fading for the path
  will be switched off in case it has been switched on by previous
  options or styles.
\begin{codeexample}[]
\begin{tikzpicture}[path fading=south]
  % Checker board
  \fill [black!20] (0,0) rectangle (4,3);
  \pattern [pattern=checkerboard,pattern color=black!30]
                   (0,0) rectangle (4,3);

  \fill [color=blue]                   (0.5,1.5) rectangle +(1,1);
  \fill [color=blue,path fading=north] (2.5,1.5) rectangle +(1,1);

  \fill [color=red,path fading]        (1,0.75) ellipse (.75 and .5);
  \fill [color=red]                    (3,0.75) ellipse (.75 and .5);
\end{tikzpicture}
\end{codeexample}

  \begin{key}{/tikz/fit fading=\meta{boolean} (default true, initially true)}
    When set to |true|, the fading is shifted and resized (in exactly
    the same way as a shading) so that it covers the current
    path. When set to |false|, the fading is only shifted so that it
    is centered on the path's center, but it is not resized. This can
    be useful for special-purpose fadings, for instance when you use a
    fading to ``punch out'' something.                                     
  \end{key}

  \begin{key}{/tikz/fading transform=\meta{transformation options}}
    The \meta{transformation options} are applied to the fading before
    it is used. For instance, if \meta{transformation options} is set
    to |rotate=90|, the fading is rotated by 90 degrees.
\begin{codeexample}[]
\begin{tikzpicture}[path fading=fade down]
  % Checker board
  \fill [black!20] (0,0) rectangle (4,1.5);
  \path [pattern=checkerboard,pattern color=black!30] (0,0) rectangle (4,1.5);

  \fill [red,path fading,fading transform={rotate=90}]
    (1,0.75) ellipse (.75 and .5);
  \fill [red,path fading,fading transform={rotate=30}]
    (3,0.75) ellipse (.75 and .5);
\end{tikzpicture}
\end{codeexample}
  \end{key}

  \begin{key}{/tikz/fading angle=\meta{degree}}
    A shortcut for |fading transform={rotate=|\meta{degree}|}|.
  \end{key}

  Note that you can ``fade just about anything.'' In particular, you
  can fade a shading.

\begin{codeexample}[]
\begin{tikzpicture}
  % Checker board
  \fill [black!20] (0,0) rectangle (4,4);
  \path [pattern=checkerboard,pattern color=black!30] (0,0) rectangle (4,4);

  \shade [ball color=blue,path fading=south] (2,2) circle (1.8);
\end{tikzpicture}
\end{codeexample}

  The |fade inside| of the following example is more transparent in the middle than on the
  outside.

\begin{codeexample}[]
\tikzfading[name=fade inside,
            inner color=transparent!80,
            outer color=transparent!30]
\begin{tikzpicture}
  % Checker board
  \fill [black!20] (0,0) rectangle (4,4);
  \path [pattern=checkerboard,pattern color=black!30] (0,0) rectangle (4,4);

  \shade [ball color=red] (3,3) circle (0.8);
  \shade [ball color=white,path fading=fade inside] (2,2) circle (1.8);
\end{tikzpicture}
\end{codeexample}

  Note that adding the |path fading| option to a node fades the
  (background) path, not the text itself. To fade the text, you need
  to use a scope fading (see below).
\end{key}

Note that using fadings in conjunction with patterns can create
visually rather pleasing effects:
\begin{codeexample}[]
\tikzfading[name=middle,
            top color=transparent!50,
            bottom color=transparent!50,
            middle color=transparent!20]
\begin{tikzpicture}
  \node      [circle,circular drop shadow,
              pattern=horizontal lines dark blue,
              path fading=south,
              minimum size=3.6cm] {};
  \pattern   [path fading=north,
              pattern=horizontal lines dark gray]
    (0,0) circle (1.8cm);
  \pattern   [path fading=middle,
              pattern=crosshatch dots light steel blue]
    (0,0) circle (1.8cm);
\end{tikzpicture}
\end{codeexample}


\subsubsection{Fading a Scope}

In addition to fading individual paths, you may also wish to ``fade a
scope,'' that is, you may wish to install a fading that is used
globally to specify the transparency for all objects drawn inside a
scope. This effect can also be thought of as a ``soft clip'' and it
works in a similar way: You add the |scope fading| option to a path in
a scope -- typically the first one -- and then all subsequent drawings
in the scope are faded. You will use a |transparency group| in
conjunction, see the end of this section.

\begin{key}{/tikz/scope fading=\meta{fading}}
  In principle, this key works in exactly the same way as the
  |path fading| key. The only difference is, that the effect of the
  fading will persist after the current path till the end of the
  scope. Thus, the \meta{fading} is applied to all subsequent drawings
  in the current scope, not just to the current path. In this regard,
  the option works very much like the |clip| option. (Note, however,
  that, unlike the |clip| option, fadings to not accumulate unless a
  transparency group is used.)

  The keys |fit fading| and |fading transform| have the same effect as
  for |path fading|. Also that, just as for |path fading|, providing
  the |scope fading| option with a |{scope}| only sets the name of the
  fading to be used. You have to explicitly provide the |scope fading|
  with a path to actually install a fading.

\begin{codeexample}[]
\begin{tikzpicture}
  \fill [black!20] (-2,-2) rectangle (2,2);
  \pattern [pattern=checkerboard,pattern color=black!30]
                   (-2,-2) rectangle (2,2);

  % The bounding box of the shading:
  \draw [red] (-50bp,-50bp) rectangle (50bp,50bp);

  \path [scope fading=south,fit fading=false] (0,0);
  % fading is centered at its natural size

  \fill[red]   ( 90:1) circle (1);
  \fill[green] (210:1) circle (1);
  \fill[blue]  (330:1) circle (1);
\end{tikzpicture}
\end{codeexample}

  In the following example we resize the fading to the size of the
  whole picture:
\begin{codeexample}[]
\begin{tikzpicture}
  \fill [black!20] (-2,-2) rectangle (2,2);
  \pattern [pattern=checkerboard,pattern color=black!30]
                   (-2,-2) rectangle (2,2);

  \path [scope fading=south] (-2,-2) rectangle (2,2);

  \fill[red]   ( 90:1) circle (1);
  \fill[green] (210:1) circle (1);
  \fill[blue]  (330:1) circle (1);
\end{tikzpicture}
\end{codeexample}

  Scope fadings are also needed if you wish to fade a node.
\begin{codeexample}[]
\tikz \node [scope fading=south,fading angle=45,text width=3.5cm]
{
  This is some text that will fade out as we go right
  and down. It is pretty hard to achieve this effect in
  other ways.
};
\end{codeexample}

\end{key}


\subsection{Transparency Groups}
\label{section-transparency-groups}

Consider the following cross and sign. They ``look wrong'' because we
can see how they were constructed, while this is not really part of
the desired effect.

\begin{codeexample}[]
\begin{tikzpicture}[opacity=.5]
  \draw [line width=5mm] (0,0) -- (2,2);
  \draw [line width=5mm] (2,0) -- (0,2);
\end{tikzpicture}
\end{codeexample}

\begin{codeexample}[]
\begin{tikzpicture}
  \node at (0,0) [forbidden sign,line width=2ex,draw=red,fill=white] {Smoking};

  \node [opacity=.5]
        at (2,0) [forbidden sign,line width=2ex,draw=red,fill=white] {Smoking};
\end{tikzpicture}
\end{codeexample}

Transparency groups are used to render them correctly:

\begin{codeexample}[]
\begin{tikzpicture}[opacity=.5]
  \begin{scope}[transparency group]
    \draw [line width=5mm] (0,0) -- (2,2);
    \draw [line width=5mm] (2,0) -- (0,2);
  \end{scope}
\end{tikzpicture}
\end{codeexample}

\begin{codeexample}[]
\begin{tikzpicture}
  \node at (0,0) [forbidden sign,line width=2ex,draw=red,fill=white] {Smoking};

  \begin{scope}[opacity=.5,transparency group]
    \node at (2,0) [forbidden sign,line width=2ex,draw=red,fill=white]
      {Smoking};
  \end{scope}
\end{tikzpicture}
\end{codeexample}

\begin{key}{/tikz/transparency group=\oarg{options}}
  This option can be given to a |scope|. It will have the following
  effect: The scope's contents is stroked\,/\penalty0\,filled
  ``ignoring any outside transparency.'' This means, all previous
  transparency settings are ignored (you can still set transparency
  inside the group, but never mind). For instance, in the forbidden
  sign example, the whole sign is first painted (conceptually) like
  the image on the left hand side. Note that some pixels of the sign
  are painted multiple times (up to three times), but only the last
  color ``wins.''

  Then, when the scope is finished, it is painted as a whole. The
  \emph{fill} transparency settings are now applied to the resulting
  picture. For instance, the pixel that has been painted three times
  is just red at the end, so this red color will be blended with
  whatever is ``behind'' the group on the page.

\begin{codeexample}[]
\begin{tikzpicture}
  \pattern[pattern=checkerboard,pattern color=black!15](-1,-1) rectangle (3,1);
  \node at (0,0) [forbidden sign,line width=2ex,draw=red,fill=white] {Smoking};

  \begin{scope}[transparency group,opacity=.5]
    \node at (2,0) [forbidden sign,line width=2ex,draw=red,fill=white]
      {Smoking};
  \end{scope}
\end{tikzpicture}
\end{codeexample}

  Note that in the example, the |opacity=.5| is not active inside the
  transparency group: The group is only established at beginning of
  the scope and all options given to the |{scope}| environment are set
  before the group is established. To change the opacity \emph{inside}
  the group, you need to open another scope inside it or use the
  |opacity| key with a command inside the group:

\begin{codeexample}[]
\begin{tikzpicture}
  \pattern[pattern=checkerboard,pattern color=black!15](-1,-1) rectangle (3,1);
  \node at (0,0) [forbidden sign,line width=2ex,draw=red,fill=white] {Smoking};

  \begin{scope}[transparency group,opacity=.5]
    \node (s) at (2,0) [forbidden sign,line width=2ex,draw=red,fill=white]
    {Smoking};

    \draw [opacity=.5, line width=2ex, blue] (1.2,0) -- (2.8,0);
  \end{scope}
\end{tikzpicture}
\end{codeexample}

  The \meta{options} are a list of comma-separated options:
  \begin{itemize}
  \item \declare{|knockout|} When this option is given inside the
    \meta{options}, the group becomes a so-called \emph{knockout}
    group. This means, essentially, that inside the group everything
    is painted as if the ``opacity'' of a line or area were just
    another color channel. In particular, if you paint a pixel with
    opacity $0$ inside a knockout group, this pixel becomes perfectly
    transparent immediately. In contrast, painting a pixel with
    something of opacity 0 normally has no effect.

    Not all renderers, let alone printers, will support
    this. At the time of writing, Apple's Preview will not show the
    following correctly (you should see the text \tikzname\ in the
    middle): 
\begin{codeexample}[]
\begin{tikzpicture}
  \shade [left color=red,right color=blue] (-2,-1) rectangle (2,1);
  \begin{scope}[transparency group=knockout]
    \fill [white] (-1.9,-.9) rectangle (1.9,.9);
    \node [opacity=0,font=\fontfamily{ptm}\fontsize{45}{45}\bfseries]
          {Ti\emph{k}Z};
  \end{scope}
\end{tikzpicture}
\end{codeexample}
   In the example, we first draw a large shading and then, inside the
   transparency group ``overwrite'' most of this shading by a big
   white rectangle. The interesting part is the text of the node,
   which has opacity |0|. Normally, this would mean that nothing is
   shown. However, in a knockout group, we ``paint'' the text with an
   ``opacity zero'' color. The effect is that part of the totally
   opaque white rectangle gets overwritten by a perfectly transparent
   area (namely exactly the area taken up by the pixels of the
   text). When this whole knockout group is then placed on top of the
   shading, the shading will ``shine through'' at the knocked-out
   pixels.

  \item \declare{|isolated|}|=false| A group can be isolated or
    not. By default, they are isolated, since this is typically what you
    want. For details on what isolated groups are, exactly, see
    Section~7.3.4 of the \textsc{pdf} Specification, version 1.7.
  \end{itemize}

  Note that when a transparency group is created, \tikzname\ must
  correctly determine the size of the material inside the
  group. Usually, this is no problem, but when you use things like
  |overlay| or |transform canvas|, trouble may result. In this case,
  please consult Section~\ref{section-transparency} on how to sidestep
  this problem in such cases.
\end{key}




%%% Local Variables:
%%% mode: latex
%%% TeX-master: "pgfmanual"
%%% End:

% Copyright 2008 by Mark Wibrow
%
% This file may be distributed and/or modified
%
% 1. under the LaTeX Project Public License and/or
% 2. under the GNU Free Documentation License.
%
% See the file doc/generic/pgf/licenses/LICENSE for more details.

\section{Decorated Paths}

\label{section-tikz-decorations}


\subsection{Overview}

Decorations are a general concept to make (sub)paths ``more
interesting.'' Before we have a look at the details, let us have a
look at some examples:

\begin{codeexample}[]
\begin{tikzpicture}[thick]
  \draw                                                (0,3)   -- (3,3);
  \draw[decorate,decoration=zigzag]                    (0,2.5) -- (3,2.5);
  \draw[decorate,decoration=brace]                     (0,2)   -- (3,2);
  \draw[decorate,decoration=triangles]                 (0,1.5) -- (3,1.5);
  \draw[decorate,decoration={coil,segment length=4pt}] (0,1)   -- (3,1);
  \draw[decorate,decoration={coil,aspect=0}]           (0,.5)  -- (3,.5);
  \draw[decorate,decoration={expanding waves,angle=7}] (0,0)   -- (3,0);
\end{tikzpicture}
\end{codeexample}

\begin{codeexample}[]
\begin{tikzpicture}
  \node [fill=red!20,decorate,decoration=bumps] {Hmm};
\end{tikzpicture}
\end{codeexample}

\begin{codeexample}[]
\begin{tikzpicture}
  \filldraw[fill=blue!20]                    (0,3)
  decorate [decoration=saw]             { -- (3,3) }
  decorate [decoration={coil,aspect=0}] { -- (2,1) }
  decorate [decoration=bumps]           { -| (0,3) };
\end{tikzpicture}
\end{codeexample}

\begin{codeexample}[]
\begin{tikzpicture}
  \node [fill=yellow!50, minimum height=2cm, minimum width=3cm,
         decorate, decoration={random steps,segment length=3pt,amplitude=1pt}]
    {Saved from trash};
\end{tikzpicture}
\end{codeexample}

The general idea of decorations is the following: First, you construct
a path using the usual path construction commands. The resulting path
is, in essence, a series of straight and curved lines. Instead of
directly using this path for filling or drawing, you can then specify
that it should form the basis for a decoration. In this case,
depending on which decoration you use, a new path is constructed
``along'' the path you specified. For instance, with the |zigzag|
decoration, the new path is a zigzagging line that goes along the old
path.

Let us have a look at an example: In the first picture, we see a path
that consists of a line, an arc, and a line. In the second picture,
this path has been used as the basis of a decoration.

\begin{codeexample}[]
\tikz \fill
  [fill=blue!20,draw=blue,thick] (0,0) -- (2,1) arc (90:-90:.5) -- cycle;
\end{codeexample}
\begin{codeexample}[]
\tikz \fill [decorate,decoration={zigzag}]
  [fill=blue!20,draw=blue,thick] (0,0) -- (2,1) arc (90:-90:.5) -- cycle;
\end{codeexample}

It is also possible to decorate only a subpath (the exact syntax will
be explained later in this section).
\begin{codeexample}[]
\tikz \fill [decoration={zigzag}]
  [fill=blue!20,draw=blue,thick] (0,0) -- (2,1)
    decorate { arc (90:-90:.5) } -- cycle;
\end{codeexample}

The |zigzag| decoration will be called a \emph{path
  morphing} decoration because it morphs a path into a different, but
topologically equivalent path. Not all decorations are path
morphing; rather there are three kinds of decorations.


\begin{enumerate}
\item The just-mentioned \emph{path morphing} decorations morph the
  path in the sense that what used to be a straight  line might
  afterwards be a squiggly line or might have bumps. However, a line
  is still and a line and path deforming decorations do not change
  the number of subpaths. 

  Examples of such decorations are the |snake| or the |zigzag|
  decoration. Many such decorations are defined in the library
  |decorations.pathmorphing|.
  
\item \emph{Path replacing} decorations completely replace the
  path by a different path that is only ``loosely based'' on the
  original path. For instance, the |crosses| decoration replaces a path
  by a path consisting of a sequence of crosses. Note how in the
  following example filling the path has no effect since the path
  consist only of (numerous) unconnected straight line subpaths:
\begin{codeexample}[]
\tikz \fill [decorate,decoration={crosses}]
  [fill=blue!20,draw=blue,thick] (0,0) -- (2,1) arc (90:-90:.5) -- cycle;
\end{codeexample}

  Examples of path replacing decorations are |crosses| or |ticks| or
  |shape backgrounds|. Such decorations are defined in the library
  |decorations.pathreplacing|, but also in |decorations.shapes|.
  
\item \emph{Path removing} decorations completely remove the
  to-be-decorated path. Thus, they have no effect on the main path
  that is being constructed. Instead, they typically have numerous
  \emph{side  effects}. For instance, they might ``write some text''
  along the (removed) path or they might place nodes along this
  path. Note that for such decorations the path usage command for the
  main path have no influence on how the decoration looks like.

\begin{codeexample}[]
\tikz \fill [decorate,decoration={text along path,
               text=This is a text along a path. Note how the path is lost.}]
  [fill=blue!20,draw=blue,thick] (0,0) -- (2,1) arc (90:-90:.5) -- cycle;
\end{codeexample}
\end{enumerate}

Decorations are defined in different decoration libraries, see
Section~\ref{section-library-decorations} for details. It is also
possible to define your own decorations, see
Section~\ref{section-base-decorations}, but you need to use the
\pgfname\ basic layer and a bit of theory is involved.

Decorations can be used to decorate already decorated paths. In the
following three graphics, we start with a simple path, then decorate
it once, and then decorate the decorated path once more.

\begin{codeexample}[]
\tikz \fill [fill=blue!20,draw=blue,thick]
  (0,0) rectangle (3,2);
\end{codeexample}
\begin{codeexample}[]
\tikz \fill [fill=blue!20,draw=blue,thick]
  decorate[decoration={zigzag,segment length=10mm,amplitude=2.5mm}]
    { (0,0) rectangle (3,2) };
\end{codeexample}
\begin{codeexample}[]
\tikz \fill [fill=blue!20,draw=blue,thick]
  decorate[decoration={crosses,segment length=2mm}] {
    decorate[decoration={zigzag,segment length=10mm,amplitude=2.5mm}] {
      (0,0) rectangle (3,2) 
    }
  };
\end{codeexample}

One final word of warning: Decorations can be pretty slow to
typeset and they can be inaccurate. The reason is that \pgfname\ has
to a \emph{lot} of rather difficult computations in the background and
\TeX\ is not very good at doing math. Decorations are fastest when
applied to straight line segments, but even then they are much slower
than other alternative. For instance, the |ticks| decoration can be
simulated by clever use of a dashing pattern and the dashing pattern
will literally be thousands of times faster to typeset. However, for
most decorations there are no real alternatives.

\begin{tikzlibrary}{decorations}
  In order to use decorations, you first have to load a decoration
  library. This |decoration| library defines the basic options
  described in the following, but it does not define any new
  decorations. This is done by libraries like
  |decorations.text|. Since these more specialized libraries include
  the |decoration| library automatically, you usually do not have to
  bother about it.
\end{tikzlibrary}



\subsection{Decorating a Subpath Using the Decorate Path Command}

The most general way to decorate a (sub)path is the following path
command.

\begin{pathoperation}{decorate}{\opt{\oarg{options}}\marg{subpath}}
  This path operation causes the \meta{subpath} to be
  decorated using the current decoration. Depending on the decoration,
  this may or may not extend the current path.
\begin{codeexample}[]
\begin{tikzpicture}
  \draw [help lines] grid (3,2);
  \draw decorate [decoration={name=zigzag}]
         { (0,0) .. controls (0,2) and (3,0) .. (3,2) |- (0,0) };
\end{tikzpicture}
\end{codeexample}
  The path can include straight lines, curves,
  rectangles, arcs, circles, ellipses, and even already decorated
  paths (that is, you can nest applications of the |decorate| path
  command, see below).

  Closed subpaths (like  rectangles or circles) may not be decorated
  succesfully with ``continuous'' decorations (those that do not
  create multiple segmented subpaths). In addition, due to the limits
  on the precision in  \TeX, some inaccuraces in positioning when
  crossing subpath boundaries may occasionally be found.

  You can use nodes normally inside the \meta{subpath}.
\begin{codeexample}[]
\begin{tikzpicture}
  \draw [help lines] grid (3,2);
  \draw decorate [decoration={name=zigzag}]
    { (0,0) -- (2,2) node (hi) [left,draw=red] {Hi!} arc(90:0:1)};

  \draw [blue] decorate [decoration={crosses}] {(3,0) -- (hi)};
\end{tikzpicture}
\end{codeexample}
  
  The following key is used to select the decoration and also to
  select further ``rendering options'' for the decoration.

  \begin{key}{/pgf/decoration=\meta{decoration options}}
    \keyalias{tikz}
    This option is used to specify which decoration is used and how it
    will look like. Note that his key will \emph{not} cause any
    decorations to be applied, immediately. It takes the |decorate| path
    command or the |decorate| option to actually decorate a path. The
    |decoration| option is only used to specify which decoration should
    be used, in principle. You can also use this option at the
    beginning of a picture or a scope to specify the decoration to be
    used with each invocation of the |decorate| path
    command. Naturally, any local options of the |decorate| path
    command override these ``global'' options.
\begin{codeexample}[]
\begin{tikzpicture}[decoration=zigzag]
  \draw       decorate                      {(0,0) -- (3,2)};
  \draw [red] decorate [decoration=crosses] {(0,2) -- (3,0)};
\end{tikzpicture}
\end{codeexample}
    
    The \meta{decoration options} are special options
    (which have the path prefix |/pgf/decoration/|) that determine the
    properties of the decoration. Which options are appropriate for a
    decoration depend strongly on the decoration, you will have to look
    up the appropriate options in the documentation of the decoration,
    see Section~\ref{section-library-decorations}.
    
    There is one option (available only in \tikzname) that is special:
    \begin{key}{/pgf/decoration/name=\meta{name} (initially none)}
      Use this key to set which decoration is to be used. The
      \meta{name} can both be a decoration or a meta-decoration (you
      need to worry about the difference only if you wish to define
      your own decorations).
      
      If you set \meta{name} to |none|, no decorations are added.
\begin{codeexample}[]
\begin{tikzpicture}
  \draw [help lines] grid (3,2);
  \draw decorate [decoration={name=zigzag}]
         { (0,0) .. controls (0,2) and (3,0) .. (3,2) };
\end{tikzpicture}
\end{codeexample}
      Since this option is used so often, you can also leave out the
      |name=| part. Thus, the above example can be rewritten more
      succinctly: 
\begin{codeexample}[]
\begin{tikzpicture}
  \draw [help lines] grid (3,2);
  \draw decorate [decoration=zigzag]
         { (0,0) .. controls (0,2) and (3,0) .. (3,2) };
\end{tikzpicture}
\end{codeexample}
      In general, when \meta{decoration options} are parsed, for each
      unknown key it is checked whether that key happens to be a
      (meta-)decoration and, if so, the |name| option is executed for
      this key.
    \end{key}

    Further options allow you to adjust the position of decorations
    relative to the to-be-decorated path. See
    Section~\ref{section-decorations-adjust} below for details.
  \end{key}

  Recall that some decorations actually completely remove the
  to-be-decorated path. In such cases, the construction of the main
  path is resumed after the |decorate| path command ends.
  
\begin{codeexample}[]
\begin{tikzpicture}[decoration={text along path,text=
      around and around and around and around we go}]

  \draw (0,0) -- (1,1) decorate { -- (2,1) } -- (3,0);
\end{tikzpicture}
\end{codeexample}

  It is permissible to nest |decorate| commands. In this case, the
  path resulting from the first decoration process is used as the
  to-be-decorated path for the second decoration process. This is
  especially useful for drawing fractals. The |Koch snowflake|
  decoration replaces a straight line like \tikz\draw (0,0) -- (1,0);
  by \tikz[decoration=Koch snowflake] \draw decorate{(0,0) --
    (1,0)};. Repeatedly applying this transformation to a triangle
  yields a fractal that looks a bit like a snowflake, hence the name. 
\begin{codeexample}[]
\begin{tikzpicture}[decoration=Koch snowflake,draw=blue,fill=blue!20,thick]
  \filldraw (0,0) -- ++(60:1) -- ++(-60:1) -- cycle ;
  \filldraw decorate{ (0,-1) -- ++(60:1) -- ++(-60:1) -- cycle };
  \filldraw decorate{ decorate{ (0,-2.5) -- ++(60:1) -- ++(-60:1) -- cycle }};
\end{tikzpicture}
\end{codeexample}
\end{pathoperation}



\subsection{Decorating a Complete Path}

You may sometimes wish to decorate a path over whose construction you
have no control. For instance, the path of the background of a node is
created without your having a chance to issue a |decorate| path
command. In such cases you can use the following option, which allows
you to decorate a path ``after the fact.''

\begin{key}{/tikz/decorate=\opt{\meta{boolean}} (default true)}
  When this key is set, the whole path is decorated after it has been
  finished. The decoration used for decorating the path is set via the
  |decoration| way, in exactly the same way as for the |decorate| path
  command. Indeed, the following two commands have the same effect:
  \begin{enumerate}
  \item |\path decorate[|\meta{options}|] {|\meta{path}|};|
  \item |\path [decorate,|\meta{options}|] |\meta{path}|;|
  \end{enumerate}
  The main use or the |decorate| option is the you can also use it
  with the nodes. It then causes the background path of the node to be
  decorated. Note that you decorate a background path only once in
  this manner. That is, in contrast to the |decorate| path command you
  cannot apply this option twice (this would just set it to |true|,
  once more).

\begin{codeexample}[]
\begin{tikzpicture}[decoration=zigzag]
  \draw [help lines] (0,0) grid (3,5);
  
  \draw [fill=blue!20,decorate] (1.5,4) circle (1cm);

  \node at (1.5,2.5) [fill=red!20,decorate,ellipse] {Ellipse};

  \node at (1.5,1) [inner sep=6mm,fill=red!20,decorate,ellipse,decoration=
    {text along path,text={This is getting silly}}] {Ellipse};
\end{tikzpicture}
\end{codeexample}

  In the last example, the |text along path| decoration removes the
  path. In such cases it is useful to use a pre- or postaction to
  cause the decoration to be applied only before or after the main
  path has been used. Incidentally, this is another application of the
  |decorate| option that you cannot achieve with the decorate path
  command. 
\begin{codeexample}[]
\begin{tikzpicture}[decoration=zigzag]
  \node at (1.5,1) [inner sep=6mm,fill=red!20,ellipse,
    postaction={decorate,decoration=
    {text along path,text={This is getting silly}}}] {Ellipse};
\end{tikzpicture}
\end{codeexample}
  Here is more useful example, where a postaction is used to add the
  path after the main path has been drawn.
\begin{codeexample}[]
\catcode`\|12
\begin{tikzpicture}
\draw [help lines] grid (3,2);
\fill [draw=red,fill=red!20,
         postaction={decorate,decoration={raise=2pt,text along path,
           text=around and around and around and around we go}}] 
  (0,1) arc (180:-180:1.5cm and 1cm);
\end{tikzpicture}
\end{codeexample} 
\end{key}


\subsection{Adjusting Decorations}

\label{section-decorations-adjust}

\subsubsection{Positioning Decorations Relative to the To-Be-Decorate Path}

The following option, which are only available with \tikzname, allow
you to modify the positioning of decorations relative to the
to-be-decorated path.

\begin{key}{/pgf/decoration/raise=\meta{dimension} (initially 0pt)}
  The segments of the decoration are raised by \meta{dimension}
  relative to the to-be-decorated path. More precisely, the segments
  of the path are offset by this much ``to the left'' of the path as
  we travel along the path. This raising is done after and in addition
  to any transformations set using the |transform| option (see below).

  A negative \meta{dimension} will offset the decoration ``to the
  right'' of the to-be-decorated path.
\begin{codeexample}[]
\begin{tikzpicture}
  \draw [help lines] (0,0) grid (3,2);

  \draw (0,0) -- (1,1) arc (90:0:2 and 1);
  \draw      decorate [decoration=crosses]
        { (0,0) -- (1,1) arc (90:0:2 and 1) };
  \draw[red] decorate [decoration={crosses,raise=5pt}]
        { (0,0) -- (1,1) arc (90:0:2 and 1) };
\end{tikzpicture}
\end{codeexample}
\end{key}

\begin{key}{/pgf/decoration/mirror=\opt{\meta{boolean}}}
  Causes the segments of the decoration to be mirrored along the
  to-be-decorated path. This is done after and in addition to any
  transformations set using the |transform| and/or |raise| options.
\begin{codeexample}[]
\begin{tikzpicture}
  \node (a)          {A};
  \node (b) at (2,1) {B};
  \draw                                                    (a) -- (b);
  \draw[decorate,decoration=brace]                         (a) -- (b);
  \draw[decorate,decoration={brace,mirror},red]            (a) -- (b);
  \draw[decorate,decoration={brace,mirror,raise=5pt},blue] (a) -- (b);
\end{tikzpicture}
\end{codeexample}
\end{key}


\begin{key}{/pgf/decoration/transform=\meta{transformations}}
  This key allows you to specify general \meta{transformations} to be
  applied to the segments of a decoration. These transformations are
  applied before and independently of |raise| and |mirror|
  transformations. The \meta{transformations} should be normal
  \tikzname\ transformations like |shift| or |rotate|.

  In the following example the |shift only| transformation is used to
  make sure that the crosses are \emph{not} sloped along the path.
\begin{codeexample}[]
\begin{tikzpicture}
  \draw [help lines] (0,0) grid (3,2);

  \draw (0,0) -- (1,1) arc (90:0:2 and 1);
  \draw[red,very thick] decorate [decoration={
               crosses,transform={shift only},shape size=1.5mm}]
        { (0,0) -- (1,1) arc (90:0:2 and 1) };
\end{tikzpicture}
\end{codeexample}
\end{key}


\subsubsection{Starting and Ending Decorations Early or Late}

You sometimes may wish to ``end'' a decoration a bit early on the
path. For instance, you might wish a |snake| decoration to stop 5mm
before the end of the path and to continue in a straight line. There
are different ways of achieving this effect, but the easiest may be
the |pre| and |post| options, which only have an effect in
\tikzname. Note, however, that they can only be used with decorations,
not with meta-decorations.

\begin{key}{/pgf/decoration/pre=\meta{decoration} (initially lineto)}
  This key sets a decoration that should be used before the main
  decoration starts. The \meta{decoration} will be used for a length
  of |pre length|, which |0pt| by default. Thus, for the |pre| option
  to have any effect, you also need to set the |pre length| option.
\begin{codeexample}[]
\begin{tikzpicture}
\tikz [decoration={zigzag,pre=lineto,pre length=1cm}]
  \draw [decorate] (0,0) -- (2,1) arc (90:0:1);
\end{tikzpicture}
\end{codeexample}
\begin{codeexample}[]
\begin{tikzpicture}
\tikz [decoration={zigzag,pre=moveto,pre length=1cm}]
  \draw [decorate] (0,0) -- (2,1) arc (90:0:1);
\end{tikzpicture}
\end{codeexample}
\begin{codeexample}[]
\begin{tikzpicture}
\tikz [decoration={zigzag,pre=crosses,pre length=1cm}]
  \draw [decorate] (0,0) -- (2,1) arc (90:0:1);
\end{tikzpicture}
\end{codeexample}

  Note that the default |pre| option is |lineto|, not |curveto|. This
  means that the default |pre| decoration will not follow curves (for
  efficiency reasons). Change the |pre| key to |curveto| if you have a
  curved path. 
\begin{codeexample}[]
\begin{tikzpicture}
\tikz [decoration={zigzag,pre length=3cm}]
  \draw [decorate] (0,0) -- (2,1) arc (90:0:1);
\end{tikzpicture}
\end{codeexample}
\begin{codeexample}[]
\begin{tikzpicture}
\tikz [decoration={zigzag,pre=curveto,pre length=3cm}]
  \draw [decorate] (0,0) -- (2,1) arc (90:0:1);
\end{tikzpicture}
\end{codeexample}
\end{key}

\begin{key}{/pgf/decoration/pre length=\meta{dimension} (initially 0pt)}
  This key sets the distance along which the pre-decoration should be
  used. If you do not need/wish a pre-decoration, set this key to
  |0pt| (exactly this string, not just to something that evaluated to
  the same things such as |0cm|).
\end{key}

\begin{key}{/pgf/decorations/post=\meta{decoration} (initially
    lineto)}
  Works like |pre|, only for the end of the decoration.  
\end{key}

\begin{key}{/pgf/decorations/post length=\meta{dimension} (initially
    0pt)}
  Works like |pre length|, only for the end of the decoration.  
\end{key}

Here is a typical example that shows how these keys can be used:

\begin{codeexample}[]
\begin{tikzpicture}
  [decoration=snake,
   line around/.style={decoration={pre length=#1,post length=#1}}]

  \draw[->,decorate]                  (0,0)    -- ++(3,0);
  \draw[->,decorate,line around=5pt]  (0,-5mm) -- ++(3,0);
  \draw[->,decorate,line around=1cm]  (0,-1cm) -- ++(3,0);
\end{tikzpicture}
\end{codeexample}



\endinput
% Copyright 2006 by Till Tantau
%
% This file may be distributed and/or modified
%
% 1. under the LaTeX Project Public License and/or
% 2. under the GNU Free Documentation License.
%
% See the file doc/generic/pgf/licenses/LICENSE for more details.

\section{Transformations}

\pgfname\ has a powerful transformation mechanism that is similar to
the transformation capabilities of \textsc{metafont}. The present
section explains how you can access it in \tikzname.


\subsection{The Different Coordinate Systems}

It is a long process from  a coordinate like, say, $(1,2)$ or
$(1\mathrm{cm},5\,mathrm{pt})$, to the position a point is finally
placed on the display or paper. In order to find out where the point
should go, it is constantly ``transformed,'' which means that it is
mostly shifted around and possibly rotated, slanted, scaled, and
otherwise mutilated. 

In detail, (at least) the following transformations are applied to a
coordinate like $(1,2)$ before a point on the screen is chosen:
\begin{enumerate}
\item
  \pgfname\ interprets a coordinate like $(1,2)$  in its
  $xy$-coordinate system as ``add the current $x$-vector once and the
  current $y$-vector twice to obtain the new point.''
\item
  \pgfname\ applies its coordinate transformation matrix to the
  resulting coordinate. This yields the final position of the point 
  inside the picture.
\item
  The backend driver (like |dvips| or |pdftex|) adds transformation
  commands such the coordinate is shifted to the correct position in
  \TeX's page coordinate system.
\item
  \textsc{pdf} (or PostScript) apply the canvas transformation
  matrix to the point, which can once more change the position on the
  page. 
\item
  The viewer application or the printer applies the device
  transformation matrix to transform the coordinate to its final pixel
  coordinate on the screen or paper.  
\end{enumerate}

In reality, the process is even more involved, but the above should
give the idea: A point is constantly transformed by changes of the
coordinate system.

In \tikzname, you only have access to the first two coordinate systems:
The $xy$-coordinate system and the coordinate transformation matrix
(these will be explained later). \pgfname\ also allows you to change
the canvas transformation matrix, but you have to use commands of
the core layer directly to do so and you ``better know what you are
doing'' when you do this. The moment you start modifying the
canvas matrix, \pgfname\ immediately looses track of all
coordinates and shapes, anchors, and bounding box computations will no
longer work.


\subsection{The XY- and XYZ-Coordinate Systems}
\label{section-xyz}

The first and easiest coordinate systems are \pgfname's $xy$- and
$xyz$-coordinate systems. The idea is very simple: Whenever you
specify a coordinate like |(2,3)| this means $2v_x + 3v_y$, where
$v_x$ is the current \emph{$x$-vector} and $v_y$ is the current
\emph{$y$-vector}. Similarly, the coordinate |(1,2,3)| means $v_x +
2v_y + 3v_z$.

Unlike other packages, \pgfname\ does not insist that $v_x$ actually
has a $y$-component of $0$, that is, that it is a horizontal
vector. Instead, the $x$-vector can point anywhere you
want. Naturally, \emph{normally} you will want the $x$-vector to point
horizontally.

One undesirable effect of this flexibility is that it is not possible
to provide mixed coordinates as in $(1,2\mathrm{pt})$. Life is hard.

To change the $x$-, $y$-, and $z$-vectors, you can use the following
options:

\begin{itemize}
\itemoption{x}|=|\meta{dimension}
  Sets the $x$-vector of \pgfname's $xyz$-coordinate system to point
  \meta{dimension} to the right, that is, to
  $(\meta{dimension},0pt)$. The default is 1cm.

\begin{codeexample}[]
\begin{tikzpicture}
  \draw                  (0,0)   -- +(1,0);
  \draw[x=2cm,color=red] (0,0.1) -- +(1,0);
\end{tikzpicture}
\end{codeexample}    

\begin{codeexample}[]
\tikz \draw[x=1.5cm] (0,0) grid (2,2);
\end{codeexample}    

The last example shows that the size of steppings in grids, just like
all other dimensions, are not affected by the $x$-vector. After all,
the $x$-vector is only used to determine the coordinate of the upper
right corner of the grid.
\itemoption{x}|=|\meta{coordinate}
  Sets the $x$-vector of \pgfname's $xyz$-coordinate system to the
  specified \meta{coordinate}. If \meta{coordinate} contains a comma,
  it must be put in braces. 

\begin{codeexample}[]
\begin{tikzpicture}
  \draw                            (0,0) -- (1,0);
  \draw[x={(2cm,0.5cm)},color=red] (0,0) -- (1,0);
\end{tikzpicture}
\end{codeexample}

  You can use this, for example, to exchange the meaning of the $x$- and
  $y$-coordinate.

\begin{codeexample}[]
\begin{tikzpicture}[smooth]
  \draw plot coordinates{(1,0) (2,0.5) (3,0) (3,1)};
  \draw[x={(0cm,1cm)},y={(1cm,0cm)},color=red]
        plot coordinates{(1,0) (2,0.5) (3,0) (3,1)};
\end{tikzpicture}
\end{codeexample}

\itemoption{y}|=|\meta{value}
  Works like the |x=| option, only if \meta{value} is a dimension, the
  resulting vector points to $(0,\meta{value})$.
\itemoption{z}|=|\meta{value}
  Works like the |z=| option, but now a dimension is means the point
  $(\meta{value},\meta{value})$.

\begin{codeexample}[]
\begin{tikzpicture}[z=-1cm,->,thick]
  \draw[color=red] (0,0,0) -- (1,0,0);
  \draw[color=blue] (0,0,0) -- (0,1,0);
  \draw[color=orange] (0,0,0) -- (0,0,1);
\end{tikzpicture}
\end{codeexample}
\end{itemize}



\subsection{Coordinate Transformations}

\pgfname\ and \tikzname\ allow you to specify \emph{coordinate
  transformations}. Whenever you specify a coordinate as in |(1,0)| or
|(1cm,1pt)| or |(30:2cm)|, this coordinate is first
``reduced'' to a position of the form ``$x$ points to the right and
  $y$ points upwards.'' For example, |(1in,5pt)| is reduced to
``$72\frac{72}{100}$ points to the right and 5 points upwards'' and
|(90:100pt)| means ``0pt to the right and 100 points upwards.''

The next step is to apply the current \emph{coordinate transformation
  matrix} to the coordinate. For example, the coordinate
transformation matrix might currently be set such that it adds a
certain constant to the $x$ value. Also, it might be setup such that
it, say, exchanges the $x$ and $y$ value. In general, any
``standard'' transformation like translation, rotation, slanting, or
scaling or any combination thereof is possible. (Internally, \pgfname\
keeps track of a coordinate transformation matrix very much like the
concatenation matrix used by \textsc{pdf} or PostScript.)

\begin{codeexample}[]
\begin{tikzpicture}
  \draw[style=help lines] (0,0) grid (3,2);
  \draw (0,0) rectangle (1,0.5);
  \begin{scope}[xshift=1cm]
    \draw             [red]    (0,0) rectangle (1,0.5);
    \draw[yshift=1cm] [blue]   (0,0) rectangle (1,0.5);
    \draw[rotate=30]  [orange] (0,0) rectangle (1,0.5);
  \end{scope}
\end{tikzpicture}
\end{codeexample}

The most important aspect of the coordinate transformation matrix is
\emph{that it applies to coordinates only!} In particular, the
coordinate transformation has no effect on things like the line width
or the dash pattern or the shading angle. In certain cases, it is not
immediately clear whether the coordinate transformation matrix
\emph{should} apply to a certain dimension. For example, should the
coordinate transformation matrix apply to grids? (It does.) And what
about the size of arced corners? (It does not.) The general rule is
``If there is no `coordinate' involved, even `indirectly,' the matrix
is not applied.'' However, sometimes, you simply have to try or look
it up in the documentation whether the matrix will be applied.

Setting the matrix cannot be done directly. Rather, all you can do is
to ``add'' another transformation to the current matrix. However, all
transformations are local to the current \TeX-group. All
transformations are added using graphic options, which are described
below.

Transformations apply immediately when they are encountered ``in the
middle of a path'' and they apply only to the coordinates on the path
following the transformation option. 

\begin{codeexample}[]
\tikz \draw (0,0) rectangle (1,0.5) [xshift=2cm] (0,0) rectangle (1,0.5);
\end{codeexample}

A final word of warning: You should refrain from using ``aggressive''
transformations like a scaling of a factor of 10000. The reason is
that all transformations are done using \TeX, which has a fairly low
accuracy. Furthermore, in certain situations it is necessary that
\tikzname\ \emph{inverts} the current transformation matrix and this will
fail if the transformation matrix is badly conditioned or even
singular (if you do not know what singular matrices are, you are blessed).   

\begin{itemize}
  \itemoption{shift}|={|\meta{coordinate}|}|
  adds the  \meta{coordinate} to all coordinates.
\begin{codeexample}[]
\begin{tikzpicture}
  \draw[style=help lines] (0,0) grid (3,2);
  \draw                       (0,0) -- (1,1) -- (1,0);
  \draw[shift={(1,1)},blue]   (0,0) -- (1,1) -- (1,0);
  \draw[shift={(30:1cm)},red] (0,0) -- (1,1) -- (1,0);
\end{tikzpicture}
\end{codeexample}

  \itemoption{shift only}
  This option does not take any parameter. Its effect is to cancel all
  current transformations except for the shifting. This means that the
  origin will remain where it is, but any rotation around the origin
  or scaling relative to the origin or skewing will no longer have an
  effect.

  This option is useful in situtations where a complicated
  transformation is used to ``get to a position,'' but you then wish
  to draw something ``normal'' at this position. 

\begin{codeexample}[]
\begin{tikzpicture}
  \draw[style=help lines] (0,0) grid (3,2);
  \draw                                      (0,0) -- (1,1) -- (1,0);
  \draw[rotate=30,xshift=2cm,blue]           (0,0) -- (1,1) -- (1,0);
  \draw[rotate=30,xshift=2cm,shift only,red] (0,0) -- (1,1) -- (1,0);
\end{tikzpicture}
\end{codeexample}

  \itemoption{xshift}|=|\meta{dimension}
  adds \meta{dimension} to the $x$ value of all coordinates.  
\begin{codeexample}[]
\begin{tikzpicture}
  \draw[style=help lines] (0,0) grid (3,2);
  \draw                   (0,0) -- (1,1) -- (1,0);
  \draw[xshift=2cm,blue]  (0,0) -- (1,1) -- (1,0);
  \draw[xshift=-10pt,red] (0,0) -- (1,1) -- (1,0);
\end{tikzpicture}
\end{codeexample}

  \itemoption{yshift}|=|\meta{dimension}
  adds \meta{dimension} to the $y$ value of all coordinates.
  
  \itemoption{scale}|=|\meta{factor}
  multiplies all coordinates by the given \meta{factor}. The
  \meta{factor} should not be excessively large in absolute terms or
  very near to zero.
\begin{codeexample}[]
\begin{tikzpicture}
  \draw[style=help lines] (0,0) grid (3,2);
  \draw               (0,0) -- (1,1) -- (1,0);
  \draw[scale=2,blue] (0,0) -- (1,1) -- (1,0);
  \draw[scale=-1,red] (0,0) -- (1,1) -- (1,0);
\end{tikzpicture}
\end{codeexample}

  \itemoption{xscale}|=|\meta{factor}
  multiplies only the $x$-value of all coordinates by the given
  \meta{factor}. 
\begin{codeexample}[]
\begin{tikzpicture}
  \draw[style=help lines] (0,0) grid (3,2);
  \draw                (0,0) -- (1,1) -- (1,0);
  \draw[xscale=2,blue] (0,0) -- (1,1) -- (1,0);
  \draw[xscale=-1,red] (0,0) -- (1,1) -- (1,0);
\end{tikzpicture}
\end{codeexample}

  \itemoption{yscale}|=|\meta{factor}
  multiplies only the $y$-value of all coordinates by \meta{factor}.
 
  \itemoption{xslant}|=|\meta{factor}
  slants the coordinate horizontally by the given \meta{factor}:
\begin{codeexample}[]
\begin{tikzpicture}
  \draw[style=help lines] (0,0) grid (3,2);
  \draw                (0,0) -- (1,1) -- (1,0);
  \draw[xslant=2,blue] (0,0) -- (1,1) -- (1,0);
  \draw[xslant=-1,red] (0,0) -- (1,1) -- (1,0);
\end{tikzpicture}
\end{codeexample}

  \itemoption{yslant}|=|\meta{factor}
  slants the coordinate vertically by the given \meta{factor}:
\begin{codeexample}[]
\begin{tikzpicture}
  \draw[style=help lines] (0,0) grid (3,2);
  \draw                (0,0) -- (1,1) -- (1,0);
  \draw[yslant=2,blue] (0,0) -- (1,1) -- (1,0);
  \draw[yslant=-1,red] (0,0) -- (1,1) -- (1,0);
\end{tikzpicture}
\end{codeexample}

  \itemoption{rotate}|=|\meta{degree}
  rotates the coordinate system by \meta{degree}:
\begin{codeexample}[]
\begin{tikzpicture}
  \draw[style=help lines] (0,0) grid (3,2);
  \draw                 (0,0) -- (1,1) -- (1,0);
  \draw[rotate=40,blue] (0,0) -- (1,1) -- (1,0);
  \draw[rotate=-20,red] (0,0) -- (1,1) -- (1,0);
\end{tikzpicture}
\end{codeexample}

  \itemoption{rotate around}|={|\meta{degree}|:|\meta{coordinate}|}|
  rotates the coordinate system by \meta{degree} around the point
  \meta{coordinate}.
\begin{codeexample}[]
\begin{tikzpicture}
  \draw[style=help lines] (0,0) grid (3,2);
  \draw                                (0,0) -- (1,1) -- (1,0);
  \draw[rotate around={40:(1,1)},blue] (0,0) -- (1,1) -- (1,0);
  \draw[rotate around={-20:(1,1)},red] (0,0) -- (1,1) -- (1,0);
\end{tikzpicture}
\end{codeexample}

  \itemoption{cm}|={|\meta{$a$}|,|\meta{$b$}|,|\meta{$c$}|,|\meta{$d$}|,|\meta{coordinate}|}|
  applies the following transformation to all coordinates: Let $(x,y)$
  be the coordinate to be transformed and let \meta{coordinate}
  specify the point $(t_x,t_y)$. Then the new coordinate is given by
  $\left(\begin{smallmatrix} a & b \\ c & d\end{smallmatrix}\right)
  \left(\begin{smallmatrix} x \\ y \end{smallmatrix}\right) +
  \left(\begin{smallmatrix} t_x \\ t_y
  \end{smallmatrix}\right)$. Usually, you do not use this option
  directly. 
\begin{codeexample}[]
\begin{tikzpicture}
  \draw[style=help lines] (0,0) grid (3,2);
  \draw                             (0,0) -- (1,1) -- (1,0);
  \draw[cm={1,1,0,1,(0,0)},blue]    (0,0) -- (1,1) -- (1,0);
  \draw[cm={0,1,1,0,(1cm,1cm)},red] (0,0) -- (1,1) -- (1,0);
\end{tikzpicture}
\end{codeexample}

  \itemoption{reset cm}
  completely resets the coordinate transformation matrix to the
  identity matrix. This will destroy not only the transformations
  applied in the current scope, but also all transformations inherited
  from surrounding scopes. Do not use this option, unless you really,
  really know what you are doing.
\end{itemize}





\part{Graph Drawing}

{\Large \emph{by Till Tantau et al.}}

\bigskip
\noindent
\emph{Graph drawing algorithms} do the tough work of computing a
layout of a graph for you. \tikzname\ comes with powerful
such algorithms, but you can also implement new algorithms in the
Lua programming language.
\vskip1cm

\ifluatex
\begin{codeexample}[graphic=white]
\tikz [nodes={text height=.7em, text depth=.2em,
              draw=black!20, thick, fill=white, font=\footnotesize},
       >=spaced stealth', rounded corners, semithick]
  \graph [layered layout, sibling distance=1.25cm] {
    "5th Edition" -> { "6th Edition", "PWB 1.0" };
    "6th Edition" -> { "LSX" [>child anchor=45],  "1 BSD", "Mini Unix", "Wollongong", "Interdata" };
    "Interdata" -> { "Unix/TS 3.0", "PWB 2.0", "7th Edition" };
    "7th Edition" -> { "8th Edition", "32V", "V7M", "Ultrix-11", "Xenix", "UniPlus+" };
    "V7M" -> "Ultrix-11";
    "8th Edition" -> "9th Edition";
    "1 BSD" -> "2 BSD" -> "2.8 BSD" -> { "Ultrix-11", "2.9 BSD" };
    "32V" -> "3 BSD" -> "4 BSD" -> "4.1 BSD" -> { "4.2 BSD", "2.8 BSD", "8th Edition" };
    "4.2 BSD" -> { "4.3 BSD", "Ultrix-32" };
    "PWB 1.0" -> { "PWB 1.2" -> "PWB 2.0", "USG 1.0" -> { "CB Unix 1", "USG 2.0" }};
    "CB Unix 1" -> "CB Unix 2" -> "CB Unix 3" -> { "Unix/TS++", "PDP-11 Sys V" };
    { "USG 2.0" -> "USG 3.0", "PWB 2.0", "Unix/TS 1.0" } -> "Unix/TS 3.0";
    { "Unix/TS++", "CB Unix 3", "Unix/TS 3.0" } -> "TS 4.0" -> "System V.0" -> "System V.2" -> "System V.3";
  };  
\end{codeexample}

% Copyright 2010 by Renée Ahrens, Olof Frahm, Jens Kluttig, Matthias Schulz, Stephan Schuster
% Copyright 2011 by Till Tantau
% Copyright 2011 by Jannis Pohlmann
%
% This file may be distributed and/or modified
%
% 1. under the LaTeX Project Public License and/or
% 2. under the GNU Free Documentation License.
%
% See the file doc/generic/pgf/licenses/LICENSE for more details.

\section{Using Algorithmic Graph Drawing}

{\noindent {\emph{by Till Tantau and Ren\'ee Ahrens, Olof-Joachim
      Frahm, Jens Kluttig, Jannis Pohlmann, Matthias Schulz, Stephan
      Schuster}}} 

\label{section-library-graphdrawing}

\begin{tikzlibrary}{graphdrawing}
  This package provides capabilities for automatic graph drawing.

  \medskip
  \textbf{Note:} Graph drawing requires that the document is typeset
  using Lua\TeX. This package should work with \LuaTeX\ 0.4 or
  higher, which is included in all current \TeX\ distributions.
\end{tikzlibrary}

\ifluatex\relax\else{LuaTeX is required for setting this manual
  section.}\expandafter\endinput\fi 


\subsection{Overview}

\emph{Algorithmic graph drawing} (or just \emph{graph drawing} in the
following) means that algorithms are used to decide where the nodes of
a graph are positioned on a page so that the graph ``looks nice.'' The
idea is that you, as human (or you, as a machine, if you happen to be
a machine and happen to be reading this document) just specify which
nodes are present in a graph and which edges are
present. Additionally, you may add some ``hints'' like ``this node
should be near the center'' or ``this edge is pretty important.'' You
do \emph{not} specify where, exactly, the nodes and edges should
be. This is something you leave to a \emph{graph drawing
  algorithm}. The algorithm gets your description of the graph as an
input and then decides where the nodes should go on the page.

Naturally, graph drawing is a bit of a (black?) art. There is no
``perfect'' way of drawing a graph, rather, depending on the
circumstances there are several different ways of drawing the same
graph and often it will just depend on the aesthetic sense of the
reader which layout he or she would prefer. For this reason, there is
a huge number of graph drawing algorithms ``out there'' and there are
scientific conference devoted to such algorithms, where each
year dozens of new algorithms are proposed.

Unlike the rest of \pgfname\ and \tikzname, which is implemented
purely in \TeX, the graph drawing algorithms are simply too complex to
implement them in \TeX. Instead, the programming language Lua is used
by the graph drawing library -- a programming language that has been
integrated into recent versions of \TeX. This means that (a) as a user
of the graph drawing engine you will can run \TeX\ on your documents
in the usual way, no external programs are called since Lua is already
integrated into \TeX\ and (b) it is pretty easy to implement new graph
drawing algorithms for \tikzname\ since Lua can be used and no \TeX\
programming knowledge is needed. 

The graph drawing engine of \tikzname\ provides two main features:
\begin{enumerate}
\item ``Users'' of the graph drawing engine can invoke the graph
  drawing algorithms often by just adding a single option to their
  picture. Here is a typical example, where the |layered layout| option
  tells \tikzname\ that the graph should be drawn (``should be layed
  out'') using a so-called ``layered graph drawing algorithm'' (what
  these are will be explained later):
\begin{codeexample}[]
\tikz
  \graph [layered layout, components go right top aligned, nodes={draw, rounded corners=2pt}]
  {
    first root -> {1 -> {2, 3} -> {4, 5}, 6 }, 4 -- 5;
    second root -> x -> {a -> {/,/}, b, c -> d -> {/,/} };
    third root -> child -> grandchild -> youngster -> third root;    
  };
\end{codeexample}
  Here is another example, where a different layout method is used
  that is more appropriate for trees:
\begin{codeexample}[]
\tikz [grow'=up, binary tree layout, nodes={circle,draw}]
  \node {1}
  child { node {2}
    child { node {3} }
    child { node {4}
      child { node {5} }
      child { node {6} }
    }
  }
  child { node {7}
    child { node {8}
      child[missing]
      child { node {9} }
    }
  };
\end{codeexample}
  An a final example, this time using a ``spring electrical layout''
  (whatever that might be\dots):
\begin{codeexample}[]
\tikz [spring electrical layout]
{
  \foreach \i in {1,...,6}
    \node (node \i) [fill=blue!50, text=white, circle] {\i};
    
  \draw (node 1) edge (node 2)
        (node 2) edge (node 3)
        (node 3) edge (node 4)
                 edge (node 5)
                 edge (node 6);
}
\end{codeexample}
  In all of the example, the positions of the nodes have only been
  computed \emph{after} all nodes have been created and the edges have
  been specified. For instance, in the last example, without the
  option |spring electrical layout|, all of the nodes would have been
  placed on top of each other.
\item The graph drawing engine is also intended to make is
  (relatively) easy to implement new graph drawing algorithms. These
  algorithms can and must be implemented in the Lua programming
  language (which is \emph{much} easier to program than \TeX\
  itself). The Lua code for a graph drawing algorithm gets an
  object-oriented model of the input graph as an input and must just
  compute the desired new positions of the nodes. The complete
  handling of passing options and configurations back-and-forth
  between the different \tikzname\ and \pgfname\ layers is handled by
  the graph drawing engine.

  The bottom line is that the graph drawing engine makes it easy
  to try out new graph drawing algorithms for medium sized graphs (up
  to a few hundred nodes).
\end{enumerate}

The documentation of the graph drawing engine is structured as
follows: The current section explains the graph drawing engine from
``the user's point of few'' and also describes the basic steps
necessary to implement a new graph drawing algorithm. The libraries
containing the different graph drawing algorithms are documented in
Sections on graph drawing
algorithms. Section~\ref{section-gd-own-algorithm} covers the
internals of how the graph drawing engine works. 



\subsection{Usage}

To use the graph drawing engine, you first need to load some
libraries. First, you should always load the |graphdrawing| library,
which will setup the basic keys. Next, you need to load another
library like |graphdrawing.trees|, see the following
Sections~\ref{section-first-graphdrawing-library-in-manual} to
\ref{section-last-graphdrawing-library-in-manual} for the different
libraries that are available. The actual graph drawing
algorithms reside in these libraries. Finally, you may also wish to
load the |graphs| library, but this is only necessary if you wish to
use the |graph| path command, which provides an easy-to-use syntax for
specifying graphs. You can also use the graph drawing engine
independently of the |graphs| library, for instance in conjunction
with the |child| or the |edge| syntax. Here is a typical setup:

\begin{codeexample}[code only]
\usetikzlibrary{graphs,graphdrawing,graphdrawing.trees}  
\end{codeexample}

Having setup things, you must then specify for which scopes the
graph drawing engine should apply an layout algorithm to the nodes in
the scope. Typically, you just add an option ending with |... layout|
to the |graph| path operation and then let the graph drawing do its
magic:

\begin{codeexample}[]
\tikz [rounded corners]
  \graph [layered layout, sibling distance=8mm, level distance=8mm]
  {
    a -> {
      b,
      c -> { d, e }
    } ->
    f -> 
    a
  };    
\end{codeexample}

Whenever you use such an option, (to be more precise, inside every
scope with the |graph drawing scope| key set either
explicitly or implicitly, which is exactly what happens when one such
an option is used) you can:
\begin{itemize}
\item Create nodes in the usual way. The nodes will be created
  completely, but then tucked away in an internal table. This means
  that all of \tikzname's options for nodes can be applied. You can
  also name a node and reference it later.
\item Create edges using either the syntax of the |graph| command
  (using |--|, |<-|, |->|, or |<->|), or using the |edge| command,
  or using the |child| command. These edges will, however, not be
  created immediately. Instead, the basic layer's command
  |\pgfgdedge| will be called, which stores ``all the information
  concerning the edge.'' The actual drawing of the edge will only
  happen after all nodes have been positioned.
\item Most of the keys that can be passed to an edge will work as
  expected. In particular, you can add labels to edges using the
  usual |node| syntax for edges.
\item The |label| and |pin| options can be used in the usual manner
  with nodes inside a graph drawing scope. Only, the labels and
  nodes will play no role in the positioning of the nodes and they
  are added when the nodes are finally positioned.
\item Similarly, nodes that are placed ``on an edge'' using the
  implicit positioning syntax can be used in the usual manner. 
\end{itemize}
Here are some things that will \emph{not} work:
\begin{itemize}
\item Only edges created using the graph syntax, the |edge| command,
  or the |child| command will correctly pass their connection
  information to the basic layer. When you write |\draw (a)--(b);|
  inside a graph drawing scope, where |a| and |b| are nodes that
  have been created inside the scope, you will get an error
  message / things will look wrong. The reason is that the usual
  |--| is not ``caught'' by the graph drawing engine and, thus,
  tries to immediately connect two nodes that do not yet exist
  (except inside some internal table).
\item The options of edges are executed twice: Once when the edge is
  ``examined'' by the |\pgfgdedge| command (using some magic to shield
  against the side effects) and then once more when the edge is
  actually created. Fortunately, in almost all cases, this will not be
  a problem; but if you do very evil magic inside your edge options,
  you must roll a D100 to see what strange things will happen. (Do no
  evil, by the way.)
\end{itemize}

The rest of this subsection describes the ``fine print'' of what
happens, in detail. You may wish to skip it.

\medskip
\noindent\textbf{The Details.}
Let us start with some background knowledge on how the graph drawing
engine works might be useful: Using a special internal key called
|graph drawing scope|, which you typically will not call directly,
the graph drawing engine can be switched on for a |{scope}|. When this
happens, a lot of things change inside \pgfname\ and \tikzname\ for
this scope: First, all nodes created inside the scope are not
immediately placed at the position where they were created. Instead,
they are ``spirited away'' to some internal table of the graph drawing
engine. Second, all edges created inside the scope using either the
|graph| command, the |edge| command, or the |child| command are also
``spirited away'' to another internal table. Then, at the end of the
scope, the graph drawing algorithm is started, which has access to
these internal tables of nodes and edge of the graph that has been
specified inside the scope. The algorithm will then compute new,
better, positions for the nodes. Finally, once the positions have been
computed, the graph drawing engine will then retrieve the nodes from
the internal table and place them at the computed positions and it
also retrieves the edges from the internal table and also adds them to
the picture.

While this theory may sound complicated, the use of the graph library
is, fortunately, pretty simple: Just add a key like |tree layout| or
|spring layout| to a scope and leave out any explicit positioning via
things like |at| -- the positioning will be done automatically by the
graph drawing algorithm.

The keys like |tree layout| or |spring layout| are explained in more detail
in the chapters on the different libraries. They all internally call
(at least) two keys: |graph drawing scope| and |algorithm|. These
keys are documented in the following, but you typically will not use
them explicitly. In addition to setting up the scope and setting the
correct algorithms, keys like |tree layout| and |spring layout| also take
some \meta{options} as arguments. These \meta{options} allow you setup
special graph parameters for the algorithm.

\begin{key}{/tikz/graph drawing scope}
  This key can (only) be used as an option when a \tikzname\ scope is
  started. Thus, you can pass it to |\tikz|, to |{tikzpicture}|, to
  |\scoped|, to |{scope}|, to |graph|, and to |{graph}|. For instance,
  the |tree layout| option (which uses |graph drawing scope| internally) can
  be used in the following ways:
\begin{codeexample}[]
\tikz [tree layout] \graph {a -> {b,c}};  

\tikz \graph [tree layout] {a -> {b,c}};

\tikz \path graph [tree layout] {a -> {b,c}};

\begin{tikzpicture}[tree layout]
  \graph {a -> {b,c}};
\end{tikzpicture}

\begin{tikzpicture}
  \draw [help lines] (0,0) grid (3,1);
  
  \scoped [tree layout] \graph {a -> {b,c}};
    
  \begin{scope}[tree layout, xshift=1cm, rotate=90]
    \graph {a -> {b,c}};
  \end{scope}
\end{tikzpicture}
\end{codeexample}

  You can \emph{not} use the |graph drawing scope| key with a single
  node or on a path. In particular, to typeset a tree given in the
  |child| syntax somewhere inside a |{tikzpicture}|, you must prefix
  it with the |\scoped| command:
\begin{codeexample}[]
\begin{tikzpicture}
  \scoped [tree layout]
    \node {root}
    child { node {left child} }
    child { node {right child} };
\end{tikzpicture}
\end{codeexample}
  Naturally, the above could have been written more succinctly as
\begin{codeexample}[]
\tikz [tree layout]
  \node {root}
  child { node {left child} }
  child { node {right child} };
\end{codeexample}
  Or even more succinctly:
\begin{codeexample}[]
\tikz \graph [tree layout] { root -- {left child, right child} };
\end{codeexample}

  In detail, adding the |graph drawing scope| command to a scope has
  the following effects:
  \begin{itemize}
  \item The basic layer is informed, using the
    |execute at begin scope| key, that the current scope will contain
    nodes that should be positioned by a graph drawing engine. Which
    algorithm is used depends on the value of the |algorithm| key.
  \item If the |graphs| library has been loaded, the default
    positioning mechanisms of this library are switched off, leaving
    the positioning to the graph drawing engine. Also, when an edge is
    created by the |graphs| library, this is signalled to the graph
    drawing library. (To be more precise: The keys |new ->| and so on
    are redefined so that they call |\pgfgdedge| instead of creating
    an edge.
  \item The |edge| path command is modified so that it also calls
    |\pgfgdedge| instead of immediately creating any edges.
  \item The |edge from parent| path command is modified so that is
    also calls |\pgfgdedge|.
  \item The keys |append after command| and |prefix after command|
    keys are modified so that they are executed only via
    |late options| when the node has ``reached its final parking
    position''. 
  \end{itemize}
\end{key}

\begin{key}{/graph drawing/algorithm=\meta{algorithm's name}}
  \label{section-gd-algorithm-key}%
  This key specifies which algorithm should be used for typesetting a
  graph. The names of these algorithm's are often a bit cryptic (like
  |Walshaw2000| or something similar), which is why you typically do
  not call this key directly. Instead, styles with more
  easy-to-remember names internally set this key.

  Setting this key has the following effects: When a scope with the
  |graph drawing scope| command is started, the current value of
  \meta{algorithm's name} is examined. Lua will try to find a class
  named \meta{algorithm's name} that has been declared using the
  |graph_drawing_algorithm| command (but any spaces inside the
  \meta{algorithm's name} are deleted). If it does not find such a
  class, the engine tries to  
  load the file called |pgfgd-algorithm-|\meta{algorithm's name}|.lua|
  (again, spaces are deleted) and then, again, tries to lookup the
  class. Thus, any class \meta{algorithm's name} mentioned inside a
  document must either have 
  already been defined in some Lua file loaded by some library or it
  must reside in a file with the corresponding name. Once the class
  has been found, the class's |new| method is called, the |graph|
  attribute of the object is set to the to-be-layouted graph, and the
  |constructor| method of the class is called for the object (if it
  exists) and, finally, the |run| method is called. Details of what
  should be done in these methods are given in
  Section~\ref{section-gd-own-algorithm}).  

  Here is an example where we switch on the graph drawing engine
  explicitly and explicitly select an algorithm:
\begin{codeexample}[]
\tikz [graph drawing scope,
       /graph drawing/algorithm=Spring Electrical Walshaw 2000]
  \graph { a <-> {b, c} };  
\end{codeexample}

  The reference of the available algorithms is in
  Sections~\ref{section-first-graphdrawing-library-in-manual} to 
  \ref{section-last-graphdrawing-library-in-manual}.
\end{key}




\subsection{Graph, Node, and Edge Parameters}

Graph drawing algorithms can typically be configured in some way. For
instance, for a graph drawing algorithm that visualizes its nodes as a
tree, it will typically be useful when the user can change the
so-called \emph{level distance} and the \emph{sibling distance}. For
other algorithms, like force-based algorithms, a large number of
parameters influence the way the algorithms work.

Options that influence graph drawing algorithms will be called
\emph{graph drawing parameters} in the following. There are three kinds of
graph drawing parameters:
\begin{itemize}
\item Graph parameters,
\item node parameters, and
\item edge parameters.
\end{itemize}
A graph drawing graph parameter influences the layout of the whole
graph. A graph drawing node parameter is an option that is attached to
a single node and should only have a direct influence on this node
(like ``place this node exactly at this position, no matter what''). A
graph drawing edge parameter in important for a single edge (like
``this edge must be exactly |2cm| long'').

A graph drawing algorithm may or may not take the different graph
parameters into account. After all, these options may even outright
contradict each other, so an algorithm can only try to ``do its
best''.

While many graph parameters are very specific to a single algorithm, a
number of graph parameters will be important for many algorithms. Such
graph parameters are called \emph{common} graph parameters, the most
important of which are documented in the following. The common graph
parameters can be used like any normal \tikzname\ option. In contrast,
specific options for algorithms must be passed to the key that
installs the algorithm. For example, the orientation of a graph
is setup with the common key |orient|, which is given alongside a key
like |spring layout|:

\begin{codeexample}[]
\tikz \graph [spring layout, orient=1|2] { 1--2--3--1 };  
\end{codeexample}

In contrast, the very specific option |iterations| must be
passed to the |spring layout| key:

\begin{codeexample}[]
\tikz \graph [spring layout={iterations=3}] { 1--2--3--1 };  
\end{codeexample}





\subsection{Padding and Node Distances}

\label{subsection-gd-dist-pad}

In many drawings, you may wish to specify how ``near'' two nodes should
be placed by a graph drawing algorithm. Naturally, this depends
strongly on the specifics of the algorithm, but there are a number of
general keys that will be used by many algorithms.



\subsubsection{Distances and Paddings Between Layers}

A number of graph drawing algorithms arrange nodes in layers; we refer
to the nodes on the same layer as siblings (although, in a tree,
siblings are only nodes with the same parent; nevertheless we use
``sibling'' loosely also for nodes that are more like ``cousins'').

\begin{key}{/graph drawing/level distance=\meta{dimension} (initially 1cm)}
  \keyalias{tikz}\keyalias{tikz/graphs}
  This is minimum distance that the centers of nodes on one
  level should have from the centers of nodes on the next level. It
  will not always be possible to satisfy this desired distance, for
  instance in case the nodes are too big. In this case, the
  \meta{dimension} is just considered as a lower bound.
\begin{codeexample}[]
\tikz \graph [layered layout, level distance=1cm] { 1--2--3--1 };  
\tikz \graph [layered layout, level distance=5mm] { 1--2--3--1 };  
\end{codeexample}
\end{key}

\begin{key}{/graph drawing/level pre sep=\meta{dimension} (initially 0.5pt)}
  \keyalias{tikz}\keyalias{tikz/graphs}
  This is a minimum ``padding'' or ``separation'' between the border
  of the nodes on a level to any nodes on the previous level. Thus, if
  nodes are so big that nodes on consecutive levels would overlap (or
  just come with \meta{dimension} distance of one another), their
  distance is enlarged so that this distance is still satisfied.

  (If a node on the previous level also has a |level post sep|, this
  post padding and the \meta{dimension} add up. Thus, these keys
  behave like the ``padding'' keys rather
  than the ``margin'' key of cascading style sheets.)
  
  Currently, this option is not yet implemented.
\end{key}

\begin{key}{/graph drawing/level post sep=\meta{dimension} (initially 0.5pt)}
  \keyalias{tikz}\keyalias{tikz/graphs}
  Works like |level pre sep|.
\end{key}

\begin{key}{/graph drawing/level sep=\meta{dimension}}
  \keyalias{tikz}\keyalias{tikz/graphs}
  Sets both |level pre sep| and |level post sep| to
  $\meta{dimension}/2$.
\end{key}

Note that if you set |level distance=0| and |level distance=1em|, you get
a layout where any two consecutive layers are ``spaced apart'' by
|1em|.


\subsubsection{Distances and Paddings Between Siblings}

The following keys work much like the |level ...| keys, only for
sibling:

\begin{key}{/graph drawing/sibling distance=\meta{dimension} (initially 1cm)}
  \keyalias{tikz}\keyalias{tikz/graphs}
  This is minimum distance that the centers of node should have to the
  center of the next node on the same level. As for levels, this is
  just a lower bound.

  For some layouts, like a circular layout, the \meta{dimension} is
  measured as the distance on the circle:
\begin{codeexample}[]
\tikz \graph [circular layout, sibling distance=1cm, nodes={circle,draw}]
  { 1--2--3--4--5--6--1 };  
\end{codeexample}
\begin{codeexample}[]
\tikz \graph [circular layout, sibling distance=0cm, sibling sep=0pt,
              nodes={circle,draw}]
  { 1--2--3--4--5--6--1 };  
\end{codeexample}
\begin{codeexample}[]
\tikz \graph [circular layout, sibling distance=0cm, sibling sep=0pt,
              nodes={circle,draw}]
  { 1--2--3[sibling distance=3cm]--4--5--6--1 };  
\end{codeexample}
\end{key}


\begin{key}{/graph drawing/sibling pre sep=\meta{dimension} (initially 0.5pt)}
  \keyalias{tikz}\keyalias{tikz/graphs}
  Works like |level pre sep|, only for siblings.
\begin{codeexample}[]
\tikz \graph [circular layout, sibling distance=0cm, nodes={circle,draw},
              sibling sep=0pt]
  { 1--2--3--4--5--6--1 };  
\end{codeexample}
\begin{codeexample}[]
\tikz \graph [circular layout, sibling distance=0cm, nodes={circle,draw},
              sibling pre sep=1em]
  { 1--2--3--4--5--6--1 };  
\end{codeexample}
\begin{codeexample}[]
\tikz \graph [circular layout, sibling distance=0cm, nodes={circle,draw},
              sibling pre sep=1em]
  { 1--2--3[sibling pre sep=1cm]--4--5--6--1 };  
\end{codeexample}
\end{key}

\begin{key}{/graph drawing/level sibling sep=\meta{dimension} (initially 0.5pt)}
  \keyalias{tikz}\keyalias{tikz/graphs}
  Works like |level sibling sep|.
\end{key}

\begin{key}{/graph drawing/sibling sep=\meta{dimension}}
  \keyalias{tikz}\keyalias{tikz/graphs}
  Sets both |sibling pre sep| and |sibling post sep| to
  $\meta{dimension}/2$.
\end{key}




\subsubsection{Paddings Between Components}

When a graph consists of several connected component, many graph
drawing algorithms will layout these components individually. The
different components will then be arranged next to each other, see
Section~\ref{section-gd-packing} for the details, such that between
the nodes of any two components the following padding is avaiable:

\begin{key}{/graph drawing/component sep=\meta{dimension} (initially 2em)}
  \keyalias{tikz}\keyalias{tikz/graphs}
  This is distance between the bounding boxes that nodes of different
  connected components will have when they are placed next to each
  other:
\begin{codeexample}[]
\tikz \graph [binary tree layout, sibling distance=4mm, level distance=8mm,
              components go right top aligned,
              component sep=1pt, nodes=draw]  
{
  1 -> 2 -> {3->4[second]->5,6,7};
  a -> b[second] -> c[second] -> d -> e;
  x -> y[second] -> z -> u[second] -> v;
};  
\end{codeexample}
\begin{codeexample}[]
\tikz \graph [binary tree layout, sibling distance=4mm, level distance=8mm,
              components go right top aligned,
              component sep=1em, nodes=draw]  
{
  1 -> 2 -> {3->4[second]->5,6,7};
  a -> b[second] -> c[second] -> d -> e;
  x -> y[second] -> z -> u[second] -> v;
};  
\end{codeexample}
\end{key}



\subsection{Anchoring a Graph}

\label{subsection-library-graphdrawing-anchoring}

A graph drawing algorithm must compute positions of the nodes of a
graph, but the computed positions are only \emph{relative} (``this
node is left of this node, but above that other node''). It is not
immediately obvious where the ``the whole graph'' should be placed
\emph{absolutely} once all relative positions have been computed. In
case that the graph consists of several unconnected components, the
situation is even more complicated.

In order to determine the absolute position of a graph, the graph
drawing engine relies on the following key:

\begin{key}{/graph drawing/desired at=\marg{coordinate}}
  \keyalias{tikz}\keyalias{tikz/graphs}
  When you add this key to a node in a graph, you ``desire'' that the
  node should be placed at the \meta{coordinate} by the graph drawing
  algorithm. Now, when you set this key for a single node of a graph,
  then, by shifting the graph around, this ``wish'' can obviously
  always be fulfill:
\begin{codeexample}[]
\begin{tikzpicture}
  \draw [help lines] (0,0) grid (3,2);
  \graph [spring layout]
  {
    a [desired at={(1,0)}] -- b -- c -- a;
  };
\end{tikzpicture}
\end{codeexample}
\begin{codeexample}[]
\begin{tikzpicture}
  \draw [help lines] (0,0) grid (3,2);
  \graph [spring layout]
  {
    a -- b[desired at={(2,1)}] -- c -- a;
  };
\end{tikzpicture}
\end{codeexample}
\begin{codeexample}[]
\begin{tikzpicture}
  \draw [help lines] (0,0) grid (3,2);
  \graph [layered layout]
  {
    a -- b[desired at={(2,1)}] -- c -- a;
  };
\end{tikzpicture}
\end{codeexample}
  Since the key's name is a but long and since the many braces and
  parentheses are a bit cumbersome, there is a special support for
  this key inside a |graph|: The standard |/tikz/at| key is redefined
  inside a |graph| so that it points to |/graph drawing/desired at|
  instead. (Which is more logical anyway, since it makes no sense to
  specify an |at| position for a node whose position it to be computed
  by a graph drawing algorithm.) A nice side effect of this is that
  you can use the |x| and |y| keys (see
  Section~\ref{section-graphs-xy}) to specify desired positions:
\begin{codeexample}[]
\begin{tikzpicture}
  \draw [help lines] (0,0) grid (3,2);
  \graph [spring layout]
  {
    a -- b[x=2,y=1] -- c -- a;
  };
\end{tikzpicture}
\end{codeexample}
\begin{codeexample}[]
\begin{tikzpicture}
  \draw [help lines] (0,0) grid (3,2);
  \graph [layered layout]
  {
    a [x=1,y=2] -- { b, c } -- {e, f} -- a
  };
\end{tikzpicture}
\end{codeexample}

  A problem arises when two or more nodes have this key set
  and when these nodes are in the same connected component, because
  then your ``desires'' for placement and the positions computed by
  the graph drawing algorithm may clash. Graph drawing algorithms are
  ``told'' about the desired positions. Most algorithms will simply
  ignore these desired positions since they will be taken care of in
  the so-called post-anchoring phase, see below. However, for some
  algorithms it makes a lot of sense to fix the positions of some
  nodes and only compute the positions 
  of the other nodes relative to these nodes. For instance, for a
  |spring layout| it makes perfect sense that some nodes are
  ``nailed to the canvas'' while other nodes can ``move freely''.
\begin{codeexample}[]
\tikz \graph [spring layout]
{
  a -- { b, c, d, e -- {f,g,h} };
  { h, g } -- a;
};
\end{codeexample}
\begin{codeexample}[]
\tikz \graph [spring layout]
{
  a -- { b, c, d[x=0], e -- {f[x=2], g, h[x=1]} };
  { h, g } -- a;
};
\end{codeexample}
\begin{codeexample}[]
\tikz \graph [spring layout]
{
  a -- { b, c, d[x=0], e -- {f[x=2,y=1], g, h[x=1]} };
  { h, g } -- a;
};
\end{codeexample}
\end{key}


\begin{key}{/graph drawing/anchor node=\meta{node name}}
  \keyalias{tikz}\keyalias{tikz/graphs}
  This option can be used with a graph to specify a node that should
  be used for anchoring the whole graph. When this option is
  specified, after the layout has been computed, the whole graph will
  be shifted in such a way that the \meta{node name} is either
  \begin{itemize}
  \item at the current value of |anchor at| or 
  \item at the position that is specified in the form of a
    |desired at| for the \meta{node name}.
  \end{itemize}
\begin{codeexample}[]
\tikz \draw (0,0)
  -- (1,0.5) graph [edges=red,  layered layout, anchor node=a] { a -> {b,c} }
  -- (2,0)   graph [edges=blue, layered layout,
                    anchor node=y, anchor at={(2,0)}]          { x -> {y,z} };
\end{codeexample}
\begin{codeexample}[]
\begin{tikzpicture}
  \draw [help lines] (0,0) grid (3,2);
  
  \graph [layered layout, anchor node=c, edges=rounded corners]
    { a -- {b [x=1,y=1], c [x=1,y=1] } -- d -- a};
\end{tikzpicture}
\end{codeexample}
  Note how in the above example |c| is placed at |(1,1)| rather than
  |b| as would happen by default.
\end{key}

\begin{key}{/graph drawing/anchor at=\meta{coordinate} (initially the origin)}
  \keyalias{tikz}\keyalias{tikz/graphs}
  The coordinate at which the graph should be anchored when no
  explicit anchor is given for any node.
\begin{codeexample}[]
\begin{tikzpicture}
  \draw [help lines] (0,0) grid (2,2);
  
  \graph [layered layout, edges=rounded corners, anchor at={(1,2)}]
    { a -- {b, c [anchor here] } -- d -- a};
\end{tikzpicture}
\end{codeexample}
\end{key}

\begin{key}{/graph drawing/anchor here=\opt{\meta{true or false}} (default true)}
  \keyalias{tikz}\keyalias{tikz/graphs}
  This option can be passed to a single node (rather than the graph as
  a whole) in order to specify that this node should be used for the
  anchoring process.
\begin{codeexample}[]
\begin{tikzpicture}
  \draw [help lines] (0,0) grid (2,2);
  
  \graph [layered layout, edges=rounded corners]
    { a -- {b, c [anchor here] } -- d -- a};
\end{tikzpicture}
\end{codeexample}
  In the above example, |c| is placed at the origin since this is the
  default |anchor at| position.
\end{key}

Let us briefly summarize the order in which \tikzname\ tries to
determine the node at which the graph should be anchored:
\begin{enumerate}
\item If the |anchor node=|\meta{node name} option given to the graph
  as a whole, the graph is anchored at \meta{node name}, provided
  there is a node of this name in the graph. (If there is no node of
  this name or if it is mispelled, the effect is the same as if this
  option had not been given at all.)
\item Otherwise, if there is a node where the |anchor here| option is
  specified, the first node with this option set is used.
\item Otherwise, if there is a node where the |desired at| option is
  set (perhaps implicitly through keys like |x|), the first such node
  is used.
\item Finally, in all other cases, the first node is used.
\end{enumerate}

In the above description, the ``first'' node refers to the node first
encountered in the specification of the graph.



\subsection{Orienting a Graph}

\label{subsection-library-graphdrawing-standard-orientation}

Just as a graph drawing algorithm cannot know \emph{where} a graph
should be placed on a page, it is also often unclear which
\emph{orientation} it should have. Some graphs, like trees, have a
natural direction in which they ``grow'', but for an ``arbitrary''
graph the ``natural orientation'' is, well, arbitrary.

As for anchoring, the graph drawing algorithm is ``told'' about
the desired ``orientation of certain edges and certain nodes'': for
each node and each edge, keys may specify a ``desired
orientation''. For edges, you can, for instance, request that an
edge should be vertical an go upwards by saying |slope=up|. For a
node, ``specifying an orientation'' means that the rest of the graph
should be rotated in such a way that certain other nodes or
vectors should lie at a certain angle relative to the current node.


\subsubsection{Orienting a Graph by Fixing the Slope of Edges}

The following keys are used to specify orientations:
\begin{key}{/graph drawing/orient=\meta{angle}}
  \keyalias{tikz}\keyalias{tikz/graphs}
  Adding this key to an edge tells the graph drawing engine that the
  edge should have a slope of the given \meta{angle}. This ``slope''
  is defined as the angle of the line connecting the start of the edge
  to the end of the edge (independently of the actual to-path of the
  edge, which might define a bend or more complicated shapes). For
  instance, a \meta{angle} of |45| requests that the end node is ``up
  and right'' relative to the start node.
  
  Instead of an \meta{angle}, you can also specify the standard
  direction texts |north| or |south east| and so forth and also
  |up|, |down|, |left|, and |right|.
    
\begin{codeexample}[]
\tikz \graph [spring layout]
{
  a -- { b, c, d, e -- {f, g, h} };
  h -- [orient=30] a;
};
\end{codeexample}
\end{key}

\begin{key}{/graph drawing/orient'=\meta{angle}}
  Same as above, only the rest of the graph should be flipped relative
  to the edge.
    
\begin{codeexample}[]
\tikz \graph [spring layout]
{
  a -- { b, c, d, e -- {f, g, h} };
  h -- [orient'=30] a;
};
\end{codeexample}
\end{key}


\subsubsection{Orienting a Graph by Fixing the Slope Between Nodes}

\begin{key}{/graph drawing/orient=\opt{|:|}\meta{angle}|:|\meta{another node}}
  \keyalias{tikz}\keyalias{tikz/graphs}
  Adding this version of the |orient| key (it is detected by the
  presence of the colon) to a node requests that the graph drawing
  engine should ensure that the straight line from the origin (typically
  the center) of the node to the origin of \meta{another node}
  should have a slope of \meta{angle}. Note that the current node
  and the \meta{another node} need not be connected by an edge.
\begin{codeexample}[]
\tikz \graph [spring layout]
{
  a [orient=:-90:f] -- { b, c, d, e -- {f, g, h} };
  { h, g } -- a;
};
\end{codeexample}
  
  Instead of an \meta{angle}, you can also specify the standard
  direction texts |north| or |south east| and so forth and also
  |up|, |down|, |left|, and |right|. Furthermore, the leading colon is
  optional: 
\begin{codeexample}[]
\tikz \graph [spring layout]
{
  a [orient=down:h] -- { b, c, d, e -- {f, g, h} };
  { h, g } -- a;
};
\end{codeexample}

  As special features, a dash somewhere inside the |orient| key is
  replaced by |:0:| and a vertical bar by |:-90:|. Thus, |orient=-a|
  is the same as |orient=:0:a|. Similarly:
\begin{codeexample}[]
\tikz \graph [spring layout]
{
  a [orient=|h] -- { b, c, d, e -- {f, g, h} };
  { h, g } -- a;
};
\end{codeexample}
\end{key}

\begin{key}{/graph drawing/orient'=\meta{angle}:\meta{another node}}
  Same as above, only the rest of the graph should be flipped relative
  to the edge.
\end{key}

Instead of specifying the slope between two nodes ``at the nodes'' it 
is sometimes more natural to specify it at the beginning of the
graph. For this, the following special key is available:

\begin{key}{/graph drawing/orient=\meta{node1}|:|\meta{angle}|:|\meta{node2}}
  \keyalias{tikz}\keyalias{tikz/graphs}
  This has nearly the same effect as specifying
  |orient=|\meta{angle}|:|\meta{node2} as an option for the node
  \meta{node1}. The only difference is that |orient| options given at
  a node always take precedence over this ``global'' option.

  As above, \meta{node1}|-|\meta{node2} gets replaced by
  \meta{node1}|:0:|\meta{node2} and \meta{node1}\verb!|!\meta{node2}
  \meta{node1}|:-90:|\meta{node2}.
\begin{codeexample}[]
\tikz \graph [spring layout] { a -- b -- c -- a };
\tikz \graph [spring layout,orient=a-b] { a -- b -- c -- a };
\tikz \graph [spring layout,orient=b-a] { a -- b -- c -- a };
\tikz \graph [spring layout,orient=b|a] { a -- b -- c -- a };
\tikz \graph [spring layout,orient=a:10:b] { a -- b -- c -- a };
\tikz \graph [spring layout,orient=1-2] { subgraph K_n[n=5] };
\tikz \graph [spring layout,orient=2-1] { subgraph K_n[n=5] };
\end{codeexample}
\end{key}

\begin{key}{/graph drawing/orient'=\meta{orientation}}
  \keyalias{tikz}\keyalias{tikz/graphs}
  Does the same as |orient| except that the nodes are flipped over the
  principal axis.
\begin{codeexample}[]
\tikz \graph [spring layout,orient=a-b]  { a -- b -- c -- a };
\tikz \graph [spring layout,orient'=a-b] { a -- b -- c -- a };
\end{codeexample}
\end{key}



\subsubsection{Orienting a Graph by Fixing the Direction of Growth of the Children}

\begin{key}{/graph drawing/grow=\meta{angle}}
  \keyalias{tikz}\keyalias{tikz/graphs}
  This key specifies in which direction the neighbors of a node
  ``should grow.'' For some graph drawing algorithms, especially for
  those that layout trees, but also for those that produce layered
  layouts, there is a natural direction in which the ``children'' of
  a node should be placed. For instance, saying |grow=down| will cause
  the children of a node in a tree to be placed in a left-to-right
  line below the node (as always, you can replace the \meta{angle}
  by direction texts). The children are requested to be placed in a
  counter-clockwise fashion, the |grow'| key will place them in a
  clockwise fashion.
  
  Note that when you say |grow=down| it is not necessarily the case
  that any particular node is actually directly below the current
  node; the key just requests that the direction of growth is
  downward.
  
  In principle, you can specify the direction of growth for each node 
  individually, but do not count on graph drawing algorithms to
  honour these wishes.
  
\begin{codeexample}[]
\tikz \graph [layered layout, sibling distance=5mm]
{
  a [grow=right] -- { b, c, d, e -- {f, g, h} };
  { h, g } -- a;
};
\end{codeexample}

  When you give the |grow=right| key to the graph as a whole, it will
  be applied to all nodes. This happens to be exactly what you want:
  
\begin{codeexample}[]
\tikz \graph [layered layout, grow=right, sibling distance=5mm]
{
  a -- { b, c, d, e -- {f, g, h} };
  { h, g } -- a;
};
\end{codeexample}
  
\begin{codeexample}[]
\tikz
  \graph [layered layout, grow'=right]
  {
    {a,b,c} --[complete bipartite] {e,d,f}
            --[complete bipartite] {g,h,i};
  };
\end{codeexample}
\end{key}
  
\begin{key}{/graph drawing/grow'=\meta{angle}}
  Same as above, only with the children in clockwise order.
\begin{codeexample}[]
\tikz \graph [layered layout, sibling distance=5mm]
{
  a [grow'=right] -- { b, c, d, e -- {f, g, h} };
  { h, g } -- a;
};
\end{codeexample}
\end{key}



\subsubsection{The Phases of the Orientation Procedure}
\label{subsection-graph-orientation-phases}

As for anchoring a graph, the different keys for orienting graphs may
easily produce conflicting demands, which need to be
resolved. The following steps are normally performed for each
connected component of the graph independently (see 
Section~\ref{section-gd-packing} for details on connected components),
but algorithms may choose to consider the graph ``as a whole''. In
this case, the following steps are performed only once for the whole
graph. 

\begin{enumerate}
\item
  The graph drawing algorithm is ``told'' about the desired
  orientations in the form of graph, node, and edge parameters. 
  An algorithm may or may not try to honor the desired
  orientations. As for the anchoring of graphs, for some algorithms it
  is natural and easy to restrict the way nodes are placed so as to
  honor orientation requests, for others this makes sense, at best, on
  a global scale. Algorithms will internally tell the graph drawing
  engine when they grow the graph in some direction. They can even
  indicate that they have already taken care of all growth demands for
  individual nodes.

  Nevertheless, the following steps are always performed:
\item
  The engine checks whether there is an edge in the graph whose
  slope has been fixed using the |orient| key. If there is at least one
  such edge, the first such edge is considered. The graph is rotated
  such that this edge has the desired slope. The orientation process
  stops at this point, all other orientation requests are ignored.
\item
  Otherwise, if there is no specified slope for any edge, it is
  checked whether there is a node with a specified |orient| to another
  node. If this is the case, the first such specification is taken for
  which the other node exists and the graph is rotated so that this
  orientation is satisfied. Again, the process stops in this case.
\item
  It is next checked whether there is a |orient| request for the graph
  as a whole like |orient=a-b|. Provided |a| and |b| exist, this
  request is honoured and the process stops.
\item
  If the algorithm has indicated that it has already taken care of all
  |grow| requests (using an internal function), the process stops at
  this point.
\item
  Otherwise, if none of the above cases is encountered, we look for
  a node with a |grow| key attached to it. If there is such a node,
  the graph is rotated so that the direction of growth of the graph is
  the desired growth direction. For this, the orientation procedure
  obviously needs to know what the direction of growth the algorithm
  was using; the algorithm signals this internally by setting the
  |growth_direction| of the algorithm object or by attaching a
  |growth_direction| to nodes. If an algorithm fails to attach such a
  direction, the direction of the first edge of the node is chosen
  and, for an isolated node, the direction is a line to the first node
  in the graph other than the current node.
  
  If no node has |grow| specified, the orientation is chosen in such a
  way as if |grow=down| had been specified for the first node of the
  graph.   
\end{enumerate}



\subsection{Packing of Connected Components}

\label{subsection-gd-component-packing}

Graphs may be composed of subgraphs or \emph{components} that are not
connected to each other. In order to draw these nicely, most graph
drawing algorithms split them into separate graphs, computes
their layouts with the same graph drawing algorithm independently and,
in a postprocessing step, arranges them next to each other. (Some
graph drawing algorithms will treat a graph ``as a whole''; for such
algorithms the following options do not apply.)

The default method for placing the different components works as
follows:

\begin{enumerate}
\item For each component, a layout is determined and the component is
  oriented as described
  Section~\ref{subsection-library-graphdrawing-standard-orientation}
  on orientation of graphs. 
\item Then, we differentiate between two kinds of components: Those
  that contain an anchored node and those that do not. As described in
  Section~\ref{subsection-library-graphdrawing-anchoring}, you can
  directly specify for a node where it should be placed on the
  page. If a component contains such a node, it is clear where the
  component should go (we call it ``anchored'').
  
  The packing collects and considers all unanchored nodes, plus the
  first anchored component, if such a component exists. The second and
  further anchored component are not considered during the packing
  process, they are simply anchored according to their anchor nodes. 
\item If the previous step has yielded two or more components that now
  need to be packed, they are sorted as prescribed by the
  |component order| key.
\item The first component is now placed (conceptually) at the
  origin. (The final position of this and all other components will be
  determined later, namely in the anchoring phase, but let us imagine
  that the first component lies at the origin at this point.)
\item The second component is now positioned relative to the first
  component. The ``direction'' in which the next component is placed
  relative to the first one is determined by the |component direction|
  key, so components can be placed from left to right or up to down or
  even in any angular direction. However, both internally and in the
  following description, we assume that the components are placed from
  left to right; other directions are achieved by doing some (clever)
  rotating of the arrangement achieved in this way.

  So, we now wish to place the second component to the right of the
  first component. The component is first shifted vertically according
  to some alignment strategy. For instance, it can be shifted so that
  the topmost node of the first component and the topmost node of the
  second component have the same vertical position. Alternatively, we
  might require that certain ``alignment nodes'' in both components
  have the same vertical position. There are several other strategies,
  which can be configured using the |component align| key.

  One the vertical position has been fixed, the horizontal position is
  computed. Here, two different strategies are available: First, image
  rectangular bounding boxed to be drawn around both components. Then
  we shift the second component such that the right border of the
  bounding box of the first component touches the left border of the
  bounding box of the second component. Instead of having the bounding
  boxes ``touch,'' we can also have a padding of |component sep|
  between them. The second strategy is more involved and also known as
  a ``skyline'' strategy. It works as follows: Imaging the second
  component to be placed far to the right of the first component. Now
  start moving the second component to the left until one of the nodes
  of the second component touches a node of the first component, and
  stop. Again, the padding |component sep| can be used to avoid the
  nodes actually touching each other. (In case you are worried that
  the second component might actually ``pass through'' the first
  component because the nodes of the two component have totally
  different vertical heights, rest assured, that such ``vertical
  holes'' are taken care of to avoid this case.)
\item
  After the second component has been placed, the third component is
  considered and positioned relative to the second one, and so on.
\item
  At the end, as hinted at earlier, the whole arrangement is rotate so
  that instead of ``going right'' the component go in the direction of
  |component direction|. Note, however, the this rotation applies only
  to the ``shift'' of the components; the components themselves are
  not rotated. Fortunately, this whole rotation process happens in the
  background and the result is normally exactly what you would expect.
\end{enumerate}

In the following, we go over the different keys that can be used to
configure the component packing.


\subsubsection{Ordering the Component}

The different connected components of the graph are collected in a
list. The ordering of the nodes in this list can be configured using
the following key:

\begin{key}{/graph drawing/component order=\meta{strategy} (initially
    by first specified node)}
  \keyalias{tikz}\keyalias{tikz/graphs}
  The following values are permissible for \meta{strategy}
  \begin{itemize}
  \item \declare{|by first specified node|}

    The components are ordered ``in the way they appear in the input
    specification of the graph''. More precisely, for each component
    consider the node that is first encountered in the description
    of the graph. Order the components in the same way as these nodes
    appear in the graph description.
\begin{codeexample}[]
\tikz \graph [tree layout, nodes={inner sep=1pt,draw,circle}]
{ a, b, c -- d -- e, f -- g };
\end{codeexample}
  \item \declare{|increasing node number|}
    
    The components are ordered by increasing number of nodes. For
    components with the same number of nodes, the first node in each
    component is considered and they are ordered according to the
    sequence in which these nodes appear in the input.
\begin{codeexample}[]
\tikz \graph [tree layout, nodes={inner sep=1pt,draw,circle},
              component order=increasing node number]
{ a, b, c -- d -- e, f -- g };
\end{codeexample}
    \begin{key}{/graph drawing/small components first}
      \keyalias{tikz}
      \keyalias{tikz/graphs}
      A shorthand for |component order=increasing node number|.
    \end{key}
  \item \declare{|decreasing node number|}
    As above, on in decreasing order.  
    \begin{key}{/graph drawing/large components first}
      \keyalias{tikz}
      \keyalias{tikz/graphs}
      A shorthand for |component order=decreasing node number|.
\begin{codeexample}[]
\tikz \graph [tree layout, nodes={inner sep=1pt,draw,circle},
              large components first]
{ a, b, c -- d -- e, f -- g };
\end{codeexample}
    \end{key}
  \end{itemize}
\end{key}


\subsubsection{Arranging Components in a Certain Direction}

\begin{key}{/graph drawing/component direction=\meta{angle} (initially 0)}
  \keyalias{tikz}\keyalias{tikz/graphs}
  The \meta{angle} is used to determine the relative position of each
  component relative to the previous one. The direction need not be a
  multiple of |90|.
\begin{codeexample}[]
\tikz \graph [tree layout, nodes={inner sep=1pt,draw,circle},
              component direction=left]
  { a, b, c -- d -- e, f -- g };
\end{codeexample}
\begin{codeexample}[]
\tikz \graph [tree layout, nodes={inner sep=1pt,draw,circle},
              component direction=10]
  { a, b, c -- d -- e, f -- g };
\end{codeexample}
  As usual, you can use texts like |up| or |right| instead of a
  number.

  As the example shows, the direction only has an influence on the
  relative positions of the components, not on the direction of growth
  inside the components. In particular, the components are not rotated
  by this option in any way. You can use the |grow| option or |orient|
  options to orient individual components:
\begin{codeexample}[]
\tikz \graph [tree layout, nodes={inner sep=1pt,draw,circle},
              component direction=up]
  { a, b, c [grow=right] -- d -- e, f[grow=45] -- g };
\end{codeexample}
\end{key}



\subsubsection{Aligning Components}

When components are placed next to each from left to right, it
is not immediately clear how the components should be aligned
vertically. What happens is the in each component a horizontal line is
determined and then all components are shifted vertically so that the
lines are aligned. There are different strategies for choosing these
``lines'', see the description of the options described later on.
When the |component direction| option is used to change the direction
in which components are placed, it certainly make no longer sense to
talk about ``horizontal'' and ``vertical'' lines. Instead, what
actually happens is that the alignment does not consider
``horizontal'' lines, but lines that go in the direction specified by
|component direction| and aligns them by moving components along a
line that is perpendicular to the line. For these reasons, let us call
the line in the component direction the \emph{alignment line} and a
line that is perpendicular to it the \emph{shift line}.

The first way of specifying through which point of a component the
alignment line should get is to use the following option:

\begin{key}{/graph drawing/align here}
  \keyalias{tikz}
  \keyalias{tikz/graphs}
  When this option is given to a node, this alignment line will go
  through the origin of this node. If this option is passed to more
  than one node of a component, the node encountered first in the
  component is used.
\begin{codeexample}[]
\tikz \graph [binary tree layout, nodes={draw}]
{ a, b -- c[align here], d -- e[second, align here] -- f };
\end{codeexample}
\end{key}

In many cases, however, you will not wish to specify an alignment node
manually in each component. Instead, you will use the following key to
specify a \emph{strategy} that should be used to automatically
determine such a node:

\begin{key}{/graph drawing/component align=\meta{strategy} (initially first node)}
  \keyalias{tikz}
  \keyalias{tikz/graphs}
  The following values are permissible:
  \begin{itemize}
  \item \declare{|first node|}
    In each component, the alignment line goes through the center of
    the first node of the component encountered during specification
    of the component.
\begin{codeexample}[]
\tikz \graph [binary tree layout, nodes={draw},
              component align=first node]
{ a, b -- c, d -- e[second] -- f };
\end{codeexample}
  \item \declare{|center|}
    
    The nodes of the component are projected onto the shift line. The
    alignment line is now chosen so that is exactly in the middle
    between the maximum and minimum value that the projected nodes
    have on the shift line.
\begin{codeexample}[]
\tikz \graph [binary tree layout, nodes={draw},
              component align=center]
{ a, b -- c, d -- e[second] -- f };
\end{codeexample}
\begin{codeexample}[]
\tikz \graph [binary tree layout, nodes={draw},
              component direction=90,
              component align=center]
{ a, b -- c, d -- e[second] -- f };
\end{codeexample}
  \item \declare{|counterclockwise|}

    As for |center|, we project the nodes of the component onto the
    shift line. The alignment line is now chosen so that it goes
    through the center of the node whose center has the highest
    projected value.
\begin{codeexample}[]
\tikz \graph [binary tree layout, nodes={draw},
              component align=counterclockwise]
{ a, b -- c, d -- e[second] -- f };
\end{codeexample}
\begin{codeexample}[]
\tikz \graph [binary tree layout, nodes={draw},
              component direction=90,
              component align=counterclockwise]
{ a, b -- c, d -- e[second] -- f };
\end{codeexample}
    The name |counterclockwise| is intended to indicate that the align
    line goes through the node that comes last if we go from the
    alignment direction in a counter-clockwise direction.
  \item \declare{|clockwise|}
    
    Works like |counterclockwise|, only in the other direction:
\begin{codeexample}[]
\tikz \graph [binary tree layout, nodes={draw},
              component align=clockwise]
{ a, b -- c, d -- e[second] -- f };
\end{codeexample}
\begin{codeexample}[]
\tikz \graph [binary tree layout, nodes={draw},
              component direction=90,
              component align=clockwise]
{ a, b -- c, d -- e[second] -- f };
\end{codeexample}
  \item \declare{|counterclockwise bounding box|}

    This method is quite similar to |counterclockwise|, only the
    alignment line does not go through the center of the node with a
    maximum projected value on the shift line, but through the maximum
    value of the projected bounding boxes. For a left-to-right
    packing, this means that the components are aligned so that the
    bounding boxes of the components are aligned at the top.
\begin{codeexample}[]
\tikz \graph [tree layout, nodes={draw, align=center},
              component align=counterclockwise]
{ a, "high\\node" -- b};
\tikz \graph [tree layout, nodes={draw, align=center},
              component align=counterclockwise bounding box]
{ a, "high\\node" -- b};
\end{codeexample}
  \item \declare{|clockwise bounding box|}
    
    Works like |counterclockwise bounding box|.
  \end{itemize}
\end{key}

Using a combination of |component direction| and |component align|,
numerous different packing strategies can be configured. However,
since names like |counterclockwise| are a bit hard to remember and to
apply in practice, a number of easier-to-remember keys are predefined
that combine an alignment and a direction:

\begin{key}{/graph drawing/components go right top aligned}
  \keyalias{tikz/graphs}
  \keyalias{tikz}
  Shorthand for |component direction=right| and
  |component align=counterclockwise|. This means that, as the name
  suggest, the components will be placed left-to-right and they are
  aligned such that their top nodes are in a line.  
\begin{codeexample}[]  
\tikz \graph [tree layout, nodes={draw, align=center},
              components go right top aligned]
  { a, "high\\node" -- b};
\end{codeexample}
\end{key}

\begin{key}{/graph drawing/components go right absolute top aligned}
  \keyalias{tikz/graphs}
  \keyalias{tikz}
  Like the previous key, but with
  |component align=counterclockwise bounding box|. This means that the
  components will be aligned with their bounding boxed being
  top-aligned: 
\begin{codeexample}[]  
\tikz \graph [tree layout, nodes={draw, align=center},
              components go right absolute top aligned]
  { a, "high\\node" -- b};
\end{codeexample}
\end{key}

\begin{key}{/graph drawing/components go right bottom aligned}
  \keyalias{tikz/graphs}
  \keyalias{tikz}
  As above, only with an alignment of the bottom nodes.
\end{key}

\begin{key}{/graph drawing/components go right absolute bottom aligned}
  \keyalias{tikz/graphs}
  \keyalias{tikz}
\end{key}

\begin{key}{/graph drawing/components go right center aligned}
  \keyalias{tikz/graphs}
  \keyalias{tikz}
  As above, but with the alignment at the centers.
\end{key}

\begin{key}{/graph drawing/components go right}
  \keyalias{tikz/graphs}
  \keyalias{tikz}
  Shorthand for |component direction=right| and
  |component align=first node|.
\end{key}

The above options are all available also with |right| replaced by
|left|. Here is an example:
\begin{codeexample}[]
\tikz \graph [tree layout, nodes={draw, align=center},
              components go left top aligned]
  { a, "high\\node" -- b};
\end{codeexample}
Next, the options are also available with |right| replaced by |up| and
also by |down|. Then, instead of |top| or |bottom| for the alignment,
|left| and |right| must be used:
\begin{codeexample}[]
\tikz \graph [tree layout, nodes={draw, align=center},
              components go down left aligned]
  { a, hello -- {world,s} };
\end{codeexample}
\begin{codeexample}[]
\tikz \graph [tree layout, nodes={draw, align=center},
              components go up absolute left aligned]
  { a, hello -- {world,s}};
\end{codeexample}



\subsubsection{The Distance Between Components}

Once the components of a graph have been oriented, sorted, aligned,
and a direction has been chosen, it remains to determine the distance
between adjacent components. Two methods are available for computing
this distance, as specified by the following option:

\begin{key}{/graph drawing/component packing=\meta{method} (initially
    skyline)}
  \keyalias{tikz}
  \keyalias{tikz/graphs}
  Given two components, their distance is computed as follows in
  depencende of \meta{method}:
  \begin{itemize}
  \item \declare{|rectangular|}

    Imagine a bounding box to be drawn around both components. They
    are then shifted such that the padding (separating distance)
    between the two boxes is the current value of |component sep|.
\begin{codeexample}[]
\tikz \graph [tree layout, nodes={draw}, component sep=0pt,
              component packing=rectangular]
  { a -- long text, longer text -- b};
\end{codeexample}
  \item \declare{|skyline|}

    The ``skyline method'' described in the introduction of 
    Section~\ref{subsection-library-graphdrawing-anchoring} is used to
    compute the distance.
\begin{codeexample}[]
\tikz \graph [tree layout, nodes={draw}, component sep=0pt,
              component packing=skyline]
  { a -- long text, longer text -- b};
\end{codeexample}
  \end{itemize}
\end{key}



\subsection{Implementing Graph Drawing Algorithms}

\label{section-gd-own-algorithm}
\label{section-library-graphdrawing-ownAlgorithm}

This section presents a simple example of how a graph drawing
algorithm can be implemented. A more detailed explanation of how
new graph drawing algorithms can be integrated and configured in given
in Section~\ref{section-gd-implementing-algorithms}.

As explained for the |algorithm| key, for each graph drawing algorithm
there must be a class of the name given to the |algorithm| key. This
class should usually reside in a file called
|pgfgd-algorithm-|\meta{algorithm name}. This class must provide (at
least) the two methods |new| and |run|. Each time a layout needs to
be computed for a graph, a new object of this algorithm class is
instantiated using the class's |new| method. For the newly created
object, an attribute |graph| will be set to an object representing the
graph. Then, the |constructor| method of the object is called,
provided it exists. Then, the |run| method is called, which should do
the actual work. (The separation into a constructor and a run method
is purely for convenience.) The |run| method should modify the
coordinates of the nodes of its |graph| attribute.

To simplify the creating of classes and constructors, the graph
drawing engine provides the function |graph_drawing_algorithm|, which
takes a table of infos about the algorithm as input and will create a
class and a constructor.

As a complete example, the following code fragment implements a
trivial graph drawing algorithm that just places all nodes on a
fixed-size circle.  It is accessed with the name 
|Simple Demo|.

\pgfgddeclareforwardedkeys{/graph drawing}{
  radius/.graph parameter=evaluate math expression,
  radius/.parameter initial=1cm,
  node radius/.node parameter=evaluate math expression
}

\begin{codeexample}[code only]
-- File pgfgd-algorithm-SimpleDemo.lua

graph_drawing_algorithm { name = "SimpleDemo" }

function SimpleDemo:run()
  local radius = 28.908  -- this is 1cm in points
  local alpha = (2 * math.pi) / #self.graph.nodes
  for i=1,#self.graph.nodes do
    self.graph.nodes[i].pos.x = radius * math.cos((i-1) * alpha)
    self.graph.nodes[i].pos.y = radius * math.sin((i-1) * alpha)
  end
end
\end{codeexample}

The algorithm computes a circular layout like in the following.

\begin{codeexample}[]
\tikz [graph drawing scope, /graph drawing/algorithm=Simple Demo]
  \graph { f -> c -> e -> a -> {b -> {c, d, f}, e -> b}};
\end{codeexample}

For details on how to make things like |radius| configurable and how
to setup keys so that users can just write
|\tikz[circular layout] ...|, please see Section~\ref{section-gd-implementing-algorithms}.


\endinput


% Copyright 2011 by Renée Ahrens, Olof Frahm, Jens Kluttig, Matthias Schulz, Stephan Schuster
% Copyright 2011 by Till Tantau
%
% This file may be distributed and/or modified
%
% 1. under the LaTeX Project Public License and/or
% 2. under the GNU Free Documentation License.
%
% See the file doc/generic/pgf/licenses/LICENSE for more details.

\section{Graph Drawing Layouts: Trees}
\label{section-first-graphdrawing-library-in-manual}
\label{section-library-graphdrawing-trees}


\begin{tikzlibrary}{graphdrawing.trees}
  Load this package when you wish to use layout trees. You should load
  the |graphdrawing| library first. 
\end{tikzlibrary}

\ifluatex\relax\else{LuaTeX is required for setting this manual section.}\expandafter\endinput\fi


\subsection{Overview}

\tikzname\ offers several different syntax to specify trees (see
Sections \ref{section-trees} and~\ref{section-library-graphs}). By
default, \tikzname\ will attempt to produce a reasonable layout of the
specified trees, but since \TeX's algorithmic capabilities are quite
limited, no advanced layout of the tree is done by the standard
algorithms. This is where the graph drawing algorithms from this
library come in: Having the full power of Lua\TeX\ at their disposal,
they will produce a far better layout of the trees.


\subsection{The Reingold--Tilford Tree Layout}

\begin{gdalgorithm}{tree layout}{Tree Reingold Tilford 1981}
  This layout arranges nodes in a tree according to the
  Reingold--Tilford method.
\begin{codeexample}[]
\tikz [binary tree layout, sibling distance=7mm, level distance=10mm]
\graph [nodes={circle, inner sep=0pt, minimum size=2mm, fill}]{
  / -- { / -- / -- { / -- /, / -- { /, / }}, / -- / -- /[second] }
};
\end{codeexample}
\begin{codeexample}[]
\tikz \graph [binary tree layout, level distance=10mm] {
  Knuth -> {
    Beeton -> Kellermann [second] -> Carnes,
    Tobin -> Plass -> { Lamport, Spivak } 
  }
};\qquad
\tikz \graph [binary tree layout, grow'=right, sibling distance=5mm] {
  Knuth -> {
    Beeton -> Kellermann [second] -> Carnes,
    Tobin -> Plass -> { Lamport, Spivak } 
  }
};
\end{codeexample}
\begin{codeexample}[]
\tikz \graph [binary tree layout, grow'=30, sibling distance=5mm] {
  Knuth -> {
    Beeton -> Kellermann [second] -> Carnes,
    Tobin -> Plass -> { Lamport, Spivak } 
  }
};
\end{codeexample}
\end{gdalgorithm}

% \begin{gdalgorithm}{centered root tree layout}{A proof-of-concept by Ahrens et al., 2011}{AhrensFKSS2011 tree}
%   This algorithm was implemented as a proof-of-concept by Ahrens,
%   Frahm, Kluttig, Schulz, and Schuster, who have implemented the graph
%   drawing engine. The idea is that the 
%   root of each subtree is centered horizontally above the child
%   trees. The |sibling distance| and |level distance| keys are taken
%   into consideration.
  
% \begin{codeexample}[]
% \tikz[centered root tree layout, nodes=draw] \graph { a -> {b -> {c,d,f->{i,j,k}}, e}};
% \end{codeexample}

% As you can see, the text nodes aren't quite aligned, so the common fix
%   is to use the |text depth| and |text height| keys to force the text
%   nodes to a specific size.

% \begin{codeexample}[]
% \tikz[AhrensFKSS2011 tree, text depth=.2em, text height=.8em]
%   \graph { a -> {b -> {c,d}, e}};
% \end{codeexample}

%   \medskip
%   \noindent\textbf{Parameters.} 
%   The keys affecting the algorithm are the following common graph
%   drawing parameters:

%   \begin{key}{/graph drawing/root}
%     \keyalias{tikz}\keyalias{tikz/graphs}
%     This is a node parameter. At most one node should have this key
%     set. If no node has it set, the first node in the graph will be
%     used. 
%   \end{key}

%   \begin{key}{/graph drawing/level distance=\meta{leveldistance} (default 1cm)}
%     \keyalias{tikz}\keyalias{tikz/graphs}
%     Determines the vertical space between the nodes on different levels:
% \begin{codeexample}[]
% \tikz [AhrensFKSS2011 tree, level distance=1cm]
%   \graph { 1 -> {2 , 3}};
% \tikz [AhrensFKSS2011 tree, level distance=2cm]
%   \graph { 1 -> {2 , 3}};
% \end{codeexample}
%   \end{key}

%   \begin{key}{/graph drawing/sibling distance=\meta{siblingdistance} (default 1cm)}
%     \keyalias{tikz}\keyalias{tikz/graphs}
%     This determines the horizontal space between the nodes. 
% \begin{codeexample}[]
% \tikz [AhrensFKSS2011 tree, sibling distance=1cm]
%   \graph { 1 ->{2 , 3}};
% \tikz [AhrensFKSS2011 tree, sibling distance=2cm]
%   \graph { 1 ->{2 , 3}};
% \end{codeexample}
% \end{key}

%   \medskip
%   \textbf{How does this algorithm work?}
%   The tree algorithm works recursively. During the recursion one step is
%   performed for each subgraph of the tree.  

%   The process builds a kind of a box structure of the given graph. This
%   means a leaf of a tree returns itself as a box. Its parent returns
%   itself and its children in a bigger box etc. as shown in the following
%   figure. 

% \begin{quote}
% \begin{tikzpicture}[
%     level 0/.style={draw=black!50,very thick},
%     level 1/.style={draw=orange!50,very thick},
%     level 2/.style={draw=blue!50,very thick},
%     level 3/.style={draw=green!50,very thick}]

%     \node[level 1] (1) {1}
%       child {node[level 2] (3) {3}
%         child {node[level 3] (4) {4}
%           child{node[level 3] (6) {6}}
%           child{node[level 3] (7) {7}}
%         }
%         child {node[level 2] (5) {5}}
%       }
%       child {node[level 1] (2) {2}};

%     \begin{pgfonlayer}{background}
%         \node [level 0, fit=(1) (6) (2)] {};
%         \node [level 1, fit=(3) (5) (6) (7)] {};
%         \node [level 2,fit=(4) (6) (7)] {};
%     \end{pgfonlayer}
% \end{tikzpicture}
% \end{quote}

%   In each step the current boxes can be compared by their size, sorted
%   and positioned. In the figure above the boxes of one step are
%   represented in the same color. 
  
%   In the tree algorithm the boxes of each tree level are first sorted
%   ascendingly by their size and then arranged as follows: The the
%   biggest box is positioned in the middle. Then the following boxes are
%   positioned alternately left and right. 
  
%   After this arrangement the relative coordinates for the position of
%   each box have to be computed. The \emph{y}-coordinate of a box (except
%   for the root node of the step) are determined by the maximum height of
%   all boxes to guarantee a uniform layout of the tree. Nodes on the same
%   level in the tree are positioned at the same height. The
%   \emph{x}-coordinate of a box depends on the coordinates of its left
%   neighbour box and an additional spacing (by default 10pt), which can
%   be influenced by the |sibling distance| key. The \emph{y}-coordinate
%   of the root node of each step is set to the maximum \emph{y}-value of
%   the other boxes adding the same spacing meantioned above (by default
%   10pt, influenced by |level distance|). Its \emph{x}-coordinate is
%   determined by the width of the other boxes divided by 2. This means
%   the root node is positioned in the middle above the other boxes. 
  
%   Because each box knows its root node, it is possible to determine the
%   absolute position of each box or node afterwards.  
  
%   At the end of the step the current boxes are added to a result box and
%   returned. 
%\end{gdalgorithm}


%%% Local Variables: 
%%% mode: latex
%%% TeX-master: "pgfmanual-pdftex-version"
%%% End: 

% Copyright 2012 by Till Tantau
%
% This file may be distributed and/or modified
%
% 1. under the LaTeX Project Public License and/or
% 2. under the GNU Free Documentation License.
%
% See the file doc/generic/pgf/licenses/LICENSE for more details.

\section{Graph Drawing Algorithms: Layered Layouts}

{\emph{by  Till Tantau and Jannis Pohlmann}}

\begin{tikzlibrary}{graphdrawing.layered}
  This library provides keys for drawing graphs using the Sugiyama
  method, which is especially useful for drawing hierachical graphs.
  You should load the |graphdrawing| library first.
\end{tikzlibrary}



\subsection{Overview}

A ``layered'' layout of a graph tries to arrange the nodes in
consecutive horizontal layers (naturally, by rotating the graph, this
can be changed in to vertical layers) such that edges tend to be only
between nodes on adjacent layers. Trees, for instance, can always be
laid out in this way. This method of laying out a graph is especially
useful for hierarchical graphs.

The method implemented in this library is often called the
\emph{Sugiyama method}, which is a rather advanced method of
assigning nodes to layers and positions on these layers. The same
method is also used in the popular GraphViz program, indeed, the
implementation in \tikzname\ is based on the same pseudo-code from the
same paper as the implementation used in GraphViz and both programs
will often generate the same layout (but not always, as explained
below). The current implementation is due to Jannis Pohlmann, who
implemented it as part of his Diploma thesis. Please consult this
thesis for a detailed explanation of the Sugiyama method and its
history:

\begin{itemize}
\item
  Jannis Pohlmann,
  \newblock \emph{Configurable Graph Drawing Algorithms
    for the \tikzname\ Graphics Description Language,}
  \newblock Diploma Thesis,
  \newblock Institute of Theoretical Computer Science, Univerist\"at
  zu L\"ubeck, 2011.\\[.5em]
  \newblock Online at 
  \url{http://www.tcs.uni-luebeck.de/downloads/papers/2011/2011-configurable-graph-drawing-algorithms-jannis-pohlmann.pdf}
  \\[.5em]
  (Note that since the publication of this thesis some option names
  have been changed. Most noticeably, the option name
  |layered drawing| was changed to |layered layout|, which is somewhat
  more consistent with other names used in the graph drawing
  libraries.) 
\end{itemize}

The Sugiyama methods lays out a graph in five steps:
\begin{enumerate}
\item Cycle removal.
\item Layer assignment (sometimes called node ranking).
\item Crossing minimization (also referred to as node ordering).
\item Node positioning (or coordinate assignment).
\item Edge routing.
\end{enumerate}
It turns out that behind each of these steps there lurks an
NP-complete problem, which means, in practice, that each step is
impossible to perform optimally for larger graphs. For this reason,
heuristics and approximation algorithms are used to find a ``good''
way of performing the steps.

A distinctive feature of Pohlmann's implementation of the Sugiyama
method for \tikzname\ is that the algorithms used for each of the
steps can easily be exchanged, just specify a different option. For
the user, this means that by specifying a different 
option and thereby using a different heuristic for one of the steps, a
better layout can often be found. For the researcher, this means that
one can very easily test new approaches and new heuristics without
having to implement all of the other steps anew. 



\subsection{Standard Layered Layout}

In order to compute a layered layout of a graph, use the following option:

\begin{gdalgorithm}{layered layout}{Sugiyama Modular Layered}
  The |layered layout| is the key used to select the modular Sugiyama
  layout algorithm. As explained in the overview of this section, this
  algorithm consists of five consecutive steps, each of which can be
  configured independently of the other ones (how this is done is
  explained later in this section). Naturally, the ``best'' heuristics
  are selected by default, so there is typically no need to change the
  settings, but what is the ``best'' method for one graph need not be
  the best one for another graph.
  
\begin{codeexample}[]
\tikz \graph [layered layout, sibling distance=7mm]
{
  a -> {
    b,
    c -> { d, e, f }
  } ->
  h ->
  a
};    
\end{codeexample}

  As can be seen in the above example, the algorithm will not only
  position the nodes of a graph, but will also perform an edge
  routing. This will look visually quite pleasing if you add the
  |rounded corners| option:

\begin{codeexample}[]
\tikz [rounded corners] \graph [layered layout, sibling distance=7mm]
{
  a -> {
    b,
    c -> { d, e, f }
  } ->
  h -> 
  a
};    
\end{codeexample}


\end{gdalgorithm}



%%% Local Variables: 
%%% mode: latex
%%% TeX-master: "pgfmanual-pdftex-version"
%%% End: 

% Copyright 2011 by Jannis Pohlmann
%
% This file may be distributed and/or modified
%
% 1. under the LaTeX Project Public License and/or
% 2. under the GNU Free Documentation License.
%
% See the file doc/generic/pgf/licenses/LICENSE for more details.

\section{Force-Based Graph Drawing Algorithms}
\label{section-library-graphdrawing-force-based}

{\emph{by Jannis Pohlmann}}


\begin{tikzlibrary}{graphdrawing.force}
  Load this package when you wish to use force-based graph drawing
  algorithms. You should load the |graphdrawing| library first.
\end{tikzlibrary}

\ifluatex\relax\else{LuaTeX is required for setting this manual section.}\expandafter\endinput\fi


\subsection{Overview}

% TODO Jannis: Explain ideas and concepts behind force-based graph
% drawing algorithms. Briefly explain the various approaches in that
% specific area of graph drawing algorithms (e.g. spring,
% spring-electrical and multidimensional embedding). 

...

\subsubsection{Spring and Spring-Electrical Layouts}

% TODO Jannis: Explain ideas and concepts behind spring and
% spring-electrical algorithms. Describe the technical as well as visual
% differences between the two techniques (think: no attractive forces
% and no peripheral effects in spring layouts). Explain why they were
% consolidated in the common family 'spring layout'.

\begin{key}{/graph drawing/spring layout=\meta{options}}
  \keyalias{tikz}\keyalias{tikz/graphs}
  Similar to the |>| option, this ``generic'' name for a spring layout
  algorithm is not hardwired to any specific algorithm. Rather, users
  can select an algorithm somewhere at the beginning of their program
  and then just write |\graph[spring layout]| to draw a tree.

  The \meta{options} will be forwarded to the currently selected
  algorithm.
\begin{codeexample}[]
\tikz \graph [spring layout] { a -> {b,c} };    
\end{codeexample}
  
  To change the algorithm, change the following key:
  \begin{key}{/graph drawing/spring layout/default algorithm=\meta{algorithm}}
    Set this key to the tree drawing algorithm of your choice. The
    default is currently set to the algorithm
    |Walshaw2000 spring electrical|, but this will change. 
  \end{key}
\end{key}


\subsection{Common Options}

The spring and and spring-electrical drawing algorithms are very similar
in terms of their parameters and the constraints they can handle. They
thus share a number of common \tikzname\ options for fine-tuning. These
options are split up into \emph{graph options} that can be specified
once for a graph, \emph{node options} that can be specified for each
node and \emph{edge options} that can be specified for each edge.

\subsubsection{Graph Options}

% TODO Jannis: This might be worth implementing. It's not very useful in
% the Hu2006 algorithm as it uses the Barnes-Hut algorithm, but the 
% Walshaw2000 algorithm can benefit from it.
%
%\begin{key}{/tikz/influence cutoff distance=\meta{dimension} (initially
%  0pt)}
%  Specifies a distance beyond which the attractive and repulsive forces 
%  between two nodes are assumed to be virtually non-existent. If 
%  \meta{dimension} is set to |0pt|, the cutoff distance is computed 
%  automatically.
%
%  Depending on the graph drawing algorithm being used, the distance
%  between two nodes is computed either based on the graph distance
%  (spring algorithm) or based on the Euclidean distance
%  (spring-electrical algorithm).
%  \begin{codeexample}[]
%  \end{codeexample}
%\end{key}

\begin{key}{/graph drawing/spring layout/maximum 
  iterations=\meta{number} (initially 500)}
  Depending on the characteristics of the input graph and the parameters
  chosen for the spring or spring-electrical algorithm, minimizing the
  system energy may require many iterations.

  In these situations it may come in handy to limit the number of
  iterations. This feature can also be useful to draw the same graph
  after different iterations and thereby demonstrate how the spring or
  spring-electrical algorithm improves the drawing step by step.
  \begin{codeexample}[]
\tikz \graph [spring layout={maximum iterations=1}]   { a -- b -- c -- a };
\tikz \graph [spring layout={maximum iterations=10}]  { a -- b -- c -- a };
\tikz \graph [spring layout={maximum iterations=500}] { a -- b -- c -- a };
  \end{codeexample}
\end{key}

\begin{key}{/graph drawing/spring layout/random seed=\meta{number} 
  (initially 42)}
  Specifies the seed used for Lua's pseudo-random number generator. If
  set to something other than |0|, the random number sequence generated
  by the pseudo-random number generator will be the same at every run.
  The resulting graph drawings will be reproducible in consecutive runs,
  despite randomized elements used in the algorithm.
  If set to |0|, the results are not guaranteed to be reproducible.
  \begin{codeexample}[width=5.5cm]
\tikz \graph [spring layout={random seed=1}] { 
  subgraph K_n[n=4]
};
\tikz \graph [spring layout={random seed=10}] { 
  subgraph K_n[n=4]
};
  \end{codeexample}
\end{key}

\begin{key}{/graph drawing/spring layout/natural spring
  dimension=\meta{dimension} (initially 1cm)}
\end{key}

\begin{key}{/graph drawing/spring layout/spring constant=\meta{number}}
\end{key}

\begin{key}{/graph drawing/spring layout/approximate repulsive
  forces=\opt{\meta{boolean}} (initially true)}
\end{key}

\begin{key}{/graph drawing/spring layout/cooling factor=\meta{number}
  (initially 0.95)}
\end{key}

\begin{key}{/graph drawing/spring layout/convergence
  tolerance=\meta{number} (initially 0.01)}
\end{key}

\begin{key}{/graph drawing/approximate repulsive 
  forces=\opt{\meta{boolean}} (initially false)}
  
  Computing the repulsive forces of the nodes in a graph requires 
  $\mathcal{O}(n^2)$ operations in each iteration of spring- and
  spring-electrical algorithms, where $n$ is the number of nodes in
  the graph. For $l$ coarse graphs, this may increase the runtime of 
  such an algorithm by up to $l\cdot\mathcal{O}(n^2)$ operations. 

  With |approximate repulsive forces| set to |true|, repulsive forces 
  are approximated using the Barnes-Hut algorithm known from solving the
  so-called $N$-body problem. This reduces the number of operations
  needed to compute the repulsive forces to $\mathcal{O}(n\log n)$ per 
  iteration and can thus lead to a significant improvement of the 
  algorithm runtime.
  
  However, this optimization \emph{can} come at the cost of slightly 
  less appealing drawings which is noticable with small graphs in 
  particular. This is why it is turned off by default. Enable it if you
  want to lay out large graphs.

  Here is an example where this disadvantage can be noticed:
  \begin{codeexample}[]
\tikz \graph [spring layout] { 
  { [clique] 3, 5, 1 } -- { [clique] 2, 4, 6 }
};
\tikz \graph [spring layout={approximate repulsive forces}] { 
  { [clique] 3, 5, 1 } -- { [clique] 2, 4, 6 }
};
  \end{codeexample}

  Sometimes, the negative effect is very subtle. Notice how the angle
  of the |2| |4| |6| clique is slightly less appealing in the drawing
  with repulsive forces approximated.
  \begin{codeexample}[]
\tikz \graph [spring layout,orient=2|1] { 
  { 1 -- 3 -- 5 -- 1, 1 -- 2, 2 -- 4 -- 6 -- 2}
};
\tikz \graph [spring layout={approximate repulsive forces},orient=2|1] { 
  { 1 -- 3 -- 5 -- 1, 1 -- 2, 2 -- 4 -- 6 -- 2}
};
  \end{codeexample}

  In the the following example the opposite is the case even. Here,
  approximating the repulsive force generates a better layout than
  computing them accurately:
  \begin{codeexample}[width=6cm]
\tikz \graph [spring layout,orient=1|2] { 
  subgraph Grid_n[n=9]
};
\tikz \graph [spring layout={approximate repulsive forces},
              orient=1|2] { 
  subgraph Grid_n[n=9]
};
  \end{codeexample}
  
  As you can see it is dependent on the graph and other parameters of
  the spring and spring-electrical algorithms as to whether or not it
  makes sense to enable |approximate repulsive forces|.
\end{key}

\begin{key}{/graph drawing/spring layout/coarsen=\opt{\meta{boolean}}
  (initially true)}
  Defines whether or not a multilevel approach is used that
  iteratively coarsens the input graph into graphs $G_1,\dots,G_l$ with 
  a smaller and smaller number of nodes. The coarsening stops as soon as
  a minimum number of nodes is reached, as set via the 
  |minimum graph size| option or when, in the last iteration, the 
  number of nodes was not reduced by at least the ratio specified via 
  |downsize ratio|. 

  A random initial layout is computed for the coarsest graph $G_l$ first.
  Afterwards, it is laid out by computing the attractive and repulsive
  forces between its nodes. 
  
  In the subsequent steps, the previous coarse graph $G_{l-1}$ is 
  restored and its node positions are interpolated from the nodes in 
  $G_l$. $G_{l-1}$ is again laid out by computing the forces between 
  its nodes. These steps are repeated with $G_{l-2},\dots,G_1$ until 
  the original input graph $G_0$ has been restored, interpolated 
  and laid out.

  There are a number of options to fine-tune the coarsening approach.
  They are consolidated in the |/graph drawing/spring layout/coarsening|
  prefix described below.
\end{key}

\begin{key}{/graph drawing/spring layout/coarsening=\marg{options}}
  Executes the \meta{options} with the path prefix 
  |/graph drawing/spring layout/coarsening|.

  These options can be used to configure the coarsening approach
  described in the documentation of the 
  |/graph drawing/spring layout/coarsen| option.
\end{key}

\begin{key}{/graph drawing/spring layout/coarsening/minimum graph
  size=\meta{number} (initially 2)}
  Defines the number of nodes down to which the graph is coarsened
  iteratively. The first graph that has a lesser or equal number of
  nodes becomes the coarsest graph $G_l$, where $l$ is the number of
  coarsening steps. The algorithm proceeds with the steps described in
  the documentation of the |/graph drawing/spring layout/coarsen|
  option.

  In the following example the same graph is coarsened down to two
  and three nodes, respectively. The layout of the original graph is 
  interpolated from the random initial layout and is not changed
  because the forces are not computed. Thus, in the first graph, the
  nodes can have exactly two (or three) possible coordinates in the
  final drawing:
  \begin{codeexample}[]
\tikz \graph [spring layout={maximum iterations=0,coarsen,
                             coarsening={minimum graph size=2}}] { 
  1 -- 2 -- 3 -- 4 
};
\tikz \graph [spring layout={maximum iterations=0,coarsen,
                             coarsening={minimum graph size=3}}] { 
  1 -- 2 -- 3 -- 4 
};
  \end{codeexample}
\end{key}

\begin{key}{/graph drawing/spring layout/coarsening/downsize
  ratio=\meta{number} (initially 0.25)}
\end{key}

\begin{key}{/graph drawing/spring layout/coarsening/collapse independent
  edges=\opt{\meta{boolean}} (initially true)}
\end{key}

\begin{key}{/graph drawing/spring layout/coarsening/connected
  independent nodes=\opt{\meta{boolean}} (initially false)}
\end{key}

%\begin{key}{/tikz/coarsening=\marg{options}}
%  Executes the \meta{options} with the path prefix |/tikz/coarsening|.
%  
%  These options define whether a multilevel approach is used that
%  successively coarsend into graphs with smaller and smaller number
%  of nodes. These graphs are arranged first and are then interpolated
%  into the finer graphs at the previous level. How this is done exactly
%  can be configured using the |coarsening| options described below.
%\end{key}
%
%\begin{key}{/tikz/coarsening/randomized=\opt{\meta{boolean}} (default
%  true, initially false)}
%  If set to |true|, nodes will be inspected in a random order. The
%  effect on the final drawing can only be seen by experimenting with the
%  option.
%  \begin{codeexample}[]
%  \end{codeexample}
%\end{key}
%
%\begin{key}{/tikz/coarsening/minimum size=\meta{number} (default 0)}
%  Defines the minimum number of nodes in a coarsened graph. If a
%  coarsened graph has less than \meta{number} nodes, then... % TODO
%  \begin{codeexample}[] 
%% the same graph with different minimum size values
%  \end{codeexample}
%\end{key}
%
%\begin{key}{/tikz/coarsening/nodes=\opt{\meta{boolean}} (default true,
%  initially false)}
%  \begin{codeexample}[]
%  \end{codeexample}
%\end{key}
%
%\begin{key}{/tikz/coarsening/nearby nodes=\opt{\meta{boolean}} (default
%  true, initially false)}
%  \begin{codeexample}[]
%  \end{codeexample}
%\end{key}
%
%\begin{key}{/tikz/coarsening/nodes with more 
%  neighbors=\opt{\meta{boolean}} (default true, initially false)}
%  \begin{codeexample}[]
%  \end{codeexample}
%\end{key}
%
%\begin{key}{/tikz/coarsening/nearby nodes with more 
%  neighbors=\opt{\meta{boolean}} (default true, initially false)}
%  \begin{codeexample}[]
%  \end{codeexample}
%\end{key}
%
%\begin{key}{/tikz/coarsening/edges=\opt{\meta{boolean}} (default true,
%  initially false)}
%  \begin{codeexample}[]
%  \end{codeexample}
%\end{key}
%
%\begin{key}{/tikz/coarsening/heavy edges=\opt{\meta{boolean}} (default
%  true, initially false)}
%  \begin{codeexample}[]
%  \end{codeexample}
%\end{key}
%
%\begin{key}{/tikz/coarsening/edges with light nodes=\opt{\meta{boolean}}
%  (default true, initially false)}
%  \begin{codeexample}[]
%  \end{codeexample}
%\end{key}
%
%\begin{key}{/tikz/minimum energy delta=\meta{number} (default TODO)}
%  \begin{codeexample}[]
%  \end{codeexample}
%\end{key}
%
%\begin{key}{/tikz/initial step size=\meta{dimension} (default TODO)}
%  \begin{codeexample}[]
%  \end{codeexample}
%\end{key}
%
%\begin{key}{/tikz/step control=\meta{text} (default TODO)}
%  Possible values: |monotonic|, |non-monotonic|, |strictly monotonic|.
%  \begin{codeexample}[]
%  \end{codeexample}
%\end{key}

\subsubsection{Node Options}

%\begin{key}{/tikz/electric charge=\meta{number} (default 1)}
%  Defines the electric charge of the node. The stronger the electric
%  charge, the higher the repulsive force between two nodes. Set this to
%  something between |0| and |1| to reduce the charge compared to the
%  normal setup. Values larger than |1| will generate stronger repulsion
%  between the node and the others.
%  \begin{codeexample}[] 
%\tikz \graph [spring electrical layout,orient=1:90:2] {
%  1 -- 2 -- 3 -- 4 -- 1,
%  1 -- 3, 2 -- 4,
%};
%\tikz \graph [spring electrical layout,orient=1:90:2] {
%  1 [electric charge=1] -- 2 -- 3 -- 4 -- 1,
%  1 -- 3, 2 -- 4,
%};
%\tikz \graph [spring electrical layout,orient=1:90:2] {
%  1 [electric charge=1000] -- 2 [electric charge=1000] -- 3 -- 4 -- 1,
%  1 -- 3, 2 -- 4,
%};
%  \end{codeexample}
%\end{key}

%\end{document}

% TODO Jannis: Explain this one better. Also, compare it to /tikz/at,
% which will move the node after the drawing has been computed, as
% opposed to /graph drawing/desired at, which will only move the node
% while computing the layout.
%
%\begin{key}{/tikz/desired at=\meta{coordinate}}
%  Nails the node down at the specified \meta{coordinate}. It will not
%  move from there despite the repulsive and attractive forces in the
%  system. Note that, while sometimes generating a similar effect, using
%  |at| is very different from altering the orientation of a graph
%  drawing (see section~\ref{subsection-library-graphdrawing-standard-orientation}).
%  Also, if an orientation is specified, it is given priority over
%  the |at| option in that nodes are first fixated at their |at|
%  coordinates but are later moved in order to satisfy the orientation 
%  desired by the user.
%  \begin{codeexample}[width=6.0cm]
%\tikz \graph [spring layout] {
%  1 -- 2 -- 3 -- 4 -- 2
%};
%\tikz \graph [spring layout] {
%  1 [at={(0,0)}] -- 2 [at={(0,1)}] -- 3 -- 4 -- 2
%};
%  \end{codeexample}
%\end{key}


% TODO what about node groups / clusters? This works via color classes
% but how do we define their layouts (cluster, line, circle)?

\subsubsection{Edge Options}

\begin{key}{/tikz/natural length=\meta{dimension} (default 10pt)}
  Defines the natural (zero energy) length of the edge. The smaller the
  length, the stronger the attractive force of the adjacent nodes. The
  \meta{dimension} has a strong influence of how far the nodes will be
  placed from each other in the final drawing.
  \begin{codeexample}[]
% two examples with the same graph
% notably change the natural length of one of the edges
  \end{codeexample}
\end{key}

\begin{key}{/tikz/stiffness=\meta{number} (default 0.5)}
  Defines how flexible the spring associated with the edge is. The
  higher this value is, the closer the final edge length will be to its
  |natural length|.
  \begin{codeexample}[]
% two examples with the same graph
% notably change the stiffness of one of the edges
  \end{codeexample}
\end{key}

\subsection{Options for the Spring Algorithm}

\subsubsection{Graph Options}

...

\subsubsection{Node Options}

...

\subsubsection{Edge Options}

...

\subsection{Options for the Spring-Electrical Algorithm}

\subsubsection{Graph Options}

...

\subsubsection{Node Options}

...

\subsubsection{Edge Options}

...

\begin{codeexample}[]
\vbox{ \hsize=16cm \rightskip=0cm plus 1fill
  \foreach \iterations in {1,...,20,100,500}
  {
    \tikz \graph [spring layout={maximum iterations=\iterations}, orient=1-2] 
      { subgraph K_n[n=4] };
    \penalty0
  }
}
\end{codeexample}

\begin{codeexample}[]
\vbox{ \hsize=16cm \rightskip=0cm plus 1fill
  \foreach \iterations in {1,...,20,100,500}
  {
    \tikz \graph [spring layout={maximum iterations=\iterations}, orient=1-2] 
      { subgraph C_n[n=7] };
    \penalty0
  }
}
\end{codeexample}

\endinput

%% TODO
%% Explain the following concepts:
%% - separation of graph drawing options and regular TikZ options
%% - generic graph drawing options:
%%   - component packing
%%   - orientation
%% - pre-defined graph drawing styles
%% - graph drawing options for fine-tuning the different algorithms

%%% Local Variables: 
%%% mode: latex
%%% TeX-master: "pgfmanual-pdftex-version"
%%% End: 

% Copyright 2011 by Jannis Pohlmann
%
% This file may be distributed and/or modified
%
% 1. under the LaTeX Project Public License and/or
% 2. under the GNU Free Documentation License.
%
% See the file doc/generic/pgf/licenses/LICENSE for more details.

\section{Graph Drawing Algorithms: Circular Layouts}


\begin{tikzlibrary}{graphdrawing.circular}
  Load this package when you wish to use the graph drawing algorithms
  that place nodes on circles. You should load the |graphdrawing| library first.
\end{tikzlibrary}


\begin{gdalgorithm}{circular layout}{Circular Layout Tantau 2012}
  This layout arranges the nodes in a circle. The centers of the nodes
  are placed on a counter-clockwise circle, starting with the first
  node at the |grow| direction (for |grow'|, the circle is
  clockwise). The order of the nodes is the order in which they appear
  in the graph, the edges are not taken into consideration.

\begin{codeexample}[]
\tikz[>=spaced stealth']
  \graph [circular layout, grow'=down, sibling sep=1em,
          nodes={draw,circle}, math nodes]
  {
    x_1 -> x_2 -> x_3 -> x_4 ->
    x_5 -> "\dots"[draw=none] -> "x_{n-1}" -> x_n -> x_1
  };    
\end{codeexample}

  The nodes are placed in such a way that
  \begin{enumerate}
  \item The (angular) distance between the centers of consecutive
    nodes is at least  |sibling distance|,
  \item the distance between the borders of consecutive nodes is at
    least |sibling sep|, and
  \item the radius is at least |circular layout/radius|.
    \begin{key}{/graph drawing/circular layout/radius=\meta{radius}}
      The minimum radius of the circle.
    \end{key}
  \end{enumerate}
  The radius of the circle is chosen near-minimal such that the above
  properties are satisfied. To be more precise, if all nodes are
  circles, the radius is chosen optimally while for, say, rectangular
  nodes there may be too much space between the nodes in order to
  satisfy the second condition.

\begin{codeexample}[]
\tikz \graph [circular layout,
          sibling sep=0pt, sibling distance=0pt,
          nodes={draw,circle}]
  { 1 -- 2 [minimum size=30pt] -- 3 --
    4 [minimum size=50pt] -- 5 [minimum size=40pt] -- 6 -- 7 }; 
\end{codeexample}

\begin{codeexample}[]
\begin{tikzpicture}
  \graph [circular layout={radius=1.25cm},
          sibling sep=0pt, sibling distance=0pt,
          nodes={draw,circle}]
  { 1 -- 2 [minimum size=30pt] -- 3 --
    4 [minimum size=50pt] -- 5 [minimum size=40pt] -- 6 -- 7 }; 
  
  \draw [red] (0,-1.25) circle [radius=1.25cm];
\end{tikzpicture}
\end{codeexample}

\begin{codeexample}[]
\tikz \graph [circular layout,
    sibling sep=0pt, sibling distance=1cm,
    nodes={draw,circle}]
  { 1 -- 2 [minimum size=30pt] -- 3 --
    4 [minimum size=50pt] -- 5 [minimum size=40pt] -- 6 -- 7 }; 
\end{codeexample}

\begin{codeexample}[]
\tikz \graph [circular layout,
    sibling sep=2pt, sibling distance=0pt,
    nodes={draw,circle}]
  { 1 -- 2 [minimum size=30pt] -- 3 --
    4 [minimum size=50pt] -- 5 [minimum size=40pt] -- 6 -- 7 }; 
\end{codeexample}

\begin{codeexample}[]
\tikz \graph [circular layout,
    sibling sep=0pt, sibling distance=0pt,
    nodes={rectangle,draw}]
  { 1 -- 2 [minimum size=30pt] -- 3 --
    4 [minimum size=50pt] -- 5 [minimum size=40pt] -- 6 -- 7 }; 
\end{codeexample}
\end{gdalgorithm}




%%% Local Variables: 
%%% mode: latex
%%% TeX-master: "pgfmanual-pdftex-version"
%%% End: 

% Copyright 2011 by Jannis Pohlmann
%
% This file may be distributed and/or modified
%
% 1. under the LaTeX Project Public License and/or
% 2. under the GNU Free Documentation License.
%
% See the file doc/generic/pgf/licenses/LICENSE for more details.

\section{Graph Drawing Layouts: Miscellaneous}
\label{section-last-graphdrawing-library-in-manual}


\begin{tikzlibrary}{graphdrawing.misc}
  Load this package when you wish to use the graph drawing algorithms
  defined in this library. You should load the |graphdrawing| library first.
\end{tikzlibrary}


\begin{gdalgorithm}{circular layout}{Circular Layout Tantau 2012}
  TODO: Document this...

\begin{codeexample}[]
\tikz[>=spaced stealth']
  \graph [circular layout, grow'=down, sibling sep=1em,
          nodes={draw,circle}, math nodes]
  {
    x_1 -> x_2 -> x_3 -> x_4 ->
    x_5 -> "\dots"[draw=none] -> "x_{n-1}" -> x_n -> x_1
  };    
\end{codeexample}

\begin{codeexample}[]
\tikz[>=spaced stealth']
  \graph [circular layout, grow'=30, sibling sep=1em,
          nodes={draw,circle}]
  { subgraph K_n [n=8] }; 
\end{codeexample}
\end{gdalgorithm}

\begin{gdalgorithm}{simple demo layout}{Simple Demo}
  The algorithm used in the examples of this manual for demonstrating
  how a trivial graph drawing can be implemented.
\end{gdalgorithm}




%%% Local Variables: 
%%% mode: latex
%%% TeX-master: "pgfmanual-pdftex-version"
%%% End: 

% Copyright 2010-2011 by Renée Ahrens
% Copyright 2010-2011 by Olof Frahm
% Copyright 2010-2011 by Jens Kluttig
% Copyright 2010-2011 by Matthias Schulz
% Copyright 2010-2011 by Stephan Schuster
% Copyright 2011 by Jannis Pohlmann
%
% This file may be distributed and/or modified
%
% 1. under the LaTeX Project Public License and/or
% 2. under the GNU Free Documentation License.
%
% See the file doc/generic/pgf/licenses/LICENSE for more details.

\section{Graph Drawing Internals}
\label{section-base-graphdrawing}

\ifluatex\relax\else{LuaTeX is required for setting this manual section.}\endinput\fi

As mentioned before (\ref{section-library-graphdrawing}), the graph
drawing library makes use of Lua. But where does the control flow
leave \TeX\ and what happens to your \tikzname\ nodes? The subsequent
sections will discuss this process in deep. The general approach is to
intercept the immediate placement of the nodes and pass them down to
Lua, which does all the placement stuff. After the selected graph
drawing algorithm has finished, it writes the nodes back to
\tikzname\ to have the graph drawn.

This proceeding consists of a front end layer for \tikzname, an
interface to Lua and of course a set of Lua classes to represent the
graph. An algorithms can be developed independently. Only knowledge
about the Lua interface is required; specific \TeX\ programming skills
not necessary.

\subsubsection{The Front End Layer}
Let's have a look at a simple example to see what the front end looks
like:

\begin{codeexample}[]
\tikzpicture[graph drawing={standard tree},scale=2]
  \graph{root [as=Hello,root] -> World[fill=blue!20]};
\endtikzpicture
\end{codeexample}

As you may see, the syntax is exactly the same as described in the
chapter about specifying graphs (section~\ref{section-library-graphs}).

You enable this library with the key |graph drawing|, which sets the
algorithm to use and its specific parameters. All other
\tikzname\ keys are accepted as well, like |scale| in the example
above. Each algorithm has its own keys to parametrize it. Please refer to
the appropriate sections for more information.

The keys are given within the |graph drawing| key family for graph options and per node for node specific options. Furthermore you can use any valid \tikzname\ keys as usual. 

There are some things which will not work with the graph drawing
library, like preordering the nodes. Consider for example the
|chain shift| key of the graphs library to place the nodes on a
certain grid: 

\begin{codeexample}[]
\tikzpicture
  \graph[chain shift=(45:1)] {
    a -> b -> c;
    d -> e;
  };
\endtikzpicture
\end{codeexample}

The graph drawing library does not take care of any predefined layout options by now, so the above example will be set differently:

\begin{codeexample}[]
\tikzpicture[graph drawing={few intersections}, scale=2]
  \graph[chain shift=(45:1)] {
    a -> b -> c;
    d -> e;
  };
\endtikzpicture
\end{codeexample}

A graph drawing algorithm will always place the nodes in its own manner. 

% what is happening in the tikz..tex file. Matthias

\subsubsection{The Interface to Lua}
The main entry point for the library to Lua is defined in the
appropriate |code| file of the library. It employs three Lua classes
to create graphs, pass down nodes and to communicate the given
options.

An overview of what happens is illustrated by the following call graph:

\begin{tikzpicture}[
    class name/.style={draw,minimum size=20pt, fill=blue!20},
    object node/.style={draw,minimum size=15pt, fill=yellow!20},
    p/.style={->,>=stealth},
    livespan/.style={thick,double},
    scale=0.9]
  % class names above
  \node (tikz) at (0,4) [class name] {\tikzname\ graph};
  \node (tex) at (5,4) [class name] {\TeX\ Interface};
  \node (interface) at (10,4) [class name] {Lua Interface};
  \node (sys) at (15,4) [class name] {Sys};
  % lines from the class names to the bottom of the picture
  \draw[livespan] (tikz) -- (0,-6.5);
  \draw[livespan] (tex) -- (5,-6.5);
  \draw[livespan] (interface) -- (10,-6.5);
  \draw[livespan] (sys) -- (15,-6.5);
  % first command: \graph{  -- generates new graph in lua interface
  \node (tikz-begin-graph) at (0,3) [object node] {|\graph{|}; %}
  \node (tex-begin-graph) at (5,3) [object node] {|\pgfgdbeginscope|};  
  \node (interface-new-graph) at (10,3) [object node] {|newGraph(|...|)|};
  \draw [p] (tikz-begin-graph.east) -- (tex-begin-graph.west);
  \draw [p] (tex-begin-graph.east) -- (interface-new-graph.west);    
  % second command: a -> b   -- generates two nodes in lua
  % and one edge
  \node (tikz-node) at (0,2) [object node] {|a -> b;|};
  \node (tex-node) at (5,2) [object node] {|\pgf@gd@positionnode@callback|};
  \node (interface-add-node-behind) at (10.1,1.9) [object node] {|addNode(|...|)|};
  \draw[p] (tikz-node.east) -- (tex-node.west);
  
  \node (interface-add-node) at (10,2) [object node] {|addNode(|...|)|};
  \draw[p] (tex-node.east) -- (interface-add-node.west);

  \node (tex-add-edge) at (5,1) [object node] {|\pgfgdaddedge|};
  \node (interface-add-edge) at (10,1) [object node] {|addEdge(|...|)|};
  \draw[p] (tikz-node.east) -- (1.5,2) -- (1.5,1) -- (tex-add-edge.west);
  \draw[p] (tex-add-edge.east) -- (interface-add-edge.west);

  % scope ends -- cloes graph, layouts it and draws it
  \node (tikz-end) at (0,0) [object node] {|};|};
  \node (tex-end) at (5,0) [object node] {|\pgfgdendscope|};
  \node (interface-draw-graph) at (10,0) [object node] {|drawGraph()|};
  \node (interface-finish-graph) at (10,-2) [object node] {|finishGraph()|};

  \node (invoke-algorithm) at (12.5,-1) [object node] {invoke algorithm};
  \draw[p] (tikz-end.east) -- (tex-end.west);
  \draw[p] (tex-end.east) -- (interface-draw-graph.west);
  \draw[p] (interface-draw-graph.east) -- (12.5,0) -- (invoke-algorithm.north);
  \draw[p] (tex-end.east) -- (7.5,0) -- (7.5,-2) -- (interface-finish-graph.west);

  % begin shipout
  \node (sys-begin-shipout) at (15,-2) [object node] {|beginShipout()|};
  \draw[p] (interface-finish-graph.east) -- (sys-begin-shipout.west);
  \node (tex-begin-shipout) at (5,-3) [object node] {|\pgfgdbeginshipout|};
  \draw[p] (sys-begin-shipout.187) -- (12,-2.2) -- (12,-3) -- (tex-begin-shipout.east);

  % put tex box
  \node (sys-puttexbox-behind) at (15.1,-4.1) [object node] {|putTeXBox(|...|)|};
  \node (sys-puttexbox) at (15,-4) [object node] {|putTeXBox(|...|)|};
  \node (tex-puttexbox) at (5,-4) [object node] {|\pgfgdinternalshipoutnode|};

  \draw[p] (12.5,-2) -- (12.5,-4) -- (sys-puttexbox.west);
  %(interface-finish-graph.east) -- (12.5,-2) -- (12.5,-4) -- (sys-puttexbox.west);
  \draw[p] (sys-puttexbox.187) -- (12,-4.2) -- (12,-4) -- (tex-puttexbox.east);

  % put edge
  \node (sys-put-edge-behind) at (15.1,-5.1) [object node] {|putEdge(|...|)|};
  \node (sys-put-edge) at (15,-5) [object node] {|putEdge(|...|)|};
  \draw[p] (12.5,-4) -- (12.5,-5) -- (sys-put-edge.west);
  %(interface-finish-graph.east) -- (12.5,-2) -- (12.5,-5) -- (sys-put-edge.west);
  % end shipout
  \node (sys-end-shipout) at (15,-6) [object node] {|endShipout()|};
  \draw[p] (12.5,-5) -- (12.5,-6) -- (sys-end-shipout.west);
  %(interface-finish-graph.east) -- (12.5,-2) -- (12.5,-6) -- (sys-end-shipout.west);
  \node (tex-end-shipout) at (5,-6) [object node] {|\pgfgdendshipout|};
  \draw[p] (sys-end-shipout.187) -- (12,-6.175) -- (12,-6) -- (tex-end-shipout.east);
\end{tikzpicture}


\paragraph{The \TeX\ side.}
\label{section-library-graphdrawing-the-tex-side}

In order to layout a graph, we need to keep \tikzname\ from placing the nodes immediately. This is done using the macro
|\pgfpositionnodelater| as described in chapter~\ref{section-shapes},
subchapter~\ref{section-shapes-deferred-node-positioning}. 

In short terms this works as follows: This macro takes another \meta{macro} as
first argument. If this is |\relax|, the behaviour is to immediately
place the node into the current picture. Any other \meta{macro} that is passed
will be executed. It works like a
callback function -- the node will be put into a box register, the
name of the node and the bounding box coordinates are stored in
separate macros and afterwards \meta{macro} will be called.

As we have to make sure, that the unplaced node will not be referenced
by \tikzname\ keys like |right of|, it is temporarily renamed to
|not yet positionedPGFGDINTERNAL|\meta{nodename}.

To finally
insert the node into the picture, we need to set the mentioned macros
and put the node into the box register. Then we can call
|\pgfpositionnodenow| with the target coordinates of the node.

The code file of the graph drawing library sets the callback function
at the beginning of a graph drawing scope, e.g.\ when a |\graph|
starts. This can also be triggered using |\pgfgdbeginscope| and
|\pgfgdendscope|, which can be used to create a sub scope in an
existing graph drawing scope. Opening a scope yields in creating a new
graph on the Lua graph stack. All subsequent operations (like adding
nodes or edges) apply to the top of the stack. 

%by now this leads to an infinite loop . when its fixed, the example
%can be uncommented :)
% \begin{codeexample}[]
% \tikzpicture[graph drawing={few intersections}, scale=2]
% \graph{
%   a->b;
% %  \graph{c->d;}; TODO: triggers an infinite loop.
%   };
% \endtikzpicture
% \end{codeexample}

The callback function gets all option keys in
|/tikz/graphs/graph drawing/|, copies the box register and passes all information down to the Lua interface class.

When the library is loaded, it initialises the Lua subsystem. This takes place by checking if \LuaTeX\ is present and then invoking the Lua loader class. 

The library code file consists mainly the following macros:

\begin{command}{\pgfgdbeginscope}
  The begin scope macro opens a new graph drawing scope. This creates a new graph object on the top of the Lua graph stack. All subsequent operations will work on this graph until |\pgfgdendscope| will be called.

It is not necessary to call it manually, because in a graph drawing environment it is executed by default at the beginning of a |\graph| statement.
\end{command}


\makeatletter
\begin{command}{\pgf@gd@positionnode@callback}
  This macro saves the keys from |/tikz/graphs/graph drawing/| into a temporary macro, sets the box register |\pgf@gd@box| to the |\pgfpositionnodelaterbox| and passes these informations down to Lua. Additionally the node name and the bounding box is passed down, too. This macro is only used internally.
\end{command}
\makeatother

\begin{command}{\pgfgdaddedge\marg{from}\marg{to}\marg{direction}}
  Adds an edge to the Lua graph object. It requires the name of the target node \meta{from}, the destination node \meta{to} with a distinct \meta{direction} like |->|.

  It is called when a |->|, |--|, |<-| or |-!-| is encountered in a graph.
\end{command}

\begin{command}{\pgfgdendscope}
  At the end of a graph drawing scope the selected algorithm runs and layouts the graph. After finishing this task the macro pops the graph from the stack.
\end{command}

\begin{command}{\pgfgdbeginshipout}
  When the layout is completed and the scope ended, this macro places a |\scope| into the output stream. The layouted graph will be placed inside an extra scope.
\end{command}

\begin{command}{\pgfgdinternalshipoutnode\marg{name}\marg{x min}\marg{x max}\marg{y min}\marg{y max}\marg{x pos}\marg{y pos}\marg{box}}
  When the algorithm finished the layout and the scope ended, the nodes have to be passed back to \tikzname. This macro takes the name of the node, the bounding box, the newly computed position and a box register number. It restores the macros set by |\pgfpositionnodelater| as mentioned above, fills the box register |\pgfpositionnodelaterbox| and then calls |\pgfpositionnodenow| with the coordinates of the node. This macro inserts the node into the current picture.
\end{command}

\begin{command}{\pgfgdendshipout}
  Issues a |\endscope| macro to close the scope opened by |\pgfgdbeginshipout|.
\end{command}

\paragraph{Lua interface class.}

The class |Interface| is the main entry point in Lua. Every communication from \TeX\ to Lua is done here.
It provides methods to create graphs, add nodes and edges to graphs and finally to invoke the selected algorithm. The |Interface| class manages the stack of graphs.

When the |newGraph()| function is called, it generates a new graph object and pushes it on the graph stack. The methods |addNode()| and |addEdge()| are called for each node and each edge, creating the actual Lua objects and adding them to the current graph.

After adding nodes and edges, when the scope ends, the interface invokes the actual algorithm to layout the graph. This is done in the |drawGraph()| function. The next step is to put the nodes back in the \TeX\ output stream. This is invoked by the |finishGraph()| method.

For a reference about the functions and their usage, please refer to section~\ref{section-library-graphdrawing-lua-documentation-interface}.

\paragraph{Lua system class.}

Communication with \TeX\ on a basic layer is done in the |Sys| class. The |beginShipout()| function opens a new scope in \tikzname\ to put all graph drawing nodes into. This prevents other graph objects outside the graph drawing scope from referencing these nodes. The |endShipout()| method closes the scope.

Nodes and edges are put in the output stream by the methods |putTeXBox()| and |putEdge()|. The first calls the |\pgfgdinternalshipoutnode| macro, which is explained in section~\ref{section-library-graphdrawing-the-tex-side}. The latter method writes the appropriate |\draw| directly to the output stream. 

For a reference about the functions and their usage, please refer to section~\ref{section-library-graphdrawing-lua-documentation-sys}.

\subsubsection{Lua Graph Representation}
Most classes in the framework (including the module objects) implement
the |__tostring| method, meaning that you can get a somewhat useful
string representation of the object via the standard |tostring|
function.

The main class which contains references to all other objects is
|Graph|.  New graphs are usually created automatically, so common ways
to get new graph objects are the |copy| method, which creates a
shallow copy (without coying nodes or edges), and the
|subGraphParent| method, which creates a deep copy of the graph, edge
and node objects starting at a designated parent node. If you need
more control by supplying your own set of already visited nodes, use
the underlying function |subGraph|.

A graph allows you to add and remove nodes and edges via |addNode|,
|addEdge|, |removeNode| and |removeEdge| respectively.  There are also
variants which remove all incident edges on a node removal and
conversely, |deleteNode| and |deleteEdge|.

Only nodes can be looked up by name with |findNode|, a
method implemented using the more generic |findNodeIf|, which supports
an arbitrary test predicate.

Lastly the |walkDepth| and |walkBreadth| methods may be used to get
iterators over all nodes and edges in a depth-first or breadth-first
order (other traversal orders may require a rewrite or extension of the
|walkAux| method).

Positions are represented using the dedicated class |Position|, the member
variables |x| and |y| contain the coordinates.  Positions can also be
relative to other positions, which can be tested using |isAbsPosition|.
The conversion to absolute coordinates is done with |getAbsCoordinates|.
The equality test method implements comparing two positions by using their
absolute positions.

For a detailed description of the mentioned classes and methods refer
to section~\ref{section-library-graphdrawing-lua-documentation-graphrep}.

\paragraph{Common graph operations.}
The following tasks are typical for manipulating the graph.
Those snippets will get you started even if you do not have any Lua
experience.

\begin{itemize}
\item Iterate over all nodes.
\begin{codeexample}[code only]
for node in values(graph.nodes) do
   ...
end
\end{codeexample}
\item Get or set width/height of a node, e.g.\ for measuring.
\begin{codeexample}[code only]
local width, height = node.width, node.height
\end{codeexample}
\item Get or set x-y-coordinates of a node.
\begin{codeexample}[code only]
node.pos.x = node.pos.x + 1
node.pos.y = node.pos.y + 1
\end{codeexample}
\item Relate the position of node to the position of another.
\begin{codeexample}[code only]
newNode.pos.x, newNode.pos.y = 1, 1
--sets position of newNode 1 pt in y- and x-direction relative to node
newNode.pos:relateTo(node.pos)
\end{codeexample}
\item Get absolute x-y-coordinate of node, with or without relative coordinates.
\begin{codeexample}[code only]
absX, absY = newNode:getAbsCoordinates()
\end{codeexample}
\item Iterate over all edges and all nodes of the current edge.
\begin{codeexample}[code only]
for edge in values(graph.edges) do
   for node in values(edge:getNodes()) do
      ...
   end
end
\end{codeexample}
\item Get the nodes connected by an edge.
\begin{codeexample}[code only]
local nodeLeft = edge:getNodes()[1]
local nodeRight = edge:getNodes()[2]
\end{codeexample}
\end{itemize}

A full example for a user-defined algorithm is shown in
section~\ref{section-library-graphdrawing-ownAlgorithm}.

\subsection{Registering graph drawing keys}
\label{section-base-graphdrawing-registerKeys}

Graphs and nodes in Lua have specific options, like the name of the
algorithm to use. These keys are registered on the \tikzname\ layer.


\begin{stylekey}{/tikz/graphs/graph drawing/register key}
  The argument of this style is registered as a new key for a
  graph. The name of the key and it's value will be passed down to the
  Lua graph object and should be used for algorithm-wide options. 

  An example is the |algorithm| key, which is required for each graph
  drawing context. 

  The key/value pair will be stored in |/tikz/graphs/graph drawing/@options/|.
\end{stylekey}

\begin{stylekey}{/tikz/graphs/graph drawing/register math key}
  Registering a new math key is like registering a new key, except
  that it's a parseable value. When a value is assigned to the key,
  pgf will parse the value. 

  Math keys can be used if a option holds a dimension value, like the
  |scale| option of \tikzname\. The value will be expanded and
  computed to the dimension |pt|. 

  A sample math key is introduced in the simpleexample algorithm
  (see \ref{section-library-graphdrawing-ownAlgorithm}) below.
\end{stylekey}

\begin{stylekey}{/tikz/graphs/graph drawing/register node key}
  A node key is not stored graph-wide; it is designated for a single
  node. The name/value pair is accessible from the node object in Lua;
  in \tikzname\ it will be stored in the key family |/tikz/graphs/graph drawing/@node@options/|.
\end{stylekey}

\begin{stylekey}{/tikz/graphs/graph drawing/register node math key}
  Like node key, but with parsing of it's value (see |register math key|).
\end{stylekey}

\subsection{Creating your own Algorithm}
\label{section-library-graphdrawing-ownAlgorithm}
There are two ways to make a user-definded algorithm
available to the graph drawing library.
You can create your own graph drawing algorithm by naming it like
|drawGraphAlgorithm_xyz| and placing it into the |pgf.graphdrawing|
Lua module, where |xyz| is the string which is supplied to the
\TeX\ interface.  This way the function is looked up before the
framework tries to load a file named
|pgflibrarygraphdrawing-algorithms-xyz.lua| anywhere in the accessible
path, which is the second way to define your algorithm.

You may load a file named according to the above-mentioned scheme that
contains an algorithm on your own using the |Interface:loadAlgorithm()|
function, which accepts the name string as single argument. This will
usually modify the module entry of the function name, so you have to
be aware of that behaviour if you rely on it to test whether an
algorithm was loaded (e.g. if you want to define a wrapper around the
loaded algorithm).

The algorithm will be called with the graph object as single argument
and should do its work by modifying this object. Any return
values are discarded.

For example, the following code fragment (taken and slightly altered
from the file |pgflibrarygraphdrawing-algorithms-simpleexample.lua|)
implements a rather simple algorithm, placing all nodes on a fixed-size
circle.  It is accessed with the name |simpleexample|, so both the
file- and function name agree on that.

\begin{codeexample}[code only]
pgf.module("pgf.graphdrawing")

--- A very, very simple node placing algorithm for demonstration purposes.
-- All nodes are positioned on a fixed-size circle.
function drawGraphAlgorithm_simpleexample(graph)
   local radius = 20
   local nodeCount = 0

   -- count nodes
   for _ in values(graph.nodes) do
      nodeCount = nodeCount + 1
   end

   local alpha = (2 * math.pi) / nodeCount
   local i = 0
   for node in values(graph.nodes) do
      -- the interesting part...
      node.pos.x = radius * math.cos(i * alpha)
      node.pos.y = radius * math.sin(i * alpha)
      i = i + 1
   end
end
\end{codeexample}

It is important not to use a |local| declaration before the function
header, because it wouldn't be available in the |pgf.graphdrawing|
module anymore.

The algorithm computes a circular layout like in the following.

\begin{codeexample}[]
\tikzpicture [graphs/.cd, graph drawing engine, algorithm=simpleexample]
  \graph { f -> c -> e -> a ->{b -> {c, d, f}, e -> b}};
\endtikzpicture
\end{codeexample}

The invocation above also shows how to use an algorithm which is not
registered as a \tikzname\ key.  In general, you will probably want to
register your algorithm with |\tikzgraphsset| to make your code more
succinct, but also to be able to change algorithm options by manipulating
\tikzname\ keys, which is not possible without registration.  

To do so, we have to modify the first line of the example algorithm.

\begin{codeexample}[code only]
   local radius = graph:getOption("radius") or 20
\end{codeexample}

Using the |getOption| method we obtain the value of the
\tikzname\ option or a |nil| value, therefore there has to be a
default value for any option or more elaborated error handling.  The 
following code block can be used to register this algorithm and
its single option.

\begin{codeexample}[code only]
\tikzgraphsset{
  simpleexample/.style={
    graph drawing engine,
    algorithm=simpleexample
  },
  graph drawing/register math key=radius
}
\end{codeexample}

\tikzgraphsset{
  simpleexample/.style={
    graph drawing engine,
    algorithm=simpleexample
  },
  graph drawing/register math key=radius
}

Eventually this fragment will have to be entered into the
|tikzlibrarygraphdrawing.code.tex| file if it is to be included in the
\pgfname\ source code.

Once registered, specifying the algorithm gets a bit easier. Note the
increased radius compared to the previous example.

\begin{codeexample}[]
\tikzpicture [graph drawing={simpleexample, radius = 30}]
  \graph { f -> c -> e -> a ->{b -> {c, d, f}, e -> b}};
\endtikzpicture
\end{codeexample}

\subsection{Module System}
The package defines its own Lua module system, which is characterised by a
more dynamic view on importing symbols.  Basically, each module has a
set of imported modules and the lookup for names first happens in the local
scope, then in the current module and subsequently in all imported
modules.  Since no name is statically imported, newly assigned
variables in other modules are still visible when those were
previously imported.

Modules are accessed with the |pgf.module| call, which enables the
module for the current context, i.e. the current file. If a module
does not exist, it will be created.  Importing modules is done via
|pgf.import|.  Both functions accept a string argument for the
module name.

Modules are named hierarchically and defined modules are exported into
each parent module.  If the module name contains no period, it is
exported into the global environment.  Nevertheless, importing is only
done on request; importing a module twice doesn't do anything.
It is recommended to dedicate a single module definition file
to create it and import other modules.  For example, the package
contains a single file containing only the following two lines for
creating the |pgf.graphdrawing| module in the first place.

\begin{codeexample}[code only]
pgf.module("pgf.graphdrawing")
pgf.import("pgf")
\end{codeexample}

Symbol lookup first happens in the local namespace, then in the
current module and subsequently in all imported modules and the global
namespace.  Assignment of new variables happens in the current module
(or for variables declared |local| in the local namespace).  If you
need to assign values to the global environment use the special table
|_G| as you'd normally do in Lua.

The |pgf| module is created during the definition of the module system
and mostly contains functions for loading and debugging.  Developers
probably shouldn't touch the |pgf| namespace and instead add new
functionality to modules below this level or in new top-level
modules.

\subsubsection{Module Examples}
Let's see what consequences this module system has in praxis.  The
following code fragment starts from a clean state after rendering it
with \LuaTeX\ and then enters the |pgf.graphdrawing| module,
overwriting the global |pgf| binding and then again reverting this
change.

\begin{codeexample}[code only]
  \input tikz

  \usetikzlibrary{graphdrawing}

  \directlua{
    pgf.graphdrawing.Sys:log("1: pgf is " .. tostring(pgf))
    pgf.graphdrawing.Sys:log("1: graphdrawing is " .. tostring(graphdrawing))
    
    pgf.module("pgf.graphdrawing")

    Sys:log("2: pgf is " .. tostring(pgf))
    Sys:log("2: graphdrawing is " .. tostring(graphdrawing))

    pgf = 1

    Sys:log("3: pgf is " .. tostring(pgf))
    Sys:log("3: graphdrawing is " .. tostring(graphdrawing))

    pgf = nil

    Sys:log("4: pgf is " .. tostring(pgf))

    pgf.graphdrawing = nil

    Sys:log("5: pgf is " .. tostring(pgf))

    _G.pgf = nil

    Sys:log("6: pgf is " .. tostring(pgf))
  }
\end{codeexample}

The result will be as follows:

\begin{codeexample}[code only]
1: pgf is <module 'pgf', table: 0x7979600>
1: graphdrawing is nil

2: pgf is <module 'pgf', table: 0x7979600>
2: graphdrawing is <module 'pgf.graphdrawing', table: 0x7973c60>

3: pgf is 1
3: graphdrawing is <module 'pgf.graphdrawing', table: 0x7973c60>

4: pgf is <module 'pgf', table: 0x7979600>
5: pgf is <module 'pgf', table: 0x7979600>
6: pgf is nil
\end{codeexample}

As you can see the |pgf| table is available in the global environment
and also after using the |pgf.graphdrawing| module, although we don't
refer to it with its full name.  Assigning a new value to |pgf|
doesn't overwrite the global object, but introduces a local binding
shadowing the global one. Assigning |nil| then removes the local
binding, therefore in the next line the global variable is available
again.

Note that in all but the first case the binding to |graphdrawing|
stays the same.  Also, using these assignments, you can't accidentally
remove your access to the |pgf| or any imported modules as the last
two assignments show (the |Sys:log| method still works).

\subsection{Lua Documentation}
This sections provides a full documentation of all relevant Lua classes
used.

Every class and function in the package (except for module handling in
|pgf|) is available in the |pgf.graphdrawing| module.

\label{section-library-graphdrawing-lua-documentation}
\subsubsection{Graph Representation}
\label{section-library-graphdrawing-lua-documentation-graphrep}
% This file has been generated from the lua sources using LuaDoc.
% To regenerate it call "make genluadoc" in
% doc/generic/pgf/version-for-luatex/en.

\begin{filedescription}{pgflibrarygraphdrawing-graph.lua}


\begin{luacommand}{{Graph:\textunderscore{}\textunderscore{}tostring}()}
Returns a string representation of this graph including all nodes and edges. 


Return value:
\begin{itemize} \item[] Graph as string.  \end{itemize}


\end{luacommand}\begin{luacommand}{{Graph:addEdge}(\meta{edge})}
Adds an edge to the graph. 

Parameters:
\begin{parameterdescription}
	\item[\meta{edge}] The edge to be added. 
\end{parameterdescription}



\end{luacommand}\begin{luacommand}{{Graph:addNode}(\meta{node})}
Adds a node to the graph. 

Parameters:
\begin{parameterdescription}
	\item[\meta{node}] The node to be added. 
\end{parameterdescription}



\end{luacommand}\begin{luacommand}{{Graph:copy}()}
Creates a shallow copy of a graph.  The nodes and edges of the original graph are not preserved in the copy. 


Return value:
\begin{itemize} \item[] A shallow copy of the graph.  \end{itemize}


\end{luacommand}\begin{luacommand}{{Graph:createEdge}(\meta{nodeA},\meta{nodeB},\meta{direction},\meta{edgenodes},\meta{options},\meta{tikzoptions})}
Creates and adds a new edge to the graph. 

Parameters:
\begin{parameterdescription}
	\item[\meta{nodeA}] The first node of the new edge.\item[\meta{nodeB}] The second node of the new edge.\item[\meta{direction}] The direction of the new edge. Possible values are |Edge.UNDIRECTED|, |Edge.LEFT|, |Edge.RIGHT|, |Edge.BOTH| and |Edge.NONE| (for invisible edges).\item[\meta{edgenodes}] A string of \tikzname\ edge nodes that needs to be passed back to the \TeX layer unmodified.\item[\meta{options}] The options of the new edge.\item[\meta{tikzoptions}] A table of \tikzname\ options to be used by graph drawing algorithms to treat the edge in special ways. 
\end{parameterdescription}


Return value:
\begin{itemize} \item[] The newly created edge.  \end{itemize}


\end{luacommand}\begin{luacommand}{{Graph:deleteEdge}(\meta{edge})}
Like removeEdge, but also removes the edge from its adjacent nodes. 

Parameters:
\begin{parameterdescription}
	\item[\meta{edge}] The edge to be deleted. 
\end{parameterdescription}


Return value:
\begin{itemize} \item[] The removed edge or |nil| if it was not found in the graph.  \end{itemize}


\end{luacommand}\begin{luacommand}{{Graph:deleteNode}(\meta{node})}
Like removeNode, but also deletes all adjacent edges of the removed node.  This function also removes the deleted adjacent edges from all neighbours of the removed node. 

Parameters:
\begin{parameterdescription}
	\item[\meta{node}] The node to be deleted together with its adjacent edges. 
\end{parameterdescription}


Return value:
\begin{itemize} \item[] The removed node or |nil| if the node was not found in the graph.  \end{itemize}


\end{luacommand}\begin{luacommand}{{Graph:findNode}(\meta{name})}
If possible, looks up the node with the given name in the graph. 

Parameters:
\begin{parameterdescription}
	\item[\meta{name}] Name of the node to look up. 
\end{parameterdescription}


Return value:
\begin{itemize} \item[] The node with the given name or |nil| if it was not found in the graph.  \end{itemize}


\end{luacommand}\begin{luacommand}{{Graph:findNodeIf}(\meta{test})}
Looks up the first node for which the function \meta{test} returns |true|. 

Parameters:
\begin{parameterdescription}
	\item[\meta{test}] A function that takes one parameter (a |Node|) and returns |true| or |false|. 
\end{parameterdescription}


Return value:
\begin{itemize} \item[] The first node for which \meta{test} returns |true|.  \end{itemize}


\end{luacommand}\begin{luacommand}{{Graph:getOption}(\meta{name})}
Returns the value of the graph option \meta{name}. 

Parameters:
\begin{parameterdescription}
	\item[\meta{name}] Name of the option. 
\end{parameterdescription}


Return value:
\begin{itemize} \item[] The value of the graph option \meta{name} or |nil|.  \end{itemize}


\end{luacommand}\begin{luacommand}{{Graph:mergeOptions}(\meta{options})}
Merges the given options into the options of the graph. 

Parameters:
\begin{parameterdescription}
	\item[\meta{options}] The options to be merged. 
\end{parameterdescription}



See also:
\begin{itemize}
	\item[] |mergeTable |
\end{itemize}

\end{luacommand}\begin{luacommand}{{Graph:new}(\meta{values})}
Creates a new graph. 

Parameters:
\begin{parameterdescription}
	\item[\meta{values}] Values to override default graph settings. The following parameters can be set:\par |nodes|: The nodes of the graph.\par |edges|: The edges of the graph.\par |pos|: Initial position of the graph.\par |options|: A table of node options passed over from \tikzname. 
\end{parameterdescription}


Return value:
\begin{itemize} \item[] A newly-allocated graph.  \end{itemize}


\end{luacommand}\begin{luacommand}{{Graph:removeEdge}(\meta{edge})}
If possible, removes an edge from the graph and returns it. 

Parameters:
\begin{parameterdescription}
	\item[\meta{edge}] The edge to be removed. 
\end{parameterdescription}


Return value:
\begin{itemize} \item[] The removed edge or |nil| if it was not found in the graph.  \end{itemize}


\end{luacommand}\begin{luacommand}{{Graph:removeNode}(\meta{node})}
If possible, removes a node from the graph and returns it. 

Parameters:
\begin{parameterdescription}
	\item[\meta{node}] The node to remove. 
\end{parameterdescription}


Return value:
\begin{itemize} \item[] The removed node or |nil| if it was not found in the graph.  \end{itemize}


\end{luacommand}\begin{luacommand}{{Graph:setOption}(\meta{name},\meta{value})}
Sets the graph option \meta{name} to \meta{value}. 

Parameters:
\begin{parameterdescription}
	\item[\meta{name}] Name of the option to be changed.\item[\meta{value}] New value for the graph option \meta{name}. 
\end{parameterdescription}



\end{luacommand}\begin{luacommand}{{Graph:subGraph}(\meta{root},\meta{graph},\meta{visited})}
Returns a subgraph.  The resulting subgraph begins at the node root, excludes all nodes and edges that are marked as visited. 

Parameters:
\begin{parameterdescription}
	\item[\meta{root}] Root node where the operation starts.\item[\meta{graph}] Result graph object or |nil| if the original graph should be used as the parent graph.\item[\meta{visited}] Set of already visited nodes/edges or |nil|. This set will be modified so make sure not to use a table that you want to remain untouched. 
\end{parameterdescription}



\end{luacommand}\begin{luacommand}{{Graph:subGraphParent}(\meta{root},\meta{parent},\meta{graph})}
Creates a new subgraph with \meta{parent} marked as visited.  This function can be useful if the graph is a tree structure (and \meta{parent} is the parent node of \meta{root}). 

Parameters:
\begin{parameterdescription}
	\item[\meta{root}] Root node where the operation starts.\item[\meta{parent}] Parent of the recursion step before.\item[\meta{graph}] Result graph object or |nil| if the original graph should be used as the parent graph. 
\end{parameterdescription}



See also:
\begin{itemize}
	\item[] |subGraph |
\end{itemize}

\end{luacommand}\begin{luacommand}{{Graph:walkAux}(\meta{root},\meta{visited},\meta{removeIndex})}
Auxiliary function to walk a graph. Does nothing if no nodes exist. 

Parameters:
\begin{parameterdescription}
	\item[\meta{root}] The first node to be visited.  If nil, chooses some node.\item[\meta{visited}] Set of already visited nodes and edges. |visited[v] == true| indicates that the node or edge |v| has already been visited.\item[\meta{removeIndex}] A numeric value or |nil| that defines the order in which nodes and edges are visited while traversing the graph. |nil| results in queue behavior, |1| in stack behavior. 
\end{parameterdescription}



See also:
\begin{itemize}
	\item[] |walkDepth|\item[] |walkBreadth |
\end{itemize}

\end{luacommand}\begin{luacommand}{{Graph:walkBreadth}(\meta{root},\meta{visited})}
Returns an iterator to walk the graph in a breadth-first traversal.  The iterator returns all edges and nodes one at a time. In case only the nodes are of interest, a filter function like |iter.filter| can be used to ignore edges. 

Parameters:
\begin{parameterdescription}
	\item[\meta{root}] The first node to be visited.  If nil, chooses some node.\item[\meta{visited}] Set of already visited nodes and edges. |visited[v] == true| indicates that the node or edge |v| has already been visited. 
\end{parameterdescription}



See also:
\begin{itemize}
	\item[] |iter.filter |
\end{itemize}

\end{luacommand}\begin{luacommand}{{Graph:walkDepth}(\meta{root},\meta{visited})}
Returns an iterator to walk the graph in a depth-first traversal.  The iterator returns all edges and nodes one at a time. In case only the nodes are of interest, a filter function like |iter.filter| can be used to ignore edges. 

Parameters:
\begin{parameterdescription}
	\item[\meta{root}] The first node to be visited.  If nil, chooses some node.\item[\meta{visited}] Set of already visited nodes and edges. |visited[v] == true| indicates that the node or edge |v| has already been visited. 
\end{parameterdescription}



See also:
\begin{itemize}
	\item[] |iter.filter |
\end{itemize}

\end{luacommand}
\end{filedescription}
% This file has been generated from the lua sources using LuaDoc.
% To regenerate it call "make genluadoc" in
% doc/generic/pgf/version-for-luatex/en.

\begin{filedescription}{pgflibrarygraphdrawing-node.lua}


\begin{luacommand}{{Node:\textunderscore{}\textunderscore{}eq}(\meta{object},\meta{other})}
Compares two nodes by their name. 

Parameters:
\begin{parameterdescription}
	\item[\meta{other}] Another node to compare with. 
\end{parameterdescription}


Return value:
\begin{itemize} \item[] |true| if both nodes have the same name. |false| otherwise.  \end{itemize}


\end{luacommand}\begin{luacommand}{{Node:\textunderscore{}\textunderscore{}tostring}()}
Returns a formated string representation of the node. 


Return value:
\begin{itemize} \item[] String represenation of the node.  \end{itemize}


\end{luacommand}\begin{luacommand}{{Node:addEdge}(\meta{edge})}
Adds new edge to the node. 

Parameters:
\begin{parameterdescription}
	\item[\meta{edge}] The edge to be added. 
\end{parameterdescription}



\end{luacommand}\begin{luacommand}{{Node:copy}()}
Creates a shallow copy of the node.  Most notably, the edges adjacent are not preserved in the copy. 


Return value:
\begin{itemize} \item[] Copy of the node.  \end{itemize}


\end{luacommand}\begin{luacommand}{{Node:getDegree}()}
Counts the adjacent edges of the node. 


Return value:
\begin{itemize} \item[] The number of adjacent edges of the node.  \end{itemize}


\end{luacommand}\begin{luacommand}{{Node:getEdges}()}
Returns all edges of the node.  Instead of calling |node:getEdges()| the edges can alternatively be accessed directly with |node.edges|. 


Return value:
\begin{itemize} \item[] All edges of the node.  \end{itemize}


\end{luacommand}\begin{luacommand}{{Node:getInDegree}(\meta{ignorereversed})}
Returns the number of incoming edges of the node. 

Parameters:
\begin{parameterdescription}
	\item[\meta{ignorereversed}] Optional parameter to consider reversed edges not reversed for this method call. Defaults to |false|. 
\end{parameterdescription}


Return value:
\begin{itemize} \item[] The number of incoming edges of the node.  \end{itemize}


See also:
\begin{itemize}
	\item[] |Node:getIncomingEdges(reversed) |
\end{itemize}

\end{luacommand}\begin{luacommand}{{Node:getIncomingEdges}(\meta{ignorereversed})}
Returns the incoming edges of the node. Undefined result for hyperedges. 

Parameters:
\begin{parameterdescription}
	\item[\meta{ignorereversed}] Optional parameter to consider reversed edges not reversed for this method call. Defaults to |false|. 
\end{parameterdescription}


Return value:
\begin{itemize} \item[] Incoming edges of the node. This includes undirected edges and directed edges pointing to the node.  \end{itemize}


\end{luacommand}\begin{luacommand}{{Node:getOption}(\meta{name})}
Returns the value of the node option \meta{name}. 

Parameters:
\begin{parameterdescription}
	\item[\meta{name}] Name of the node option. 
\end{parameterdescription}


Return value:
\begin{itemize} \item[] The value of the node option \meta{name} or |nil|.  \end{itemize}


\end{luacommand}\begin{luacommand}{{Node:getOutDegree}(\meta{ignorereversed})}
Returns the number of edges starting at the node. 

Parameters:
\begin{parameterdescription}
	\item[\meta{ignorereversed}] Optional parameter to consider reversed edges not reversed for this method call. Defaults to |false|. 
\end{parameterdescription}


Return value:
\begin{itemize} \item[] The number of outgoing edges of the node.  \end{itemize}


See also:
\begin{itemize}
	\item[] |Node:getOutgoingEdges() |
\end{itemize}

\end{luacommand}\begin{luacommand}{{Node:getOutgoingEdges}(\meta{ignorereversed})}
Returns the outgoing edges of the node. Undefined result for hyperedges. 

Parameters:
\begin{parameterdescription}
	\item[\meta{ignorereversed}] Optional parameter to consider reversed edges not reversed for this method call. Defaults to |false|. 
\end{parameterdescription}


Return value:
\begin{itemize} \item[] Outgoing edges of the node. This includes undirected edges and directed edges leaving the node.  \end{itemize}


\end{luacommand}\begin{luacommand}{{Node:getTexHeight}()}
Computes the heigth of the node. 


Return value:
\begin{itemize} \item[] Height of the node.  \end{itemize}


\end{luacommand}\begin{luacommand}{{Node:getTexWidth}()}
Computes the width of the node. 


Return value:
\begin{itemize} \item[] Width of the node.  \end{itemize}


\end{luacommand}\begin{luacommand}{{Node:new}(\meta{values})}
Creates a new node. 

Parameters:
\begin{parameterdescription}
	\item[\meta{values}] Values to override default node settings. The following parameters can be set:\par |name|: The name of the node. It is obligatory to define this.\par |tex|:  Information about the corresponding \TeX\ node.\par |edges|: Edges adjacent to the node.\par |pos|: Initial position of the node.\par |options|: A table of node options passed over from \tikzname. 
\end{parameterdescription}


Return value:
\begin{itemize} \item[] A newly allocated node.  \end{itemize}


\end{luacommand}\begin{luacommand}{{Node:removeEdge}(\meta{edge})}
Removes an edge from the node. 

Parameters:
\begin{parameterdescription}
	\item[\meta{edge}] The edge to remove. 
\end{parameterdescription}



\end{luacommand}\begin{luacommand}{{Node:setOption}(\meta{name},\meta{value})}
Sets the node option \meta{name} to \meta{value}. 

Parameters:
\begin{parameterdescription}
	\item[\meta{name}] Name of the node option to be changed.\item[\meta{value}] New value for the node option \meta{name}. 
\end{parameterdescription}



\end{luacommand}
\end{filedescription}
% This file has been generated from the lua sources using LuaDoc.
% To regenerate it call "make genluadoc" in
% doc/generic/pgf/version-for-luatex/en.

\begin{filedescription}{pgflibrarygraphdrawing-edge.lua}


\begin{luacommand}{{Edge:\textunderscore{}\textunderscore{}eq}(\meta{other})}
Returns whether or not the two edges have the same adjacent nodes. 

Parameters:
\begin{parameterdescription}
	\item[\meta{other}] Another edge to compare with. 
\end{parameterdescription}


Return value:
\begin{parameterdescription} 
  \item[] |true| if the two edges have exactly the same adjacent nodes. 
\end{parameterdescription}


\end{luacommand}
\begin{luacommand}{{Edge:\textunderscore{}\textunderscore{}tostring}()}
Returns a readable string representation of the edge. 


Return value:
\begin{parameterdescription} 
  \item[] String representation of the edge. 
\end{parameterdescription}


\end{luacommand}
\begin{luacommand}{{Edge:addNode}(\meta{node})}
If possible, adds a node to the edge. 

Parameters:
\begin{parameterdescription}
	\item[\meta{node}] The node to be added to the edge. 
\end{parameterdescription}



\end{luacommand}
\begin{luacommand}{{Edge:containsNode}(\meta{node})}
Returns whether or not a node is adjacent to the edge. 

Parameters:
\begin{parameterdescription}
	\item[\meta{node}] The node to check. 
\end{parameterdescription}


Return value:
\begin{parameterdescription} 
  \item[] |true| if the node is adjacent to the edge. |false| otherwise. 
\end{parameterdescription}


\end{luacommand}
\begin{luacommand}{{Edge:copy}()}
Copies an edge (preventing accidental use).  The nodes of the edge are not preserved and have to be added to the copy manually if necessary. 


Return value:
\begin{parameterdescription} 
  \item[] Shallow copy of the edge. 
\end{parameterdescription}


\end{luacommand}
\begin{luacommand}{{Edge:getDegree}()}
Counts the nodes on this edge. 


Return value:
\begin{parameterdescription} 
  \item[] The number of nodes on the edge. 
\end{parameterdescription}


\end{luacommand}
\begin{luacommand}{{Edge:getNeighbour}(\meta{node})}
Gets first neighbour of the node (disregarding hyperedges). 

Parameters:
\begin{parameterdescription}
	\item[\meta{node}] The node which first neighbour should be returned. 
\end{parameterdescription}


Return value:
\begin{parameterdescription} 
  \item[] The first neighbour of the node. 
\end{parameterdescription}


\end{luacommand}
\begin{luacommand}{{Edge:getNeighbours}(\meta{node})}
Returns all neighbours of a node adjacent to the edge.  The edge direction is not taken into account, so this method always returns all neighbours even if called on a directed edge. 

Parameters:
\begin{parameterdescription}
	\item[\meta{node}] A node. Typically but not necessarily adjacent to the edge. If the node is not an intermediate or end point of the edge, an empty array is returned. 
\end{parameterdescription}


Return value:
\begin{parameterdescription} 
  \item[] An array of nodes that are adjacent to the input node via the edge the method is called on. 
\end{parameterdescription}


\end{luacommand}
\begin{luacommand}{{Edge:getNodes}()}
Returns all nodes of the edge.  Instead of calling |edge:getNodes()| the nodes can alternatively be accessed directly with |edge.nodes|. 


Return value:
\begin{parameterdescription} 
  \item[] All edges of the node. 
\end{parameterdescription}


\end{luacommand}
\begin{luacommand}{{Edge:getOption}(\meta{name})}
Returns the value of the edge option \meta{name}. 

Parameters:
\begin{parameterdescription}
	\item[\meta{name}] Name of the option. 
\end{parameterdescription}


Return value:
\begin{parameterdescription} 
  \item[] The value of the edge option \meta{name} or |nil|. 
\end{parameterdescription}


\end{luacommand}
\begin{luacommand}{{Edge:isHead}(\meta{node},\meta{ignore\_reversed})}
Checks whether a node is the head of the edge. Does not work for hyperedges.  This method only works for edges with two adjacent nodes.  For undirected edges or edges that point into both directions, the result will always be true. Directed edges may be reversed internally, so their head and tail might be switched. Whether or not this internal reversal is handled by this method can be specified with the optional second \meta{ignore\_reversed} parameter which is |false| by default. 

Parameters:
\begin{parameterdescription}
	\item[\meta{node}] The node to check.\item[\meta{ignore\_reversed}] Optional parameter. Set this to true if reversed edges should not be considered reversed for this method call. 
\end{parameterdescription}


Return value:
\begin{parameterdescription} 
  \item[] True if the node is the head of the edge. 
\end{parameterdescription}


\end{luacommand}
\begin{luacommand}{{Edge:isHyperedge}()}
Returns whether or not the edge is a hyperedge.  A hyperedge is an edge with more than two adjacent nodes. 


Return value:
\begin{parameterdescription} 
  \item[] |true| if the edge is a hyperedge. |false| otherwise. 
\end{parameterdescription}


\end{luacommand}
\begin{luacommand}{{Edge:isTail}(\meta{node},\meta{ignore\_reversed})}
Checks whether a node is the tail of the edge. Does not work for hyperedges.  This method only works for edges with two adjacent nodes.  For undirected edges or edges that point into both directions, the result will always be true.  Directed edges may be reversed internally, so their head and tail might be switched. Whether or not this internal reversal is handled by this method can be specified with the optional second \meta{ignore\_reversed} parameter which is |false| by default. 

Parameters:
\begin{parameterdescription}
	\item[\meta{node}] The node to check.\item[\meta{ignore\_reversed}] Optional parameter. Set this to true if reversed edges should not be considered reversed for this method call. 
\end{parameterdescription}


Return value:
\begin{parameterdescription} 
  \item[] True if the node is the tail of the edge. 
\end{parameterdescription}


\end{luacommand}
\begin{luacommand}{{Edge:new}(\meta{values})}
Creates an edge between nodes of a graph. 

Parameters:
\begin{parameterdescription}
	\item[\meta{values}] Values to override default edge settings. The following parameters can be set:\par |nodes|: TODO \par |edge_nodes|: TODO \par |options|: TODO \par |tikz_options|: TODO \par |direction|: TODO \par |bend_points|: TODO \par |bend_nodes|: TODO \par |reversed|: TODO \par 
\end{parameterdescription}


Return value:
\begin{parameterdescription} 
  \item[] A newly-allocated edge. 
\end{parameterdescription}


\end{luacommand}
\begin{luacommand}{{Edge:setOption}(\meta{name},\meta{value})}
Sets the edge option \meta{name} to \meta{value}. 

Parameters:
\begin{parameterdescription}
	\item[\meta{name}] Name of the option to be changed.\item[\meta{value}] New value for the edge option \meta{name}. 
\end{parameterdescription}



\end{luacommand}

\end{filedescription}
% This file has been generated from the lua sources using LuaDoc.
% To regenerate it call "make genluadoc" in
% doc/generic/pgf/version-for-luatex/en.

\begin{filedescription}{pgflibrarygraphdrawing-position.lua}


\begin{luacommand}{{Position.calcCoordsTo}(\meta{posFrom},\meta{posTo})}
Returns a vector between two positions.

Parameters:
\begin{parameterdescription}
	\item[\meta{posFrom}] Position A.\item[\meta{posTo}] Position B.
\end{parameterdescription}


Return value:
\begin{parameterdescription} 
  \item[] x- and y-coordinates of the vector between posFrom and posTo.
\end{parameterdescription}


\end{luacommand}
\begin{luacommand}{{Position:\textunderscore{}\textunderscore{}tostring}()}
Returns a readable string representation of the position.


Return value:
\begin{parameterdescription} 
  \item[] string representation of the position.
\end{parameterdescription}


\end{luacommand}
\begin{luacommand}{{Position:copy}()}
Creates a copy of this position object.


Return value:
\begin{parameterdescription} 
  \item[] Copy of the position.
\end{parameterdescription}


\end{luacommand}
\begin{luacommand}{{Position:equals}(\meta{pos})}
Returns a boolean value whether the object is equal to the given position.


Return value:
\begin{parameterdescription} 
  \item[] true if the position is equal to the given position pos.
\end{parameterdescription}


\end{luacommand}
\begin{luacommand}{{Position:getAbsCoordinates}(\meta{x},\meta{y})}
Computes absolute coordinates of a position.

Parameters:
\begin{parameterdescription}
	\item[\meta{x}] Just used internally for recrusion.\item[\meta{y}] Just used internally for recrusion.
\end{parameterdescription}


Return value:
\begin{parameterdescription} 
  \item[] Absolute position.
\end{parameterdescription}


\end{luacommand}
\begin{luacommand}{{Position:isAbsPosition}()}
Determines if the position is absolute.


Return value:
\begin{parameterdescription} 
  \item[] True if the position is absolute, else false.
\end{parameterdescription}


\end{luacommand}
\begin{luacommand}{{Position:new}(\meta{values})}
Represents a relative postion.

Parameters:
\begin{parameterdescription}
	\item[\meta{values}] Values (e.g. x- and y-coordinate) to be merged with the default-metatable of a position.
\end{parameterdescription}


Return value:
\begin{parameterdescription} 
  \item[] A new position object.
\end{parameterdescription}


\end{luacommand}
\begin{luacommand}{{Position:relateTo}(\meta{pos},\meta{keepAbsPosition})}
Relates a position to the given position.

Parameters:
\begin{parameterdescription}
	\item[\meta{pos}] The relative position.\item[\meta{keepAbsPosition}] If true, the coordinates of the position are computed in the relation to the given position pos.
\end{parameterdescription}



\end{luacommand}

\end{filedescription}
% This file has been generated from the lua sources using LuaDoc.
% To regenerate it call "make genluadoc" in
% doc/generic/pgf/version-for-luatex/en.

\begin{filedescription}{pgflibrarygraphdrawing-vector.lua}


\begin{luacommand}{{Vector:copy}()}
Creates a copy of the vector that holds the same elements as the original. 


Return value:
\begin{parameterdescription} 
  \item[] A newly-allocated copy of the vector holding exactly the same elements. 
\end{parameterdescription}


\end{luacommand}
\begin{luacommand}{{Vector:dividedBy}(\meta{other})}
Performs a vector division and returns the result in a new vector. 

Parameters:
\begin{parameterdescription}
	\item[\meta{other}] Vector to divide by. 
\end{parameterdescription}


Return value:
\begin{parameterdescription} 
  \item[] A new vector with the result of the division. 
\end{parameterdescription}


\end{luacommand}
\begin{luacommand}{{Vector:dividedByScalar}(\meta{scalar})}
Divides a vector by a scalar value and returns the result in a new vector. 

Parameters:
\begin{parameterdescription}
	\item[\meta{scalar}] Scalar value to divide the vector by. 
\end{parameterdescription}


Return value:
\begin{parameterdescription} 
  \item[] A new vector with the result of the division. 
\end{parameterdescription}


\end{luacommand}
\begin{luacommand}{{Vector:dotProduct}(\meta{other})}
Performs the dot product of two vectors and returns the result in a new vector. 

Parameters:
\begin{parameterdescription}
	\item[\meta{other}] Vector to perform the dot product with. 
\end{parameterdescription}


Return value:
\begin{parameterdescription} 
  \item[] A new vector with the result of the dot product. 
\end{parameterdescription}


\end{luacommand}
\begin{luacommand}{{Vector:get}(\meta{index})}
Returns the element at the given \meta{index}. 


Return value:
\begin{parameterdescription} 
  \item[] The element at the given \meta{index}. 
\end{parameterdescription}


\end{luacommand}
\begin{luacommand}{{Vector:limit}(\meta{limit\_function})}
Limits all elements of the vector in-place. 

Parameters:
\begin{parameterdescription}
	\item[\meta{limit\_function}] A function that is called for each index/element pair. It is supposed to return minimum and maximum values for the element. The element is then clamped to these values. 
\end{parameterdescription}



\end{luacommand}
\begin{luacommand}{{Vector:minus}(\meta{other})}
Subtracts two vectors and returns the result in a new vector. 

Parameters:
\begin{parameterdescription}
	\item[\meta{other}] Vector to subtract. 
\end{parameterdescription}


Return value:
\begin{parameterdescription} 
  \item[] A new vector with the result of the subtraction. 
\end{parameterdescription}


\end{luacommand}
\begin{luacommand}{{Vector:minusScalar}(\meta{scalar})}
Subtracts a scalar value from a vector and returns the result in a new vector. 

Parameters:
\begin{parameterdescription}
	\item[\meta{scalar}] Scalar value to subtract from all elements. 
\end{parameterdescription}


Return value:
\begin{parameterdescription} 
  \item[] A new vector with the result of the subtraction. 
\end{parameterdescription}


\end{luacommand}
\begin{luacommand}{{Vector:new}(\meta{n},\meta{fill\_function})}
Creates a new vector with \meta{n} values using an optional \meta{fill\_function}. 

Parameters:
\begin{parameterdescription}
	\item[\meta{n}] The number of elements of the vector.\item[\meta{fill\_function}] Optional function that takes a number between 1 and \meta{n} and is expected to return a value for the corresponding element of the vector. If omitted, all elements of the vector will be initialized with 0. 
\end{parameterdescription}


Return value:
\begin{parameterdescription} 
  \item[] A newly-allocated vector with \meta{n} elements. 
\end{parameterdescription}


\end{luacommand}
\begin{luacommand}{{Vector:norm}()}
Computes the Euclidean norm of the vector. 


Return value:
\begin{parameterdescription} 
  \item[] The Euclidean norm of the vector. 
\end{parameterdescription}


\end{luacommand}
\begin{luacommand}{{Vector:normalized}()}
Normalizes the vector and returns the result in a new vector. 


Return value:
\begin{parameterdescription} 
  \item[] Normalized version of the original vector. 
\end{parameterdescription}


\end{luacommand}
\begin{luacommand}{{Vector:plus}(\meta{other})}
Performs a vector addition and returns the result in a new vector. 

Parameters:
\begin{parameterdescription}
	\item[\meta{other}] The vector to add. 
\end{parameterdescription}


Return value:
\begin{parameterdescription} 
  \item[] A new vector with the result of the addition. 
\end{parameterdescription}


\end{luacommand}
\begin{luacommand}{{Vector:plusScalar}(\meta{scalar})}
Performs an addition with a scalar value and returns the result in a new vector.  The scalar value is added to all elements of the vector. 

Parameters:
\begin{parameterdescription}
	\item[\meta{scalar}] Scalar value to add to all elements. 
\end{parameterdescription}


Return value:
\begin{parameterdescription} 
  \item[] A new vector with the result of the addition. 
\end{parameterdescription}


\end{luacommand}
\begin{luacommand}{{Vector:reset}()}
Resets all vector elements to 0 in-place. 



\end{luacommand}
\begin{luacommand}{{Vector:set}(\meta{index},\meta{value})}
Changes the element at the given \meta{index}. 

Parameters:
\begin{parameterdescription}
	\item[\meta{index}] The index of the element to change.\item[\meta{value}] New value of the element. 
\end{parameterdescription}



\end{luacommand}
\begin{luacommand}{{Vector:timesScalar}(\meta{scalar})}
Multiplies a vector by a scalar value and returns the result in a new vector. 

Parameters:
\begin{parameterdescription}
	\item[\meta{scalar}] Scalar value to multiply the vector with. 
\end{parameterdescription}


Return value:
\begin{parameterdescription} 
  \item[] A new vector with the result of the multiplication. 
\end{parameterdescription}


\end{luacommand}
\begin{luacommand}{{Vector:update}(\meta{update\_function})}
Updates the values of the vector in-place. 

Parameters:
\begin{parameterdescription}
	\item[\meta{update\_function}] A function that is called for each element of the vector. The elements are replaced by the values returned from this function. 
\end{parameterdescription}



\end{luacommand}
\begin{luacommand}{{Vector:x}()}
Convenience method that returns the first element of the vector. 


Return value:
\begin{parameterdescription} 
  \item[] The first element of the vector. 
\end{parameterdescription}


\end{luacommand}
\begin{luacommand}{{Vector:y}()}
Convenience method that returns the second element of the vector. 


Return value:
\begin{parameterdescription} 
  \item[] The second element of the vector. 
\end{parameterdescription}


\end{luacommand}

\end{filedescription}
% This file has been generated from the lua sources using LuaDoc.
% To regenerate it call "make genluadoc" in
% doc/generic/pgf/version-for-luatex/en.

\begin{filedescription}{pgflibrarygraphdrawing-box.lua}


\begin{luacommand}{{Box:addBox}(\meta{box})}
Adds new internal Box.

Parameters:
\begin{parameterdescription}
	\item[\meta{box}] The box to be added.
\end{parameterdescription}



\end{luacommand}\begin{luacommand}{{Box:getPaths}()}
Provides all Paths this box contains.


Return value:
\begin{itemize} \item[] Recursive iteration over all paths. \end{itemize}


\end{luacommand}\begin{luacommand}{{Box:getPosAt}(\meta{place},\meta{absolute})}
Calculates the coordinates of the box according to the place parameter.

Parameters:
\begin{parameterdescription}
	\item[\meta{place}] Determines of which position of the box the coordinates should be returned (e.g. the center of the box requires the param Box.CENTER).  Possible values are: \begin{itemize} \item Box.UPPERLEFT \item Box.UPPERRIGHT \item Box.CENTER \item Box.LOWERRIGHT \item Box.LOWERLEFT \end{itemize}\item[\meta{absolute}] If true the absolute coordinates of the box will be returned, otherwise its relative coordinates.
\end{parameterdescription}


Return value:
\begin{itemize} \item[] X- and y-coordinates of the box. \end{itemize}


\end{luacommand}\begin{luacommand}{{Box:new}(\meta{values})}
Creates a new box.

Parameters:
\begin{parameterdescription}
	\item[\meta{values}] Values (e.g. height) to be merged with the default-metatable of a box.
\end{parameterdescription}


Return value:
\begin{itemize} \item[] The new box. \end{itemize}


\end{luacommand}\begin{luacommand}{{Box:recalculateSize}()}
Checks internal Boxes and resets width and height.



\end{luacommand}\begin{luacommand}{{Box:removeBox}(\meta{box})}
Removes internal Box.

Parameters:
\begin{parameterdescription}
	\item[\meta{box}] The box to remove.
\end{parameterdescription}



\end{luacommand}
\end{filedescription}
%% This file has been generated from the lua sources using LuaDoc.
% To regenerate it call "make genluadoc" in
% doc/generic/pgf/version-for-luatex/en.

\begin{filedescription}{pgflibrarygraphdrawing-path.lua}


\begin{luacommand}{{Path:\textunderscore{}\textunderscore{}tostring}()}
Returns a readable string representation of the path.


Return value:
\begin{parameterdescription} 
  \item[] String representation of the path
\end{parameterdescription}


\end{luacommand}
\begin{luacommand}{{Path:\textunderscore{}intersects}(\meta{a1},\meta{a2},\meta{b1},\meta{b2},\meta{allowedIntersections})}
Checks if the lines a1a2 and b1b2 intersect.

Parameters:
\begin{parameterdescription}
	\item[\meta{a1}] Start of the first line.\item[\meta{a2}] End of the first line.\item[\meta{b1}] Start of the second line.\item[\meta{b2}] End of the second line.\item[\meta{allowedIntersections}] A boolean table with the keys a1, a2, b1 and b2. If two or three of those values are true, the corresponding start and/or end points are allowed to match without being seen as intersection. If all four keys are true any matching of start and end points is allowed as long as the two lines are not coincedent. If three of the keys are true or start and end of a line are allowed to match, nill will be returned. If this optional parameter is not given, any matching points will be seen as intersections.
\end{parameterdescription}


Return value:
\begin{parameterdescription} 
  \item[] true, if lines intersect, false otherwise. If allowedIntersections contained an invalid value, nil will be returned.
\end{parameterdescription}


\end{luacommand}
\begin{luacommand}{{Path:addPoint}(\meta{point},\meta{keepAbsPosition})}
Appends new point at the end of path.

Parameters:
\begin{parameterdescription}
	\item[\meta{point}] Point to be added to the path\item[\meta{keepAbsPosition}] true if the coordinates of the point are absolute
\end{parameterdescription}



\end{luacommand}
\begin{luacommand}{{Path:createPath}(\meta{posStart},\meta{posEnd},\meta{keepAbsPosition})}
Adds a new segment to the path.

Parameters:
\begin{parameterdescription}
	\item[\meta{posStart}] Startposition of the new segment\item[\meta{posEnd}] Endposition of the new segment
\end{parameterdescription}



\end{luacommand}
\begin{luacommand}{{Path:getLastPoint}()}
Returns last point in path.


Return value:
\begin{parameterdescription} 
  \item[] last point
\end{parameterdescription}


\end{luacommand}
\begin{luacommand}{{Path:getLength}()}
Returns the length of the whole path.


Return value:
\begin{parameterdescription} 
  \item[] Length of the whole path.
\end{parameterdescription}


\end{luacommand}
\begin{luacommand}{{Path:getPoints}()}
Copies the internal points of a path.


Return value:
\begin{parameterdescription} 
  \item[] array of points
\end{parameterdescription}


\end{luacommand}
\begin{luacommand}{{Path:intersects}(\meta{path})}
Tests if the path is intersected by path.

Parameters:
\begin{parameterdescription}
	\item[\meta{path}] other path
\end{parameterdescription}



\end{luacommand}
\begin{luacommand}{{Path:move}(\meta{x},\meta{y})}
Adds new point with x,y relative to last point.

Parameters:
\begin{parameterdescription}
	\item[\meta{x}] x-coordinate of the new point\item[\meta{y}] y-coordinate of the new point
\end{parameterdescription}



\end{luacommand}
\begin{luacommand}{{Path:new}(\meta{values})}
Creates a new path.

Parameters:
\begin{parameterdescription}
	\item[\meta{values}] Values to be merged with the default-metatable of a path
\end{parameterdescription}


Return value:
\begin{parameterdescription} 
  \item[] A new path.
\end{parameterdescription}


\end{luacommand}

\end{filedescription}

\subsubsection{Base Layer}
% This file has been generated from the lua sources using LuaDoc.
% To regenerate it call "make genluadoc" in
% doc/generic/pgf/version-for-luatex/en.

\begin{filedescription}{pgflibrarygraphdrawing-interface.lua}


\begin{luacommand}{{Interface:addEdge}(\meta{from},\meta{to},\meta{direction},\meta{edge\_nodes},\meta{options},\meta{tikz\_options})}
Adds an edge from one node to another by name.  Both parameters are node names and have to exist before an edge can be created between them. 

Parameters:
\begin{parameterdescription}
	\item[\meta{from}] Name of the node the edge begins at.\item[\meta{to}] Name of the node the edge ends at.\item[\meta{direction}] Direction of the edge (e.g. |--| for an undirected edge or |->| for a directed edge from the first to the second node).\item[\meta{edge\_nodes}] A string for \tikzname\ to generate the edge label nodes later. Needs to be passed back to TikZ unmodified.\item[\meta{options}] A string of |{key}{value}| pairs of edge options that are relevant to graph drawing algorithms.\item[\meta{tikz\_options}] A string of |{key}{value}| pairs that need to be passed back to \tikzname\ unmodified. 
\end{parameterdescription}



See also:
\begin{itemize}
	\item[] |addNode |
\end{itemize}

\end{luacommand}
\begin{luacommand}{{Interface:addNode}(\meta{name},\meta{xMin},\meta{yMin},\meta{xMax},\meta{yMax},\meta{options})}
Adds a new node to the graph.  The options string of |{key}{value}| pairs is parsed and assigned to the node. Graph drawing algorithms may use these options to treat the node in special ways. 

Parameters:
\begin{parameterdescription}
	\item[\meta{name}] Name of the node.\item[\meta{xMin}] Minimum x point of the bouding box.\item[\meta{yMin}] Minimum y point of the bouding box.\item[\meta{xMax}] Maximum x point of the bouding box.\item[\meta{yMax}] Maximum y point of the bouding box.\item[\meta{options}] Options for the node. 
\end{parameterdescription}



\end{luacommand}
\begin{luacommand}{{Interface:drawEdge}(\meta{edge})}
Passes an edge back to the \TeX\ layer.  Edges with a direction of |Edge.NONE| are skipped and not passed back to \TeX. 

Parameters:
\begin{parameterdescription}
	\item[\meta{edge}] The edge to pass back to the \TeX\ layer. 
\end{parameterdescription}



\end{luacommand}
\begin{luacommand}{{Interface:drawGraph}()}
Arranges the current graph using the specified algorithm.  The algorithm is derived from the graph options and is loaded on demand from the corresponding algorithm file. For a fictitious algorithm |simple| this file is per convention called |pgflibrarygraphdrawing-algorithms-simple.lua|. It is required to define at least one function as an entry point to the algorithm. The name of the function is again predetermined as |graph_drawing_algorithm_simple|. When a graph is to be layed out, this function is called with the graph as its only parameter. 



\end{luacommand}
\begin{luacommand}{{Interface:drawNode}(\meta{node})}
Passes a node back to the \TeX\ layer. 

Parameters:
\begin{parameterdescription}
	\item[\meta{node}] The node to pass back to the \TeX\ layer. 
\end{parameterdescription}



\end{luacommand}
\begin{luacommand}{{Interface:finishGraph}()}
Passes the current graph back to the \TeX\ layer and removes it from the stack. 



\end{luacommand}
\begin{luacommand}{{Interface:getOption}(\meta{name})}
Returns the value of the graph option \meta{name}. 

Parameters:
\begin{parameterdescription}
	\item[\meta{name}] Name of the option. 
\end{parameterdescription}


Return value:
\begin{parameterdescription} 
  \item[] The value of the \meta{name} option or |nil|. 
\end{parameterdescription}


\end{luacommand}
\begin{luacommand}{{Interface:loadAlgorithm}(\meta{name})}
Attempts to load the algorithm with the given \meta{name}.  This function tries to look up the corresponding algorithm file |pgflibrarygraphdrawing-algorithms-name.lua| and attempts to look up the main function for calling the algorithm. 

Parameters:
\begin{parameterdescription}
	\item[\meta{name}] Name of the algorithm. 
\end{parameterdescription}


Return value:
\begin{parameterdescription} 
  \item[] The algorithm function or nil. 
\end{parameterdescription}


\end{luacommand}
\begin{luacommand}{{Interface:newGraph}(\meta{options})}
Creates a new graph and adds it to the graph stack.  The options string consisting of |{key}{value}| pairs is parsed and assigned to the graph. These options are used to configure the different graph drawing algorithms shipped with \tikzname. 

Parameters:
\begin{parameterdescription}
	\item[\meta{options}] A string containing |{key}{value}| pairs of \tikzname\ options. 
\end{parameterdescription}



See also:
\begin{itemize}
	\item[] |finishGraph |
\end{itemize}

\end{luacommand}
\begin{luacommand}{{Interface:setOption}(\meta{name},\meta{value})}
Sets the graph option \meta{name} to \meta{value}. Only affects the current graph. 

Parameters:
\begin{parameterdescription}
	\item[\meta{name}] The name of the option to set.\item[\meta{value}] New value for the option. 
\end{parameterdescription}



\end{luacommand}

\end{filedescription}
\label{section-library-graphdrawing-lua-documentation-interface}
% This file has been generated from the lua sources using LuaDoc.
% To regenerate it call "make genluadoc" in
% doc/generic/pgf/version-for-luatex/en.

\begin{filedescription}{pgflibrarygraphdrawing-sys.lua}


\begin{luacommand}{{Sys:beginShipout}()}
Begins the shipout of nodes by opening a scope in pgf.



\end{luacommand}\begin{luacommand}{{Sys:endShipout}()}
Ends the shipout by closing the opened scope.



See also:
\begin{itemize}
	\item[] |Sys:beginShipout()|
\end{itemize}

\end{luacommand}\begin{luacommand}{{Sys:escapeTeXNodeName}(\meta{nodename})}
Adds a ``not yet positionedPGFGDINTERNAL'' prefix to a node name. The prefix is required by pgf to place the node. Actually, when deferring the node placement, the prefix is added to avoid references to the node.

Parameters:
\begin{parameterdescription}
	\item[\meta{nodename}] Name of the node to prefix.
\end{parameterdescription}


Return value:
\begin{itemize} \item[] A newly composed string. \end{itemize}


\end{luacommand}\begin{luacommand}{{Sys:getTeXBox}()}
Retrieves a box from the transfer box register.



See also:
\begin{itemize}
	\item[] |putTeXBox|
\end{itemize}

\end{luacommand}\begin{luacommand}{{Sys:getVerboseMode}()}
Checks the verbosity of the subsystems output.


Return value:
\begin{itemize} \item[] Boolean value specifying the verbosity. \end{itemize}


\end{luacommand}\begin{luacommand}{{Sys:logMessage}(\meta{...})}
Prints objects to the TeX output, formatting them with tostring and separated by spaces.

Parameters:
\begin{parameterdescription}
	\item[\meta{...}] List of parameters.
\end{parameterdescription}



\end{luacommand}\begin{luacommand}{{Sys:putEdge}(\meta{edge},\meta{Edge})}
Assembles and outputs the TeX command to draw an edge.

Parameters:
\begin{parameterdescription}
	\item[\meta{Edge}] A lua edge object.
\end{parameterdescription}



\end{luacommand}\begin{luacommand}{{Sys:putTeXBox}(\meta{nodename},\meta{texnode},\meta{minX},\meta{minY},\meta{maxX},\meta{maxY},\meta{posX},\meta{posY},\meta{nodeName})}
Saves a box from the transfer box register.

Parameters:
\begin{parameterdescription}
	\item[\meta{texnode}] The box which contains the \TeX\ node.\item[\meta{minX}] Maximum y of the bounding box.\item[\meta{minY}] Minimal y of the bounding box.\item[\meta{posX}] X coordinate where to put the node in the output.\item[\meta{posY}] Y coordinate where to put the node in the output.\item[\meta{nodeName}] The name of the node in the box.
\end{parameterdescription}



\end{luacommand}\begin{luacommand}{{Sys:setBoxNumber}(\meta{bn})}
Init method, sets the box register number. This method is called when the \tikzname\ (pgf) library is loaded.

Parameters:
\begin{parameterdescription}
	\item[\meta{bn}] Number of the box register used for transfering boxes of the current graph.
\end{parameterdescription}



\end{luacommand}\begin{luacommand}{{Sys:setVerboseMode}(\meta{mode})}
Enables or disables verbose logging for the graph drawing library.

Parameters:
\begin{parameterdescription}
	\item[\meta{mode}] If true, enable verbose logging. Otherwise it'll be disabled.
\end{parameterdescription}



\end{luacommand}\begin{luacommand}{{Sys:unescapeTeXNodeName}(\meta{nodename})}
Removes the ``not yet positionedPGFGDINTERNAL'' prefix from a node name.

Parameters:
\begin{parameterdescription}
	\item[\meta{nodename}] Nodename without prefix.
\end{parameterdescription}


Return value:
\begin{itemize} \item[] The substring in question. \end{itemize}


See also:
\begin{itemize}
	\item[] |Sys:escapeTeXNodeName(nodename)|
\end{itemize}

\end{luacommand}
\end{filedescription}
\label{section-library-graphdrawing-lua-documentation-sys}
% This file has been generated from the lua sources using LuaDoc.
% To regenerate it call "make genluadoc" in
% doc/generic/pgf/version-for-luatex/en.

\begin{filedescription}{pgflibrarygraphdrawing-texboxregister.lua}


\begin{luacommand}{{TeXBoxRegister:getBox}(\meta{boxReference})}
Gets a box by its reference.

Parameters:
\begin{parameterdescription}
	\item[\meta{boxReference}] Reference id of the box to get.
\end{parameterdescription}



See also:
\begin{itemize}
	\item[] |TeXBoxRegister:insertBox(texbox)|
\end{itemize}

\end{luacommand}
\begin{luacommand}{{TeXBoxRegister:insertBox}(\meta{texbox})}
Adds the content of a \TeX\ box to the box register class. Contents of the box will be stored. 



\end{luacommand}

\end{filedescription}

\subsubsection{Helper Classes}
% This file has been generated from the lua sources using LuaDoc.
% To regenerate it call "make genluadoc" in
% doc/generic/pgf/version-for-luatex/en.

\begin{filedescription}{pgflibrarygraphdrawing-helper.lua}


\begin{luacommand}{{parseBraces}(\meta{str},\meta{default})}
Parses a braced list of {key}{value} pairs and returns a table mapping keys to values.



\end{luacommand}

\end{filedescription}
% This file has been generated from the lua sources using LuaDoc.
% To regenerate it call "make genluadoc" in
% doc/generic/pgf/version-for-luatex/en.

\begin{filedescription}{pgflibrarygraphdrawing-table-helpers.lua}


\begin{luacommand}{{table.combine\textunderscore{}pairs}(\meta{table},\meta{combine\_func},\meta{initial\_value})}
Combine all key/value pairs of \meta{table} to a single value using a combine function.  This is a very powerful function. It can be used for combining the key/value pairs of a table into a single string but can also be used to compute mathematical operations on tables, such as finding the maximum value in a table etc.  The main difference to |table.combine_values| is that keys and values are used to determine the combination value and that the key/value pairs are are passed to \meta{combine\_func} in a random order. 

Parameters:
\begin{parameterdescription}
	\item[\meta{table}] Table to iterate over.\item[\meta{combine\_func}] Function to be called for each key/value pair. It takes three parameters, the current combination value and the key/value pair. It is supposed to return a new combination value.\item[\meta{initial\_value}] Initial combination value. 
\end{parameterdescription}


Return value:
\begin{parameterdescription} 
  \item[] The final combination value after all key/value pairs have been passed over to \meta{combine\_func}. 
\end{parameterdescription}


\end{luacommand}
\begin{luacommand}{{table.combine\textunderscore{}values}(\meta{input},\meta{combine\_func},\meta{initial\_value})}
Combine all values of \meta{input} to a single value using a combine function.  This is a very powerful function. It can be used for combining the values of a table into a single string but can also be used to compute mathematical operations on tables, such as finding the maximum value in a table etc.  The main difference to |table.combine_pairs| is that the keys are ignored and that the values are passed to \meta{combine\_func} in the order they appear in the table. 

Parameters:
\begin{parameterdescription}
	\item[\meta{input}] Table to iterate over.\item[\meta{combine\_func}] Function to be called for each value. It takes two parameters, the current combination value and the current value. It is supposed to return a new combination value.\item[\meta{initial\_value}] Initial combination value. 
\end{parameterdescription}


Return value:
\begin{parameterdescription} 
  \item[] The final combination value after all values of \meta{input} have been passed over to \meta{combine\_func}. 
\end{parameterdescription}


\end{luacommand}
\begin{luacommand}{{table.copy}(\meta{source},\meta{target})}
Copies a table while preserving its metatable. 

Parameters:
\begin{parameterdescription}
	\item[\meta{source}] The table to copy.\item[\meta{target}] The table to which values are to be copied or |nil| if a new table is to be allocated. 
\end{parameterdescription}


Return value:
\begin{parameterdescription} 
  \item[] The \meta{target} table or a newly allocated table containing all keys and values of the \meta{source} table. 
\end{parameterdescription}


\end{luacommand}
\begin{luacommand}{{table.count\textunderscore{}pairs}(\meta{input})}
Count the key/value pairs in the table. 

Parameters:
\begin{parameterdescription}
	\item[\meta{input}] The table whose key/value pairs to count. 
\end{parameterdescription}


Return value:
\begin{parameterdescription} 
  \item[] Number of key/value pairs in the table. 
\end{parameterdescription}


\end{luacommand}
\begin{luacommand}{{table.filter\textunderscore{}keys}(\meta{table},\meta{filter\_func})}
Copies a table and filters out all keys using a function. 

Parameters:
\begin{parameterdescription}
	\item[\meta{table}] The table whose values are to be filtered.\item[\meta{filter\_func}] The test function to be called for each key of \meta{table}. If it returns |false| or |nil| for a key, that key will not be part of the result table. 
\end{parameterdescription}


Return value:
\begin{parameterdescription} 
  \item[] Copy of \meta{table} with its keys filtered using \meta{filter\_func}. 
\end{parameterdescription}


\end{luacommand}
\begin{luacommand}{{table.filter\textunderscore{}pairs}(\meta{table},\meta{filter\_func})}
Copies a table and filters out all key/value pairs using a function. 

Parameters:
\begin{parameterdescription}
	\item[\meta{table}] The table whose values are to be filtered.\item[\meta{filter\_func}] The test function to be called for each pair of \meta{table}. If it returns |false| or |nil| for a pair, that pair will not be part of the result table. 
\end{parameterdescription}


Return value:
\begin{parameterdescription} 
  \item[] Copy of \meta{table} with its pairs filtered using \meta{filter\_func}. 
\end{parameterdescription}


\end{luacommand}
\begin{luacommand}{{table.filter\textunderscore{}values}(\meta{input},\meta{filter\_func})}
Copies a table and filters out all values using a function. 

Parameters:
\begin{parameterdescription}
	\item[\meta{input}] The table whose values are to be filtered.\item[\meta{filter\_func}] The test function to be called for each value of the input table. If it returns |false| or |nil| for a value, that value will not be part of the result table. 
\end{parameterdescription}


Return value:
\begin{parameterdescription} 
  \item[] Copy of \meta{input} with its values filtered using \meta{filter\_func}. 
\end{parameterdescription}


\end{luacommand}
\begin{luacommand}{{table.find}(\meta{table},\meta{find\_func})}
Returns the first value in \meta{table} for which \meta{find\_func} returns |true|. 

Parameters:
\begin{parameterdescription}
	\item[\meta{table}] The table to search in.\item[\meta{find\_func}] A function to test values with. It receives a single parameter (a value of \meta{table}) and is supposed to return either |true| or |false|. 
\end{parameterdescription}


Return value:
\begin{parameterdescription} 
  \item[] The first value of \meta{table} for which \meta{find\_func} returns true. Returns |nil| if the function was |false| for al of the values in \meta{table}. 
\end{parameterdescription}


\end{luacommand}
\begin{luacommand}{{table.find\textunderscore{}index}(\meta{table},\meta{find\_func})}
Returns the index of the first value in \meta{table} for which \meta{find\_func} returns |true|. 

Parameters:
\begin{parameterdescription}
	\item[\meta{table}] The table to search in.\item[\meta{find\_func}] A function to test values with. It receives a single parameter (a value of \meta{table}) and is supposed to return either |true| or |false|. 
\end{parameterdescription}


Return value:
\begin{parameterdescription} 
  \item[] Index of the first value of \meta{table} for which \meta{find\_func} returns |true|. Returns |nil| if the function was |false| for all of the values in \meta{table}. 
\end{parameterdescription}


\end{luacommand}
\begin{luacommand}{{table.key\textunderscore{}iter}(\meta{table})}
Iterate over all keys of a table in random order. 

Parameters:
\begin{parameterdescription}
	\item[\meta{table}] The table whose keys to iterate over. 
\end{parameterdescription}


Return value:
\begin{parameterdescription} 
  \item[] An iterator for the keys of the table. 
\end{parameterdescription}


\end{luacommand}
\begin{luacommand}{{table.map}(\meta{input},\meta{map\_func})}
Maps key/value pairs of an \meta{input} table to a flat table of new values. 

Parameters:
\begin{parameterdescription}
	\item[\meta{input}] Table whose key/value pairs are to be mapped to new values.\item[\meta{map\_func}] The mapping function to be called for each key/value pair of \meta{input}. The value it returns for a pair will be inserted into the result table. 
\end{parameterdescription}


Return value:
\begin{parameterdescription} 
  \item[] A new table containing all values returned by \meta{map\_func} for the key/value pairs of the \meta{input} table. 
\end{parameterdescription}


\end{luacommand}
\begin{luacommand}{{table.map\textunderscore{}keys}(\meta{table},\meta{map\_func})}
Maps keys of a table to new keys in a copy of the table. 

Parameters:
\begin{parameterdescription}
	\item[\meta{table}] The table whose keys are to be mapped to new keys.\item[\meta{map\_func}] A function to be called for each key of \meta{table} in order to generate a new key to replace the old one in the result table. 
\end{parameterdescription}


Return value:
\begin{parameterdescription} 
  \item[] A new table with all keys of \meta{table} having been replaced with the keys returned from \meta{map\_func}. The original values are preserved. 
\end{parameterdescription}


\end{luacommand}
\begin{luacommand}{{table.map\textunderscore{}pairs}(\meta{table},\meta{map\_func})}
Maps keys and values of a table to new pairs of keys and values. 

Parameters:
\begin{parameterdescription}
	\item[\meta{table}] The table whose key and value pairs are to be replaced.\item[\meta{map\_func}] A function to be called for each key and value pair of \meta{table} in order to generate a new pair to replace the old one. 
\end{parameterdescription}


Return value:
\begin{parameterdescription} 
  \item[] A new table with all key and value pairs of \meta{table} having been replaced with the pairs returned from \meta{map\_func}. 
\end{parameterdescription}


\end{luacommand}
\begin{luacommand}{{table.map\textunderscore{}values}(\meta{input},\meta{map\_func})}
Maps values of a table to new values in a new table. 

Parameters:
\begin{parameterdescription}
	\item[\meta{input}] The table whose values are to be mapped to new values.\item[\meta{map\_func}] A function to be called for each value in order to generate a new value to replace the old one in the result table. 
\end{parameterdescription}


Return value:
\begin{parameterdescription} 
  \item[] A new table with all values of the \meta{input} table having been replaced with the values returned from \meta{map\_func}. 
\end{parameterdescription}


\end{luacommand}
\begin{luacommand}{{table.merge}(\meta{table1},\meta{table2},\meta{first\_metatable})}
Merges the key/value pairs of two tables.  This function merges the key/value pairs of the two input tables.  All |nil| values of the first table are overwritten by the corresponding values of the second table.  By default the metatable of the second input table is applied to the resulting table. If \meta{first\_metatable} is set to |true| however, the metatable of the first input table will be used. 

Parameters:
\begin{parameterdescription}
	\item[\meta{table1}] First table with key/value pairs.\item[\meta{table2}] Second table with key/value pairs.\item[\meta{first\_metatable}] Whether to inherit the metatable of \meta{table1} or not. 
\end{parameterdescription}


Return value:
\begin{parameterdescription} 
  \item[] A new table with the key/value pairs of the two input tables merged together. 
\end{parameterdescription}


\end{luacommand}
\begin{luacommand}{{table.randomized\textunderscore{}pair\textunderscore{}iter}(\meta{table})}
Iterate over the key/value pairs of \meta{table} in a truely random order. 

Parameters:
\begin{parameterdescription}
	\item[\meta{table}] The table whose key/value pairs to iterate over. 
\end{parameterdescription}


Return value:
\begin{parameterdescription} 
  \item[] A randomized iterator for the values of \meta{table}. 
\end{parameterdescription}


\end{luacommand}
\begin{luacommand}{{table.randomized\textunderscore{}value\textunderscore{}iter}(\meta{table})}
Iterate over the values of \meta{table} in a truely random order. 

Parameters:
\begin{parameterdescription}
	\item[\meta{table}] The table whose values to iterate over. 
\end{parameterdescription}


Return value:
\begin{parameterdescription} 
  \item[] A randomized iterator for the values of the table. 
\end{parameterdescription}


\end{luacommand}
\begin{luacommand}{{table.remove\textunderscore{}values}(\meta{input},\meta{remove\_func})}
Removes all values from \meta{input} for which \meta{remove\_func} returns |true|.  Important note: this method does not work with dictionaries. Make sure only to process number-indexed arrays with it. 

Parameters:
\begin{parameterdescription}
	\item[\meta{input}] The table to remove values from.\item[\meta{remove\_func}] Function to be called for each value of \meta{input}. If it returns |false|, the value will be removed from the table in-place. 
\end{parameterdescription}


Return value:
\begin{parameterdescription} 
  \item[] \meta{input} which was edited in-place. 
\end{parameterdescription}


\end{luacommand}
\begin{luacommand}{{table.update\textunderscore{}values}(\meta{table},\meta{update\_func})}
Update values of \meta{table} in-place using an update function. 

Parameters:
\begin{parameterdescription}
	\item[\meta{table}] The table whose values are to be updated.\item[\meta{update\_func}] A function that takes two parameters, the key/value pairs of \meta{table} and returns a new value to replace the old one. 
\end{parameterdescription}


Return value:
\begin{parameterdescription} 
  \item[] The input \meta{table}. 
\end{parameterdescription}


\end{luacommand}
\begin{luacommand}{{table.value\textunderscore{}iter}(\meta{table})}
Iterate over all values of a table.  FIXME: The iterators stops if a key's value is nil. But we actually want to continue iterating until the end of the table. 

Parameters:
\begin{parameterdescription}
	\item[\meta{table}] The table whose values to iterate over. 
\end{parameterdescription}


Return value:
\begin{parameterdescription} 
  \item[] An iterator for the values of the table. 
\end{parameterdescription}


\end{luacommand}

\end{filedescription}
% This file has been generated from the lua sources using LuaDoc.
% To regenerate it call "make genluadoc" in
% doc/generic/pgf/version-for-luatex/en.

\begin{filedescription}{pgflibrarygraphdrawing-iter-helpers.lua}


\begin{luacommand}{{iter.filter}(\meta{iterator},\meta{filter\_func})}
Skips all values of an iterator for which \meta{filter\_func} returns |false|. 

Parameters:
\begin{parameterdescription}
	\item[\meta{iterator}] Original \meta{iterator} of values.\item[\meta{filter\_func}] Filter function that takes a value of the original \meta{iterator} and is expected to return |false| if the value should be skipped. 
\end{parameterdescription}


Return value:
\begin{parameterdescription} 
  \item[] A modified iterator that skips values of \meta{iterator} for which \meta{filter\_func} returns |false|. 
\end{parameterdescription}


\end{luacommand}
\begin{luacommand}{{iter.map}(\meta{iterator},\meta{map\_func})}
Maps all values of an iterator to new values.  This function will cause loops to iterate over the values of the original \meta{iterator} replaced by the values returned from \meta{map\_func}. 

Parameters:
\begin{parameterdescription}
	\item[\meta{iterator}] Original iterator whose values are to be mapped to new ones.\item[\meta{map\_func}] Mapping function that takes a value of the original \meta{iterator} and maps it to a new value that is then returned to the loop instead. 
\end{parameterdescription}


Return value:
\begin{parameterdescription} 
  \item[] A modified iterator. 
\end{parameterdescription}


\end{luacommand}
\begin{luacommand}{{iter.times}(\meta{n})}
Causes a loop to run multiple times.  Use this iterator like this to perform 100 loops:\\ |for n in iter.times(100) do ... end|.  To iterate over the values $0, 10, 20, 30, ..., 100$ do:\\ |for n in iter.filter(iter.times(100), function (n) return n % 10 == 0 end)| 

Parameters:
\begin{parameterdescription}
	\item[\meta{n}] Number of loops. 
\end{parameterdescription}



\end{luacommand}

\end{filedescription}
% This file has been generated from the lua sources using LuaDoc.
% To regenerate it call "make genluadoc" in
% doc/generic/pgf/version-for-luatex/en.

\begin{filedescription}{pgflibrarygraphdrawing-traversal-helpers.lua}


\begin{luacommand}{{traversal.depth\textunderscore{}first\textunderscore{}dag}(\meta{graph},\meta{initial\_nodes})}
Iterator for traversing a directed acyclic \meta{graph} in depth-first order. 

Parameters:
\begin{parameterdescription}
	\item[\meta{graph}] A directed acyclic graph. 
\end{parameterdescription}


Return value:
\begin{parameterdescription} 
  \item[] An iterator for traversing \meta{graph} in a depth-first order. 
\end{parameterdescription}


\end{luacommand}
\begin{luacommand}{{traversal.topological\textunderscore{}sorting}(\meta{graph})}
Iterator for traversing a directed \meta{graph} using a topological sorting.  A topological sorting of a directed graph is a linear ordering of its nodes such that, for every edge |(u,v)|, |u| comes before |v|.  Important note: if performed on a graph with at least one cycle a topological sorting is impossible. Thus, the nodes returned from the iterator are not guaranteed to satisfy the ``|u| comes before |v|'' criterion. The iterator may even terminate early or loop forever. 

Parameters:
\begin{parameterdescription}
	\item[\meta{graph}] A directed acyclic graph. 
\end{parameterdescription}


Return value:
\begin{parameterdescription} 
  \item[] An iterator for traversing \meta{graph} in a topological order. 
\end{parameterdescription}


\end{luacommand}

\end{filedescription}

\else
You need to use Lua\TeX\ to typeset this part of the manual (and,
also, to use algorithmic graph drawing). 
\fi
  

\part{Libraries}
\label{part-libraries}

{\Large \emph{by Till Tantau}}


\bigskip
\noindent
In this part the library packages are documented. They
provide additional predefined graphic objects like new arrow heads or
new plot marks, but sometimes also extensions of the basic \pgfname\
or \tikzname\ system. The libraries are not loaded by default since
many users will not need them.

\medskip
\noindent
\begin{codeexample}[graphic=white]
\tikzset{
  ld/.style={level distance=#1},lw/.style={line width=#1},
  level 1/.style={ld=4.5mm, trunk,          lw=1ex ,sibling angle=60},
  level 2/.style={ld=3.5mm, trunk!80!leaf a,lw=.8ex,sibling angle=56},
  level 3/.style={ld=2.75mm,trunk!60!leaf a,lw=.6ex,sibling angle=52},
  level 4/.style={ld=2mm,   trunk!40!leaf a,lw=.4ex,sibling angle=48},
  level 5/.style={ld=1mm,   trunk!20!leaf a,lw=.3ex,sibling angle=44},
  level 6/.style={ld=1.75mm,leaf a,         lw=.2ex,sibling angle=40},
}
\pgfarrowsdeclare{leaf}{leaf}
  {\pgfarrowsleftextend{-2pt} \pgfarrowsrightextend{1pt}}
{
  \pgfpathmoveto{\pgfpoint{-2pt}{0pt}}
  \pgfpatharc{150}{30}{1.8pt}
  \pgfpatharc{-30}{-150}{1.8pt}
  \pgfusepathqfill
}

\newcommand{\logo}[5]
{
  \colorlet{border}{#1}
  \colorlet{trunk}{#2}
  \colorlet{leaf a}{#3}
  \colorlet{leaf b}{#4}
  \begin{tikzpicture}
    \scriptsize\scshape
    \draw[border,line width=1ex,yshift=.3cm,
          yscale=1.45,xscale=1.05,looseness=1.42]
      (1,0) to [out=90, in=0]    (0,1)  to [out=180,in=90]  (-1,0)
            to [out=-90,in=-180] (0,-1) to [out=0,  in=-90] (1,0) -- cycle;

    \coordinate (root) [grow cyclic,rotate=90]
    child {
      child [line cap=round] foreach \a in {0,1} {
        child foreach \b in {0,1} {
          child foreach \c in {0,1} {
            child foreach \d in {0,1} {
              child foreach \leafcolor in {leaf a,leaf b}
                { edge from parent [color=\leafcolor,-#5] }
        } } }
      } edge from parent [shorten >=-1pt,serif cm-,line cap=butt]
    };

    \node [align=center,below] at (0pt,-.5ex)
    { \textcolor{border}{T}heoretical \\ \textcolor{border}{C}omputer \\
      \textcolor{border}{S}cience };
  \end{tikzpicture}
}
\begin{minipage}{3cm}
  \logo{green!80!black}{green!25!black}{green}{green!80}{leaf}\\
  \logo{green!50!black}{black}{green!80!black}{red!80!green}{leaf}\\
  \logo{red!75!black}{red!25!black}{red!75!black}{orange}{leaf}\\
  \logo{black!50}{black}{black!50}{black!25}{}
\end{minipage}
\end{codeexample}

% Copyright 2003 by Till Tantau <tantau@cs.tu-berlin.de>.
%
% This program can be redistributed and/or modified under the terms
% of the LaTeX Project Public License Distributed from CTAN
% archives in directory macros/latex/base/lppl.txt.


\section{Arrow Tip Library}
\label{section-library-arrows}

\begin{package}{pgflibraryarrows}
  The package defines additional arrow tips, which are described
  below. See page~\pageref{standard-arrows} for the arrows tips that
  are defined by default. Note that neither the standard packages nor
  this package defines an arrow name containing |>| or |<|. These are
  left for the user to defined as he or she sees fit.
\end{package}

\subsection{Triangular Arrow Tips}

\begin{tabular}{ll}
  \symarrow{latex'} \\
  \symarrow{latex' reversed}  \\
  \symarrow{stealth'} \\
  \symarrow{stealth' reversed}\\
  \symarrow{triangle 90} \\
  \symarrow{triangle 90 reversed}   \\
  \symarrow{triangle 60} \\
  \symarrow{triangle 60 reversed}   \\
  \symarrow{triangle 45} \\
  \symarrow{triangle 45 reversed}   \\
  \symarrow{open triangle 90} \\
  \symarrow{open triangle 90 reversed}   \\
  \symarrow{open triangle 60} \\
  \symarrow{open triangle 60 reversed}   \\
  \symarrow{open triangle 45} \\
  \symarrow{open triangle 45 reversed}   \\
\end{tabular}

\subsection{Barbed Arrow Tips}

\begin{tabular}{ll}
  \symarrow{angle 90} \\
  \symarrow{angle 90 reversed}   \\
  \symarrow{angle 60} \\
  \symarrow{angle 60 reversed}   \\
  \symarrow{angle 45} \\
  \symarrow{angle 45 reversed}   \\
  \symarrow{hooks} \\
  \symarrow{hooks reversed} \\
\end{tabular}


\subsection{Bracket-Like Arrow Tips}

\begin{tabular}{ll}
  \sarrow{[}{]} \\
  \sarrow{]}{[} \\
  \sarrow{(}{)} \\
  \sarrow{)}{(}
\end{tabular}

\subsection{Circle and Diamond Arrow Tips}


\begin{tabular}{ll}
  \symarrow{o} \\
  \symarrow{*} \\
  \symarrow{diamond} \\
  \symarrow{open diamond}   \\
\end{tabular}



\subsection{Serif-Like Arrow Tips}

\begin{tabular}{ll}
  \symarrow{serif cm}
\end{tabular}


\subsection{Partial Arrow Tips}

\begin{tabular}{ll}
  \symarrow{left to} \\
  \symarrow{left to reversed} \\
  \symarrow{right to} \\
  \symarrow{right to reversed} \\
  \symarrow{left hook} \\
  \symarrow{left hook reversed} \\
  \symarrow{right hook} \\
  \symarrow{right hook reversed}
\end{tabular}



\subsection{Line Caps}

\begin{tabular}{ll}
  \carrow{round cap} \\
  \carrow{butt cap} \\
  \carrow{triangle 90 cap} \\
  \carrow{triangle 90 cap reversed} \\
  \carrow{fast cap} \\
  \carrow{fast cap reversed} \\
\end{tabular}


%%% Local Variables: 
%%% mode: latex
%%% TeX-master: "pgfmanual-pdftex-version"
%%% End: 

% Copyright 2006 by Till Tantau
%
% This file may be distributed and/or modified
%
% 1. under the LaTeX Project Public License and/or
% 2. under the GNU Free Documentation License.
%
% See the file doc/generic/pgf/licenses/LICENSE for more details.


\section{Automata Drawing Library}

\begin{tikzlibrary}{automata}
  This packages provides shapes and styles for drawing finite state
  automata and Turing machines. 
\end{tikzlibrary}


\subsection{Drawing Automata}

The automata drawing library is intended to make it easy to draw
finite automata and Turing machines. It does not cover every
situation imaginable, but most finite automata and Turing machines
found in text books can be drawn in a nice and convenient fashion
using this library. 

To draw an automaton, proceed as follows:
\begin{enumerate}
\item For each state of the automaton, there should be one node with
  the option |state|.
\item To place the states, you can either use absolute positions or
  relative positions, using options like |above| or |right|.
\item Give a unique name to each state node.
\item Accepting and initial states are indicated by adding the
  options |accepting| and |initial|, respectively, to the state
  nodes.
\item Once the states are fixed, the edges can be added. For this, the
  |edge| operation is most useful. It is, however, also possible to
  add edges after each node has been placed.
\item For loops, use the |edge [loop]| operation.
\end{enumerate}

Let us now see how this works for a real example. Let us consider a
nondeterminsitic four state automaton that checks whether an contains
the sequence $0^*1$ or the sequence $1^*0$. 
\begin{codeexample}[]
\begin{tikzpicture}[shorten >=1pt,node distance=2cm,on grid,auto]
  \draw[help lines] (0,0) grid (3,2);

  \node[state,initial]  (q_0)                      {$q_0$};
  \node[state]          (q_1) [above right=of q_0] {$q_1$};
  \node[state]          (q_2) [below right=of q_0] {$q_2$};
  \node[state,accepting](q_3) [below right=of q_1] {$q_3$};

  \path[->] (q_0) edge              node        {0} (q_1)
                  edge              node [swap] {1} (q_2)
            (q_1) edge              node        {1} (q_3)
                  edge [loop above] node        {0} ()
            (q_2) edge              node [swap] {0} (q_3)
                  edge [loop below] node        {1} ();
\end{tikzpicture}
\end{codeexample}


\subsection{States With and Without Output}

The |state| style actually just ``selects'' a default underlying
style. Thus, you can define multiple new complicated state style and
then simply set the |state| style to your given style to get the
desired kind of styles.

By default, the following state styles are defined:
\begin{stylekey}{/tikz/state without output}
  This node style causes nodes to be drawn circles. Also, this style
  calls |every state|.
\end{stylekey}

\begin{stylekey}{/tikz/state with output}
  This node style causes nodes to be drawn as split circles, that is,
  using the |circle split| shape. In the upper part of the shape you
  have the name of the style, in the lower part the output is
  placed. To specify the output, use the command |\nodepart{lower}|
  inside the node. This style also calls |every state|.
\begin{codeexample}[]
\begin{tikzpicture}
  \draw[help lines] (0,0) grid (3,2);

  \node[state without output] {$q_0$};
  
  \node[state with output] at (2,0) {$q_1$ \nodepart{lower} $00$};
\end{tikzpicture}
\end{codeexample}
\end{stylekey}

\begin{stylekey}{/tikz/state (initially state without output)}
  You should redefine it to something else, if you wish to use states
  of a different nature.
\begin{codeexample}[]
\begin{tikzpicture}[state/.style=state with output]
  \node[state]          {$q_0$ \nodepart{lower} $11$};
  \node[state] at (2,0) {$q_1$ \nodepart{lower} $00$};
\end{tikzpicture}
\end{codeexample}
\end{stylekey}

\begin{stylekey}{/tikz/every state (initially \normalfont empyt)}
  This style is used by |state with output| and also by
  |state without output|. By default, it does nothing, but you can use
  it to make your state look more fancy:
\begin{codeexample}[]
\begin{tikzpicture}[shorten >=1pt,node distance=2cm,on grid,>=stealth',
    every state/.style={draw=blue!50,very thick,fill=blue!20}]

  \node[state,initial]  (q_0)                      {$q_0$};
  \node[state]          (q_1) [above right=of q_0] {$q_1$};
  \node[state]          (q_2) [below right=of q_0] {$q_2$};

  \path[->] (q_0) edge              node [above left]  {0} (q_1)
                  edge              node [below left]  {1} (q_2)
            (q_1) edge [loop above] node               {0} ()
            (q_2) edge [loop below] node               {1} ();
\end{tikzpicture}
\end{codeexample}
\end{stylekey}


\subsection{Initial and Accepting States}

The styles |initial| and |accepting| are similar to the |state| style
as they also just select an ``underlying'' style, which installs the
actual settings for initial and accepting states.

Let us start with the initial states.
\begin{stylekey}{/tikz/initial (initially initial by arrow)}
  This style is used to draw initial states.
\end{stylekey}
\begin{stylekey}{/tikz/initial by arrow}
  This style causes an arrow and, possibly, some text to be added to
  the node. The arrow points from the text to the node. The node text
  and the direction and the distance can be set using the following
  key:
  \begin{key}{/tikz/initial text=\meta{text} (initially start)}
    This key sets the text to be used. Use an empty text to suppress
    all text.
  \end{key}
  \begin{key}{/tikz/initial where=\meta{direction} (initially left)}
    Set the place where the text should be shown. Allowed values are
    |above|, |below|, |left|, and |right|.
  \end{key}
  \begin{key}{/tikz/intial distance=\meta{distance} (initially 3ex)}
    Sets the length of the arrow leading from the text to the state
    node.
  \end{key}
  \begin{stylekey}{/tikz/every initial by arrow (initially \normalfont empty)}
    This style is executed at the beginning of every path that contains
    the arrow and the text. You can use it to, say, make the text red or
    whatever.
  \end{stylekey}
\begin{codeexample}[]
\begin{tikzpicture}[every initial by arrow/.style={text=red,->>}]
  \node[state,initial,initial distance=2cm] {$q_0$};
\end{tikzpicture}
\end{codeexample}
  %<<
\end{stylekey}
\begin{stylekey}{/tikz/initial above}
  This is a shorthand for |initial by arrow,initial where=above|.
\end{stylekey}
\begin{stylekey}{/tikz/initial below}
  Works similarly to the previous option.
\end{stylekey}
\begin{stylekey}{/tikz/initial left}
  Works similarly to the previous option.
\end{stylekey}
\begin{stylekey}{/tikz/initial right}
  Works similarly to the previous option.
\end{stylekey}

\begin{stylekey}{/tikz/initial by diamond}
  This style uses a diamond to indicate an initial node. 
\end{stylekey}

For the accepting states, the sitation is similar: There is also an
|accepting| style that selects the way accepting states are
rendered. There are now two options: First,
|accepting by arrow|, which works the same way as |initial by arrow|,
only with the direction of arrow reversed, and |accepting by double|,
where accepting states get a double line around them.

\begin{stylekey}{/tikz/accepting (initially accepting by double)}
  This style is used to draw accepting states.  You can replace
  this by the style |accepting by arrow| to get accepting states with
  an arrow leaving them.
\end{stylekey}

\begin{stylekey}{/tikz/accepting by double}
  This style causes a double line to be drawn arond a state.
\end{stylekey}

\begin{stylekey}{/tikz/accepting by arrow}
  This style causes an arrow and, possibly, some text to be added to
  the node. The arrow points to the text from the node.

  The same options as for initial states can be used, only with
  |initial| replaced by |accepting|:
  \begin{key}{/tikz/accepting text=\meta{text} (initially \normalfont empty)}
    This key sets the text to be used.
  \end{key}
  \begin{key}{/tikz/accepting where=\meta{direction} (initially right)}
    Set the place where the text should be shown. Allowed values are
    |above|, |below|, |left|, and |right|.
  \end{key}
  \begin{key}{/tikz/intial distance=\meta{distance} (initially 3ex)}
    Sets the length of the arrow leading from the text to the state
    node.
  \end{key}
  \begin{stylekey}{/tikz/every accepting by arrow (initially \normalfont empty)}
    Executed at the beginning of every path that contains
    the arrow and the text.
  \end{stylekey}  
\begin{codeexample}[]
\begin{tikzpicture}
  [shorten >=1pt,node distance=2cm,on grid,>=stealth',initial text=,
   every state/.style={draw=blue!50,very thick,fill=blue!20},
   accepting/.style=accepting by arrow]

  \node[state,initial]  (q_0)                      {$q_0$};
  \node[state]          (q_1) [above right=of q_0] {$q_1$};
  \node[state]          (q_2) [below right=of q_0] {$q_2$};
  \node[state,accepting](q_3) [below right=of q_1] {$q_3$};

  \path[->] (q_0) edge              node [above left]  {0} (q_1)
                  edge              node [below left]  {1} (q_2)
            (q_1) edge              node [above right] {1} (q_3)
                  edge [loop above] node               {0} ()
            (q_2) edge              node [below right] {0} (q_3)
                  edge [loop below] node               {1} ();
\end{tikzpicture}
\end{codeexample}
\end{stylekey}

\begin{stylekey}{/tikz/accepting above}
  This is a shorthand for |accepting by arrow,accepting where=above|.
\end{stylekey}
\begin{stylekey}{/tikz/accepting below}
  Works similarly to the previous option.
\end{stylekey}
\begin{stylekey}{/tikz/accepting left}
  Works similarly to the previous option.
\end{stylekey}
\begin{stylekey}{/tikz/accepting right}
  Works similarly to the previous option.
\end{stylekey}



\subsection{Examples}

In the following example, we once more typeset the automaton presented
in the previous sections. This time, we use the following rule for
accepting/initial state: Initial states are red, accepting states are
green, and normal states are orange. Then, we must find a path from a
red state to a green state. 

\begin{codeexample}[]
\begin{tikzpicture}[shorten >=1pt,node distance=2cm,on grid,>=stealth',thick,
    every state/.style={fill,draw=none,orange,text=white,circular drop shadow},
    accepting/.style  ={green!50!black,text=white},
    initial/.style    ={red,text=white}]

  \node[state,initial]  (q_0)                      {$q_0$};
  \node[state]          (q_1) [above right=of q_0] {$q_1$};
  \node[state]          (q_2) [below right=of q_0] {$q_2$};
  \node[state,accepting](q_3) [below right=of q_1] {$q_3$};

  \path[->] (q_0) edge              node [above left]  {0} (q_1)
                  edge              node [below left]  {1} (q_2)
            (q_1) edge              node [above right] {1} (q_3)
                  edge [loop above] node               {0} ()
            (q_2) edge              node [below right] {0} (q_3)
                  edge [loop below] node               {1} ();
\end{tikzpicture}
\end{codeexample}

The next example is the current candidate for the five-state busiest
beaver:

\begin{codeexample}[]
\begin{tikzpicture}[->,>=stealth',shorten >=1pt,%
                    auto,node distance=2cm,on grid,semithick,
                    inner sep=2pt,bend angle=45]
  \node[initial,state] (A)                    {$q_a$};
  \node[state]         (B) [above right=of A] {$q_b$};
  \node[state]         (D) [below right=of A] {$q_d$};
  \node[state]         (C) [below right=of B] {$q_c$};
  \node[state]         (E) [below=of D]       {$q_e$};

  \path [every node/.style={font=\footnotesize}]
        (A) edge              node {0,1,L} (B)
            edge              node {1,1,R} (C)
        (B) edge [loop above] node {1,1,L} (B)  
            edge              node {0,1,L} (C)
        (C) edge              node {0,1,L} (D)
            edge [bend left]  node {1,0,R} (E)    
        (D) edge [loop below] node {1,1,R} (D)
            edge              node {0,1,R} (A)
        (E) edge [bend left]  node {1,0,R} (A);
\end{tikzpicture}
\end{codeexample}


%%% Local Variables: 
%%% mode: latex
%%% TeX-master: "pgfmanual-pdftex-version"
%%% End: 

% Copyright 2006 by Till Tantau
%
% This file may be distributed and/or modified
%
% 1. under the LaTeX Project Public License and/or
% 2. under the GNU Free Documentation License.
%
% See the file doc/generic/pgf/licenses/LICENSE for more details.



\section{Background Library}

\label{section-tikz-backgrounds}

\begin{tikzlibrary}{backgrounds}
  This library defines ``backgrounds'' for pictures. This does not
  refer to background pictures, but rather to frames drawn around and
  behind pictures. For example, this package allows you to just add
  the |framed| option to a picture to get a rectangular box around
  your picture or |gridded| to put a grid behind your picture.
\end{tikzlibrary}

When this package is loaded, the following styles become available:
\begin{stylekey}{/tikz/show background rectangle}
  This style causes a rectangle to be drawn behind your graphic. This
  style option must be given to the |{tikzpicture}| environment or to
  the |\tikz| command.
\begin{codeexample}[]
\begin{tikzpicture}[show background rectangle]
  \draw (0,0) ellipse (10mm and 5mm);
\end{tikzpicture}
\end{codeexample}
  The size of the background rectangle is determined as follows:
  We start with the bounding box of the picture. Then, a certain
  separator distance is added on the sides. This distance can be
  different for the $x$- and $y$-directions and can be set using the
  following options:
  \begin{key}{/tikz/inner frame xsep=\meta{dimension} (initially 1ex)}
    Sets the additional horizontal separator distance for the
    background rectangle.    
  \end{key}
  \begin{key}{/tikz/inner frame ysep=\meta{dimension} (initially 1ex)}
    Same for the vertical separator distance.
  \end{key}  
  \begin{key}{/tikz/inner frame sep=\meta{dimension}}
    Sets the horizontal and vertical separator distances
    simultaneously. 
  \end{key}
  The following two styles make setting the inner separator a bit
  easier to remember:
  \begin{stylekey}{/tikz/tight background}
    Sets the inner frame separator to 0pt. The background rectangle
    will have the size of the bounding box. 
  \end{stylekey}
  \begin{stylekey}{/tikz/loose background}
    Sets the inner frame separator to 2ex.
  \end{stylekey}  

  You can influence how the background rectangle is rendered by setting
  the following style:
  \begin{stylekey}{/tikz/background rectangle (initially draw)}
    This style dictates how the background rectangle is drawn or
    filled. The default setting causes the path of the background
    rectangle to be drawn in the usual 
    way. Setting this style to, say, |fill=blue!20| causes a light
    blue background to be added to the picture. You can also use more
    fancy settings as shown in the following example:
\begin{codeexample}[]
\begin{tikzpicture}
  [background rectangle/.style=
     {double,ultra thick,draw=red,top color=blue,rounded corners},  
   show background rectangle]
  \draw (0,0) ellipse (10mm and 5mm);
\end{tikzpicture}
\end{codeexample}
    Naturally, no one in their right mind would use the above, but
    here is a nice background: 
\begin{codeexample}[]
\begin{tikzpicture}
  [background rectangle/.style=
     {draw=blue!50,fill=blue!20,rounded corners=1ex},
   show background rectangle]
  \draw (0,0) ellipse (10mm and 5mm);
\end{tikzpicture}
\end{codeexample}
  \end{stylekey}
\end{stylekey}

\begin{stylekey}{/tikz/framed}
  This is a shorthand for |show background rectangle|.
\end{stylekey}

\begin{stylekey}{/tikz/show background grid}
  This style behaves similarly to the |show background rectangle|
  style, but it will not use a rectangle path, but a grid. The lower
  left and upper right corner of the grid is computed in the same way
  as for the background rectangle:
\begin{codeexample}[]
\begin{tikzpicture}[show background grid]
  \draw (0,0) ellipse (10mm and 5mm);
\end{tikzpicture}
\end{codeexample}
  You can influence the background grid by setting
  the following style:
  \begin{stylekey}{/tikz/background grid (initially draw,help lines)}
    This style dictates how the background grid path is drawn. 
\begin{codeexample}[]
\begin{tikzpicture}
  [background grid/.style={thick,draw=red,step=.5cm},
   show background grid]
  \draw (0,0) ellipse (10mm and 5mm);
\end{tikzpicture}
\end{codeexample}
  \end{stylekey}
  This option can be combined with the |framed| option (use the
  |framed| option first):
\begin{codeexample}[]
\tikzset{background grid/.style={thick,draw=red,step=.5cm},
         background rectangle/.style={rounded corners,fill=yellow}}
\begin{tikzpicture}[framed,gridded]
  \draw (0,0) ellipse (10mm and 5mm);
\end{tikzpicture}
\end{codeexample}
\end{stylekey}

\begin{stylekey}{/tikz/gridded}
  This is a shorthand for |show background grid|.
\end{stylekey}

\begin{stylekey}{/tikz/show background top}
  This style causes a single line to be drawn at the top of the
  background rectangle. Normally, the line coincides exactly with the
  top line of the background rectangle:
\begin{codeexample}[]
\begin{tikzpicture}[
    background rectangle/.style={fill=yellow},
    framed,show background top]
  \draw (0,0) ellipse (10mm and 5mm);
\end{tikzpicture}
\end{codeexample}
  The following option allows you to lengthen (or shorten) the line:
  \begin{key}{/tikz/outer frame xsep=\meta{dimension} (initially 0pt)}
    The \meta{dimension} is added at the left and right side of the
    line. 
\begin{codeexample}[]
\begin{tikzpicture}
  [background rectangle/.style={fill=yellow},
   framed,
   show background top,
   outer frame xsep=1ex]
  \draw (0,0) ellipse (10mm and 5mm);
\end{tikzpicture}
\end{codeexample}
  \end{key}
  \begin{key}{/tikz/outer frame ysep=\meta{dimension} (initially 0pt)}
    This option does not apply to the top line, but to the left and
    right lines, see below.
  \end{key}
  \begin{key}{/tikz/outer frame sep=\meta{dimension}}
    Sets both the $x$- and $y$-separation.
  \end{key}
\begin{codeexample}[]
\begin{tikzpicture}
  [background rectangle={fill=blue!20},
   outer frame sep=1ex,%
   show background top,%
   show background bottom,%
   show background left,%
   show background right]
  \draw (0,0) ellipse (10mm and 5mm);
\end{tikzpicture}
\end{codeexample}
  You can influence how the line is drawn grid by setting
  the following style:
  \begin{stylekey}{/tikz/background top (initially draw)}
\begin{codeexample}[]
\tikzset{background rectangle/.style={fill=blue!20},
         background top/.style={draw=blue!50,line width=1ex}}
\begin{tikzpicture}[framed,show background top]
  \draw (0,0) ellipse (10mm and 5mm);
\end{tikzpicture}
\end{codeexample}
  \end{stylekey}
\end{stylekey}

\begin{stylekey}{/tikz/show background bottom}
  Works like the style for the top line.
\end{stylekey}
\begin{stylekey}{/tikz/show background left}
  Works similarly.
\end{stylekey}
\begin{stylekey}{/tikz/show background right}
  Works similarly.
\end{stylekey}



%%% Local Variables: 
%%% mode: latex
%%% TeX-master: "pgfmanual-pdftex-version"
%%% End: 

\section{Calc Library}

\begin{tikzlibrary}{calc}
  The library allows advanced Coordinate Calculations. It is documented in all detail in Section~\ref{tikz-lib-calc} on page~\pageref{tikz-lib-calc}.
\end{tikzlibrary}


% Copyright 2006 by Till Tantau
%
% This file may be distributed and/or modified
%
% 1. under the LaTeX Project Public License and/or
% 2. under the GNU Free Documentation License.
%
% See the file doc/generic/pgf/licenses/LICENSE for more details.


\section{Calendar Library}

\label{section-calender-snakes}

\begin{tikzlibrary}{calendar}
  The library defines the |\calendar| command, which can be used to
  typeset calendars. The command relies on the |\pgfcalendar| command
  from the |pgfcalendar| package, which is loaded automatically.

  The |\calendar| command is quite configurable, allowing you to
  produce all kinds of different calendars.
\end{tikzlibrary}


\subsection{Calendar Command}

The core command for creating calendars in \tikzname\ is the
|\calendar| command. It is available only inside |{tikzpicture}|
environments (similar to, say, the |\draw| command). 

\begin{command}{\calendar \meta{calendar specification}|;|}
  The syntax for this command is similar to commands like |\node| or
  |\matrix|. However, it has its complete own parser and only those
  commands described in the following will be recognized, nothing
  else. Note, furthermore, that a \meta{calendar specification} is not
  a path specification, indeed, no path is created for the calendar.

  \medskip
  \textbf{The specification syntax.}
  The \meta{calendar specification} must be a sequence of
  elements, each of which has one of the following structures:
  \begin{itemize}
  \item |[|\meta{options}|]|

    You provide \meta{options} in square brackets as
    in |[red,draw=none]|. These \meta{options} can be any \tikzname\ 
    option and they apply to the whole calendar. You can provide this
    element multiple times, the effect accumulates.
  \item |(|\meta{name}|)|

    This has the same effect as saying |[name=|\meta{name}|]|. The
    effect of providing a \meta{name} is explained later. Note
    alreadys that \emph{a calendar is not a node} and the \meta{name}
    is \emph{not the name of a node}.
  \item |at (|\meta{coordinate}|)|

    This has the same effect as saying |[at=(|\meta{coordinate}|)]|.
  \item |if (|\meta{date condition}|)| \meta{options or
      commands}\opt{|else|\meta{else options or commands}}

    The effect of such an |if| is explained later.
  \end{itemize}

  At the beginning of every calendar, the following style is used:
  \begin{itemize}
  \itemstyle{every calendar} This style is empty be default.
  \end{itemize}
  
  \medskip
  \textbf{The date range.}
  The overall effect of the |\calendar| command is to execute code for
  each day of a range of dates. This range of dates is set using the
  following option:
  \begin{itemize}
    \itemoption{dates}|=|\meta{start date}| to |\meta{end date} This
    option specifies the date range. Both the start and end date are
    specified as described on page~\pageref{calendar-date-format}. In
    short: You can provide ISO-format type dates like |2006-01-02|, you
    can replace the day of month by |last| to refer to the last day of a
    month (so |2006-02-last| is the same as |2006-02-28|), and you can
    add a plus sign followed by a number to specify an offset (so
    |2006-01-01+-1| is the same as |2005-12-31|).
  \end{itemize}
  It will be useful to fix two pieces of terminology for the following
  descriptions: The |\calendar| command iterates over the dates in the
  range. The \emph{current date} refers to the current date the
  command is processing as it iterates over the dates. For each
  current date code is executed, which will be called the
  \emph{current date code}. The current date code consists of
  different parts, to be detailed later.
  
  The central part of the current date code is the execution of the
  code |\tikzdaycode|. By default, this code simply produces a node
  whose text is set to the day of month. This means that unless further
  action is taken, all days of a calendar will be put on top of each
  other! To avoid this, you must modify the current date code to shift
  days around appropriately. Predefined arrangements like 
  |day list downward| or |week list| do this for you, but you can
  define arrangements yourself. Since defining an arrangement is a bit
  tricky, it is explained only later on. For the time being, let us
  use a predefined arrangement to produce our first calendar:

\begin{codeexample}[]
\tikz \calendar[dates=2000-01-01 to 2000-01-31,week list];  
\end{codeexample}

  \medskip
  \textbf{Changing the spacing.}
  In the above calendar, the spacing between the days is determined by
  the numerous options. Most arrangement do not use all of these
  options, but only those that apply naturally.
  \begin{itemize}
    \itemoption{day xshift}|=|\meta{dimension} specifies the
    horizontal shift between days. This is not the gap between days,
    but the shift between the anchors of their nodes. The default is
    |3.5ex|. 
\begin{codeexample}[]
\tikz \calendar[dates=2000-01-01 to 2000-01-31,week list,day xshift=3ex];  
\end{codeexample}
    \itemoption{day yshift}|=|\meta{dimension} specifies the
    vertical shift between days. Again, this is the shift between the
    anchors of their nodes. The default is |3ex|. 
\begin{codeexample}[]
\tikz \calendar[dates=2000-01-01 to 2000-01-31,week list,day yshift=2ex];  
\end{codeexample}
    \itemoption{month xshift}|=|\meta{dimension} specifies an
    additional  horizontal shift between different months.
    \itemoption{month yshift}|=|\meta{dimension} specifies an
    additional  vertical shift between different months. 
\begin{codeexample}[]
\tikz \calendar[dates=2000-01-01 to 2000-02-last,week list,
                month yshift=0pt];  
\end{codeexample}
\begin{codeexample}[]
\tikz \calendar[dates=2000-01-01 to 2000-02-last,week list,
                month yshift=1cm];  
\end{codeexample}
  \end{itemize}

  \medskip
  \textbf{Changing the position of the calendar.}
  The calendar is placed in such a way that, normally, the anchor of
  the first day label is at the origin. This can be changed by using
  the |at| option. When you say |at={(1,1)}|, this anchor of the first
  day will lie at coordinate $(1,1)$.

  In general, arrangements will not always place the anchor of the
  first day at the origin. Sometimes, additional spacing rules get in
  the way. There are different ways of addressing this problem: First,
  you can just ignore it. Since calendars are often placed in their own
  |{tikzpicture}| and since their size if computed automatically, the
  exact position of the origin often does not matter at all. Second,
  you can put the calendar inside a node as in
  |...node {\tikz \calendar...}|. This allows you to position the node
  in the normal ways using the node's anchors. Third, you can be very
  clever and use a single-cell matrix. The advantage is that a matrix
  allows you to provide any anchor of any node inside the matrix as an
  anchor for the whole matrix. For example, the following calendar is
  placed in such a way the center of 2000-01-20 lies on the position
  $(2,2)$:
\begin{codeexample}[]
\begin{tikzpicture}    
  \draw[help lines] (0,0) grid (3,2);
  \matrix [anchor=cal-2000-01-20.center] at (2,2)
  { \calendar(cal)[dates=2000-01-01 to 2000-01-31,week list]; \\};
\end{tikzpicture}
\end{codeexample}
  Unfortunately, the matrix-base positions, which is the cleanest way,
  isnot as portable as the other approaches (it currently does not
  work with the \textsc{svg} backend for instance). 

  \medskip
  \textbf{Changing the appearance of days.}
  As mentioned before, each day in the above calendar is produced by
  an execution of the |\tikzdaycode|. Each time this code is executed,
  the coordinate system will have been setup appropriately to place
  the day of the month correctly. You can change both the code and its
  appearance using the following options.
  \begin{itemize}
    \itemoption{day code}|=|\meta{code}
    This option allows you to change the code that is executed for
    each day. The default is to create a node with an appropriate
    name, but you can change this:
\begin{codeexample}[]
\tikz \calendar[dates=2000-01-01 to 2000-01-31,week list,
                day code={\fill[blue] (0,0) circle (2pt);}];  
\end{codeexample}
    The default code is the following:
\begin{codeexample}[code only]
\node[name=\pgfcalendarsuggestedname,every day]{\tikzdaytext};
\end{codeexample}
    The first part causes the day nodes to be accessible via the
    following names: If \meta{name} is the name given to the calendar
    via a |name=| option or via the specification element
    |(|\meta{name}|)|, then |\pgfcalendarsuggestedname| will expand to
    \meta{name}|-|\meta{date}, where \meta{date} is the date of the
    day that is currently being processed in ISO format .

    For example, if January 1, 2006 is being processed and the
    calendar has been named |mycal|, then the node containg the |1|
    for this date will be names |mycal-2006-01-01|. You can later
    reference this node.
\begin{codeexample}[]
\begin{tikzpicture}
  \calendar (mycal) [dates=2000-01-01 to 2000-01-31,week list];

  \draw[red] (mycal-2000-01-20) circle (4pt);
\end{tikzpicture}
\end{codeexample}

    \itemoption{day text}|=|\meta{text}
    This option changes the setting of the |\tikzdaytext|. By default,
    this macro simply yields the current day of month, but you can
    change it arbitrarily. Here is a silly example:
\begin{codeexample}[]
\tikz \calendar[dates=2000-01-01 to 2000-01-31,week list,
                day text=x];  
\end{codeexample}
    More useful examples are based on using the |\%| command. This
    command is redefined inside a |\pgfcalendar| to mean the same as
    |\pgfcalendarshorthand|. (The original meaning of |\%| is lost
    inside the calendar, you need to save if before the calendar if
    you really need it.)

    The |\%| inserts the current day/month/year/day of week in a
    certain format into the text. The first letter following the |\%|
    selects the type (permisslbe values are |d|, |m|, |y|, |w|), the
    second letter specifies how the value should be displayed (|-|
    means numerically, |=| means numerically with leading
    space, |0| means numerically with leading zeros, |t| means
    textual, and |.| means textual, abbreviated). For example |\%d0|
    gives the day with a leading zero (for more details see
    the description of |\pgfcalendarshorthand| on
    page~\pageref{pgfcalendarshorthand}).

    Let us redefine the |day text| so that it yields the day with a
    leading zero:
\begin{codeexample}[leave comments]
\tikz \calendar[dates=2000-01-01 to 2000-01-31,week list,
                day text=\%d0];  
\end{codeexample}
    \itemstyle{every day}
    This style is executed by the default node code for each day. The
    default setting is
\begin{codeexample}[code only]
anchor=base east
\end{codeexample}
    The |every day| style is useful for changing the way days
    look. For example, let us make all days red:
\begin{codeexample}[leave comments]
\tikzstyle{every day}+=[red]
\tikz \calendar[dates=2000-01-01 to 2000-01-31,week list];
\end{codeexample}
  \end{itemize}

  \medskip
  \textbf{Changing the appearance of month and year labels.}
  In addition to the days of a calendar, labels for the months and
  even years (for really long calendars) can be added. These labels
  are only added once per month or year and this is not done by
  default. Rather, special styles starting with |month label|
  place these labels and make them visible:
\begin{codeexample}[]
\tikz \calendar[dates=2000-01-01 to 2000-02-last,week list,
                month label above centered];
\end{codeexample}

  The following options change the appearance of the month and year
  label:
  \begin{itemize}
    \itemoption{month code}|=|\meta{code}
    This option allows you to specify what the macro |\tikzmonthcode|
    should expand to.

    By default, the |\tikzmonthcode| it is set to
\begin{codeexample}[code only]
\node[every month]{\tikzmonthtext};
\end{codeexample}
    Note that this node is not named by default.
    \itemoption{month text}|=|\meta{text}
    This option allows you to change the macro |\tikzmonthtext|. By
    default, the month text is a long textual presentation of the
    current month being typeset. 
\begin{codeexample}[leave comments]
\tikz \calendar[dates=2000-01-01 to 2000-01-31,week list,
                month label above centered,  
                month text=\textcolor{red}{\%mt} \%y-];
\end{codeexample}
    \itemstyle{every month}
    This style, which is empty by default, can be used to change the
    appearance of month labels.
    \itemoption{year code}|=|\meta{code} Works like |month code|,
    only for years.
    \itemoption{year text}|=|\meta{text} Works like |month text|,
    only for years.
    \itemstyle{every year} Works like |every month|,
    only for years.
  \end{itemize}
  
  \medskip
  \textbf{Date ifs.}
  Much of the power of the |\calendar| command comes from the use of
  conditionals. There are two equivalent way of specifying such a
  conditional. First, you can add the text
  |if (|\meta{conditions}|) |\meta{code or options} to your
  \meta{calendar specification}, possibly followed by |else|\meta{else
    code or options}. You can have multiple such conditionals (but
  you cannot nest them in this simple manner). The second way is to
  use the following option:
  \begin{itemize}
    \itemoption{if}|=(|\meta{coditions}|)|\meta{code or
      options}\opt{|else|\meta{else code or options}} This option has
    the same effect as giving a corresponding if in the \meta{calendar
      specification}.  The option is mostly useful for use in the
    |every calendar| style, where you cannot provide if conditionals
    otherwise. 
  \end{itemize}
  Now, regardless of how you specify a conditional, it has the
  following effect (individually and independently for each date in
  the calendar):
  \begin{enumerate}
  \item It is checked whether the current date is one of the
    possibilities listed in \meta{coditions}. An example of such a
    condition is |Sunday|. Thus, when you write
    |if (Saturday,Sunday) {foo}|,  then |foo| will be executed for
    every day in the calendar that is a Saturday \emph{or} a Sunday.

    The command |\ifdate| and, thereby, |\pgfcalendarifdate| are used
    to evaluate the \meta{conditions}, see
    page~\pageref{pgfcalendarifdate} for a complete list of possible
    tests. The most useful tests are: Tests like |Monday| and so on,
    |workday| for the days Monday to Friday, |weekend| for Saturday
    and Sunday, |equals| for testing whether the current date equals a
    given date, |at least| and  |at least| for comparing the current
    date with a given date.
  \item If the date passes the check, the \meta{code or options} is
    evaluated in a manner to be described in a moment; if the date
    fails, the \meta{else code or options} is evaluated, if present.

    The \meta{code or options} can either be some code. This is
    indicated by surrounding the code with curly braces. It can also
    be a list of \tikzname\ options. This is indicated by surrounding
    the options with square brackets. For example in the date test
    |if (Sunday) {\draw...} else {\fill...}| there are two pieces of
    code involved. By comparison, |if (Sunday) [red] else [green]|
    involves two options.

    If \meta{code or options} is code, it is simply executed (for the
    current day). If it is a list of options, these options are passed
    to a scope surrounding the current date.
  \end{enumerate}
  Let us now have a look at some examples. First, we use a conditional
  to make all Sundays red.
\begin{codeexample}[]
\tikz
  \calendar
    [dates=2000-01-01 to 2000-01-31,week list]
    if (Sunday) [red];
\end{codeexample}
  Next, let us do something on a specific date:
\begin{codeexample}[]
\tikz
  \calendar
    [dates=2000-01-01 to 2000-01-31,week list]
    if (Sunday)            [red]
    if (equals=2000-01-20) {\draw (0,0) circle (8pt);};
\end{codeexample}
  You might wonder why the circle seems to be ``off'' the
  date. Actually, it is centered on the date, it is just that the date
  label uses the |base east| anchor, which shifts the label up and
  right. To overcome this problem we can change the anchor:
\begin{codeexample}[]
\tikzstyle{every day}=[anchor=mid]
\tikz
  \calendar
    [dates=2000-01-01 to 2000-01-31,week list]
    if (Sunday)            [red]
    if (equals=2000-01-20) {\draw (0,0) circle (8pt);};
\end{codeexample}
  However, the single day dates are now no longer aligned
  correctly. For this, we can change the day text to |\%d=|,
  which adds a space at the beginning of single day
  text.
  
  In the following, more technical information is covered. Most
  readers may wish to skip it.

  \medskip
  \textbf{The current date code.}
  As mentioned earlier, for each date in the calendar the current date
  code is executed. It is the job of this code to shift around date
  nodes, to render the date nodes, to draw the month labels and to do
  all other stuff that is necessary to draw a calendar.

  The current daet code consists of the following parts, in this order:
  \begin{enumerate}
  \item The before-scope code.
  \item A scope is opened.
  \item The at-begin-scope code.
  \item All date-ifs from the \meta{calendar specification} are
    executed.
  \item The at-end-scope code.
  \item The scope is closed.
  \item The after-scope code.
  \end{enumerate}
  All of the codes mentioned above can be changed using appropriate
  options, see below. In case you wonder why so many are needed, the
  reason is that the current date code as a whole is not
  surrounded by a scope or \TeX\ group. This means that code executed
  in the before-scope code and in the after-scope code has an effect
  on all following days. For example, if the after-scope code modifies
  the transformation matrix by shifting everything downward, all
  following days will be shifted downward. If each day does this, you
  get a list of days, one below the other.

  However, you do not always want code to have an effect on everything
  that follows. For instance, if a day has the date-if
  |if (Sunday) [red]|, we only want this Sunday to red, not all
  following days also. Similarly, sometimes it is easier to compute
  the position of a day relative to a fixed origin and we do not want
  any modifications of the transformation matrix to have an effect
  outside the scope.

  By cleverly adjusting the different codes, all sorts of different
  day arrangements are possible.

  \begin{itemize}
    \itemoption{execute before day scope}|=|\meta{code} This
    code is executed before everything else for each date. Multiple
    calls of this option have an accumulative effect. Thus, if you use
    this option twice, the code from the first use is used first for
    each day, followed by the code given the second time.
    \itemoption{execute at begin day scope}|=|\meta{code}
    This code is execute before everything else inside the scope of
    the current date. Again, the effect is accumulative.
    \itemoption{execute at end day scope}|=|\meta{code}
    This code is executed just before the day scope is
    closed. The effect is also accumulative, however, in reverse
    order. This is useful to pair, say, |\scope| and |\endscope|
    commands in at-begin- and at-end-code.
    \itemoption{execute after day scope}|=|\meta{code} This
    is executed at the very end of the current date, outside the
    scope. The accumulation is also in reverse.
  \end{itemize}
\end{command}


In the rest of the following subsections we have a look at how the
different scope codes can be used to create different calendar
arrangements. 


\subsubsection{Creating a Simple List of Days}

We start with a list the days of the calendar, one day below the
other. For this, we simply shift the coordinate system downward at the
end of the code for each day. This shift must be \emph{outside} the
day scope as we want day shifts to accumulate. Thus, we use the
following code: 
\begin{codeexample}[]
\tikz
  \calendar [dates=2000-01-01 to 2000-01-08,
             execute after day scope=
               {\pgftransformyshift{-1em}}];
\end{codeexample}
Clearly, we can use this approach to create day lists going up,
down, right, left, or even diagonally.


\subsubsection{Adding a Month Label}

We now want to add a month label to the left of the beginning of each
month. The idea is to do two things:
\begin{enumerate}
\item We add code that is executed only on the first of each month.
\item The code is executed before the actual day is rendered. This
  ensures that options applying to the days do not affect the month
  rendering.
\end{enumerate}
We have two options where we should add the month code: Either we add
it at the beginning of the day scope or before. Either will work fine,
but it might be safer to put the code inside the scope to ensure that
settings to not inadventerdly ``leak outside.''
\begin{codeexample}[]
\tikz
  \calendar
    [dates=2000-01-01 to 2000-01-08,
     execute after day scope={\pgftransformyshift{-1em}},
     execute at begin day scope=
       {\ifdate{day of month=1}{\tikzmonthcode}{}},
     set style={{every month}+=[anchor=base east,xshift=-2em]}];
\end{codeexample}

In the above code we used the |\ifdate|\marg{condition}\marg{then
  code}\marg{else code} command, which is described on
page~\pageref{ifdate} in detail and which has much the same effect as
|if (|\meta{condition}|)|\marg{then code}| else |\marg{else code}, but
works in normal code.


\subsubsection{Creating a Week List Arrangement}

Let us now address a more complicated arrangement: A week list. In
this arrangement there is line for each week. The horizontal placement
of the days is thus that all Mondays lie below each other, likewise
for all Tuesdays, and so on.

In order to typeset this arrangement, we can use the following
approach: The origin of the coordinate system rests at the anchor for
the Monday of each week. That means that at the end of each week the
origin is moved downward one line. On all other days, the origin at
the end of the day code is the same as at the beginning. To position
each day correctly, we use code inside and at the beginning of the day
scope to horizontally shift the day according to its day of week. 
\begin{codeexample}[]
\tikz
  \calendar
    [dates=2000-01-01 to 2000-01-20,
     % each day is shifted right according to the day of week
     execute at begin day scope=
       {\pgftransformxshift{\pgfcalendarcurrentweekday em}},
     % after each week, the origin is shifted downward:
     execute after day scope=
       {\ifdate{Sunday}{\pgftransformyshift{-1em}}{}}];
\end{codeexample}


\subsubsection{Creating a Month List Arrangement}

For another example, let us create an arrangment that contains one
line for each month. This is easy enough to do as for weeks, unless we
add the following requirement: Again, we want all days in a column to
have the same day of week. Since months start on different days of
week, this means that each row has to have an individual offset.

One possible way is to use the following approach: After each month
(or at the beginning of each month) we advance the vertical position
of the offset by one line. For horizontal placement, inside the day
scope we locally shift the day by its day of month. Furthermore, we
must additionally shift the day to ensure that the first day of the
month lies on the correct day of week column. For this, we remember
this day of week the first time we see it.
\begin{codeexample}[]
\newcount\mycount
\tikz
  \calendar
    [dates=2000-01-01 to 2000-02-last,
     execute before day scope=
     {
       \ifdate{day of month=1} {
         % Remember the weekday of first day of month
         \mycount=\pgfcalendarcurrentweekday
         % Shift downward
         \pgftransformyshift{-1em}
       }{}
     },
     execute at begin day scope=
     {
       % each day is shifted right according to the day of month
       \pgftransformxshift{\pgfcalendarcurrentday em}
       % and additionally according to the weekday of the first
       \pgftransformxshift{\the\mycount em}
     }];
\end{codeexample}


\subsection{Arrangements}

An \emph{arrangement} specifies how the days of calendar are arranged
on the page. The calendar library defines a number of predefined
arrangements.

We start with arrangements in which the days are listed in a long
line. 

\begin{itemize}
  \itemstyle{day list downward}
  This style causes the days of a month to be typeset one below the
  other. The shift between days is given by |day yshift|. Between
  month an additional shift of |month yshift| is added.
\begin{codeexample}[]
\tikz
  \calendar [dates=2000-01-28 to 2000-02-03,
             day list downward,month yshift=1em];
\end{codeexample}
  \itemstyle{day list upward}
  works as above, only the list grows upward instead of downward.
\begin{codeexample}[]
\tikz
  \calendar [dates=2000-01-28 to 2000-02-03,
             day list upward,month yshift=1em];
\end{codeexample}
  \itemstyle{day list right}
  This style also works as before, but the list of days grows to the
  right. Instead of |day yshift| and |month yshift|, the values of
  |day xshift| and |month xshift| are used.
\begin{codeexample}[]
\tikz
  \calendar [dates=2000-01-28 to 2000-02-03,
             day list right,month xshift=1em];
\end{codeexample}
  \itemstyle{day list left}
  as above, but the list grows left.
\end{itemize}

The next arrangement lists days weekwise.

\begin{itemize}
  \itemstyle{week list}
  This style creates one row for each week in the range. The value
  of |day xshift| is used for the distance between days in each week
  row, the value of |day yshift| is used for the distance between
  rows. In both cases, ``distance''  refers to the distance between
  the anchors of the nodes of the days (or, more generally, the
  distance between the origins of the little pictures created for each
  day).

  The days inside each week are shifted such that Monday is always
  at the first position (to change this, you need to copy and then
  modify the code appropriately). If the date range does not start on
  a Monday, the first line will not start in the first column, but
  rather in the column appropriate for the first date in the range.

  At the beginning of each month (except for the first month in the
  range) an additional vertical space of |month yshift| is added. If
  this is set to |0pt| you get a continuous list of days.
\begin{codeexample}[]
\tikz
  \calendar [dates=2000-01-01 to 2000-02-last,week list];
\end{codeexample}
\begin{codeexample}[]
\tikz
  \calendar [dates=2000-01-01 to 2000-02-last,week list,
             month yshift=0pt];
\end{codeexample}
\end{itemize}

The following arrangement gives a very compact view of a whole year.
\begin{itemize}
  \itemstyle{month list}
  In this arrangement there is a row for each month. As for the
  |week list|, the |day xshift| is used for the horizontal distance.
  For the vertical shft, |month yshift| is used.

  In each row, all days of the month are listed alongside each
  other. However, it is once more ensured that days in each column lie
  on the same day of week. Thus, the very first column contains only
  Mondays. If a month does not start with a Monday, its days are
  shifted to the right such that the days lie on the correct columns. 
\end{itemize}
\begin{codeexample}[]
\sffamily\scriptsize    
\tikz
  \calendar [dates=2000-01-01 to 2000-12-31,
             month list,month label left,month yshift=1.25em]
            if (Sunday) [black!50];
\end{codeexample}


\subsection{Month Labels}

For many calendars you may wish to add a labels to each month. We have
already covered how month nodes are created and rendered in the
description of the |\calendar| command: use |month text|,
|every month|, and also |month code| (if necessary) to change the
appearance of the month labels.

What we have not yet covered is where these labels are placed. By
default, they are not placed at all as there is no good ``default
position'' for them. Instead, you can use one of the following options
to specify a position for the labels:
\begin{itemize}
  \itemstyle{month label left}
  Places the month label to the left of the first day of the
  month. (For |week list| and |month list| where a month does not
  start on a Monday, the position is chosen ``as if'' the month had
  started on a Monday --  which is usually exactly what you want.)
\begin{codeexample}[]
\tikz
  \calendar [dates=2000-01-28 to 2000-02-03,
             day list downward,month yshift=1em,
             month label left];
\end{codeexample}
  \itemstyle{month label left vertical}
  This style works like the above style, only the label is rotated
  counterclockwise by 90 degrees.
\begin{codeexample}[]
\tikz
  \calendar [dates=2000-01-28 to 2000-02-03,
             day list downward,month yshift=1em,
             month label left vertical];
\end{codeexample}
  \itemstyle{month label right}
  This style places the month label to the right of the row in which
  the first day of the month lies. This means that for a day list the
  label is to the right of the first day, for a week list it is to the
  right of the first week, and for a month list it is to the right of
  the whole month.
\begin{codeexample}[]
\tikz
  \calendar [dates=2000-01-28 to 2000-02-03,
             day list downward,month yshift=1em,
             month label right];
\end{codeexample}
  \itemstyle{month label right vertical}
  as above, only the label is rotated clockwise by 90 degrees.
\begin{codeexample}[]
\tikz
  \calendar [dates=2000-01-28 to 2000-02-03,
             day list downward,month yshift=1em,
             month label right vertical];
\end{codeexample}

  \itemstyle{month label above left}
  This style places the month label above of the row of the first day,
  flushed left to the leftmost column. The amount by which the label
  is raised is fixed to |1.25em|; use the |yshift| option with the
  month node to modify this.
\begin{codeexample}[]
\tikz
  \calendar [dates=2000-01-28 to 2000-02-03,
             day list right,month xshift=1em,
             month label above left];
\end{codeexample}
\begin{codeexample}[]
\tikz
  \calendar [dates=2000-01-20 to 2000-02-10,
             week list,month label above left];
\end{codeexample}
  
  \itemstyle{month label above centered}
  works as above, only the label is centered above the row containing
  the first day.
\begin{codeexample}[]
\tikz
  \calendar [dates=2000-02-01 to 2000-02-last,
             day list right,month label above centered];
\end{codeexample}
\begin{codeexample}[]
\tikz
  \calendar [dates=2000-01-20 to 2000-02-10,
             week list,month label above centered];
\end{codeexample}
   
  \itemstyle{month label above right}
  works as above, but flushed right
\begin{codeexample}[]
\tikz
  \calendar [dates=2000-01-20 to 2000-02-10,
             week list,month label above right];
\end{codeexample}

  \itemstyle{month label below left}
  Works like |month label above left|, only the label is placed below
  the row. This placement is not really useful with the |week list|
  arrangement, but rather with the |day list right| or |month list|
  arrangement.
\begin{codeexample}[]
\tikz
  \calendar [dates=2000-02-01 to 2000-02-last,
             day list right,month label below left];
\end{codeexample}
  \itemstyle{month label below centered}
  works like |month label above centered|, only below.
\begin{codeexample}[]
\tikz
  \calendar [dates=2000-02-01 to 2000-02-last,
             day list right,month label below centered];
\end{codeexample}
\end{itemize}


\subsection{Examples}

In the following, some example calendars are shown that come either
from real applications or are just nice to look at.

Let us start with a year-2100-countdown, in which we cross out dates
as we approach the big celebration. For
this, we set the shape to |strike out| for these dates.

\begin{codeexample}[leave comments]
\begin{tikzpicture}
  \calendar
  [
    dates=2099-12-01 to 2100-01-last,
    week list,inner sep=2pt,month label above centered,
    month text=\%mt \%y0
  ]
  if (at most=2099-12-29) [nodes={strike out,draw}]
  if (weekend)            [black!50,nodes={draw=none}]
  ;
\end{tikzpicture}
\end{codeexample}

The next calendar shows a deadline, which is 10 days in the future
from the current date. The last three days before the deadline are in
red, because we really should be done by then. All days on which we
can no longer work on the project are crossed out.

\begin{codeexample}[leave comments]
\begin{tikzpicture}
  \calendar
  [
    dates=\year-\month-\day+-25 to \year-\month-\day+25,
    week list,inner sep=2pt,month label above centered,
    month text=\textit{\%mt \%y0}
  ]
  if (at least=\year-\month-\day) {}
    else [nodes={strike out,draw}]
  if (at most=\year-\month-\day+7)
    [green!50!black]
  if (between=\year-\month-\day+8 and \year-\month-\day+10)
    [red]
  if (Sunday)
    [gray,nodes={draw=none}]
  ;
\end{tikzpicture}
\end{codeexample}

The following example is a futuristic calendar that is all about circles:

\begin{codeexample}[]
\sffamily

\colorlet{winter}{blue}  
\colorlet{spring}{green!60!black}  
\colorlet{summer}{orange}  
\colorlet{fall}{red}  

% A counter, since TikZ is not clever enough (yet) to handle
% arbitrary angle systems.
\newcount\mycount

\begin{tikzpicture}[transform shape]
  \tikzstyle{every day}=[anchor=mid,font=\fontsize{6}{6}\selectfont]
  \node{\normalsize\the\year};
  \foreach \month/\monthcolor in
    {1/winter,2/winter,3/spring,4/spring,5/spring,6/summer,
     7/summer,8/summer,9/fall,10/fall,11/fall,12/winter}
  {
    % Computer angle:
    \mycount=\month
    \advance\mycount by -1
    \multiply\mycount by 30
    \advance\mycount by -90

    % The actual calendar
    \calendar at (\the\mycount:6.4cm)
    [
      dates=\the\year-\month-01 to \the\year-\month-last,
    ]
    if (day of month=1) {\color{\monthcolor}\tikzmonthcode}
    if (Sunday) [red]
    if (all) 
    {
      % Again, compute angle
      \mycount=1
      \advance\mycount by -\pgfcalendarcurrentday
      \multiply\mycount by 11
      \advance\mycount by 90
      \pgftransformshift{\pgfpointpolar{\mycount}{1.4cm}}
    };
  }
\end{tikzpicture}
\end{codeexample}

Next, lets us have a whole year in a tight column:
\begin{codeexample}[leave comments]
\begin{tikzpicture}
  \small\sffamily
  \colorlet{darkgreen}{green!50!black}
  \calendar[dates=\year-01-01 to \year-12-31,week list,
            month label left,month yshift=0pt,
            month text=\textcolor{darkgreen}{\%m0}]
           if (Sunday) [black!50];
\end{tikzpicture}
\end{codeexample}


%%% Local Variables: 
%%% mode: latex
%%% TeX-master: "pgfmanual-pdftex-version"
%%% End: 

% Copyright 2008 by Till Tantau
%
% This file may be distributed and/or modified
%
% 1. under the LaTeX Project Public License and/or
% 2. under the GNU Free Documentation License.
%
% See the file doc/generic/pgf/licenses/LICENSE for more details.


\section{Chains}

\label{section-chains}

\begin{tikzlibrary}{chains}
  This library defines options for creating chains.
\end{tikzlibrary}


\subsection{Overview}

\emph{Chains} are sequences of nodes that are -- typically -- arranged
in an o row or a column and that are -- typically -- connected by
edges. More generally, they can be used to position nodes of a
branching network in a systematic manner. For the positioning of nodes
in rows and columns you can also use matrices, see
Section~\ref{section-matrices}, but chains can also be used
to describe the connections between nodes that have already been
connected using, say, matrices. Thus, it often makes sense to use
matrices for the positioning of elements and chains to describe the
connections.



\subsection{Starting and Continuing a Chain}

Typically, you construct one chain at a time, but it is
permissible to have construct multiple chains simultaneously. In this
case, the chains must be named differently and you must specify for
each node which chain it belongs to.

The first step toward creating a chain is to use the |start chain|
option.

\begin{key}{/tikz/start chain=\opt{\meta{chain name}}\opt{\meta{direction}}}
  This key should, but need not, be given as an option to a scope
  enclosing all nodes of the chain. Typically, this will be a |scope|
  or the whole |tikzpicture|, but it might just be a path on which all
  nodes of the chain are found.
  If no \meta{chain name} is given, the default value |chain| will be
  used instead.

  The key starts a chain named \meta{chain name} and makes it
  \emph{active}, which means that is currently being constructed. The
  |start chain| can be issued only once to activate a chain, inside a
  scope in which a chain is active you cannot use this option once
  more (for the same chain name). The chain stops being active at the
  end of the scope in which the |start chain| command was given.

  Although chains are only locally active (that is, active inside the
  scope the |start chain| command was issued), the information
  concerning the chains is stored globally and it is possible
  to \emph{continue} a chain after a scope has ended. For this, the
  |continue chain| option can be used, which allows you to reactivate
  an existing chain in another scope.

  The \meta{direction} is used to determine the placement rule for
  nodes on the chain. If it is omitted, the current value of the
  following key is used:
  \begin{key}{/tikz/chain default direction=\meta{direction}
      (initially going right)}
    This \meta{direction} is used in a |chain| option, if no other
    \meta{direction} is specified.
  \end{key}

  The \meta{direction} can have two different forms:
  \declare{|going |\meta{options}} or
  \declare{|placed |\meta{options}}. The effect of these rules will be
  explained in the description of the |on chain| option. Right now,
  just remember that the \meta{direction} you provide with the |chain|
  option applies to the whole chain.

  Other than this, this key has no further effect. In particular, to
  place nodes on the chain, you must use the |on chain| option,
  described next.
\begin{codeexample}[]
\begin{tikzpicture}[start chain]
  % The chain is called just "chain"
  \node [on chain] {A};
  \node [on chain] {B};
  \node [on chain] {C};
\end{tikzpicture}
\end{codeexample}

\begin{codeexample}[]
\begin{tikzpicture}
  % Same as above, using the scope shorthand
  { [start chain]
    \node [on chain] {A};
    \node [on chain] {B};
    \node [on chain] {C};
  }
\end{tikzpicture}
\end{codeexample}

\begin{codeexample}[]
\begin{tikzpicture}[start chain=1 going right,
                    start chain=2 going below,
                    node distance=5mm,
                    every node/.style=draw]
  \node [on chain=1] {A};
  \node [on chain=1] {B};
  \node [on chain=1] {C};

  \node [on chain=2] at (0.5,-.5) {0};
  \node [on chain=2] {1};
  \node [on chain=2] {2};

  \node [on chain=1] {D};
\end{tikzpicture}
\end{codeexample}
\end{key}

\begin{key}{/tikz/continue chain=\opt{\meta{chain name}}\opt{\meta{direction}}}
  This option allows you to (re)activate an existing chain and to
  possibly change the default direction. If the |chain name| is
  missing, the name of the innermost activated chain is used. If no
  chain is activated, |chain| is used.

  Let us have a look at the two different applications of this
  option. The first is to change the direction of a chain as it is
  begin constructed. For this, just give this option somewhere inside
  the scope of the chain.
\begin{codeexample}[]
\begin{tikzpicture}[start chain=going right,node distance=5mm]
  \node [draw,on chain] {Hello};
  \node [draw,on chain] {World};
  \node [draw,continue chain=going below,on chain] {,};
  \node [draw,on chain] {this};
  \node [draw,on chain] {is};
\end{tikzpicture}
\end{codeexample}

  The second application is to reactivate a chain after it ``has
  already been closed down.''

\begin{codeexample}[]
\begin{tikzpicture}[node distance=5mm,
                    every node/.style=draw]
  { [start chain=1]
    \node [on chain] {A};
    \node [on chain] {B};
    \node [on chain] {C};
  }

  { [start chain=2 going below]
    \node [on chain=2] at (0.5,-.5) {0};
    \node [on chain=2] {1};
    \node [on chain=2] {2};
  }

  { [continue chain=1]
    \node [on chain] {D};
  }
\end{tikzpicture}
\end{codeexample}
\end{key}


\subsection{Nodes on a Chain}

\begin{key}{/tikz/on chain=\opt{\meta{chain name}}\opt{\meta{direction}}}
  This key should be given as an option to a node. When the option is
  used, the \meta{chain name} must be the name of a chain that has
  been started using the |start chain| option. If \meta{chain name} is
  the empty string, the current value of the innermost activated chain
  is used. If this option is used several times for a node, only the
  last invocation ``wins.'' (To place a node on several chains, use
  the |\chainin| command repeatedly.)

  The \meta{direction} part is optional. If present it sets the
  direction used for this node, otherwise the \meta{direction}
  that was given to the original |start chain| option is used (or of
  the last |continue chain| option, which allows you to change this
  default).

  The effects of this option are the following:
  \begin{enumerate}
  \item An internal counter (there is one, local, counter
    for each chain) is increased. This counter reflects the current
    number of the node in the chain, where the first node is node 1,
    the second is node 2, and so on.

    This value of this internal counter is globally stored in the
    macro \declare{|\tikzchaincount|}.
  \item If the node does not yet have a name, (having been given using
    the |name| option or the name-syntax), the name of the node is set to
    \meta{chain name}|-|\meta{value of the internal chain
      counter}. For instance, if the chain is called |nums|, the first
    node would be named |nums-1|, the second |nums-2|, and so on. For
    the default chain name |chain|, the first node is named |chain-1|,
    the second |chain-2|, and so on.
  \item Independently of whether the name has been provided
    automatically or via the |name| option, the name of the node is
    globally stored in the macro \declare{|\tikzchaincurrent|}.
  \item Except for the first node, the macro
    \declare{|\tikzchainprevious|} is now globally set to the name of
    the node of the previous node on the chain. For the first node of
    the chain, this macro is globally set to the empty string.
  \item Except possibly for the first node of the chain, the placement
    rule is now executed. The placement rule is just a \tikzname\ option
    that is applied automatically to each node on the chain. Depending
    on the form of the \meta{direction} parameter (either the locally
    given one or the one given to the |start chain| option), different
    things happen.

    First, it makes a difference whether the \meta{direction} starts
    with |going| or with |placed|. The difference is that in the first
    case, the placement rule is not applied to the first node of the
    chain, while in the second case the placement rule is applied also
    to this first node. The idea is that a |going|-direction indicates
    that we are ``going somewhere relative to the previous node''
    whereas a |placed| indicates that we are ``placing nodes according
    to their number.''

    Independently of which form is used, the \meta{text} inside
    \meta{direction} that follows |going| or |placed| (separated by a
    compulsory space) can have two different effects:
    \begin{enumerate}
    \item If it contains an equal sign, then this \meta{text} is used
      as the placement rule, that is, it is simply executed.
    \item If it does not contain an equal sign, then
      \meta{text}|=of \tikzchainprevious| is used as the placement
      rule.
    \end{enumerate}

    Note that in the first case, inside the \meta{text} you have
    access to |\tikzchainprevious| and |\tikzchaincount| for doing
    your positioning calculations.
\begin{codeexample}[]
\begin{tikzpicture}[start chain=circle placed {at=(\tikzchaincount*30:1.5)}]
  \foreach \i in {1,...,10}
    \node [on chain] {\i};

  \draw (circle-1) -- (circle-10);
\end{tikzpicture}
\end{codeexample}
  \item
    The following style is executed:
    \begin{stylekey}{/tikz/every on chain}
      This key is executed for every node of a chain, including the
      first one.
    \end{stylekey}
  \end{enumerate}

  Recall that the standard replacement rule has a form like
  |right=of (\tikzchainprevious)|. This means that each
  new node is placed to the right of the previous one, spaced by the
  current value of |node distance|.
\begin{codeexample}[]
\begin{tikzpicture}[start chain,node distance=5mm]
  \node [draw,on chain] {};
  \node [draw,on chain] {Hallo};
  \node [draw,on chain] {Welt};
\end{tikzpicture}
\end{codeexample}

  The optional \meta{direction} allows us to temporarily change the
  direction in the middle of a chain:
\begin{codeexample}[]
\begin{tikzpicture}[start chain,node distance=5mm]
  \node [draw,on chain] {Hello};
  \node [draw,on chain] {World};
  \node [draw,on chain=going below] {,};
  \node [draw,on chain] {this};
  \node [draw,on chain] {is};
\end{tikzpicture}
\end{codeexample}

  You can also use more complicated computations in the \meta{direction}:
\begin{codeexample}[]
\begin{tikzpicture}[start chain=going {at=(\tikzchainprevious),shift=(30:1)}]
  \draw [help lines] (0,0) grid (3,2);
  \node [draw,on chain] {1};
  \node [draw,on chain] {Hello};
  \node [draw,on chain] {World};
  \node [draw,on chain] {.};
\end{tikzpicture}
\end{codeexample}
\end{key}

For each chain, two special ``pseudo nodes'' are created.

\begin{predefinednode}{\meta{chain name}-begin}
  This node is the same as the first node on the chain. It is only
  defined after a first node has been defined.
\end{predefinednode}

\begin{predefinednode}{\meta{chain name}-end}
  This node is the same as the (currently) last node on the chain. As
  the chain is extended, this node changes.
\end{predefinednode}

The |on chain| option can also be used, in conjunction with
|late options|, to add an already existing node to a chain. The
following command, which is only defined inside scopes where a
|start chain| option is present, simplifies this process.

\begin{command}{\chainin |(|\meta{existing name}|)| \opt{\oarg{options}}}
  This command makes it easy to add a node to chain that has already
  been constructed. This node may even be part of a another chain.

  When you say |\chainin (some node);|, the node |some node| must
  already exist. It will then be made part of the current chain. This
  does not mean that the node can be changed (it is already
  constructed, after all), but the |join| option can be used to join
  |some node| to the previous last node on the chain and subsequent
  nodes will be placed relative to |some node|.

  It is permissible to give the |on chain| option inside the
  \meta{options} in order to specify on which chain the node should be
  put.

  This command is just a shortcut for
\begin{quote}
|\path (|\meta{existing name}|) [late options={on chain,every chain in,|\meta{options}|}]|
\end{quote}
  In particular, it is possible to continue to path after a |\chainin|
  command, though that does not seem very useful.

\begin{codeexample}[]
\begin{tikzpicture}[node distance=5mm,
                    every node/.style=draw,every join/.style=->]
  \draw [help lines] (0,0) grid (3,2);

  \node[red] (existing) at (0,2) {existing};

  { [start chain]
    \node [draw,on chain,join] {Hello};
    \node [draw,on chain,join] {World};
    \chainin (existing) [join];
    \node [draw,on chain,join] {this};
    \node [draw,on chain,join] {is};
  }
\end{tikzpicture}
\end{codeexample}

  Here is an example where nodes are positioned using a matrix and
  then connected using a chain
{\catcode`\|=12
\begin{codeexample}[]
\begin{tikzpicture}[every node/.style=draw]
  \matrix [matrix of nodes,column sep=5mm,row sep=5mm]
  {
    |(a)|           World & |(b) [circle]|             peace \\
    |(c)|           be    & |(d) [isosceles triangle]| would \\
    |(e) [ellipse]| great & |(f)|                      ! \\
  };

  { [start chain,every on chain/.style={join=by ->}]
    \chainin (a);
    \chainin (b);
    \chainin (d);
    \chainin (c);
    \chainin (e);
    \chainin (f);
  }
\end{tikzpicture}
\end{codeexample}
}
\end{command}



\subsection{Joining Nodes on a Chain}

\begin{key}{/tikz/join=\opt{|with |\meta{with} }\opt{|by |\meta{options}}}
  When this key is given to any node on a chain (except possibly for
  the first node), an |edge| command is added after the node. The
  |with| part specifies which node should be used for the start point
  of the edge; if the |with| part is omitted, the |\tikzchainprevious|
  is used. This |edge| command gets the \meta{options} as parameter
  and the current node as its target. If there is no
  previous node and no |with| is given, no |edge| command gets
  executed.
  \begin{stylekey}{/tikz/every join}
    This style is executed each time this command is used.
  \end{stylekey}

  Note that is makes sense to call this option several times for a
  node, in order to connect it to several nodes. This is especially
  useful for joining in branches, see the next section.
\begin{codeexample}[]
\begin{tikzpicture}[start chain,node distance=5mm,
                    every join/.style={->,red}]
  \node [draw,on chain,join] {};
  \node [draw,on chain,join] {Hallo};
  \node [draw,on chain,join] {Welt};
  \node [draw,on chain=going below,
         join,join=with chain-1 by {blue,<-}] {foo};
\end{tikzpicture}
\end{codeexample}
\end{key}


\subsection{Branches}

A \emph{branch} is a chain that (typically only temporarily) extends
an existing chain. The idea is the following: Suppose we are
constructing a chain and at some node |x| there is a fork. In this
case, one (or even more) branches starts at this fork. For each branch
a chain is created, but the first node on this chain should
be~|x|. For this, it is useful to use |\chainin| on the node |x| to
make it part of the different branch chains and to name the branch
chains in some way that reflects the name of the main chain.

The |start branch| option provides a shorthand for doing exactly what
was just described.

\begin{key}{/tikz/start branch=\meta{branch name}\opt{\meta{direction}}}
  This key is used in the same manner as the |start chain| command,
  however, the effect is slightly different:
  \begin{itemize}
  \item This option may only be used if some chain is already active
    and there is a (last) node on this chain. Let us call this node
    the \meta{fork node}.
  \item The chain is not just called \meta{branch name}, but
    \meta{current chain}|/|\meta{branch name}. For instance, if the
    \meta{fork node} is part of the chain called |trunk| and the
    \meta{branch name} is set to |left|, the complete chain name of
    the branch is |trunk/left|. The \meta{branch name} must be given,
    there is no default value.
  \item The \meta{fork node} is automatically ``chained into'' the
    branch chain as its first node. Thus, for the first node on the
    branch that you provide, the |join| option will cause it to be
    connected to the fork node.
  \end{itemize}
\begin{codeexample}[]
\begin{tikzpicture}[every on chain/.style=join,every join/.style=->,
                    node distance=2mm and 1cm]
  { [start chain=trunk]
    \node [on chain] {A};
    \node [on chain] {B};

    { [start branch=numbers going below]
      \node [on chain] {1};
      \node [on chain] {2};
      \node [on chain] {3};
    }
    { [start branch=greek going above]
      \node [on chain] {$\alpha$};
      \node [on chain] {$\beta$};
      \node [on chain] {$\gamma$};
    }

    \node [on chain,join=with trunk/numbers-end,join=with trunk/greek-end] {C};
    { [start branch=symbols going below]
      \node [on chain] {$\star$};
      \node [on chain] {$\circ$};
      \node [on chain] {$\int$};
    }
  }
\end{tikzpicture}
\end{codeexample}
\end{key}

\begin{key}{/tikz/continue branch=\meta{branch name}\opt{\meta{direction}}}
  This option works like the |continue chain| option, only
  \meta{current chain}|/|\meta{branch name} is used as the chain name,
  rather than just \meta{branch name}.
\begin{codeexample}[]
\begin{tikzpicture}[every on chain/.style=join,every join/.style=->,
                    node distance=2mm and 1cm]
  { [start chain=trunk]
    \node [on chain] {A};
    \node [on chain] {B};
    { [start branch=numbers going below] } % just a declaration,
    { [start branch=greek   going above] } % we will come back later
    \node [on chain] {C};

    % Now come the branches...
    { [continue branch=numbers]
      \node [on chain] {1};
      \node [on chain] {2};
    }
    { [continue branch=greek]
      \node [on chain] {$\alpha$};
      \node [on chain] {$\beta$};
    }
  }
\end{tikzpicture}
\end{codeexample}
\end{key}

% Copyright 2008 by Till Tantau and Mark Wibrow
%
% This file may be distributed and/or modified
%
% 1. under the LaTeX Project Public License and/or
% 2. under the GNU Free Documentation License.
%
% See the file doc/generic/pgf/licenses/LICENSE for more details.

\section{Circuit Libraries}
\label{section-library-circuits}

\emph{Written and documented by Till Tantau, and Mark Wibrow. Inspired
by the work of Massimo Redaelli.}

\subsection{Introduction}

The circuit libraries can be used to draw different kinds of
electrical or logical circuits. There is not a single library for
this, but a whole hierarchy of libraries that work in concert. The
main design goal was to create a balance between ease-of-use and
ease-of-extending, while creating high-quality graphical
representations of circuits.

\subsubsection{A First Example}

\begin{codeexample}[]
\begin{tikzpicture}[circuit ee IEC,x=3cm,y=2cm,semithick,
                    every info/.style={font=\footnotesize},
                    small circuit symbols,
                    set resistor graphic=var resistor IEC graphic,
                    set diode graphic=var diode IEC graphic,
                    set make contact graphic= var make contact IEC graphic]
  % Let us start with some contacts:
  \foreach \contact/\y in {1/1,2/2,3/3.5,4/4.5,5/5.5}
  {
    \node [contact] (left contact \contact) at (0,\y) {};
    \node [contact] (right contact \contact) at (1,\y) {};
  }
  \draw (right contact 1) -- (right contact 2) -- (right contact 3)
     -- (right contact 4) -- (right contact 5);

  \draw (left contact 1) to [diode] ++(down:1)
                         to [voltage source={near start,
                                             direction info={volt=3}},
                             resistor={near end,ohm=3}] ++(right:1)
                         to (right contact 1);
  \draw (left contact 1) to [resistor={ohm=4}] (right contact 1);
  \draw (left contact 1) to [resistor={ohm=3}] (left contact 2);
  \draw (left contact 2) to [voltage source={near start,
                                             direction info={<-,volt=8}},
                             resistor={ohm=2,near end}] (right contact 2);
  \draw (left contact 2) to [resistor={near start,ohm=1},
                             make contact={near end,info'={[red]$S_1$}}]
                         (left contact 3);
  \draw (left contact 3) to [current direction'={near start,info=$\iota$},
                             resistor={near end,info={$R=4\Omega$}}]
                         (right contact 3);
  \draw (left contact 4) to [voltage source={near start,
                                             direction info={<-,volt=8}},
                             resistor={ohm=2,near end}] (right contact 4);
  \draw (left contact 3) to [resistor={ohm=1}] (left contact 4);
  \draw (left contact 4) to [resistor={ohm=3}] (left contact 5);
  \draw (left contact 5) to [resistor={ohm=4}] (right contact 5);
  \draw (left contact 5) to [diode] ++(up:1)
                         to [voltage source={near start,
                                             direction info={volt=3}},
                             resistor={near end,ohm=3}] ++(right:1)
                         to (right contact 5);
\end{tikzpicture}
\end{codeexample}

An important feature of the circuit library is that the appearance of
a circuit can be configured in general ways and that the labels are
placed automatically by default. Here is the graphic once more,
generated from \emph{exactly the same source code}, with only
the options of the |{tikzpicture}| environment replaced by
|[rotate=-90,circuit ee IEC,x=3.25cm,y=2.25cm]|:

\begin{tikzpicture}[rotate=-90,circuit ee IEC,x=3cm,y=2.25cm]
  % Let us start with some contacts:
  \foreach \contact/\y in {1/1,2/2,3/3.5,4/4.5,5/5.5}
  {
    \node [contact] (left contact \contact) at (0,\y) {};
    \node [contact] (right contact \contact) at (1,\y) {};
  }
  \draw (right contact 1) -- (right contact 2) -- (right contact 3)
     -- (right contact 4) -- (right contact 5);

  \draw (left contact 1) to [diode] ++(down:1)
                      to [voltage source={near start,direction info={volt=3}},
                          resistor={near end,ohm=3}] ++(right:1)
                      to (right contact 1);
  \draw (left contact 1) to [resistor={ohm=4}] (right contact 1);
  \draw (left contact 1) to [resistor={ohm=3}] (left contact 2);
  \draw (left contact 2) to [voltage source={near start,
                                          direction info={<-,volt=8}},
                          resistor={ohm=2,near end}] (right contact 2);
  \draw (left contact 2) to [resistor={near start,ohm=1},
                          make contact={near end,info'={[red]$S_1$}}] (left contact 3);
  \draw (left contact 3) to [current direction'={near start,info=$\iota$},
                          resistor={near end,info={$R=4\Omega$}}]
                            (right contact 3);
  \draw (left contact 4) to [voltage source={near start,
                                          direction info={<-,volt=8}},
                          resistor={ohm=2,near end}] (right contact 4);
  \draw (left contact 3) to [resistor={ohm=1}] (left contact 4);
  \draw (left contact 4) to [resistor={ohm=3}] (left contact 5);
  \draw (left contact 5) to [resistor={ohm=4}] (right contact 5);
  \draw (left contact 5) to [diode] ++(up:1)
                      to [voltage source={near start,direction info={volt=3}},
                          resistor={near end,ohm=3}] ++(right:1)
                      to (right contact 5);
\end{tikzpicture}


\subsubsection{Symbols}

A circuit typically consists of numerous electronic elements like
logical gates or resistors or diodes that are connected by wires. In
\pgfname/\tikzname, we use nodes for the
electronic elements and normal lines for the wires. \tikzname\ offers
a large number of different ways of positioning and connecting nodes
in general, all of which can be used here. Additionally, the
|circuits| library defines an additional useful |to|-path that is
particularly useful for elements like a resistor on a line.

There are many different names that are used to refer to electrical
``elements,'' so a bit of terminology standardization is useful: We
will call such elements \emph{symbols}. A \emph{symbol shape} is a
\pgfname\ shape declared using the |\pgfdeclareshape| command. A
\emph{symbol node} is a node whose shape is a symbol shape.


\subsubsection{Symbol Graphics}

Symbols can be created by
|\node[shape=some symbol shape]|. However, in order to represent some
symbols correctly, just using standard \pgfname\ shapes is not
sufficient. For instance, most symbols have a visually appealing
``default size,'' but the size of a symbol shape depends only on the
current values of parameters like |minimum height| or |inner xsep|.

For these reasons, the circuit libraries introduce the concept of a
\emph{symbol graphic}. This is a style that causes a |\node| to
not only have the correct shape, but also the correct size and the
correct path usage. More generally, this style may set up things in any
way so that the ``symbol looks correct''. When you write, for
instance, |\node[diode]|, then the style called |diode graphic| is
used, which in turn is set to something like
|shape=diode IEC,draw,minimum height=...|.

Here is an overview of the different kinds of circuit libraries:

\begin{itemize}
\item The \tikzname-library |circuits| defines general keys for
  creating circuits. Mostly, these keys are useful for defining more
  specialized libraries.

  You normally do not use this library directly since it does not
  define any symbol graphics.
\item The \tikzname-library |circuits.logic| defines keys for creating
  logical gates like and-gates or xor-gates. However, this library
  also does not actually define any symbol graphics; this is done by
  two sublibraries:
  \begin{itemize}
  \item The library |circuits.logic.US| defines symbol graphics that
    cause the logical gates to be rendered in the ``US-style.'' It
    includes all of the above libraries and you can use this library
    directly.
  \item The library |circuits.logic.IEC| also defines symbol graphics
    for logical gates, but it uses rectangular gates rather that the
    round US-gates. This library can coexist peacefully with the above
    library, you can change which symbol graphics are used ``on the
    fly.''
  \end{itemize}
\item The \tikzname-library |cirucits.ee| defines keys for symbols
  from electrical engineering like resistors or capacitors. Again,
  sublibraries define the actual symbol graphics.
  \begin{itemize}
  \item The library |circuits.ee.IEC| defines symbol shapes that
    follow the IEC norm.
  \end{itemize}
\item The \pgfname-libraries |shapes.gates.*| define (circuit) symbol
  shapes. However, you normally do not use these shapes directly,
  rather you use a style that uses an appropriate symbol graphic,
  which in turn uses one of these shapes.
\end{itemize}

Let us have a look at a simple example. Suppose we wish to create a
logical circuit. Then we first have to decide which symbol graphics we
would like to use. Suppose we wish to use the US-style, then we would
include the library |circuits.logic.US|. If you wish to use IEC-style
symbols, use |circuits.logic.IEC|. If you cannot decide, include both:
\begin{codeexample}[code only]
\usetikzlibrary{circuits.logic.US,circuits.logic.IEC}
\end{codeexample}
To create a picture that contains a US-style circuit you can now use
the option |circuit logic US|. This will set up keys like |and gate| to
create use an appropriate symbol graphic for rendering an |and gate|. Using the |circuit logic IEC| instead will set up |and gate| to
use another symbol graphic.

\begin{codeexample}[]
\begin{tikzpicture}[circuit logic US]
  \matrix[column sep=7mm]
  {
    \node (i0) {0}; &                            & \\
                    & \node [and gate] (a1) {};  & \\
    \node (i1) {0}; &                            & \node [or gate] (o) {};\\
                    & \node [nand gate] (a2) {}; & \\
    \node (i2) {1}; &                            & \\
  };
  \draw (i0.east) -- ++(right:3mm) |- (a1.input 1);
  \draw (i1.east) -- ++(right:3mm) |- (a1.input 2);
  \draw (i1.east) -- ++(right:3mm) |- (a2.input 1);
  \draw (i2.east) -- ++(right:3mm) |- (a2.input 2);
  \draw (a1.output) -- ++(right:3mm) |- (o.input 1);
  \draw (a2.output) -- ++(right:3mm) |- (o.input 2);
  \draw (o.output) -- ++(right:3mm);
\end{tikzpicture}
\end{codeexample}


\begin{codeexample}[]
\begin{tikzpicture}[circuit logic IEC]
  \matrix[column sep=7mm]
  {
    \node (i0) {0}; &                            & \\
                    & \node [and gate] (a1) {};  & \\
    \node (i1) {0}; &                            & \node [or gate] (o) {};\\
                    & \node [nand gate] (a2) {}; & \\
    \node (i2) {1}; &                            & \\
  };
  \draw (i0.east) -- ++(right:3mm) |- (a1.input 1);
  \draw (i1.east) -- ++(right:3mm) |- (a1.input 2);
  \draw (i1.east) -- ++(right:3mm) |- (a2.input 1);
  \draw (i2.east) -- ++(right:3mm) |- (a2.input 2);
  \draw (a1.output) -- ++(right:3mm) |- (o.input 1);
  \draw (a2.output) -- ++(right:3mm) |- (o.input 2);
  \draw (o.output) -- ++(right:3mm);
\end{tikzpicture}
\end{codeexample}



\subsubsection{Annotations}

An \emph{annotation} is a little extra drawing that can be added to a
symbol. For instance, when you add two little parallel arrows pointing
away from some electrical element, this usually means that the element
is light emitting.

Instead of having one symbol for ``diode'' and another for ``light
emitting diode,'' there is just one |diode| symbol, but you can add
the |light emitting| annotation to it. This is done by passing the
annotation as a parameter to the symbol as in the following example:

\begin{codeexample}[]
\tikz [circuit ee IEC]
  \draw (0,0) to [diode={light emitting}] (3,0)
              to [resistor={adjustable}]  (3,2);
\end{codeexample}



\subsection{The Base Circuit Library}

\begin{tikzlibrary}{circuits}
  This library is a base library that is included by other circuit
  libraries. You do not include it directly, but you will typically
  use some of the general keys, described below.
\end{tikzlibrary}

\begin{key}{/tikz/circuits}
  This key should be passed as an option to a picture or a scope that
  contains a circuit. It will do some internal setups. This key is
  normally called by more specialized keys like |circuit ee IEC|.
\end{key}




\subsubsection{Symbol Size}

\begin{key}{/tikz/circuit symbol unit=\meta{dimension} (initially 7pt)}
  This dimension is a ``unit'' for the size of symbols. The libraries
  generally define the sizes of symbols relative to this
  dimension. For instance, the longer side of an inductor is, by
  default, in the IEC library equal to five times this
  \meta{dimension}. When you change this \meta{dimension}, the size of
  all symbols will automatically change accordingly.

  Note, that it is still possible to overwrite the size of any
  particular symbol. These settings apply only to the default sizes.

\begin{codeexample}[]
\begin{tikzpicture}[circuit ee IEC]
  \draw (0,1) to [resistor] (3.5,1);
  \draw[circuit symbol unit=14pt]
        (0,0) to [resistor] (3.5,0);
\end{tikzpicture}
\end{codeexample}
\end{key}

\begin{stylekey}{/tikz/huge circuit symbols}
  This style sets the default circuit symbol unit to |10pt|.
\end{stylekey}
\begin{stylekey}{/tikz/large circuit symbols}
  This style sets the default circuit symbol unit to |8pt|.
\end{stylekey}
\begin{stylekey}{/tikz/medium circuit symbols}
  This style sets the default circuit symbol unit to |7pt|.
\end{stylekey}
\begin{stylekey}{/tikz/small circuit symbols}
  This style sets the default circuit symbol unit to |6pt|.
\end{stylekey}
\begin{stylekey}{/tikz/tiny circuit symbols}
  This style sets the default circuit symbol unit to |5pt|.
\end{stylekey}

\begin{key}{/tikz/circuit symbol size=|width| \meta{width} |height|
    \meta{height}}
  This key sets |minimum height| to \meta{height} times the current
  value of the circuit symbol unit and the |minimum width| to
  \meta{width} times this value. Thus, this option can be used with a
  node command to set the size of the node as a multiple of the
  circuit symbol unit.

\begin{codeexample}[]
\begin{tikzpicture}[circuit ee IEC]
  \draw (0,1) to [resistor] (2,1) to[inductor] (4,1);

  \begin{scope}
    [every resistor/.style={circuit symbol size=width 3 height 1}]
    \draw (0,0) to [resistor] (2,0) to[inductor] (4,0);
  \end{scope}
\end{tikzpicture}
\end{codeexample}
\end{key}



\subsubsection{Declaring New Symbols}


\begin{key}{/tikz/circuit declare symbol=\meta{name}}
  This key is used to declare a symbol. It does not cause this symbol
  to be shown nor does it set a graphic to be used for the symbol, it
  simply ``prepares'' several keys that can later be used to draw a
  symbol and to configure it.

  In detail, the first key that is defined is just called
  \meta{name}. This key should be given as an option to a |node| or on
  a |to| path, as explained below. The key will take options, which
  can be used to influence the way the symbol graphic is rendered.

  Let us have a look at an example. Suppose we want to define a symbol
  called |foo|, which just looks like a simple rectangle. We could
  then say
\begin{codeexample}[code only]
\tikzset{circuit declare symbol=foo}
\end{codeexample}
  The symbol could now be used like this:
\begin{codeexample}[code only]
  \node [foo]       at (1,1) {};
  \node [foo={red}] at (2,1) {};
\end{codeexample}

  However, in the above example we would not actually see anything
  since we have not yet set up the graphic to be used by |foo|. For
  this, we must use a key called |set foo graphic| or, generally,
  |set| \meta{name} |graphic|. This key gets graphic options as parameter
  that will be set when a symbol |foo| should be shown:
\begin{codeexample}[]
\begin{tikzpicture}
  [circuit declare symbol=foo,
   set foo graphic={draw,shape=rectangle,minimum size=5mm}]

  \node [foo]       at (1,1) {};
  \node [foo={red}] at (2,1) {};
\end{tikzpicture}
\end{codeexample}

  In detail, when you use the key \meta{name}=\meta{options} with a
  node, the following happens:
  \begin{enumerate}
  \item The |inner sep| is set to |0.5pt|.
  \item The following style is executed:
    \begin{stylekey}{/tikz/every circuit symbol}
      Use this style to set up things in general.
    \end{stylekey}
  \item The graphic options that have been set using
    |set| \meta{name} |graphic| are set.
  \item The style |every |\meta{name} is executed. You can use it to
    configure the symbol further.
  \item The \meta{options} are executed.
  \end{enumerate}

  The key \meta{name} will have a different effect when it is used on
  a |to| path command inside a |circuit| environment (the |circuit|
  environment sets up |to| paths in such a way that the use of a key
  declared using |circuit declare symbol| is automatically detected).
  When \meta{name} is used on a |to| path, the above actions also
  happen (setting the inner separation, using the symbol graphic, and
  so on), but they are passed to the key |circuit handle symbol|,
  which is explained next.
\end{key}


\begin{key}{/tikz/circuit handle symbol=\meta{options}}
  This key is mostly used internally. Its purpose is to render a
  symbol. The effect of this key differs, depending on whether it is
  used as the optional argument of a |to| path command or elsewhere.

  If the key is not used as an argument of a |to| path command, the
  \meta{options} are simply executed.

  The more interesting case happens when the key is given on a |to|
  path command. In this case, several things happen:
  \begin{enumerate}
  \item The |to| path is locally changed and set to an internal
    path (which you should not try to change) that consists mostly of
    a single straight line.
  \item The \meta{options} are tentatively executed with filtering
    switched on. Everything is filtered out, except for the key |pos|
    and also the styles |at start|, |very near start|, |near start|,
    |midway|, |near end|, |very near end|, and |at end|. If none of
    them is found, |midway| is used.
  \item The filtered option is used to determine a position for the
    symbol on the path. At the given position (with |pos=0|
    representing the start and |pos=1| representing the end), a node
    will be added to the path (in a manner to be described presently).
  \item This node gets \meta{options} as its option list.
  \item The node is added by virtue of a special |markings|
    decoration. This means that a |mark| command is executed that
    causes the node to be placed as a mark on the path.
  \item The marking decoration will automatically subdivide the path
    and cause a line to be drawn from the start of the path to the
    node's border (at the position that lies on a line from the node's
    center to the start of the path) and then from the node's border
    (at a position on the other side of the node) to the end of the
    path.
  \item The marking decoration will also take care of the case that
    multiple marks are present on a path, in this case the lines from
    and to the borders of the nodes are only between consecutive
    nodes.
  \item The marking decoration will also rotate the coordinate system
    in such a way that the $x$-axis points along the path. Thus, if
    you use the |transform shape| option, the node will ``point
    along'' the path.
  \item In case a node is at |pos=0| or at |pos=1| some special code
    will suppress the superfluous lines to the start or end of the
    path.
  \end{enumerate}

  The net effect of all of the above is that a node will be placed
  ``on the path'' and the path will have a ``gap'' just large enough
  to encompass the node. Another effect is that you can use this key
  multiple times on a path to add several node to a path, provided
  they do not overlap.
\begin{codeexample}[]
\begin{tikzpicture}[circuit]
  \draw (0,0) to [circuit handle symbol={draw,shape=rectangle,near start},
                  circuit handle symbol={draw,shape=circle,near end}] (3,2);
  \end{tikzpicture}
\end{codeexample}
\begin{codeexample}[]
\begin{tikzpicture}[transform shape,circuit]
  \draw (0,0) to [circuit handle symbol={draw,shape=rectangle,at start},
                  circuit handle symbol={draw,shape=circle,near end}] (3,2);
\end{tikzpicture}
\end{codeexample}
\end{key}



\subsubsection{Pointing Symbols in the Right Direction}

Unlike normal nodes, which generally should not be rotated since this
will make their text hard to read, symbols often need to be
rotated. There are two ways of achieving such rotations:

\begin{enumerate}
\item When you place a symbol on a |to| path, the graphic symbol is
  automatically rotated such that it ``points along the path.'' Here
  is an examples that shows how the inductor shape (which looks,
  unrotated, like this: \tikz[circuit ee IEC]\node[inductor]{};) is
  automatically rotated around:
\begin{codeexample}[]
\tikz [circuit ee IEC]
  \draw (3,0) to[inductor] (1,0) to[inductor] (0,2);
\end{codeexample}
\item Many shapes cannot be placed ``on'' a path in this way, namely
  whenever there are more than two possible inputs. Also, you may wish
  to place the nodes first, possibly using a matrix, and connect them
  afterwards. In this case, you can simply add rotations like
  |rotate=90| to the shapes to rotate them. The following four keys
  make this slightly more convenient:
  \begin{key}{/tikz/point up}
    This is the same as |rotate=90|.
\begin{codeexample}[]
\tikz [circuit ee IEC] \node [diode,point up] {};
\end{codeexample}
  \end{key}
  \begin{key}{/tikz/point down}
    This is the same as |rotate=-90|.
\begin{codeexample}[]
\tikz [circuit ee IEC] \node [diode,point down] {};
\end{codeexample}
  \end{key}
  \begin{key}{/tikz/point left}
    This is the same as |rotate=-180|.
\begin{codeexample}[]
\tikz [circuit ee IEC] \node [diode,point left] {};
\end{codeexample}
  \end{key}
  \begin{key}{/tikz/point right}
    This key has no effect.
\begin{codeexample}[]
\tikz [circuit ee IEC] \node [diode,point right] {};
\end{codeexample}
  \end{key}
\end{enumerate}


\subsubsection{Info Labels}

Info labels are used to add text to a circuit symbol. Unlike normal
nodes like a rectangle, circuit symbols typically do not have text
``on'' them, but the text is placed next to them (like the text
``$3\,\Omega$'' next to a resistor).

\tikzname\ already provides the |label| option for this purpose. The
|info| option is built on top of this option, but it comes in some
predefined variants that are especially useful in conjunction with
circuits.

\begin{key}{/tikz/info=\opt{|[|\meta{options}|]|\meta{angle}|:|}\meta{text}}
  This key has nearly the same effect as the |label| key, only the
  following style is used additionally automatically:
  \begin{stylekey}{/tikz/every info}
    Set this style to configure the styling of info labels. Since this
    key is \emph{not} used with normal labels, it provides an easy way
    of changing the way info labels look without changing other
    labels.
  \end{stylekey}
  The \meta{options} and \meta{angle} are passed directly to the
  |label| command.
\begin{codeexample}[]
\begin{tikzpicture}[circuit ee IEC,every info/.style=red]
  \node [resistor,info=$3\Omega$] {};
\end{tikzpicture}
\end{codeexample}

  You will find a detailed discussion of the |label| option on
  page~\pageref{label-option}.

  Hint: To place some text \emph{on} the main node, use |center| as
  the \meta{angle}:
\begin{codeexample}[]
\begin{tikzpicture}[circuit ee IEC,every info/.style=red]
  \node [resistor,info=center:$3\Omega$] {};
  \node [resistor,point up,info=center:$R_1$] at (2,0) {};
\end{tikzpicture}
\end{codeexample}
\end{key}

\begin{key}{/tikz/info'=\opt{|[|\meta{options}|]|\meta{angle}|:|}\meta{text}}
  This key works exactly like the |info| key, only in case the
  \meta{angle} is missing, it defaults to |below| instead of the
  current value of  |label position|, which is usually |above|. This
  means that when you use |info|, you get a label above the node,
  while when you use the |info'| key you get a label below the
  node. In case the node has been rotated, the positions of the info
  nodes are rotated accordingly.
\begin{codeexample}[]
\begin{tikzpicture}[circuit ee IEC,every info/.style=red]
  \draw (0,0) to[resistor={info={$3\Omega$},info'={$R_1$}}] (3,0)
              to[resistor={info={$4\Omega$},info'={$R_2$}}] (3,2);
\end{tikzpicture}
\end{codeexample}
\end{key}

\begin{key}{/tikz/info sloped=\opt{|[|\meta{options}|]|\meta{angle}|:|}\meta{text}}
  This key works like |info|, only the |transform shape| option is set
  when the label is drawn, causing it to follow the sloping of the
  main node.
\begin{codeexample}[]
\begin{tikzpicture}[circuit ee IEC,every info/.style=red]
  \draw (0,0) to[resistor={info sloped={$3\Omega$}}] (3,0)
              to[resistor={info sloped={$4\Omega$}}] (3,2);
\end{tikzpicture}
\end{codeexample}
\end{key}

\begin{key}{/tikz/info' sloped=}
  This is a combination of |info'| and |info sloped|.
\begin{codeexample}[]
\begin{tikzpicture}[circuit ee IEC,every info/.style=red]
  \draw (0,0) to[resistor={info' sloped={$3\Omega$}}] (3,0)
              to[resistor={info' sloped={$4\Omega$}}] (3,2);
\end{tikzpicture}
\end{codeexample}
\end{key}

\begin{key}{/tikz/circuit declare unit=\marg{name}\marg{unit}}
  This key is used to declare keys that make it easy to attach
  physical units to nodes. The idea is that instead of
  |info=$3\Omega$| you can write |ohm=3| or instead of
  |info'=$5\mathrm{S}$| you can write |siemens'=5|.

  In detail, four keys are defined, namely |/tikz/|\meta{name},
  |/tikz/|\meta{name}|'|, |/tikz/|\meta{name} |sloped|, and
  |/tikz/|\meta{name}|'| |sloped|. The arguments of all of these keys
  are of the form
  \opt{|[|\meta{options}|]|\meta{angle}|:|}\meta{value} and it is
  passed (slightly modified) to the corresponding key |info|, |info'|,
  |info| |sloped|, or |info'| |sloped|. The ``slight modification'' is the
  following: The text that is passed to the, say, |info| key is not
  \meta{value}, but rather |$\mathrm{|\meta{value}\meta{unit}|}$|

  This means that after you said |circuit declare unit={ohm}{\Omega}|,
  then |ohm=5k| will have the same effect as
  |info={[every ohm]$\mathrm{5k\Omega}$}|. Here, |every ohm| is a
  style that allows you to configure the appearance of this unit.
  Since the |info| key is used internally, by changing the
  |every info| style, you can change the appearance of all units
  infos.
\begin{codeexample}[]
\begin{tikzpicture}[circuit ee IEC,circuit declare unit={my ohm}{O}]
  \draw (0,0) to[resistor={my ohm' sloped=3}] (3,2);
\end{tikzpicture}
\end{codeexample}
\end{key}


\subsubsection{Declaring and Using Annotations}

Annotations are quite similar to info labels. The main difference is
that they generally cause something to be drawn by default rather than
some text to be added (although an annotation might also add some
text).

Annotations can be declared using the following key:

\begin{key}{/tikz/circuit declare annotation=\marg{name}\marg{distance}\marg{path}}
  This key is used to declare an annotation named \meta{name}. Once
  declared, it can be used as an argument of a symbol and will add the
  drawing in \meta{path} to the symbol. In detail, the following
  happens:

  \textbf{The Main Keys.}
  Two keys called \meta{name} and \meta{name}|'| are
  defined. The second causes the annotation to be ``mirrored and
  placed on the other side'' of the symbol. Both of these keys may
  also take further keys as parameter like |info| keys.
  Whenever the \meta{name} key is used, a local scope is opened and in
  this scope the following things are done:
  \begin{enumerate}
  \item The style |every| \meta{name} is executed.
  \item The following style is executed and then |arrows=->|:
    \begin{stylekey}{/tikz/annotation arrow}
      This style should set the |>| key to some desirable arrow tip.
    \end{stylekey}
  \item The coordinate system is shifted such that the origin is at
    the north anchor of the symbol. (For the \meta{name}|'| key the
    coordinate system is flipped and shifted such that the origin is
    at the south anchor of the symbol.)
  \item The |label distance| is locally set to \meta{distance}.
  \item The parameter options given to the \meta{name} key are
    executed.
  \item The \meta{path} is executed.
  \end{enumerate}

  \textbf{Usage.}
  What all of the above amounts to is best explained by an
  example. Suppose we wish to create an annotation that looks like a
  little circular arrow (like \tikz \draw [->] (0,0) arc
  (-270:80:1ex);). We could then say:
\begin{codeexample}[code only]
\tikzset{circuit declare annotation=
  {circular annotation}
  {9pt}
  {(0pt,8pt) arc (-270:80:3.5pt)}
}
\end{codeexample}
  We can then use it like this:
\tikzset{circuit declare annotation=
  {circular annotation}
  {8pt}
  {(0pt,8pt) arc (-270:80:3.5pt)}
}
\begin{codeexample}[]
\tikz[circuit ee IEC]
  \draw (0,0) to [resistor={circular annotation}]   (3,0);
\end{codeexample}
  Well, not very impressive since we do not see anything. This is due to
  the fact that the \meta{path} becomes part of a path that contains
  the symbol node an nothing else. This path is not drawn or filled,
  so we do not see anything. What we must do is to use an |edge| path
  operation:
\begin{codeexample}[]
\tikzset{circuit declare annotation={circular annotation}{9pt}
  {(0pt,8pt) edge[to path={arc(-270:80:3.5pt)}] ()}
}
\tikz[circuit ee IEC]
  \draw (0,0) to [resistor={circular annotation}]   (3,0)
              to [capacitor={circular annotation'}] (3,2);
\end{codeexample}
  The \meta{distance} is important for the correct placement of
  additional |info| labels. When an annotation is present, the info
  labels may need to be moved further away from the symbol, but not
  always. For this reason, an annotation defines an additional
  \meta{distance} that is applied to all info labels given
  as parameters to the annotation. Here is an example, that shows the
  difference:
\tikzset{circuit declare annotation={circular annotation}{9pt}
  {(0pt,8pt) edge[to path={arc (-270:80:3.5pt)}] ()}
}
\begin{codeexample}[]
\tikz[circuit ee IEC]
  \draw (0,0) to [resistor={circular annotation,ohm=5}]   (2,0)
              to [resistor={circular annotation={ohm=5}}] (4,0);
\end{codeexample}
\end{key}


\subsubsection{Theming Symbols}
\label{section-theming-symbols}
For each symbol, a certain graphical representation is chosen to
actually show the symbol. You can modify this graphical representation
in several ways:

\begin{itemize}
\item You can select a different library and use a different
  |circuit ...| key. This will change all graphics used for the
  symbols.
\item You can generally change the size of graphic symbols by setting
  |circuit size unit| to a different value or using a key like
  |small circuit symbols|.
\item
  You can add options to the graphics used by symbols either globally
  by setting the |every circuit| |symbol| style or locally by setting the
  |every| \meta{name} style, where \meta{name} is the name of a
  symbol. For instance, in the following picture the symbols are
  ridiculously thick and resistors are red.
\begin{codeexample}[]
\begin{tikzpicture}
  [circuit ee IEC,
   every circuit symbol/.style={ultra thick},
   every resistor/.style={red}]

  \draw (0,0) to [inductor] ++(right:3) to [resistor] ++(up:2);
\end{tikzpicture}
\end{codeexample}
\item You can selectively change the graphic used for a symbol
  by saying |set resistor graphic=|.
\item You can change one or more of the following styles:
  \begin{stylekey}{/tikz/circuit symbol open (initially draw)}
    This style is used with symbols that consist of lines that
    surround some area. For instance, the IEC version of a resistor is
    an open symbol.
\begin{codeexample}[]
\tikz [circuit ee IEC,
       circuit symbol open/.style={thick,draw,fill=yellow}]
  \draw (0,0) to [inductor] ++(right:3) to [resistor] ++(up:2);
\end{codeexample}
  \end{stylekey}
  \begin{stylekey}{/tikz/circuit symbol filled (initially {draw,fill=black})}
    This style is used with symbols that are completely filled. For
    instance, the variant IEC version of an inductor is a filled,
    black rectangle.
  \end{stylekey}
  \begin{stylekey}{/tikz/circuit symbol lines (initially draw)}
    This style is used with symbols that consist only of lines that do
    not surround anything. Examples are a capacitor.
\begin{codeexample}[]
\tikz [circuit ee IEC,
       circuit symbol lines/.style={thick,draw=red}]
  \draw (0,0) to [capacitor] ++(right:3) to [resistor] ++(up:2);
\end{codeexample}
  \end{stylekey}
  \begin{stylekey}{/tikz/circuit symbol wires (initially draw)}
    This style is used for symbols that consist only of ``wires.'' The
    difference to the previous style is that a symbol consisting of
    wires will look strange when the lines are thicker than the lines
    of normal wires, while for symbols consisting of lines (but not
    wires) it may look nice to make them thicker. An example is the
    |make contact| symbol.

    Compare
\begin{codeexample}[]
\tikz [circuit ee IEC,circuit symbol lines/.style={draw,very thick}]
  \draw (0,0) to [capacitor={near start},
                  make contact={near end}] (3,0);
\end{codeexample}
    to
\begin{codeexample}[]
\tikz [circuit ee IEC,circuit symbol wires/.style={draw,very thick}]
  \draw (0,0) to [capacitor={near start},
                  make contact={near end}] (3,0);
\end{codeexample}
  \end{stylekey}
\end{itemize}

All circuit environments like |circuit logic IEC| mainly use options
like |set and gate graphic=...| to set up the graphics used for a certain
symbol. It turns out that graphic hidden in the ``|...|'' part is also
always available as a separate style, whose name contains the
library's initials. For instance, the |circuit logic IEC| option
actually contains the following command:
\begin{codeexample}[code only]
  set and gate graphic = and gate IEC graphic,
\end{codeexample}
The |and gate IEC graphic| style, in turn, is defined as follows:
\begin{codeexample}[code only]
\tikzset{and gate IEC graphic/.style=
  {
    circuit symbol open,
    circuit symbol size=width 2.5 height 4,
    shape=and gate IEC,
    inner sep=.5ex
  }
}
\end{codeexample}

Normally, you do not need to worry about this, since you will not need
to access a style like |and gate IEC graphic| directly; you will only
use the |and gate| key. However, sometimes libraries define
\emph{variants} of a graphic; for instance, there are two variants for
the resistor graphic in the IEC library. In this case you can set the
graphic for the resistor to this variant (or back to the original) by
saying |set resistor graphic| yourself:

\begin{codeexample}[]
\begin{tikzpicture}[circuit ee IEC]
  % Standard resistor
  \draw (0,2) to [resistor] (3,2);

  % Var resistor
  \begin{scope}[set resistor graphic=var resistor IEC graphic]
    \draw (0,1) to [resistor] (3,1);

    % Back to original
    \draw [set resistor graphic=resistor IEC graphic]
      (0,0) to [resistor] (3,0);
  \end{scope}
\end{tikzpicture}
\end{codeexample}



\subsection{Logical Circuits}


\subsubsection{Overview}

A \emph{logical circuit} is a circuit that contains what we call
\emph{logical gates} like an |and gate| or an |xor gate|. The logical
libraries are intended to make it easy to draw such circuits.

In the following, we first have a look at the different libraries that
can be used in principle and how the symbols look like. Then we have a
more detailed look at how the symbols are used. Finally, we discuss
the implementation details.

There are different ways of depicting logical gates, which is why there
are different (sub-)libraries for drawing them. They provide the
necessary graphical representations of the symbols declared in the
following library:

\begin{tikzlibrary}{circuits.logic}
  This library declares the logical gate symbols, but does not
  provide the symbol graphics.
  The library also defines the following key which, however, is also
  only used indirectly, namely by other libraries:
  \begin{key}{/tikz/circuit logic}
    This style calls the keys |circuit| (which internally calls
    |every circuit|, then it defines the |inputs| key and it calls the
    |every circuit logic| key.
    \begin{key}{/tikz/inputs=\meta{inputs}}
      This key is defined only inside the scope of a
      |circuit logic|. There, it has the same effect as
      |logic gate inputs|, described on
      page~\pageref{logic-gate-inputs}.
    \end{key}
    \begin{stylekey}{/tikz/every circuit logic}
      Use this key to configure the appearance of logical circuits.
    \end{stylekey}
  \end{key}
\end{tikzlibrary}

Since the |circuit.logic| library does not define any actual graphics,
you need to use one of the following libraries, instead:

\begin{pgflibrary}{circuits.logic.IEC}
  This library provides graphics based on gates
  recommended by the International Electrotechnical Commission.  When
  you include this library, you can use the following key to
  set up a scope that contains a logical circuit where the gates are
  shown in this style.

  \begin{key}{/tikz/circuit logic IEC}
    This key calls |circuit logic| and installs the IEC-like
    graphics for the logical symbols like |and gate|.

    As explained in Section~\ref{section-theming-symbols}, for each
    graphic symbol of the library there is also a style that stores this
    particular appearance. These keys are called |and gate IEC graphic|,
    |or gate IEC graphic|, and so on.
\begin{codeexample}[]
\begin{tikzpicture}[circuit logic IEC,
                    every circuit symbol/.style={
                      logic gate IEC symbol color=black,
                      fill=blue!20,draw=blue,very thick}]
  \matrix[column sep=7mm]
  {
    \node (i0) {0}; &                            & \\
                    & \node [and gate] (a1) {};  & \\
    \node (i1) {0}; &                            & \node [or gate] (o) {};\\
                    & \node [nand gate] (a2) {}; & \\
    \node (i2) {1}; &                            & \\
  };
  \draw (i0.east) -- ++(right:3mm) |- (a1.input 1);
  \draw (i1.east) -- ++(right:3mm) |- (a1.input 2);
  \draw (i1.east) -- ++(right:3mm) |- (a2.input 1);
  \draw (i2.east) -- ++(right:3mm) |- (a2.input 2);
  \draw (a1.output) -- ++(right:3mm) |- (o.input 1);
  \draw (a2.output) -- ++(right:3mm) |- (o.input 2);
  \draw (o.output) -- ++(right:3mm);
\end{tikzpicture}
\end{codeexample}
  \end{key}
\end{pgflibrary}

\begin{pgflibrary}{circuits.logic.US}
  This library provides graphics showing ``American'' logic
  gates. It defines the following key:

  \begin{key}{/tikz/circuit logic US}
    This style calls |circuit logic| and installs US-like graphics
    for the logical symbols like |and gate|. For instance, it says
    \begin{codeexample}[code only]
set and gate graphic = and gate US graphic
    \end{codeexample}

    Here is an example:
\begin{codeexample}[]
\begin{tikzpicture}[circuit logic CDH,
                    tiny circuit symbols,
                    every circuit symbol/.style={
                      fill=white,draw}]
  \matrix[column sep=7mm]
  {
    \node (i0) {0}; &                            & \\
                    & \node [and gate] (a1) {};  & \\
    \node (i1) {0}; &                            & \node [or gate] (o) {};\\
                    & \node [nand gate] (a2) {}; & \\
    \node (i2) {1}; &                            & \\
  };
  \draw (i0.east) -- ++(right:3mm) |- (a1.input 1);
  \draw (i1.east) -- ++(right:3mm) |- (a1.input 2);
  \draw (i1.east) -- ++(right:3mm) |- (a2.input 1);
  \draw (i2.east) -- ++(right:3mm) |- (a2.input 2);
  \draw (a1.output) -- ++(right:3mm) |- (o.input 1);
  \draw (a2.output) -- ++(right:3mm) |- (o.input 2);
  \draw (o.output) -- ++(right:3mm);
\end{tikzpicture}
\end{codeexample}
  \end{key}
\end{pgflibrary}

\begin{pgflibrary}{circuits.logic.CDH}
  This library provides graphics based on the logic symbols used in
  A. Croft, R. Davidson, and M. Hargreaves (1992),  \emph{Engineering
    Mathematics}, Addison-Wesley, 82--95. They are identical to the
  US-style symbols, except for the and- and nand-gates.

  \begin{key}{/tikz/circuit logic CDH}
    This key calls |circuit logic US| and installs the two special
    and- and nand-gates, that is, it uses |set and gate graphic| with
    |and gate CDH graphic| and likewise for nand-gates.
  \end{key}
\end{pgflibrary}


Inside |circuit logic XYZ| scopes, you can now use the keys shown in
Section~\ref{section-logic-symbols}. We have a more detailed look at
one of them, all the other work the same way:

\begin{key}{/tikz/and gate}
  This key should be passed to a |node| command. It will cause the
  node to ``look like'' an |and gate|, where the exact appearance of the
  gate is dictated by the which circuit environment is used.   To
  further configure the appearance of the |and gate|, see
  Section~\ref{section-theming-symbols}.

\begin{codeexample}[]
\tikz [circuit logic IEC] \node [and gate] {$A$};
\end{codeexample}
\begin{codeexample}[]
\tikz [circuit logic US]
{
  \node [and gate,point down] {$A$};
  \node [and gate,point down,info=center:$A$] at (1,0) {};
}
\end{codeexample}

  \medskip\textbf{Inputs.}
  Multiple inputs can be specified for a logic gate (provided they
  support multiple inputs: a not gate---also known as an
  inverter---does not). However, there is an upper limit for the
  number of inputs which has been set to 1024, which should be \emph{way}
  more than would ever be needed.

  The following key is used to configure the inputs. It is available
  only inside a |circuit logic| environment.

  \begin{key}{/tikz/inputs=\meta{input list} (initially \char`\{normal,normal\char`\})}
    If a gate has $n$ inputs, the \meta{input list} should consists of
    $n$ letters, each being |i| for ``inverted'' or |n| for
    ``normal.'' Inverted gates will be indicated by a little
    circle. In any case the anchors for the inputs will be set
    up appropriately, numbered from top to bottom |input 1|, |input 2|,
    \ldots and so on. If the gate only supports one input the anchor
    is simply called |input| with no numerical index.
\begin{codeexample}[]
\begin{tikzpicture}[circuit logic IEC]
  \node[and gate,inputs={inini}] (A) {};
  \foreach \a in {1,...,5}
    \draw (A.input \a -| -1,0) -- (A.input \a);
  \draw (A.output) -- ++(right:5mm);
\end{tikzpicture}
\end{codeexample}
  \end{key}

  (This key is just a shorthand for |logic gate inputs|, described
  in detail on page~\pageref{logic-gate-inputs}. There you will also
  find descriptions of how to configure the size of the inverted
  circles and the way the symbol size increases when there are too
  many inputs.)

  \textbf{Output.}
  Every logic gate has one anchor called |output|.
\end{key}


\subsubsection{Symbols: The Gates}

\label{section-logic-symbols}

The following table shows which symbols are declared by the main
|circuits.logic| library and their appearance in the different
sublibraries.
\medskip

\def\gateexamples#1{%
  \texttt{#1}
  \indexkey{#1} &
  \tikz[baseline,circuit logic IEC] \node[#1,label=] {}; &
  \tikz[baseline,circuit logic US]  \node[#1] {}; &
  \tikz[baseline,circuit logic CDH] \node[#1] {};
}
\begin{tabular}{lccc}
  \emph{Key} & \emph{Appearance inside} & \emph{Appearance inside} & \emph{Appearance inside} \\
      & |circuit logic IEC| & |circuit logic US| & |circuit logic CDH| \\
  \gateexamples{/tikz/and gate}\\
  \gateexamples{/tikz/nand gate}\\
  \gateexamples{/tikz/or gate}\\
  \gateexamples{/tikz/nor gate}\\
  \gateexamples{/tikz/xor gate}\\
  \gateexamples{/tikz/xnor gate}\\
  \gateexamples{/tikz/not gate}\\
  \gateexamples{/tikz/buffer gate}
\end{tabular}


\subsubsection{Implementation: The Logic Gates Shape Library}

The previous sections described the \tikzname\ interface for creating
logical circuits. In this section we take a closer look at the
underlying \pgfname\ libraries.

Just as there are several \tikzname\ circuit libraries, there are two
underlying \pgfname\ shape libraries, one for creating US-style gates
and one for IEC-style gates. These libraries define \emph{shapes}
only. It is the job of the circuit libraries to ``theme'' them so that
they ``look nice.'' However, in principle, you can also use these
shapes directly.

Let us begin with the base library that defines the handling of
inputs.

\begin{pgflibrary}{shapes.gates.logic}
  This library defines common keys used by all logical gate shapes.

\begin{key}{/pgf/logic gate inputs=\meta{input list} (initially \char`\{normal,normal\char`\})}
  \label{logic-gate-inputs}%
  Specify the inputs for the logic gate. The keyword |inverted|
  indicates an inverted input which will mean \pgfname{} will draw a
  circle attached to the main shape of the logic gate. Any keyword
  that is not |inverted| will be treated as a ``normal'' or
  ``non-inverted'' input (however, for readability, you may wish to
  use |normal| or |non-inverted|), and \pgfname{} will not draw the
  circle.
  In both cases the anchors for the inputs will be set
  up appropriately, numbered from top to bottom |input 1|, |input 2|,
  \ldots and so on. If the gate only supports one input the anchor
  is simply called |input| with no numerical index.

\begin{codeexample}[]
\begin{tikzpicture}[minimum height=0.75cm]
  \node[and gate IEC, draw, logic gate inputs={inverted, normal, inverted}]
    (A) {};
  \foreach \a in {1,...,3}
    \draw (A.input \a -| -1,0) -- (A.input \a);
  \draw (A.output) -- ([xshift=0.5cm]A.output);
\end{tikzpicture}
\end{codeexample}

  For multiple inputs it may be somewhat unwieldy to specify a long
  list, thus, the following ``shorthand'' is permitted (this is an
  extension of ideas due to Juergen Werber and Christoph Bartoschek):
  Using |i| for inverted and |n| for normal inputs, \meta{input list}
  can be specified \emph{without the commas}. So, for example,
  |ini| is equivalent to |inverted, normal, inverted|.

\begin{codeexample}[]
\begin{tikzpicture}[minimum height=0.75cm]
  \node[or gate US, draw,logic gate inputs=inini] (A) {};
  \foreach \a in {1,...,5}
    \draw (A.input \a -| -1,0) -- (A.input \a);
  \draw (A.output) -- ([xshift=0.5cm]A.output);
\end{tikzpicture}
\end{codeexample}

\end{key}

The height of the gate may be increased to accommodate the number
of inputs. In fact, it depends on three variables:
$n$, the number of inputs, $r$, the radius of the circle used
to indicate an inverted input and $s$, the distance between
the centers of the inputs.
The default height is then calculated according to the expression
$(n+1)\times\max(2r,s)$. This then may
be increased to accommodate the node contents or any
minimum size specifications.

The radius of the inverted input circle and the distance between the
centers of the inputs can be customized using the following keys:

\begin{key}{/pgf/logic gate inverted radius=\meta{length} (initially 2pt)}
  Set the radius of the circle that is used to indicate inverted
  inputs. This is also the radius of the circle used for the inverted
  output of the |nand|, |nor|, |xnor| and |not| gates.

\begin{codeexample}[]
\begin{tikzpicture}[minimum height=0.75cm]
  \tikzset{every node/.style={shape=nand gate CDH, draw, logic gate inputs=ii}}
  \node[logic gate inverted radius=2pt] {A};
  \node[logic gate inverted radius=4pt] at (0,-1) {B};
\end{tikzpicture}
\end{codeexample}
\end{key}

\begin{key}{/pgf/logic gate input sep=\meta{length} (initially .125cm)}
  Set the distance between the \emph{centers} of the inputs to the
  logic gate.

\begin{codeexample}[]
\begin{tikzpicture}[minimum size=0.75cm]
  \draw [help lines] grid (3,2);
  \tikzset{every node/.style={shape=and gate IEC, draw, logic gate inputs=ini}}
  \node[logic gate input sep=0.33333cm] at (1,1)(A) {A};
  \node[logic gate input sep=0.5cm]     at (3,1) (B) {B};
  \foreach \a in {1,...,3}
    \draw (A.input \a -| 0,0) -- (A.input \a)
          (B.input \a -| 2,0) -- (B.input \a);
\end{tikzpicture}
\end{codeexample}
\end{key}

  \pgfname{} will increase the size of the
  logic gate to accommodate the number of inputs, and the size
  of the inverted radius and the separation between the inputs.
  However with all shapes in this library, any increase in size
  (including any minimum size requirements) will be applied so that
  the default aspect ratio is unaltered. This means that changing
  the height will change the width and vice versa.

\end{pgflibrary}


\subsubsection{Implementation: The US-Style Logic Gates Shape Library}

\begin{pgflibrary}{shapes.gates.logic.US}
  This library provides ``American'' logic gate shapes whose names are
  suffixed with the identifier |US|. Additionally,
  alternative |and| and |nand| gates are provided which are based on the
  logic symbols used in A. Croft, R. Davidson, and M. Hargreaves (1992),
  \emph{Engineering Mathematics}, Addison-Wesley, 82--95. These two
  shapes are suffixed with |CDH|.

  The ``compass point'' anchors apply to the main part of the shape
  and do not include any inverted inputs or outputs. This library
  provides an additional feature to facilitate the relative positioning
  of logic gates:

\begin{key}{/pgf/logic gate anchors use bounding box=\meta{boolean} (initially false)}
  When set to |true| this key will ensure that the
  compass point anchors use the bounding rectangle of the
  main shape, which, ignore any inverted inputs or outputs, but
  includes any |outer sep|.
  This \emph{only} affects the compass point anchors
  and is not set on a shape by shape basis: whether the bounding
  box is used is determined by value of this key when the anchor
  is accessed.

\begin{codeexample}[]
\begin{tikzpicture}[minimum height=1.5cm]
  \node[xnor gate US, draw, gray!50,line width=2pt] (A) {};
  \foreach \x/\y/\z in {false/blue/1pt, true/red/2pt}
    \foreach \a in {north, south, east, west, north east,
      south east, north west, south west}
      \draw[logic gate anchors use bounding box=\x, color=\y]	
        (A.\a) circle(\z);
\end{tikzpicture}
\end{codeexample}

\end{key}

  The library defines a number of shapes. For each shape the allowed
  number of inputs is also shown:
  \begin{itemize}
  \item |and gate US|, two or more inputs
  \item |and gate CDH|, two or more inputs
  \item |nand gate US|, two or more inputs
  \item |nand gate CDH|, two or more inputs
  \item |or gate US|, two or more inputs
  \item |nor gate US|, two or more Inputs
  \item |xor gate US|, two inputs
  \item |xnor gate US|, two inputs
  \item |not gate US|, one input
  \item |buffer gate US|, one input
  \end{itemize}

  In the following, we only have a detailed look at the anchors
  defined by one of them. We choose the |nand gate US| because it
  shows all the ``interesting'' anchors.

  \begin{shape}{nand gate US}
    This shape is a nand gate, which supports two or more inputs. If
    less than two inputs are specified an error will result.
    The anchors for this gate with two
    non-inverted inputs (using the normal compass point anchors) are
    shown below. Anchor |30| is an example of a border anchor.

\begin{codeexample}[]
\Huge
\begin{tikzpicture}
  \node[name=s,shape=nand gate US,shape example, inner sep=0cm,
  logic gate inputs={in},
  logic gate inverted radius=.5cm] {Nand Gate\vrule width1pt height2cm};
  \foreach \anchor/\placement in
    {center/above, text/above, 30/above right,
     mid/right, mid east/left, mid west/above,
     base/below, base east/below, base west/left,
     north/above, south/below, east/above, west/above,
     north east/above, south east/below, south west/below, north west/above,
     output/right, input 1/above, input 2/below}
     \draw[shift=(s.\anchor)] plot[mark=x] coordinates{(0,0)}
       node[\placement] {\scriptsize\texttt{(s.\anchor)}};
\end{tikzpicture}
\end{codeexample}

    (For the definition of the |shape example| style, see
    Section~\ref{section-libs-shapes}.) 
  \end{shape}
\end{pgflibrary}



\subsubsection{Implementation: The IEC-Style Logic Gates Shape Library}

\begin{pgflibrary}{shapes.gates.logic.IEC}
  This library provides rectangular logic gate shapes. These shapes
  are suffixed with |IEC| as they are based on gates recommended by
  the International Electrotechnical Commission.

  By default each gate is drawn with a symbol, $\char`\&$ for |and| and
  |nand| gates, $\geq1$ for |or| and |nor| gates, $1$ for |not| and
  |buffer| gates, and $=1$ for |xor| and |xnor| gates. These symbols
  are drawn automatically (internally they are drawn using the
  ``foreground'' path), and are not strictly speaking part of the node
  contents. However, the gate is enlarged to make sure the symbols are
  within the border of the node.
  It is possible to change
  the symbols and their position within the node using the following
  keys:

\begin{key}{/pgf/and gate IEC symbol=\meta{text} (initially \char`\\char\char`\`\char`\\\char`\&)}
  Set the symbol for the |and gate|. Note that if the node is filled,
  this color will be used for the symbol, making it invisible, so
  it will be necessary set \meta{text} to something like
  |\color{black}\char`\&|. Alternatively, the
  |logic gate IEC symbol color| key can be used to set the color
  of all symbols simultaneously.

  In \tikzname, when the |use IEC style logic gates| key has been
  used, this key can be replaced by |and gate symbol|.
\end{key}

\begin{key}{/pgf/nand gate IEC symbol=\meta{text} (initially \char`\\char\char`\`\char`\\\char`\&)}
  Set the symbol for the |nand gate|.
  In \tikzname, when the |use IEC style logic gates| key has been
  used, this key can be replaced by |nand gate symbol|.
\end{key}

\begin{key}{/pgf/or gate IEC symbol=\meta{text} (initially \char`\$\char`\\geq1\char`\$)}
  Set the symbol for the |or gate|.
  In \tikzname, when the |use IEC style logic gates| key has been
  used, this key can be replaced by |or gate symbol|.
\end{key}

\begin{key}{/pgf/nor gate IEC symbol=\meta{text} (initially \char`\$\char`\\geq1\char`\$)}
  Set the symbol for the |nor gate|.
  In \tikzname, when the |use IEC style logic gates| key has been
  used, this key can be replaced by |nor gate symbol|.
\end{key}

\begin{key}{/pgf/xor gate IEC symbol=\meta{text} (initially \char`\{\char`\$=1\char`\$\char`\})}
  Set the symbol for the |xor gate|. Note the necessity for braces,
  as the symbol contains |=|.
  In \tikzname, when the |use IEC style logic gates| key has been
  used, this key can be replaced by |xor gate symbol|.
\end{key}

\begin{key}{/pgf/xnor gate IEC symbol=\meta{text} (initially  \char`\{\char`\$=1\char`\$\char`\})}
  Set the symbol for the |xnor gate|.
  In \tikzname, when the |use IEC style logic gates| key has been
  used, this key can be replaced by |xnor gate symbol|.
\end{key}

\begin{key}{/pgf/not gate IEC symbol=\meta{text} (initially 1)}
  Set the symbol for the |not gate|.
  In \tikzname, when the |use IEC style logic gates| key has been
  used, this key can be replaced by |not gate symbol|.
\end{key}

\begin{key}{/pgf/buffer gate IEC symbol=\meta{text} (initially 1)}
  Set the symbol for the |buffer gate|.
  In \tikzname, when the |use IEC style logic gates| key has been
  used, this key can be replaced by |buffer gate symbol|.
\end{key}

\begin{key}{/pgf/logic gate IEC symbol align=\meta{align} (initially top)}
  Set the alignment of the logic gate symbol (in \tikzname, when the
  |use IEC style logic gates| key has been used, |IEC| can be omitted.
  The specification in \meta{align} is a comma separated list from
  |top|, |bottom|, |left| or |right|. The distance between the border
  of the node and the outer edge of the symbol is determined by the values
  of the |inner xsep| and |inner ysep|.

\begin{codeexample}[]
\begin{tikzpicture}[minimum size=1cm, use IEC style logic gates]
	\tikzset{every node/.style={nor gate, draw}}
  \node (A) at (0,1.5) {};
  \node [logic gate symbol align={bottom, right}] (B) at (0,0) {};
  \foreach \g in {A, B}{
    \foreach \i in {1,2}
      \draw ([xshift=-0.5cm]\g.input \i) -- (\g.input \i);
    \draw (\g.output) -- ([xshift=0.5cm]\g.output);
  }
\end{tikzpicture}
\end{codeexample}

\end{key}


\begin{key}{/pgf/logic gate IEC symbol color=\meta{color}}
  This key sets the color for all symbols simultaneously. This color
  can be overridden on a case by case basis by specifying a color
  when setting the symbol text.
\end{key}

  The library defines the following shapes:
  \begin{itemize}
  \item |and gate IEC|, two or more inputs
  \item |nand gate IEC|, two or more inputs
  \item |or gate IEC|, two or more inputs
  \item |nor gate IEC|, two or more inputs
  \item |xor gate IEC|, two inputs
  \item |xnor gate IEC|, two inputs
  \item |not gate IEC|, one input
  \item |buffer gate IEC|, one input
  \end{itemize}

  Again, we only have a look at the nand-gate in more detail:

\begin{shape}{nand gate IEC}
  This shape is a nand gate. It supports two or more inputs.
  If less than two inputs are specified an error will result.
	The anchors for this gate with two
  inverted inputs are
  shown below. Anchor |30| is an example of a border anchor.

\begin{codeexample}[]
\Huge
\begin{tikzpicture}
  \node[name=s,shape=nand gate IEC ,shape example, inner xsep=1cm, inner ysep=1cm,
    minimum height=6cm, nand gate IEC symbol=\color{black!30}\char`\&,
    logic gate inputs={in},
    logic gate inverted radius=0.65cm]
  {Nand Gate\vrule width1pt height2cm};
  \foreach \anchor/\placement in
    {center/above, text/above, 30/above right,
     mid/right, mid east/left, mid west/above,
     base/below, base east/below, base west/left,
     north/above, south/below, east/above, west/above,
     north east/above, south east/below, south west/below, north west/above,
     output/right, input 1/above, input 2/below}
     \draw[shift=(s.\anchor)] plot[mark=x] coordinates{(0,0)}
       node[\placement] {\scriptsize\texttt{(s.\anchor)}};
\end{tikzpicture}
\end{codeexample}
\end{shape}
\end{pgflibrary}




\subsection{Electrical Engineering Circuits}

\subsubsection{Overview}

An \emph{electrical engineering circuit} contains symbols like
resistors or capacitors or voltage sources and annotations like the
two arrows pointing toward an element whose behaviour is light
dependent. The electrical engineering libraries, abbreviated
ee-libraries, provide such symbols and annotations.

Just as for logical gates, there are different ways of drawing
ee-symbols. Currently, there is one main library for drawing circuits,
which uses the graphics from the International Electrotechnical
Commission, but you can add your own libs. This is why, just as for
logical gates, there is a base library and more specific libraries.

\begin{tikzlibrary}{circuits.ee}
  This library declares the ee symbols, but (mostly) does not
  provide the symbol graphics, which is left to the sublibraries.
  Just like the logical gates library, a key is defined that is
  normally only used internally:
  \begin{key}{/tikz/circuit ee}
    This style calls the keys |circuit| (which internally calls
    |every circuit| and the following style:
    \begin{stylekey}{/tikz/every circuit ee}
      Use this key to configure the appearance of logical circuits.
    \end{stylekey}
  \end{key}

  The library also declares some standard annotations and units.
\end{tikzlibrary}

As for logical circuits, to draw a circuit the first step is to
include a library containing the symbols graphics. Currently, you have to
include |circuits.ee.IEC|.

\begin{tikzlibrary}{circuit.ee.IEC}
  When this library is loaded, you can use the following style:
  \begin{key}{/tikz/circuit ee IEC}
    This style calls |circuit ee| and installs the IEC-like
    graphics for the logical symbols like |resistor|.
  \end{key}
\end{tikzlibrary}

Inside the |circuit ee IEC| scope, you can now use the keys for
symbols, units, and annotations listed in the later sections. We have
a more detailed look at one of each of them, all the others work the
same way.

Let us start with an example of a symbol: the resistor symbol. The
other predefined symbols are listed in
Section~\ref{section-circuits-ee-symbols} and later sections.

\begin{key}{/tikz/resistor=\opt{\meta{options}}}
  This key should be used with a |node| path command or with the |to|
  path command.

  \medskip\textbf{Using the Key with Normal Nodes.}
  When used with a node, it will cause this node to
  ``look like'' a resistor (by default, in the IEC library, this is
  just a simple rectangle).
\begin{codeexample}[]
\tikz [circuit ee IEC]
  \node [resistor] {};
\end{codeexample}

  Unlike normal nodes, a resistor node generally should not take any
  text (as in |node [resistor] {foo}|). Instead, the labeling of
  resistors should be done using the |label|, |info| and |ohm|
  options.
\begin{codeexample}[]
\tikz [circuit ee IEC]
  \node [resistor,ohm=5] {};
\end{codeexample}

  The \meta{options} make no real sense when the |resistor| option is
  used with a normal node, you can just as well given them to the
  |node| itself. Thus, the following has the same effect as the above
  example:
\begin{codeexample}[]
\tikz [circuit ee IEC]
  \node [resistor={ohm=5}] {};
\end{codeexample}

  In a circuit, you will often wish to rotate elements. For this, the
  options |point up|, |point down|, |point left| or |point right| may
  be especially useful. They are just shorthands for appropriate
  rotations like |rotate=90|.
\begin{codeexample}[]
\tikz [circuit ee IEC] {
  \node (R1) [resistor,point up,ohm=5] at (3,1) {};
  \node (R2) [resistor,ohm=10k]        at (0,0) {};
  \draw (R2) -| (R1);
}
\end{codeexample}

  \medskip\textbf{Using the Key on a To Path.}
  When the |resistor| key is used on a |to| path inside a
  |circuit ee IEC|, the |circuit handle symbol| key is called
  internally. This has a whole bunch of effects:
  \begin{enumerate}
  \item The path currently being constructed is cut up to make place
    for a node.
  \item This node will be a |resistor node| that is rotated so that it
    points ``along'' the path (unless an option like |shift only| or
    an extra rotation is used to change this).
  \item The \meta{options} passed to the |resistor| key are passed on
    to the node.
  \item The \meta{options} are pre-parsed to identify a |pos| key or
    a key like |at start| or |midway|. These keys are used to
    determine where on the |to| path the node will lie.
  \end{enumerate}

  Since the \meta{options} of the |resistor| key are passed on to the
  resistor node on the path, you can use it to add labels to the
  node. Here is a simple example:

\begin{codeexample}[]
\tikz [circuit ee IEC]
  \draw (0,0) to [resistor=red]        (3,0)
              to [resistor={ohm=2\mu}] (3,2);
\end{codeexample}

  You can add multiple labels to a resistor and you can have multiple
  resistors (or other elements) on a single path.

  \medskip\textbf{Inputs, Outputs, and Anchors.}
  Like the logical gates, all ee-symbols have an |input|
  and an |output| anchor. Special-purpose-nodes may have even more
  anchors of this type. Furthermore, the ee-symbols-nodes also have four
  standard compass direction anchors.

  \medskip\textbf{Changing the Appearance.}
  To configure the appearance of all |resistor|s, see
  Section~\ref{section-theming-symbols}. You can use the
  \meta{options} to locally change the appearance of a single
  resistor.
\end{key}

Let us now have a look at an example of a unit: the Ohm unit. The
other predefined units are listed in Section~\ref{section-circuits-units}.

\begin{key}{/tikz/ohm=\meta{value}}
  This key is used to add an |info| label to a node with a special
  text: |$\mathrm{|\meta{value}|\Omega}$|. In other words, the |ohm|
  key can only be used with the options of a node and, when used, it
  will cause the \meta{value} to be placed next to the node, followed
  by $\Omega$. Since the \meta{value} is typeset inside a |\mathrm|
  command, when you write |ohm=5k| you get $\mathrm{5k\Omega}$,
  |ohm=5p| yields $\mathrm{5p\Omega}$, and |ohm=5.6\cdot 10^{2}\mu|
  yields $\mathrm{5.6\cdot 10^{2}\mu\Omega}$.
\begin{codeexample}[]
\tikz [circuit ee IEC] \draw (0,0) to [resistor={ohm=5M}] (0,2);
\end{codeexample}

  Instead of |ohm| you can also use |ohm'|, which places the label on
  the other side.
\begin{codeexample}[]
\tikz [circuit ee IEC] \draw (0,0) to [resistor={ohm'=5M}] (0,2);
\end{codeexample}

  Finally, there are also keys |ohm sloped| and |ohm' sloped| for
  having the info label rotate together with the main node.
\begin{codeexample}[]
\tikz [circuit ee IEC]
  \draw (0,0) to [resistor={ohm sloped=5M}] (0,2)
        (2,0) to [resistor={ohm' sloped=6f}]  (2,2);
\end{codeexample}

  You can configure the appearance of an Ohm info label using the key
  |every ohm|.
\end{key}

Finally, let us have a look at an annotation: the |light emitting|
annotation. The other predefined units are listed in
Section~\ref{section-circuits-annotations}.

\begin{key}{/tikz/light emitting=\opt{\meta{options}}}
  Like a unit, an annotation should be given as an additional option
  to a node. It causes some drawings (in this case, two parallel
  lines) to be placed next to the node.

\begin{codeexample}[]
\tikz [circuit ee IEC] \draw (0,0) to [diode=light emitting] (2,0);
\end{codeexample}

  The \meta{options} can be used for three different things:
  \begin{enumerate}
  \item You can use keys like |red| to change the appearance of this
    annotation, locally.
  \item You can use keys like |<-|  or |-latex| to change the
    direction and kinds of arrows used in the annotation.
  \item You can use info labels like |ohm=5| or |info=foo| inside the
    \meta{options}. These info labels will be added to the main node
    (not to the annotation itself), but the label distance will have
    been changed to accommodate for the space taken up by the
    annotation.
\begin{codeexample}[]
\tikz [circuit ee IEC]
{
  \draw (0,2) to [diode={light emitting,info=not good}] (2,2);
  \draw (0,0) to [diode={light emitting={info=better},
                         info'=also good}]  (2,0);
}
\end{codeexample}
  \end{enumerate}

  In addition to |light emitting| there is also a key called
  |light emitting'|, which simply places the annotation on the other
  side of the node.

  You can configure the appearance of annotations in three ways:
  \begin{itemize}
  \item You can set the |every circuit annotation| style.
  \item You can set the |every light emitting| style.
  \item You can set the following key:
    \begin{stylekey}{/tikz/annotation arrow}
      This style should set the default |>| arrow to some nice value.
    \end{stylekey}
  \end{itemize}
\end{key}





\def\eelineexample#1#2{%
  \texttt{#1}\indexkey{#1}
   &
  \tikz[baseline=-.5ex,circuit ee IEC] \draw (0,0) to [#1] (3,0);
  &
  \relax\def\temp{#2}
  \ifx\temp\empty\else
  {\tikz[baseline=-.5ex,circuit ee IEC,set #2 graphic=var #2 IEC graphic]
    \draw (0,0) to [#2] (3,0);}
  \fi \\[.2em]
}
\def\eeendexample#1#2{%
  \texttt{#1}\indexkey{#1}
   &
  \tikz[baseline=-.5ex,circuit ee IEC] \draw (0,0) to [#1={at end}] (1.5,0)(3,0);
  &
  \relax\def\temp{#2}
  \ifx\temp\empty\else
  {\tikz[baseline=-.5ex,circuit ee IEC,set #2 graphic=var #2 IEC graphic]
    \draw (0,0) to [#2={at end}] (1.5,0)(3,0);}
  \fi \\[.2em]
}
\def\unitexample#1{%
  \texttt{#1}\indexkey{#1}
  &
  \tikz [baseline,inner sep=0pt] \node[#1=1] {};\\
}
\def\annotationexample#1{%
  \texttt{#1}\indexkey{#1}
  &
  \tikz[baseline=-.5ex,circuit ee IEC]
    \draw (0,0) to [resistor={#1}] (2,0)
                to [diode   ={#1'}] (4,0);\\
}
\def\empty{}


\subsubsection{Symbols: Indicating Current Directions}

\label{section-ee-symbols}
\label{section-circuits-ee-symbols}

There are two symbols for indicating current directions. These symbols
are defined directly inside |circuit ee|.
\medskip

\noindent
\begin{tabular}{p{5cm}ll}
  \emph{Key} & \emph{Appearance}\\[.25em]
  \eelineexample{/tikz/current direction}{}
  \eelineexample{/tikz/current direction'}{}
\end{tabular}

\medskip
The examples have been produced by (in essence)
|\draw (0,0) to[|\meta{symbol name}|] (3,0);|.


\subsubsection{Symbols: Basic Elements}

The following table show basic symbols as they are depicted inside the
|circuit ee IEC| environment. To install one of alternate graphics,
you have to say |set| \meta{symbol name} |graphic=var| \meta{symbol name}
|IEC graphic|.
\medskip

\noindent
\begin{tabular}{p{5cm}ll}
  \emph{Key} & \emph{Appearance}  & \emph{Alternate appearance} \\[.25em]
  \eelineexample{/tikz/resistor}{resistor}
  \eelineexample{/tikz/inductor}{inductor}
  \eelineexample{/tikz/capacitor}{}
  \eelineexample{/tikz/battery}{}
  \eelineexample{/tikz/bulb}{}
  \eelineexample{/tikz/current source}{}
  \eelineexample{/tikz/voltage source}{}
  \eeendexample{/tikz/ground}{}
\end{tabular}


\subsubsection{Symbols: Diodes}

The following table shows diodes as they are depicted inside the
|circuit ee IEC| environment.
\medskip

\noindent
\begin{tabular}{p{5cm}ll}
  \emph{Key} & \emph{Appearance}  & \emph{Alternate appearance} \\[.25em]
  \eelineexample{/tikz/diode}{diode}
  \eelineexample{/tikz/Zener diode}{Zener diode}
  \eelineexample{/tikz/Schottky diode}{Schottky diode}
  \eelineexample{/tikz/tunnel diode}{tunnel diode}
  \eelineexample{/tikz/backward diode}{backward diode}
  \eelineexample{/tikz/breakdown diode}{breakdown diode}
\end{tabular}


\subsubsection{Symbols: Contacts}

The following table shows contacts as they are depicted inside the
|circuit ee IEC| environment.
\medskip

\noindent
\begin{tabular}{p{5cm}ll}
  \emph{Key} & \emph{Appearance}  & \emph{Alternate appearance} \\[.25em]
  \eelineexample{/tikz/contact}{}
  \eelineexample{/tikz/make contact}{make contact}
  \eelineexample{/tikz/break contact}{}
\end{tabular}


\subsubsection{Units}

\label{section-circuits-units}

The |circuit.ee| library predefines the following unit keys:
\medskip

\noindent
\begin{tabular}{p{5cm}c}
  \emph{Key} & \emph{Appearance of $1$ unit} \\[.25em]
  \unitexample{/tikz/ampere}
  \unitexample{/tikz/volt}
  \unitexample{/tikz/ohm}
  \unitexample{/tikz/siemens}
  \unitexample{/tikz/henry}
  \unitexample{/tikz/farad}
  \unitexample{/tikz/coulomb}
  \unitexample{/tikz/voltampere}
  \unitexample{/tikz/watt}
  \unitexample{/tikz/hertz}
\end{tabular}



\subsubsection{Annotations}

\label{section-circuits-annotations}

The |circuit.ee.IEC| library defines the following annotations:
\medskip

\noindent
\begin{tabular}{p{5cm}ll}
  \emph{Key} & \emph{Appearance} \\[.25em]
  \annotationexample{/tikz/light emitting}
  \annotationexample{/tikz/light dependent}
  \annotationexample{/tikz/direction info}
  \annotationexample{/tikz/adjustable}
\end{tabular}
\medskip

The lines have been produced using, in essence,
\begin{codeexample}[code only]
\draw (0,0)  to [resistor=light emitting] (2,0)  to [diode=light emitting'] (4,0);
\end{codeexample}
and similarly for the other annotations.


\subsubsection{Implementation: The EE-Symbols Shape Library}

The \tikzname\ libraries depend on two shape libraries, which are
included automatically. Usually, you will not need to use these shapes
directly.

\begin{pgflibrary}{shapes.gates.ee}
  This library defines basic shapes that can be used by all ee-circuit
  libraries. Currently, it defines the following shapes:
  \begin{itemize}
  \item |rectangle ee|
  \item |circle ee|
  \item |direction ee|
  \end{itemize}
  Additionally, the library defines the following arrow tip:
  The |direction ee| arrow tip is basically the same as a |triangle 45|
  arrow tip with rounded joins.

  \begin{tabular}{ll}
    \symarrow{direction ee}
  \end{tabular}

  However, unlike normal arrow tips, its size does \emph{not} depend on
  the current line width. Rather, it depends on the value of its
  arrow options, which should be set to the desired size. Thus, you
  should say something like |\pgfsetarrowoptions{direction ee}{5pt}| to
  set the size of the arrow.
\end{pgflibrary}

\begin{shape}{rectangle ee}
  This shape is completely identical to a normal |rectangle|, only
  there are two additional anchors: The |input| anchor is an alias for
  the |west| anchor, while the |output| anchor is an alias for the
  |east| anchor.
\end{shape}

\begin{shape}{circle ee}
  Like the |rectangle ee| shape, only for circles.
\end{shape}

\begin{shape}{direction ee}
  This shape is rather special. It is intended to be used to ``turn an
  arrow tip into a shape.'' First, you should set the following key to
  the name of an arrow tip:
  \begin{key}{/pgf/direction ee arrow=\meta{right arrow tip name}}
    The value of this key will be used for the arrow tip depicted in
    an |direction ee| shape.
  \end{key}
  When a node of shape |direction ee| is created, several things
  happen:
  \begin{enumerate}
  \item The size of the shape is computed according to the following
    rules: The width of the shape is set up so that the left border of
    the shape is at the left end of the arrow tip and the right border
    is at the right end of the arrow tip. These left and right
    ``ends'' of the arrow are the tip end and the back end specified
    by the arrow itself (see Section~\ref{section-arrow-terminology}
    for details). You usually need not worry about this width
    setting.

    By comparison, the height of the arrow is given by the current
    setting of |minimum height|. Thus, this key must have been set up
    correctly to reflect the ``real'' height of the arrow tip. The
    reason is that the height of an arrow is not specified when arrows
    are declared and is, thus, not available, here.

    Possibly, the height computation will change in the future to
    reflect the real height of the arrow, so you should generally
    set up the |minimum height| to be the same as the real height.
  \item A straight line from left to right inside the shape's
    boundaries is added to the background path.
  \item The arrow tip, pointing right, is drawn before the background
    path.
  \end{enumerate}
  The anchors of this shape are just the compass anchors, which lie on
  a rectangle whose width and height are the above-computed height and
  width.

\begin{codeexample}[]
\begin{tikzpicture}
  \pgfsetarrowoptions{direction ee}{6cm}
  \node[name=s,shape=direction ee,shape example,minimum height=0.7654*6cm] {};
  \foreach \anchor/\placement in
    {center/above, 30/above right,
     north/above, south/below, east/left, west/right,
     north east/above, south east/below, south west/below, north west/above,
     input/left,output/right}
     \draw[shift=(s.\anchor)] plot[mark=x] coordinates{(0,0)}
       node[\placement] {\scriptsize\texttt{(s.\anchor)}};
\end{tikzpicture}
\end{codeexample}

\begin{codeexample}[]
\begin{tikzpicture}[direction ee arrow=angle 45]
  \node[name=s,shape=direction ee,shape example,minimum height=1.75cm] {};
  \foreach \anchor/\placement in {north/above, south/below,
                                  output/right, input/left}
     \draw[shift=(s.\anchor)] plot[mark=x] coordinates{(0,0)}
       node[\placement] {\scriptsize\texttt{(s.\anchor)}};
\end{tikzpicture}
\end{codeexample}
\end{shape}


\subsubsection{Implementation: The IEC-Style EE-Symbols Shape Library}

\begin{pgflibrary}{shapes.gates.ee.IEC}
  This library defines shapes for depicting ee symbols according to
  the IEC recommendations. These shapes will typically be
  used in conjunction with the graphic mechanism detailed earlier, but
  you can also used them directly.
\end{pgflibrary}

\begin{shape}{generic circle IEC}
  This shape inherits from |circle ee|, which in turn is just a normal
  |circle| with additional |input| and |output| anchors at the left
  and right ends. However, additionally, this shape allows you to
  specify a path that should be added before the background path using
  the following key:
  \begin{key}{/pgf/generic circle IEC/before background=\meta{code}}
    When a node of shape |generic circle IEC| is created, the current
    setting of this key is used as the ``before background path.''
    This means that after the circle's background has been
    drawn/filled/whatever, the \meta{code} is executed.

    When the \meta{code} is executed, the coordinate system will have
    been transformed in such a way that the point
    $(1\mathrm{pt},0\mathrm{pt})$ lies at the right end of the circle
    and $(0\mathrm{pt},1\mathrm{pt})$ lies at the top of the
    circle. (More precisely, these points will lie exactly on the
    middle of the radial line.)
  \end{key}
  Here is an examples of how to use this shape:
\begin{codeexample}[]
\tikz \node [generic circle IEC,
             /pgf/generic circle IEC/before background={
               \pgfpathmoveto{\pgfpointorigin}
               \pgfpathlineto{\pgfpoint{1pt}{0pt}}
               \pgfpathlineto{\pgfpoint{0pt}{1pt}}
               \pgfpathlineto{\pgfpoint{-0.5pt}{-0.5pt}}
               \pgfusepathqstroke
             },
             draw] {Hello world};
\end{codeexample}
\end{shape}


\begin{shape}{generic diode IEC}
  This shape is used to depict diodes. The main shape is taken up by a
  ``right pointing'' triangle. The anchors are positioned on the border of
  a rectangle around the diode, see the below example.  The
  diode's size is based on the current settings of |minimum width| and
  |minimum height|.
\begin{codeexample}[]
\begin{tikzpicture}
  \node[name=s,shape=generic diode IEC,shape example,minimum size=6cm] {};
  \foreach \anchor/\placement in
    {center/above, 30/above right,
     north/above, south/below, east/left, west/right,
     north east/above, south east/below, south west/below, north west/above,
     input/left,output/right}
     \draw[shift=(s.\anchor)] plot[mark=x] coordinates{(0,0)}
       node[\placement] {\scriptsize\texttt{(s.\anchor)}};
\end{tikzpicture}
\end{codeexample}

  This shape, like the |generic circle IEC| shape, is generic in the
  sense that there is a special key that is used for the before
  background drawings:
  \begin{key}{/pgf/generic diode IEC/before background=\meta{code}}
    Similarly to the |generic circle IEC| shape, when a node of shape
    |generic diode IEC| is created, the current setting of this key is
    used as the ``before background path.''
    When the \meta{code} is executed, the coordinate system will have
    been transformed in such a way that the origin is at the ``tip''
    of the diode's triangle, the point $(0\mathrm{pt},1\mathrm{pt})$
    is exactly half the diode's height above this origin, and
    the point $(1\mathrm{pt},0\mathrm{pt})$ is half the diode's height
    to the right of the origin.

    The idea is that you use this key to draw different kinds of diode
    endings.
\begin{codeexample}[]
\tikz \node [minimum size=1cm,generic diode IEC,
             /pgf/generic diode IEC/before background={
               \pgfpathmoveto{\pgfqpoint{-.5pt}{-1pt}}
               \pgfpathlineto{\pgfqpoint{.5pt}{-1pt}}
               \pgfpathmoveto{\pgfqpoint{0pt}{-1pt}}
               \pgfpathlineto{\pgfqpoint{0pt}{1pt}}
               \pgfpathmoveto{\pgfqpoint{-.5pt}{1pt}}
               \pgfpathlineto{\pgfqpoint{.5pt}{1pt}}
               \pgfusepathqstroke
             },
             draw] {};
\end{codeexample}
  \end{key}
\end{shape}



\begin{shape}{breakdown diode IEC}
  This shape is used to depict a bidirectional breakdown diode. The
  diode's size is based on the current settings of |minimum width| and
  |minimum height|.
\begin{codeexample}[]
\begin{tikzpicture}
  \node[name=s,shape=breakdown diode IEC,shape example,minimum width=6cm,minimum height=4cm] {};
  \foreach \anchor/\placement in
    {center/above, 30/above right,
     north/above, south/below, east/left, west/right,
     north east/above, south east/below, south west/below, north west/above,
     input/left,output/right}
     \draw[shift=(s.\anchor)] plot[mark=x] coordinates{(0,0)}
       node[\placement] {\scriptsize\texttt{(s.\anchor)}};
\end{tikzpicture}
\end{codeexample}
\end{shape}

\begin{shape}{var resistor IEC}
  This shape is used to depict a variant version of a resistor. Its
  size is computed as for a rectangle (thus, its size depends things
  like the |minimum height|). Then, inside this rectangle, a
  background path is set up according to the following rule: Starting
  from the left end, zigzag segments are added to the path. Each
  segment consists of a line at a 45 degree angle going up to the top
  of the rectangle, then going down to the bottom, then going up to
  mid height of the node. As many segments as possible are put inside
  as possible. The last segment is then connected to the output anchor
  via a straight line.

  All of this means that, in general, the shape should be much wider
  than high.
\begin{codeexample}[]
\begin{tikzpicture}
  \node[name=s,shape=var resistor IEC,shape example,minimum width=7cm,minimum height=1cm] {};
  \foreach \anchor/\placement in
    {center/above, 30/above right,
     north/above, south/below, east/left, west/right,
     north east/above, south east/below, south west/below, north west/above,
     input/left,output/right}
     \draw[shift=(s.\anchor)] plot[mark=x] coordinates{(0,0)}
       node[\placement] {\scriptsize\texttt{(s.\anchor)}};
\end{tikzpicture}
\end{codeexample}
\end{shape}


\begin{shape}{inductor IEC}
  This shape is used to depict an inductor, using a bumpy line. Its
  size is computed as follows: Any text and |inner sep| are ignored
  (and should normally not be given). The |minimum height| plus
  (twice) the |outer ysep| specify the distance between the |north|
  and |south| anchors, similarly for the |minimum width| plus the
  |outer xsep| for the |east| and |west|.
  The bumpy line is drawn starting from the lower left corner to the
  lower right corner with bumps being half-circles whose height is
  exactly the |minimum height|. The |center| of the shape is just
  above the |south| anchor, at a distance of the |outer ysep|.
\begin{codeexample}[]
\begin{tikzpicture}
  \node[name=s,shape=inductor IEC,shape example,minimum width=7cm,minimum height=1cm] {};
  \foreach \anchor/\placement in
    {center/above, 30/above right,
     north/above, south/below, east/left, west/right,
     north east/above, south east/below, south west/below, north west/above,
     input/left,output/right}
     \draw[shift=(s.\anchor)] plot[mark=x] coordinates{(0,0)}
       node[\placement] {\scriptsize\texttt{(s.\anchor)}};
\end{tikzpicture}
\end{codeexample}
  Just as for a |var resistor IEC|, as many bumps as possible are
  added and the last bump is connected to the output anchor via a
  straight line.
\end{shape}

\begin{shape}{capacitor IEC}
  This shape is based on a |rectangle ee|. However, instead of a
  rectangle as the background path, only the ``left and right lines''
  that make up the rectangle are drawn.
\begin{codeexample}[]
\begin{tikzpicture}
  \node[name=s,shape=capacitor IEC,shape example,
        minimum width=2cm,minimum height=3cm,inner sep=0pt] {};
  \foreach \anchor/\placement in
    {center/above, 30/above right,
     north/above, south/below, east/left, west/right,
     north east/above, south east/below, south west/below, north west/above,
     input/left,output/right}
     \draw[shift=(s.\anchor)] plot[mark=x] coordinates{(0,0)}
       node[\placement] {\scriptsize\texttt{(s.\anchor)}};
\end{tikzpicture}
\end{codeexample}
\end{shape}

\begin{shape}{battery IEC}
  This shape is similar to a |capacitor IEC|, however, the right line is
  only half the height of the left line.
\begin{codeexample}[]
\tikz \node[shape=battery IEC,shape example,minimum size=2cm,
            inner sep=0pt] {};
\end{codeexample}
\end{shape}

\begin{shape}{ground IEC}
  This shape is similar to a |batter IEC|, only three lines of
  different heights are drawn.
\begin{codeexample}[]
\tikz \node[shape=ground IEC,shape example,minimum size=2cm,
            inner sep=0pt] {};
\end{codeexample}
\end{shape}

\begin{shape}{make contact IEC}
  This shape consists of a line going from the lower left corner to
  the upper right corner. The size and anchors of this shape are
  computed in the same way as for an |inductor IEC|.
\begin{codeexample}[]
\begin{tikzpicture}
  \node[name=s,shape=make contact IEC,shape example,minimum width=3cm,minimum height=1cm] {};
  \foreach \anchor/\placement in
    {center/above, 30/above right,
     north/above, south/below, east/left, west/right,
     north east/above, south east/below, south west/below, north west/above,
     input/left,output/right}
     \draw[shift=(s.\anchor)] plot[mark=x] coordinates{(0,0)}
       node[\placement] {\scriptsize\texttt{(s.\anchor)}};
\end{tikzpicture}
\end{codeexample}
\end{shape}

\begin{shape}{var make contact IEC}
  This shape works like |make contact IEC|, only a little circle is
  added to the path at the lower left corner. The radius of this
  circle is one twelfth of the width of the node.
\begin{codeexample}[]
\tikz \node[shape=var make contact IEC,shape example,
            minimum height=1cm,minimum width=3cm,inner sep=0pt] {};
\end{codeexample}
\end{shape}

\begin{shape}{break contact IEC}
  This shape depicts a contact that can be broken. It works like
  |make contact IEC|.
\begin{codeexample}[]
\tikz \node[shape=break contact IEC,shape example,
            minimum height=1cm,minimum width=3cm,inner sep=0pt] {};
\end{codeexample}
\end{shape}

\endinput

% Copyright 2013 by Mark Wibrow and Till Tantau
%
% This file may be distributed and/or modified
%
% 1. under the LaTeX Project Public License and/or
% 2. under the GNU Free Documentation License.
%
% See the file doc/generic/pgf/licenses/LICENSE for more details.

\section{Decoration Library}
\label{section-library-decorations}



\subsection{Overview and Common Options}

The decoration libraries define a number of (more or less useful)
decorations that can be applied to paths. The usage of decorations is
not covered in the present section, please consult
Sections~\ref{section-tikz-decorations}, which explains how
decorations are used in \tikzname, and
\ref{section-base-decorations}, which  explains how new
decorations can be defined.

The decorations are influenced by a number of parameters that can be
set using the |decoration| option. These parameters are
typically shared between different decorations. In the following, the
general options are documented (they are defined directly in the
|decoration| module), special-purpose keys are documented
with the decoration that uses it.

Since you are encouraged to use these keys to make your own
decorations configurable, it is indicated for each key where the value
is stored (so that you can access it). Note that some values are
stored in \TeX\ dimension registers while others are stored in macros.

\begin{key}{/pgf/decoration/amplitude=\meta{dimension} (initially 2.5pt)}
  This key determines the ``desired height'' (or amplitude) of
  decorations for which this makes sense. For instance, the initial
  value of |2.5pt| means that deforming decorations should deform a
  path by up to 2.5pt away from the original path.

  This key sets the \TeX-dimension |\pgfdecorationsegmentamplitude|.
\end{key}

\begin{key}{/pgf/decoration/meta-amplitude=\meta{dimension} (initially 2.5pt)}
  This key determines the amplitude for a meta-decoration.

  The key sets the \TeX-macro (!) |\pgfmetadecorationsegmentamplitude|.
\end{key}

\begin{key}{/pgf/decoration/segment length=\meta{dimension} (initially 10pt)}
  Many decorations are made up of small segments. This key determines
  the desired length of such segments.

  This key sets the \TeX-dimension |\pgfdecorationsegmentlength|.
\end{key}

\begin{key}{/pgf/decoration/meta-segment length=\meta{dimension} (initially 1cm)}
  This determined the length of the meta-segments from which a
  meta-decoration is made up.

  This key sets the \TeX-macro (!) |\pgfmetadecorationsegmentlength|.
\end{key}

\begin{key}{/pgf/decoration/angle=\meta{degree} (initially 45)}
  The way some decorations look like depends on a configurable angle. For
  instance, a |wave| decoration consists of arcs and the opening angle
  of these arcs is given by the |angle|.

  This key sets the \TeX-macro |\pgfdecorationsegmentangle|.
\end{key}

\begin{key}{/pgf/decoration/aspect=\meta{factor} (initially 0.5)}
  For some decorations there is a natural aspect ratio. For instance,
  for a |brace| decoration the aspect ratio determines where the brace
  point will be.

  This key sets the \TeX-macro |\pgfdecorationsegmentaspect|.
\end{key}

\begin{key}{/pgf/decoration/start radius=\meta{dimension} (initially 2.5pt)}
  For some decorations there is a natural start radius (of some circle, presumably).

  This key stores the value directly inside the key.
\end{key}

\begin{key}{/pgf/decoration/end radius=\meta{dimension} (initially 2.5pt)}
  For some decorations there is a natural end radius (of some circle, presumably).

  This key stores the value directly inside the key.
\end{key}

\begin{stylekey}{/pgf/decoration/radius=\meta{dimension}}
  Sets the start and end radius simultaneously.
\end{stylekey}


\begin{key}{/pgf/decoration/path has corners=\meta{boolean} (initially false)}
  This is a hint to the decoration code as to whether the path has
  corners or not. If a path has a sharp corner, setting this option to
  |true| may result in better rendering of the decoration because the
  joins of input segments are approached ``more carefully'' than
  when this key is set to false. However, if the path is, say, a
  smooth circle, setting this key to |true| will usually look
  worse. Most decorations ignore this key, anyway. Internally, it sets
  the \TeX-if |\ifpgfdecoratepathhascorners|.
\end{key}


\subsection{Path Morphing Decorations}

\begin{pgflibrary}{decorations.pathmorphing}
  A \emph{path morphing decoration} ``morphs'' or ``deforms'' the
  to-be-decorated path. This means that what used to be a straight
  line might afterwards be a snaking curve and have bumps. However, a
  line is still a line and path deforming decorations do not
  change the number of subpaths. For instance, if the path used to
  consist of two circles and an open arc, the path will, after the
  decoration process, still consist of two closed subpaths and one open
  subpath.
\end{pgflibrary}


\subsubsection{Decorations Producing Straight Line Paths}

The following deformations use only straight lines in order to morph
the paths.

\begin{decoration}{lineto}
  This decoration replaces the path by straight lines. For each curve,
  the path simply goes directly from the start point to the end point.
  In the following example, the arc actually consist of two
  subcurves.

  This decoration is actually always defined when the decoration
  module is loaded, but it is documented here for consistency.
\begin{codeexample}[]
\begin{tikzpicture}[decoration=lineto]
  \draw [help lines] grid (3,2);
  \draw [decorate,fill=yellow!80!black]
    (0,0) -- (3,1) arc (0:180:1.5 and 1) -- cycle;
\end{tikzpicture}
\end{codeexample}
\end{decoration}


\begin{decoration}{straight zigzag}
  This (meta-)decoration decorates the path by alternating between
  |curveto| and |zigzag| decorations. It always finishes
  with the |curveto| decoration. The following parameters influence
  the decoration:
  \begin{itemize}
  \item |amplitude|
    determines how much the zigzag line raises above and falls below
    a straight line to the target point.
  \item |segment length|
    determines the length of a complete ``up-down'' cycle.
  \item  |meta-segment length|
    determines the length of the |curveto| and the |zigzag| decorations.
  \end{itemize}

\begin{codeexample}[]
\begin{tikzpicture}[decoration={straight zigzag,meta-segment length=1.1cm}]
  \draw [help lines] grid (3,2);
  \draw [decorate,fill=yellow!80!black]
    (0,0) -- (3,1) arc (0:180:1.5 and 1) -- cycle;
\end{tikzpicture}
\end{codeexample}
\end{decoration}


\begin{decoration}{random steps}
  This decoration consists of straight line segments. The line segments
  head towards the target, but each step is randomly shifted a little
  bit. The following parameters influence the decorations:
  \begin{itemize}
  \item |segment length|
    determines the basic length of each step.
  \item |amplitude|
    The end of each step is perturbed both in $x$- and in
    $y$-direction by two values drawn uniformly from the interval
    $[-d,d]$, where $d$ is the value of |amplitude|.
  \end{itemize}
\begin{codeexample}[]
\begin{tikzpicture}
    [decoration={random steps,segment length=2mm}]
  \draw [help lines] grid (3,2);
  \draw [decorate,fill=yellow!80!black]
    (0,0) -- (3,1) arc (0:180:1.5 and 1) -- cycle;
\end{tikzpicture}
\end{codeexample}
\end{decoration}


\begin{decoration}{saw}
  This decoration looks like the blade of a saw. The following parameters
  influence the decoration:
  \begin{itemize}
  \item |amplitude|
    determines how much each spike raises above the straight line.
  \item |segment length|
    determines the length each spike.
  \end{itemize}
\begin{codeexample}[]
\begin{tikzpicture}[decoration=saw]
  \draw [help lines] grid (3,2);
  \draw [decorate,fill=yellow!80!black]
    (0,0) -- (3,1) arc (0:180:1.5 and 1) -- cycle;
\end{tikzpicture}
\end{codeexample}
\end{decoration}


\begin{decoration}{zigzag}
  This decoration looks like a zigzag line. The following parameters
  influence the decoration:
  \begin{itemize}
  \item |amplitude|
    determines how much the zigzag line raises above and falls below
    a straight line to the target point.
  \item |segment length|
    determines the length of a complete ``up-down'' cycle.
  \end{itemize}
\begin{codeexample}[]
\begin{tikzpicture}[decoration=zigzag]
  \draw [help lines] grid (3,2);
  \draw [decorate,fill=yellow!80!black]
    (0,0) -- (3,1) arc (0:180:1.5 and 1) -- cycle;
\end{tikzpicture}
\end{codeexample}
\end{decoration}



\subsubsection{Decorations Producing Curved Line Paths}

\begin{decoration}{bent}
  This decoration adds a slightly bent line from the start to the
  target. The amplitude of the bend is given |amplitude|
  (an amplitude of zero gives a straight line).
  \begin{itemize}
  \item |amplitude|
    determines the amplitude of the bend.
  \item |aspect|
    determines how tight the bend is. A good value is around |0.3|.
  \end{itemize}
  Note that this decoration makes only little sense for curves. You
  should apply it only to straight lines.
\begin{codeexample}[]
\begin{tikzpicture}[decoration=bent]
  \draw [help lines] grid (3,2);
  \draw [decorate] (0,0) -- (3,1) -- (1.5,2) -- (0,1);
\end{tikzpicture}
\end{codeexample}
\begin{codeexample}[]
\begin{tikzpicture}[decoration={bent,aspect=.3}]
  \draw [decorate,fill=yellow!80!black] (0,0) rectangle (3.5,2);
  \node[circle,draw] (A) at (.5,.5) {A};
  \node[circle,draw] (B) at (3,1.5) {B};
  \draw[->,decorate] (A) -- (B);
  \draw[->,decorate] (B) -- (A);
\end{tikzpicture}
\end{codeexample}
\end{decoration}


\begin{decoration}{bumps}
  This decoration replaces the path by little half ellipses. The
  following parameters influence it.
  \begin{itemize}
  \item |amplitude|
    determines the height of the half ellipse.
  \item |segment length|
    determines the width of the half ellipse.
  \end{itemize}
\begin{codeexample}[]
\begin{tikzpicture}[decoration=bumps]
  \draw [help lines] grid (3,2);
  \draw [decorate,fill=yellow!80!black]
    (0,0) -- (3,1) arc (0:180:1.5 and 1) -- cycle;
\end{tikzpicture}
\end{codeexample}
\end{decoration}


\begin{decoration}{coil}
  This decoration replaces the path by a coiled line. To understand how this works,
  imagine a three-dimensional spring. The spring's axis points along
  the path toward the target. Then, we ``view'' the spring from a
  certain angle. If we look ``straight from the side'' we will see a
  perfect sine curve, if we look ``more from the front'' we will see a
  coil. The following parameters influence the decoration:
  \begin{itemize}
  \item |amplitude|
    determines how much the coil rises above the path and falls below
    it. Thus, this is the radius of the coil.
  \item |segment length|
    determines the distance between two consecutive ``curls.'' Thus,
    when the spring is see ``from the side'' this will be the wave
    length of the sine curve.
  \item |aspect|
    determines the ``viewing direction.'' A value of |0| means
    ``looking from the side'' and a value of |0.5|, which is the
    default, means ``look more from the front.''
  \end{itemize}
\begin{codeexample}[]
\begin{tikzpicture}[decoration=coil]
  \draw [help lines] grid (3,2);
  \draw [decorate,fill=yellow!80!black]
    (0,0) -- (3,1) arc (0:180:1.5 and 1) -- cycle;
\end{tikzpicture}
\end{codeexample}
\begin{codeexample}[]
\begin{tikzpicture}
    [decoration={coil,aspect=0.3,segment length=3mm,amplitude=3mm}]
  \draw [help lines] grid (3,2);
  \draw [decorate,fill=yellow!80!black]
    (0,0) -- (3,1) arc (0:180:1.5 and 1) -- cycle;
\end{tikzpicture}
\end{codeexample}
\end{decoration}



\begin{decoration}{curveto}
  This decoration simply yields a line following the original
  path. This means that (ideally) it does not change the path and
  follows any curves in the path (hence the name). In
  reality, due to the internals of how decorations are implemented,
  this decoration actually replaces the path by numerous small
  straight lines.

  This decoration is mostly useful in conjunction with
  meta-decorations. It is also actually defined in the decoration
  module and is always available.

\begin{codeexample}[]
\begin{tikzpicture}[decoration=curveto]
  \draw [help lines] grid (3,2);
  \draw [decorate,fill=yellow!80!black]
    (0,0) -- (3,1) arc (0:180:1.5 and 1) -- cycle;
\end{tikzpicture}
\end{codeexample}
\end{decoration}



\begin{decoration}{snake}
  This decoration replaces the path by a line that looks like a snake
  seen from above. More precisely, the snake is a sine wave with a
  ``softened'' start and ending. The following parameters influence
  the snake:
  \begin{itemize}
  \item |amplitude|
    determines the sine wave's amplitude.
  \item |segment length|
    determines the sine wave's wavelength.
  \end{itemize}
\begin{codeexample}[]
\begin{tikzpicture}[decoration=snake]
  \draw [help lines] grid (3,2);
  \draw [decorate,fill=yellow!80!black]
    (0,0) -- (3,1) arc (0:180:1.5 and 1) -- cycle;
\end{tikzpicture}
\end{codeexample}
\end{decoration}




\subsection{Path Replacing Decorations}

\begin{pgflibrary}{decorations.pathreplacing}
  This library defines decorations that replace the to-be-decorated
  path by another path. Unlike morphing decorations, the replaced path
  might be quite different, for instance a straight line might be
  replaced by a set of circles. Note that filling a path that has been
  replaced using one of the decorations in this library typically does
  not fill the original area but, rather, the smaller area of the
  newly-created path segments.
\end{pgflibrary}

\begin{decoration}{border}
  This decoration adds straight lines to the path that are at a specific
  angle to the line toward the target. The idea is to add these little
  lines to indicate the ``border'' of an area. The following
  parameters influence the decoration:
  \begin{itemize}
  \item |segment length|
    determines the distance between consecutive ticks.
  \item |amplitude|
    determines the length of the ticks.
  \item |angle|
    determines the angle between the ticks and the line of the path.
  \end{itemize}
\begin{codeexample}[]
\begin{tikzpicture}[decoration=border]
  \draw [help lines] grid (3,2);
  \draw [postaction={decorate,draw,red}]
        (0,0) -- (3,1) arc (0:180:1.5 and 1);
\end{tikzpicture}
\end{codeexample}
\end{decoration}


\begin{decoration}{brace}
  This decoration replaces a straight line path by a long brace. The
  left and right end of the brace will be exactly on the start and
  endpoint of the decoration. The decoration really only makes sense
  for paths that are a straight line.
  \begin{itemize}
  \item |amplitude|
    determines how much the brace rises above the path.
  \item |aspect|
    determines the fraction of the total length where the ``middle
    part'' of the brace will be.
  \end{itemize}
\begin{codeexample}[]
\begin{tikzpicture}[decoration=brace]
  \draw [help lines] grid (3,2);
  \draw [decorate] (0,0) -- (3,1);
\end{tikzpicture}
\end{codeexample}
\end{decoration}



\begin{decoration}{expanding waves}
  This decoration adds arcs to the path that get bigger along the line
  towards the target. The following parameters influence the decoration:
  \begin{itemize}
  \item |segment length|
    determines the distance between consecutive arcs.
  \item |angle|
    determines the opening angle below and above the path. Thus, the
    total opening angle is twice this angle.
  \end{itemize}
\begin{codeexample}[]
\begin{tikzpicture}[decoration={expanding waves,angle=5}]
  \draw [help lines] grid (3,2);
  \draw [decorate] (0,0) -- (3,1) arc (0:180:1.5 and 1);
\end{tikzpicture}
\end{codeexample}
\end{decoration}


\begin{decoration}{moveto}
  This decoration simply jumps to the end of the path using a move-to
  path operation. It is mainly useful as |pre=moveto| or |post=moveto|
  decorations.

  This decoration is actually always defined when the decoration
  module is loaded, but it is documented here for consistency.
\end{decoration}


\begin{decoration}{ticks}
  This decoration replaces the path by straight lines that are
  orthogonal to the path. The following parameters influence the
  decoration:
  \begin{itemize}
  \item |segment length|
    determines the distance between consecutive ticks.
  \item |amplitude|
    determines half the length of the ticks.
  \end{itemize}
\begin{codeexample}[]
\begin{tikzpicture}[decoration=ticks]
  \draw [help lines] grid (3,2);
  \draw [decorate] (0,0) -- (3,1) arc (0:180:1.5 and 1);
\end{tikzpicture}
\end{codeexample}
\end{decoration}



\begin{decoration}{waves}
  This decoration replaces the path by arcs that have a constant
  size. The following parameters influence the decoration:
  \begin{itemize}
  \item |segment length|
    determines the distance between consecutive arcs.
  \item |angle|
    determines the opening angle below and above the path. Thus, the
    total opening angle is twice this angle.
  \item |radius|
    determines the radius of each arc.
  \end{itemize}
\begin{codeexample}[]
\begin{tikzpicture}[decoration={waves,radius=4mm}]
  \draw [help lines] grid (3,2);
  \draw [decorate] (0,0) -- (3,1) arc (0:180:1.5 and 1);
\end{tikzpicture}
\end{codeexample}
\end{decoration}




\begin{decoration}{show path construction}
  This decoration allows ``something different'' to be done
  for each \emph{type} of input segment (i.e., moveto, lineto,
  curveto or closepath). Typically, each segment will be replaced
  with another path, but this need not necessarily be the case.
  
\begin{codeexample}[]
\begin{tikzpicture}[>=stealth, every node/.style={midway, sloped, font=\tiny},
  decoration={show path construction,
    moveto code={
      \fill [red] (\tikzinputsegmentfirst) circle (2pt)
        node [fill=none, below] {moveto};},
    lineto code={
      \draw [blue,->] (\tikzinputsegmentfirst) -- (\tikzinputsegmentlast)
        node [above] {lineto};
    },
    curveto code={
      \draw [green!75!black,->] (\tikzinputsegmentfirst) .. controls
        (\tikzinputsegmentsupporta) and (\tikzinputsegmentsupportb)
        ..(\tikzinputsegmentlast) node [above] {curveto};
    },
    closepath code={
      \draw [orange,->] (\tikzinputsegmentfirst) -- (\tikzinputsegmentlast)
        node [above] {closepath};}
  }]
  \draw [help lines] grid (3,2);
  \path [decorate] (0,0) -- (3,1) arc (0:180:1.5 and 1) -- cycle;
\end{tikzpicture}
\end{codeexample}


  The following keys can be used to specify the code to execute
  for each type of input segment.

\begin{key}{/pgf/decoration/moveto code=\meta{code} (initially \char`\{\char`\})}
  Set the code to be executed for every moveto input segment.
  It is important to remember that the transformations applied
  by the decoration automaton are turned \emph{off} when \meta{code}
  is executed.
\end{key}

\begin{key}{/pgf/decoration/lineto code=\meta{code} (initially \char`\{\char`\})}
  Set the code to be executed for every lineto input segment.
\end{key}

\begin{key}{/pgf/decoration/curveto code=\meta{code} (initially \char`\{\char`\})}
  Set the code to be executed for every curveto input segment.
\end{key}

\begin{key}{/pgf/decoration/closepath code=\meta{code} (initially \char`\{\char`\})}
  Set the code to be executed for every closepath input segment.
\end{key}

Within \meta{code} the first and last
points on the current input segment can be accessed using
|\pgfpointdecoratedinputsegmentfirst| and
|\pgfpointdecoratedinputsegmentlast|. For curves, the
control (support) points can be accessed using
|\pgfpointdecoratedinputsegmentsupporta| and
|\pgfpointdecoratedinputsegmentsupportb|.

In \tikzname, you can use the following macros
inside a \tikzname{} coordinate.

\begin{command}{\tikzinputsegmentfirst}
  The first point on the current input segment path.
\end{command}

\begin{command}{\tikzinputsegmentlast}
  The last point on the current input segment path.
\end{command}

\begin{command}{\tikzinputsegmentsupporta}
  The first support on the curveto input segment path.
\end{command}

\begin{command}{\tikzinputsegmentsupportb}
  The second support on the curveto input segment path.
\end{command}

{\tikzexternaldisable
\begin{codeexample}[]
\tikzset{
  show curve controls/.style={
    decoration={
      show path construction,
      curveto code={
        \draw [blue, dashed]
          (\tikzinputsegmentfirst)    -- (\tikzinputsegmentsupporta)
          node [at end, cross out, draw, solid, red, inner sep=2pt]{};
        \draw [blue, dashed]
          (\tikzinputsegmentsupportb) -- (\tikzinputsegmentlast)
          node [at start, cross out, draw, solid, red, inner sep=2pt]{};
      }
    },decorate
  }
}

\tikzpicture
  \draw [postaction=show curve controls, thick]
    (0,2) .. controls (2.5,1.5) and (0.5,0.5) .. (3,0);
\endtikzpicture
\end{codeexample}
}%
\end{decoration}




\subsection{Marking Decorations}

\subsubsection{Overview}

A \emph{marking on a path} is any kind of graphic that is placed on a
specific position on a path. Markings are useful in rather diverse
situations: you can use them to, say, place little ``footsteps'' along
a path as if someone where walking along the path; to place arrow tips
on the middle of a path to indicate the ``direction'' in which
something is flowing; or you can use them to place informative
information at certain positions of a path.

For historical reasons there are three different libraries for placing
marks on a path. They differ in what kind of markings can be added to
a path. We start with the most general and most useful of these libraries.



\subsection{Arbitrary Markings}

\begin{pgflibrary}{decorations.markings}
  Markings are arbitrary ``marks'' that can be put on a path. Marks
  can be arrow tips or nodes or even whole pictures.
\end{pgflibrary}

\begin{decoration}{markings}
  A \emph{marking} can be thought of a ``little picture'' or more
  precisely of ``some scope contents'' that is placed ``on'' a path at
  a certain position. Suppose the marking should be a simple cross. We
  can produce this with the following code:
\begin{codeexample}[code only]
\draw (-2pt,-2pt) -- (2pt,2pt);
\draw (2pt,-2pt) -- (-2pt,2pt);
\end{codeexample}
  If we use this code as a marking at position |2cm| on a path, then
  the following happens: \pgfname\ determines the position on the path
  that is 2cm along the path. Then is translates the coordinate system
  to this position and rotates it such that the positive $x$-axis is
  tangent to the path. Then a protective scope is created, inside
  which the above code is executed -- resulting in a little cross on
  the path.

  The |markings| decoration allows you to place one or more such
  markings on a path. The decoration destroys the input path (except
  in certain cases, detailed later), which means that it uses the path
  for determining positions on the path, but after the decoration is
  done this path is gone. You typically need to use a |postaction| to
  add markings.

  Let us start with the above example in real code:
\begin{codeexample}[]
\begin{tikzpicture}[decoration={
    markings,% switch on markings
    mark=% actually add a mark
      at position 2cm
      with
      {
        \draw (-2pt,-2pt) -- (2pt,2pt);
        \draw (2pt,-2pt) -- (-2pt,2pt);
      }
    }
    ]
  \draw [help lines] grid (3,2);
  \draw [postaction={decorate}] (0,0) -- (3,1) arc (0:180:1.5 and 1);
\end{tikzpicture}
\end{codeexample}

  We can also add the cross repeatedly:
\begin{codeexample}[]
\begin{tikzpicture}[decoration={
    markings,% switch on markings
    mark=% actually add a mark
      between positions 0 and 1 step 5mm
      with
      {
        \draw (-2pt,-2pt) -- (2pt,2pt);
        \draw (2pt,-2pt) -- (-2pt,2pt);
      }
    }
    ]
  \draw [help lines] grid (3,2);
  \draw [postaction={decorate}] (0,0) -- (3,1) arc (0:180:1.5 and 1);
\end{tikzpicture}
\end{codeexample}

  The |mark| decoration option is used to specify a marking. It comes
  in two versions:
  \begin{key}{/pgf/decoration/mark=\texttt{at position}
      \meta{pos}| with |\meta{code}}
    The options specifies that when a |marking| decoration is applied,
    there should be a marking at position \meta{pos} on the path whose
    code is given by \meta{code}.

    The \meta{pos} can have four different forms:
    \begin{enumerate}
    \item It can be a non-negative dimension like |0pt| or |2cm| or
      |5cm/2|. In this case, it refers to the position along the path
      that is this far displaced from the start.
    \item It can be a negative dimension like |-1cm-2pt| or |-1sp|. In
      this case, the position is taken from the end of the path. Thus,
      |-1cm| is the position that is $-1$cm displaced from the end of
      the path.
    \item It can be a dimensionless non-negative number like |1/2| or
      |0.333+2*0.1|. In this case, the \meta{pos} is interpreted as a
      factor of the total path length. Thus, a \meta{pos} or |0.5|
      refers to the middle of the path, |0.1| is near the start, and
      so on.
    \item It can be a dimensionless negative number like |-0.1|. Then,
      again, the fraction of the path length counts ``from the end.''
    \end{enumerate}

    The \meta{pos} determines a position on the path. When the marking
    is applied, the (high level) coordinate system will have been
    transformed so that the origin lies at this position and the
    positive $x$-axis points along the path. For this coordinate
    system, the \meta{code} is executed. It can contain all sorts of
    graphic drawing commands, including (even named) nodes.

    If the position lies past the end of the path (for
    instance if \meta{pos} is set to |1.2|), the marking will not be
    drawn.

    It is possible to give the |mark| option several times, which
    causes several markings to be applied. In this case, however, it
    is necessary that the positions on the path are in increasing
    order. That is, it is not allowed (and will result in chaos) to
    have a marking that lies earlier on the path to follow a marking
    that is later on the path.

\begin{codeexample}[]
\begin{tikzpicture}[decoration={
    markings,% switch on markings
    mark=at position 1cm  with \node[red]{1cm};,
    mark=at position .5   with \node[green]{mid};,
    mark=at position -1cm with {\node[blue,transform shape]{1cm from end};}}
    ]
  \draw [help lines] grid (3,2);
  \draw [postaction={decorate}] (0,0) -- (3,1) arc (0:180:1.5 and 1);
\end{tikzpicture}
\end{codeexample}

    Here is an example that shows how markings can be used to place text
    on plots:
\begin{codeexample}[]
\begin{tikzpicture}[domain=0:4,label/.style={postaction={
      decorate,
      decoration={
        markings,
        mark=at position .75 with \node #1;}}}]
  \draw[very thin,color=gray] (-0.1,-1.1) grid (3.9,3.9);

  \draw[->] (-0.2,0) -- (4.2,0) node[right] {$x$};
  \draw[->] (0,-1.2) -- (0,4.2) node[above] {$f(x)$};

  \draw[red,label={[above left]{$f(x)=x$}}]                       plot (\x,\x);
  \draw[blue,label={[below left]{$f(x)=\sin x$}}]                 plot (\x,{sin(\x r)});
  \draw[orange,label={[right]{$f(x)= \frac{1}{20} \mathrm e^x$}}] plot (\x,{0.05*exp(\x)});
\end{tikzpicture}
\end{codeexample}

    When the \meta{code} is being executed, two special keys will have
    been set up, whose value may be of interest:
    \begin{key}{/pgf/decoration/mark info/sequence number}
      This key can only be read. Its value (which can be obtained
      using the |\pgfkeysvalueof| command) is a ``sequence number'' of
      the mark. The first mark that is added to a path has number |1|,
      the second number |2|, and so on. This key is mainly useful in
      conjunction with repeated markings (see below).
    \end{key}
    \begin{key}{/pgf/decoration/mark info/distance from start}
      This key can only be read. Its value is the distance of the
      marking from the start of the path in points. For instance, if
      the path length is 100pt and the marking is in the middle of the
      path, the value of this key would be |50.0pt|.
    \end{key}
  \end{key}

  A second way to use the |mark| key is the following:
  \begin{key}{/pgf/decoration/mark=|between positions|
      \meta{start pos} |and| \meta{end pos} |step| \meta{stepping}
      |with| \meta{code}}
    This works similarly to the |at position| version of this option,
    only multiple marks are placed, starting at \meta{start pos} and
    then spaced apart by \meta{stepping}. The \meta{start pos}, the
    \meta{end pos}, and also the \meta{stepping} may all be specified
    in the same way as for the |at position| version, that is, either
    using units or no units and also using positive or negative
    values.

    Let us start with a simple example in which we place ten crosses
    along a path starting with the beginning of the
    path ($\meta{start pos} = 0$) and ending at the end ($\meta{end
      pos} = 1$).
\begin{codeexample}[]
\begin{tikzpicture}[decoration={markings,
    mark=between positions 0 and 1 step 0.1
         with { \draw (-2pt,-2pt) -- (2pt,2pt);
                \draw (2pt,-2pt) -- (-2pt,2pt); }} ]
  \draw [help lines] grid (3,2);
  \draw [postaction={decorate}] (0,0) -- (3,1) arc (0:180:1.5 and 1);
\end{tikzpicture}
\end{codeexample}

    In the next example we place arrow shapes on the path instead of
    crosses. Note the use of the |transform shape| option to ensure
    that the nodes are actually rotated.
\begin{codeexample}[]
\begin{tikzpicture}[decoration={markings,
    mark=between positions 0 and 1 step 1cm
      with { \node [single arrow,fill=red,
                    single arrow head extend=3pt,transform shape] {};}}]
  \draw [help lines] grid (3,2);
  \draw [postaction={decorate}] (0,0) -- (3,1) arc (0:180:1.5 and 1);
\end{tikzpicture}
\end{codeexample}

    Using the key |sequence number| we can also ``number'' the nodes
    and even refer to them later on.
  % FIXME : the automatic key highlighting fails here!
\begin{codeexample}[]
\begin{tikzpicture}[decoration={markings,
    mark=between positions 0 and 1 step 1cm with {
      \node [draw,
        name=mark-\pgfkeysvalueof{/pgf/decoration/mark info/sequence number},
        transform shape]
      {\pgfkeysvalueof{/pgf/decoration/mark info/sequence number}};}}]
  \draw [help lines] grid (3,2);
  \draw [postaction={decorate}] (0,0) -- (3,1) arc (0:180:1.5 and 1);
  \draw [red,->] (mark-3) -- (mark-7);
\end{tikzpicture}
\end{codeexample}

    In the following example we use the distance info to place
    ``length information'' on a path:
\begin{codeexample}[]
\begin{tikzpicture}[decoration={markings,
    % Main marks
    mark=between positions 0 and 1 step 40pt with
      { \draw [help lines] (0,0) -- (0,0.5)
        node[above,font=\tiny]{
          \pgfkeysvalueof{/pgf/decoration/mark info/distance from start}}; },
    mark=at position -0.1pt with
      { \draw [help lines] (0,0) -- (0,0.5)
        node[above,font=\tiny]{
          \pgfkeysvalueof{/pgf/decoration/mark info/distance from start}}; }}]
  \draw [help lines] grid (5,3);
  \draw [postaction={decorate}]  (0,0) .. controls (8,3) and (0,3) .. (5,0) ;
\end{tikzpicture}
\end{codeexample}
  \end{key}

  \begin{key}{/pgf/decoration/reset marks}
    Since |mark| options accumulate, there needs to be a way to
    ``reset'' things, so that any |mark| options set in an enclosing
    scope do not interfere. This option does exactly this. Note that
    when the \meta{code} of a marking is executed, the markings are
    automatically reset.
  \end{key}

  As mentioned earlier, the decoration usually destroys the
  path. However, this is no longer the case when the following key is
  set:
  \begin{key}{/pgf/decoration/mark connection node=\meta{node name} (initially empty)}
    When this key is set to a non-empty \meta{node name} while the
    decoration is being processed, the following happens: The marking
    code should, among possibly other things, define a node named
    \meta{node name}. Then, the output path of this decoration will
    contain a line-to to ``one end'' of this node, followed by a
    moveto to the ``other end'' of the node. More precisely, the first
    end is given by the position on the border of \meta{node name}
    that lies in the direction ``from which the path heads toward the
    node'' while the other end lies on the border ``where the path
    heads away from the node.'' Furthermore, this option causes the
    decoration to end with a line-to to the end instead of a move-to.

    The net effect of all this is that when you decorate a straight
    line with one or more markings that contain just a node, the line
    will effectively connect these nodes.

    Here are two examples that show how this works:
\begin{codeexample}[]
\begin{tikzpicture}[decoration={markings,
    mark connection node=my node,
    mark=at position .5 with
      {\node [draw,blue,transform shape] (my node) {my node};}}]
  \draw [help lines] grid (3,2);
  \draw decorate { (0,0) -- (3,2) };
\end{tikzpicture}
\end{codeexample}

\begin{codeexample}[]
\begin{tikzpicture}[decoration={markings,
    mark connection node=my node,
    mark=at position .25 with
      {\node [draw,red] (my node) {my node};}}]
  \draw [help lines] grid (3,2);
  \draw decorate { (0,0) -- (3,2) };
\end{tikzpicture}
\end{codeexample}

  \end{key}
\end{decoration}


\subsubsection{Arrow Tip Markings}

Frequent markings that are hard to create correctly are arrow
tips. For them, two special commands are available when the \meta{code} of
a |mark| option is executed. (They are only defined in this code):

\begin{command}{\arrow\opt{\oarg{options}}\marg{arrow end tip}}
  This command simply draws the \meta{arrow end tip} at the origin,
  pointing right. This is exactly what you need when you want to
  draw an arrow tip as a marking.

  The \meta{options} can only be given when \tikzname\ is used. In
  this case, they are executed in a scope that contains the arrow
  tip.
\begin{codeexample}[]
\begin{tikzpicture}[decoration={
    markings,% switch on markings
    mark=at position 1cm  with {\node[red]{1cm};},
    mark=at position .75  with {\arrow[blue,line width=2mm]{>}},
    mark=at position -1cm with {\arrowreversed[black]{stealth}}}
    ]
  \draw [help lines] grid (3,2);
  \draw [postaction={decorate}] (0,0) -- (3,1) arc (0:180:1.5 and 1);
\end{tikzpicture}
\end{codeexample}

  Here is a more useful example:
\begin{codeexample}[]
\begin{tikzpicture}[decoration={
    markings,% switch on markings
    mark=between positions 0 and .75 step 4mm with {\arrow{stealth}},
    mark=between positions .75 and 1 step 4mm with {\arrowreversed{stealth}}}
    ]
  \draw [help lines] grid (3,2);
  \draw [postaction={decorate}] (0,0) -- (3,1) arc (0:180:1.5 and 1);
\end{tikzpicture}
\end{codeexample}
\end{command}

\begin{command}{\arrowreversed\opt{\oarg{options}}\marg{arrow end tip}}
  As above, only the arrow end tip is flipped and points in the
  other direction.
\end{command}


\subsubsection{Footprint Markings}

\begin{pgflibrary}{decorations.footprints}
  The decorations of this library can be used to decorate a path with
  little footprints, as if someone had ``walked'' along the path.
\end{pgflibrary}

\begin{decoration}{footprints}
  The footprint decoration adds little footprints around the
  path. They start with the left foot.
\begin{codeexample}[]
\begin{tikzpicture}[decoration={footprints,foot length=5pt,stride length=10pt}]
  \draw [help lines] grid (3,3);
  \fill [decorate] (0,0) -- (3,2) arc (0:180:1.5 and 1);
\end{tikzpicture}
\end{codeexample}
  You can influence the way this decoration looks using the following
  options:
  \begin{key}{/pgf/decoration/foot length (initially 10pt)}
    The length or size of the footprint itself. A larger value makes
    the footprint larger, but does not change the stride length.
\begin{codeexample}[]
\begin{tikzpicture}[decoration={footprints,foot length=20pt}]
  \fill [decorate] (0,0) -- (3,0);
\end{tikzpicture}
\end{codeexample}
  \end{key}
  \begin{key}{/pgf/decoration/stride length (initially 30pt)}
    The length of strides. This is the distance between the beginnings
    of left footprints along the path.
\begin{codeexample}[]
\begin{tikzpicture}[decoration={footprints,stride length=50pt}]
  \fill [decorate] (0,0) -- (3,0);
\end{tikzpicture}
\end{codeexample}
  \end{key}
  \begin{key}{/pgf/decoration/foot sep (initially 4pt)}
    The separation in the middle between the footprints. The
    footprints are moved away from the path by half this amount.
\begin{codeexample}[]
\begin{tikzpicture}[decoration={footprints,foot sep=10pt}]
  \fill [decorate] (0,0) -- (3,0);
\end{tikzpicture}
\end{codeexample}
  \end{key}
  \begin{key}{/pgf/decoration/foot angle (initially 10)}
    Footprints are rotated by this much.
\begin{codeexample}[]
\begin{tikzpicture}[decoration={footprints,foot angle=60}]
  \fill [decorate] (0,0) -- (3,0);
\end{tikzpicture}
\end{codeexample}
  \end{key}
  \begin{key}{/pgf/decoration/foot of (initially human)}
    The species whose footprints are shown. Possible values are:

    \def\render#1{
      \texttt{#1} &
      \tikz [baseline,decoration={footprints,foot of=#1}]
        \fill [decorate] (0,0) -- (6,0); \\[3em]
    }
    \begin{tabular}{ll}
      \emph{Species} & \emph{Result} \\[1em]
      \render{gnome}
      \render{human}
      \render{bird}
      \render{felis silvestris}
    \end{tabular}
  \end{key}
\end{decoration}




\subsubsection{Shape Background Markings}

The third library for adding markings uses the background paths of
certain shapes. This library is included mostly for historical
reasons, using the |markings| library is usually preferable.

\begin{pgflibrary}{decorations.shapes}
  This library defines decorations that use shapes or shape-like
  drawings to decorate a path. The following options are common
  options used by the decorations in this library:

  \begin{key}{/pgf/decoration/shape width=\meta{dimension}  (initially 2.5pt)}
    The desired width of the shapes. For decorations that support
    varying shape sizes, this key sets both the start and end width
    (which can be overwritten using options like |shape start width|).
  \end{key}

  \begin{key}{/pgf/decoration/shape height=\meta{dimension} (initially 2.5pt)}
    Works like the previous key, only for the height.
  \end{key}

  \begin{key}{/pgf/decoration/shape size=\meta{dimension}}
    Sets the desired width and height simultaneously.
  \end{key}

  For the exact places and macros where these keys store the values,
  please consult the beginning of the code of the library.
\end{pgflibrary}



\begin{decoration}{crosses}
  This decoration replaces the path by (diagonal) crosses. The
  following parameters influence the decoration:
  \begin{itemize}
  \item |segment length|
    determines the distance between (the centers of) consecutive crosses.
  \item |shape height|
    determines the height of each cross.
  \item |shape width|
    determines the width of each cross.
  \end{itemize}
\begin{codeexample}[]
\begin{tikzpicture}[decoration=crosses]
  \draw [help lines] grid (3,2);
  \draw [decorate] (0,0) -- (3,1) arc (0:180:1.5 and 1);
\end{tikzpicture}
\end{codeexample}
\end{decoration}

\begin{decoration}{triangles}
  This decoration replaces the path by triangles that point along the
  path. The following parameters influence the decoration:
  \begin{itemize}
  \item |segment length|
    determines the distance between consecutive triangles.
  \item |shape height|
    determines the height of the triangle side that is orthogonal
    to the path.
  \item |shape width|
    determines the width of the triangle.
  \end{itemize}
\begin{codeexample}[]
\begin{tikzpicture}[decoration=triangles]
  \draw [help lines] grid (3,2);
  \draw [decorate,fill=yellow!80!black] (0,0) -- (3,1) arc (0:180:1.5 and 1);
\end{tikzpicture}
\end{codeexample}
\end{decoration}


\begin{decoration}{shape backgrounds}
  This is a general decoration that replaces the to-be-decorated path by repeated
  copies of the background path of an arbitrary shape that has
  previously been defined using the |\pgfdeclareshape| command (that is,
  you can use any shape in the shape libraries).

  Please note that the background path of the shapes is used, but
  \emph{no nodes are created}. This means that \emph{you cannot have
    text inside the shapes of this path, you cannot name them, or
    refer to them.} Finally, this decoration \emph{will not work with
    shapes that depend  strongly on the size of the text box (like the
    arrow shapes).}  If any of these restrictions pose a problem, use
  the |markings| library instead.

\begin{codeexample}[]
\begin{tikzpicture}[decoration={shape backgrounds,shape=star,shape size=5pt}]
  \draw [help lines] grid (3,2);
  \draw [decorate] (0,0) -- (3,1) arc (0:180:1.5 and 1);
\end{tikzpicture}
\end{codeexample}

\begin{codeexample}[]
\tikzset{paint/.style={ draw=#1!50!black, fill=#1!50 },
         decorate with/.style=
           {decorate,decoration={shape backgrounds,shape=#1,shape size=2mm}}}
\begin{tikzpicture}
  \draw [decorate with=dart,      paint=red]    (0,1.5) -- (3,1.5);
  \draw [decorate with=diamond,   paint=green]  (0,1)   -- (3,1);
  \draw [decorate with=rectangle, paint=blue]   (0,0.5) -- (3,0.5);
  \draw [decorate with=circle,    paint=yellow] (0,0)   -- (3,0);
\end{tikzpicture}
\end{codeexample}

  All shapes are positioned by the anchor that is specified via the
  |anchor| decoration option:

  \begin{key}{/pgf/decoration/anchor=\meta{anchor} (initially center)}
    The anchor used to position the shape backgrounds.
  \end{key}

  A shape background path is added at the start point of the path and,
  if the distance between the shapes is appropriate, at the end point
  of the path.
  
\begin{codeexample}[]
\begin{tikzpicture}[decoration={
      shape backgrounds,shape=regular polygon,shape size=4mm}]
  \draw [help lines] grid (3,2);
  \draw [thick] (0,0) -- (2,2) (1,0) -- (3,0);
  \draw [red, decorate, decoration={shape sep=.5cm}]  (1,0) -- (3,0);
  \draw [blue, decorate, decoration={shape sep=.5cm}] (0,0) -- (2,2);
\end{tikzpicture}
\end{codeexample}

  Keys for customizing specific shapes can be specified (e.g.,
  |star points|, |cloud puffs|, |kite angles|, and so on). The size of
  the shape is ``enforced'' using transformations. This means that the
  shape is typeset with an empty text box and some default size
  values, resulting in an initial shape. This shape is then rescaled
  using coordinate transformations so that it has the desired size
  (which may vary as we travel along the to-be-decorated path). This
  means that settings involving angles and distances may not appear
  entirely accurate. More general options such as |inner sep| and
  |minimum size| will be ignored,  but transformations can be applied
  to each segment as described below.

\begin{codeexample}[]
\tikzset{
  paint/.style={draw=#1!50!black, fill=#1!50},
  my star/.style={decorate,decoration={shape backgrounds,shape=star},
                  star points=#1}
}
\begin{tikzpicture}[decoration={shape sep=.5cm, shape size=.5cm}]
  \draw [my star=9, paint=red]                            (0,1.5) -- (3,1.5);
  \draw [my star=5, paint=blue]                           (0,.75) -- (3,.75);
  \draw [my star=5, paint=yellow, shape border rotate=30] (0,0) -- (3,0);
\end{tikzpicture}
\end{codeexample}

  There are various keys to control the drawing of the shape
  decoration.

  \begin{key}{/pgf/decoration/shape=\meta{shape name} (initially circle)}
    The shape whose background path is used.
  \end{key}

  \begin{key}{/pgf/decoration/shape sep=\meta{spacing} (initially {.25cm, between centers})}
    Set the spacing between the shapes on the decorations path. This can be
    just a distance on its own, but the additional keywords
    |between centers|, and |between borders| (which must be preceded by a
    comma), specify that the distance  is between the center anchors of
    the shapes or between the edges of the \emph{boundaries} of
    the shape borders.
  
\begin{codeexample}[]
\begin{tikzpicture}[
    decoration={shape backgrounds,shape size=.5cm,shape=signal},
    signal from=west, signal to=east,
    paint/.style={decorate, draw=#1!50!black, fill=#1!50}]
  \draw [help lines] grid (3,2);
  \draw [paint=red, decoration={shape sep=.5cm}]
    (0,2) -- (3,2);
  \draw [paint=green, decoration={shape sep={1cm, between centers}}]
    (0,1) -- (3,1);
  \draw [paint=blue, decoration={shape sep={1cm, between borders}}]
    (0,0) -- (3,0);
\end{tikzpicture}
\end{codeexample}
  \end{key}

  \begin{key}{/pgf/decoration/shape evenly spread=\meta{number}}
    This key overrides the |shape sep| key and forces the decoration to
    fit \meta{number} shapes evenly across the path.
    If \meta{number} is less than |1|, then no shapes will be used.
    If \meta{number} equals |1|, then one shape is put in the middle
    of the path.
    The additional keywords |by centers| (the default, if no keyword is
    specified) and |by borders| can be used (both preceded by a comma),
    to specify how the distance between shapes is determined. These
    keywords will only have a noticeable effect if the shapes sizes
    differ over time.

\begin{codeexample}[]
\tikzset{
  paint/.style={draw=#1!50!black, fill=#1!50},
  spreading/.style={
    decorate,decoration={shape backgrounds, shape=rectangle,
    shape start size=4mm,shape end size=1mm,shape evenly spread={#1}}}
}
\begin{tikzpicture}
  \fill [paint=green,spreading={5, by borders},
         decoration={shape scaled}]            (0,2)   -- (3,2);
  \fill [paint=blue,spreading={5, by centers},
         decoration={shape scaled}]            (0,1.5) -- (3,1.5);
  \fill [paint=red,    spreading=5]            (0,1)   -- (3,1);
  \fill [paint=orange, spreading=4]            (0,.5)  -- (3,.5);
  \fill [paint=gray,    spreading=1]            (0,0)   -- (3,0);
\end{tikzpicture}
\end{codeexample}
  \end{key}

  \begin{key}{/pgf/decoration/shape sloped=\opt{\meta{boolean}} (initially true)}
    By default, shapes are rotated to the slope of the decorations path. If
    \meta{boolean} is the value |false|, then this rotation is turned
    off. Internally this sets the \TeX-if |\ifpgfshapedecorationsloped|
    accordingly.

\begin{codeexample}[]
\tikzset{
  paint/.style={draw=#1!50!black, fill=#1!50}
}
\begin{tikzpicture}[decoration={
    shape width=.65cm, shape height=.45cm,
    shape=isosceles triangle, shape sep=.75cm,
    shape backgrounds}]
  \draw [help lines] grid (3,2);
  \draw [paint=red,decorate] (0,0) -- (2,2);
  \draw [paint=blue,decorate,decoration={shape sloped=false}]
                             (1,0) -- (3,2);
\end{tikzpicture}
\end{codeexample}
  \end{key}%

  It is possible to scale the width and height of the shapes along the
  length of the decorations path. The shapes are scaled between the starting
  size and the ending size. The following keys customize the way the
  decoration shapes are scaled:
  
  \begin{key}{/pgf/decoration/shape scaled=\meta{boolean} (initially false)}
\begin{codeexample}[]
\tikzset{
  bigger/.style={decoration={shape start size=.125cm, shape end size=.5cm}},
  smaller/.style={decoration={shape start size=.5cm, shape end size=.125cm}},
  decoration={shape backgrounds,
              shape sep={.25cm, between borders},shape scaled}
}
\begin{tikzpicture}
  \draw [help lines] grid (3,2);
  \fill [decorate, bigger, red!50]   (0,1) -- (3,2);
  \fill [decorate, smaller, blue!50] (0,0) -- (3,1);
\end{tikzpicture}
\end{codeexample}

    If this key is set to false (which is the default), then only the
    start width and height are used. Note that the keys |shape width|
    and |shape height| set the start and end height simultaneously.
  \end{key}

  \begin{key}{/pgf/decoration/shape start width=\meta{length} (initially 2.5pt)}
    The starting width of the shape.
  \end{key}%

  \begin{key}{/pgf/decoration/shape start height=\meta{length} (initially 2.5pt)}
    The starting height of the shape.
  \end{key}%

  \begin{stylekey}{/pgf/decoration/shape start size=\meta{length}}
    Sets both the start height and start width simultaneously.
  \end{stylekey}%

  \begin{key}{/pgf/decoration/shape end width=\meta{length} (initially 2.5pt)}
    The recommended ending width of the shape. Note that this is the
    width that a shape will take only if it is drawn exactly at the end
    of the path.


\begin{codeexample}[]
\tikzset{
  bigger/.style={decoration={shape start size=.25cm, shape end size=1cm}},
  smaller/.style={decoration={shape start size=1cm, shape end size=.25cm}},
  decoration={shape backgrounds,
              shape sep={.25cm, between borders},shape scaled}
}
\begin{tikzpicture}
  \draw [help lines] grid (3,2);
  \fill [decorate,bigger,
         decoration={shape sep={.25cm, between borders}}, blue!50]
    (0,1.5) -- (3,1.5);
  \fill [decorate,smaller,
         decoration={shape sep={1cm, between centers}},   red!50]
    (0,.5)  -- (3,.5);
  \draw [gray, dotted] (0,1.625) -- (3,2)    (0,1.375) -- (3,1)
                       (0,1)     -- (3,.625) (0,0)     -- (3,.375);
\end{tikzpicture}
\end{codeexample}
  \end{key}%

  \begin{key}{/pgf/decoration/shape end height=\meta{length}}
    The recommended ending height of the shape.
  \end{key}%

  \begin{stylekey}{/pgf/decoration/shape end size=\meta{length}}
    Set both the end height and end width simultaneously.
  \end{stylekey}
\end{decoration}





\subsection{Text Decorations}

\begin{pgflibrary}{decorations.text}
  The decoration in this library decorates the path with some
  text. This can be used to draw text that follows a curve.
\end{pgflibrary}

\begin{decoration}{text along path}
  This decoration decorates the path with text. This drawing of the
  text is a ``side effect'' of the decoration. The to-be-decorated
  path is only used to determine where the characters should be put
  and it is thrown away after the decoration is done. This is why no line is shown in
  the following example.
  
\begin{codeexample}[]
\catcode`\|12
\begin{tikzpicture}[decoration={text along path,
    text={Some long text along a ridiculously long curve that}}]
  \draw [help lines] grid (3,2);
  \draw [decorate] (0,0) -- (3,1) arc (0:180:1.5 and 1);
\end{tikzpicture}
\end{codeexample}

  \pgfname{} ``does its best'' to typeset the text, however you
  should note the following points:
  \begin{itemize}
  \item
    Each character in the text is typeset in a separate |\hbox|. This
    means that if you want fancy things like kerning or ligatures you
    will have to manually annotate the characters in the decoration
    text within a group, for example, |W{\kern-1ptA}TER|.
  \item
    Each character is positioned using the center of its baseline. To
    move the text vertically (relative to the path), the additional
    transform key should be used.
  \item
    No attempt is made to ensure characters do not overlap when
    the angle between segments is considerably less than 180$^\circ$
    (this is tricky to do in \TeX{} without a huge processing
    overhead). In general this should not be too much of a problem,
    but, once again, kerning can be used in most cases to overcome
    any undesirable effects.
  \item      
    It is only possible to typeset text in math mode under considerable
    restrictions. Math mode is entered and exited using any character  
    of category code 3 (e.g., in plain \TeX{} this is |$|). %$
    Math subscripts and superscripts need to be  contained within braces
    (e.g., |{^y_i}|) as do commands like |\times| or |\cdot|.
    However, even modestly complex mathematical  typesetting is unlikely
    to be successful along a path (or even desirable).
  \item
    Some inaccuracies in positioning may be particularly apparent
    at input segment boundaries. This can (unfortunately) only be solved
    on a case-by-case basis by individually kerning the offending
    characters within a group.
  \end{itemize}

  The following keys are used by the |text| decoration:
  \begin{key}{/pgf/decoration/text=\meta{text}
      (initially \normalfont empty)}
    Sets the text to typeset along the curve.
    Consecutive spaces are ignored, so |\ | (or |\space| in \LaTeX)
    should be used to insert multiple spaces.  It is possible to
    format the text using normal formatting commands, such as |\it|, |\bf|
    and |\color|, within customizable delimiters. Initially these
    delimiters are both {\tt\char`\|} (however, care will be needed
    regarding  the category codes of delimiters --- see below).

{\catcode`\|12
\begin{codeexample}[]
\catcode`\|12
\begin{tikzpicture}
  \draw [help lines] grid (3,2);
  \path [decorate,decoration={text along path,
           text={a big |\color{green}|green|| juicy apple.}}]
    (0,0) .. controls (0,2) and (3,0) .. (3,2);
\end{tikzpicture}
\end{codeexample}
}
    By following the first delimiter
    with |+|, the formatting commands are added to any existing
    formatting.

{\catcode`\|12
\begin{codeexample}[]
\begin{tikzpicture}
  \draw [help lines] grid (3,2);
  \path [decorate,decoration={text along path,
           text={a |\large|big |+\bf\color{red}|red|| juicy apple.}}]
    (0,0) .. controls (0,2) and (3,0) .. (3,2);
\end{tikzpicture}
\end{codeexample}
}
  
    Internally, the text is stored in the macro |\pgfdecorationtext|.
    Any characters that have not been typeset when the end of the
    path has been reached will be stored in |\pgfdecorationrestoftext|.

  \end{key}

{\catcode`\|12
  \begin{key}{/pgf/decoration/text format delimiters=\marg{before}\marg{after} (initially \char`\{|\char`\}\char`\{\char`\})}

    \catcode`\|13
  
    Set the characters that the text decoration will use to parse
    formatting commands.
    If \meta{after} is empty, then \meta{before} will be used for both
    delimiters.
    In general you should stick to characters  whose category codes are
    |11| or |12|.
    As |+| is used to indicate that the specified format commands
    are added  to any existing ones, you should avoid using |+| as
    a delimiter.

\begin{codeexample}[]
\begin{tikzpicture}
  \draw [help lines] grid (3,2);
  \path [decorate, decoration={text along path,text format delimiters={[}{]},
           text={A big [\color{red}]red[] and [\color{green}]green[] apple.}}]
    (0,0) .. controls (0,2) and (3,0) .. (3,2);
\end{tikzpicture}
\end{codeexample}
  \end{key}
}

  \begin{key}{/pgf/decoration/text color=\meta{color}  (initially black)}
    The color of the text.
  \end{key}


\begin{key}{/pgf/decoration/reverse path=\meta{boolean} (initially false)}
  This key reverses the path. This is especially useful for typesetting
  text along different sides of curves.

\begin{codeexample}[]
\begin{tikzpicture}
  \draw [help lines] grid (3,2);
  \draw [gray, ->]
    [postaction={decoration={text along path,
      text={a big juicy apple}, text color=red}, decorate}]
    [postaction={decoration={text along path,
      text={a big juicy apple}, text color=blue, reverse path}, decorate}]
    (3,0) .. controls (3,2) and (0,2) .. (0,0);
\end{tikzpicture}
\end{codeexample}

\end{key}

\begin{key}{/pgf/decoration/text align={\ttfamily\char`\{}\meta{alignment options}{\ttfamily\char`\}}}
  This changes the key path to |/pgf/decoration/text align| and
  executes \meta{alignment options}.
\end{key}

\begin{key}{/pgf/decoration/text align/align=\meta{alignment} (initially left)}
  Aligns the text according to \meta{alignment}, which should
  be one of |left|, |right|, or |center|.

\begin{codeexample}[]
\begin{tikzpicture}
  \draw [help lines] grid (3,2);
  \draw [red, dashed]
    [postaction={decoration={text along path, text={a big juicy apple},
      text align={align=right}}, decorate}]
    (0,0) .. controls (0,2) and (3,2) .. (3,0);
\end{tikzpicture}
\end{codeexample}

\end{key}

\begin{stylekey}{/pgf/decoration/text align/left}
  Aligns the text to the left end of the path.

\end{stylekey}

\begin{stylekey}{/pgf/decoration/text align/right}
  Aligns the text to the right end of the path.
\end{stylekey}

\begin{stylekey}{/pgf/decoration/text align/center}
  Aligns the text to the center of the path.
\end{stylekey}

\begin{key}{/pgf/decoration/text align/left indent=\meta{length} (initially 0pt)}
  Specifies a distance which the automaton should move along
  before it starts typesetting the text.
\end{key}

\begin{key}{/pgf/decoration/text align/right indent=\meta{length} (initially 0pt)}
  Specifies a distance before the end of the path, where
  the automaton should stop typesetting the text.
\end{key}

\begin{key}{/pgf/decoration/text align/fit to path=\meta{boolean} (initially false)}
  This key makes the decoration automaton try to fit the text
  to the length of the path. The automaton shifts forward
  by a small amount between each character in order to fit the
  text to the path. If, however, the length of the text is longer
  than the length of the path (i.e., the automaton would have to
  shift \emph{backwards} between characters) this key will
  have no effect.

\begin{codeexample}[]
\begin{tikzpicture}
  \draw [help lines] grid (3,2);
  \draw [red, dashed]
    [postaction={decoration={text along path, text={a big juicy apple},
      text align=fit to path}, decorate}]
    (0,0) .. controls (0,2) and (3,2) .. (3,0);
\end{tikzpicture}
\end{codeexample}

\end{key}

\begin{key}{/pgf/decoration/text align/fit to path stretching spaces=\meta{boolean} (initially false)}
  This key works like the previous key except the automaton
  shifts forward only for space characters (including |\space|, but
  \emph{excluding} |\ |).

\begin{codeexample}[]
\begin{tikzpicture}
  \draw [help lines] grid (3,2);
  \draw [red, dashed]
    [postaction={decoration={text along path, text={a big juicy apple},
      text align={fit to path stretching spaces}}, decorate}]
    (0,0) .. controls (0,2) and (3,2) .. (3,0);
\end{tikzpicture}
\end{codeexample}
\end{key}

\end{decoration}




\begin{decoration}{text effects along path}

  This decoration is similar to the |text along path| decoration
  except that each character is inserted into the picture
  as a \tikzname\ node, and node options (such as |text|, |scale| and |opacity|)
  can be used to create `text effects'.

\begin{codeexample}[]
\bfseries\large
\begin{tikzpicture}[decoration={text effects along path, 
  text={text effects along path!}, text align=center,
  text effects/.cd,
    character count=\i, character total=\n,
    characters={evaluate={\c=\i/\n*100;}, text along path, text=red!\c!orange},
    character widths={text along path, xslant=0, yscale=1}}]

\path [postaction={decorate}, preaction={decorate, 
  text effects={characters/.append={yscale=-1.5, opacity=0.5, 
    text=gray, xslant=(\i/\n-0.5)*3}}}] 
   (0,0) .. controls ++(2,1) and ++(-2,-1) .. (3,4);
\end{tikzpicture}
\end{codeexample}


  There are some important differences between this decoration and the
  |text along path| decoration:
  
  \begin{itemize}
  \item 
    formatting (e.g., font and color)
  	cannot be specified in the decoration text. They can only be specified
  	using the keys described below.  	
 	\item
 	  as a consequence of using the \tikzname\ node options, this
  	decoration is only available in \tikzname.
  \item
  	due to the number of computations involved, this is
  	quite a slow decoration.
  	
  \end{itemize}
 
  The following keys are shared with the |text along path|
  decoration:
  
\begin{key}{/pgf/decoration/text=\marg{text}}


  Set the text this decoration will use. Braces can be
  used to group multiple characters together, 
  or commands that should not be expanded until they are typset, for example
  |gr{\"o}{\ss}eren|. You should \emph{not} use the formatting 
  delimiters or math mode characters 
  that the |text along path| decoration  supports.
  
\end{key}



\begin{key}{/pgf/decoration/text align=\meta{align}}

  This sets the alignment of the text along the path.
  The \meta{align} argument should be |left|, |right| or |center|.
  Spreading the text out, or stretching the spaces between
  words is \emph{not} supported.
  
\end{key}

	The decoration text can be thought of as consisting
	of \emph{characters} arranged in to sequences of \emph{letters}
	to make \emph{words} which are separated by a \emph{word separator}.
	This, however, does not mean that you are limited to using only
	natural language as the decoration text.

\begin{codeexample}[]
\begin{tikzpicture}[decoration={text effects along path,
  text={000-001-010-011-100-101-110-111},
  text effects/.cd,
    path from text,
    word separator=-,
    every letter/.style={shape=rectangle, fill=blue!20, draw=blue!40}}]
    
\path [decorate] (0,0);
\end{tikzpicture}
\end{codeexample}
	
	 In addition, it is possible to replace characters
	 with \tikzname\ code:
	 
\begin{codeexample}[]
\begin{tikzpicture}[decoration={text effects along path,
  text={000-001-010-011-100-101-110-111}, text align=center,
  text effects/.cd,
    word separator=-,
    replace characters=0 with {\fill [purple] circle [radius=2pt]; },
    replace characters=1 with {\fill [orange] circle [radius=2pt]; },
    replace characters=- with {\path circle [radius=2pt]; },
    every letter/.style={shape=rectangle, fill=blue!20, draw=blue!40}}]
    
\path [decorate] (0,0) .. controls ++(2,0) and ++(-2,0) .. (3,4);
\end{tikzpicture}
\end{codeexample}

	 
  There are many keys and styles that can be used to add effects
	 to the decoration text.
	 Many of these keys have the parent path |/pgf/decoration/text effects/|,
	 but for convenience, these keys can be accessed using the following
	 key:
	 
\begin{key}{/tikz/text effects=\marg{options}}

  Execute every option in \marg{options} with the
  key path for each option temporarily set to  |/pgf/decoration/text effects/|.
  
\end{key}


  The following keys can be used to customise the
  appearance of text in the |text effects along path|
  decoration.
	 



\begin{stylekey}{/pgf/decoration/text effects/every character}

	Set the effects that will be applied to 
	every character in the decoration text. The effects
	will typically be \tikzname\ node options.
	Initially, this style is empty so the decoration simply positions
	nodes at the appropriate position along the path. In order to
	make the text `follow the path' like the |text along path| decoration
	the following key can be added to the |every character| style. 
	
\end{stylekey}


\begin{stylekey}{/pgf/decoration/text effects/text along path}

	This style automatically sets the \tikzname\ keys 
	|transform shape| (to make the character slope with the path),
	|anchor=baseline| (to make the baseline of the characters `sit' on 
	the path) and |inner xsep=0pt| (to horizontally fit each node to the character
	it contains, reducing the spacing between characters).
	
\begin{codeexample}[]
\begin{tikzpicture}[decoration={text effects along path,
	text={text effects along path!}}]
	
\path [draw=red, dotted, postaction={decorate}] 
  (0,0) .. controls ++(1,0) and ++(-1,0) .. (3,2);
\path [draw=blue, dotted, yshift=1cm, postaction={decorate},
  text effects={text along path}] 
  (0,0) .. controls ++(1,0) and ++(-1,0) .. (3,2);
\end{tikzpicture}
\end{codeexample}
\end{stylekey}


\begin{key}{/pgf/decoration/text effects/characters=\marg{effects}}

	Shorthand for the |every character|.
	
\end{key}

\begin{stylekey}{/pgf/decoration/text effects/character \meta{number}}

  Specify additional effects for the character \meta{number}.
  
\end{stylekey}


\begin{stylekey}{/pgf/decoration/text effects/every letter}

	Specify additional effects for every letter (i.e., every character
	that isn't the word separator) in the decoration text.
	
\end{stylekey}

\begin{stylekey}{/pgf/decoration/text effects/letter \meta{number}}

  Specify the effects for letter \meta{number} in \emph{every} word.
  
\end{stylekey}


\begin{stylekey}{/pgf/decoration/text effects/every first letter}

   Specify additional effects for the first letter in \emph{every} word.
   
\end{stylekey}


\begin{stylekey}{/pgf/decoration/text effects/every last letter}

   Specify additional effects for the last letter in \emph{every} word.
   
\end{stylekey}


\begin{stylekey}{/pgf/decoration/text effects/every word}

   Specify additional effects for every word in the decoration text.
   
\end{stylekey}



\begin{stylekey}{/pgf/decoration/text effects/word \meta{number}}

  Specify additional effects for word \meta{number} in the decoration text.
  
\end{stylekey}



\begin{stylekey}{/pgf/decoration/text effects/word \meta{m} letter \meta{n}}

  Specify additional effects for letter \meta{n} in word \meta{m} in the decoration text.
  
\end{stylekey}



\begin{stylekey}{/pgf/decoration/text effects/every word separator}

  Specify additional effects for every character that is a word separator.
  
\end{stylekey}



\begin{key}{/pgf/decoration/text effects/word separator=\meta{character} (initially space)}

  Specify the character that is to be used as the word separator.
  This \emph{must} be a single character such as |a| or |-| or
  the special value |space| (which should be used to indicate that
  spaces should be used as the separator).
  
\end{key}

	By default, the width for each character is calculated according
	to the bounding box of the node in which it is contained.
	However, if the node is rotated or slanted, or has a substantial
	|inner sep|, this bounding box
	will be quite big. The following key enables different effects
	to be applied to the node that is used to calculate the 
	width.




\begin{stylekey}{/pgf/decoration/text effects/every character width}

	This style is appplied to the (invisible) nodes used for calculating
	the width of a character node.
	
\end{stylekey}



\begin{key}{/pgf/decoration/text effects/character widths=\marg{effects}}

  Shorthand for the |every character width| style.

\begin{codeexample}[]
\begin{tikzpicture}[decoration={text effects along path, 
  text={text effects along path!}, text align=center,
  text effects/.cd,
  	character count=\i,
    characters={xslant=0.5, text along path, name=c-\i}}]

\path [decorate] (0,0) -- (3,2);
\path [decorate, 
  text effects={character widths={inner xsep=0pt, xslant=0}}]
  (0,1) -- (3,3);
\end{tikzpicture}
\end{codeexample}
\end{key}

  It is possible to parametrize effects, perhaps for doing
  calculations, or labelling nodes based on the number of the
  character in the decoration text. To access the number
  of the character, and the total number of characters
  the following keys can be used. 
  However, these keys should \emph{not}
  be used inside the style keys given above.
   
\begin{key}{/pgf/decoration/text effects/character count=\meta{macro}}
	Store the number of the character being typeset in \meta{macro}.
	
\begin{codeexample}[]
\begin{tikzpicture}[decoration={text effects along path,
  text={text effects along path!}, 
  text effects/.cd,
    path from text, 
    character count=\i, every word separator/.style={fill=red!30},
    characters={text along path, shape=circle, fill=gray!50}}]
    
\path [decorate, text effects={characters/.append={label=above:\footnotesize\i}}] (0,0);
\end{tikzpicture}
\end{codeexample}
	
\end{key}

\begin{key}{/pgf/decoration/text effects/character total=\meta{macro}}
	Store the total number of the characters in the decoration text
	in \meta{macro}.
	This key can be used with the |character count| key to produce
	some quite pleasing effects:
	
\begin{codeexample}[]
\begin{tikzpicture}[decoration={text effects along path,
  text={text effects along path!}, 
  text effects/.cd,
    character count=\i, character total=\n, 
    characters={text along path, evaluate={\c=\i/\n*100;},
      text=orange!\c!blue, scale=\i/\n+0.5}}]
      
\path [decorate] 
  (0,0) .. controls ++(1,0) and ++(-1,0) .. (3,2);
\end{tikzpicture}
\end{codeexample}

\end{key}

\begin{key}{/pgf/decoration/text effects/letter count=\meta{macro}}

  Store the number of letter being typeset (i.e., the position of the character
  in the word) in \meta{macro}. Numbering starts at |1| and
  the character acting as a word separator is numbered |0|.
  
\begin{codeexample}[]
\begin{tikzpicture}[decoration={text effects along path,
  text={text effects along path!}, 
  text effects/.cd,
    path from text, letter count=\i, every word separator/.style={fill=red!30},
    characters={text along path, shape=circle, fill=gray!50}}]
    
\path [decorate, text effects={characters/.append={label=above:\footnotesize\i}}] (0,0);
\end{tikzpicture}
\end{codeexample}

\end{key}

\begin{key}{/pgf/decoration/text/effetcs/letter total=\meta{macro}}

  Store the number of letters in the current word in \meta{macro}.
  When the character is the word separator, this value is |0|.

\end{key}

\begin{key}{/pgf/decoration/text effects/word count=\meta{macro}}
   Store the number of words in the decoration text in	\meta{macro}. 
   Numbering starts at |1|.
 		When the character is the word separator, \meta{macro}
 		takes the number of the previous word. If the decoration text
 		starts with a word separator \meta{macro} will be |0|.

\begin{codeexample}[]
\begin{tikzpicture}[decoration={text effects along path,
  text={text effects along path!},  
  text effects/.cd,
    path from text, word count=\i, every word separator/.style={fill=red!30},
    characters={text along path, shape=circle, fill=gray!50}}]
    
 \path [decorate, text effects={characters/.append={label=above:\footnotesize\i}}] (0,0);
\end{tikzpicture}
\end{codeexample}
\end{key}

\begin{key}{/pgf/decoration/text effects/word total=\meta{macro}}

  Store the total number of words in the decoration text in \meta{macro}.

\end{key}


  It is also possible to apply effects to specific
  characters such as coloring every instance of the
  character |a|,
  or changing the font of every |T| in the decoration text:
  
\begin{key}{/pgf/decoration/text effects/style characters=\marg{characters} with \marg{effects}} 

  This key enables \meta{effects} to be applied to every character
  in the decoration text that is specified in \meta{characters}.

\begin{codeexample}[]
\begin{tikzpicture}[decoration={text effects along path,  
  text={Falsches {\"U}ben von Xylophonmusik qu{\"a}lt jeden gr{\"o}{\ss}eren Zwerg},
  text effects/.cd,
    path from text,
    style characters=aeiou{\"U}{\"a}{\"o} with {text=blue},
    characters={text along path}}]

\path [decorate] (0,0);
\end{tikzpicture}
\end{codeexample}

\end{key}

\begin{key}{/pgf/decoration/text effects/path from text=\opt{\marg{true or false}} (default true)}
  When this key is set to |true| and the decorated path
  consists only of a single point, the decoration will calculate
  the width of the decoration text using all the specified parameters
  as if the decorated path was actually a straight line
  starting from the given point. This `virtual'
  straight line is then decorated with the text.
  
\begin{codeexample}[]
\begin{tikzpicture}[decoration={text effects along path,
  text={text effects along path!}, 
  text effects/.cd,
    path from text,
    character count=\i, character total=\n,
    characters={text along path, scale=\i/\n+0.5}}]
    
\path [decorate] (0,0);
\end{tikzpicture}
\end{codeexample}

\end{key}



\begin{key}{/pgf/decoration/text effects/path from text angle=\meta{angle}}

  When used in conjunction with the |path from text| key,
  the straight line that is used as the decorated path
  is rotated by \meta{angle} around the starting point.
  
\begin{codeexample}[]
\begin{tikzpicture}[decoration={text effects along path,
  text={text effects along path!}, 
  text effects/.cd,
    path from text, path from text angle=60,
    character count=\i, character total=\n,
    characters={text along path, scale=\i/\n+0.5}}]
    
\path [decorate] (0,0);
\end{tikzpicture}
\end{codeexample}

\end{key}


\begin{key}{/pgf/decoration/text effects/fit text to path=\opt{\meta{true or false}} (default true)}
  
  This key will make the decoration increase the space between
  characters so that the entire path is used by the decoration.
  
\begin{codeexample}[]
\begin{tikzpicture}[decoration={text effects along path,
  text={text effects along path!},
  text effects/every character/.style={text along path}}]
    
\path [draw=gray, postaction={decorate}, rotate=90] 
  (0,0) .. controls ++(2,0) and ++(-1,0) .. (5,-1);
\path [draw=gray, postaction={decorate}, rotate=90, yshift=-1cm,
  text effects={fit text to path}] 
  (0,0) .. controls ++(2,0) and ++(-1,0) .. (5,-1);
\end{tikzpicture}
\end{codeexample}
\end{key}


\begin{key}{/pgf/decoration/text effects/scale text to path=\opt{\meta{true or false}} (default true)}

 This key will make the decoration scale the text
  so that the entire path is used by the decoration.
  
\begin{codeexample}[]
\begin{tikzpicture}[decoration={text effects along path,
  text={text effects along path!},
  text effects/every character/.style={text along path}}]
    
\path [draw=gray, postaction={decorate}, rotate=90] 
  (0,0) .. controls ++(2,0) and ++(-1,0) .. (5,-1);
\path [draw=gray, postaction={decorate}, rotate=90, yshift=-1cm,
  text effects={scale text to path}] 
  (0,0) .. controls ++(2,0) and ++(-1,0) .. (5,-1);
\end{tikzpicture}
\end{codeexample}
\end{key}


\begin{key}{/pgf/decoration/text effects/reverse text}

  Reverse the order of the characters in the decoration text.
  This may be useful if using `right-to-left` languages. 
  Unfortunately, any
  leading `soft' spaces in the original text will be lost.
  
\begin{codeexample}[]
\begin{tikzpicture}[decoration={text effects along path,
  text={text effects along path!}, 
  text effects/.cd,
    path from text, path from text angle=60,
    reverse text,
    character count=\i, character total=\n,
    characters={text along path, scale=\i/\n+0.5}}]
    
\path [decorate] (0,0) .. controls ++(1,0) and ++(-1,0) .. (3,2);
\end{tikzpicture}
\end{codeexample}

  It is important to note that the |reverse text| key reverses the
  text \emph{before} doing anything else. This means that the
  numbering of characters, letters and words will still
  still be in the normal order, so any parametrized effects
  will have to take this into account.
  Alternatively, to get the numbering 
  to follow the reversed text, it is possible to reverse the path and then
  invert the scale:
  
\begin{codeexample}[]

\begin{tikzpicture}[decoration={text effects along path,
  text={text effects along path!}, 
  text effects/.cd,
    path from text, path from text angle=60,
    character count=\i, character total=\n,
    characters={text along path, scale=\i/\n+0.5}}]

\path [decorate, text effects={reverse text}] (0,0);
\path [blue, decorate, decoration={reverse path},
    text effects={characters/.append={scale=-1}}] (1,0);
\end{tikzpicture}
\end{codeexample}
\end{key}

\begin{key}{/pgf/decoration/text effects/group letters}

  Group sequences of letters together so they are treated
  as a single `character'.

\begin{codeexample}[]
\begin{tikzpicture}[decoration={text effects along path,
  text={text effects along path!}, 
  text effects/.cd,
    path from text, path from text angle=60,
    every word separator/.style={fill=none},
    character count=\i, character total=\n,
    characters={text along path, fill=gray!50, scale=\i/\n+0.5}}]
    
\path [decorate] (0,0);
\path [decorate, text effects={group letters,
  characters/.append={fill=red!20}}]
  (1,0);
\end{tikzpicture}
\end{codeexample}

\end{key}

  The order in which the |reverse text| and |group letters| keys 
  are applied is important:
  
\begin{codeexample}[]
\begin{tikzpicture}[decoration={text effects along path,
  text={text effects along path!}, 
  text effects/.cd,
    path from text, path from text angle=60,
    every word separator/.style={fill=none},
    character count=\i, character total=\n,
    characters={text along path, fill=gray!50, scale=\i/\n+0.5}}]
    
\path [decorate, text effects={reverse text, group letters}] (0,0);
\path [decorate, text effects={group letters, reverse text,
  characters/.append={fill=red!20}}] (1,0);
\end{tikzpicture}
\end{codeexample}

\begin{key}{/pgf/decoration/text effects/repeat text=\opt{\meta{times}}}

  Usually, when the decoration runs out of text, it simply stops.
  This key will make the decoration repeat the decoration text
  for the specified number of \meta{times}. If no value is given
  the text will be repeated until the path is finished. 
  There are two points to remember however.
  Firstly the numbering of characters, letters and words
  will be restarted each time the text is repeated.
  Secondly, the options for alignment, scaling or fitting
  the text to the path, fitting the path to the text, 
  and so on, are computed using the decoration text before the
  decoration starts. If any of these options are given
  the behaviour of the |repeat text| key is undefined, but
  typically it will be ignored.

\begin{codeexample}[]
\begin{tikzpicture}[decoration={text effects along path,
  text={text effects along path!\ },
  text effects/.cd,
    repeat text,
    character count=\m, character total=\n,
    characters={text along path, scale=0.5+\m/\n/2}}]
    
\path [draw=gray, ultra thin, postaction=decorate] 
  (180:2) \foreach \a in {0,...,12}{ arc (180-\a*90:90-\a*90:1.5-\a/10) };
\end{tikzpicture}
\end{codeexample}
\end{key}

\begin{key}{/pgf/decoration/text effects/character command=\meta{macro}}
  
  This key specifies a command that is executed when
  each character is placed in the node. The \meta{macro} should
  be an ordinary \TeX\ macro which takes one argument. The
  argument will be a macro which when expanded will
  contain the current character.
  
\begin{codeexample}[]
\def\mycommand#1{#1$_\n$}
\begin{tikzpicture}[decoration={text effects along path,
  text={text effects along path!}, 
  text effects/.cd,
    path from text, path from text angle=60, group letters,
    word count=\n,
    every word/.style={character command=\mycommand},
    characters={text along path}}]
    
\path [decorate] (0,0);
\end{tikzpicture}
\end{codeexample}
\end{key}


\begin{key}{/pgf/decoration/text effects/replace characters=\meta{characters} with \marg{code}}

  Replace the node for each character in \meta{characters} with \meta{code}. 
  The \meta{code} can be thought of as describing
  a little picture or marking which will be
  used instead of the character node. 
  The origin will be the current point along the 
  decoration path.
  Any transformations associated with
  the \meta{characters} (e.g., applied with the |every character|
  or |every letter| styles) will also be applied to \meta{code}.
  
\begin{codeexample}[]
\begin{tikzpicture}[decoration={text effects along path,
  text={text effects along path!}, 
  text effects/.cd,
    path from text, path from text angle=60,
    replace characters=e with {\fill [red!20]   (0,1mm) circle [radius=1mm];},
    replace characters=a with {\fill [black!20] (0,1mm) circle [radius=1mm];},
    character count=\i, character total=\n,
    characters={text along path, scale=\i/\n+0.5}}]
    
\path [decorate] (0,0);
\end{tikzpicture}
\end{codeexample}

\end{key}
\end{decoration}



\subsection{Fractal Decorations}

\begin{pgflibrary}{decorations.fractals}
  The decorations of this library can be used to create fractal
  lines. To use them, you typically have to apply the decoration
  repeatedly to an originally straight path.
\end{pgflibrary}


\begin{decoration}{Koch curve type 1}
  This decoration replaces a straight line by a ``rectangular bump.''
  By repeatedly applying this replacement, different levels of the
  Koch curve fractal can be created. Its Hausdorff dimension is $\log
  5/\log 3$.
\begin{codeexample}[]
\begin{tikzpicture}[decoration=Koch curve type 1]
  \draw decorate{ (0,0) -- (3,0) };
  \draw decorate{ decorate{ (0,-1.5) -- (3,-1.5) }};
  \draw decorate{ decorate{ decorate{ (0,-3) -- (3,-3) }}};
\end{tikzpicture}
\end{codeexample}
\end{decoration}


\begin{decoration}{Koch curve type 2}
  This decoration replaces a straight line by a ``rectangular sine.''
  Its Hausdorff dimension is $3/2$.
\begin{codeexample}[]
\begin{tikzpicture}[decoration=Koch curve type 2]
  \draw decorate{ (0,0) -- (3,0) };
  \draw decorate{ decorate{ (0,-2) -- (3,-2) }};
  \draw decorate{ decorate{ decorate{ (0,-4) -- (3,-4) }}};
\end{tikzpicture}
\end{codeexample}
\end{decoration}

\begin{decoration}{Koch snowflake}
  This decoration replaces a straight line by a ``line with a spike.''
  The Hausdorff dimension of Koch's snowflake's is $\log 4/\log 3$.
\begin{codeexample}[]
\begin{tikzpicture}[decoration=Koch snowflake]
  \draw decorate{ (0,0) -- (3,0) };
  \draw decorate{ decorate{ (0,-1) -- (3,-1) }};
  \draw decorate{ decorate{ decorate{ (0,-2) -- (3,-2) }}};
  \draw decorate{ decorate{ decorate{ decorate{ (0,-3) -- (3,-3) }}}};
\end{tikzpicture}
\end{codeexample}
\end{decoration}

\begin{decoration}{Cantor set}
  This decoration replaces a straight line by a ``line with a gap in
  the middle.'' The Hausdorff dimension of the Cantor set is $\log
  2/\log 3$.
\begin{codeexample}[]
\begin{tikzpicture}[decoration=Cantor set,very thick]
  \draw decorate{ (0,0) -- (3,0) };
  \draw decorate{ decorate{ (0,-.5) -- (3,-.5) }};
  \draw decorate{ decorate{ decorate{ (0,-1) -- (3,-1) }}};
  \draw decorate{ decorate{ decorate{ decorate{ (0,-1.5) -- (3,-1.5) }}}};
\end{tikzpicture}
\end{codeexample}
\end{decoration}





\endinput

% Copyright 2003 by Till Tantau <tantau@cs.tu-berlin.de>.
%
% This program can be redistributed and/or modified under the terms
% of the LaTeX Project Public License Distributed from CTAN
% archives in directory macros/latex/base/lppl.txt.



\section{Entity-Relationship Diagram Drawing Library}

\begin{package}{pgflibrarytikzer}
  This packages provides styles for drawing entity-relationship
  diagrams. 
\end{package}

This library is intended to help you in creating E/R-diagrams. It defines
only very little new styles, but using these style |entity| instead of
saying |rectangle,draw| makes the code more expressive.


\subsection{Entities}

The package defines a simple style for drawing entities:

\begin{itemize}
  \itemstyle{entity}
  This style is to be used with nodes that represent entity types. It
  causes the node's shape to be set to a rectangle that is drawn and
  whose minimum size and width are set to sensible values.

  Note that this style is called |entity| despite the fact that it is
  to be used for nodes representing entity \emph{types} (the
  difference between an entity and an entity type is the same as the
  difference between an object and a class in object-oriented
  programming). If this bothers you, feel free to define a style
  |entity type| instead.
\begin{codeexample}[]
\begin{tikzpicture}[node distance=2cm]
  \node[entity] (sheep)                   {Sheep};
  \node[entity] (genome) [right of=sheep] {Genome};
\end{tikzpicture}
\end{codeexample}
  
  \itemstyle{every entity}
  This stype is envoked by the style |entity|. To change the
  appearance of entities, you can change this style.
\begin{codeexample}[]
\begin{tikzpicture}[node distance=2cm]
  \tikzstyle{every entity}=[draw=blue!50,fill=blue!20,thick]
  \node[entity] (sheep)                   {Sheep};
  \node[entity] (genome) [right of=sheep] {Genome};
\end{tikzpicture}
\end{codeexample}
\end{itemize}



\subsection{Relationships}

Relationships are drawn using styles that are very similar to the
styles for entities.

\begin{itemize}
  \itemstyle{relationship}
  This style works like |entity|, only it is to be used for
  relationships. Again, |relationship|s are actually relationship types. 
\begin{codeexample}[]
\begin{tikzpicture}
  \node[entity] (sheep)  at (0,0)   {Sheep};
  \node[entity] (genome) at (2,0)   {Genome};
  \node[relationship]    at (1,1.5) {has}
    edge (sheep) 
    edge (genome);
\end{tikzpicture}
\end{codeexample}
  \itemstyle{every relationship}
  works like |every entity|
\begin{codeexample}[]
\begin{tikzpicture}
  \tikzstyle{every entity}=[fill=blue!20,draw=blue,thick]
  \tikzstyle{every relationship}=[fill=orange!20,draw=orange,thick,aspect=1.5]
  \node[entity] (sheep)  at (0,0)   {Sheep};
  \node[entity] (genome) at (2,0)   {Genome};
  \node[relationship]    at (1,1.5) {has}
    edge (sheep) 
    edge (genome);
\end{tikzpicture}
\end{codeexample}
\end{itemize}



\subsection{Attributes}

\begin{itemize}
  \itemstyle{attribuate}
  This style is used to indicate that a node is an attribute. To
  connect an attribute to its entity, you can use, for example, the
  |child| command or the |pin| option. 
\begin{codeexample}[]
\begin{tikzpicture}
  \node[entity] (sheep)  {Sheep}
    child {node[attribute] {name}}
    child {node[attribute] {color}};
\end{tikzpicture}
\end{codeexample}
\begin{codeexample}[]
\begin{tikzpicture}
  \tikzstyle{every pin edge}=[draw]    
  \node[entity,pin={[attribute]60:name},pin={[attribute]120:color}] {Sheep};
\end{tikzpicture}
\end{codeexample}
  \itemstyle{key attribute}
  This style is intended for key attributes. By default, the will
  cause the attribute to be typeset in italics. Typcially, underlining
  is used instead, but that looks ugly and it is difficult to
  implement in \TeX.
  \itemstyle{every attribute}
  This style is used with every (key) attribute.
\begin{codeexample}[]
\begin{tikzpicture}[text depth=1pt]
  \tikzstyle{every attribute}=[fill=black!20,draw=black]
  \tikzstyle{every entity}=[fill=blue!20,draw=blue,thick]
  \tikzstyle{every relationship}=[fill=orange!20,draw=orange,thick,aspect=1.5]
  \node[entity] (sheep)  at (0,0)   {Sheep}
    child {node  [key attribute] {name}};
  \node[entity] (genome) at (2,0)   {Genome};
  \node[relationship]    at (1,1.5) {has}
    edge (sheep) 
    edge (genome);
\end{tikzpicture}
\end{codeexample}
\end{itemize}



%%% Local Variables: 
%%% mode: latex
%%% TeX-master: "pgfmanual-pdftex-version"
%%% End: 

% Copyright 2008 by Christian Feuersaenger
%
% This file may be distributed and/or modified
%
% 1. under the LaTeX Project Public License and/or
% 2. under the GNU Free Documentation License.
%
% See the file doc/generic/pgf/licenses/LICENSE for more details.
\section{Externalization Library}
\label{section-libs-external}
{\noindent {\emph{by Christian Feuers\"anger}}}

\begin{tikzlibrary}{external}
	This library provides a high-level automatic or semi--automatic export feature for \tikzname\ pictures.
	Its purpose is to convert each picture to a separate \pdf.
\end{tikzlibrary}

\subsection{Overview}

There are several reasons why external images for at least some pictures are of interest:
\begin{enumerate}
	\item Larger picture require a considerable amount of time, which is necessary for every compilation. However, only few images will change from run to run. It can simply save time to export finished images and include them as final graphics.
	\item It may be desirable to have final images for some graphics, for example to include them in third-party programs or to communicate them electronically.
	\item It may be necessary to typeset a file in environments where \pgfname\ and \tikzname\ are not available. In this case, external images are the only way to ensure compatibility.
\end{enumerate}
The purpose of this library is to provide a way to export any \tikzname-picture to separate \pdf\ (or \eps) images without changing the main document. It employs the |\beginpgfgraphicnamed| $\dotsc$ |\endpgfgraphicnamed| framework of \pgfname\ which is discussed in section~\ref{section-external}.

\subsection{Requirements}
For most users, the library does not need special attention since requirements are met anyway. It collects all tokens between |\begin{tikzpicture}| and the next following |\end{tikzpicture}| and replaces them by the appropriate graphics or it takes steps to generate such an image.% For Con\TeX t and plain \TeX\ users, the appropriate begin and end picture statements apply.

It can't expand macros during this step, so the only requirement is that every picture's end is directly reachable from its beginning, without further macro expansion. Furthermore, the library assumes that all \LaTeX\ pictures are ended with |\end{tikzpicture}|.% In Con\TeX t, the end command is assumed to be |\stoptikzpicture| and for plain \TeX\ it is |\endtikzpicture|.

The library always searches for the \emph{next} picture's end, |\end{tikzpicture}|. As a consequence, you can't use nested pictures directly. You \emph{can} nest pictures, but you have to avoid that the nested picture's |\end| command is found before the outer |\end| command (for example using bracing constructs or by writing the nested picture into a separate macro call).

\subsection{A word about Con\TeX t and plain \TeX}
Currently, the basic layer backend |\beginpgfgraphicnamed| $\dotsc$ |\endpgfgraphicnamed| relies on \LaTeX\ only, so externalization is only supported for \LaTeX\ yet.
%The library comes in three different versions, one for \LaTeX, one for Con\TeX t and one for plain \TeX. For reasons of simplicity, examples in this manual only refer to \LaTeX\ (especially |pdflatex|).

\subsection{Externalizing graphics}
After loading the library, a call to |\tikzexternalize| is necessary to activate the externalization.
\begin{codeexample}[code only]
\documentclass{article}
% main document, called main.tex
\usepackage{tikz}

\usetikzlibrary{external}
\tikzexternalize{main} % provide the file's real name

\begin{document}
\begin{tikzpicture}
  \node {root}
    child {node {left}}
    child {node {right}
      child {node {child}}
      child {node {child}}
    };
\end{tikzpicture}

A simple image is \tikz \fill (0,0) circle(5pt);.
\end{document}
\end{codeexample}
It is necessary to configure the file's name, in our case |main.tex|.

The method works as follows: if the document is typeset normally, the library searches for replacement images for every picture. Filenames are generated automatically if there is no explicit file name. In our case, the two file names will be |main-figure0| and |main-figure1|. If they exist, those images are simply included and the pictures as such are not processed. If graphics file do not exists, steps are taken to generate the missing ones. 
Missing images need to be generated by a separate run of \LaTeX\ in which the |\jobname| is set to the desired image file name.
\begin{codeexample}[code only]
pdflatex -jobname main-figure0 main
pdflatex -jobname main-figure1 main
\end{codeexample}
In the default configuration |mode=convert with system call|, these commands are issued automatically by using the |\write18| method to call system commands. It is also possible to output every required file name; users will need to issue these command manually (or with a script). The probably most comfortable way is to use the default configuration with
\begin{codeexample}[code only]
pdflatex -shell-escape main
\end{codeexample}
\noindent which automatically calls |pdflatex -jobname |\marg{image} \marg{main}.

If the library realizes that the jobname differs from the argument of |\tikzexternalize|, it enters its conversion mode. During the evaluation of
\begin{codeexample}[code only]
pdflatex -jobname main-figure0 main
\end{codeexample}
\noindent \pgfname\ changes the shipout routine. The complete file |main.tex| is typeset as normal, but only the part of the desired picture will be written to the output file, in our case |main-figure0.pdf|. The rest of the document is silently thrown away. Of course, such a conversion process is quite expensive since we need to do it for every picture. Since everything except the current picture is thrown away, the library skips all other pictures. Furthermore, any |\includegraphics| commands which are outside of the converted \tikzname-picture will be skipped as well. Thus, the conversion process should be much faster than typesetting the complete document, but it still requires its time.

Finally, all images will be converted. From this point on, successive runs of \LaTeX\ will use the final graphics files, the pictures won't be used anymore. Section~\ref{section-libs-external-nopgf} contains details about how to submit such a file to environments where \pgfname\ is not available.

\begin{command}{\tikzexternalize[\meta{optional arguments}]\marg{real file name}}
	This command activates the externalization. It installs commands to replace every \tikzname-picture. It needs to be called before |\begin{document}| because it may need to install its separate shipout routine.

	The argument \marg{real file name} denotes the file's name, without the |.tex| extension. It is used to generate picture filenames (by appending |-figure|\marg{number}) and to decide whether a new shipout routine needs to be installed (if |\jobname| has a different value).

	The \meta{optional argument} can be any of the keys described below.
\end{command}

\begin{key}{/tikz/external/system call=\marg{template}}
\label{extlib:systemcall:option}
	A template string used to generate system calls. Inside of \marg{template}, the macro |\image| can be used as placeholder for the image which is about to be generated while |\texsource| contains the main file name.

	The default is 
\begin{codeexample}[code only]
\tikzset{external/system call={pdflatex -shell-escape -halt-on-error 
    -interaction=batchmode -jobname "\image" "\texsource"}
\end{codeexample}

	For |eps| output, you can (and need to) use
\begin{codeexample}[code only]
\tikzset{external/system call={latex -shell-escape -halt-on-error
    -interaction=batchmode -jobname "\image" "\texsource"; 
    dvips -o "\image".ps "\image".dvi}}
\end{codeexample}
	
	The argument \marg{template} will be expanded using |\edef|, so any control sequences will be expanded. During this evaluation, `|\\|' will result in a normal backslash, `|\|'. Furthermore, double quotes `|"|', single quotes `|'|', semicolons and dashes `|-|' will be made to normal characters if any package uses them as macros. This ensures compatibility with the |german| package, for example.
\end{key}

\subsubsection{Customizing the Generated File Names}
The default filename for externalized graphics is `\meta{real file name}|-figure_|\meta{number}' where \meta{number} ranges from $0$ to whatever is required. However, there are a couple of ways to change the generated filenames:
\begin{itemize}
	\item Changing the overall file name using a |prefix|,
	\item Changing the file name for a single figure using |\tikzsetnextfilename|,
	\item Changing the file name for a restricted set of figures using |figure name|.
\end{itemize}

\begin{key}{/tikz/external/prefix=\marg{file name prefix} (initially empty)}
	A shortcut for |\tikzsetexternalprefix|\marg{file name prefix}.
\end{key}

\begin{command}{\tikzsetexternalprefix\marg{file name prefix}}
	Assigns a common prefix used by all file names. For example,
\begin{codeexample}[code only]
\tikzsetexternalprefix{figures/}
\end{codeexample}
	will prepend |figure/| to every external graphics file name.

	Please note that |\tikzsetexternalprefix| is the \emph{only} way to assign a prefix in case you want to prepare your document for environments where \pgfname\ is not installed (see section~\ref{section-libs-external-nopgf}).
\end{command}

\begin{command}{\tikzsetnextfilename\marg{file name}}
	Sets the file name for the \emph{next} \tikzname\ picture or |\tikz| short command. It will \emph{only} be used for the next picture.

	Pictures for which no explicit file name has been set will get automatically generated file names.

	Please note that |prefix| will still be prepended to \marg{file name}.
\begin{codeexample}[code only]
\documentclass{article}
% main document, called main.tex
\usepackage{tikz}

\usetikzlibrary{external}
\tikzexternalize[prefix=figures/]{main} % provide the file's real name

\begin{document}

\tikzsetnextfilename{trees}
\begin{tikzpicture} % will be written to 'figures/trees.pdf'
  \node {root}
    child {node {left}}
    child {node {right}
      child {node {child}}
      child {node {child}}
    };
\end{tikzpicture}

\tikzsetnextfilename{simple}
A simple image is \tikz \fill (0,0) circle(5pt);. % will be written to 'figures/simple.pdf'

\begin{tikzpicture} % will be written to 'figures/main-figure0.pdf'
   \draw[help lines] (0,0) grid (5,5);
\end{tikzpicture}
\end{document}
\end{codeexample}
\begin{codeexample}[code only]
pdflatex -shell-escape main
\end{codeexample}
\end{command}

\begin{key}{/tikz/external/figure name=\marg{name}}
	Changes the names of \emph{all} figures. It is possible to change |figure name| during the document using |\tikzset{external/figure name|=\marg{name}|}|. A unique counter will be used for each different \marg{name}, and each counter will start at $0$.

	The value of |prefix| will be applied after |figure name| has been evaluated.
\begin{codeexample}[code only]
\documentclass{article}
% main document, called main.tex
\usepackage{tikz}

\usetikzlibrary{external}
\tikzexternalize{main} % provide the file's real name

\begin{document}

\begin{tikzpicture} % will be written to 'main-figure0.pdf'
  \node {root}
    child {node {left}}
    child {node {right}
      child {node {child}}
      child {node {child}}
    };
\end{tikzpicture}

{
  \tikzset{external/figure name={subset_}}
  A simple image is \tikz \fill (0,0) circle(5pt);. % will be written to 'subset_0.pdf'

  \begin{tikzpicture} % will be written to 'subset_1.pdf'
     \draw[help lines] (0,0) grid (5,5);
  \end{tikzpicture}
}% here, the old file name will be restored:

\begin{tikzpicture} % will be written to 'main-figure1.pdf'
   \draw (0,0) -- (5,5);
\end{tikzpicture}
\end{document}
\end{codeexample}
	The scope of |figure name| ends with the next closing brace (as all values set by |\tikzset| do).

	Remark: Use |\tikzset{external/figure name/.add={prefix_}{_suffix_}}| to add a |prefix_| and a |_suffix_| to the actual value of |figure name|.
\end{key}


\subsubsection{Remaking Figures or Skipping Figures}
\begin{key}{/tikz/external/force remake=\marg{boolean} (default true)}
	A boolean which is used by |mode=convert with system call|. Every up-to-date check will fail, resulting in automatic regeneration of every figure.

\begin{codeexample}[code only]
\tikzset{external/force remake}
\begin{tikzpicture}
	\draw (0,0) circle(5pt);
\end{tikzpicture}
\end{codeexample}
	You can also use |force remake| inside of a local \TeX\ group to remake only selected pictures. The example
\begin{codeexample}[code only]
\tikz \draw (0,0) -- (1,1);

{
\tikzset{external/force remake}
\begin{tikzpicture}
   \draw (0,0) circle(5pt);
\end{tikzpicture}
}

\tikz \draw (0,0) -- (1,1);
\end{codeexample}
	will only apply |force remake| to the second figure.
\end{key}

\begin{key}{/tikz/external/remake next=\marg{boolean} (default true)}
	A variant of |force remake| which applies only to the next image.
\end{key}

\begin{key}{/tikz/external/export next=\marg{boolean} (default true)}
	A boolean which can be used to disable the export mechanism for single pictures.
\end{key}

\begin{key}{/tikz/external/export=\marg{boolean} (initially true)}
	A boolean which can be used to disable the export mechanism for all pictures inside of the current \TeX-scope. 

\begin{codeexample}[code only]
\begin{document}
\begin{tikzpicture} % will be exported
	...
\end{tikzpicture}

{
\tikzset{external/export=false}
\begin{tikzpicture} % won't be exported
	...
\end{tikzpicture}
...
} 

\begin{tikzpicture} % will be exported
	...
\end{tikzpicture}
\end{document}
\end{codeexample}
	For \LaTeX, the feature lasts until the next |\end|\marg{$\cdot$} (this holds for every call to |\tikzset|).
\end{key}

\begin{command}{\tikzexternaldisable}
	Allows to disable the complete externalization. While |export next| will still collect the contents of picture environments, this command uninstalls the hooks for the external library completely. Thus, nested picture environments or environments where |\end{tikzpicture}| is not directly reachable won't produce compilation failures -- although it is not possible to externalize them automatically.

	The externalization remains disabled until the end of the next \TeX\ group (or environment) or until the next call to |\tikzexternalenable|.
\end{command}

\begin{command}{\tikzexternalenable}
	Re-enables a previously running externalization after |\tikzexternaldisable|.
\end{command}


\subsubsection{Customizing the Externalization}
\begin{key}{/tikz/external/figure list=\marg{boolean} (initially true)}
	A boolean which configures whether a figure list shall be generated. A figure list is an output file named \marg{jobname}|.figlist| which is filled with file names of each figure, one per line.

	This file is not used by \TeX\ anymore, its purpose is to issue the required conversion commands |pdflatex -jobname |\marg{picture file name} \marg{main file} manually (or in a script).

\begin{codeexample}[code only]
\documentclass{article}
% main document, called main.tex
\usepackage{tikz}

\usetikzlibrary{external}
\tikzexternalize[
   mode=graphics if exists,
   figure list=true,
   prefix=figures/]{main} % provide the file's real name

\begin{document}

\tikzsetnextfilename{trees}
\begin{tikzpicture}
  \node {root}
    child {node {left}}
    child {node {right}
      child {node {child}}
      child {node {child}}
    };
\end{tikzpicture}

\tikzsetnextfilename{simple}
A simple image is \tikz \fill (0,0) circle(5pt);.

\begin{tikzpicture}
   \draw[help lines] (0,0) grid (5,5);
\end{tikzpicture}
\end{document}
\end{codeexample}

\begin{codeexample}[code only]
pdflatex main
\end{codeexample}
generates |main.figlist| containing
\begin{codeexample}[code only]
figures/trees
figures/simple
figures/main-figure0
\end{codeexample}
\end{key}

\begin{key}{/tikz/external/mode=\marg{choice} (initially convert with system call)}
	Configures what to do with \tikzname\ pictures (unless we are currently externalising one particular image, in that case, these modes are ignored).

	The preconfigured mode |convert with system call| checks whether external graphics file are up-to-date and includes them if that is the case. Any picture which is not up-to-date will be generated automatically using a system call. The system call can be configured using the |system call| template. The up-to-date check is simple: if the file does not exist, it is not up-to-date. Furthermore, if one of the |force remake| or |remake next| keys is true, the figure is not up-to-date. In all other case, the file is considered to be up-to-date. Setting this mode automatically disables |figure list| because such a file is not required. If you still need it, you can enable it after setting |mode|.

	Please note that system calls may be disabled for security reasons. For pdflatex, they can be enabled using
\begin{codeexample}[code only]
pdflatex -shell-escape
\end{codeexample}
	while other \TeX\ variants may need other switches. The feature is sometimes called |\write18|.
	
	The choice |only graphics| always tries to replace pictures with external graphics. It is an error if the graphics file does not exist.

	The choice |no graphics| (or, equivalently, |only pictures|) typesets \tikzname\ pictures without checking for external graphics.

	A mixture is |graphics if exists|, it checks whether a suitable graphics file exists and includes it if that is the case. If it does not exist, the picture is typeset using \TeX.

	Mode |list only| skips every \tikzname\ picture; it only generates the file \marg{main file}|.figlist| containing file names for every picture, the contents of any picture environment is thrown away and a replacement text is shown. This implies |figure list=true|. See also the |list and make| mode which includes available graphics.

	The mode |list and make| is similar to |list only|: it generates the same file \marg{main file}|.figlist|, but any images which exist already are included as graphics instead of ignoring them. Furthermore, this mode generates an additional file: \marg{main file}.makefile. This allows to use a work flow like
\begin{codeexample}[code only]
% step 1: generate main.makefile:
pdflatex main
% step 2: generate ALL graphics on 2 processors:
make -j 2 main.makefile
% step 3: include the graphics:
pdflatex main
\end{codeexample}
	\noindent This last make method is, however unnecessary: |list and make| just assumes that images are generated somehow (not necessary with the generated makefile). However, the generated makefile allows parallel externalisation of graphics on multi-core systems.

\end{key}


\begin{key}{/tikz/external/verbose IO=\marg{boolean} (initially true)}
	A boolean which configures whether I/O operations shall be listed in the logfile.
\end{key}
\begin{key}{/tikz/external/verbose optimize=\marg{boolean} (initially true)}
	A boolean which configures whether optimization operations shall be listed in the logfile.
\end{key}

\begin{key}{/tikz/external/optimize=\marg{boolean} (initially true)}
	Configures whether the conversion process shall be optimized. This affects only the case when |\jobname| differs from the main file name, i.e. when single pictures are converted.

	In that case, the main file is compiled as usual - but everything except the selected picture is thrown away. If optimization is enabled, all other pictures won't be processed at all. Furthermore, expensive commands which do not contribute to the selected picture will be thrown away as well.

	The default implementation discards |\includegraphics| commands which are \emph{not} inside of the selected picture to reduce conversion time.

	It is possible to add commands which shall be optimized away, see below.
\end{key}

\begin{key}{/tikz/external/optimize command away=\meta{\textbackslash command}\marg{required argument count}}
	Installs commands to optimize \meta{\textbackslash command} away. As is described above, optimization applies to the case when single pictures are converted: one usually doesn't need to process (probably expensive) commands which do not contribute to the selected picture.

	The argument \marg{required argument count} is either empty or a non-negativ integer between $0$ and $9$. It denotes the number of arguments which should be consumed after \meta{\textbackslash command}. In any case, one argument in square brackets after the command will be recognised as well. To be more precise, the following cases for arguments of \meta{\textbackslash command} are supported:
	\begin{enumerate}
		\item If \marg{required argument count} is empty (the default), \meta{\textbackslash command} may take one optional argument in square brackets and one in curly braces (which is also optional).
		\item If \marg{required argument count} is not empty, \marg{\textbackslash command} may take one optional argument in square brackets. Furthermore, it expects exactly \marg{required argument count} following arguments.
	\end{enumerate}

	Example:
\begin{codeexample}[code only]
\tikzset{external/optimize command away=\includegraphics}
\end{codeexample}

\begin{codeexample}[code only]
\newcommand{\myExpensiveMacro}[1]{Very expensive!}

\tikzset{external/optimize command away=\myExpensiveMacro}
\end{codeexample}

\begin{codeexample}[code only]
\newcommand{\myExpensiveMacroWithThreeArgs}[3]{Very expensive!}

\tikzset{external/optimize command away={\myExpensiveMacroWithThreeArgs}{3}}
\end{codeexample}
\begin{codeexample}[code only]
% A command with optional argument:
\newcommand{\aFurtherExample}[3][]{Very expensive!}

% consume only two arguments: the first optional one will be processed
% anyway:
\tikzset{external/optimize command away={\myExpensiveMacroWithThreeArgs}{2}}
\end{codeexample}
	The argument \meta{\textbackslash command} must be the name of a single macro. Any occurance of this macro, together with its arguments, will be removed.
\begin{codeexample}[code only]
\begin{tikzpicture}
	% this picture is currently converted!
\end{tikzpicture}

This here is outside of the converted picture and contains \myExpensiveMacro. It will be discarded.

This call: \myExpensiveMacro[argument=value]{Argument} as well. 
And this here: \myExpensiveMacro{Argument} also.
\end{codeexample}

	The default is to optimize |\includegraphics| away.

	This key is actually a style which sets the |optimize/install| and |optimize/restore| keys.
\end{key}

\begin{key}{/tikz/external/optimize/install}
	A command key which contains code to install optimizations. You can append code here (or clear the macro) if you need to modify the optimization.
\end{key}
\begin{key}{/tikz/external/optimize/restore}
	A command key which contains code to undo optimizations. You can append code here (or clear the macro) if you need to modify the optimization.
\end{key}

\begin{key}{/tikz/external/only named=\marg{boolean} (initially false)}
	If enabled, only pictures for which file names have been set explicitly using |\tikzsetnextfilename| will be considered, no file names will be generated automatically.
\end{key}

\begin{key}{/pgf/images/include external (initially \textbackslash pgfimage\{\#1\})}
\index{External Graphics!Bounding Box Issues}
	This key constitutes the public interface to exchange the |\includegraphics| command used for the image inclusion.

	Its description can be found in the corresponding basic layer documentation on page~\pageref{pgf:includeexternalkey}.

	Just one example here: you can use
\begin{codeexample}[code only]
\pgfkeys{/pgf/images/include external/.code={\includegraphics[viewport=0 0 211.28 175.686]{#1}}}
\end{codeexample}
	to manually change the viewport (bounding box) for included graphics.
	
	If you want to limit the effects of this key to just one externalized figure, use
\begin{codeexample}[code only]
{
\pgfkeys{/pgf/images/include external/.code={\includegraphics[viewport=0 0 211.28 175.686]{#1}}}
\begin{tikzpicture}
	...
\end{tikzpicture}
}% this brace ends the effect of `include external'
\end{codeexample}
\end{key}

\begin{command}{\tikzifexternalizing\marg{true code}\marg{false code}}
	This command can be used to check whether an image is currently written to its separate graphics file (if the ``grab'' procedure is running). If so, the \marg{true code} will be executed. If not, that means if the main document is being typeset normally, the \marg{false code} will be invoked.

	This command must be used \emph{after} |\tikzexternalize|.
\end{command}

\subsection{Using external graphics without \textmd{\pgfname}\ installed}
\label{section-libs-external-nopgf}
Given that every picture has been exported correctly, one may want to compile a file without \pgfname\ and \tikzname\ installed. \tikzname\ comes with a minimal package which contains just enough commands to replace every |tikzpicture| environment and the |\tikz| short command with the appropriate external graphics. It can be found at
\begin{codeexample}[code only]
latex/pgf/utilities/tikzexternal.sty
\end{codeexample}
\noindent and needs to be used instead of |\usepackage{tikz}|. Our example from the beginning thus becomes
\begin{codeexample}[code only]
\documentclass{article}
% main document, called main.tex
%\usepackage{tikz}

\usepackage{graphicx}
\usepackage{tikzexternal}

%\usetikzlibrary{external}
\tikzexternalize{main} % provide the file's real name

\begin{document}
\begin{tikzpicture}
  \node {root}
    child {node {left}}
    child {node {right}
      child {node {child}}
      child {node {child}}
    };
\end{tikzpicture}

A simple image is \tikz \fill (0,0) circle(5pt);.
\end{document}
\end{codeexample}
\noindent where the following files are necessary to compile the document:
\begin{codeexample}[code only]
tikzexternal.sty
main.tex
figures/main-figure0.pdf
figures/main-figure1.pdf
figures/main-figure2.pdf
\end{codeexample}
\noindent If there are any `|.dpth|' files, for example |figures/main-figure0.dpth|, these files are also required. They contain information for the \tikzname\ |baseline| option.

Just copy the |.sty| file into the directory of your |main.tex| file and use it as part of your document.

Please keep in mind, that only |tikzpicture| environments and |\tikz| short images are available within the externalization framework. Additionally, calls to |\tikzset| and |\pgfkeys| won't lead to compilation errors because they are simply ignored. But since |pgfkeys| is not available, any option supplied to |\tikzexternalize| is \emph{ignored}.

Please use |\tikzsetexternalprefix| instead of the |prefix| option.

\subsection{Externalization and bounding box restrictions}
Bounding box restrictions provide no problem when used with \eps\ graphics (see the next section). However, they pose problems for |pdflatex|, so you may need to use the |latex| / |dvips| combination if you use bounding box restrictions and externalization.

\subsection{\eps\ Graphics}
It is also possible to use \eps\ graphics instead of \pdf\ files. There are different ways to produce them, for example to use |pdflatex| and call |pdftops -eps |\marg{pdf file} \marg{eps file} afterwards. You could add this command to the |system call| option.

Alternatively, you can use |latex| and |dvips| for image conversion as is explained for the |system call| option, see page~\pageref{extlib:systemcall:option}. See the documentation for the basic level externalization in section~\ref{section-external} for restrictions of other drivers.

\subsection{Interoperability with the basic layer externalization}
This library is fully compatible with |\beginpgfgraphicnamed|$\dotsc$|\endpgfgraphicnamed| environments. However, you will need to use the |export next=false| key to avoid conflicts:
\begin{codeexample}[code only]
\beginpgfgraphicnamed{picture4}
\tikzset{external/export next=false}
\begin{tikzpicture}
   \draw (0,0) -- (4,4);
\end{tikzpicture}
\endpgfgraphicnamed
\end{codeexample}
Please keep in mind that file prefixes do not apply to the basic layer.

% Copyright 2006 by Till Tantau
%
% This file may be distributed and/or modified
%
% 1. under the LaTeX Project Public License and/or
% 2. under the GNU Free Documentation License.
%
% See the file doc/generic/pgf/licenses/LICENSE for more details.


\section{Fading Library}
\label{section-library-fadings}

\begin{pgflibrary}{fadings}
  The package defines a number of fadings, see
  Section~\ref{section-tikz-transparency} for an introduction.  The
  \tikzname\ version defines special \tikzname\ commands for creating
  fadings. These commands are explained in
  Section~\ref{section-tikz-transparency}.   
\end{pgflibrary}

\newcommand\fadingindex[1]{%
  \index{#1@\protect\texttt{#1} fading}%
  \index{Fadings!#1@\protect\texttt{#1}}%
  \texttt{#1}& 
  \begin{tikzpicture}[baseline=5mm-.5ex]
    \fill [black!20] (0,0) rectangle (1,1);
    \path [pattern=checkerboard,pattern color=black!30] (0,0) rectangle (1,1);

    \fill [fading=#1,blue] (0,0) rectangle (1,1);
  \end{tikzpicture} \\[4.5mm]
}

\noindent
\begin{tabular}{ll}
  \emph{Fading name} & \emph{Example (solid blue faded on checkerboard)} \\[1mm]
  \fadingindex{west}  
  \fadingindex{east}  
  \fadingindex{north}  
  \fadingindex{south} 
\end{tabular}


%%% Local Variables: 
%%% mode: latex
%%% TeX-master: "pgfmanual-pdftex-version"
%%% End: 

% Copyright 2006 by Till Tantau
%
% This file may be distributed and/or modified
%
% 1. under the LaTeX Project Public License and/or
% 2. under the GNU Free Documentation License.
%
% See the file doc/generic/pgf/licenses/LICENSE for more details.


\section{Fitting Library}
\label{section-library-fit}

\begin{tikzlibrary}{fit}
  The library defines (currently only two) options for fitting a node
  so that it contains a set of coordinates.
\end{tikzlibrary}

When you load this library, the following options become available:

\begin{key}{/tikz/fit=\meta{coordinates or nodes}}
  This option must be given to a |node| path command. The
  \meta{coordinates or nodes} should be a sequence of \tikzname\
  coordinates or node names,  one directly after the other without
  commas (like with the |plot coordinates| path operation). Examples
  as |(1,0) (2,2)| or |(a) (1,0) (b)|, where |a| and |b| are nodes.

  For this sequence of coordinates, a minimal bounding box is computed
  that encompasses all the listed \meta{coordinates or nodes}. For
  coordinates in the list, the bounding box is guaranteed to contain
  this coordinate, for nodes it is guaranteed to contain the |east|,
  |west|, |north| and |south| anchors of the node. In principle (the
  details will be explained in a moment), things are now setup such
  that the text box of the node will be exactly this bounding box.

  Here is an example: We fit several points in a rectangular node. By
  setting the |inner sep| to zero, we see exactly the text box of the
  node. Then we fit these points again in circular node. Note how
  the circle encompasses exactly the same bounding box.
\begin{codeexample}[]
\begin{tikzpicture}[inner sep=0pt,thick,
                    dot/.style={fill=blue,circle,minimum size=3pt}]
  \draw[help lines] (0,0) grid (3,2);
  \node[dot] (a) at (1,1) {};
  \node[dot] (b) at (2,2) {};
  \node[dot] (c) at (1,2) {};
  \node[dot] (d) at (1.25,0.25) {};
  \node[dot] (e) at (1.75,1.5) {};

  \node[draw=red,   fit=(a) (b) (c) (d) (e)] {box};
  \node[draw,circle,fit=(a) (b) (c) (d) (e)] {};
\end{tikzpicture}  
\end{codeexample}

  Every time the |fit| option is used, the following style is also
  applied to the node:
  \begin{stylekey}{/tikz/every fit (initially \normalfont empty)}
    Set this style to change the appearance of a node that uses the
    |fit| option.
  \end{stylekey}

  The exact effects of the |fit| option are the following:
  \begin{enumerate}
  \item A minimal bounding box containg all coordinates is
    computed. Note that if a coordinate like |(a)| is used that
    contain a node name, this has the same effect as explicitly
    providing the |(a.north)| and |(a.south)| and |(a.west)| and
    |(a.east)|. If you wish to refer only to the center of the |a|
    node, use  |(a.center)| instead.
  \item The |text width| option is set to the width of this bounding box.
  \item The |align=center| option is set.
  \item The |anchor| is set to |center|.
  \item The |at| position of the node is set to the center of the
    computed bounding box.
  \item After the node has been typeset, its height and depth are
    adjusted such that they add up to the height of the computed
    bounding box and such that the text of the node is vertically
    centered inside the box.
  \end{enumerate}
  The above means that, generally speaking, if the node contains text
  like |box| in the above example, it will be centered inside the
  box. It will be difficult to put the text elsewhere, in particular,
  changing the |anchor| of the node will not have the desired
  effect. Instead, what you should do is to create a node with the
  |fit| option that does not contain any text, give it a name, and
  then use normal nodes to add text at the desired
  positions. Alternatively, consider using the |label| or |pin|
  options. 

  Suppose, for instance, that in the above example we want the word
  ``box'' to appear inside the box, but at its top. This can be
  achieved as follows: 
\begin{codeexample}[]
\begin{tikzpicture}[inner sep=0pt,thick,
                    dot/.style={fill=blue,circle,minimum size=3pt}]
  \draw[help lines] (0,0) grid (3,2);
  \node[dot] (a) at (1,1) {};
  \node[dot] (b) at (2,2) {};
  \node[dot] (c) at (1,2) {};
  \node[dot] (d) at (1.25,0.25) {};
  \node[dot] (e) at (1.75,1.5) {};

  \node[draw=red,fit=(a) (b) (c) (d) (e)] (fit) {};
  \node[below] at (fit.north) {box};
\end{tikzpicture}  
\end{codeexample}

 Here is a real-life example that uses fitting:

\begin{codeexample}[]
\begin{tikzpicture}
  [vertex/.style={minimum size=2pt,fill,draw,circle},
   open/.style={fill=none},
   sibling distance=1.5cm,level distance=.75cm,
   every fit/.style={ellipse,draw,inner sep=-2pt},
   leaf/.style={label={[name=#1]below:$#1$}},auto]

  \node [vertex] (root) {}
  child { node [vertex,open] {}
    child { node [vertex,open] {}
      child { node [vertex] (b's parent) {}
        child { node [vertex] {}
          child { node [vertex,leaf=d] {} }
          child { node [vertex,leaf=e] {} } }
        child { node [vertex,leaf=b] {} } }
      child { node [vertex,leaf=a] {} } }
    child { node [coordinate] {}
      child[missing] 
      child { node [vertex] (f's parent) {}
        child { node [vertex,leaf=c] {} }
        child { node [vertex,leaf=f] {} } } }
    edge from parent node {$\rho$} };
  
  \node [fit=(d) (e) (b) (b's parent),label=above left:$F^{(b,R)}$] {};
  \node [fit=(c) (f) (f's parent),label=above right:$F^{(c,R)}$]    {};
\end{tikzpicture}
\end{codeexample}

\end{key}

\begin{key}{/tikz/rotate fit=\meta{angle} (initially 0)}
  This key fits \meta{coordinates or nodes} inside a node that is
  rotated by \meta{angle}. As a side effect, it also sets the
  |/tikz/rotate| key.

\begin{codeexample}[]  
\begin{tikzpicture}[inner sep=0pt,thick,
  dot/.style={fill=blue,circle,minimum size=3pt}]
  \draw[help lines] (0,0) grid (3,2);
  \node[dot] (a) at (1,1) {};
  \node[dot] (b) at (2,2) {};
  \node[dot] (c) at (1,2) {};
  \node[dot] (d) at (1.25,0.25) {};
  \node[dot] (e) at (1.75,1.5) {};
  \node[draw, fit=(a) (b) (c) (d) (e)] {};
  \node[draw=red, rotate fit=30, fit=(a) (b) (c) (d) (e)] {};
\end{tikzpicture}
\end{codeexample}

\end{key}





% Copyright 2008 by Mark WIbrow
%
% This file may be distributed and/or modified
%
% 1. under the LaTeX Project Public License and/or
% 2. under the GNU Free Documentation License.
%
% See the file doc/generic/pgf/licenses/LICENSE for more details.


\section{Fixed Point Arithmetic Library}

{\bf\emph{This library is provisional, and may change/disappear without warning}}

\begin{pgflibrary}{fixedpointarithmetic}
  This library provides an interface to the \LaTeX{} package 
  |fp| for fixed point arithmetic. 	In addition to loading this 
  library you must ensure |fp| is loaded otherwise errors
  will occur. 
\end{pgflibrary}

\subsection{Overview}

  Whilst the mathematical engine that comes with \pgfname{} is 
  reasonably fast and flexible when it comes to parsing, the accuracy
  tends to be fairly low, particularly for expressions involving many 
  operations chained together. In addition the range of values that
  can be computed is very small: $\pm16383.99999$.
	Conversely, the |fp| package has a reasonably high accuracy, and 
	can	perform computations over a wide range of 
	values (approximately $\pm9.999\times10^{17}$), but is comparatively 
	slow and not very 
	flexible, particularly regarding parsing.
  
  This library enables the combination of the two: the flexible parser 
  of the \pgfname{} mathematical engine with the evaluation accuracy 
  of |fp|. There are, however, a number of important points to
  bear in mind: 

\begin{itemize}
	
  \item 
  
  Whilst |fp| supports very large numbers, \pgfname{} and
  \tikzname{} do not. It is possible to calculate the result of 
  |2^20| or |1.2e10+3.4e10|, but it is not possible to use these
  results in pictures directly without some ``extra work''.
  
  \item
  
  The \pgfname{} mathematical engine will still be used to evaluate
  lengths, such as |10pt| or |3em|, so it is not possible for an
  length to exceed the range of values supported by 
  \TeX-dimensions ($\pm16383.99999$pt), even though the resulting 
  expression is within the range of |fp|. So, for example, one can
  calculate |3cm*10000|, but not |3*10000cm|.
  
  \item
  
  All the functions listed in Section~\ref{pgfmath-syntax}, have been
  mapped onto |fp| equivalents. However, it is not guaranteed that
  functions will perform in the same way as they do 
  in \pgfname. Reference should be made to the documentation for |fp|.
  
  \item
  
  In \pgfname, trigonometric functions such as |sin| and |cos| assume 
  arguments are in degrees, and functions sucn as |asin| and |acos|
  return results in degrees. Although |fp| uses radians for such
  functions, \pgfname{} automtically converts arguments from degrees 
  to radians, and converts results from radians to degrees, to ensure 
  everything ``works properly''.
   
  \item
  
  The overall speed will actually be slower than using 
  \pgfname{} mathematical engine. The calculating power of |fp| 
  comes at the cost of an increased processing time.
  
  
\end{itemize}
  
\subsection{Using Fixed Point Arithmetic in PGF and \tikzname}

  The following key is provided to use |fp| in \pgfname{} 
  and \tikzname:
  
\begin{key}{/pgf/fixed point arithmetic=\meta{options}}
\keyalias{tikz}

  This key will set the key path to |/pgf/fixed point|, and 
  execute \meta{options}. Then it will install the necessary 
  commands so that the \pgfname{} parser will use |fp| to perform 
  calculations. 
  The best way to use this key is as an argument to a scope or 
  picture. This means that |fp| does not always have to be used, 
  and \pgfname{} can use its own mathematical engine at other times, 
  which can lead to a significant reduction in the time for a document 
  to compile.
  
\end{key}

  Currently there are only a few keys key suported for \meta{options}:
  
\begin{key}{/pgf/fixed point/scale results=\meta{factor}}

	As noted above, |fp| can process a far greater range of numbers
	than \pgfname{} and \tikzname{}. In order to use results from 
	|fp| in a |{pgfpicture}| or a |{tikzpicture}| they need to be
	scaled. When this key is used \pgfname{} will scale results
	of any evaulation by \meta{factor}. However, as it is not
	desirable for every part of every expression to be scaled,
	scaling will only take place if a special prefix |*| is used.
	If |*| is used at the begining of an expression the evaluation
	of the expression will evaluated and then multiplied by 
	\meta{factor}.

\begin{codeexample}[]	
\begin{tikzpicture}[fixed point arithmetic={scale results=10^-6}]
  \draw [help lines] grid (3,2);
  \draw (0,0) -- (2,2);
  \draw [red, line width=4pt] (*1.0e6,0) -- (*3.0e6,*2.0e6);
\end{tikzpicture}
\end{codeexample}
  
  A special case of scaling involves plots of data containing
  large numbers from files.
  It is possible to ``pre-process'' a file, typically using the 
  application that generates the data, to either precede
  the relavent column with |*| or to perform the scaling as part
  of the calculation process. However, it may be desirable for
  the data in a plot to appear in a table as well, so, two files would
  be required, one pre-processed for plotting, and one not. This
  extra work may be undesirable so the following keys are provided:
  
\begin{key}{/pgf/fixed point/scale file plot x=\meta{factor}}
  This key will scale the first column of data read from
  a file before it is plotted. It is independent of the
  |scale results| key.
\end{key}

\begin{key}{/pgf/fixed point/scale file plot y=\meta{factor}}
  This key will scale the second column of data read from
  a file before it is plotted. 
\end{key}

\begin{key}{/pgf/fixed point/scale file plot z=\meta{factor}}
  This key will scale the third column of data read from
  a file before it is plotted. 
\end{key}

\end{key} 


% Copyright 2008 by Christian Feuersaenger
%
% This file may be distributed and/or modified
%
% 1. under the LaTeX Project Public License and/or
% 2. under the GNU Free Documentation License.
%
% See the file doc/generic/pgf/licenses/LICENSE for more details.


\section{Floating Point Unit Library}
{\noindent {\emph{by Christian Feuers\"anger}}}
\label{pgfmath-floatunit}

\begingroup
\pgfqkeys{/pgf/number format}{sci}
\pgfkeys{/pgf/fpu}

\begin{pgflibrary}{fpu}
	The floating point unit (fpu) allows the full data range of scientific computing for use in \pgfname. Its core is the \pgfname\ math routines for mantissa operations, leading to a reasonable trade--of between speed and accuracy. It does not require any third--party packages or external programs.
\end{pgflibrary}

\subsection{Overview}
The fpu provides a replacement set of math commands which can be installed in isolated placed to archieve large data ranges at reasonable accuracy. It provides at least\footnote{To be more precise, the FPU's exponent is currently a 32 bit integer. That means it supports a significantly larger data range than an IEEE double precision number -- but if a future \TeX\ version may provide lowlevel access to doubles, this may change.} the IEEE double precision data range, $\pgfmathprintnumber{-1e+324}, \dotsc, \pgfmathprintnumber{+1e324}$. The absolute smallest number bigger than zero is $\pgfmathprintnumber{1e-324}$. The FPU's relative precision is at least $\pgfmathprintnumber{1e-4}$ although operations like addition have a relative precision of $\pgfmathprintnumber{1e-6}$.

\subsection{Usage}
\begin{key}{/pgf/fpu=\marg{boolean} (default true)}
	This key installs or uninstalls the FPU. The installation exchanges any routines of the standard math parser with those of the FPU: |\pgfmathadd| will be replaced with |\pgfmathfloatadd| and so on. Furthermore, any number will be parsed with |\pgfmathfloatparsenumber|.

\begin{codeexample}[]
\pgfkeys{/pgf/fpu}
\pgfmathparse{1+1}\pgfmathresult
\end{codeexample}
\noindent The FPU uses a lowlevel number representation consisting of flags, mantisse and exponent\footnote{Users should \emph{always} use high level routines to manipulate floating point numbers as the format may change in a future release.}. To avoid unnecessary format conversions, |\pgfmathresult| will usually contain such a cryptic number. Depending on the context, the result may need to be converted into something which is suitable for \pgfname\ processing (like coordinates) or may need to be typeset. The FPU provides such methods as well.

%--------------------------------------------------
% \begin{codeexample}[]
% \begin{tikzpicture}
% 	\fill[red,fpu,/pgf/fpu/scale results=1e-10] (*1.234e10,*1e10) -- (*2e10,*2e10);
% \end{tikzpicture}
% \end{codeexample}
%-------------------------------------------------- 

	Use |fpu=false| to deactivate the FPU. This will restore any change. Please note that this is not necessary if the FPU is uses inside of a \TeX\ group -- it will be deactivated afterwards anyway.

	It does not hurt to call |fpu=true| or |fpu=false| multiple times.

	Please note that if the |fixed point arithmetics| library of \pgfname\ will be activated after the FPU, the FPU will be deactivated automatically.
\end{key}

\begin{key}{/pgf/fpu/output format=\mchoice{float,sci,fixed} (initially float)}
	This key allows to change the number format in which the FPU assigns |\pgfmathresult|.

	The predefined choice |float| uses the low-level format used by the FPU. This is useful for further processing inside of any library.
\begin{codeexample}[]
\pgfkeys{/pgf/fpu,/pgf/fpu/output format=float}
\pgfmathparse{exp(50)*42}\pgfmathresult
\end{codeexample}

	The choice |sci| returns numbers in the format \meta{mantisse}|e|\meta{exponent}. It provides almost no computational overhead.
\begin{codeexample}[]
\pgfkeys{/pgf/fpu,/pgf/fpu/output format=sci}
\pgfmathparse{4.22e-8^-2}\pgfmathresult
\end{codeexample}

	The choice |fixed| returns normal fixed point numbers and provides the highest compatibility with the \pgfname\ engine. It is activated automatically in case the FPU scales results.
\begin{codeexample}[]
\pgfkeys{/pgf/fpu,/pgf/fpu/output format=fixed}
\pgfmathparse{sqrt(1e-12)}\pgfmathresult
\end{codeexample}
\end{key}

\begin{key}{/pgf/fpu/scale results=\marg{scale}}
	A feature which allows semi--automatic result scaling. Setting this key has two effects: first, the output format for \emph{any} computation will be set to |fixed| (assuming results will be processed by \pgfname's kernel). Second, any expression which starts with a star, |*|, will be multiplied with \marg{scale}.
\end{key}

\begin{keylist}{
	/pgf/fpu/scale file plot x=\marg{scale},%
	/pgf/fpu/scale file plot y=\marg{scale},%
	/pgf/fpu/scale file plot z=\marg{scale}}%
	These keys will patch \pgfname's |plot file| command to automatically scale single coordinates by \marg{scale}.

	The initial setting does not scale |plot file|.
\end{keylist}

\begin{command}{\pgflibraryfpuifactive\marg{true-code}\marg{false-code}}
	This command can be used to execute dependend on whether the FPU has been activated or not.
\end{command}

\subsection{Comparison to the fixed point arithmetics library}
There are other ways to increase the data range and/or the precision of \pgfname's math parser. One of them is the |fp| package, preferrable combined with \pgfname's |fixed point arithmetic| library. The differences between the FPU and |fp| are:
\begin{itemize} 
	\item The FPU supports at least the complete IEEE double precision number range, while |fp| covers only numbers of magnitude $\pm\pgfmathprintnumber{1e17}$.
	\item The FPU has a uniform relative precision of about 4--5 correct digits. The fixed point library has an absolute precision which may perform good in many cases -- but will fail at the ends of the data range (as every fixed point routines does).
	\item The FPU has potential to be faster than |fp| as it has access to fast mantisse operations using \pgfname's math capabilities (which use \TeX\ registers).
\end{itemize}

\subsection{Command Reference and Programmer's Manual}
  
\subsubsection{Float--specific commands}
\begin{command}{\pgfmathfloatparsenumber\marg{x}}
	Reads a number of arbitrary magnitude and precision and stores its result into |\pgfmathresult| as floating point number $m \cdot 10^e$ with mantisse and exponent base~$10$.

	The algorithm and the storage format is purely text-based. The number is stored as a triple of flags, a positive mantisse and an exponent, such as
\begin{codeexample}[]
\pgfmathfloatparsenumber{2}
\pgfmathresult
\end{codeexample}
	Please do not rely on the low-level representation here, use |\pgfmathfloattomacro| (and its variants) and |\pgfmathfloatcreate| if you want to work with these components.

	The flags encoded in |\pgfmathresult| are represented as a digit where `$0$' stands for the number $\pm 0\cdot 10^0$, `$1$' stands for a positive sign, `$2$' means a negative sign, `$3$' stands for `not a number', `$4$' means $+\infty$ and `$5$' stands for $-\infty$.

	The mantisse is a normalized real number $m \in \mathbb{R}$, $1 \le m < 10$. It always contains a period and at least one digit after the period. The exponent is an integer.

	Examples:
\begin{codeexample}[]
\pgfmathfloatparsenumber{0}
\pgfmathfloattomacro{\pgfmathresult}{\F}{\M}{\E}
Flags: \F; Mantisse \M; Exponent \E.
\end{codeexample}

\begin{codeexample}[]
\pgfmathfloatparsenumber{0.2}
\pgfmathfloattomacro{\pgfmathresult}{\F}{\M}{\E}
Flags: \F; Mantisse \M; Exponent \E.
\end{codeexample}

\begin{codeexample}[]
\pgfmathfloatparsenumber{42}
\pgfmathfloattomacro{\pgfmathresult}{\F}{\M}{\E}
Flags: \F; Mantisse \M; Exponent \E.
\end{codeexample}

\begin{codeexample}[]
\pgfmathfloatparsenumber{20.5E+2}
\pgfmathfloattomacro{\pgfmathresult}{\F}{\M}{\E}
Flags: \F; Mantisse \M; Exponent \E.
\end{codeexample}

\begin{codeexample}[]
\pgfmathfloatparsenumber{1e6}
\pgfmathfloattomacro{\pgfmathresult}{\F}{\M}{\E}
Flags: \F; Mantisse \M; Exponent \E.
\end{codeexample}

\begin{codeexample}[]
\pgfmathfloatparsenumber{5.21513e-11}
\pgfmathfloattomacro{\pgfmathresult}{\F}{\M}{\E}
Flags: \F; Mantisse \M; Exponent \E.
\end{codeexample}
	The argument \marg{x} may be given in fixed point format or the scientific `e' (or `E') notation. The scientific notation does not necessarily need to be normalised. Its exponent should be limited to the range $-16000 \le e \le +16000$ (the \TeX-integer range).
\end{command}

\begin{key}{/pgf/fpu/handlers/empty number=\marg{input}\marg{unreadable part}}
	This command key is invoked in case an empty string is parsed inside of |\pgfmathfloatparsenumber|. You can overwrite it to assign a replacement |\pgfmathresult| (in float!).

	The initial setting is to invoke |invalid number|, see below.
\end{key}
\begin{key}{/pgf/fpu/handlers/invalid number=\marg{input}\marg{unreadable part}}
	This command key is invoked in case an invalid string is parsed inside of |\pgfmathfloatparsenumber|. You can overwrite it to assign a replacement |\pgfmathresult| (in float!).

	The initial setting is to generate an error message.
\end{key}

\begin{command}{\pgfmathfloatqparsenumber\marg{x}}
	The same as |\pgfmathfloatparsenumber|, but does not perform sanity checking.
\end{command}

\begin{command}{\pgfmathfloattofixed{\marg{x}}}
	Converts a number in floating point representation to a fixed point number. It is a counterpart to |\pgfmathfloatparsenumber|. The algorithm is purely text based and defines |\pgfmathresult| as a string sequence which represents the floating point number \marg{x} as a fixed point number (of arbitrary precision).

\begin{codeexample}[]
\pgfmathfloatparsenumber{0.00052}
\pgfmathfloattomacro{\pgfmathresult}{\F}{\M}{\E}
Flags: \F; Mantisse \M; Exponent \E
$\to$ 
\pgfmathfloattofixed{\pgfmathresult}
\pgfmathresult
\end{codeexample}

\begin{codeexample}[]
\pgfmathfloatparsenumber{123.456e4}
\pgfmathfloattomacro{\pgfmathresult}{\F}{\M}{\E}
Flags: \F; Mantisse \M; Exponent \E
$\to$
\pgfmathfloattofixed{\pgfmathresult}
\pgfmathresult 
\end{codeexample}
\end{command}

\begin{command}{\pgfmathfloattosci\marg{float}}
	Converts a number from low-level floating point representation to scientific format, $1.234e4$.
\end{command}

\begin{command}{\pgfmathfloatcreate{\marg{flags}}{\marg{mantisse}}{\marg{exponent}}}
	Defines |\pgfmathresult| as the floating point number encoded by
	\marg{flags}, \marg{mantisse} and \marg{exponent}.
	
	All arguments are characters and will be expanded using |\edef|.
\begin{codeexample}[]
\pgfmathfloatcreate{1}{1.0}{327}
\pgfmathfloattomacro{\pgfmathresult}{\F}{\M}{\E}
Flags: \F; Mantisse \M; Exponent \E
\end{codeexample}
\end{command}

\begin{command}{\pgfmathfloattomacro{\marg{x}}{\marg{flagsmacro}}{\marg{mantissemacro}}{\marg{exponentmacro}}}
	Extracts the flags of a floating point number \marg{x} to \marg{flagsmacro}, the mantisse to \marg{mantissemacro} and the exponent to \marg{exponentmacro}.
\end{command}

\begin{command}{\pgfmathfloattoregisters{\marg{x}}{\marg{flagscount}}{\marg{mantissedimen}}{\marg{exponentcount}}}
	Takes a floating point number \marg{x} as input and writes flags to count
	register \marg{flagscount}, mantisse to dimen register \marg{mantissedimen} and exponent to count
	register \marg{exponentcount}.

	Please note that this method rounds the mantisse to \TeX-precision.
\end{command}

\begin{command}{\pgfmathfloattoregisterstok{\marg{x}}{\marg{flagscount}}{\marg{mantissetoks}}{\marg{exponentcount}}}
	A variant of |\pgfmathfloattoregisters| which writes the mantisse into a token register. It maintains the full input precision.
\end{command}

\begin{command}{\pgfmathfloatgetflags{\marg{x}}{\marg{flagscount}}}
	Extracts the flags of \marg{x} into the count register \marg{flagscount}.
\end{command}

\begin{command}{\pgfmathfloatgetflagstomacro{\marg{x}}{\marg{\textbackslash macro}}}
	Extracts the flags of \marg{x} into the macro \meta{\textbackslash macro}.
\end{command}

\begin{command}{\pgfmathfloatgetmantisse{\marg{x}}{\marg{mantissedimen}}}
	Extracts the mantisse of \marg{x} into the dimen register \marg{mantissedimen}.
\end{command}
\begin{command}{\pgfmathfloatgetmantissetok{\marg{x}}{\marg{mantissetoks}}}
	Extracts the mantisse of \marg{x} into the token register \marg{mantissetoks}.
\end{command}
\begin{command}{\pgfmathfloatgetexponent{\marg{x}}{\marg{exponentcount}}}
	Extracts the exponent of \marg{x} into the count register \marg{exponentcount}.
\end{command}

\begin{command}{\pgfmathfloatifflags\marg{floating point number}\marg{flag}\marg{true-code}\marg{false-code}}
	Invokes \marg{true-code} if the flag of \marg{floating point number} equals \marg{flag} and \marg{false-code} otherwise.

	The argument \marg{flag} can be one of
	\begin{description}
		\item[0] to test for zero,
		\item[1] to test for positive numbers,
		\item[+] to test for positive numbers,
		\item[2] to test for negative numbers,
		\item[-] to test for negative numbers,
		\item[3] for ``not-a-number'',
		\item[4] for $+\infty$,
		\item[5] for $-\infty$.
	\end{description}
	
\begin{codeexample}[]
\pgfmathfloatparsenumber{42}
\pgfmathfloatifflags{\pgfmathresult}{0}{It's zero!}{It's not zero!}
\pgfmathfloatifflags{\pgfmathresult}{1}{It's positive!}{It's not positive!}
\pgfmathfloatifflags{\pgfmathresult}{2}{It's negative!}{It's not negative!}

% or, equivalently
\pgfmathfloatifflags{\pgfmathresult}{+}{It's positive!}{It's not positive!}
\pgfmathfloatifflags{\pgfmathresult}{-}{It's negative!}{It's not negative!}
\end{codeexample}
\end{command}


\begin{command}{\pgfmathfloatshift{\marg{x}}{\marg{num}}}
	Defines |\pgfmathresult| to be $\meta{x} \cdot 10^{\meta{num}}$. The operation is an arithmetic shift base ten and modifies only the exponent of \marg{x}. The argument \marg{num} is expected to be a (positive or negative) integer.
\end{command}


\begin{command}{\pgfmathfloatabserror\marg{x}\marg{y}}
	Defines |\pgfmathresult| to be the absolute error between two floating point numbers $x$ and $y$, $\lvert x - y\rvert $ and returns the result as floating point number.
\end{command}

\begin{command}{\pgfmathfloatrelerror\marg{x}\marg{y}}
	Defines |\pgfmathresult| to be the relative error between two floating point numbers $x$ and $y$, $\lvert x - y\rvert / \lvert y \rvert $ and returns the result as floating point number.
\end{command}
\begin{command}{\pgfmathfloattoextentedprecision{\marg{x}}}
Renormalizes \marg{x} to extended precision mantisse, meaning
$100 \le m < 1000$ instead of $1 \le m < 10$.

The `extended precision' means we have higher accuracy when we apply pgfmath operations to mantisses.

The input argument is expected to be a normalized floating point number; the output argument is a non-normalized floating point number (well, normalized to extended precision).

The operation is supposed to be very fast.
\end{command}
\begin{command}{\pgfmathfloatint\marg{x}}
Returns the integer part of the floating point number \marg{x}, by truncating any digits after the period. This methods is applied to the absolute value $\rvert x \lvert$, so negative numbers are treated in the same way as positive ones.

The result is returned as floating point number as well.
\end{command}

\begin{command}{\pgfmathfloatsetextprecision\marg{shift}}
	Sets the precision used inside of |\pgfmathfloattoextentedprecision| to \marg{shift}.

	The different choices are
	
	\begin{tabular}{llrll}
	0 & normalization to &    $0$ & $\le m < 1$ 	& (disable extended precision)\\
	1 & normalization to &   $10$ & $\le m < 100$	\\
	2 & normalization to & 	$100$ & $\le m < 1000$	& (default of |\pgfmathfloattoextentedprecision|)\\
	3 & normalization to & $1000$ & $\le m < 10000$	\\
	\end{tabular}
\end{command}

\begin{command}{\pgfmathroundto{\marg{x}}}
	Rounds a fixed point number to prescribed precision and writes the result to |\pgfmathresult|.

	The desired precision can be configured with |/pgf/number format/precision|, see section~\ref{pgfmath-numberprinting}. This section does also contain application examples.
	
	Any trailing zeros after the period are discarded. The algorithm is purely text based and allows to deal with precisions beyond \TeX's fixed point support.

	As a side effect, the global boolean |\ifpgfmathfloatroundhasperiod| will be set to true if and only if the resulting mantisse has a period. Furthermore, |\ifpgfmathfloatroundmayneedrenormalize| will be set to true if and only if the rounding result's floating point representation would have a larger exponent than \marg{x}. 
\begin{codeexample}[]
\pgfmathroundto{1}
\pgfmathresult
\end{codeexample}
\begin{codeexample}[]
\pgfmathroundto{4.685}
\pgfmathresult
\end{codeexample}
\begin{codeexample}[]
\pgfmathroundto{19999.9996}
\pgfmathresult
\end{codeexample}
\end{command}

\begin{command}{\pgfmathroundtozerofill{\marg{x}}}
	A variant of |\pgfmathroundto| which always uses a fixed number of digits behind the period. It fills missing digits with zeros.
\begin{codeexample}[]
\pgfmathroundtozerofill{1}
\pgfmathresult
\end{codeexample}
\begin{codeexample}[]
\pgfmathroundto{4.685}
\pgfmathresult
\end{codeexample}
\begin{codeexample}[]
\pgfmathroundtozerofill{19999.9996}
\pgfmathresult
\end{codeexample}
\end{command}

\begin{command}{\pgfmathfloatround{\marg{x}}}
	Rounds a normalized floating point number to a prescribed precision and writes the result to |\pgfmathresult|.

	The desired precision can be configured with |/pgf/number format/precision|, see section~\ref{pgfmath-numberprinting}. 
	
	This method employs |\pgfmathroundto| to round the mantisse and applies renormalizations if necessary.

	As a side effect, the global boolean |\ifpgfmathfloatroundhasperiod| will be set to true if and only if the resulting mantisse has a period.
\begin{codeexample}[]
\pgfmathfloatparsenumber{52.5864}
\pgfmathfloatround{\pgfmathresult}
\pgfmathfloattomacro{\pgfmathresult}{\F}{\M}{\E}
Flags: \F; Mantisse \M; Exponent \E.
\end{codeexample}
\begin{codeexample}[]
\pgfmathfloatparsenumber{9.995}
\pgfmathfloatround{\pgfmathresult}
\pgfmathfloattomacro{\pgfmathresult}{\F}{\M}{\E}
Flags: \F; Mantisse \M; Exponent \E.
\end{codeexample}
\end{command}

\begin{command}{\pgfmathfloatroundzerofill{\marg{x}}}
	A variant of |\pgfmathfloatround| produces always the same number of digits after the period (it includes zeros if necessary).
\begin{codeexample}[]
\pgfmathfloatparsenumber{52.5864}
\pgfmathfloatroundzerofill{\pgfmathresult}
\pgfmathfloattomacro{\pgfmathresult}{\F}{\M}{\E}
Flags: \F; Mantisse \M; Exponent \E.
\end{codeexample}
\begin{codeexample}[]
\pgfmathfloatparsenumber{9.995}
\pgfmathfloatroundzerofill{\pgfmathresult}
\pgfmathfloattomacro{\pgfmathresult}{\F}{\M}{\E}
Flags: \F; Mantisse \M; Exponent \E.
\end{codeexample}
\end{command}

\begin{command}{\pgfmathlog{\marg{x}}}
	Defines |\pgfmathresult| to be the natural logarithm of \marg{x}, $\ln(\meta{x})$. This method is logically the same as |\pgfmathln|, but it applies floating point arithmetics to read number \marg{x} and employs the logarithm identity 
		\[ \ln(m \cdot 10^e) = \ln(m) + e \cdot \ln(10) \]
	to get the result. The factor $\ln(10)$ is a constant, so only $\ln(m)$ with $1 \le m < 10$ needs to be computed. This is done using standard pgf math operations.

	Please note that \marg{x} needs to be a number, expression parsing is not possible here.

	If \marg{x} is \emph{not} a bounded positive real number (for example $\meta{x} \le 0$), |\pgfmathresult| will be \emph{empty}, no error message will be generated.
\begin{codeexample}[]
\pgfmathlog{1.452e-7}
\pgfmathresult
\end{codeexample}
\begin{codeexample}[]
\pgfmathlog{6.426e+8}
\pgfmathresult
\end{codeexample}
\end{command}

\subsubsection{Replacement commands}
This sections describes some of the replacement commands in more details.
\begin{command}{\pgfmathfloatlessthan{\marg{x}}{\marg{y}}}
	Defines |\pgfmathresult| as $1.0$ if $\meta{x} < \meta{y}$, but $0.0$ otherwise. It also sets the global \TeX-boolean |\pgfmathfloatcomparison| accordingly. The arguments \marg{x} and \marg{y} are expected to be numbers which have already been processed by |\pgfmathfloatparsenumber|. Arithmetics is carried out using \TeX-registers for exponent- and mantisse comparison.
\end{command}

\begin{command}{\pgfmathfloatmax{\marg{x}}{\marg{y}}}
	Defines |\pgfmathresult| as the maximum of two floating point numbers \marg{x} and \marg{y}. The arguments \marg{x} and \marg{y} are expected to be numbers which have already been processed by |\pgfmathfloatparsenumber|. Arithmetics is carried out using \TeX-registers for exponent- and mantisse comparison.
\end{command}

\begin{command}{\pgfmathfloatmin{\marg{x}}{\marg{y}}}
	Defines |\pgfmathresult| as the minimum of two floating point numbers \marg{x} and \marg{y}. The arguments \marg{x} and \marg{y} are expected to be numbers which have already been processed by |\pgfmathfloatparsenumber|. Arithmetics is carried out using \TeX-registers for exponent- and mantisse comparison.
\end{command}


\begin{command}{\pgfmathfloatadd{\marg{x}}{\marg{y}}}
	Defines |\pgfmathresult| to be $\meta{x} + \meta{y}$ for two floating point numbers, returning another floating point number.

	It invokes the usual math engine on mantisses and employs 8 significant decimal digits for its computation (using |\pgfmathfloattoextentedprecision|).
\end{command}

\begin{command}{\pgfmathfloatsubtract{\marg{x}}{\marg{y}}}
	Defines |\pgfmathresult| to be $\meta{x} - \meta{y}$ for two floating point numbers, returning a floating point number.

	It invokes the usual math engine on mantisses and employs 8 significant decimal digits for its computation (using |\pgfmathfloattoextentedprecision|).
\end{command}

\begin{command}{\pgfmathfloatmultiply{\marg{x}}{\marg{y}}}
	Defines |\pgfmathresult| to be $x \cdot y$ for two floating point numbers, returning a floating point number.
	
	It invokes the usual math engine on mantisses.
\end{command}
\begin{command}{\pgfmathfloatmultiplyfixed\marg{float}\marg{fixed}}
	Defines |\pgfmathresult| to be $\meta{float} \cdot \meta{fixed}$ where \meta{float} is a floating point number and \meta{fixed} is a fixed point number. The computation is performed in floating point arithmetics, that means we compute $m \cdot \meta{fixed}$ and renormalize the result where $m$ is the mantisse of \meta{float}.

	This operation renormalizes \meta{float} with |\pgfmathfloattoextentedprecision| before the operation, that means it is intended for relatively small arguments of \meta{fixed}. The result is a floating point number.
\end{command}

\begin{command}{\pgfmathfloatdivide{\marg{x}}{\marg{y}}}
	Defines |\pgfmathresult| to be $x / y$ for two floating point numbers, returning a floating point number.
	
	It invokes the usual math engine on mantisses.
\end{command}

\begin{command}{\pgfmathfloatsqrt{\marg{x}}}
	Defines |\pgfmathresult| to be $\sqrt x$ for a floating point $x$ and returns the result as floating point number.
	
	It invokes the usual math engine on mantisses. It has a relative precision of about $10^{-5}$.
\end{command}
\begin{command}{\pgfmathfloatabs{\marg{x}}}
	Defines |\pgfmathresult| to be $\lvert x\rvert $ for a floating point number $x$ and returns the result as floating point number.
\end{command}

\subsubsection{Accessing the Original Math Routines for Programmers}
As soon as the library is loaded, every private math routine will be copied to a new name.
This allows library and package authors to access the \TeX-register based math routines even if the FPU is activated. And, of course, it allows the FPU as such to perform its own mantissa computations.

The private implementations of \pgfname\ math commands, which are of the form |\pgfmath|\meta{name}|@|, will be available as|\pgfmath@basic@|\meta{name}|@| as soon as the library is loaded.


%--------------------------------------------------
% \subsubsection{Implementation details: Accessing flags, mantisse and exponent}
% Floating point representations can be read using some private macros which are only available if `|@|' is a letter (that means inside of package/module implementations).
% 
% \begin{command}{\pgfmathfloat@decompose\marg{x}\relax\marg{f}\marg{m}\marg{e}}
% 	Assigns floating point flags of number \marg{x} to integer register \marg{f}, the mantisse to the dimen register \marg{m} and the exponent to integer register \marg{e}.
% 
% 	The input argument \marg{x} needs to be fully expanded and must not be enclosed by braces.
% \end{command}
% 
% \begin{command}{\pgfmathfloat@decompose@tok\marg{x}\relax\marg{f}\marg{m}\marg{e}}
% 	Works in the same way as |\pgfmathfloat@decompose|, but assumes that \marg{m} is a token register (i.e. reads the mantisse as a character sequence).
% \end{command}
% 
% \begin{command}{\pgfmathfloat@decompose@F\marg{x}\relax\marg{f}}
% 	Reads only flags into integer register \marg{f}.
% \end{command}
% \begin{command}{\pgfmathfloat@decompose@M\marg{x}\relax\marg{m}}
% 	Reads only the mantisse into dimen register \marg{m}.
% \end{command}
% \begin{command}{\pgfmathfloat@decompose@Mtok\marg{x}\relax\marg{m}}
% 	Reads only the mantisse into token register \marg{m}.
% \end{command}
% \begin{command}{\pgfmathfloat@decompose@Mtok\marg{x}\relax\marg{e}}
% 	Reads only the exponent into integer register \marg{e}.
% \end{command}
%-------------------------------------------------- 
\endgroup

% Copyright 2008 by Mark Wibrow
%
% This file may be distributed and/or modified
%
% 1. under the LaTeX Project Public License and/or
% 2. under the GNU Public License.
%
% See the file doc/generic/pgf/licenses/LICENSE for more details.

\section{Lindemayer System Drawing Library}
\subsection{Overview}

Lindenmayer systems (also commonly known as ``L-systems''), were
originally developed by Aristid Lindenmayer as a theory of algae 
growth patterns and then subsequently used to model branching 
patterns in plants and produce fractal patterns.
Typically, an L-system consists of a set of symbols, each of which
is associated with some graphical action (such as ``turn left'' or 
``move forward'') and a set of rules (``production'' or ``rewrite'' 
rules). Given a string of symbols, the rewrite rules are applied 
several times and the when resulting string is processed the action 
associated with each symbol is executed. 

In \pgfname, L-systems can be used to create simple 2-dimensional
fractal patterns\ldots	
\begin{codeexample}[pre={\expandafter\let\csname pgf@lsystem@Koch curve\endcsname=\relax}]
\begin{tikzpicture}
\pgfdeclarelindenmayersystem{Koch curve}{
  \rule{F -> F-F++F-F}
}

\filldraw [fill=blue!50, draw=blue!50!black]
  [l-system={Koch curve, step=2pt, angle=60, axiom=F++F++F, order=3}]
  lindenmayer system -- cycle;
\end{tikzpicture}
\end{codeexample}

\noindent\ldots and ``plant like'' patterns\ldots

\begin{codeexample}[]
\begin{tikzpicture}
\draw [green!50!black, rotate=90]
  [l-system={rules={F -> FF-[-F+F]+[+F-F]}, axiom=F, order=4, step=2pt,
   randomize step percent=25, angle=30, randomize angle percent=5}]
  lindenmayer system;
\end{tikzpicture}
\end{codeexample}

\noindent
\ldots but it is important to bear in mind that even moderately
complex L-systems can exceed the available memory of \TeX, 
and can be very slow. 
If possible, you are advised to increase the main memory and save 
stack to their maximum possible values for your particular 
\TeX{} distribution. 
However, even by doing this you may find you still run out of memory
quite quickly.

For an excellent introduction to L-systems (containing some
``really cool'' pictures -- many of which are sadly not possible in 
\pgfname)
see \emph{The Algorithmic Beauty of Plants} by 
Przemyslaw Prusinkiewicz and Aristid Lindenmayer (which is freely
available via the internet). 

\begin{pgflibrary}{lindenmayersystems}
  This \pgfname-library provides basic commands for defining and using 
  simple L-systems. The \tikzname-library provides, furthermore, a
  front end for using L-systems in  \tikzname. 
\end{pgflibrary}



\subsubsection{Declaring L-systems}
  Before an L-system can be used, it must be declared using the
  following command:
  
\begin{command}{\pgfdeclarelindenmayersystem\marg{name}\marg{specification}}

This command declares a Lindenmayer system called \meta{name}.
The \meta{specification} argument contains a description of the
L-system's symbols and rules. Two commands |\symbol| and |\rule| are
only defined when the \meta{specification} argument is executed.

\begin{command}{\symbol\marg{name}\marg{code}}
  This defines a symbol called \meta{name} for a specific L-system,
  and associates it with \meta{code}. 
  
  A symbol should consist of a single 
  alpha-numeric character (i.e., |A|-|Z|, |a|-|z| or |0|-|9|). 
  The symbols
  |F|, |f|, |+|, |-|, |[| and |]| are available by default so do 
  not need to be defined for each L-system. However, if you are
  feeling adventurous, they can be redefined for specific L-systems 
  if required. The L-system treats the default symbols as follows
  (the commands they execute are described below):
  
  \begin{itemize}
  	\item 
  	|F| move forward a certain distance, drawing a line. Uses
  	|\pgflsystemdrawforward|.
  	
  	\item 
  	|f| move forward a certain distance, without drawing a line.
  	Uses |\pgflsystemmoveforward|.
  	
  	\item 
  	|+| turn left by some angle.
  	Uses |\pgflsystemturnleft|.
  	
  	\item 
  	|-| turn right by some angle.
  	Uses |\pgflsystemturnright|.
  	
  	\item 
  	|[| save the current state (i.e., the position and direction).
  	Uses |\pgflsystemsavestate|.
  	
  	\item 
  	|]| restore the last saved state.
  	Uses |\pgflsystemrestorestate|.
  	 
  \end{itemize}
  
  The symbols |[| and |]| act like a stack: |[| pushes the state of the
  L-system on to the stack, and |]| pops a state off the stack. 
  
   When \meta{code} is executed the transformation matrix is set up
  so that the origin is at the current position and the positive 
  x-axis ``points forward'', so |\pgfpathlineto{\pgfpoint{1cm}{0cm}}| 
  draws a line 1cm forward.

The following keys store values which can alter the production of an
L-system. 

\begin{key}{/pgf/lindenmayer system/step=\meta{length} (initially 5pt)}
  This key sets how far the L-system moves forward.
\end{key}

\begin{key}{/pgf/lindenmayer system/randomize step percent=\meta{percentage} (initially 0)}
  If the step is to be randomized, this key specifies by how much. 
\end{key}

\begin{key}{/pgf/lindenmayer system/left angle=\meta{angle} (initially 90)}
  This key sets the angle through which the L-system turns when it
  turns left.
\end{key}

\begin{key}{/pgf/lindenmayer system/right angle=\meta{angle} (initially 90)}
  This key sets the angle through which the L-system turns when it
  turns right.
\end{key}

\begin{key}{/pgf/lindenmayer system/randomize angle percent=\meta{percentage} (initially 0)}
  If the angles are to be randomized, this key specifies by how much. 
\end{key}

For speed and convenience, when the code for a symbol is executed the 
following commands are available.

\begin{command}{\pgflsystemstep}
	The current ``step'' of the L-system (i.e., how far the system
	will move foward if required). The value stored in this macro
	may be changed if |\pgflsystemrandomizestep| is used (see below). 
\end{command}

\begin{command}{\pgflsystemleftangle}
	The angle the L-system will turn when it turns left. 
	The value stored in this macro may be changed if 
	|\pgflsystemrandomizeleftangle| is used. 
\end{command}

\begin{command}{\pgflsystemrightangle}
	The angle the L-system will turn when it turns right. 
	The value stored in this macro may be changed if 
	|\pgflsystemrandomizerightangle| is used. 
\end{command}


The following commands may be useful if you wish to define your own
symbols.

\begin{command}{\pgflsystemrandomizestep}
	Randomizes the value in |\pgflsystemstep| according to the value of
	the |randomize| |step| |percent| key.
\end{command}

\begin{command}{\pgflsystemrandomizeleftangle}
	Randomizes the value in |\pgflsystemleftangle| according to the value of
	the |randomize| |angle| |percent| key.
\end{command}

\begin{command}{\pgflsystemrandomizerightangle}
	Randomizes the value in |\pgflsystemrightangle| according to the value of
	the |randomize| |angle| key.
\end{command}

\begin{command}{\pgflsystemdrawforward}
	Move forward in the current direction, by |\pgflsystemstep|,
	drawing a line in the process. This macro calls 
	|\pgflsystemrandomizestep|. Internally, \pgfname{} simply
	shifts the transformation matrix in the positive direction of 
	the current (transformed) x-axis by |\pgflsystemstep| 
	and then executes	a line-to to the (newly transformed) origin.
\end{command}

\begin{command}{\pgflsystemmoveforward}
	Move forward in the current direction, by |\pgflsystemstep|,
	without drawing a line. This macro calls 
	|\pgflsystemrandomizestep|. \pgfname{} executes a transformation
	as abvove, but executes	a move-to to the (newly transformed) 
	origin.
\end{command}

\begin{command}{\pgflsystemturnleft}
  Turn left by |\pgflsystemleftangle|. This macro calls 
	|\pgflsystemrandomizeleftangle|. Internally, \pgfname{}
	simply rotates the transformation matrix.
\end{command}

\begin{command}{\pgflsystemturnright}
	Turn right by |\pgflsystemrightangle|. This macro calls 
	|\pgflsystemrandomizerightangle|. Internally, \pgfname{}
	simply rotates the transformation matrix.
\end{command}

\begin{command}{\pgflsystemsavestate}
	Save the current position and orientation. Internally,
	\pgfname{} simply starts a new \TeX-group. 
\end{command}

\begin{command}{\pgflsystemrestorestate}
	Restore the lased saved position and orientation. Internally,
	\pgfname{} closes a \TeX-group, restoring the transformation 
	matrix of the outer scope, and a move-to command is executed to
	the (transformed) origin.
\end{command}
  

\end{command}

\begin{command}{\rule{\ttfamily\char`\{}\meta{head}{\ttfamily->}\meta{body}{\ttfamily\char`\}}}
  Declare a rule. \meta{head} should consist of a single symbol, which
  need not have been declared using |\symbol| or exist as a defualt
  symbol (in fact, the more intersting L-systems depend on using
  symbols with no corresponding code, to control the ``growth'' of the
  system).
 	\meta{body} consists of a string of symbols, which again need not
 	necessarily have any code associated with them.
 	
\end{command}

  As an example, the following shows an L-system that uses
  some of these commands. This example illustrates the point
  that some symbols, in this case |A| and |B|, do not have to 
  have code associated with them. They simply control the
  growth of the system.

\begin{codeexample}[pre={\nullfont\expandafter\let\csname pgf@lsystem@Hilbert curve\endcsname=\relax}]
\pgfdeclarelindenmayersystem{Hilbert curve}{
  \symbol{X}{\pgflsystemdrawforward}
  \symbol{+}{\pgflsystemturnright} % Explicitly define + and - symbols.
  \symbol{-}{\pgflsystemturnleft}
  \rule{A -> +BX-AXA-XB+}
  \rule{B -> -AX+BXB+XA-}
}
\tikz\draw[lindenmayer system={Hilbert curve, axiom=A, order=4, angle=90}]
  lindenmayer system;
\end{codeexample}


\end{command}

\subsection{Using Lindenmayer Systems}
\subsubsection{Using L-Systems in PGF}

The following command is used to run an L-system in \pgfname:
\begin{command}{\pgflindenmayersystem\marg{name}\marg{axiom}\marg{order}}
  Runs the L-system called \meta{name} using the input string \meta{axiom}
  for \meta{order} iteraions.
  In general, prior to calling this command the 
  transformation matrix should be set appropriately for shifting and
  rotating, and a move-to to the (transformed) origin should be 
  executed. This origin will be where the L-system starts.
  In addition the relavent keys should be set appropriately.
  
\begin{codeexample}[]
\begin{tikzpicture}
  \draw [help lines] grid (3,2);
  \pgfset{lindenmayer system/.cd, angle=60, step=2pt}
  \foreach \x/\y in {0cm/1cm, 1.5cm/1.5cm, 2.5cm/0.5cm, 1cm/0cm}{
    \pgftransformshift{\pgfqpoint{\x}{\y}}
    \pgfpathmoveto{\pgfpointorigin}
    \pgflindenmayersystem{Koch curve}{F++F++F}{2}
    \pgfusepath{stroke}
  }
\end{tikzpicture}
\end{codeexample}

  Note that, it is perfectly feasible for an L-system to define
  special symbols which perform the move-to and use-path 
  operations.
  
\end{command}

\subsubsection{Using L-Systems in Ti\emph{k}Z}

  In \tikzname, an L-system is created using a path operation. 
  However, \tikzname{} is more flexible regarding the positioning
  of the L-system and also provides keys to create L-systems
  ``on-line''.
  
\begin{pathoperation}{lindenmayer system}{ \opt{|[|\meta{keys}|]|}}
  This will run an L-system according to the parameters
  specified in \meta{keys} (which can also contain normal \tikz{} keys
  such as |draw| or |thin|). The syntax is flexible
  regarding the L-system parameters and the following all do
  the same thing:

\begin{codeexample}[code only]
\draw lindenmayer system [lindenmayer system={Hilbert curve, axiom=4, order=3}];
\end{codeexample} 

\begin{codeexample}[code only]
\draw [lindenmayer system={Hilbert curve, axiom=4, order=3}] lindenmayer system;
\end{codeexample} 

\begin{codeexample}[code only]
\tikzset{lindenmayer system={Hilbert curve, axiom=4, order=3}}
\draw lindenmayer system;
\end{codeexample} 

\end{pathoperation}

\begin{pathoperation}{l-system}{ \opt{|[|\meta{keys}|]|}}
  A more compact version of the |lindenmayer system| path command.
\end{pathoperation}

This library adds some additional keys for specifying L-systems. They
all have the same path, namely, |/pgf/lindenmayer| |system|, but so 
you do not have to keep repeating this path the following keys are 
provided:
 
\begin{stylekey}{/pgf/lindenmayer system=\marg{keys}}
\keyalias{tikz}
This key changes the key path to |/pgf/lindenmayer systems| and
executes \meta{keys}.
\end{stylekey}

\begin{stylekey}{/pgf/l-system=\marg{keys}}
\keyalias{tikz}
A more compact version of the previous key.
\end{stylekey}

\begin{key}{/pgf/lindenmayer system/name=\marg{name}}
  Set the name for the L-system. 
\end{key}

\begin{key}{/pgf/lindenmayer system/axiom=\marg{string}}
  Set the axiom (or input string) for the L-system. 
\end{key}

\begin{key}{/pgf/lindenmayer system/order=\marg{integer}}
  Set the number of iterations the L-system will perform.
\end{key}

\begin{key}{/pgf/lindenmayer system/rules=\marg{rule list}}
  This key allows an (anonymous) L-system to be declared ``on-line''.
  There is, however, a restriction that only the defualt symbols can be
  used for drawing (empty symbols can still be used to control
  the growth of the system). The rules in \meta{rule list} should
  be separated by commas.
  
\begin{codeexample}[]
\tikz[rotate=65]\draw [green!60!black] l-system
  [l-system={rules={F -> F[+F]F[-F]}, axiom=F, order=4, angle=25,step=3pt}];
\end{codeexample} 
\end{key}

\begin{key}{/pgf/lindenmayer system/anchor=\meta{anchor}}
  Be default, when this key is not used, the L-system will start from 
  the last specified coordinate. By using this key, the L-system
  will be placed inside a special (rectangle) node which can be
	positioned using \meta{anchor}.

 
\begin{codeexample}[]
\begin{tikzpicture}[l-system={step=1.75pt, order=5, angle=60}]
  \pgfdeclarelindenmayersystem{Sierpinski triangle}{
    \symbol{X}{\pgflsystemdrawforward}
    \symbol{Y}{\pgflsystemdrawforward}
    \rule{X -> Y-X-Y}
    \rule{Y -> X+Y+X}
  }
  \draw [help lines] grid (3,2);
  \draw [red] (0,0) l-system 
    [l-system={Sierpinski triangle, axiom=+++X, anchor=south west}];
  \draw [blue] (3,2) l-system 
    [l-system={Sierpinski triangle, axiom=X, anchor=north east}];
\end{tikzpicture}
\end{codeexample} 
\end{key}

% Copyright 2006 by Till Tantau
%
% This file may be distributed and/or modified
%
% 1. under the LaTeX Project Public License and/or
% 2. under the GNU Free Documentation License.
%
% See the file doc/generic/pgf/licenses/LICENSE for more details.


\section{Matrix Library}

\begin{tikzlibrary}{matrix}
  This library packages defines additional styles and options for
  creating matrices.
\end{tikzlibrary}


\subsection{Matrices of Nodes}

A \emph{matrix of nodes} is a \tikzname\ matrix in which each cell
contains a node. In this case it is bothersome having to write
|\node{| at the beginning of each cell and |};| at the end of each
cell. The following key simplifies typesetting such matrices.

\begin{key}{/tikz/matrix of nodes}
  Conceptually, this key adds |\node{| at the beginning and |};| at
  the end of each cell and sets the |anchor| of the node to
  |base|. Furthermore, it adds  the option |name| option to each node,
  where the name is set to  \meta{matrix name}|-|\meta{row
    number}|-|\meta{column number}. For  example, if the matrix has
  the name |my matrix|, then the node in  the upper left cell will get
  the name |my matrix-1-1|. 
\begin{codeexample}[]
\begin{tikzpicture}
  \matrix (magic) [matrix of nodes]
  {
    8 & 1 & 6 \\
    3 & 5 & 7 \\
    4 & 9 & 2 \\
  };

  \draw[thick,red,->] (magic-1-1) |- (magic-2-3);
\end{tikzpicture}
\end{codeexample}

  You may wish to add options to certain nodes in the matrix. This can
  be achieved in three ways.
  \begin{enumerate}
  \item You can modify, say, the
    |row 2 column 5| style to pass special options to this particular
    cell.

\begin{codeexample}[]
\begin{tikzpicture}[row 2 column 3/.style=red]
  \matrix [matrix of nodes]
  {
    8 & 1 & 6 \\
    3 & 5 & 7 \\
    4 & 9 & 2 \\
  };
\end{tikzpicture}
\end{codeexample}
    
  \item At the beginning of a cell, you can use a special syntax. If a
    cell starts with a vertical bar, then everything between this bar
    and the next bar is passed on to the |node| command.
{\catcode`\|=12
\begin{codeexample}[]
\begin{tikzpicture}
  \matrix [matrix of nodes]
  {
    8 & 1 &         6 \\
    3 & 5 & |[red]| 7 \\
    4 & 9 &         2 \\
  };
\end{tikzpicture}
\end{codeexample}
}
  You can also use an option like \verb!|[red] (seven)|! to give a
  different name to the node.

  Note that the |&| character also takes an optional argument, which
  is an extra column skip.
{\catcode`\|=12  
\begin{codeexample}[]
\begin{tikzpicture}
  \matrix [matrix of nodes]
  {
    8 &[1cm] 1 &[3mm] |[red]| 6 \\
    3 &      5 &      |[red]| 7 \\
    4 &      9 &              2 \\
  };
\end{tikzpicture}
\end{codeexample}
}
  \item If your cell starts with a |\path| command or any command that
    expands to |\path|, which includes |\draw|, |\node|, |\fill| and
    others, the |\node{| startup code and the |};| code are
    suppressed. This means that for this particular cell you can
    provide a totally different contents.

\begin{codeexample}[]
\begin{tikzpicture}
  \matrix [matrix of nodes]
  {
    8 & 1 & 6 \\
    3 & 5 & \node[red]{7}; \draw(0,0) circle(10pt);\\
    4 & 9 & 2 \\
  };
\end{tikzpicture}
\end{codeexample}
  \end{enumerate}
\end{key}

\begin{key}{/tikz/matrix of math nodes}
  This style is almost the same as the previous style, only |$| is %$
  added at the beginning and at the end of each node, so math mode
  will be switched on in all nodes.
{\catcode`\|=12
\begin{codeexample}[]
\begin{tikzpicture}
  \matrix [matrix of math nodes]
  {
    a_8 & a_1 &         a_6 \\
    a_3 & a_5 & |[red]| a_7 \\
    a_4 & a_9 &         a_2 \\
  };
\end{tikzpicture}
\end{codeexample}
}
\end{key}

\begin{key}{/tikz/nodes in empty cells=\meta{true or false} (default true)}
  When set to |true|, a node (with an empty contents) is put in empty
  cells. Normally, empty cells are just, well, empty. The style can be
  used together with both a |matrix of nodes| and a
  |matrix of math nodes|.
\begin{codeexample}[]
\begin{tikzpicture}
  \matrix [matrix of math nodes,nodes={circle,draw}]
  {
    a_8 &     & a_6 \\
    a_3 &     & a_7 \\
    a_4 & a_9 &     \\
  };
\end{tikzpicture}
\end{codeexample}
\begin{codeexample}[]
\begin{tikzpicture}
  \matrix [matrix of math nodes,nodes={circle,draw},nodes in empty cells]
  {
    a_8 &     & a_6 \\
    a_3 &     & a_7 \\
    a_4 & a_9 &     \\
  };
\end{tikzpicture}
\end{codeexample}
\end{key}


\subsection{End-of-Lines and End-of-Row Characters in Matrices of Nodes}

Special care must be taken about the usage of the |\\| command inside
a matrix of nodes. The reason is that this character is overloaded in
\TeX: On the one hand, it is used to denote the end of a line in
normal text; on the other hand it is used to denote the end of a row
in a matrix. Now, if a matrix contains node which in turn may have
multiple lines, it is unclear which meaning of |\\| should be used.

This problem arises only when you use the |text width| option of
nodes. Suppose you write a line like
\begin{codeexample}[code only]
\matrix [text width=5cm,matrix of nodes]
{
  first row & upper line \\ lower line \\
  second row & hmm \\
};
\end{codeexample}
This leaves \TeX\ trying to riddle out how many rows this matrix
should have. Do you want two rows with the upper right cell containing
a two-line text. Or did you mean a three row matrix with the second
row having only one cell?

Since \TeX\ is not clairvoyant, the following rules are used:
\begin{enumerate}
\item Inside a matrix, the |\\| command, by default, signals the end
  of the row, not the end of a line in a cell.
\item However, there is an exception to this rule: If a cell
  starts with a \TeX-group (this is, with |{|), % }
  then inside this first group the |\\| command retains the meaning of
  ``end of line'' character. Note that this special rule works only
  for the first group in a cell and this group must be at the
  beginning. 
\end{enumerate}

The net effect of these rules is the following: Normally, |\\| is an
end-of-row indicator; if you want to use it as an end-of-line
indicator in a cell, just put the whole cell in curly braces. The
following example illustrates the difference:
\begin{codeexample}[]
\begin{tikzpicture}
  \matrix [matrix of nodes,nodes={text width=16mm,draw}]
  {
    row 1 & upper line \\ lower line \\
    row 2 & hmm \\
  };
\end{tikzpicture}
\end{codeexample}
\begin{codeexample}[]
\begin{tikzpicture}
  \matrix [matrix of nodes,nodes={text width=16mm,draw}]
  {
    row 1 & {upper line \\ lower line} \\
    row 2 & hmm \\
  };
\end{tikzpicture}
\end{codeexample}

Note that this system is not fool-proof. If you write things like
|a&b{c\\d}\\| in a matrix of nodes, an error will result (because the
second cell did not start with a brace, so |\\| retained its normal
meaning and, thus, the second cell contained the text |b{c|, which is
not balanced with respect to the number of braces).

\subsection{Delimiters}

Delimiters are parantheses or braces to the left and right of a
formula or a matrix. The matrix library offers options for adding such
delimiters to a matrix. However, delimiters can actually be added to
any node that has the standard anchors |north|, |south|, |north west|
and so on. In particular, you can add delimiters to any |rectangle|
box. They are implemented by ``measuring the height'' of the node and
then adding a delimiter of the correct size to the left or right using
some after node magic.

\begin{key}{/tikz/left delimiter=\meta{delimiter}}
  This option can be given to a any node that has the standard anchors
  |north|, |south| and so on. The \meta{delimiter} can be any
  delimiter that is acceptable to \TeX's |\left| command.
\begin{codeexample}[]
\begin{tikzpicture}
  \matrix [matrix of math nodes,left delimiter=(,right delimiter=\}]
  {
    a_8 & a_1 & a_6 \\
    a_3 & a_5 & a_7 \\
    a_4 & a_9 & a_2 \\
  };
\end{tikzpicture}
\end{codeexample}

\begin{codeexample}[]
\begin{tikzpicture}
  \node [fill=red!20,left delimiter=(,right delimiter=\}]
    {$\displaystyle\int_0^1 x\,dx$};
\end{tikzpicture}
\end{codeexample}

  \begin{stylekey}{/tikz/every delimiter (initially \normalfont empyt)}
    This style is executed for every delimiter. You can use it to shift
    or color delimiters or do whatever.
  \end{stylekey}

  \begin{stylekey}{/tikz/every left delimiter (initially \normalfont empty)}
    This style is additionally executed for every left delimiter.
\begin{codeexample}[]
\begin{tikzpicture}
  [every left delimiter/.style={red,xshift=1ex},
   every right delimiter/.style={xshift=-1ex}]
  \matrix [matrix of math nodes,left delimiter=(,right delimiter=\}]
  {
    a_8 & a_1 & a_6 \\
    a_3 & a_5 & a_7 \\
    a_4 & a_9 & a_2 \\
  };
\end{tikzpicture}
\end{codeexample}
  \end{stylekey}
\end{key}

\begin{key}{/tikz/right delimiter=\meta{delimiter}}
  Works as above.
  \begin{stylekey}{/tikz/every right delimiter (initially \normalfont empty)}
    Works as above.
  \end{stylekey}
\end{key}


\begin{key}{/tikz/above delimiter=\meta{delimiter}}
  This option allows you to add a delimiter above the node. It is
  implementing by rotating a left delimiter.
\begin{codeexample}[]
\begin{tikzpicture}
  \matrix [matrix of math nodes,%
           left delimiter=\|,right delimiter=\rmoustache,%
           above delimiter=(,below delimiter=\}]
  {
    a_8 & a_1 & a_6 \\
    a_3 & a_5 & a_7 \\
    a_4 & a_9 & a_2 \\
  };
\end{tikzpicture}
\end{codeexample}

  \begin{stylekey}{/tikz/every above delimiter (initially \normalfont empty)}
    Works as above.
  \end{stylekey}
\end{key}

\begin{key}{/tikz/below delimiter=\meta{delimiter}}
  Works as above.
  \begin{stylekey}{/tikz/every below delimiter (initially \normalfont empty)}
    Works as above.
  \end{stylekey}
\end{key}



%%% Local Variables: 
%%% mode: latex
%%% TeX-master: "pgfmanual-pdftex-version"
%%% End: 

\section{Mindmap Drawing Library}

\begin{package}{pgflibrarytikzmindmap}
  This packages provides styles for drawing mindmap diagrams.
\end{package}

\subsection{Overview}

This library is intended to make the creation of mindmaps easier. A
\emph{mindmap} is a graphical representation of a concept together
with related concepts and annotations. Mindmaps are, essentially,
trees, possibly with a few extra edges added, but they are usually
drawn in a special way: The root concept is placed in the middle of
the page and is drawn as a huge circle, ellipse, or cloud. The related
concepts then ``leave'' this root concept in branch-like tendrils.

The mindmap library of \tikzname\ produces mindmaps that look a bit
different from the standard mindmaps: While the big root concept is
still a circle, related concepts are also depicted as (smaller)
circles. The related concepts are linked to the root concept via
organic-looking connections. The overall effect is visually rather
pleasing, but readers may not immediately think of a mindmap when they
see a picture created with this library.

Although it is not strictly necessary, you will usually create
mindmaps using \tikzname's tree mechanism and some of the styles and
macros of the package work best when used inside trees. However, it is
still possible and sometimes necessary to treat parts of a mindmap as
a graph with arbitrary edges and this is also possible.


\subsection{The Mindmap Style}

Every mindmap should be put in a scope or a picture where the
|mindmap| style is used. This style installs some internal settings.

\begin{itemize}
  \itemstyle{mindmap}
  Use this style with all pictures or at least scopes that contain a
  mindmap. It installs a whole bunch of settings that are useful for
  drawing mindmaps. 
\begin{codeexample}[]
\tikz[mindmap,concept color=red!50]
  \node [concept] {Root concept}
    child[grow=right] {node[concept] {Child concept}};
\end{codeexample}
  The sizes of concepts are predefined in such a way that a
  medium-size mindmap will fit on an A4 page (more or less).  
  \itemstyle{every mindmap}
  This style is included by the |mindmap| style. Change this style to
  add special settings to your mindmaps.
\begin{codeexample}[]
\tikz[large mindmap,concept color=red!50]
  \node [concept] {Root concept}
    child[grow=right] {node[concept] {Child concept}};
\end{codeexample}
  \itemstyle{large mindmap}
  This style includes the |mindmap| style, but additionally changes
  the default size of concepts and of distances so that a medium-sized
  mindmap will fit on an A3 page (A3 pages are twice as large as A4
  pages).
  \itemstyle{huge mindmap}
  This style causes conepts to be even bigger and it is best used with
  A2 paper and above.
\end{itemize}

\subsection{Concepts Nodes}

The basic entities of mindmaps are called \emph{concepts} in
\tikzname. A concept is a node of style |concept| and it must be
circular for some of the connection macros to work.


\subsubsection{Isolated Concepts}

The following styles influence how isolated concepts are rendered:

\begin{itemize}
  \itemstyle{concept}
  This style should be used with all nodes that are concepts, although
  some styles like |extra concept| install this style automatically.

  Bascially, this style makes the concept node circular and installs a
  uniform color called |concept color|, see below. Additionally, the
  style |every concept| is called.
\begin{codeexample}[]
\tikz[mindmap,concept color=red!50] \node [concept] {Some concept};
\end{codeexample}
  \itemstyle{every concept}
  In order to change the appearance of concept nodes, you should
  change this style. Note, however, that the color of a concept should
  be uniform for some of the connection bar stuff to work, so you
  should not change the color or the draw/fill state of concepts using
  this option. It is mostly useful for changing the text color and
  font.
  \itemoption{concept color}|=|\meta{color}
  This option tells \tikzname\ which color should be used for filling
  and stroking concepts. The difference between this option and just
  setting |every concept| to the desired color is that this option
  allows \tikzname\ to keep track of the colors used for
  concepts. This is important when you \emph{change} the color between
  two connected concepts. In this case, \tikzname\ can automatically
  create a shading that provides a smooth transition between the old
  and the new concept color; we will come back to this in the next
  section. 
  \itemstyle{extra concept}
  This style is intended for concepts that are not part of the
  ``mindmap tree,'' but stand beside it. Typically, they will have a
  subdued color are be smaller. In order to have these concepts appear
  in a uniform way and in order to indicate in the code that these
  concepts are extra, you can use this style.
\begin{codeexample}[]
\begin{tikzpicture}[mindmap,concept color=blue!80]
  \node [concept]                 {Root concept};
  \node [extra concept] at (10,0) {extra concept};
\end{tikzpicture}
\end{codeexample}
  \itemstyle{every extra concept}
  Change this style to change the appearance of extra concepts.
\end{itemize}


\subsubsection{Concepts in Trees}

As pointed out earlier, \tikzname\ assumes that your mindmap is build
using the |child| facilities of \tikzname. There are numerous options
that influence how concepts are rendered at the different levels of a
tree. 

\begin{itemize}
  \itemstyle{root concept}
  This style is used for the roots of mindmap trees. More precisely,
  this style is included by the |mindmap| style. Thus, by adding
  something to this, you can change how the root of a mindmap will be
  rendered.
\begin{codeexample}[]
\tikzstyle{root concept}+=[concept color=blue!80,minimum size=3.5cm]    
\tikz[mindmap] \node [concept] {Root concept};
\end{codeexample}

  Note that styles like |large mindmap| redefine these styles, so you
  should add something to this style only inside the picture.
  \itemstyle{level 1 concept}
  The |mindmap| style adds this style to the |level 1| style. This
  means that the first level children of a mindmap tree will use this
  style. 
\begin{codeexample}[]
\tikzstyle{root concept}+=[concept color=blue!80]    
\tikzstyle{level 1 concept}+=[concept color=red!50]    
\tikz[mindmap]
  \node [concept] {Root concept}
    child[grow=30] {node[concept] {child}}
    child[grow=0 ] {node[concept] {child}};
\end{codeexample}
  \itemstyle{level 2 concept}
  works like |level 1 concept|, only for second level children. 
  \itemstyle{level 3 concept}
  works like |level 1 concept|.
  \itemstyle{level 4 concept}
  works like |level 1 concept|. Note that there are not fifth and
  higher level styles, you need to modify |level 5| directly in such
  cases. 
  
  \itemoption{concept color}|=|\meta{color}
  We saw already that this option is used to change the color of
  concepts. We now have a look at its effect when used on child nodes
  of a concept. Normally, this option simply changes the color of the
  children. However, when the option is given as an option to the
  |child| operation (and not to the |node| operation and also not as
  an option to all children via the |level 1| style), \tikzname\ will
  smoothly change the concept color from the parent's color to the
  color of the child concept. 

  Here is an example:
\begin{codeexample}[]
\tikz[mindmap,concept color=blue!80]
  \node [concept] {Root concept}
    child[concept color=red,grow=30] {node[concept] {Child concept}}
    child[concept color=orange,grow=0]  {node[concept] {Child concept}};
\end{codeexample}

  In order to have all children of a certain level have a different
  concept color, a tiny bit of magic is needed:
\begin{codeexample}[]
\tikzstyle{root concept}+=[concept color=blue]    
\tikzstyle{level 1 concept}+=[set style={{every child}=[concept color=blue!50]}]    
\tikz[mindmap,text=white]
  \node [concept] {Root concept}
    child[grow=30] {node[concept] {child}}
    child[grow=0 ] {node[concept] {child}};
\end{codeexample}
\end{itemize}

\subsection{Connecting Concepts}

\subsection{Adding Annotations}

An \emph{annotation} is some text outside a mindmap that, unlike an
extra concept, simply explains something in the mindmap. The following
style is mainly intended to help readers of the code see that a node
in an annotation node.

\begin{itemize}
  \itemstyle{annotation}
  This style indicates that a node is an annotation node. It includes
  |every annotation|, which allows you to change this style in a
  convenient fashion.
\begin{codeexample}[]
\tikzstyle{every annotation}=[fill=red!20]    
\begin{tikzpicture}[mindmap,concept color=blue!80]
  \node [concept] (root)  {Root concept};

  \node [annotation,right] at (root.east)
  {The root concept is, in general, the most important concept.};
\end{tikzpicture}
\end{codeexample}
  \itemstyle{every annotation}
    This style is included by |annotation|.
\end{itemize}



%%% Local Variables: 
%%% mode: latex
%%% TeX-master: "pgfmanual-pdftex-version"
%%% End: 

% Copyright 2006 by Till Tantau
%
% This file may be distributed and/or modified
%
% 1. under the LaTeX Project Public License and/or
% 2. under the GNU Free Documentation License.
%
% See the file doc/generic/pgf/licenses/LICENSE for more details.


\section{Paper Folding Diagrams Library}

\label{section-calender-folding}

\begin{tikzlibrary}{folding}
  This library defines commands for creating paper folding
  diagrams. Many thanks to Nico Van Cleemput for providing the code
  for additional Platonic solids.
\end{tikzlibrary}

Here is a big example that produces a diagram for a calendar:

\begin{codeexample}[leave comments]
\sffamily\scriptsize
\begin{tikzpicture}
  [transform shape,
   every calendar/.style=
   {
     at={(-8ex,4ex)},
     week list,
     month label above centered,
     month text=\bfseries\textcolor{red}{\%mt} \%y0,
     if={(Sunday) [black!50]}
   }]
  \tikzfoldingdodecahedron
  [
    folding line length=2.5cm,
    face 1={ \calendar [dates=\the\year-01-01 to \the\year-01-last];},
    face 2={ \calendar [dates=\the\year-02-01 to \the\year-02-last];},
    face 3={ \calendar [dates=\the\year-03-01 to \the\year-03-last];},
    face 4={ \calendar [dates=\the\year-04-01 to \the\year-04-last];},
    face 5={ \calendar [dates=\the\year-05-01 to \the\year-05-last];},
    face 6={ \calendar [dates=\the\year-06-01 to \the\year-06-last];},
    face 7={ \calendar [dates=\the\year-07-01 to \the\year-07-last];},
    face 8={ \calendar [dates=\the\year-08-01 to \the\year-08-last];},
    face 9={ \calendar [dates=\the\year-09-01 to \the\year-09-last];},
    face 10={\calendar [dates=\the\year-10-01 to \the\year-10-last];},
    face 11={\calendar [dates=\the\year-11-01 to \the\year-11-last];},
    face 12={\calendar [dates=\the\year-12-01 to \the\year-12-last];}
  ];
\end{tikzpicture}
\end{codeexample}

The foldings are sorted by number of faces.


\begin{command}{\tikzfoldingtetrahedron|[|\meta{options}|];|}
  This command draws a folding diagram for a tetrahedron. The syntax
  is intended to remind of the |\path| command, but (currently) you
  must specify the \meta{options} and nothing else may be specified
  between the command name and the closing semicolon.

  The following keys may be used in the \meta{options}:
  \begin{key}{/tikz/folding line length=\meta{dimension}}
    Sets the length of the base line for folding. For the dodecahedron
    this is the length of all the sides of the pentagons.
  \end{key}
  \begin{key}{/tikz/face 1=\meta{code}}
    The \meta{code} is executed for the first face of the
    dodecahedron. When it is executed, the coordinate system will have
    been shifted and rotated such that it lies at the middle of the
    first face of the dodecahedron.
  \end{key}
  \begin{key}{/tikz/face 2=\meta{code}}
    Same as |face 1|, but for the second face.
  \end{key}
  \begin{key}{/tikz/face 3=\meta{code}}
    Same as |face 1|, but for the third face.
  \end{key}
  There are further similar options for up to 20 faces (for commands
  shown later).

  Here is a simple example:
\begin{codeexample}[]
\begin{tikzpicture}[transform shape]
  \tikzfoldingtetrahedron
  [folding line length=6mm,
  face 1={ \node[red] {1};},
  face 2={ \node      {2};},
  face 3={ \node      {3};},
  face 4={ \node      {4};}];
\end{tikzpicture}
\end{codeexample}

  The appearance of the cut and folding lines can be influenced using
  the following styles:
  \begin{stylekey}{/tikz/every cut (initially \normalfont empty)}
    Executed for every line that should be cut using scissors.
  \end{stylekey}
  \begin{stylekey}{/tikz/every fold (initially help lines)}
    Executed for every line that should be
    folded.
\begin{codeexample}[]
\begin{tikzpicture}[every cut/.style=red,every fold/.style=dotted]
  \tikzfoldingtetrahedron[folding line length=6mm];
\end{tikzpicture}
\end{codeexample}
  \end{stylekey}
\end{command}


\begin{command}{\tikzfoldingcube|[|\meta{options}|];|}
  A folding of a cube.
\begin{codeexample}[]
\begin{tikzpicture}[transform shape]
  \tikzfoldingcube
  [folding line length=6mm,
  face 1={ \node[red] {1};},
  face 2={ \node      {2};},
  face 3={ \node      {3};},
  face 4={ \node      {4};},
  face 5={ \node      {5};},
  face 6={ \node      {6};}];
\end{tikzpicture}
\end{codeexample}
\end{command}

\begin{command}{\tikzfoldingoctahedron|[|\meta{options}|];|}
  A folding of an octahedron.
\begin{codeexample}[]
\begin{tikzpicture}[transform shape]
  \tikzfoldingoctahedron
  [folding line length=6mm,
  face 1={ \node[red] {1};},
  face 2={ \node      {2};},
  face 3={ \node      {3};},
  face 4={ \node      {4};},
  face 5={ \node      {5};},
  face 6={ \node      {6};},
  face 7={ \node      {7};},
  face 8={ \node      {8};}];
\end{tikzpicture}
\end{codeexample}
\end{command}

\begin{command}{\tikzfoldingtruncatedtetrahedron|[|\meta{options}|];|}
  A folding of a truncated tetrahedron.
\begin{codeexample}[]
\begin{tikzpicture}[transform shape]
  \tikzfoldingtruncatedtetrahedron
  [folding line length=6mm,
  face 1={ \node[red] {1};},
  face 2={ \node      {2};},
  face 3={ \node      {3};},
  face 4={ \node      {4};},
  face 5={ \node      {5};},
  face 6={ \node      {6};},
  face 7={ \node      {7};},
  face 8={ \node      {8};}];
\end{tikzpicture}
\end{codeexample}
\end{command}

\begin{command}{\tikzfoldingdodecahedron|[|\meta{options}|];|}
  A folding of a dodecahedron.

\begin{codeexample}[]
\begin{tikzpicture}[transform shape]
  \tikzfoldingdodecahedron
  [folding line length=6mm,
  face 1={ \node[red] {1};},
  face 2={ \node      {2};},
  face 3={ \node      {3};},
  face 4={ \node      {4};},
  face 5={ \node      {5};},
  face 6={ \node      {6};},
  face 7={ \node      {7};},
  face 8={ \node      {8};},
  face 9={ \node      {9};},
  face 10={\node      {10};},
  face 11={\node      {11};},
  face 12={\node      {12};}];
\end{tikzpicture}
\end{codeexample}
\end{command}

\begin{command}{\tikzfoldingalternatedodecahedron|[|\meta{options}|];|}
  This is an alternative folding of a dodecahedron.
\begin{codeexample}[]
\begin{tikzpicture}[transform shape]
  \tikzfoldingalternatedodecahedron
  [folding line length=6mm,
  face 1={ \node[red] {1};},
  face 2={ \node      {2};},
  face 3={ \node      {3};},
  face 4={ \node      {4};},
  face 5={ \node      {5};},
  face 6={ \node      {6};},
  face 7={ \node      {7};},
  face 8={ \node      {8};},
  face 9={ \node      {9};},
  face 10={\node      {10};},
  face 11={\node      {11};},
  face 12={\node      {12};}];
\end{tikzpicture}
\end{codeexample}
\end{command}


\begin{command}{\tikzfoldingcuboctahedron|[|\meta{options}|];|}
  A folding of a cuboctahedron.
\begin{codeexample}[]
\begin{tikzpicture}[transform shape]
  \tikzfoldingcuboctahedron
  [folding line length=6mm,
  face 1={ \node[red] {1};},
  face 2={ \node      {2};},
  face 3={ \node      {3};},
  face 4={ \node      {4};},
  face 5={ \node      {5};},
  face 6={ \node      {6};},
  face 7={ \node      {7};},
  face 8={ \node      {8};},
  face 9={ \node      {9};},
  face 10={\node      {10};},
  face 11={\node      {11};},
  face 12={\node      {12};}];
\end{tikzpicture}
\end{codeexample}
\end{command}


\begin{command}{\tikzfoldingicosahedron|[|\meta{options}|];|}
  A folding of an icosahedron.
\begin{codeexample}[]
\begin{tikzpicture}[transform shape]
  \tikzfoldingicosahedron
  [folding line length=6mm,
  face 1={ \node[red] {1};},    face 11={ \node {11};},     
  face 2={ \node      {2};},    face 12={ \node {12};},     
  face 3={ \node      {3};},    face 13={ \node {13};},     
  face 4={ \node      {4};},    face 14={ \node {14};},     
  face 5={ \node      {5};},    face 15={ \node {15};},     
  face 6={ \node      {6};},    face 16={ \node {16};},     
  face 7={ \node      {7};},    face 17={ \node {17};},     
  face 8={ \node      {8};},    face 18={ \node {18};},     
  face 9={ \node      {9};},    face 19={ \node {19};},     
  face 10={\node      {10};},   face 20={ \node {20};}];
\end{tikzpicture}
\end{codeexample}
\end{command}

%%% Local Variables: 
%%% mode: latex
%%% TeX-master: "pgfmanual-pdftex-version"
%%% End: 

% Copyright 2003 by Till Tantau <tantau@cs.tu-berlin.de>.
%
% This program can be redistributed and/or modified under the terms
% of the LaTeX Project Public License Distributed from CTAN
% archives in directory macros/latex/base/lppl.txt.


\section{Pattern Library}
\label{section-library-patterns}

\begin{package}{pgflibrarypatterns}
  The package defines some useful patterns.
\end{package}

\begin{tabular}{ll}
  \emph{Pattern name} & \emph{Example} \\
  \patternindex{horizontal lines} \\
  \patternindex{vertical lines} \\
  \patternindex{north east lines} \\
  \patternindex{north west lines} \\
  \patternindex{grid} \\
  \patternindex{crosshatch} \\
  \patternindex{dots} \\
  \patternindex{crosshatch dots} \\
  \patternindex{fivepointed stars} \\
  \patternindex{sixpointed stars} \\
  \patternindex{bricks}
\end{tabular}
  


%%% Local Variables: 
%%% mode: latex
%%% TeX-master: "pgfmanual-pdftex-version"
%%% End: 

% Copyright 2006 by Till Tantau
%
% This file may be distributed and/or modified
%
% 1. under the LaTeX Project Public License and/or
% 2. under the GNU Free Documentation License.
%
% See the file doc/generic/pgf/licenses/LICENSE for more details.


\section{Petri-Net Drawing Library}

\begin{tikzlibrary}{petri}
  This packages provides shapes and styles for drawing Petri nets.
\end{tikzlibrary}



\subsection{Places}

The package defines a style for drawing places of Petri nets.

\begin{stylekey}{/tikz/place}
  This style indicates that a node is a place of a Petri net. Usually,
  the text of the node should be empty since places do not contain any
  text. You should use the |label| option to add text outside the node
  like its name or its capacity. You should use the |tokens| options,
  explained in Section~\ref{section-tokens}, to add tokens inside the
  place.

\begin{codeexample}[]
\begin{tikzpicture}
  \node[place,label=above:$p_1$,tokens=2]        (p1) {};
  \node[place,label=below:$p_2\ge1$,right=of p1] (p2) {};
\end{tikzpicture}
\end{codeexample}

  \begin{stylekey}{/tikz/every place}
    This style is evoked by the style |place|. To change the
    appearance of places, you can change this style.
\begin{codeexample}[]
\begin{tikzpicture}
  [every place/.style={draw=blue,fill=blue!20,thick,minimum size=9mm}]
  \node[place,tokens=7,label=above:$p_1$]  (p1) {};
  \node[place,structured tokens={3,2,9},
        label=below:$p_2\ge1$,right=of p1] (p2) {};
\end{tikzpicture}
\end{codeexample}
  \end{stylekey}
\end{stylekey}




\subsection{Transitions}

Transitions are also nodes. They should be drawn using the following
style:

\begin{stylekey}{/tikz/transition}
  This style indicates that a node is a transition. As for places, the
  text of a transition should be empty and the |label| option should
  be used for adding labels.

  To connect a transition to places, you can use the |edge| command as
  in the following example:

\begin{codeexample}[]
\begin{tikzpicture}
  \node[place,tokens=2,label=above:$p_1$]        (p1) {};
  \node[place,label=above:$p_2\ge1$,right=of p1] (p2) {};

  \node[transition,below right=of p1,label=below:$t_1$] {}
    edge[pre]                 (p1)
    edge[post] node[auto] {2} (p2);
\end{tikzpicture}
\end{codeexample}

  \begin{stylekey}{/tikz/every transition}
    This style is evoked by the style |transition|.
  \end{stylekey}

  \begin{stylekey}{/tikz/pre}
    This style can be used with paths leading \emph{from} a transition
    \emph{to} a place to indicate that the place is in the pre-set of
    the transition. By default, this style is |<-,shorten <=1pt|, but
    feel free to redefine it.
  \end{stylekey}

  \begin{stylekey}{/tikz/post}
    This style is also used with paths leading \emph{from} a transition
    \emph{to} a place, but this time the place is in the post-set of
    the transition. Again, feel free to redefine it.
  \end{stylekey}

  \begin{stylekey}{/tikz/pre and post}
    This style is to be used to indicate that a place is both in the
    pre- and post-set of a transition.
  \end{stylekey}
\end{stylekey}


\subsection{Tokens}
\label{section-tokens}

Interestingly, the most complicated aspect of drawing Petri nets in
\tikzname\ turns out to be the placement of tokens.

Let us start with a single token. They are also nodes and there is a
simple style for typesetting them.

\begin{stylekey}{/tikz/token}
  This style indicates that a node is a token. By default, this causes
  the node to be a small black circle. Unlike places and transitions,
  it \emph{does} make sense to provide text for the token node. Such
  text will be typeset in a tiny font and in white on black
  (naturally, you can easily change this by setting the style
  |every token|).

\begin{codeexample}[]
\begin{tikzpicture}
  \node[place,label=above:$p_1$]             (p1) {};
  \node[token] at (p1) {};

  \node[place,label=above:$p_2$,right=of p1] (p2) {};
  \node[token] at (p2) {$y$};
\end{tikzpicture}
\end{codeexample}
  \begin{stylekey}{/tikz/every token}
    Change this style to change the appearance of tokens.
  \end{stylekey}
\end{stylekey}

In the above example, it is bothersome that we need an extra command
for the token node. Worse, when we have \emph{two} tokens on a node,
it is difficult to place both nodes inside the node without overlap.

The Petri net library offers a solution to this problem: The
|children are tokens| style.


\begin{stylekey}{/tikz/children are tokens}
  The idea behind this style is to use trees mechanism for placing
  tokens. Every token lying on a place is treated as a child of the
  node. Normally this would have the effect that the tokens are placed
  below the place and they would be connected to the place by an
  edge. The |children are tokens| style, however, redefines the growth
  function of trees such that it places the children next to each
  other inside (or, rather, on top) of the place node. Additionally,
  the edge from the parent node is not drawn.
\begin{codeexample}[]
\begin{tikzpicture}
  \node[place,label=above:$p_1$] {}
  [children are tokens]
  child {node [token] {1}}
  child {node [token] {2}}
  child {node [token] {3}};
\end{tikzpicture}
\end{codeexample}

  In detail, what happens is the following: Tree growth functions tell
  \tikzname\ where it should place the children of nodes. These
  functions get passed the number of children that a node has an the
  number of the child that should be placed. The special tree growth
  function for tokens has a special mapping for each possible number
  of children up to nine children. This mapping decides for each child
  where it should be placed on top of the place. For example, a single
  child is placed directly on top of the place. Two children are
  placed next to each other, separated by the |token distance|. Three
  children are placed in a triangle whose side lengths are
  |token distance|; and so on up to nine tokens. If you wish to place
  more than nice tokens on a place, you will have to write your own
  placement code.
\begin{codeexample}[]
\begin{tikzpicture}
  \node[place,label=above:$p_2$] {}
  [children are tokens]
  child {node [token] {1}}
  child {node [token,fill=red] {2}}
  child {node [token,fill=red] {2}}
  child {node [token] {1}};
\end{tikzpicture}
\end{codeexample}

  \begin{key}{/tikz/token distance=\meta{distance}}
  This specifies the distance between the centers of the tokens in the
  arrangements of the option |children are tokens|.
\begin{codeexample}[]
\begin{tikzpicture}
  \node[place,label=above:$p_3$] {}
  [children are tokens,token distance=1.1ex]
  child {node [token] {}}
  child {node [token,red] {}}
  child {node [token,blue] {}}
  child {node [token] {}};
\end{tikzpicture}
\end{codeexample}
  \end{key}
\end{stylekey}

The |children are tokens| option gives you a lot of flexibility, but
it is a bit cumbersome to use. For this reason there are some options
that help in standard situations. They all use |children are tokens|
internally, so any change to, say, the |every token| style will
affect how these options depict tokens.

\begin{key}{/tikz/tokens=\meta{number}}
  This option is given to a |place| node, not to a |token| node. The
  effect of this option is to add \meta{number} many child nodes to
  the place, each having the style |token|. Thus, the following two
  pieces of codes have the same effect:
\begin{codeexample}[]
\tikz
  \node[place] {}
  [children are tokens]
  child {node [token] {}}
  child {node [token] {}}
  child {node [token] {}};
\tikz
  \node[place,tokens=3] {};
\end{codeexample}
  It is legal to say |tokens=0|, no tokens are drawn in this
  case. This option does not handle ten or more tokens correctly. If
  you need this many tokens, you will have to program your own code.
\begin{codeexample}[]
\begin{tikzpicture}[every place/.style={minimum size=9mm}]

  \foreach \x/\y/\tokennumber in {0/2/1,1/2/2,2/2/3,
                                  0/1/4,1/1/5,2/1/6,
                                  0/0/7,1/0/8,2/0/9}
    \node [place,tokens=\tokennumber] at (\x,\y) {};
\end{tikzpicture}
\end{codeexample}
\end{key}

\begin{key}{/tikz/colored tokens=\meta{color list}}
  This option, which must also be given when a place node is being
  created, gets a list of colors as parameter. It will then add as
  many tokens to the place as there are colors in this list, each filled correspondingly.
\begin{codeexample}[]
\tikz  \node[place,colored tokens={black,black,red,blue}] {};
\end{codeexample}
\end{key}
\begin{key}{/tikz/structured tokens=\meta{token texts}}
  This option, which must again be passed to a place, gets a list of 
  texts for tokens. For each text, a new token will be added to the place.
\begin{codeexample}[]
\tikz  \node[place,structured tokens={$x$,$y$,$z$}] {};
\end{codeexample}
\begin{codeexample}[]
\begin{tikzpicture}[every place/.style={minimum size=9mm}]

  \foreach \x/\y/\tokennumber in {0/2/1,1/2/2,2/2/3,
                                  0/1/4,1/1/5,2/1/6,
                                  0/0/7,1/0/8,2/0/9}
    \node [place,structured tokens={1,...,\tokennumber}] at (\x,\y) {};
\end{tikzpicture}
\end{codeexample}
  If you use lots of structured tokens, consider redefining the
  |every token| style so that the tokens are larger.
\end{key}


\subsection{Examples}


\begin{codeexample}[]
\begin{tikzpicture}[yscale=-1.6,xscale=1.5,thick,
  every transition/.style={draw=red,fill=red!20,minimum size=3mm},
  every place/.style={draw=blue,fill=blue!20,minimum size=6mm}]

  \foreach \i in {1,...,6} {
    \node[place,label=left:$p_\i$] (p\i) at (0,\i) {};
    \node[place,label=right:$q_\i$] (q\i) at (8,\i) {};
  }
  \foreach \name/\var/\vala/\valb/\height/\x in
      {m1/m_1/f/t/2.25/3,m2/m_2/f/t/2.25/5,h/\mathit{hold}/1/2/4.5/4} {
    \node[place,label=above:{$\var = \vala$}] (\name\vala) at (\x,\height) {};
    \node[place,yshift=-8mm,label=below:{$\var = \valb$}] (\name\valb) at (\x,\height) {};
  }
  \node[token] at (p1) {};   \node[token] at (q1) {};
  \node[token] at (m1f) {};  \node[token] at (m2f) {};
  \node[token] at (h1) {};

  \node[transition] at (1.5,1.5) {}  edge [pre] (p1)  edge [post] (p2);
  \node[transition] at (1.5,2.5) {}  edge [pre] (p2)  edge[pre]   (m1f)
                                     edge [post](p3)  edge[post]  (m1t);
  \node[transition] at (1.5,3.3) {}  edge [pre] (p3)  edge [post] (p4)
                                     edge [pre and post] (h1);
  \node[transition] at (1.5,3.7) {}  edge [pre] (p3)  edge [pre] (h2)
                                     edge [post] (p4) edge [post] (h1.west);
  \node[transition] at (1.5,4.3) {}  edge [pre] (p4)  edge [post] (p5)
                                     edge [pre and post] (m2f);
  \node[transition] at (1.5,4.7) {}  edge [pre] (p4)  edge [post] (p5)
                                     edge [pre and post] (h2);
  \node[transition] at (1.5,5.5) {}  edge [pre] (p5)  edge [pre] (m1t)
                                     edge [post] (p6) edge [post] (m1f);
  \node[transition] at (1.5,6.5) {}  edge [pre] (p6)  edge [post] (p1.south east);
  \node[transition] at (6.5,1.5) {}  edge [pre] (q1)  edge [post] (q2);
  \node[transition] at (6.5,2.5) {}  edge [pre] (q2)  edge [pre] (m2f)
                                     edge [post] (q3) edge [post] (m2t);
  \node[transition] at (6.5,3.3) {}  edge [pre] (q3)  edge [post] (q4)
                                     edge [pre and post] (h2);
  \node[transition] at (6.5,3.7) {}  edge [pre] (q3)  edge [pre] (h1)
                                     edge [post] (q4) edge [post] (h2.east);
  \node[transition] at (6.5,4.3) {}  edge [pre] (q4)  edge [post] (q5)
                                     edge [pre and post] (m1f);
  \node[transition] at (6.5,4.7) {}  edge [pre] (q4)  edge [post] (q5)
                                     edge [pre and post] (h1);
  \node[transition] at (6.5,5.5) {}  edge [pre] (q5)  edge [pre] (m2t)
                                     edge [post] (q6) edge [post] (m2f);
  \node[transition] at (6.5,6.5) {}  edge [pre] (q6)  edge [post] (q1.south west);
\end{tikzpicture}
\end{codeexample}

Here is the same net once more, but with these styles changes:
\begin{codeexample}[code only]
\begin{tikzpicture}[yscale=-1.1,thin,>=stealth,
  every transition/.style={fill,minimum width=1mm,minimum height=3.5mm},
  every place/.style={draw,thick,minimum size=6mm}]
\end{codeexample}

\begin{tikzpicture}[yscale=-1.1,thin,>=stealth,
  every transition/.style={fill,minimum width=1mm,minimum height=3.5mm},
  every place/.style={draw,thick,minimum size=6mm}]

  \foreach \i in {1,...,6} {
    \node[place,label=left:$p_\i$] (p\i) at (0,\i) {};
    \node[place,label=right:$q_\i$] (q\i) at (8,\i) {};
  }
  \foreach \name/\var/\vala/\valb/\height/\x in
      {m1/m_1/f/t/2.25/3,m2/m_2/f/t/2.25/5,h/\mathit{hold}/1/2/4.5/4} {
    \node[place,label=above:{$\var = \vala$}] (\name\vala) at (\x,\height) {};
    \node[place,yshift=-8mm,label=below:{$\var = \valb$}] (\name\valb) at (\x,\height) {};
  }
  \node[token] at (p1) {};   \node[token] at (q1) {};
  \node[token] at (m1f) {};  \node[token] at (m2f) {};
  \node[token] at (h1) {};

  \node[transition] at (1.5,1.5) {}  edge [pre] (p1)  edge [post] (p2);
  \node[transition] at (1.5,2.5) {}  edge[pre] (p2)   edge[pre] (m1f)
                                     edge[post] (p3)  edge[post] (m1t);
  \node[transition] at (1.5,3.3) {}  edge [pre] (p3)  edge [post] (p4)
                                     edge [pre and post] (h1);
  \node[transition] at (1.5,3.7) {}  edge [pre] (p3)  edge [pre] (h2)
                                     edge [post] (p4) edge [post] (h1.west);
  \node[transition] at (1.5,4.3) {}  edge [pre] (p4)  edge [post] (p5)
                                     edge [pre and post] (m2f);
  \node[transition] at (1.5,4.7) {}  edge [pre] (p4)  edge [post] (p5)
                                     edge [pre and post] (h2);
  \node[transition] at (1.5,5.5) {}  edge [pre] (p5)  edge [pre] (m1t)
                                     edge [post] (p6) edge [post] (m1f);
  \node[transition] at (1.5,6.5) {}  edge [pre] (p6)  edge [post] (p1.south east);
  \node[transition] at (6.5,1.5) {}  edge [pre] (q1)  edge [post] (q2);
  \node[transition] at (6.5,2.5) {}  edge [pre] (q2)  edge [pre] (m2f)
                                     edge [post] (q3) edge [post] (m2t);
  \node[transition] at (6.5,3.3) {}  edge [pre] (q3)  edge [post] (q4)
                                     edge [pre and post] (h2);
  \node[transition] at (6.5,3.7) {}  edge [pre] (q3)  edge [pre] (h1)
                                     edge [post] (q4) edge [post] (h2.east);
  \node[transition] at (6.5,4.3) {}  edge [pre] (q4)  edge [post] (q5)
                                     edge [pre and post] (m1f);
  \node[transition] at (6.5,4.7) {}  edge [pre] (q4)  edge [post] (q5)
                                     edge [pre and post] (h1);
  \node[transition] at (6.5,5.5) {}  edge [pre] (q5)  edge [pre] (m2t)
                                     edge [post] (q6) edge [post] (m2f);
  \node[transition] at (6.5,6.5) {}  edge [pre] (q6)  edge [post] (q1.south west);
\end{tikzpicture}

%%% Local Variables:
%%% mode: latex
%%% TeX-master: "pgfmanual-pdftex-version"
%%% End:

% Copyright 2006 by Till Tantau
%
% This file may be distributed and/or modified
%
% 1. under the LaTeX Project Public License and/or
% 2. under the GNU Free Documentation License.
%
% See the file doc/generic/pgf/licenses/LICENSE for more details.


\section{Plot Handler Library}
\label{section-library-plothandlers}

\begin{pgflibrary}{plothandlers}
  This library packages defines additional plot handlers, see
  Section~\ref{section-plot-handlers} for an introduction to plot
  handlers. The additional handlers are described in the following.

  This library is loaded automatically by \tikzname.
\end{pgflibrary}


\subsection{Curve Plot Handlers}
  
\begin{command}{\pgfplothandlercurveto}
  This handler will issue a |\pgfpathcurveto| command for each point of
  the plot, \emph{except} possibly for the first. As for the line-to
  handler, what happens with the first point can be specified using
  |\pgfsetmovetofirstplotpoint| or |\pgfsetlinetofirstplotpoint|.

  Obviously, the |\pgfpathcurveto| command needs, in addition to the
  points on the path, some control points. These are generated
  automatically using a somewhat ``dumb'' algorithm: Suppose you have
  three points $x$, $y$, and $z$ on the curve such that $y$ is between
  $x$ and $z$:
\begin{codeexample}[]
\begin{tikzpicture}    
  \draw[gray] (0,0) node {x} (1,1) node {y} (2,.5) node {z};
  \pgfplothandlercurveto
  \pgfplotstreamstart
  \pgfplotstreampoint{\pgfpoint{0cm}{0cm}}
  \pgfplotstreampoint{\pgfpoint{1cm}{1cm}}
  \pgfplotstreampoint{\pgfpoint{2cm}{.5cm}}
  \pgfplotstreamend
  \pgfusepath{stroke}
\end{tikzpicture}
\end{codeexample}

  In order to determine the control points of the curve at the point
  $y$, the handler computes the vector $z-x$ and scales it by the
  tension factor (see below). Let us call the resulting vector
  $s$. Then $y+s$ and $y-s$ will be the control points around $y$. The
  first control point at the beginning of the curve will be the
  beginning itself, once more; likewise the last control point is the
  end itself.
\end{command}

\begin{command}{\pgfsetplottension\marg{value}}
  Sets the factor used by the curve plot handlers to determine the
  distance of the control points from the points they control. The
  higher the curvature of the curve points, the higher this value
  should be. A value of $1$ will cause four points at quarter
  positions of a circle to be connected using a circle. The default is
  $0.5$. 

\begin{codeexample}[]
\begin{tikzpicture}    
  \draw[gray] (0,0) node {x} (1,1) node {y} (2,.5) node {z};
  \pgfsetplottension{0.75}
  \pgfplothandlercurveto
  \pgfplotstreamstart
  \pgfplotstreampoint{\pgfpoint{0cm}{0cm}}
  \pgfplotstreampoint{\pgfpoint{1cm}{1cm}}
  \pgfplotstreampoint{\pgfpoint{2cm}{0.5cm}}
  \pgfplotstreamend
  \pgfusepath{stroke}
\end{tikzpicture}
\end{codeexample}
\end{command}


\begin{command}{\pgfplothandlerclosedcurve}
  This handler works like the curve-to plot handler, only it will
  add a new part to the current path that is a closed curve through
  the plot points.
\begin{codeexample}[]
\begin{tikzpicture}    
  \draw[gray] (0,0) node {x} (1,1) node {y} (2,.5) node {z};
  \pgfplothandlerclosedcurve
  \pgfplotstreamstart
  \pgfplotstreampoint{\pgfpoint{0cm}{0cm}}
  \pgfplotstreampoint{\pgfpoint{1cm}{1cm}}
  \pgfplotstreampoint{\pgfpoint{2cm}{0.5cm}}
  \pgfplotstreamend
  \pgfusepath{stroke}
\end{tikzpicture}
\end{codeexample}
\end{command}

\subsection{Constant Plot Handlers}
There are three plot handlers which produce piecewise constant interpolations between successive points.

\begin{command}{\pgfplothandlerconstantlineto}
  This handler works like the line-to plot handler, only it will
  produce a connected, piecewise constant path to connect the points.
\begin{codeexample}[]
\begin{tikzpicture}    
  \draw[gray] (0,0) node {x} (1,1) node {y} (2,.5) node {z};
  \pgfplothandlerconstantlineto
  \pgfplotstreamstart
  \pgfplotstreampoint{\pgfpoint{0cm}{0cm}}
  \pgfplotstreampoint{\pgfpoint{1cm}{1cm}}
  \pgfplotstreampoint{\pgfpoint{2cm}{0.5cm}}
  \pgfplotstreamend
  \pgfusepath{stroke}
\end{tikzpicture}
\end{codeexample}
\end{command}

\begin{command}{\pgfplothandlerconstantlinetomarkright}
A variant of |\pgfplothandlerconstantlineto| which places its mark on
the right line ends.
\begin{codeexample}[]
\begin{tikzpicture}    
  \draw[gray] (0,0) node {x} (1,1) node {y} (2,.5) node {z};
  \pgfplothandlerconstantlinetomarkright
  \pgfplotstreamstart
  \pgfplotstreampoint{\pgfpoint{0cm}{0cm}}
  \pgfplotstreampoint{\pgfpoint{1cm}{1cm}}
  \pgfplotstreampoint{\pgfpoint{2cm}{0.5cm}}
  \pgfplotstreamend
  \pgfusepath{stroke}
\end{tikzpicture}
\end{codeexample}
\end{command}

\begin{command}{\pgfplothandlerjumpmarkleft}
  This handler works like the line-to plot handler, only it will
  produce a non-connected, piecewise constant path to connect the points. If there are any plot marks, they will be placed on the left open pieces.
\begin{codeexample}[]
\begin{tikzpicture}    
  \draw[gray] (0,0) node {x} (1,1) node {y} (2,.5) node {z};
  \pgfplothandlerjumpmarkleft
  \pgfplotstreamstart
  \pgfplotstreampoint{\pgfpoint{0cm}{0cm}}
  \pgfplotstreampoint{\pgfpoint{1cm}{1cm}}
  \pgfplotstreampoint{\pgfpoint{2cm}{0.5cm}}
  \pgfplotstreamend
  \pgfusepath{stroke}
\end{tikzpicture}
\end{codeexample}
\end{command}

\begin{command}{\pgfplothandlerjumpmarkright}
  This handler works like the line-to plot handler, only it will
  produce a non-connected, piecewise constant path to connect the points. If there are any plot marks, they will be placed on the right open pieces.
\begin{codeexample}[]
\begin{tikzpicture}    
  \draw[gray] (0,0) node {x} (1,1) node {y} (2,.5) node {z};
  \pgfplothandlerjumpmarkright
  \pgfplotstreamstart
  \pgfplotstreampoint{\pgfpoint{0cm}{0cm}}
  \pgfplotstreampoint{\pgfpoint{1cm}{1cm}}
  \pgfplotstreampoint{\pgfpoint{2cm}{0.5cm}}
  \pgfplotstreamend
  \pgfusepath{stroke}
\end{tikzpicture}
\end{codeexample}
\end{command}

\subsection{Comb Plot Handlers}

There are three ``comb'' plot handlers. There name stems from the fact
that the plots they produce look like ``combs'' (more or less).

\begin{command}{\pgfplothandlerxcomb}
  This handler converts each point in the plot stream into a line from
  the $y$-axis to the point's coordinate, resulting in a ``horizontal
  comb.''

  
\begin{codeexample}[]
\begin{tikzpicture}    
  \draw[gray] (0,0) node {x} (1,1) node {y} (2,.5) node {z};
  \pgfplothandlerxcomb
  \pgfplotstreamstart
  \pgfplotstreampoint{\pgfpoint{0cm}{0cm}}
  \pgfplotstreampoint{\pgfpoint{1cm}{1cm}}
  \pgfplotstreampoint{\pgfpoint{2cm}{0.5cm}}
  \pgfplotstreamend
  \pgfusepath{stroke}
\end{tikzpicture}
\end{codeexample}
\end{command}


\begin{command}{\pgfplothandlerycomb}
  This handler converts each point in the plot stream into a line from
  the $x$-axis to the point's coordinate, resulting in a ``vertical
  comb.''
  
  This handler is useful for creating ``bar diagrams.''
\begin{codeexample}[]
\begin{tikzpicture}    
  \draw[gray] (0,0) node {x} (1,1) node {y} (2,.5) node {z};
  \pgfplothandlerycomb
  \pgfplotstreamstart
  \pgfplotstreampoint{\pgfpoint{0cm}{0cm}}
  \pgfplotstreampoint{\pgfpoint{1cm}{1cm}}
  \pgfplotstreampoint{\pgfpoint{2cm}{0.5cm}}
  \pgfplotstreamend
  \pgfusepath{stroke}
\end{tikzpicture}
\end{codeexample}
\end{command}

\begin{command}{\pgfplothandlerpolarcomb}
  This handler converts each point in the plot stream into a line from
  the origin to the point's coordinate.
  
\begin{codeexample}[]
\begin{tikzpicture}    
  \draw[gray] (0,0) node {x} (1,1) node {y} (2,.5) node {z};
  \pgfplothandlerpolarcomb
  \pgfplotstreamstart
  \pgfplotstreampoint{\pgfpoint{0cm}{0cm}}
  \pgfplotstreampoint{\pgfpoint{1cm}{1cm}}
  \pgfplotstreampoint{\pgfpoint{2cm}{0.5cm}}
  \pgfplotstreamend
  \pgfusepath{stroke}
\end{tikzpicture}
\end{codeexample}
\end{command}

\subsubsection{Changing the start of comb/bar plots}
\pgfname\ bar or comb plots usually draw something from zero to the current plot's coordinate.

The `zero' offset can be changed using an input stream which returns the desired offset successively for each processed coordinate.

There are two such streams which can be configured independently.
The first one returns `zeros' for coordinate~$x$, the second one
returns `zeros' for coordinate~$y$. They are used as follows.

\begin{codeexample}[code only]
\pgfplotxzerolevelstreamstart
\pgfplotxzerolevelstreamnext % assigns \pgf@x
\pgfplotxzerolevelstreamnext
\pgfplotxzerolevelstreamnext
\pgfplotxzerolevelstreamend
\end{codeexample}
%
\begin{codeexample}[code only]
\pgfplotyzerolevelstreamstart
\pgfplotyzerolevelstreamnext % assigns \pgf@x
\pgfplotyzerolevelstreamend
\end{codeexample}
Different zero level streams can be implemented by overwriting these macros.

\begin{command}{\pgfplotxzerolevelstreamconstant\marg{dimension}}
	This zero level stream always returns \marg{dimension} instead of $x=0$pt.

	It is used for |xcomb| and |xbar|.
\end{command}

\begin{command}{\pgfplotyzerolevelstreamconstant\marg{dimension}}
	This zero level stream always returns \marg{dimension} instead of $y=0$pt.

	It is used for |ycomb| and |ybar|.
\end{command}

\subsection{Bar Plot Handlers}
\label{section-plotlib-bar-handlers}

While comb plot handlers produce a line-to operation to generate combs, bar plot handlers employ rectangular shapes, allowing filled bars (or pattern bars).

\begin{command}{\pgfplothandlerybar}
  This handler converts each point in the plot stream into a rectangle from
  the $x$-axis to the point's coordinate. The rectangle is placed centered at the $x$-axis.
  
\begin{codeexample}[]
\begin{tikzpicture}    
  \draw[gray] (0,0) node {x} (1,1) node {y} (2,.5) node {z};
  \pgfplothandlerybar
  \pgfplotstreamstart
  \pgfplotstreampoint{\pgfpoint{0cm}{0cm}}
  \pgfplotstreampoint{\pgfpoint{1cm}{1cm}}
  \pgfplotstreampoint{\pgfpoint{2cm}{0.5cm}}
  \pgfplotstreamend
  \pgfusepath{stroke}
\end{tikzpicture}
\end{codeexample}
\end{command}

\begin{command}{\pgfplothandlerxbar}
  This handler converts each point in the plot stream into a rectangle from
  the $y$-axis to the point's coordinate. The rectangle is placed centered at the $y$-axis.
  
\begin{codeexample}[]
\begin{tikzpicture}    
  \draw[gray] (0,0) node {x} (1,1) node {y} (2,.5) node {z};
  \pgfplothandlerxbar
  \pgfplotstreamstart
  \pgfplotstreampoint{\pgfpoint{0cm}{0cm}}
  \pgfplotstreampoint{\pgfpoint{1cm}{1cm}}
  \pgfplotstreampoint{\pgfpoint{2cm}{0.5cm}}
  \pgfplotstreamend
  \pgfusepath{stroke}
\end{tikzpicture}
\end{codeexample}
\end{command}

\label{key-bar-width}%
\begin{key}{/pgf/bar width=\marg{dimension} (initially 10pt)}
	\keyalias{tikz}
	Sets the width of |\pgfplothandlerxbar| and |\pgfplothandlerybar| to \marg{dimension}. The argument \marg{dimension} will be evaluated using the math parser.
\end{key}

\label{key-bar-shift}%
\begin{key}{/pgf/bar shift=\marg{dimension} (initially 0pt)}
	\keyalias{tikz}
	Sets a shift used by |\pgfplothandlerxbar| and |\pgfplothandlerybar| to \marg{dimension}. It has the same effect as |xshift|, but it applies only to those bar plots. The argument \marg{dimension} will be evaluated using the math parser.
\end{key}

\begin{command}{\pgfplotbarwidth}
	Expands to the value of |/pgf/bar width|.
\end{command}


\begin{command}{\pgfplothandlerybarinterval}
  This handler is a variant of |\pgfplothandlerybar| which works with intervals instead of points.
  
  Bars are drawn between successive input coordinates and the width is determined relatively to the interval length.
  
\begin{codeexample}[]
\begin{tikzpicture}    
  \draw[gray] (0,2) node {$x_1$} (1,1) node {$x_2$} (2,.5) node {$x_3$} (4,0.7) node {$x_4$};
  \pgfplothandlerybarinterval
  \pgfplotstreamstart
  \pgfplotstreampoint{\pgfpoint{0cm}{2cm}}
  \pgfplotstreampoint{\pgfpoint{1cm}{1cm}}
  \pgfplotstreampoint{\pgfpoint{2cm}{0.5cm}}
  \pgfplotstreampoint{\pgfpoint{4cm}{0.7cm}}
  \pgfplotstreamend
  \pgfusepath{stroke}
\end{tikzpicture}
\end{codeexample}

In more detail, if $(x_i,y_i)$ and $(x_{i+1},y_{i+1})$ denote successive input coordinates, the bar will be placed above the interval $[x_i,x_{i+1}]$, centered at
\[ x_i + \text{\meta{bar interval shift}} \cdot (x_{i+1} - x_i) \]
with width
\[ \text{\meta{bar interval width}} \cdot (x_{i+1} - x_i). \]
Here, \meta{bar interval shift} and \meta{bar interval width} denote the current values of |/pgf/bar interval shift| and |/pgf/bar interval width|.

If you have $N+1$ input points, you will get $N$ bars (one for each interval). The $y$~value of the last point will be ignored.
\end{command}

\begin{command}{\pgfplothandlerxbarinterval}
   As |\pgfplothandlerybarinterval|, this handler provides bar plots with relative bar sizes and offsets, one bar for each $y$~coordinate interval.
\end{command}

\label{key-bar-interval-shift}%
\begin{key}{/pgf/bar interval shift=\marg{shift} (initially 0.5)}
	\keyalias{tikz}
	Sets the \emph{relative} shift of |\pgfplothandlerxbarinterval| and |\pgfplothandlerybarinterval| to \marg{shift}. As |/pgf/bar interval width|, the argument is relative to the interval length of the input coordinates.
	
	The argument \marg{scale} will be evaluated using the math parser.
\end{key}

\label{key-bar-interval-width}%
\begin{key}{/pgf/bar interval width=\marg{scale} (initially 1)}
	\keyalias{tikz}
	Sets the \emph{relative} width of |\pgfplothandlerxbarinterval| and |\pgfplothandlerybarinterval| to \marg{scale}. The argument is relative to $(x_{i+1} - x_i)$ for $y$~bar plots and relative to $(y_{i+1}-y_i)$ for $x$~bar plots.
	
	The argument \marg{scale} will be evaluated using the math parser.
\begin{codeexample}[]
\begin{tikzpicture}[bar interval width=0.5]  
  \draw[gray] 
  	(0,3) -- (0,-0.1) 
    (1,3) -- (1,-0.1)
    (2,3) -- (2,-0.1)
    (4,3) -- (4,-0.1);
  \pgfplothandlerybarinterval
  \begin{scope}[bar interval shift=0.25,fill=blue]
  \pgfplotstreamstart
  \pgfplotstreampoint{\pgfpoint{0cm}{2cm}}
  \pgfplotstreampoint{\pgfpoint{1cm}{1cm}}
  \pgfplotstreampoint{\pgfpoint{2cm}{0.5cm}}
  \pgfplotstreampoint{\pgfpoint{4cm}{0.7cm}}
  \pgfplotstreamend
  \pgfusepath{fill}
  \end{scope}
  \begin{scope}[bar interval shift=0.75,fill=red]
  \pgfplotstreamstart
  \pgfplotstreampoint{\pgfpoint{0cm}{3cm}}
  \pgfplotstreampoint{\pgfpoint{1cm}{0.2cm}}
  \pgfplotstreampoint{\pgfpoint{2cm}{0.7cm}}
  \pgfplotstreampoint{\pgfpoint{4cm}{0.2cm}}
  \pgfplotstreamend
  \pgfusepath{fill}
  \end{scope}
\end{tikzpicture}
\end{codeexample}
Please note that bars are always centered, so we have to use shifts $0.25$ and $0.75$ instead of $0$ and $0.5$.
\end{key}

\subsection{Mark Plot Handler}

\label{section-plot-marks}

\begin{command}{\pgfplothandlermark\marg{mark code}}
  This command will execute the \meta{mark code} for some points of the
  plot, but each time the coordinate transformation matrix will be
  setup such that the origin is at the position of the point to be
  plotted. This way, if the \meta{mark code} draws a little circle
  around the origin, little circles will be drawn at some point of the
  plot.

  By default, a mark is drawn at all points of the plot. However, two
  parameters $r$ and $p$ influence this. First, only every $r$th mark
  is drawn. Second, the first mark drawn is the $p$th. These
  parameters can be influenced using the commands below.
  
\begin{codeexample}[]
\begin{tikzpicture}    
  \draw[gray] (0,0) node {x} (1,1) node {y} (2,.5) node {z};
  \pgfplothandlermark{\pgfpathcircle{\pgfpointorigin}{4pt}\pgfusepath{stroke}}
  \pgfplotstreamstart
  \pgfplotstreampoint{\pgfpoint{0cm}{0cm}}
  \pgfplotstreampoint{\pgfpoint{1cm}{1cm}}
  \pgfplotstreampoint{\pgfpoint{2cm}{0.5cm}}
  \pgfplotstreamend
  \pgfusepath{stroke}
\end{tikzpicture}
\end{codeexample}

  Typically, the \meta{code} will be |\pgfuseplotmark{|\meta{plot mark
      name}|}|, where \meta{plot mark name} is the name of a
  predefined plot mark.
\end{command}

\begin{command}{\pgfsetplotmarkrepeat\marg{repeat}}
  Sets the $r$ parameter to \meta{repeat}, that is, only every $r$th
  mark will be drawn.
\end{command}

\begin{command}{\pgfsetplotmarkphase\marg{phase}}
  Sets the $p$ parameter to \meta{phase}, that is, the first mark to
  be drawn is the $p$th, followed by the $(p+r)$th, then the
  $(p+2r)$th, and so on.
\end{command}

\begin{command}{\pgfplothandlermarklisted\marg{mark code}\marg{index list}}
  This command works similar to the previous one. However, marks will
  only be placed at those indices in the given \meta{index list}. The
  syntax for the list is the same as for the |\foreach| statement. For
  example, if you provide the list |1,3,...,25|, a mark will be placed
  only at every second point. Similarly, |1,2,4,8,16,32| yields marks
  only at those points that are powers of two.
  
\begin{codeexample}[]
\begin{tikzpicture}    
  \draw[gray] (0,0) node {x} (1,1) node {y} (2,.5) node {z};
  \pgfplothandlermarklisted
    {\pgfpathcircle{\pgfpointorigin}{4pt}\pgfusepath{stroke}}
    {1,3}
  \pgfplotstreamstart
  \pgfplotstreampoint{\pgfpoint{0cm}{0cm}}
  \pgfplotstreampoint{\pgfpoint{1cm}{1cm}}
  \pgfplotstreampoint{\pgfpoint{2cm}{0.5cm}}
  \pgfplotstreamend
  \pgfusepath{stroke}
\end{tikzpicture}
\end{codeexample}
\end{command}

\begin{command}{\pgfuseplotmark\marg{plot mark name}}
  Draws the given \meta{plot mark name} at the origin. The \meta{plot
    mark name} must previously have been declared using
  |\pgfdeclareplotmark|. 

\begin{codeexample}[]
\begin{tikzpicture}    
  \draw[gray] (0,0) node {x} (1,1) node {y} (2,.5) node {z};
  \pgfplothandlermark{\pgfuseplotmark{pentagon}}
  \pgfplotstreamstart
  \pgfplotstreampoint{\pgfpoint{0cm}{0cm}}
  \pgfplotstreampoint{\pgfpoint{1cm}{1cm}}
  \pgfplotstreampoint{\pgfpoint{2cm}{0.5cm}}
  \pgfplotstreamend
  \pgfusepath{stroke}
\end{tikzpicture}
\end{codeexample}
\end{command}

\begin{command}{\pgfdeclareplotmark\marg{plot mark name}\marg{code}}
  Declares a plot mark for later used with the |\pgfuseplotmark|
  command.

\begin{codeexample}[]
\pgfdeclareplotmark{my plot mark}
  {\pgfpathcircle{\pgfpoint{0cm}{1ex}}{1ex}\pgfusepathqstroke}  
\begin{tikzpicture}    
  \draw[gray] (0,0) node {x} (1,1) node {y} (2,.5) node {z};
  \pgfplothandlermark{\pgfuseplotmark{my plot mark}}
  \pgfplotstreamstart
  \pgfplotstreampoint{\pgfpoint{0cm}{0cm}}
  \pgfplotstreampoint{\pgfpoint{1cm}{1cm}}
  \pgfplotstreampoint{\pgfpoint{2cm}{0.5cm}}
  \pgfplotstreamend
  \pgfusepath{stroke}
\end{tikzpicture}
\end{codeexample}
\end{command}


\begin{command}{\pgfsetplotmarksize\marg{dimension}}
  This command sets the \TeX\ dimension |\pgfplotmarksize| to
  \meta{dimension}. This dimension is a ``recommendation'' for plot
  mark code at which size the plot mark should be drawn; plot mark
  code may choose to ignore this \meta{dimension} altogether. For
  circles, \meta{dimension} should  be the radius, for other shapes it
  should be about half the width/height.

  The predefined plot marks all take this dimension into account.

\begin{codeexample}[]
\begin{tikzpicture}    
  \draw[gray] (0,0) node {x} (1,1) node {y} (2,.5) node {z};
  \pgfsetplotmarksize{1ex}
  \pgfplothandlermark{\pgfuseplotmark{*}}
  \pgfplotstreamstart
  \pgfplotstreampoint{\pgfpoint{0cm}{0cm}}
  \pgfplotstreampoint{\pgfpoint{1cm}{1cm}}
  \pgfplotstreampoint{\pgfpoint{2cm}{0.5cm}}
  \pgfplotstreamend
  \pgfusepath{stroke}
\end{tikzpicture}
\end{codeexample}
\end{command}

\begin{textoken}{\pgfplotmarksize}
  A \TeX\ dimension that is a ``recommendation'' for the size of plot
  marks.
\end{textoken}

The following plot marks are predefined (the filling color has been
set to yellow):

\medskip
\begin{tabular}{lc}
  \plotmarkentry{*}
  \plotmarkentry{x}
  \plotmarkentry{+}
\end{tabular}


%%% Local Variables: 
%%% mode: latex
%%% TeX-master: "pgfmanual-pdftex-version"
%%% End: 

% Copyright 2003 by Till Tantau <tantau@cs.tu-berlin.de>.
%
% This program can be redistributed and/or modified under the terms
% of the LaTeX Project Public License Distributed from CTAN
% archives in directory macros/latex/base/lppl.txt.


\section{Plot Mark Library}

\begin{pgflibrary}{plotmarks}
  This library defines a number of plot marks.
\end{pgflibrary}

This library defines the following plot marks in
addition to |*|, |x|, and |+| (the filling color has been set to a
dark yellow):

{
\catcode`\|=12
\medskip
\begin{tabular}{lc}
  \plotmarkentry{-}
  \index{*vbar@\protect\texttt{\protect\myvbar} plot mark}%
  \index{Plot marks!*vbar@\protect\texttt{\protect\myvbar}}
  \texttt{\char`\\pgfuseplotmark\char`\{\declare{|}\char`\}} &
  \tikz\draw[color=black!25] plot[mark=|,mark options={fill=yellow,draw=black}]
  coordinates {(0,0) (.5,0.2) (1,0) (1.5,0.2)};\\
  \plotmarkentry{o}
  \plotmarkentry{asterisk}
  \plotmarkentry{star}
  \plotmarkentry{oplus}
  \plotmarkentry{oplus*}
  \plotmarkentry{otimes}
  \plotmarkentry{otimes*}
  \plotmarkentry{square}
  \plotmarkentry{square*}
  \plotmarkentry{triangle}
  \plotmarkentry{triangle*}
  \plotmarkentry{diamond}
  \plotmarkentry{diamond*}
  \plotmarkentry{pentagon}
  \plotmarkentry{pentagon*}
\end{tabular}
}


%%% Local Variables: 
%%% mode: latex
%%% TeX-master: "pgfmanual-pdftex-version"
%%% End: 

% Copyright 2010 by Christian Feuersaenger
%
% This file may be distributed and/or modified
%
% 1. under the LaTeX Project Public License and/or
% 2. under the GNU Free Documentation License.
%
% See the file doc/generic/pgf/licenses/LICENSE for more details.

\section{Profiler Library}
{\noindent {\emph{by Christian Feuers\"anger}}}

\begin{pgflibrary}{profiler}
	A library to simplify the optimization of runtime speed of \TeX\ programs.

	It relies on the |pdftex| primitive
        \declareandlabel{\pdfelapsedtime}\footnote{The primitive is
          emulated in lua\TeX.} to count (fractional) seconds and counts total time and self time for macro invocations.
\end{pgflibrary}

\subsection{Overview}
The intended audience for this library are people writing \TeX\ code which should be optimized. It is certainly \emph{not} useful for the end-user.

The work flow for the optimization is simple: the preamble contains configuration commands like
\begin{codeexample}[code only]
\usepgflibrary{profiler}
\pgfprofilenewforenvironment{tikzpicture}
\pgfprofilenewforcommand{\pgfkeys}{1}
\end{codeexample}
\noindent and then, the time between |\begin{tikzpicture}| and |\end{tikzpicture}| and the time required to call |\pgfkeys| will be collected.

At the end, a short usage summary like
\begin{codeexample}[code only]
 pgflibraryprofiler(main job) {total time=1.07378sec; (100.0122%) self time=0.034sec; (3.1662%)}
 pgflibraryprofiler(<ENV>tikzpicture) {total time=1.03978sec; (96.84601%) self time=1.00415sec; (93.52722%)}
 pgflibraryprofiler(<CS>pgfkeys) {total time=0.03563sec; (3.31726%) self time=0.03563sec; (3.31726%)}
\end{codeexample}
\noindent will be provided in the log file, furthermore, the same information is available in a text table called |\jobname.profiler.|\meta{datetime}|.dat| which is of the form:
\begin{codeexample}[code only]
profilerentry       totaltime[s]        totaltime[percent]  selftime[s]         selftime[percent]   
main job            1.07378             100.0122            0.034               3.1662              
<ENV>tikzpicture    1.03978             96.84601            1.00415             93.52722            
<CS>pgfkeys         0.03563             3.31726             0.03563             3.31726             
\end{codeexample}
Here, the |totaltime| means the time used for all invocations of the respective profiler entry (one row in the table). The |selftime| measures time which is not already counted for in another profiler entry which has been invoked within the current one. The example above is not very exciting: the main job consists only of several (quite complex) pictures and nothing else. Thus, its total time is large. However, the self time is very small because the |tikzpicture|s are counted separately, and they have been invoked within the |main job|. The |\pgfkeys| control sequence has been invoked within the |tikzpicture|, that's why the |selftime| for the |tikzpicture| is a little bit smaller than its |totaltime|.

\subsection{Requirements}
The library works with |pdftex| and |luatex|. Furthermore, it requires
a more or less recent version of |pdftex| which supports the |\pdfelapsedtime| directive.

\subsection{Defining Profiler Entries}
Unlike profilers for C/C++ or java, this library doesn't extract information about every \TeX\ macro automatically, nor does it collect information for each of them. Instead, every profiler entry needs to be defined explicitly. Only defined profiler entries will be processed.

\begin{command}{\pgfprofilenew\marg{name}}
	Defines a new profiler entry named \meta{name}.

	This updates a set of internal registers used to track the profiler entry. The \meta{name} can be arbitrary, it doesn't need to be related to any \TeX\ macro.

	The actual job of counting seconds is accomplished using |\pgfprofilestart|\marg{name} followed eventually by the command |\pgfprofileend|\marg{name}.

	It doesn't hurt if |\pgfprofilenew| is called multiple times with the same name.
\end{command}

\begin{command}{\pgfprofilenewforcommand\oarg{profiler entry name}\marg{\textbackslash macro}\marg{arguments}}
	Defines a new profiler entry which will measure the time spent in \meta{\textbackslash macro}. This calls |\pgfprofilenew| and replaces the current definition of \meta{\textbackslash macro} with a new one. 
	
	If \oarg{profiler entry name} has been provided, this defines the argument for |\pgfprofilenew|. It is allowed to use the same name for multiple commands; in this case, they are treated as if it where the same command. If the optional argument is not used, the profiler entry will be called `\declareandlabel{\pgfprofilecs}\meta{macro}' (\meta{macro} without backslash) where |\pgfprofilecs| is predefined to be |<CS>|. 
	
	The replacement macro will collect all required arguments, start counting, invoke the original macro definition and stop counting.

	The following macro types are supported within |\pgfprofilenewforcommand|:
	\begin{itemize}
 \item commands which take one (optional) argument in square brackets
   followed by one optional argument which has to be delimited by
   curly braces (use an empty argument for \meta{arguments} in this case),
 \item commands which take one (optional) argument in square brackets
   and \emph{exactly} \meta{arguments} arguments afterwards.
	\end{itemize}

	Take a look at |\pgfprofilenewforcommandpattern| in case you have more complicated commands.


	Note that the library can't detect if a command has been redefined somewhere.
\end{command}

\begin{command}{\pgfprofilenewforcommandpattern\oarg{profiler entry name}\marg{\textbackslash macro}\marg{argument pattern}\marg{invocation pattern}}
	A variant of |\pgfprofilenewforcommand| which can be used with arbitrary \meta{argument patterns}. Example:
\begin{codeexample}[code only]
\def\mymacro#1\to#2\in#3{ ... }
\pgfprofilenewforcommandpattern{\mymacro}{#1\to#2\in#3}{{#1}\to{#2}\in{#3}}	
\end{codeexample}

Note that |\pgfprofilenewforcommand| is a special case of |\pgfprofilenewforcommandpattern|:
\begin{codeexample}[code only]
\def\mymacro#1#2{ ... }
\pgfprofilenewforcommand\macro{2}
\pgfprofilenewforcommandpattern{\mymacro}{#1#2}{{#1}{#2}}	
\end{codeexample}
	Thus, \meta{argument pattern} is a copy-paste from the definition of your command. The \meta{invocation pattern} is used by the profiler library to invoke the \emph{original} command, so it is closely related to \meta{argument pattern}, but it needs extra curly braces around each argument.

	The behavior of |\pgfprofilenewforcommandpattern| is the same as discussed above: it defines a new profiler entry which will measure the time spent in \meta{\textbackslash macro}. The details about this definition has already been described. Note that up to one optional argument in square brackets is also checked automatically.
	

	If you like to profile a command which doesn't match here for whatever reasons, you'll have to redefine it manually and insert |\pgfprofilestart| and |\pgfprofileend| in appropriate places.
\end{command}

\begin{command}{\pgfprofileshowinvocationsfor\marg{profiler entry name}}
	Enables verbose output for \emph{every} invocation of \meta{profiler entry name}.

	This is only available for profiler entries for commands (those created by |\pgfprofilenewforcommand| for example). It will also show all given arguments.
\end{command}
\begin{command}{\pgfprofileshowinvocationsexpandedfor\marg{profiler entry name}}
	A variant of |\pgfprofileshowinvocationsfor| which will expand all arguments for \meta{profiler entry name} before showing them. The invocation as such is not affected by this expansion.

	This expansion (with |\edef|) might yield unrecoverable errors for some commands. Handle with care.
\end{command}

\begin{command}{\pgfprofilenewforenvironment\oarg{profiler entry name}\marg{environment name}}
	Defines a new profiler entry which measures time spent in the environment \meta{environment name}. 

	This calls |\pgfprofilenew| and handles the begin/end of the environment automatically.
	
	The argument for |\pgfprofilenew| is \meta{profiler entry name}, or, if this optional argument is not used, it is `\declareandlabel{\pgfprofileenv}\meta{environment name}' where |\pgfprofileenv| is predefined as |<ENV>|. Again, it is permitted to use the same \meta{profiler entry name} multiple times to merge different commands into one output section.

\end{command}


\begin{command}{\pgfprofilestart\marg{profiler entry name}}
	Starts (or resumes) timing of \meta{profiler entry name}. The argument must have been declared in the preamble using |\pgfprofilenew|.

	Nested calls of |\pgfprofilestart| with the same argument will be ignored.

	The invocation of this command doesn't change the environment: it doesn't introduce any \TeX\ groups nor does it modify the token list.
\end{command}

\begin{command}{\pgfprofileend\marg{profiler entry name}}
	Stops (or interrupts) timing of \meta{profiler entry name}. 
	
	This command finishes a preceding call to |\pgfprofilestart|.
\end{command}

\begin{command}{\pgfprofilepostprocess}
	For \LaTeX, this command is installed automatically in |\end{document}|. It stops all running timings, evaluates them and returns the result into the logfile. Furthermore, it generates a text table called |\jobname.profiler.|\meta{YYYY}|-|\meta{MM}|-|\meta{DD}|_|\meta{HH}|h_|\meta{MM}|m.dat| with the same information.

	Note that the profiler library predefines two profiler entries, namely |main job| which counts time from the beginning of the document until |\pgfprofilepostprocess| and |preamble| which counts time from the beginning of the document until |\begin{document}|.
\end{command}

\begin{command}{\pgfprofilesetrel\marg{profiler entry name} (initially main job)}
	Sets the profiler entry whose total time will be used to compute all other relative times. Thus, \meta{profiler entry name} will use $100\%$ of the total time per definition, all other relative times are relative to this one.
\end{command}

\begin{command}{\pgfprofileifisrunning\marg{profiler entry name}\marg{true code}\marg{false code}}
	Invokes \marg{true code} if \marg{profiler entry name} is currently running and \marg{false code} otherwise.
\end{command}

% Copyright 2006 by Till Tantau
%
% This file may be distributed and/or modified
%
% 1. under the LaTeX Project Public License and/or
% 2. under the GNU Free Documentation License.
%
% See the file doc/generic/pgf/licenses/LICENSE for more details.


\section{Shadings Library}
\label{section-library-shadings}

\begin{pgflibrary}{shadings}
  The package defines a number of shadings in addition to the ball and
  axis shadings that are available by default.
\end{pgflibrary}

In the following, the shadings defined in the library are listed in
alphabetical order. The colors of some of these shadings can be
configured using special options (like |left color|). These options
implicitly select the shading.

The three shadings |axis|, |ball|, and |radial| are always defined,
even when this library is not used.


\begin{shading}{axis}
  In this always-defined shading the colors change gradually
  between three horizontal lines. The top line is at the top
  (uppermost) point of the path, the middle is in the middle, the
  bottom line is at the bottom of the path.
  
  \begin{key}{/tikz/top color=\meta{color}}
    This option prescribes the color to be used at the top in an |axis|
    shading. When this option is given, several things happen:
    \begin{enumerate}
    \item
      The |shade| option is selected.
    \item
      The |shading=axis| option is selected.
    \item
      The middle color of the axis shading is set to the average of the
      given top color \meta{color} and of whatever color is currently
      selected for the bottom.
    \item
      The rotation angle of the shading is set to 0.
  \end{enumerate}

\begin{codeexample}[]
\tikz \draw[top color=red] (0,0) rectangle (2,1);
\end{codeexample}
  \end{key}  

  \begin{key}{/tikz/bottom color=\meta{color}}
    This option works like |top color|, only for the bottom color.
  \end{key}

  \begin{key}{/tikz/middle color=\meta{color}}
    This option specifies the color for the middle of an axis
    shading. It also sets the |shade| and |shading=axis| options, but it
    does not change the rotation angle.
    
    \emph{Note:} Since both |top color| and |bottom color| change the
    middle color, this option should be given \emph{last} if all of
    these options need to be given:
    
\begin{codeexample}[]
\tikz \draw[top color=white,bottom color=black,middle color=red]
  (0,0) rectangle (2,1);
\end{codeexample}  
  \end{key}

  \begin{key}{/tikz/left color=\meta{color}}
    This option does exactly the same as |top color|, except that the
    shading angle is set to $90^\circ$.
  \end{key}
  
  \begin{key}{/tikz/right color=\meta{color}}
    Works like |left color|.
  \end{key}
\end{shading}


\begin{shading}{ball}
  This always-defined shading fills the path with a shading that ``looks like a
  ball.'' The default ``color'' of the ball is blue (for no
  particular reason).
  
  \begin{key}{/tikz/ball color=\meta{color}}
    This option sets the color used for the ball shading. It sets the
    |shade| and |shading=ball| options. Note that the ball will never
    ``completely'' have the color \meta{color}. At its ``highlight'' spot
    a certain amount of white is mixed in, at the border a certain
    amount of black. Because of this, it also makes sense to say
    |ball color=white| or |ball color=black|
    
\begin{codeexample}[]
\begin{tikzpicture}
  \shade[ball color=white] (0,0) circle (2ex);
  \shade[ball color=red] (1,0) circle (2ex);
  \shade[ball color=black] (2,0) circle (2ex);
\end{tikzpicture}
\end{codeexample}
  \end{key}
\end{shading}



\begin{shading}{bilinear interpolation}
  This shading fills a rectangle with colors that a bilinearly
  interpolated between the colors in the four corners of the
  rectangle. These four colors are called |lower left|, |lower right|,
  |upper left|, and |upper right|. By changing these color, you can
  change the way the shading looks. The library also defines four
  options, called the same way, that can be used to set these colors
  and select the shading implicitly.

\begin{codeexample}[]
\tikz
  \shade[upper left=red,upper right=green,
         lower left=blue,lower right=yellow]
    (0,0) rectangle (3,2);
\end{codeexample}

  \begin{key}{/tikz/lower left=\meta{color} (initially white)}
    Sets the color to be used in a |bilinear interpolation| shading
    for the lower left corner. Also, this options selects this shading
    and sets the |shade| option.
  \end{key}

  \begin{key}{/tikz/upper left=\meta{color} (initially white)}
    Like |lower left|.
  \end{key}
  \begin{key}{/tikz/upper right=\meta{color} (initially white)}
    Like |lower left|.
  \end{key}
  \begin{key}{/tikz/lower left=\meta{color} (initially white)}
    Like |lower left|.
  \end{key}
\end{shading}


\begin{shading}{color wheel}
  \label{shading-color-wheel}
  This shading fills the path with a color wheel.
\begin{codeexample}[]
\tikz \shade[shading=color wheel] (0,0) circle (1.5);
\end{codeexample}
  To produce a color ring, cut out a circle from the color wheel:
\begin{codeexample}[]
\tikz \shade[shading=color wheel] [even odd rule]
  (0,0) circle (1.5)
  (0,0) circle (1);
\end{codeexample}
\end{shading}


\begin{shading}{color wheel black center}
  This shading looks like a color wheel, but the brightness drops to
  zero in the center.
\begin{codeexample}[]
\tikz \shade[shading=color wheel black center] (0,0) circle (1.5);
\end{codeexample}
\end{shading}


\begin{shading}{color wheel white center}
  This shading looks like a color wheel, but the saturation drops to
  zero in the center.
\begin{codeexample}[]
\tikz \shade[shading=color wheel white center] (0,0) circle (1.5);
\end{codeexample}
\end{shading}



\begin{shading}{Mandelbrot set}
  This shading is just for fun. It fills the path with a zoomable
  Mandelbrot set. Note that this is \emph{not} a bitmap
  graphic. Rather, the Mandelbrot set is \emph{computed by the
    \textsc{pdf} renderer} and can be zoomed arbitrarily (give it a
  try, if you have a fast computer).

\begin{codeexample}[]
\tikz \shade[shading=Mandelbrot set] (0,0) rectangle (2,2);
\end{codeexample}
\end{shading}



\begin{shading}{radial}
  This always-defined shading fills the path with a gradual sweep from
  a certain color in the middle to another color at the border. If the path is
  a circle, the outer color will be reached exactly at the
  border. If the shading is not a circle, the outer color will
  continue a bit towards the corners. The default inner color is
  gray, the default outer color is white.
  
  \begin{key}{/tikz/inner color=\meta{color}}
    This option sets the color used at the center of a |radial|
    shading. When this option is used, the |shade| and |shading=radial|
    options are set.
  
\begin{codeexample}[]
\tikz \draw[inner color=red] (0,0) rectangle (2,1);
\end{codeexample}
  \end{key}

  \begin{key}{/tikz/outer color=\meta{color}}
    This option sets the color used at the border and outside of a
    |radial| shading.
  
\begin{codeexample}[]
\tikz \draw[outer color=red,inner color=white]
  (0,0) rectangle (2,1);
\end{codeexample}
  \end{key}
\end{shading}


%%% Local Variables: 
%%% mode: latex
%%% TeX-master: "pgfmanual-pdftex-version"
%%% End: 

% Copyright 2007 by Till Tantau and Mark Wibrow
%
% This file may be distributed and/or modified
%
% 1. under the LaTeX Project Public License and/or
% 2. under the GNU Free Documentation License.
%
% See the file doc/generic/pgf/licenses/LICENSE for more details.

\section{Shadow Library}
\label{section-libs-shadows}

\begin{pgflibrary}{shadows}
  This library defines styles that help adding a (partly) transparent
  shadow to a path or node.
\end{pgflibrary}


\subsection{Overview}

A \emph{shadow} is usually a black or gray area that is drawn behind a
path or a node, thereby adding visual depth to a picture. The shadows
library defines options that make it easy to add shadows to
paths. Internally, these options are based on using the |preaction|
option to use a path twice: Once for drawing the shadow (slightly
shifted) and once for actually using the path.

Note that you can only add shadows to \emph{paths}, not to whole
scopes.

In addition to the general |shadow| option, there exist special
options like |circular shadow|. These can only (sensibly) be used with
a special kind of path (for |circular shadow|, a circle) and, thus,
there are not as general. The advantage is, however, that they are
more visually pleasing since these shadows blend smoothly with the
background. Note that these special shadows use fadings, which few
printers will support.


\subsection{The General Shadow Option}

The shadows are internally created by using a single option called
|general shadow|. The different options like |drop shadow| or
|copy shadow| only differ in the commands that they preset.

You will not need to use this option directly under normal
circumstances.


\begin{key}{/tikz/general shadow=\meta{shadow options} (default \normalfont empty)}
  This option should be given to a |\path| or a |node|. It has the
  following effect: Before the path is used normally, it is used once
  with the \meta{shadow options} in force. Furthermore, when the path
  is ``preused'' in this way, it is shifted and scaled a little bit.

  In detail, the following happens: A |preaction| is used to
  paint the path in a special manner before it is actually
  painted. This ``special'' manner is as follows: The options in
  \meta{shadow options} are used for painting this path. Typically,
  the \meta{shadow options} will contain options like |fill=black| to
  create, say, a black shadow. Furthermore, after the \meta{shadow
    options} have been setup, the following extra canvas
  transformations are applied to the path: It is scaled by
  |shadow scale| (with the origin of scaling at the path's center) and
  it is shifted by |shadow xshift| and |shadow yshift|. 

  Note that since scaling and shifting is done using canvas
  transformations, shadows are not taken into account when the
  picture's bounding box is computed.
\begin{codeexample}[]
\tikz [even odd rule]
  \draw [general shadow={fill=red}] (0,0) circle (.5) (0.5,0) circle (.5);  
\end{codeexample}
  
  \begin{key}{/tikz/shadow scale=\meta{factor} (initially 1)}
    Shadows are scaled by this amount.
\begin{codeexample}[]
\tikz [even odd rule]
  \draw [general shadow={fill=red,shadow scale=1.25}]
    (0,0) circle (.5) (0.5,0) circle (.5);  
\end{codeexample}
  \end{key}
  \begin{key}{/tikz/shadow xshift=\meta{factor} (initially 0pt)}
    Shadows are shifted horizontally by this amount.
\begin{codeexample}[]
\tikz [even odd rule]
  \draw [general shadow={fill=red,shadow xshift=-5pt}]
    (0,0) circle (.5) (0.5,0) circle (.5);  
\end{codeexample}
  \end{key}
  \begin{key}{/tikz/shadow yshift=\meta{factor} (initially 0pt)}
    Shadows are shifted vertically by this amount.
  \end{key}
\end{key}



\subsection{Shadows for Arbitrary Paths and Shapes}

\subsubsection{Drop Shadows}

\begin{key}{/tikz/drop shadow=\meta{shadow options} (default \normalfont empty)}
  This option adds a drop shadow to a path or node. |\path| or a
  |node|. It uses the |general shadow| and passes the \meta{shadow
    options} to it plus, before them, the following extra options:
\begin{codeexample}[code only]
  shadow scale=1, shadow xshift=.5ex, shadow yshift=-.5ex,
  opacity=.5, fill=black!50, every shadow
\end{codeexample}

\begin{codeexample}[]
\tikz [even odd rule]
  \filldraw [drop shadow,fill=white] (0,0) circle (.5) (0.5,0) circle (.5);  
\end{codeexample}
\begin{codeexample}[]
\begin{tikzpicture}
  \foreach \i in {1,...,4}
    \node[starburst,drop shadow,fill=white,draw] at (0,\i) {Burst \i};
\end{tikzpicture}
\end{codeexample}

\begin{codeexample}[]
\begin{tikzpicture}
  \draw [help lines] (0,0) grid (3,2);
  \filldraw [drop shadow={opacity=0.25},fill=white]
    (1,.5) circle (.5) (1.5,.5) circle (.5);  

  \filldraw [drop shadow={opacity=1},fill=white]
    (1,2)  circle (.5) (1.5,2)  circle (.5);  
\end{tikzpicture}
\end{codeexample}
\end{key}

\begin{stylekey}{/tikz/every shadow (initially \normalfont empty)}
  This style is executed in addition to any \meta{shadow options} for
  each shadow. Use this style to reconfigure the way shadows are
  drawn.
\begin{codeexample}[]
\begin{tikzpicture}[every shadow/.style={opacity=.8,fill=blue!50!black}]    
  \filldraw [drop shadow,fill=white] (0,0) circle (.5) (0.5,0) circle (.5);  
\end{tikzpicture}
\end{codeexample}
\end{stylekey}



\subsubsection{Copy Shadows}

A \emph{copy shadow} is not really a shadow. Rather, it looks like
another copy of the path drawn behind the path and a little bit
offset. This creates the visual impression of having multiple copies
of the path/object present.

\begin{key}{/tikz/copy shadow=\meta{shadow options} (default \normalfont empty)}
  This shadow installs the following default options:
\begin{codeexample}[code only]
  shadow scale=1, shadow xshift=.5ex, shadow yshift=-.5ex, every shadow
\end{codeexample}
  Furthermore, the options |fill=|\meta{fill color} and
  |draw=|\meta{draw color} are also set, where the \meta{fill color}
  and \meta{draw color} are the fill and draw colors used for the main 
  path. 
\begin{codeexample}[]
\begin{tikzpicture}
  \node [copy shadow,fill=blue!20,draw=blue,thick] {Hello World!};

  \node at (0,-1) [copy shadow={shadow xshift=1ex,shadow yshift=1ex},
                   fill=blue!20,draw=blue,thick]
    {Hello World!};

  \node at (0,-2) [copy shadow={opacity=.5},tape,
                   fill=blue!20,draw=blue,thick]
    {Hello World!};

  % We have to repeat the left color since shadings are not
  % automatically applied to shadows
  \node at (0,-3) [copy shadow={left color=blue!50},
                   left color=blue!50,draw=blue,thick]
    {Hello World!};
\end{tikzpicture}
\end{codeexample}
\end{key}

\begin{key}{/tikz/double copy shadow=\meta{shadow options} (default \normalfont empty)}
  This shadow works like a |copy shadow|, only the shadow is added
  twice, the first time with the double |xshift| and |yshift|.
\begin{codeexample}[]
\begin{tikzpicture}
  \node [double copy shadow,fill=blue!20,draw=blue,thick] {Hello World!};

  \node at (0,-1) [double copy shadow={shadow xshift=1ex,shadow yshift=1ex},
                   fill=blue!20,draw=blue,thick]
    {Hello World!};

  \node at (0,-2) [double copy shadow={opacity=.5},tape,
                   fill=blue!20,draw=blue,thick]
    {Hello World!};

  \node at (0,-3) [double copy shadow={left color=blue!50},
                   left color=blue!50,draw=blue,thick]
    {Hello World!};
\end{tikzpicture}
\end{codeexample}
\end{key}


\subsection{Shadows for Special Paths and Nodes}

The shadows in this section should normally be added only to paths
that have a special shape. They will look strange with other shapes.

\begin{key}{/tikz/circular drop shadow=\meta{shadow options}}
  This shadow works like a drop shadow, only it adds a circular
  fading to the shadow. This means that the shadow will fade out at
  the border. The following options are preset for this shadow:
\begin{codeexample}[code only]
  shadow scale=1.1, shadow xshift=.3ex, shadow yshift=-.3ex,
  fill=black, path fading={circle with fuzzy edge 15 percent},
  every shadow,
\end{codeexample}

\begin{codeexample}[]
\begin{tikzpicture}
  \foreach \i in {1,...,8}
    \node[circle,circular drop shadow,draw=blue,fill=blue!20,thick]
      at (\i*45:1) {Circle \i};
\end{tikzpicture}
\end{codeexample}
\end{key}


\begin{key}{/tikz/circular glow=\meta{shadow options}}
  This shadow works much like the |circular shadow|, only it is not
  shifted. This creates a visual effect of a ``glow'' behind the
  circle.  The following options are preset for this shadow:
\begin{codeexample}[code only]
  shadow scale=1.25, shadow xshift=0pt, shadow yshift=0pt,
  fill=black, path fading={circle with fuzzy edge 15 percent},
  every shadow,
\end{codeexample}

\begin{codeexample}[]
\begin{tikzpicture}
  \foreach \i in {1,...,8}
  \node[circle,circular glow,fill=red!20,draw=red,thick]
    at (\i*45:1) {Circle \i};
\end{tikzpicture}
\end{codeexample}
\begin{codeexample}[]
\begin{tikzpicture}
  \foreach \i in {1,...,8}
  \node[circle,circular glow={fill=white},fill=red!20,draw=red,thick]
    at (\i*45:1) {Circle \i};
\end{tikzpicture}
\end{codeexample}
\begin{codeexample}[]
\begin{tikzpicture}
  \foreach \i in {1,...,8}
  \node[circle,circular glow={fill=green},fill=black,text=green!50!black]
    at (\i*45:1) {Circle \i};
\end{tikzpicture}
\end{codeexample}
  An especially interesting effect can be achieved by only using the
  glow and not filling the path:
\begin{codeexample}[]
\begin{tikzpicture}
  \foreach \i in {1,...,8}
  \node[circle,circular glow={fill=red!\i0}]
    at (\i*45:1) {Circle \i};
\end{tikzpicture}
\end{codeexample}
\end{key}


%%% Local Variables: 
%%% mode: latex
%%% TeX-master: "pgfmanual-pdftex-version"
%%% End: 

% Copyright 2007 by Till Tantau and Mark Wibrow
%
% This file may be distributed and/or modified
%
% 1. under the LaTeX Project Public License and/or
% 2. under the GNU Free Documentation License.
%
% See the file doc/generic/pgf/licenses/LICENSE for more details.


\section{Shape Library}
\label{section-libs-shapes}


\subsection{Overview}

In addition to the standard shapes |rectangle|, |circle| and
|coordinate|, there exist a number of additional shapes defined in
different shape libraries. Most of these shapes have been 
contributed by Mark Wibrow. In the present section, these shapes are
described. Note that the library |shapes| is provided for
compatibility only. Please include sublibraries like
|shapes.geometric| or |shapes.misc| directly.

The appearance of shapes is influenced by numerous parameters like
|minimum height| or |inner xsep|. These general parameters are documented in
Section~\ref{section-shape-common-options} 


\subsection{Predefined Shapes}
\label{section-predefined-shapes}

The three shapes |rectangle|, |circle|, and |coordinate| are always
defined and no library needs to be loaded for them. While the
|coordinate| shape defines only the |center| anchor, the other two
shapes define a standard set of anchors.

\begin{shape}{circle}
  This shape draws a tightly fitting circle around the text. The
  following figure shows the anchors this shape defines; the anchors
  |10| and |130| are example of border anchors. 
\begin{codeexample}[]
\Huge
\begin{tikzpicture}
  \node[name=s,shape=circle,shape example] {Circle\vrule width 1pt height 2cm};
  \foreach \anchor/\placement in
    {north west/above left, north/above, north east/above right, 
     west/left, center/above, east/right, 
     mid west/right, mid/above, mid east/left, 
     base west/left, base/below, base east/right, 
     south west/below left, south/below, south east/below right, 
     text/left, 10/right, 130/above}
     \draw[shift=(s.\anchor)] plot[mark=x] coordinates{(0,0)}
       node[\placement] {\scriptsize\texttt{(s.\anchor)}};
\end{tikzpicture}
\end{codeexample}
\end{shape}

\begin{shape}{rectangle}
  This shape, which is the standard, is a rectangle around the
  text. The inner   and outer separations (see
  Section~\ref{section-shape-seps}) influence the white space around
  the text. The following figure shows the anchors this
  shape defines; the anchors |10| and |130| are example of border anchors.
\begin{codeexample}[]
\Huge
\begin{tikzpicture}
  \node[name=s,shape=rectangle,shape example] {Rectangle\vrule width 1pt height 2cm};
  \foreach \anchor/\placement in
    {north west/above left, north/above, north east/above right, 
     west/left, center/above, east/right, 
     mid west/right, mid/above, mid east/left, 
     base west/left, base/below, base east/right, 
     south west/below left, south/below, south east/below right, 
     text/left, 10/right, 130/above}
     \draw[shift=(s.\anchor)] plot[mark=x] coordinates{(0,0)}
       node[\placement] {\scriptsize\texttt{(s.\anchor)}};
\end{tikzpicture}
\end{codeexample}
\end{shape}



\subsection{Geometric Shapes}

\begin{pgflibrary}{shapes.geometric}
  This library defines different shapes that correspond to basic
  geometric objects like ellipses or polygons.
\end{pgflibrary}


\begin{shape}{diamond}
  This shape is a diamond tightly fitting the text box. The ratio
  between width and height is 1 by default, but can be changed by
  setting the shape aspect ratio using the following \pgfname{}
  key (to use this key in \tikzname{} simply remove the 
  \declare{|/pgf/|} path). 

  \begin{key}{/pgf/aspect=\meta{value} (initially 1.0)}
    The aspect is a recommendation for the quotient of the width and 
    the height of a shape. This key calls the macro
    |\pgfsetshapeaspect|.
  \end{key}
  
  The following figure shows the anchors this
  shape defines; the anchors |10| and |130| are example of border
  anchors.
  
\begin{codeexample}[]
\Huge
\begin{tikzpicture}
  \node[name=s,shape=diamond,shape example] {Diamond\vrule width 1pt height 2cm};
  \foreach \anchor/\placement in
    {north west/above left, north/above, north east/above right, 
     west/left, center/above, east/right, 
     mid/above, 
     base/below,  
     south west/below left, south/below, south east/below right, 
     text/left, 10/right, 130/above}
     \draw[shift=(s.\anchor)] plot[mark=x] coordinates{(0,0)}
       node[\placement] {\scriptsize\texttt{(s.\anchor)}};
\end{tikzpicture}
\end{codeexample}
\end{shape}

\begin{shape}{ellipse}
  This shape is an ellipse tightly fitting the text box, if no inner
  separation is given. The following figure shows the anchors this
  shape defines; the anchors |10| and |130| are example of border anchors.
\begin{codeexample}[]
\Huge
\begin{tikzpicture}
  \node[name=s,shape=ellipse,shape example] {Ellipse\vrule width 1pt height 2cm};
  \foreach \anchor/\placement in
    {north west/above left, north/above, north east/above right, 
     west/left, center/above, east/right, 
     mid west/right, mid/above, mid east/left, 
     base west/left, base/below, base east/right, 
     south west/below left, south/below, south east/below right, 
     text/left, 10/right, 130/above}
     \draw[shift=(s.\anchor)] plot[mark=x] coordinates{(0,0)}
       node[\placement] {\scriptsize\texttt{(s.\anchor)}};
\end{tikzpicture}
\end{codeexample}
\end{shape}





\begin{shape}{trapezium}
  This shape is a trapezium, that is, a quadrilateral with a single
  pair of parallel lines (this can sometimes be known as a trapezoid).
  The trapezium shape supports the rotation of the shape border, as 
  described in Section~\ref{section-rotating-shape-borders}. 
  
  The lower internal angles at the lower corners of the trapezium can 
  be specified independently, and the resulting extensions are in 
  addition to the natural dimensions of the node contents (which
  includes any |inner sep|.
	
\begin{codeexample}[]
\begin{tikzpicture}
   \tikzstyle{every node}=[trapezium, draw]
   \node at (0,2) {A};
   \node[trapezium left angle=75, trapezium right angle=45pt]
         at (0,1) {B};
   \node[trapezium left angle=120, trapezium right angle=60pt]
         at (0,0) {C};
\end{tikzpicture}
\end{codeexample}

         
  The \pgfname{} keys to set the lower internal angles of the trapezium 
  are shown below. 
  To use these keys in \tikzname, simply remove the \declare{|/pgf/|} path.
	
  \begin{key}{/pgf/trapezium left angle=\meta{angle} (initially 60)}
    Set the lower internal angle of the left side. 
  \end{key}
   
  \begin{key}{/pgf/trapezium right angle=\meta{angle} (initially 60)}
    Set the lower internal angle of the right side. 
  \end{key}
  
  \begin{stylekey}{/pgf/trapezium angle=\meta{angle}}
    This key stores no value itself, but sets the value of the
    previous two keys to \meta{angle}. 
  \end{stylekey}
     
  Regardless of the rotation of the shape border, the width
  and height of the trapezium are as follows:

\begin{codeexample}[]
\begin{tikzpicture}[>=stealth, every node/.style={text=black}, 
    shape border uses incircle, shape border rotate=60]
  \node [trapezium, fill=gray!25, minimum width=2cm] (t) {};
  \draw [red, <->] (t.bottom left corner) -- (t.bottom right corner) 
    node [midway, below right] {width};
  \draw [red, <->] (t.top side) -- (t.bottom side) 
    node [at start, above] {height};
\end{tikzpicture}
\end{codeexample}

  \begin{key}{/pgf/trapezium stretches=\meta{boolean} (default true)}
    This key controls whether \pgfname{} allows the width and the height
    of the trapezium to be enlarged independently, 
    when considering any minimum size specification. This is initially 
    |false|,  ensuring that the shape ``looks the same but bigger'' when 
    enlarged.
  
\begin{codeexample}[]
\tikzset{my node/.style={trapezium, fill=#1!20, draw=#1!75, text=black}}
\begin{tikzpicture}
  \draw [help lines] grid (3,2);
  \node [my node=red]                                      {A};
  \node [my node=green, minimum height=1.5cm] at (1, 1.25) {B};
  \node [my node=blue,  minimum width=1.5cm]  at (2, 0)    {C};
\end{tikzpicture}
\end{codeexample} 

    By setting \meta{boolean} to |true|, the trapezium can be stretched
    horizontally or vertically.
  
\begin{codeexample}[]
\tikzset{my node/.style={trapezium, fill=#1!20, draw=#1!75, text=black}}
\begin{tikzpicture}
\tikzset{trapezium stretches=true}
  \draw [help lines] grid (3,2);
  \node [my node=red]                                      {A};
  \node [my node=green, minimum height=1.5cm] at (1, 1.25) {B};
  \node [my node=blue,  minimum width=1.5cm]  at (2, 0)    {C};
\end{tikzpicture}
\end{codeexample}
\end{key}

  \begin{key}{/pgf/trapezium stretches body=\meta{boolean} (default true)}
    This is similar to the |trapezium stretches| key except that
    when \meta{boolean} is |true|, \pgfname{} enlarges only the body 
    of the trapezium when applying minimum width.
  
\begin{codeexample}[]
\tikzset{my node/.style={trapezium, fill=#1!20, draw=#1!75, text=black}}
\begin{tikzpicture}
  \draw [help lines] grid (3,2);
  \node [my node=red]                      at (1.5,.25)  {A};
  \node [my node=green, minimum width=3cm, trapezium stretches] 
    at (1.5,1)    {B};
  \node [my node=blue,  minimum width=3cm, trapezium stretches body] 
    at (1.5,1.75) {C};
\end{tikzpicture}
\end{codeexample}
  \end{key}

  The anchors for the trapezium are shown below. The anchor |160| is an
  example of a border anchor.

\begin{codeexample}[]
\Huge
\begin{tikzpicture}
  \node[name=s, shape=trapezium, shape example, inner sep=1cm] 
    {Trapezium\vrule width 1pt height 2cm};
  \foreach \anchor/\placement in
    {bottom left corner/below, top right corner/right, 
     top left corner/left,     bottom right corner/below,
     bottom side/below,        left side/left, 
     right side/right,         top side/above,
     center/above,   text/below,      mid/right,       base/below, 
     mid west/right, base west/below, mid east/left,   base east/below, 
     west/above,     east/above,      north/below,     south/above,
     north west/above, north east/above, 
     south west/below, south east/below, 160/above}    
  \draw[shift=(s.\anchor)] plot[mark=x] coordinates{(0,0)}
    node[\placement] {\scriptsize\texttt{(s.\anchor)}};
\end{tikzpicture}
\end{codeexample}  
\end{shape}


\begin{shape}{semicircle}
  This shape is a semicircle, which tightly fits the node contents.
  This shape supports the rotation of the shape border, as described in 
  Section~\ref{section-rotating-shape-borders}.
  The anchors for the |semicircle| shape are shown below. 
  Anchor |30| is an example of a border anchor.
	
\begin{codeexample}[]
\Huge
\begin{tikzpicture}
  \node[name=s,shape=semicircle,shape border rotate=0,shape example, inner sep=1cm] 
  	{Semicircle\vrule width 1pt height 2cm};
  \foreach \anchor/\placement in
    {apex/above,      arc start/below, arc end/below,  chord center/below,
     center/above,    base/below,      mid/right,      text/left,
     base west/below, base east/below, mid west/left, mid east/right, 
     north/below,     south/above,     east/above,     west/above,
     north west/above left, north east/above right,
     south west/below,      south east/below, 30/right}
     \draw[shift=(s.\anchor)] plot[mark=x] coordinates{(0,0)}
       node[\placement] {\scriptsize\texttt{(s.\anchor)}};
\end{tikzpicture}
\end{codeexample}
\end{shape}





\begin{shape}{regular polygon}
  This shape is a regular polygon, which, by default, is drawn so that 
  a side (rather than a corner) is always at the bottom. 
  This shape supports the rotation as described in 
  Section~\ref{section-rotating-shape-borders}, but the border of the 
  polygon is \emph{always} constructed using the incircle, whose
  radius is calculated to tightly fit the node contents (including
  any |inner sep|).
  
\begin{codeexample}[]
\begin{tikzpicture}
  \foreach \a in {3,...,7}{
    \draw[red, dashed] (\a*2,0)  circle(0.5cm);
    \node[regular polygon, regular polygon sides=\a, draw,
     inner sep=0.3535cm] at (\a*2,0) {};
   }  
\end{tikzpicture}
\end{codeexample}	
	
  If the node is enlarged to any specified minimum size, 
  this is interpreted as the diameter of the the 
  circumcircle, that is, the circle that passes through all the 
  corners of the polygon border.

\begin{codeexample}[]
\begin{tikzpicture}
  \foreach \a in {3,...,7}{
    \draw[blue, dashed] (\a*2,0)  circle(0.5cm);
    \node[regular polygon, regular polygon sides=\a, minimum size=1cm, draw] at (\a*2,0) {};
   }  
\end{tikzpicture}
\end{codeexample}	

  There is a \pgfname{} key to set the number of sides for the regular
  polygon.
  To use this key in \tikzname, simply remove the \declare{|/pgf/|} path.
	
  \begin{key}{/pgf/regular polygon sides=\meta{integer} (initially 5)}
  \end{key}
  
  The anchors for a regular polygon shape are shown below.  
  The anchor |75| is an example of a border anchor.
  
\begin{codeexample}[]
\Huge
\begin{tikzpicture}
  \node[name=s, shape=regular polygon, shape example, inner sep=.5cm] 
    {Regular Polygon\vrule width 1pt height 2cm};
  \foreach \anchor/\placement in
    {corner 1/above, corner 2/above, corner 3/left, corner 4/right, corner 5/above, 
     side 1/above,   side 2/left,    side 3/below,  side 4/right,   side 5/above,  
     center/above, text/left,  mid/right,   base/below, 75/above,
     west/above,   east/above, north/below, south/above,
     north east/below, south east/above, north west/below, south west/above}
  \draw[shift=(s.\anchor)] plot[mark=x] coordinates{(0,0)}
    node[\placement] {\scriptsize\texttt{(s.\anchor)}};
\end{tikzpicture}
\end{codeexample}

\end{shape}

\begin{shape}{star}
  This shape is a star, which by default (minus any transformations) is
  drawn with the first point pointing upwards.  
  This shape supports the rotation as described in 
  Section~\ref{section-rotating-shape-borders}, but the border of the 
  star is \emph{always} constructed using the incircle.
  
  A star should be thought of as having an set of ``inner points'' and
  and ``outer points''. 
  The inner points of the border are based on the radius of the circle
  which tightly fits the node contents, and the outer points are based
  on the circumcircle, the circle that passes through every outer
  point.
  Any specified minimum size, width or height, is interpreted as the 
  diameter of the circumcircle.
 
\begin{codeexample}[]
\begin{tikzpicture}
   \draw [help lines]   (0,0) grid (2,2);
   \draw [blue, dashed]  (1,1) circle(1cm);
   \draw [red, dashed] (1,1) circle(.5cm);
   \node [star, star point height=.5cm, minimum size=2cm, draw] 
       at (1,1) {S};
\end{tikzpicture}
\end{codeexample} 
  
  The \pgfname{} keys to set the number of star points, and the height
  of the star points, are shown below. To use these keys in \tikzname,
  simply remove the \declare{|/pgf/|} path.
  
  \begin{key}{/pgf/star points=\meta{integer} (initially 5)}
    Sets the number of points for the star.
  \end{key}
  
  \begin{key}{/pgf/star point height=\meta{distance} (initially .5cm)}
    Sets the height of the star points. This is the distance between the
    inner point and outer point radii. If the star is enlarged to some
    specified minimum size, the inner radius is increased to maintain
    the point height.	
  \end{key}
  
  \begin{key}{/pgf/star point ratio=\meta{number} (initially 1.5)}
    Sets the ratio between the inner point and outer point radii.		
    If the star is enlarged to some specified minimum size, the
    inner radius is increased to maintain the ratio.	
  \end{key}

	The inner and outer points form the principle anchors for the star,
   as shown below (anchor |75| is an example of a border anchor).
  
  \begin{codeexample}[]
\Huge
\begin{tikzpicture}
  \node[name=s, shape=star, star points=5, star point ratio=1.65, shape example, inner sep=1.5cm] 
    {Star\vrule width 1pt height 2cm};
  \foreach \anchor/\placement in
     {inner point 1/above, inner point 2/above, inner point 3/below, inner point 4/right, 
      inner point 5/above, outer point 1/above, outer point 2/above, outer point 3/left,  
      outer point 4/right, outer point 5/above,
      center/above, text/left,  mid/right,   base/below, 75/above,
     	west/above,   east/above, north/below, south/above,
     	north east/below, south east/above, north west/below, south west/above}
  \draw[shift=(s.\anchor)] plot[mark=x] coordinates{(0,0)}
    node[\placement] {\scriptsize\texttt{(s.\anchor)}};
\end{tikzpicture}
\end{codeexample}
\end{shape}





\begin{shape}{isosceles triangle}
  This shape is an isosceles triangle, which supports the rotation of 
  the shape border, as described in 
  Section~\ref{section-rotating-shape-borders}. The angle of rotation
  determines the direction in which the apex of the triangle points
  (provided no other transformations are applied). However, regardless
  of the rotation of the shape border, the width and height are 
  always considered as follows:
	
\begin{codeexample}[]
\begin{tikzpicture}[>=stealth, every node/.style={text=black},
    shape border uses incircle, shape border rotate=-30]
  \node [isosceles triangle, fill=gray!25, minimum width=1.5cm] (t) {};
  \draw [red, <->] (t.left corner) -- (t.right corner)
    node [midway, above left] {width};
  \draw [red, <->] (t.apex) -- (t.lower side)
    node [midway, above right] {height};
\end{tikzpicture}
\end{codeexample}

	There are \pgfname{} keys to customise this shape. 
	To use these keys in \tikzname, simply remove the \declare{|/pgf/|} 
	path.
    
  \begin{key}{/pgf/isosceles triangle apex angle=\meta{angle} (initially 45)}
    Sets the angle of the apex of the isosceles triangle. 
  \end{key}

\begin{key}{/pgf/isosceles triangle stretches=\meta{boolean} (default true)}
  
	By default \meta{boolean} is |false|. This means, that when applying
	any minimum width or minimum height requirements, increasing the 
	height will increase the width (and	vice versa), in order to keep the
	apex angle the same.
   
\begin{codeexample}[]
\begin{tikzpicture}[paint/.style={draw=#1!75, fill=#1!20}]
  \tikzset{every node/.style={isosceles triangle, draw, inner sep=0pt, 
    anchor=left corner, shape border rotate=90}}
  \draw[help lines] grid(4,2);
  \foreach \a/\c in {1.5/blue, 1/green, 0.5/red}{
    \node[paint=\c, minimum height=\a cm] at (0,0) {};
    \node[paint=\c, minimum width=\a cm] at (2,0) {};
  }
\end{tikzpicture}
\end{codeexample}	

	However, by setting \meta{boolean} to |true|, minimum width and 
	height can be applied	independently.
	
\begin{codeexample}[]
\begin{tikzpicture}[paint/.style={draw=#1!75, fill=#1!20}]
  \tikzset{every node/.style={isosceles triangle, draw, inner sep=0pt, 
     anchor=south, shape border rotate=90, isosceles triangle stretches}}
  \draw[help lines] grid(4,2);
  \foreach \a/\c in {1.5/blue, 1/green, 0.5/red}{
    \node[paint=\c, minimum height=\a cm, minimum width=1.5cm] at (0.75,0) {};
    \node[paint=\c, minimum width=\a cm, minimum height=1.5cm] at (3,0)    {};
  }
\end{tikzpicture}
\end{codeexample}	
\end{key}
	
   The anchors for the |isosceles triangle| are shown below 
   (anchor |150| is an	example of a border anchor). Note that,
   somewhat confusingly, the anchor names such as |left side| and
   |right corner| are named as if the triangle is rotated to 
   90 degrees. Note also that the |center| anchor
   does not	necessarily correspond to any kind of geometric center.
	
\begin{codeexample}[]
\Huge
\begin{tikzpicture}
  \node[name=s, shape=isosceles triangle, shape example, inner xsep=1cm]
    {Isosceles Triangle\vrule width 1pt height 2cm};
  \foreach \anchor/\placement in
    {apex/above,      left corner/right, right corner/right,
     left side/above, right side/below,  lower side/right,    
     center/above,    text/right,        150/above,
     mid/right,       mid west/above,    mid east/right,
     base/below,      base west/below,   base east/below,
     west/above, east/below, north/below, south/above,
     north west/below, north east/below, 
     south west/above, south east/above}  
  \draw[shift=(s.\anchor)] plot[mark=x] coordinates{(0,0)}
    node[\placement] {\scriptsize\texttt{(s.\anchor)}};
\end{tikzpicture}
\end{codeexample} 
\end{shape}






\par\leavevmode
\begin{shape}{kite}

	This shape is a kite, which supports the rotation of the shape border, 
	as described in Section~\ref{section-rotating-shape-borders}. 
	There are \pgfname{} keys to specify the upper and lower vertex angles
	of the kite. 
	To use these keys in \tikzname, simply remove the \declare{|/pgf/|} 
	path.
	
	\begin{key}{/pgf/kite upper vertex angle=\meta{angle} (initially 120)}
	Set the upper internal angle of the kite.
	\end{key}
	
	\begin{key}{/pgf/kite lower vertex angle=\meta{angle} (initially 60)}
	Set the lower internal angle of the kite.
	\end{key}
	
	\begin{key}{/pgf/kite vertex angles=\meta{angle specification}}
		This key sets the keys for both the upper and lower vertex angles
		(it stores no value itself).
	   \meta{angle specification} can be pair of angles in the form
	   \meta{upper angle} |and| \meta{lower angle}, or a single angle.
	   In this latter case, both the upper and lower vertex angles will 
	   be the same.
	\end{key}%
    
\begin{codeexample}[]
\begin{tikzpicture}
  \tikzstyle{every node}=[kite, draw]
  \node[kite upper vertex angle=135, kite lower vertex angle=70] at (0,0) {A};
  \node[kite vertex angles=90 and 45] at (1,0) {B};
  \node[kite vertex angles=60]        at (2,0) {C};
\end{tikzpicture}
\end{codeexample}


	The anchors for the |kite| are shown below. Anchor |110| is an 
	example of a border anchor.
	
\begin{codeexample}[]
\Huge
\begin{tikzpicture}
  \node[name=s, shape=kite, shape example, inner sep=1.5cm] 
    {Kite\vrule width 1pt height 2cm};
  \foreach \anchor/\placement in
    {upper vertex/above, left vertex/above,    lower vertex/below, 
     right vertex/above, upper left side/above, upper right side/above,
     lower left side/below, lower right side/below,
     center/above,   text/left,       mid/right,        base/below, 
     mid west/left,  base west/below, mid east/right,   base east/below,
     west/above,     east/above,      north/below,     south/above,
     north west/left, north east/right, 
     south west/above, south east/above, 110/above}  
  \draw[shift=(s.\anchor)] plot[mark=x] coordinates{(0,0)}
    node[\placement] {\scriptsize\texttt{(s.\anchor)}};
\end{tikzpicture}
\end{codeexample}
\end{shape}


\begin{shape}{dart}


	This shape is a dart (which can also be known as an arrowhead or
	concave kite). This shape supports the rotation of the shape border, 
	as described in Section~\ref{section-rotating-shape-borders}. 
	The angle of the border rotation determines the direction in which 
	the dart points (unless other transformations have been applied).
	
	There are \pgfname{} keys to set the 
	angle for the `tip' of the dart and the angle between the `tails'
	of the dart. 
	To use these keys in \tikzname, simply remove the \declare{|/pgf/|} 
	path.

\begin{codeexample}[]
\begin{tikzpicture}
   \node[dart, draw, gray, shape border uses incircle, shape border rotate=45] 
       (d) {dart};
   \draw [<->] (d.tip)++(202.5:.5cm) arc(202.5:247.5:.5cm);
   \node [left=.5cm] at (d.tip) {tip angle};
   \draw [<->] (d.tail center)++(157.5:.5cm) arc(157.5:292.5:.5cm);
   \node [right] at (d.tail center) {tail angle};
\end{tikzpicture}
\end{codeexample}

	\begin{key}{/pgf/dart tip angle=\meta{angle} (initially 45)}
		Set the angle at the tip of the dart.
	\end{key}
	
	\begin{key}{/pgf/dart tail angle=\meta{angle} (initially 135)}
		Set the angle between the tails of the dart.
	\end{key}
		
	The anchors for the |dart| shape are shown below (note that the 
	shape is rotated 90 degrees anti-clockwise). Anchor |110| is an 
	example of a border anchor.
\begin{codeexample}[]
\Huge
\begin{tikzpicture}
  \node[name=s, shape=dart, shape border rotate=90, shape example, inner sep=1.25cm] 
    {Dart\vrule width 1pt height 2cm};
  \foreach \anchor/\placement in
    {tip/above,       tail center/below, right tail/below, 
     left tail/below, right tail/below,  left side/left,   right side/right,
     center/above,    text/left,         mid/right,        base/below, 
     mid west/left,   base west/below,   mid east/right,   base east/below,
     west/above,      east/above,        north/below,      south/above,
     north west/left, north east/right,  south west/above, south east/above,
     110/above}    
  \draw[shift=(s.\anchor)] plot[mark=x] coordinates{(0,0)}
    node[\placement] {\scriptsize\texttt{(s.\anchor)}};
\end{tikzpicture}
\end{codeexample}
\end{shape}




\begin{shape}{circular sector}

	This shape is a circular sector (which can also be known as a
	wedge).
	This shape supports the rotation of the shape border, 
	as described in Section~\ref{section-rotating-shape-borders}. 
	The angle of the border rotation determines the direction in which 
	the `apex' of the sector points (unless other transformations have 
	been applied).
	
\begin{codeexample}[]
\begin{tikzpicture}
	\tikzstyle{every node}=[circular sector, shape border uses incircle, draw];
   \node at (0,0) {A};
   \node [shape border rotate=30] at (1.5,0) {A};
\end{tikzpicture}
\end{codeexample}

	There is a \pgfname{} key to set the central angle of the sector, 
	which is expected to be less than 180 degrees. 
	To use this key in \tikzname,	simply remove the \declare{|/pgf/|} 
	path.
	
	\begin{key}{/pgf/circular sector angle=\meta{angle} (initially 60)}
		Set the central angle of the sector. 
	\end{key}
	
	The anchors for the circular sector shape are shown below.
	Anchor |30| is an example of a border anchor.
	
\begin{codeexample}[]
\Huge
\begin{tikzpicture}
  \node[name=s,shape=circular sector,  style=shape example, inner sep=1cm] 
  	{Circular Sector\vrule width 1pt height 2cm};
  \foreach \anchor/\placement in
   {sector center/above, arc start/below, arc end/below, arc center/below,
    center/above,        base/below,      mid/right,     text/below,
    north/below,         south/above,     east/below,    west/above,
    north west/above left, north east/above right,
    south west/below,      south east/below, 30/right}
     \draw[shift=(s.\anchor)] plot[mark=x] coordinates{(0,0)}
       node[\placement] {\scriptsize\texttt{(s.\anchor)}};
\end{tikzpicture}
\end{codeexample}
\end{shape}




\begin{shape}{cylinder}
	This shape is a 2-dimensional representation of a cylinder, which 
	supports the rotation of the shape border as described in
	Section~\ref{section-rotating-shape-borders}.

\begin{codeexample}[]
\begin{tikzpicture}
  \node[cylinder, draw, shape aspect=.5] {ABC};
\end{tikzpicture}
\end{codeexample}
		
	Regardless the rotation of the shape border, the height is always the
	distance between the curved ends, and the width is always the	
	distance between the straight sides. 

\begin{codeexample}[]
\begin{tikzpicture}[>=stealth]
  \node [cylinder, gray!50, rotate=30, draw, 
    minimum height=2cm, minimum width=1cm] (c) {Cylinder};
  \draw[red, <->] (c.top)   -- (c.bottom) 
    node [at end, below, black]   {height};
  \draw[red, <->] (c.north) -- (c.south) 
    node [at start, above, black] {width};
\end{tikzpicture}
\end{codeexample}

	Enlarging the shape to some minimum height will stretch only the body
	of the cylinder. By contrast, enlarging the shape to some minimum 
	width will stretch the curved ends.
	
\begin{codeexample}[]
\begin{tikzpicture}[>=stealth, shape aspect=.5]
  \tikzset{every node/.style={cylinder, shape border rotate=90, draw}}
  \node [minimum height=1.5cm]            {A};  
  \node [minimum width=1.5cm]  at (1.5,0) {B};  
\end{tikzpicture}
\end{codeexample}

  There are various keys to customize this shape (to use \pgfname{}
  keys in \tikzname{}, simply remove the \declare{|/pgf/|} path).
  
\begin{key}{/pgf/aspect=\meta{value} (initially 1.0)}
  The aspect is a recommendation for the quotient of the radii of
  the cylinder end. This may be ignored if the shape is enlarged
  to some minimum width.

\begin{codeexample}[]
\begin{tikzpicture}[>=stealth]
  \tikzset{every node/.style={cylinder, shape border rotate=90, draw}}
  \node [aspect=1.0]           {A};  
  \node [aspect=0.5]  at (1,0) {B};  
  \node [aspect=0.25] at (2,0) {C};  
\end{tikzpicture}
\end{codeexample}

\end{key}

\begin{key}{/pgf/cylinder uses custom fill=\meta{boolean} (default true)}
	This enables the use of a custom fill for the body and the end of 
	the cylinder. The background path for the shape should not be 
	filled (e.g., in \tikzname{}, the |fill| option for the node must 
	be implicity or explicitly set to |none|).
  Internally, this key sets the \TeX-if 
  |\ifpgfcylinderusescustomfill| appropriately.
\end{key}

\begin{codeexample}[]
\begin{tikzpicture}[>=stealth, aspect=0.5]
  \node [cylinder, cylinder uses custom fill, cylinder end fill=red!50,
         cylinder body fill=red!25] {Cylinder};  
\end{tikzpicture}
\end{codeexample}

\begin{key}{/pgf/cylinder end fill=\meta{color} (initially white)}
	Set the color for the end of the cylinder.
\end{key}
\begin{key}{/pgf/cylinder body fill=\meta{color} (initially white)}
	Set the color for the body of the cylinder.
\end{key}


  The anchors this shape are shown below (anchor |160| is an
	example of a border anchor). Note the the cylinder shape does not 
	distinguish between |outer xsep| and |outer ysep|. Only the larger 
	of the two values is used for the shape. Note also the difference 
	between the |center| and |shape center| anchors: |center| is the
	center of the cylinder body and also the center of rotation. 
	The |shape center| is the center of the shape which includes the 
	2-dimensional representation of the cylinder top.	
	 

\begin{codeexample}[]
\Huge
\begin{tikzpicture}
  \node[name=s, shape=cylinder, shape example, aspect=.5, inner xsep=3cm,
        inner ysep=1cm] {Cylinder\vrule width 1pt height 2cm};
  \foreach \anchor/\placement in
    {before top/above,    top/above,       after top/below,
     before bottom/below, bottom/above,    after bottom/above,
     mid/right,           mid west/right,  mid east/left,  
     base/below,          base west/below, base east/below,
     center/above,        text/above,      shape center/right, 
     west/right, east/left, north/above, south/below,
     north west/below, north east/above, 
     south west/above, south east/below, 160/above}    
  \draw[shift=(s.\anchor)] plot[mark=x] coordinates{(0,0)}
    node[\placement] {\scriptsize\texttt{(s.\anchor)}};
\end{tikzpicture}
\end{codeexample}  


\end{shape}






\subsection{Symbol Shapes}

\begin{pgflibrary}{shapes.symbols}
  This library defines shapes that can be used for drawing symbols
  like a forbidden sign or a cloud.
\end{pgflibrary}



\begin{shape}{forbidden sign}
  This shape places the node inside a circle with a diagonal from the
  lower left to the upper right added. The circle is part of the
  background, the diagonal line part of the foreground path; thus, the
  diagonal line is on top of the text.
  
\begin{codeexample}[]
\begin{tikzpicture}
  \node [forbidden sign,line width=1ex,draw=red,fill=white] {Smoking};
\end{tikzpicture}
\end{codeexample}

  The shape inherits all anchors from the |circle| shape, see also the
  following figure:
\begin{codeexample}[]
\Huge
\begin{tikzpicture}
  \node[name=s,shape=forbidden sign,shape example] {Forbidden\vrule width 1pt height 2cm};
  \foreach \anchor/\placement in
    {north west/above left, north/above, north east/above right, 
     west/left, center/above, east/right, 
     mid west/right, mid/above, mid east/left, 
     base west/left, base/below, base east/right, 
     south west/below left, south/below, south east/below right, 
     text/left, 10/right, 130/above}
     \draw[shift=(s.\anchor)] plot[mark=x] coordinates{(0,0)}
       node[\placement] {\scriptsize\texttt{(s.\anchor)}};
\end{tikzpicture}
\end{codeexample}
\end{shape}


\begin{shape}{cloud}

	This shape is a cloud, drawn to tightly fit the node contents 
	(strictly speaking, using an ellipse which tightly fits the node
	contents -- including any |inner sep|). 
	
\begin{codeexample}[]
\begin{tikzpicture}
  \node[cloud, draw, fill=gray!20, aspect=2] {ABC};
  \node[cloud, draw, fill=gray!20] at (1.5,0) {D};
\end{tikzpicture}
\end{codeexample}

	A cloud should be thought of as having a number of ``puffs'', which
	are the individual arcs drawn around the border. There are \pgfname{}
	keys to specify how the cloud is drawn (to use these keys in 
	\tikzname{}, simply remove the \declare{|/pgf/|} path).
	
	\begin{key}{/pgf/cloud puffs=\meta{integer} (initially 10)}
	  Set the number of puffs for the cloud.
	\end{key}
	
	\begin{key}{/pgf/cloud puff arc=\meta{angle} (initially 135)}
	  Set the length of the puff arc (in degrees). A shorter arc can 
	  produce better looking joins between puffs for larger line widths.
	\end{key}
	
	Like the diamond shape, the cloud shape also uses the 
	\declare{|aspect|} key, to determine the ratio of the width and the 
	height of the cloud. However there may be cirumstances where it may
	be undesirable to continually specify the |aspect| for the cloud.
	Therefore, the following key is implemented:
	
	\begin{key}{/pgf/cloud ignores aspect=\meta{boolean} (default true)}
		Instruct \pgfname{} to ignore the |aspect| key. Internally, the
		\TeX-if |\ifpgfcloudignoresaspect| is set appropriately. The initial
		value is |false|.

\begin{codeexample}[]
\begin{tikzpicture}[aspect=1, every node/.style={cloud, cloud puffs=11, draw}]
  \node [fill=gray!20]                                {rain};
  \node [cloud ignores aspect, fill=white] at (1.5,0) {snow};
\end{tikzpicture}
\end{codeexample}	
	
	\end{key}
	
	
	Any minimum size requirements are applied to the ``circum-ellipse'',
	which is the ellipse which passes through all the midpoints of the
	puff arcs. These requirements are considered \emph{after} any 
	aspect specification is applied.
	
\begin{codeexample}[]
\begin{tikzpicture}
  \draw [help lines] grid (3,2);
  \draw [blue, dashed] (1.5, 1) ellipse (1.5cm and 1cm);
  \node [cloud, cloud puffs=9, draw, minimum width=3cm, minimum height=2cm] 
    at (1.5, 1) {};
\end{tikzpicture}
\end{codeexample}
	
  The anchors for the cloud shape are shown below for a cloud with
  eleven puffs. Anchor 70 is an example of a border anchor. 
  
\begin{codeexample}[]
\Huge
\begin{tikzpicture}
  \node[name=s, shape=cloud, style=shape example, cloud puffs=11, aspect=1.5,
       cloud puff arc=120,inner ysep=1cm] {Cloud\vrule width 1pt height 2cm};
  \foreach \anchor/\placement in
   {puff 1/above, puff 2/above,  puff 3/above,  puff 4/below, 
    puff 5/left,  puff 6/below,  puff 7/below,  puff 8/right,
    puff 9/below, puff 10/above, puff 11/above, 70/right,
    center/above, base/below,    mid/right,     text/left, 
    north/below,  south/below,   east/above,    west/above,
    north west/left,             north east/right, 
    south west/below,            south east/below}
     \draw[shift=(s.\anchor)] plot[mark=x] coordinates{(0,0)}
       node[\placement] {\scriptsize\texttt{(s.\anchor)}};
\end{tikzpicture}
\end{codeexample}
\end{shape} 






\begin{shape}{starburst}

	This shape is a randomly generated eliptical star,
	which supports the rotating of the shape border as described in 
	Section~\ref{section-rotating-shape-borders}. 
\begin{codeexample}[]
\begin{tikzpicture}
  \node[starburst, fill=yellow, draw=red, line width=2pt] {\bf BANG!};
\end{tikzpicture}
\end{codeexample}	
	Like the |star| shape, the starburst should be thought of as having a set
	of inner points and outer points. The inner points lie on the ellipse
	which tightly fits the node contents (including any |inner sep|).
	
	Using a specified `starburst point height' value, the outer points
	are generated randomly between this value and one quarter of this 
	value. For a given starburst shape the angle between each point is 
	fixed, and is determined by the number of points specified for
	the starburst.
	
	It is important to note that, whilst the maximum possible point 
	height is used to calculate minimum width or height requirements, 
	the outer points are randomly generated, so there is (unfortunately) 
	no guarantee that any such requirements will be fully met. 
	
\begin{codeexample}[]
\begin{tikzpicture}
  \draw[help lines] grid(3,2);
  \node[starburst, draw, minimum width=3cm, minimum height=2cm] 
    at (1.5, 1) {\bf BOOM!};
\end{tikzpicture}
\end{codeexample}

	There are \pgfname{} keys to control the drawing of the starburst
	shape. To use these keys in \tikzname,	simply remove the 
	\declare{|/pgf/|}	path.

	\begin{key}{/pgf/starburst points=\meta{integer} (initially 17)}
		Set the number of points for the starburst.
	\end{key}
	\begin{key}{/pgf/starburst point height=\meta{length} (initially .5cm)}
      Set the \emph{maximum} distance between the inner point radius  
      and the outer point radius.
	\end{key}
	
	\begin{key}{/pgf/random starburst=\meta{integer} (initially 100)}
      Set the seed for the random number generator for creating the
      starburst.  The maximum value for \meta{integer} is |16383|.
      If \meta{integer}|=0|, the random number generator will not be 
      used, and the maximum point height will be used for all outer 
      points. If \meta{integer} is omitted, a seed will be randomly
      chosen.
	\end{key}
	
	The basic anchors for a nine point |starburst| shape are shown below. 
	Anchor |80| is an example of a border anchor.
\begin{codeexample}[]
\Huge
\begin{tikzpicture}
  \node[name=s, shape=starburst, starburst points=9, starburst point height=3.5cm, 
        style=shape example,inner sep=1cm] 
    {Starburst\vrule width 1pt height 2cm};
  \foreach \anchor/\placement in
    {outer point 1/above, outer point 2/above, outer point 3/right,
     outer point 4/above, outer point 5/below, outer point 6/above,
     outer point 7/left,  outer point 8/above, outer point 9/above,
     inner point 1/below, inner point 2/above, inner point 3/left,
     inner point 4/above, inner point 5/above, inner point 6/above,
     inner point 7/below, inner point 8/above, inner point 9/below,
     center/above, text/left,   mid/right, base/below, 80/above,
     north/below,  south/below, east/left, west/right,
     north east/below, south west/below, south east/below, north west/below}
  \draw[shift=(s.\anchor)] plot[mark=x] coordinates{(0,0)}
    node[\placement] {\scriptsize\texttt{(s.\anchor)}};
\end{tikzpicture}
\end{codeexample}
\end{shape}

\begin{shape}{signal}

	This shape is a ``signal'' or sign shape, that is, a rectangle, with
	optionally pointed sides. A signal can point ``to'' somewhere, with 
	outward points in that direction. It can also be ``from'' 
	somewhere, with inward points from that direction. The resulting 
	points extend the node contents (which include the |inner sep|).
	
\begin{codeexample}[]
\begin{tikzpicture}[every node/.style={signal, draw,  text=white}]
  \node[fill=green!65!black, signal to=east] at (0,1) {To East};
  \node[fill=red!65!black, signal from=east] at (0,0) {From East};
\end{tikzpicture}
\end{codeexample}

	There are \pgfname{} keys for drawing the signal shape (to use these
	keys in \tikzname{}, simply remove the \declare{|/pgf/|} path):
	
	\begin{key}{/pgf/signal pointer angle=\meta{angle} (initially 90)} 
		Set the angle for the pointed sides of the shape. This angle is
		maintained when enforcing any minimum size requirements, so
		any adjustment to the width will affect the height, and vice versa.
	\end{key}
	
	\begin{key}{/pgf/signal from=\meta{direction}\space\opt{and \meta{opposite direction}} (initially nowhere)} 
		Set which sides take an inward pointer (i.e., that points towards the
		center of the shape). The possible values for \meta{direction} and 
		\meta{opposide direction} are the compass point directions |north|,
		|south|, |east| and |west| (or |above|, |below|, |right| and |left|).
		An additional keyword |nowhere| can be used to reset the sides so 
		they have no pointers. When used with |signal from| key, this only 
		resets inward pointers;	used with the |signal to| key, it only 
		resets outward pointers. 
		
	\end{key}
	
	\begin{key}{/pgf/signal to=\meta{direction}\space\opt{and \meta{opposite direction}} (initially east)} 
		Set which sides take an outward pointer (i.e., that points away from 
		the	the shape). 
	\end{key}
	
	Note that \pgfname{} will ignore any instruction to use directions
	that are not opposites (so using the value |east and north|, will
	result in only |north| being assigned a pointer). This is also 
	the case if non-opposite values are used in the |signal to| and
	|signal from| keys at the same time. So, for example, it is not 
	possible for a signal to have an outward point to the left, and also
	have an inward point from below.
	
	The anchors for the signal shape are shown below. Anchor |70| is an
	example of a border anchor.

\begin{codeexample}[]
\Huge
\begin{tikzpicture}
  \node[name=s, shape=signal, signal from=west, shape example, inner sep=2cm] 
    {Signal\vrule width1pt height2cm};
  \foreach \anchor/\placement in
    {text/left,   center/above,    70/above,
     base/below,  base east/below, base west/below,
     mid/right,   mid east/above left,  mid west/above left, 
     north/above,      south/below, 
     east/above,       west/above,        
     north west/above, north east/above, 
     south west/below, south east/below}
     \draw[shift=(s.\anchor)] plot[mark=x] coordinates{(0,0)}
       node[\placement] {\scriptsize\texttt{(s.\anchor)}};
\end{tikzpicture}
\end{codeexample}

\end{shape}





\begin{shape}{tape}
	This shape is a rectangle with optional, ``bendy'' top and bottom
	sides, which tightly fits the node contents (including the 
	|inner sep|).
	
\begin{codeexample}[]
\begin{tikzpicture}
  \node[tape, draw]{ABCD};
  \node[tape, draw, tape bend top=none] at (1.5, 0) {EFGH};
\end{tikzpicture}
\end{codeexample}

  There are \pgfname{} keys to specify which sides bend and how high
  the bends are (to use these keys in \tikzname{}, simply remove the
  \declare{|/pgf/|} path):
  
  \begin{key}{/pgf/tape bend top=\meta{bend style} (initially in and out)}
  	Specify how the top side bends. The \meta{bend style} is either
  	|in and out|, |out and in| or |none| (i.e., a straight line). 
  	The bending sides are drawn in a 
  	clockwise direction, and using the bend style |in and out| will mean 
  	the side will first	bend inwards and then bend outwards. 
  	The opposite holds true for	|out and in|. 
  	
\begin{codeexample}[]
\begin{tikzpicture}[-stealth]
  \node[tape, draw, gray, minimum width=2cm](t){Tape};
  \draw [blue]([yshift=5pt] t.north west) -- ([yshift=5pt]t.north east) 
         node[midway, above, black]{in and out};
  \draw [blue]([yshift=-5pt]t.south east) -- ([yshift=-5pt]t.south west) 
         node[sloped, allow upside down, midway, above, black]{in and out};
\end{tikzpicture}
\end{codeexample}  

    This might take a bit of getting used to, but just remember that 
    when you want the bendy sides to be parallel, the sides take the 
    same bend style. It is possible for the top and bottom sides to 
    take opposite bend styles, but the author of this shape cannot 
    think of a single use for such a combination.
    
\begin{codeexample}[]
\begin{tikzpicture}
  \tikzstyle{every node}=[tape, draw]
  \node [tape bend top=out and in, tape bend bottom=out and in] {Parallel};
  \node at (2,0) [tape bend bottom=out and in]                  {Why?};
\end{tikzpicture}
\end{codeexample} 

	\end{key}
	
	\begin{key}{/pgf/tape bend bottom=\meta{bend style} (initially in and out)}
		Specify how the bottom side bends.
	\end{key}%
	
	\begin{key}{/pgf/tape bend height=\meta{length} (initially 5pt)}
		Set the total height for a side with a bend.
		
\begin{codeexample}[]
\begin{tikzpicture}[>=stealth]
  \draw [help lines] grid(3,2);
  \node [tape, fill, minimum size=2cm, red!50, tape bend top=none,
         tape bend height=1cm] at (1.5,1.5) (t) {};
  \draw [|<->|, blue] (1.5,0) -- (1.5,1) 
         node [at end, above, black]{tape bend height};
\end{tikzpicture}
\end{codeexample} 
 
	\end{key}
	
	The anchors for the tape shape are shown below. Anchor |60| is an
	example of a border anchor. Note that border anchors will snap to
	the center of convex curves (i.e. when bending in). 
	
\begin{codeexample}[]
\Huge
\begin{tikzpicture}
  \node[name=s, shape=tape, tape bend height=1cm, shape example, inner xsep=3cm] 
    {Tape\vrule width1pt height2cm};
   \foreach \anchor/\placement in
    {text/left,  center/above,    60/above, 
     base/below, base east/below, base west/below,
     mid/right,  mid east/left,   mid west/right,  
     north/above, south/below,  east/above, west/above,        
     north west/above, north east/above, 
     south west/below, south east/below}
     \draw[shift=(s.\anchor)] plot[mark=x] coordinates{(0,0)}
       node[\placement] {\scriptsize\texttt{(s.\anchor)}};
\end{tikzpicture}
\end{codeexample}

\end{shape}%



\subsection{Arrow Shapes}

\begin{pgflibrary}{shapes.arrows}
  This library defines arrow shapes. Note that an arrow shape is
  something quite different from a (normal) arrow tip: It is a shape
  that just ``happens'' to ``look like'' an arrow. In particular, you
  cannot use these shapes as arrow tips.
\end{pgflibrary}

\begin{shape}{single arrow}
	This shape is an arrow, which tightly fits the note contents 
	(including any |inner sep|). 
	This shape supports the rotation of the shape border, as 
	described in Section~\ref{section-rotating-shape-borders}. 
	The angle of rotation determines which direction the arrow
	points (provided no other rotational transformations are applied).
	
\begin{codeexample}[]
\begin{tikzpicture}[every node/.style={single arrow, draw},
    rotate border/.style={shape border uses incircle, shape border rotate=#1}]
  \node {right};
  \node at (2,0) [shape border rotate=90]{up};
  \node at (1,1) [rotate border=37, inner sep=0pt]{$37^\circ$};
\end{tikzpicture}
\end{codeexample}

	Regardless of the rotation of the arrow border, the width is 
  measured between the back ends of the arrow head, and the 
  height is measured from the arrow tip to the end of the arrow 
  tail.

\begin{codeexample}[]
\begin{tikzpicture}[>=stealth, 
    rotate border/.style={shape border uses incircle, shape border rotate=#1}]
  \node[rotate border=-30, fill=gray!25, minimum height=3cm, single arrow, 
    single arrow head extend=.5cm, single arrow head indent=.25cm] (arrow) {};
  \draw[red, <->] (arrow.before tip) -- (arrow.after tip)
    node [near end, left, black] {width};
  \draw[red, <->] (arrow.tip) -- (arrow.tail)
    node [near end, below left, black] {height};
\end{tikzpicture}
\end{codeexample}

	There are \pgfname{} keys that can be used to customize this shape (to
	use these keys in \tikzname{}, simply remove the \declare{|/pgf/|}
	path).
	
\begin{key}{/pgf/single arrow tip angle=\meta{angle} (initially 90)}
  Set the angle for the arrow tip. Enlarging the arrow to some
  minimum width may increase the the height of the shape to maintain
  this angle.
\end{key}

\begin{key}{/pgf/single arrow head extend=\meta{length} (initially .5cm)}
  This sets the distance between the tail of the arrow and the outer
  end of the arrow head. This may change if the shape is enlarged to
  some minimum width.
  
\begin{codeexample}[]
\begin{tikzpicture}
  \node[single arrow, draw, single arrow head extend=.5cm, gray!50, rotate=60] 
     (a) {Arrow};
  \draw[red, |<->|] (a.before tip) -- (a.before head) 
    node [midway, below, sloped, black] {head extend};
\end{tikzpicture}
\end{codeexample}
\end{key}

\begin{key}{/pgf/single arrow head indent=\meta{length} (initially 0cm)}
  This moves the point where the arrow head joins the shaft of the
  arrow \emph{towards} the arrow tip, by \meta{length}.
  
\begin{codeexample}[]
\begin{tikzpicture}[every node/.style={single arrow, draw=none, rotate=60}]
  \node [fill=red!50]                                           {arrow 1};
  \node [fill=blue!50, single arrow head indent=1ex] at (1.5,0) {arrow 2};
\end{tikzpicture}
\end{codeexample}
\end{key}

  The anchors for this shape are shown below (anchor |20| is an 
  example of a border anchor).
  
\begin{codeexample}[]
\Huge
\begin{tikzpicture}
  \node[name=s,shape=single arrow, shape example, single arrow head extend=1.5cm] 
    {Single Arrow\vrule width1pt height2cm};
  \foreach \anchor/\placement in
    {text/above,      center/above, 20/above,
     mid west/left,   mid/above,    mid east/above left,
     base west/below, base/below,   base east/below,
     tip/above, before tip/above, after tip/below, before head/above, 
     after head/below, after tail/above, before tail/below, tail/right,   
     north/above, south/below, east/below, west/above,
     north west/above, north east/below, south west/below, south east/above}    
     \draw[shift=(s.\anchor)] plot[mark=x] coordinates{(0,0)}
       node[\placement] {\scriptsize\texttt{(s.\anchor)}};
\end{tikzpicture}
\end{codeexample}

\end{shape}





\begin{shape}{double arrow}
  This shape is a double arrow, which tightly fits the note contents 
	(including any |inner sep|), and supports the rotation of the shape
	 border, as described in Section~\ref{section-rotating-shape-borders}. 
	 
	
\begin{codeexample}[]
\begin{tikzpicture}[every node/.style={double arrow, draw}]
  \node [double arrow, draw] {Left or Right};
\end{tikzpicture}
\end{codeexample}

  The double arrow behaves exactly like the single arrow, so you
  need to remember that the width is \emph{always} the distance
  between the back ends of the arrow heads, and the height
  is \emph{always} the the tip-to-tip distance.
  
\begin{codeexample}[]
\begin{tikzpicture}[>=stealth, 
    rotate border/.style={shape border uses incircle, shape border rotate=#1}]
  \node[rotate border=210, fill=gray!25, minimum height=3cm, double arrow, 
    double arrow head extend=.5cm, double arrow head indent=.25cm] (arrow) {};
  \draw[red, <->] (arrow.before tip 1) -- (arrow.after tip 1)
    node [near start, right, black] {width};
  \draw[red, <->] (arrow.tip 1) -- (arrow.tip 2)
    node [near end, above left, black] {height};
\end{tikzpicture}
\end{codeexample}

  The \pgfname{} keys that can be used to customize the double arrow 
  behave similarly to the keys for the single arrow (to
	use these keys in \tikzname{}, simply remove the \declare{|/pgf/|}
	path).
  
\begin{key}{/pgf/double arrow tip angle=\meta{angle} (initially 90)}
  Set the angle for the arrow tip. Enlarging the arrow to some
  minimum width may increase the the height of the shape to maintain
  this angle.
\end{key}

\begin{key}{/pgf/double arrow head extend=\meta{length} (initially .5cm)}
  This sets the distance between the shaft of the arrow and the outer
  end of the arrow heads. This may change if the shape is enlarged to
  some minimum width.
\end{key}

\begin{key}{/pgf/double arrow head indent=\meta{length} (initially 0cm)}
  This moves the point where the arrow heads join the shaft of the
  arrow \emph{towards} the arrow tips, by \meta{length}.
  \begin{codeexample}[]
\begin{tikzpicture}[every node/.style={double arrow, draw=none, rotate=-60}]
  \node [fill=red!50]                                           {arrow 1};
  \node [fill=blue!50, double arrow head indent=1ex] at (1.5,0) {arrow 2};
\end{tikzpicture}
\end{codeexample}
\end{key}


  The anchors for this shape are shown below (anchor |20| is an 
  example of a border anchor).
  
  
\begin{codeexample}[]
\Huge
\begin{tikzpicture}
  \node[name=s,shape=double arrow, double arrow head extend=1.5cm, shape example, inner xsep=2cm] 
    {Double Arrow\vrule width1pt height2cm};
  \foreach \anchor/\placement in
    {text/above, center/above, 20/above,
     mid west/above right, mid/above, mid east/above left,
     base west/below, base/below, base east/below,
     before head 1/above, before tip 1/above, tip 1/above, after tip 1/below, after head 1/below,
     before head 2/above, before tip 2/below, tip 2/above, after tip 2/above, after head 2/below,
     north/above, south/below, east/below, west/below,
     north west/below, north east/below, south west/above, south east/above}    
     \draw[shift=(s.\anchor)] plot[mark=x] coordinates{(0,0)}
       node[\placement] {\scriptsize\texttt{(s.\anchor)}};
\end{tikzpicture}
\end{codeexample}
\end{shape}




\begin{shape}{arrow box}
This shape is a rectangle with optional arrows which extend from the
four sides.

\begin{codeexample}[]
\begin{tikzpicture}
  \node[arrow box, draw] {A};
  \node[arrow box, draw, arrow box arrows={north:.5cm, west:0.75cm}] 
    at (2,0) {B};
\end{tikzpicture}
\end{codeexample}

Any minimum size requirements are applied to the main rectangle
\emph{only}. This does not pose too many problems if you wish to
accommodate the length of the arrows, as it is possible to specify 
the length of each arrow independently, from either the border of the
rectangle (the default) or the center of the rectangle.

\begin{codeexample}[]
\begin{tikzpicture}
	\tikzset{box/.style={arrow box, fill=#1}}
	\draw [help lines] grid(3,2);
  \node[box=blue!50, arrow box arrows={east:2cm}]             at (1,1.5){One};
  \node[box=red!50,  arrow box arrows={east:2cm from center}] at (1,0.5){Two};
\end{tikzpicture}
\end{codeexample}

There are various \pgfname{} keys for drawing this shape (to use these
keys in \tikzname, simply remove the \declare{/pgf/} path).

\begin{key}{/pgf/arrow box tip angle=\meta{angle} (initially 90)}
  Set the angle at the arrow tip for all four arrows. 
\end{key}

\begin{key}{/pgf/arrow box head extend=\meta{length} (initially .125cm)}
  Set the the distance the arrow head extends away from the the shaft
  of the arrow. This applies to all arrows.
\end{key}

\begin{key}{/pgf/arrow box head indent=\meta{length} (initially 0cm)}
  Move the point where the arrow head joins the shaft of the arrow
  \emph{towards} the arrow tip. This applies to all arrows.
\end{key}

\begin{key}{/pgf/arrow box shaft width=\meta{length} (initially .125cm)}
  Set the width of the shaft of all arrows.
\end{key}

\begin{key}{/pgf/arrow box north arrow=\meta{distance} (initially .5cm)}
  Set distance the north arrow extends from the node. By default this
  is from the border of the shape, but by using the additional keyword
  |from center|, the distance will be measured from the center of the
  shape. If \meta{distance} is |0pt| or a negative distance, the arrow
  will not be drawn.
\end{key}

\begin{key}{/pgf/arrow box south arrow=\meta{distance} (initially .5cm)}
	Set distance the south arrow extends from the node.
\end{key}

\begin{key}{/pgf/arrow box east arrow=\meta{distance} (initially .5cm)}
	Set distance the east arrow extends from the node.
\end{key}

\begin{key}{/pgf/arrow box west arrow=\meta{distance} (initially .5cm)}
	Set distance the west arrow extends from the node.
\end{key}

\begin{key}{/pgf/arrow box arrows={\ttfamily\char`\{}\meta{list}{\ttfamily\char`\}}}
  Set the distance that all arrows extend from the node. The 
  specification in \meta{list} consists of the four compass points
  |north|, |south|, |east| or |west|, separated by commas (so the list
  must be contained within braces). 
  The distances can be specified after each side separated by a colon 
  (e.g., |north:1cm|, or |west:5cm from center|). 
  If an item specifies no distance, the most recently specifed 
  distance will be used (at the start of the list this is |0cm|,
  so the first item in the list should specify a distance). 
  Any sides not specified will not be drawn with an arrow.
\end{key}

The anchors for this shape are shown below (unfortunately due to its
size, this example must be rotated). Anchor |75| is an example of a
border anchor.
If a side is drawn without an arrow, the anchors for that arrow should 
be considered unavailable. They are (unavoidably) defined, but default 
to the center of the appropriate side.

\begin{codeexample}[]
\Huge
\begin{tikzpicture}
  \node[shape=arrow box, shape example, inner xsep=1cm, inner ysep=1.5cm, arrow box shaft width=2cm,
    arrow box arrows={north:3.5cm from border, south, east:5cm from border, west}, 
    arrow box head extend=0.75cm, rotate=-90](s) {Arrow Box\vrule width1pt height2cm};
  \foreach \anchor/\placement in
   {center/above, text/above, mid/right, base/below, 75/above,
    mid east/right, mid west/left, base east/right, base west/left,
    north/below, south/below, east/below, west/below,
    north east/above, south east/above, south west/below, north west/below,
    north arrow tip/above,south arrow tip/above, east arrow tip/above, west arrow tip/above,
    before north arrow/above, before north arrow head/below left, before north arrow tip/above left, 
    after north arrow tip/above right, after north arrow head/below right, after north arrow/below,
    before south arrow/below, before south arrow head/above right, before south arrow tip/below right, 
    after south arrow tip/below left, after south arrow head/above left, after south arrow/above,
    before east arrow/above, before east arrow head/above right, before east arrow tip/above, 
    after east arrow tip/below, after east arrow head/below right, after east arrow/below, 
    before west arrow/below, before west arrow head/below left, before west arrow tip/below, 
    after west arrow tip/above, after west arrow head/above left, after west arrow/below}
      \draw[shift=(s.\anchor)] plot[mark=x] coordinates{(0,0)}
        node[\placement, rotate=-90] {\scriptsize\texttt{(s.\anchor)}};
\end{tikzpicture}
\end{codeexample}

\end{shape}



\subsection{Shapes with Multiple Text Parts}

\begin{pgflibrary}{shapes.multipart}
  This library defines general-purpose shapes that are composed of
  multiple (text) parts. 
\end{pgflibrary}


\begin{shape}{circle split}
  This shape is a multi-part shape consisting of a circle with a line
  in the middle. The upper part is the main part (the |text| part),
  the lower part is the |lower| part.
  
\begin{codeexample}[]
\begin{tikzpicture}
  \node [circle split,draw,double,fill=red!20]
  {
    $q_1$
    \nodepart{lower}
    $00$
  };
\end{tikzpicture}
\end{codeexample}

  The shape inherits all anchors from the |circle| shape and defines
  the |lower| anchor in addition. See also the
  following figure:
\begin{codeexample}[]
\Huge
\begin{tikzpicture}
  \node[name=s,shape=circle split,shape example] {text\nodepart{lower}lower};
  \foreach \anchor/\placement in
    {north west/above left, north/above, north east/above right, 
     west/left, center/below, east/right, 
     mid west/right, mid/above, mid east/left, 
     base west/left, base/below, base east/right, 
     south west/below left, south/below, south east/below right, 
     text/left, lower/left, 130/above}
     \draw[shift=(s.\anchor)] plot[mark=x] coordinates{(0,0)}
       node[\placement] {\scriptsize\texttt{(s.\anchor)}};
\end{tikzpicture}
\end{codeexample}
\end{shape}


\begin{shape}{circle solidus}
  This shape (due to Manuel Lacruz) is similar to the split circle,
  but the two text parts are arranged diagonally.
  
\begin{codeexample}[]
\begin{tikzpicture}
  \node [circle solidus,draw,double,fill=red!20]
  {
    $q_1$
    \nodepart{lower}
    $00$
  };
\end{tikzpicture}
\end{codeexample}

\begin{codeexample}[]
\Huge
\begin{tikzpicture}
  \node[name=s,shape=circle solidus,shape example,inner xsep=1cm] {text\nodepart{lower}lower};
  \foreach \anchor/\placement in
    {north west/above left, north/above, north east/above right, 
     west/left, center/below, east/right, 
     mid west/right, mid/above, mid east/left, 
     base west/left, base/below, base east/right, 
     south west/below left, south/below, south east/below right, 
     text/left, lower/left, 130/above}
     \draw[shift=(s.\anchor)] plot[mark=x] coordinates{(0,0)}
       node[\placement] {\scriptsize\texttt{(s.\anchor)}};
\end{tikzpicture}
\end{codeexample}
\end{shape}


\begin{shape}{ellipse split}
  This shape is a multi-part shape consisting of an ellipse with a line
  in the middle. The upper part is the main part (the |text| part),
  the lower part is the |lower| part.  
  The anchors for this shape are shown below. Anchor |60| is a border anchor.
\begin{codeexample}[]
\Huge
\begin{tikzpicture}
  \node[name=s,shape=ellipse split,shape example] {text\nodepart{lower}lower};
  \foreach \anchor/\placement in
    {center/below, text/left, lower/left, 60/above right,
     mid/above, mid east/above, mid west/above,
     base/right, base east/left, base west/right,
     north/above, south/below, east/below, west/below,
     north east/above, south east/below, south west/below, north west/above}
     \draw[shift=(s.\anchor)] plot[mark=x] coordinates{(0,0)}
       node[\placement] {\scriptsize\texttt{(s.\anchor)}};
\end{tikzpicture}
\end{codeexample}
\end{shape}


\begin{shape}{rectangle split}
  This shape is a rectangle which can be either split horizontally 
  or vertically into several parts.
	
\begin{codeexample}[]
\begin{tikzpicture}[my shape/.style={
  rectangle split, rectangle split parts=#1, draw, anchor=center}]
  \node [my shape=5] at (0,1)
    {a\nodepart{two}b\nodepart{three}c\nodepart{four}d\nodepart{five}e};
  \node [my shape=5, rectangle split horizontal] at (2,2)
    {1\nodepart{two}2\nodepart{three}3\nodepart{four}4\nodepart{five}5};
  \node [my shape=3] at (3,0.5)
    {A\nodepart{two}B\nodepart{three}C};
  \node [my shape=4, rectangle split horizontal] at (1.5,0.5)
    {1\nodepart{two}2\nodepart{three}3\nodepart{four}4};   
\end{tikzpicture}
\end{codeexample} 


The shape can be split into a maximum of twenty parts. However, to 
avoid allocating a lot of unnecessary boxes, \pgfname{} only allocates 
four boxes by default. 
To use the |rectangle| |split| shape with more than four boxes, the 
extra boxes must be allocated manually in advance (perhaps using |\newbox| 
or |\let|). 
The boxes take the form |\pgfnodepart|\meta{number}|box|, where \meta{number} 
is from the cardinal numbers |one|, |two|, |three|, \ldots{} and so on. 
|\pgfnodepartonebox| is special in that it is synonymous with
|\pgfnodeparttextbox|. For compatability with earlier versions of
this shape, the boxes |\pgfnodeparttwobox|, |\pgfnodepartthreebox|
and |\pgfnodepartfourbox|, can be referred to using the ordinal 
numbers: |\pgfnodepartsecondbox|, |\pgfnodepartthirdbox|
and |\pgfnodepartfourthbox|. In order to facilitate the allocation of 
these extra boxes, the following key is provided:

\begin{key}{/pgf/rectangle split allocate boxes=\meta{number}}
  This key checks if \meta{number} boxes have been allocated, and if
  not allocates the required boxes using |\newbox| (some ``magic'' is
  peformed to get around the fact that |\newbox| is declared |\outer|
  in plain \TeX).
\end{key}

	When split vertically, the rectangle split will observe any 
	|minimum width| requirements but any |minimum height| will be ignored.
	Conversely when split horizontally, |minimum height| requirements
	will be observed, but any |minimum width| will be ignored.	
	In addition, |inner sep| is applied 
  to every part that is used, so it cannot be specified 
  independently for a particular part.
  
  There are several \pgfname{} keys to specify how the shape is
  drawn. To use these keys in \tikzname, simply remove the 
  \declare{|/pgf/|} path:
  
\begin{key}{/pgf/rectangle split parts=\meta{number} (initially 4)}
  Split the rectangle into \meta{number} parts, 
  which should be in the range |1| to |20|. If more than four parts
  are need, the boxes should be allocated in advance as
  described above.
  
\begin{codeexample}[]
\begin{tikzpicture}[every text node part/.style={text centered}]
  \node[rectangle split, rectangle split parts=3, draw, text width=2.75cm] 
    {Student
     \nodepart{two}
       age:int \\
       name:String
     \nodepart{three}
       getAge():int \\
       getName():String};
\end{tikzpicture}
\end{codeexample} 
\end{key}

\begin{key}{/pgf/rectangle split horizontal=\opt{\meta{boolean}} (default true)}
  This key determines whether the rectangle is split horizontally or vertically
 \end{key}
  
  \begin{key}{/pgf/rectangle split ignore empty parts=\opt{\meta{boolean}} (default true)}
    When \meta{boolean} is true, \pgfname{} will ignore any part 
    that is empty \emph{except the text part}. 
    This effectively overrides the |rectangle split parts| key in that, if 
    3 parts (for example) are specified, but one is empty, only
    two will be shown.
    
\begin{codeexample}[]
\begin{tikzpicture}[every node/.style={draw, anchor=text, rectangle split, 
    rectangle split parts=3}]
  \node {text \nodepart{second} \nodepart{third}third};
  \node [rectangle split ignore empty parts] at (2,0)
        {text \nodepart{second} \nodepart{third}third};
\end{tikzpicture}
\end{codeexample}
  \end{key}
%  
	\begin{key}{/pgf/rectangle split empty part width=\meta{length} (initially 1ex)}
    Set the default width for a node part box if it is empty and
    empty parts are not ignored.
  \end{key}
  
  \begin{key}{/pgf/rectangle split empty part height=\meta{length} (initially 1ex)}
    Set the default height for a node part box if it is empty and
    empty parts are not ignored.
  \end{key}
  
  \begin{key}{/pgf/rectangle split empty part depth=\meta{length} (initially 0ex)}
    Set the default depth for a node part box if it is empty and
    empty parts are not ignored.
  \end{key}
  
  \begin{key}{/pgf/rectangle split part align={\ttfamily\char`\{}\meta{list}{\ttfamily\char`\}} (initially center)}
  	Set the alignment of the boxes inside the node parts.
  	Each item in \meta{list} should be
  	separated by commas (so if there is more than one item in 
  	\meta{list} it must be surrounded by braces).
  	
  	When the rectangle is split vertically, the entries in \meta{list} 
  	must be one of |left|, |right|, or |center|. If \meta{list} has less 
  	entries than node parts then the remaining boxes are aligned 
  	according to the last entry in the list.    
    Note that this only aligns the boxes in each part and \emph{does not} 
    affect the alignment of the contents of the boxes.
    
\begin{codeexample}[]
\def\x{one \nodepart{two} 2 \nodepart{three} three \nodepart{four} 4}
\begin{tikzpicture}[
  every node/.style={rectangle split, rectangle split parts=4, 
    draw}
  ]
  \node[rectangle split part align={center, left, right}] at (0,0)    {\x};
  \node[rectangle split part align={center, left}]        at (1.25,0) {\x};
  \node[rectangle split part align={center}]              at (2.5,0)  {\x};
\end{tikzpicture}
\end{codeexample}
 
 	When the rectangle is split horizontally, the entries in \meta{list} 
	must be one of |top|, |bottom|, |center| or |base|. Note that using
	the value |base| will only makes sense if all the node part boxes are 
	being	aligned in this way. This is because the |base| value aligns
	the boxes in relation to each other, whereas the other values align 
	the boxes in relation to the part of the shape they occupy.


\begin{codeexample}[]
\def\x{\Large w\nodepart{two}x\nodepart{three}\Huge y\nodepart{four}\tiny z}
\begin{tikzpicture}[
  every node/.style={rectangle split, rectangle split parts=4, 
    draw, rectangle split horizontal}
  ]
  \node[rectangle split part align={center, top, bottom}] at (0,0)     {\x};
  \node[rectangle split part align={center, top}]         at (0,-1.25) {\x};
  \node[rectangle split part align={center}]              at (0,-2.5)  {\x};
  \node[rectangle split part align=base]                  at (0,-3.75) {\x};
\end{tikzpicture}
\end{codeexample}

  \end{key}
   
  \begin{key}{/pgf/rectangle split draw splits=\opt{\meta{boolean}} (defualt true)}
  	Set whether the line or lines between node parts will be drawn.
  	Internally, this sets the \TeX-if |\ifpgfrectanglesplitdrawsplits| 
  	appropriately.
  \end{key}
  
  \begin{key}{/pgf/rectangle split use custom fill=\opt{\meta{boolean}} (default true)}
    This enables the use of a custom fill for each of the node
    parts (including the area covered by the |inner sep|). The 
    background path for the shape should not be filled (e.g., in
    \tikzname{}, the |fill|
    option for the node must be implicity or explicitly set to |none|).
    Internally, this key sets the \TeX-if 
    |\ifpgfrectanglesplitusecustomfill| appropriately.
  \end{key}
  
  \begin{key}{/pgf/rectangle split part fill={\ttfamily\char`\{}\meta{list}{\ttfamily\char`\}} (initially white)}
  	Set the custom fill color for each node part shape. 
  	Each item in \meta{list} should be separated by commas (so if 
  	there is more than one item in \meta{list} it must be surrounded 
  	by braces).
  	If \meta{list}  has less entries than node
    parts then the remaining node parts use the color from
    the last entry in the list. This key will automatically set
    |/pgf/rectangle split use custom fill|.
    
\begin{codeexample}[]
\begin{tikzpicture}
  \tikzset{every node/.style={rectangle split, draw, minimum width=.5cm}}
  \node[rectangle split part fill={red!50, green!50, blue!50, yellow!50}]  {};
  \node[rectangle split part fill={red!50, green!50, blue!50}] at (0.75,0) {};
  \node[rectangle split part fill={red!50, green!50}]          at (1.5,0)  {};
  \node[rectangle split part fill={red!50}]                    at (2.25,0) {};
\end{tikzpicture}
\end{codeexample}

\end{key}
	
  The anchors for the |rectangle split| shape split vertically into four, 
  are shown below (anchor |70| is an example of a border angle). When a 
  node part is missing, the anchors prefixed with name of that node part
  should be considered unavailable. They are (unavoidably) defined, but 
  default to other anchor positions.
  
\begin{codeexample}[]
\Huge
\begin{tikzpicture}
  \node[name=s,shape=rectangle split, rectangle split parts=4, shape example,
    inner ysep=0.75cm] 
    {\nodepart{text}text\nodepart{two}two
		\nodepart{three}three\nodepart{four}four};
  \foreach \anchor/\placement in
    {text/left, text east/above, text west/above, 
     two/left, two east/above, two west/above,    
     three/left, three east/below, three west/below,
     four/left, four east/below, four west/below,
     text split/left, text split east/above, text split west/above,
     two split/left, two split east/above, two split west/above,    
     three split/left, three split east/below, three split west/below, 
     north/above, south/below, east/below, west/below,
     north west/above, north east/above, south west/below, south east/below,  
     center/above, 70/above, mid/above, base/below}     
     \draw[shift=(s.\anchor)] plot[mark=x] coordinates{(0,0)}
       node[\placement] {\scriptsize\texttt{(s.\anchor)}};
\end{tikzpicture}
\end{codeexample}


\end{shape}






\subsection{Callout Shapes}

\begin{pgflibrary}{shapes.callout}
  Producing basic callouts can be done quite easily in \pgfname{} and 
	\tikzname{} by creating a node and then subsequently drawing a path
	from the border of the node to the required point. This library
	provides more fancy, `balloon'-style callouts.
    
\end{pgflibrary}

Callouts consist of a 
main shape, and a pointer (which is part of the shape) which points
to something in (or outside) the picture. The position on the border
of the main shape to which the pointer is connected is determined
automatically. However, the pointer is ignored when calculating the 
minimum size of the shape, and also when positioning anchors.

\begin{codeexample}[]
\begin{tikzpicture}[remember picture]
  \node[ellipse callout, draw] (hallo) {Hallo!};
\end{tikzpicture}
\end{codeexample}

There are two kinds of pointer:	the ``relative'' pointer and the 
``absolute'' pointer.	The relative pointer calculates the angle of a
specified coordinate relative to the center of the main shape, locates 
the point on the border to which this angle corresponds, and then adds 
the coordinate to this point. This seemingly over-complex approach 
means than you do not have to guess the size of the main shape: the 
relative pointer will always be outside. 
The absolute pointer, on the 
other hand, is much simpler: it points to the specified coordinate
absolutely (and can even point to named coordinates in different 
pictures).


\begin{codeexample}[]
\begin{tikzpicture}[remember picture, note/.style={rectangle callout, fill=#1}]
  \draw [help lines] grid(3,2);
  \node [note=red!50,  callout relative pointer={(0,1)}] at (3,1) {Relative};
  \node [note=blue!50, callout absolute pointer={(0,1)}] at (1,0) {Absolute};
  \node [note=green!50, opacity=.5, overlay, 
         callout absolute pointer={(hallo.south)}]       at (1,2) {Outside};
\end{tikzpicture}
\end{codeexample}


The following keys are common to all callouts. Please remember
that the |callout| |relative| |pointer|, and |callout| |absolute|
|pointer| keys take a different format for their value depending 
on whether they are being used in \pgfname{} or \tikzname{}.
  
  
\begin{key}{/pgf/callout relative pointer=\meta{coordinate} (initially {\ttfamily\char`\\pgfpointpolar\char`\{315\char`\}\char`\{.5cm\char`\}})}
  
  Set the vector of the callout pointer `relative' to the callout 
  shape. 
  
\end{key}

\begin{key}{/pgf/callout absolute pointer=\meta{coordinate}}
  
  Set the vector of the callout pointer absolutely within the picture.
  
\end{key}



\begin{key}{/tikz/callout relative pointer=\meta{coordinate} (initially {(315:.5cm)})}
	The \tikzname{} version of the |callout relative pointer| key. Here,
	\meta{coordinate} can be specified using the \tikzname{} format for
	coordinates.
\end{key}

\begin{key}{/tikz/callout absolute pointer=\meta{coordinate}}
	The \tikzname{} version of the |callout absolute pointer| key. Here,
	\meta{coordinate} can be specified using the \tikzname{} format for
	coordinates.
\end{key}

	It is also possible to shorten the pointer by some distance, using 
	the following key:
	
\begin{key}{/pgf/callout pointer shorten=\meta{distance} (initially 0cm)}
	Move the callout pointer towards the center of the callouts main 
	shape by \meta{distance}. 
	
\begin{codeexample}[]
\begin{tikzpicture}
	\tikzset{callout/.style={ellipse callout, callout pointer arc=30,
	  callout absolute pointer={#1}}}
  \draw (0,0) grid (3,2);
  \node[callout={(3,1.5)}, fill=red!50] at (0,1.5) {A};
  \node[callout={(3,.5)},  fill=green!50, callout pointer shorten=1cm]          
    at (0,.5)  {B};
\end{tikzpicture} 
\end{codeexample}
\end{key}
	
  
\begin{shape}{rectangle callout}%
	
  This shape is a callout whose main shape is a rectangle, which 
  tightly fits the node contents (including any |inner sep|).
  It  supports the keys described above and also the following 
  key:
  
  
\begin{key}{/pgf/callout pointer width=\meta{length} (initially .25cm)} 
  Set the width of the pointer at the border of the rectangle.
\end{key}
		
	The anchors for this shape are shown below (anchor |60| is an
	example of a border anchor). The pointer direction is ignored when 
	placing anchors. 
	Additionally, when using an absolute pointer, the |pointer| 
	anchor should not be used to used to position the shape as it is 
	calculated whilst the shape is being drawn.
	
\begin{codeexample}[]
\Huge
\begin{tikzpicture}
  \node[name=s,shape=rectangle callout, callout relative pointer={(1.25cm,-1cm)}, 
     callout pointer width=2cm, shape example, inner xsep=2cm, inner ysep=1cm] 
  	{Rectangle Callout\vrule width 1pt height 2cm};
  \foreach \anchor/\placement in
    {center/above, text/below,      60/above,
     mid/right,    mid west/left,  mid east/right, 
     base/below,   base west/below, base east/below, 
     north/above,  south/below, east/above, west/above,
     north west/above, north east/above,     
     south west/below, south east/below,
     pointer/below}
     \draw[shift=(s.\anchor)] plot[mark=x] coordinates{(0,0)}
       node[\placement] {\scriptsize\texttt{(s.\anchor)}};
\end{tikzpicture}
\end{codeexample}

\end{shape}%


\begin{shape}{ellipse callout}%
	
  This shape is a callout whose main shape is a ellipse, which 
  tightly fits the node contents (including any |inner sep|).
  It uses the |absolute callout pointer|, 
	|relative callout pointer| and |callout pointer shorten| keys, and
	also the following key:
  
  
\begin{key}{/pgf/callout pointer arc=\meta{angle} (initially 15)} 
  Set the width of pointer at the border of the ellipse according
  to an arc of length \meta{angle}.
\end{key}
	
	
	The anchors for this shape are shown below (anchor |60| is an
	example of a border anchor). The pointer direction is ignored 
	when placing anchors and the |pointer| anchor can only be
	used to position the shape when the relative anchor is 
	specified.
	
\begin{codeexample}[]
\Huge
\begin{tikzpicture}
  \node[name=s,shape=ellipse callout, callout relative pointer={(1.25cm,-1cm)}, 
    callout pointer width=2cm, shape example, inner xsep=1cm, inner ysep=.5cm] 
  	{Ellipse Callout\vrule width 1pt height 2cm};
  \foreach \anchor/\placement in
    {center/above, text/below,      60/above,
     mid/above,    mid west/right,  mid east/left, 
     base/below,   base west/below, base east/below, 
     north/above,  south/below, east/above, west/above,
     north west/above left,    north east/above right,     
     south west/below left,    south east/below right,
     pointer/below}
     \draw[shift=(s.\anchor)] plot[mark=x] coordinates{(0,0)}
       node[\placement] {\scriptsize\texttt{(s.\anchor)}};
\end{tikzpicture}
\end{codeexample}

\end{shape}


\begin{shape}{cloud callout}
This shape is a callout whose main shape is a cloud which fits the 
node contents. The pointer is segmented, consisting of a series of 
shrinking ellipses. This callout requires the symbol shape library 
(for the cloud shape). If this library is not loaded an error will 
result.

\begin{codeexample}[]
\begin{tikzpicture}
  \node[cloud callout, cloud puffs=15, aspect=2.5, cloud puff arc=120, 
    shading=ball,text=white] {\bf Imagine...};
\end{tikzpicture}
\end{codeexample}

The |cloud callout| supports the |absolute callout pointer|,
|relative callout pointer| and |callout pointer shorten| keys, as 
described above.
The main shape can be modified using the same keys as the |cloud| 
shape. The following keys are also supported:

\begin{key}{/pgf/callout pointer start size=\meta{value} (initially .2 of callout)}
	Set the size of the first segment in the pointer (i.e., the segment
	nearest the main cloud shape). There are three possible forms for
	\meta{value}:
	\begin{itemize}
		\item
			A single dimension (e.g., |5pt|), in which case the first ellipse 
			will have equal diameters of 5pt.
		\item
			Two dimensions (e.g., |10pt and 2.5pt|), which sets the $x$ and 
			$y$ diameters of the first ellipse.
		\item
			A decimal fraction (e.g., |.2 of callout|), in which case
			the $x$ and $y$ diameters of the first ellipse will be set as 
			fractions of the width and height of the main shape. The keyword
			|of callout| cannot be omitted.	
	\end{itemize}
\end{key}

\begin{key}{/pgf/callout pointer end size=\meta{value} (initially .1 of callout)}
	Set the size of the last ellipse in the pointer.
\end{key}

\begin{key}{/pgf/callout pointer segments=\meta{number} (initially 2)}
	Set the number of segments in the pointer. Note that \pgfname{} will 
	happily	overlap segments if too many are specified. 
\end{key}

The anchors for this shape are shown below (anchor |70| is an example 
of a border anchor). The pointer direction is ignored when placing 
anchors and the pointer anchor can only be used to position the
shape when the relative anchor is specified. Note that the center
of the last segment is drawn at the |pointer| anchor.

\begin{codeexample}[]
\Huge
\begin{tikzpicture}
  \node[name=s, shape=cloud callout, style=shape example, cloud puffs=11, aspect=1.5,
    cloud puff arc=120,inner xsep=.5cm, callout pointer start size=.25 of callout,
    callout pointer end size=.15 of callout, callout relative pointer={(315:4cm)},
    callout pointer segments=2] {Cloud Callout\vrule width 1pt height 2cm};
  \foreach \anchor/\placement in
    {puff 1/above, puff 2/above, puff 3/above, puff 4/below,
     puff 5/left, puff 6/below, puff 7/below, puff 8/right,
     puff 9/below, puff 10/above, puff 11/above, 70/right,
     center/above, base/below, mid/right, text/left,
     north/below, south/below, east/above, west/above,
     north west/left, north east/right,
     south west/below, south east/below,pointer/above}
  \draw[shift=(s.\anchor)] plot[mark=x] coordinates{(0,0)}
    node[\placement] {\scriptsize\texttt{(s.\anchor)}};
\end{tikzpicture}
\end{codeexample}%
\end{shape}




\subsection{Logic Gate Shapes}

\subsubsection{Overview}
	\pgfname{} provides two libraries of logic gates, one providing
	``American'' style gates and the other, providing ``rectangular'' 
	logic gates.	
	Each library suffixes the gate names with an identifer:
	|US| for the American style gates, and |IEC| for the rectangular
	gates (additionally, two shapes in the |US| library use the
	suffix |CDH|). Keys which are specific
	to a particular library	also contain this identifier (e.g., 
	|/pgf/and gate IEC symbol|).
	However, as described below, a \tikzname{} key is provided which
	sets up several styles allowing the identifier to be omitted,
	for example, |and gate| can become a synonym for |shape=and gate US|.
	
	Multiple inputs can be specified for a logic gate (provided they
  support multiple inputs: a not gate --- also known as an inverter ---
  does not). However, there is an upper limit for the number of inputs 
  which has been set at 1024, which should be \emph{way} 
  more than would ever be needed.
    
  There are some \pgfname{} keys which are common to both 
  libraries, which have no library identifier contained in them:
  
  
  \begin{key}{/pgf/logic gate inputs=\meta{input list} (initially \char`\{normal,normal\char`\})}
  Specify the inputs for for the logic gate. The keyword |inverted|
  indicates an inverted input which will mean \pgfname{} will draw a
  circle attached to the main shape of the logic gate. Any keyword
  that is not |inverted| will be treated as a ``normal'' or 
  ``non-inverted'' input (however, for readability, you may wish to 
  use |normal| or |non-inverted|), and \pgfname{} will not draw the 
  circle.  
  In both cases the anchors for the inputs will be set 
  up appropriately, numbered from top to bottom |input 1|, |input 2|,
  \ldots and so on. If the gate only supports one input the anchor
  is simply called |input| with no numerical index.
  
\begin{codeexample}[]
\begin{tikzpicture}[minimum height=0.75cm]
  \node[and gate IEC, draw, logic gate inputs={inverted, normal, inverted}] 
  (A) {};
  \foreach \a in {1,...,3}
    \draw (A.input \a -| -1,0) -- (A.input \a);
  \draw (A.output) -- ([xshift=0.5cm]A.output);
\end{tikzpicture}
\end{codeexample} 
  
  For multiple inputs it may be somewhat unweildy to specify a long
  list, thus, the following ``shorthand'' is permitted (this is an  
  extension of ideas due to Juergen Werber and Christoph Bartoschek):
  Using |i| for inverted and |n| for normal inputs, \meta{input list}
  can be specfied \emph{without the commas}. So, for example,
  |ini| is equivalent to |inverted, normal, inverted|.
  
\begin{codeexample}[]
\begin{tikzpicture}[minimum height=0.75cm]
  \node[or gate US, draw,logic gate inputs=inini] (A) {};
  \foreach \a in {1,...,5}
    \draw (A.input \a -| -1,0) -- (A.input \a);
  \draw (A.output) -- ([xshift=0.5cm]A.output);
\end{tikzpicture}
\end{codeexample} 
 
\end{key}


The height of the gate may be increased to accommodate the number 
of inputs. In fact, it depends on three variables:
$n$, the number of inputs, $r$, the radius of the circle used
to indicate an inverted input and $s$, the distance between
the centers of the inputs.
The default height is then calculated according to the expression 
$(n+1)\times\max(2r,s)$. This then may
be increased to accommodate the node contents or any
minimum size specifications.

The radius of the inverted input circle and the distance between the 
centers of the inputs can be customised using the following keys:

\begin{key}{/pgf/logic gate inverted radius=\meta{length} (initially 2pt)}
  Set the radius of the circle that is used to indicate inverted
  inputs. This is also the radius of the circle used for the inverted
  output of the |nand|, |nor|, |xnor| and |not| gates. 
    
\begin{codeexample}[]
\begin{tikzpicture}[minimum height=0.75cm]
  \tikzset{every node/.style={shape=nand gate CDH, draw, logic gate inputs=ii}}
  \node[logic gate inverted radius=2pt] {A};
  \node[logic gate inverted radius=4pt] at (0,-1) {B};
\end{tikzpicture}
\end{codeexample} 
\end{key}

\begin{key}{/pgf/logic gate input sep=\meta{length} (initially .125cm)}
  Set the distance between the \emph{centers} of the inputs to the
  logic gate. 
  
\begin{codeexample}[]
\begin{tikzpicture}[minimum size=0.75cm]
  \draw [help lines] grid (3,2);
  \tikzset{every node/.style={shape=and gate IEC, draw, logic gate inputs=ini}}
  \node[logic gate input sep=0.33333cm] at (1,1)(A) {A};
  \node[logic gate input sep=0.5cm]     at (3,1) (B) {B};
  \foreach \a in {1,...,3}
    \draw (A.input \a -| 0,0) -- (A.input \a)
          (B.input \a -| 2,0) -- (B.input \a);
\end{tikzpicture}
\end{codeexample} 
\end{key}


\subsubsection{US Logic Gates}

\begin{pgflibrary}{shapes.gates.logic.US}
 This library provides ``American'' logic gate shapes whose names are 
  suffixed with the identifier |US|. Additionally,
  alternative |and| and |nand| gates are provided which are based on the 
  logic symbols used in A. Croft, R. Davidson, and M. Hargreaves (1992), 
  \emph{Engineering Mathematics}, Addison-Wesley, 82--95. These two 
  shapes are suffixed with |CDH|. 
\end{pgflibrary}

  To use the shapes in \tikzname{} without their suffixes, the 
  following keys are provided:
  
\begin{key}{/tikz/use US style logic gates}
	This allows the the shapes suffixed with |US| to be used without
	the suffix. So, for example, |and gate| becomes a synonym for
	|shape=and gate US|.
\begin{codeexample}[]
\tikz\node[draw, and gate US, red]{and};
\space
\tikz[use US style logic gates,blue]\node[draw, and gate]{and};
\end{codeexample}
\end{key}

\begin{key}{/tikz/use CDH style logic gates}
	This key again allows the the shapes suffixed with |US| to be used 
	without	the |US| suffix. However, |and gate| becomes a synonym for
	|shape=and gate CDH| and |nand gate| becomes a synonym for
	|shape=nand gate CDH|, providing alternative symbols for these
	gates.
	
\begin{codeexample}[]
\begin{tikzpicture}[minimum height=1cm]
  \node[draw, and gate US, red]  at  (0,1.5) {and};
  \node[draw, nand gate US, red] at (2,1.5) {nand};
  \tikzset{use CDH style logic gates}
  \node[draw, and gate, blue]  at (0,0) {and};
  \node[draw, nand gate, blue] at (2,0) {nand};
\end{tikzpicture}
\end{codeexample}
\end{key}



As described above, \pgfname{} will increase the size of the 
logic gate to accommodate the number of inputs, and the size
of the inverted radius and the separation between the inputs.
However with all shapes in this library, any increase in size 
(including any minimum size requirements) will be applied so that 
the default aspect ratio is unaltered. This means that changing
the height will change the width and vice versa. 

The ``compass point'' anchors apply to the main part of the shape
and do not include any inverted inputs or outputs. This library
provides an additonal feature to facilitate the relative positioning
of logic gates:

\begin{key}{/pgf/logic gate anchors use bounding box=\meta{boolean} (initially false)}
When set to |true| this key will ensure that the 
compass point anchors use the bounding rectangle of the
main shape, which, ignore any inverted inputs or outputs, but
includes any |outer sep|. 
This \emph{only} affects the compass point anchors
and is not set on a shape by shape basis: whether the bounding
box is used is determined by value of this key when the anchor
is accessed.

\begin{codeexample}[]
\begin{tikzpicture}[minimum height=1.5cm]
  \node[xnor gate US, draw, gray!50,line width=2pt] (A) {};
  \foreach \x/\y/\z in {false/blue/1pt, true/red/2pt}
    \foreach \a in {north, south, east, west, north east, 
      south east, north west, south west}
      \draw[logic gate anchors use bounding box=\x, color=\y]	
        (A.\a) circle(\z);
\end{tikzpicture}
\end{codeexample} 



\end{key}




\begin{shape}{and gate US}
  This shape is an and gate which supports two or more inputs. If
	less than two inputs are specified an error will result. 
	The anchors for this gate with two
  non-inverted inputs (using the normal compass point anchors) are
  shown below. Anchor |30| is an example of a border anchor.
  
\begin{codeexample}[]
\Huge
\begin{tikzpicture}
  \node[name=s,shape=and gate US,shape example, inner sep=0cm,
    logic gate inverted radius=.5cm] {And Gate\vrule width1pt height2cm};
  \foreach \anchor/\placement in
    {center/above, text/above, 30/above right,
     mid/right, mid east/left, mid west/above,
     base/below, base east/right, base west/left,
     north/above, south/below, east/above, west/above,
     north east/above, south east/below, south west/below, north west/above,
     output/right, input 1/above, input 2/below}
     \draw[shift=(s.\anchor)] plot[mark=x] coordinates{(0,0)}
       node[\placement] {\scriptsize\texttt{(s.\anchor)}};
\end{tikzpicture}
\end{codeexample}
\end{shape}

\begin{shape}{nand gate US}
  This shape is a nand gate, which supports two or more inputs. If
	less than two inputs are specified an error will result. 
	The anchors for this gate with two
  non-inverted inputs (using the normal compass point anchors) are
  shown below. Anchor |30| is an example of a border anchor.
 
\begin{codeexample}[]
\Huge
\begin{tikzpicture}
  \node[name=s,shape=nand gate US,shape example, inner sep=0cm,
  logic gate inverted radius=.5cm] {Nand Gate\vrule width1pt height2cm};
  \foreach \anchor/\placement in
    {center/above, text/above, 30/above right,
     mid/right, mid east/left, mid west/above,
     base/below, base east/below, base west/left,
     north/above, south/below, east/above, west/above,
     north east/above, south east/below, south west/below, north west/above,
     output/right, input 1/above, input 2/below}
     \draw[shift=(s.\anchor)] plot[mark=x] coordinates{(0,0)}
       node[\placement] {\scriptsize\texttt{(s.\anchor)}};
\end{tikzpicture}
\end{codeexample}

\end{shape}

\begin{shape}{or gate US}

	This shape is an or gate, which supports two or more inputs. If
	less than two inputs are specified an error will result. 
	The anchors for this gate with two
  non-inverted inputs (using the normal compass point anchors) are
  shown below. Anchor |30| is an example of a border anchor.
  
\begin{codeexample}[]
\Huge
\begin{tikzpicture}
  \node[name=s,shape=or gate US,shape example, inner sep=0cm,
  logic gate inverted radius=.5cm] {Or Gate\vrule width1pt height2cm};
  \foreach \anchor/\placement in
    {center/above, text/above, 30/above right,
     mid/right, mid east/left, mid west/above,
     base/below, base east/below, base west/left,
     north/above, south/below, east/above, west/above,
     north east/above, south east/below, south west/below, north west/above,
     output/right, input 1/left, input 2/below}
     \draw[shift=(s.\anchor)] plot[mark=x] coordinates{(0,0)}
       node[\placement] {\scriptsize\texttt{(s.\anchor)}};
\end{tikzpicture}
\end{codeexample}
\end{shape}

\begin{shape}{nor gate US}
	This shape is a nor gate, which supports two or more inputs. If
	less than two inputs are specified an error will result. 
	The anchors for this gate with two
  non-inverted inputs (using the normal compass point anchors) are
  shown below. Anchor |30| is an example of a border anchor.

\begin{codeexample}[]
\Huge
\begin{tikzpicture}
  \node[name=s,shape=nor gate US,shape example, inner sep=0cm,
  logic gate inverted radius=.5cm] {Nor Gate\vrule width1pt height2cm};
  \foreach \anchor/\placement in
    {center/above, text/above, 30/above right,
     mid/right, mid east/left, mid west/above,
     base/below, base east/below, base west/left,
     north/above, south/below, east/above, west/above,
     north east/above, south east/below, south west/below, north west/above,
     output/right, input 1/left, input 2/below}
     \draw[shift=(s.\anchor)] plot[mark=x] coordinates{(0,0)}
       node[\placement] {\scriptsize\texttt{(s.\anchor)}};
\end{tikzpicture}
\end{codeexample}

\end{shape}

\begin{shape}{xor gate US}
  This shape is an xor gate, which supports only two inputs. If
  less than two inputs are specified an error will result. If more
  than two inputs are specified, the extra inputs are ignored.
  The anchors for this gate with two
  non-inverted inputs (using the normal compass point anchors) are
  shown below. Anchor |30| is an example of a border anchor.
  
\begin{codeexample}[]
\Huge
\begin{tikzpicture}
  \node[name=s,shape=xor gate US,shape example, inner sep=0cm,
    logic gate inverted radius=.5cm] {Xor Gate\vrule width1pt height2cm};
  \foreach \anchor/\placement in
    {center/above, text/above, 30/above right,
     mid/right, mid east/left, mid west/above,
     base/below, base east/below, base west/left,
     north/above, south/below, east/above, west/above,
     north east/above, south east/below, south west/below, north west/above,
     output/right, input 1/above, input 2/below}
     \draw[shift=(s.\anchor)] plot[mark=x] coordinates{(0,0)}
       node[\placement] {\scriptsize\texttt{(s.\anchor)}};
\end{tikzpicture}
\end{codeexample}

\end{shape}

\begin{shape}{xnor gate US}

	This shape is an xnor gate, which supports only two inputs. If
	less than two inputs are specified an error will result. If more
	than two inputs are specified, the extra inputs are ignored.
	The anchors for this gate with two
  non-inverted inputs (using the normal compass point anchors) are
  shown below. Anchor |30| is an example of a border anchor.
  
\begin{codeexample}[]
\Huge
\begin{tikzpicture}
  \node[name=s,shape=xnor gate US,shape example, inner sep=0cm,
  logic gate inverted radius=.5cm] {Xnor Gate\vrule width1pt height2cm};
  \foreach \anchor/\placement in
    {center/above, text/above, 30/above right,
     mid/right, mid east/left, mid west/above,
     base/below, base east/below, base west/left,
     north/above, south/below, east/above, west/above,
     north east/above, south east/below, south west/below, north west/above,
     output/above, input 1/above, input 2/below}
     \draw[shift=(s.\anchor)] plot[mark=x] coordinates{(0,0)}
       node[\placement] {\scriptsize\texttt{(s.\anchor)}};
\end{tikzpicture}
\end{codeexample}


\end{shape}

\begin{shape}{not gate US}
	This shape is a not gate, which supports only one input. If
	no inputs are specified an error will result. If more
	than one input is specified, the extra inputs are ignored.
	The anchors for this gate with two
  non-inverted inputs (using the normal compass point anchors) are
  shown below. Anchor |30| is an example of a border anchor.
  
\begin{codeexample}[]
\Huge
\begin{tikzpicture}
  \node[name=s,shape=not gate US,shape example, inner sep=1.5cm,
  logic gate inverted radius=.5cm] 
  {Not Gate\vrule width1pt height2cm};
  \foreach \anchor/\placement in
    {center/above, text/above, 30/above right,
     mid/right, mid east/left, mid west/above,
     base/below, base east/below, base west/below,
     north/above, south/below, east/above, west/above,
     north east/above, south east/below, south west/below, north west/above,
     output/above}
     \draw[shift=(s.\anchor)] plot[mark=x] coordinates{(0,0)}
       node[\placement] {\scriptsize\texttt{(s.\anchor)}};
\end{tikzpicture}
\end{codeexample}

\end{shape}

\begin{shape}{buffer gate US}
	This shape is a not gate, which supports only one input. If
	no inputs are specified an error will result. If more
	than one input is specified, the extra inputs are ignored.
	The anchors for this gate with two
  non-inverted inputs (using the normal compass point anchors) are
  shown below. Anchor |30| is an example of a border anchor.
  
\begin{codeexample}[]
\Huge
\begin{tikzpicture}
  \node[name=s,shape=buffer gate US,shape example, inner sep=1.5cm,
  logic gate inverted radius=.5cm] 
  {Buffer Gate\vrule width1pt height2cm};
  \foreach \anchor/\placement in
    {center/above, text/above, 30/above right,
     mid/right, mid east/left, mid west/above,
     base/below, base east/below, base west/below,
     north/above, south/below, east/above, west/above,
     north east/above, south east/below, south west/below, north west/above,
     output/below}
     \draw[shift=(s.\anchor)] plot[mark=x] coordinates{(0,0)}
       node[\placement] {\scriptsize\texttt{(s.\anchor)}};
\end{tikzpicture}
\end{codeexample}

\end{shape}





\begin{shape}{and gate CDH}
  This shape is the alternative and gate. It supports two or more inputs.
  If less than two inputs are specified an error will result. 
	The anchors for this gate with two
  non-inverted inputs (using the normal compass point anchors) are
  shown below. Anchor |30| is an example of a border anchor.
  
\begin{codeexample}[]
\Huge
\begin{tikzpicture}
  \node[name=s,shape=and gate CDH,shape example, inner sep=0cm,
    logic gate inverted radius=.5cm] {And Gate\vrule width1pt height2cm};
  \foreach \anchor/\placement in
    {center/above, text/above, 30/above right,
     mid/right, mid east/left, mid west/above,
     base/below, base east/below, base west/left,
     north/above, south/below, east/above, west/above,
     north east/above, south east/below, south west/below, north west/above,
     output/right, input 1/above, input 2/below}
     \draw[shift=(s.\anchor)] plot[mark=x] coordinates{(0,0)}
       node[\placement] {\scriptsize\texttt{(s.\anchor)}};
\end{tikzpicture}
\end{codeexample}
\end{shape}

\begin{shape}{nand gate CDH}
  This shape is the alternative nand gate. It supports two or more inputs.
  If less than two inputs are specified an error will result. 
	The anchors for this gate with two
  non-inverted inputs (using the normal compass point anchors) are
  shown below. Anchor |30| is an example of a border anchor.
  
\begin{codeexample}[]
\Huge
\begin{tikzpicture}
  \node[name=s,shape=nand gate CDH,shape example, inner xsep=0cm,
    logic gate inverted radius=.5cm] {Nand Gate\vrule width1pt height2cm};
  \foreach \anchor/\placement in
    {center/above, text/above, 30/above right,
     mid/right, mid east/left, mid west/above,
     base/below, base east/below, base west/left,
     north/above, south/below, east/above, west/above,
     north east/above, south east/below, south west/below, north west/above,
     output/right, input 1/above, input 2/below}
     \draw[shift=(s.\anchor)] plot[mark=x] coordinates{(0,0)}
       node[\placement] {\scriptsize\texttt{(s.\anchor)}};
\end{tikzpicture}
\end{codeexample}

\end{shape}





\subsubsection{IEC Logic Gates}

\begin{pgflibrary}{shapes.gates.logic.IEC}
  This library provides rectangular logic gate shapes. These shapes
  are suffixed with |IEC| as they are based on gates recommended by
  the International Electrotechincal Commission.
\end{pgflibrary}

  In order to use these shapes in \tikzname{} without the |IEC|
  suffix, the following key is provided:
  
\begin{key}{/tikz/use IEC style logic gates}
	This allows the the shapes suffixed with |IEC| to be used without
	the suffix. So, for example, |and gate| becomes a synonym for
	|shape=and gate IEC|. In addtion the |IEC| specific keys can be
	used without |IEC|, so |and gate symbol| can be
	used for |and gate IEC symbol|.
\end{key}

  By default each gate is drawn with a symbol, $\char`\&$ for |and| and 
  |nand| gates, $\geq1$ for |or| and |nor| gates, $1$ for |not| and 
  |buffer| gates, and $=1$ for |xor| and |xnor| gates. These symbols 
  are drawn automatically (internally they are drawn using the 
  ``foreground'' path), and are not strictly speaking part of the node
  contents. However, the gate is enlarged to make sure the symbols are 
  within the border of the node.
  It is possible to change
  the symbols and their position within the node using the following
  keys:
  
\begin{key}{/pgf/and gate IEC symbol=\meta{text} (initially \char`\\char\char`\`\char`\\\char`\&)}
  Set the symbol for the |and gate|. Note that if the node is filled,
  this color will be used for the symbol, making it invisible, so
  it will be necessary set \meta{text} to something like
  |\color{black}\char`\&|. Alternatively, the 
  |logic gate IEC symbol color| key can be used to set the color
  of all symbols simultaneously.
  
  In \tikzname, when the |use IEC style logic gates| key has been 
  used, this key can be replaced by |and gate symbol|.
\end{key}

\begin{key}{/pgf/nand gate IEC symbol=\meta{text} (initially \char`\\char\char`\`\char`\\\char`\&)}
  Set the symbol for the |nand gate|.  
  In \tikzname, when the |use IEC style logic gates| key has been 
  used, this key can be replaced by |nand gate symbol|.
\end{key}

\begin{key}{/pgf/or gate IEC symbol=\meta{text} (initially \char`\$\char`\\geq1\char`\$)}
  Set the symbol for the |or gate|.  
  In \tikzname, when the |use IEC style logic gates| key has been 
  used, this key can be replaced by |or gate symbol|.
\end{key}

\begin{key}{/pgf/nor gate IEC symbol=\meta{text} (initially \char`\$\char`\\geq1\char`\$)}
  Set the symbol for the |nor gate|.  
  In \tikzname, when the |use IEC style logic gates| key has been 
  used, this key can be replaced by |nor gate symbol|.
\end{key}

\begin{key}{/pgf/xor gate IEC symbol=\meta{text} (initially \char`\{\char`\$=1\char`\$\char`\})}
  Set the symbol for the |xor gate|. Note the necessity for braces,
  as the symbol contains |=|.
  In \tikzname, when the |use IEC style logic gates| key has been 
  used, this key can be replaced by |or gate symbol|.
\end{key}

\begin{key}{/pgf/xnor gate IEC symbol=\meta{text} (initially  \char`\{\char`\$=1\char`\$\char`\})}
  Set the symbol for the |xnor gate|.  
  In \tikzname, when the |use IEC style logic gates| key has been 
  used, this key can be replaced by |xnor gate symbol|.
\end{key}

\begin{key}{/pgf/not gate IEC symbol=\meta{text} (initially 1)}
  Set the symbol for the |not gate|.  
  In \tikzname, when the |use IEC style logic gates| key has been 
  used, this key can be replaced by |not gate symbol|.
\end{key}

\begin{key}{/pgf/buffer gate IEC symbol=\meta{text} (initially 1)}
  Set the symbol for the |buffer gate|.  
  In \tikzname, when the |use IEC style logic gates| key has been 
  used, this key can be replaced by |buffer gate symbol|.
\end{key}

\begin{key}{/pgf/logic gate IEC symbol align=\meta{align} (initially top)}
  Set the alignment of the logic gate symbol (in \tikzname, when the 
  |use IEC style logic gates| key has been used, |IEC| can be omitted.
  The specification in \meta{align} is a comma separated list from
  |top|, |bottom|, |left| or |right|. The distance between the border
  of the node and the outer edge of the symbol is determined by the values 
  of the |inner xsep| and |inner ysep|.
  
\begin{codeexample}[]
\begin{tikzpicture}[minimum size=1cm, use IEC style logic gates]
	\tikzset{every node/.style={nor gate, draw}}
  \node (A) at (0,1.5) {};
  \node [logic gate symbol align={bottom, right}] (B) at (0,0) {}; 
  \foreach \g in {A, B}{
    \foreach \i in {1,2}
      \draw ([xshift=-0.5cm]\g.input \i) -- (\g.input \i);
    \draw (\g.output) -- ([xshift=0.5cm]\g.output);
  }
\end{tikzpicture}
\end{codeexample} 

\end{key}


\begin{key}{/pgf/logic gate IEC symbol color=\meta{color}}
  This key sets the color for all symbols simultaneously. This color
  can be overridden on a case by case basis by specifying a color
  when seting the symbol text.
\end{key}


\begin{shape}{and gate IEC}
  This shape is an and gate. It supports two or more inputs.
  If less than two inputs are specified an error will result. 
	The anchors for this gate with two
  non-inverted inputs are
  shown below. Anchor |30| is an example of a border anchor.
  
\begin{codeexample}[]
\Huge
\begin{tikzpicture}
  \node[name=s,shape=and gate IEC ,shape example, inner xsep=1cm, inner ysep=1cm,
    minimum height=6cm, and gate IEC symbol=\color{black!30}\char`\&] 
  {And Gate\vrule width1pt height2cm};
  \foreach \anchor/\placement in
    {center/above, text/above, 30/above right,
     mid/right, mid east/left, mid west/above,
     base/below, base east/below, base west/left,
     north/above, south/below, east/above, west/above,
     north east/above, south east/below, south west/below, north west/above,
     output/right, input 1/above, input 2/below}
     \draw[shift=(s.\anchor)] plot[mark=x] coordinates{(0,0)}
       node[\placement] {\scriptsize\texttt{(s.\anchor)}};
\end{tikzpicture}
\end{codeexample}
\end{shape}


\begin{shape}{nand gate IEC}
  This shape is a nand gate. It supports two or more inputs.
  If less than two inputs are specified an error will result. 
	The anchors for this gate with two
  non-inverted inputs are
  shown below. Anchor |30| is an example of a border anchor.
  
\begin{codeexample}[]
\Huge
\begin{tikzpicture}
  \node[name=s,shape=nand gate IEC ,shape example, inner xsep=1cm, inner ysep=1cm,
    minimum height=6cm, nand gate IEC symbol=\color{black!30}\char`\&,
    logic gate inverted radius=0.65cm] 
  {Nand Gate\vrule width1pt height2cm};
  \foreach \anchor/\placement in
    {center/above, text/above, 30/above right,
     mid/right, mid east/left, mid west/above,
     base/below, base east/below, base west/left,
     north/above, south/below, east/above, west/above,
     north east/above, south east/below, south west/below, north west/above,
     output/right, input 1/above, input 2/below}
     \draw[shift=(s.\anchor)] plot[mark=x] coordinates{(0,0)}
       node[\placement] {\scriptsize\texttt{(s.\anchor)}};
\end{tikzpicture}
\end{codeexample}
\end{shape}

\begin{shape}{or gate IEC}
  This shape is an or gate. It supports two or more inputs.
  If less than two inputs are specified an error will result. 
	See the |and gate IEC| shape for the anchors.
	
\begin{codeexample}[]
\begin{tikzpicture}[minimum width=.875cm, minimum height=1cm]
  \node[or gate IEC, draw, logic gate inputs=in] (A) {};
  \draw (A.input 1 -| -1,0) -- (A.input 1) (A.input 2 -| -1,0) -- (A.input 2)
        (A.output) -- ([xshift=0.5cm]A.output);
\end{tikzpicture}
\end{codeexample} 

\end{shape}


\begin{shape}{nor gate IEC}
  This shape is an nor gate. It supports two or more inputs.
  If less than two inputs are specified an error will result. 
	See the |nand gate IEC| shape for the anchors.
	
\begin{codeexample}[]
\begin{tikzpicture}[minimum width=.875cm, minimum height=1cm]
  \node[nor gate IEC, draw, logic gate inputs=in] (A) {};
  \draw (A.input 1 -| -1,0) -- (A.input 1) (A.input 2 -| -1,0) -- (A.input 2)
        (A.output) -- ([xshift=0.5cm]A.output);
\end{tikzpicture}
\end{codeexample}

\end{shape}

\begin{shape}{xor gate IEC}
  This shape is an xor gate. It supports only two inputs.
   If less than two inputs are specified an error will result.
  Any extra inputs are ignored.  
		See the |and gate IEC| shape for the anchors.
	
\begin{codeexample}[]
\begin{tikzpicture}[minimum width=.875cm, minimum height=1cm]
  \node[xor gate IEC, draw, logic gate inputs=in] (A) {};
  \draw (A.input 1 -| -1,0) -- (A.input 1) (A.input 2 -| -1,0) -- (A.input 2)
        (A.output) -- ([xshift=0.5cm]A.output);
\end{tikzpicture}
\end{codeexample}

\end{shape}


\begin{shape}{xnor gate IEC}
  This shape is an xnor gate. It supports only two inputs.
  If less than two inputs are specified an error will result.
  Any extra inputs are ignored.  
	See the |nand gate IEC| shape for the anchors.
	
\begin{codeexample}[]
\begin{tikzpicture}[minimum width=.875cm, minimum height=1cm]
  \node[xnor gate IEC, draw, logic gate inputs=in] (A) {};
  \draw (A.input 1 -| -1,0) -- (A.input 1) (A.input 2 -| -1,0) -- (A.input 2)
        (A.output) -- ([xshift=0.5cm]A.output);
\end{tikzpicture}
\end{codeexample}

\end{shape}

\begin{shape}{buffer gate IEC}
   This shape is a buffer gate. It supports only one input.
  If less than one input is specified an error will result.
  Any extra inputs are ignored. 
	See the |and gate IEC| shape for the anchors.

\begin{codeexample}[]
\begin{tikzpicture}[minimum width=.875cm, minimum height=1cm]
  \node[buffer gate IEC, draw] (A) {};
  \draw (A.input -| -1,0) -- (A.input) (A.output) -- ([xshift=0.5cm]A.output);
\end{tikzpicture}
\end{codeexample}

\end{shape}


\begin{shape}{not gate IEC}
  This shape is a not gate. It supports only one input.
  If less than one input is specified an error will result.
  Any extra inputs are ignored. 
  See the |nand gate IEC| shape for the anchors.

\begin{codeexample}[]
\begin{tikzpicture}[minimum width=.875cm, minimum height=1cm]
  \node[not gate IEC, draw] (A) {};
  \draw (A.input -| -1,0) -- (A.input) (A.output) -- ([xshift=0.5cm]A.output);
\end{tikzpicture}
\end{codeexample}


\end{shape}






\subsection{Miscellaneous Shapes}

\begin{pgflibrary}{shapes.misc}
  This library defines general-purpose shapes that do not fit in the
  previous categories.
\end{pgflibrary}

\begin{shape}{cross out}
  This shape ``crosses out'' the node. Its foreground path are simply
  two diagonal lines that between the corners of the node's bounding
  box. Here is an example:  
\begin{codeexample}[]
\begin{tikzpicture} 
  \draw [help lines] (0,0) grid (3,2); 
  \node [cross out,draw=red] at (1.5,1) {cross out}; 
\end{tikzpicture}
\end{codeexample}

  A useful application is inside text as in the following example:
\begin{codeexample}[]
Cross \tikz[baseline] \node [cross out,draw,anchor=text] {me}; out!  
\end{codeexample}

  This shape inherits all anchors from the |rectangle| shape, see also
  the following figure:
\begin{codeexample}[]
\Huge
\begin{tikzpicture}
  \node[name=s,shape=cross out,shape example] {cross out\vrule width 1pt height 2cm};
  \foreach \anchor/\placement in
    {north west/above left, north/above, north east/above right, 
     west/left, center/above, east/right, 
     mid west/right, mid/above, mid east/left, 
     base west/left, base/below, base east/right, 
     south west/below left, south/below, south east/below right, 
     text/left, 10/right, 130/above}
     \draw[shift=(s.\anchor)] plot[mark=x] coordinates{(0,0)}
       node[\placement] {\scriptsize\texttt{(s.\anchor)}};
\end{tikzpicture}
\end{codeexample}
\end{shape}


\begin{shape}{cross out}
  This shape ``crosses out'' the node. Its foreground path are simply
  two diagonal lines that between the corners of the node's bounding
  box. Here is an example:

\begin{codeexample}[]
\begin{tikzpicture}
  \draw[help lines] (0,0) grid (3,2);
  \node [cross out,draw=red] at (1.5,1) {cross out};
\end{tikzpicture}
\end{codeexample}

  A useful application is inside text as in the following example:
\begin{codeexample}[]
Cross \tikz[baseline] \node [cross out,draw,anchor=text] {me}; out!  
\end{codeexample}

  This shape inherits all anchors from the |rectangle| shape, see also
  the following figure:
\begin{codeexample}[]
\Huge
\begin{tikzpicture}
  \node[name=s,shape=cross out,shape example] {cross out\vrule width 1pt height 2cm};
  \foreach \anchor/\placement in
    {north west/above left, north/above, north east/above right, 
     west/left, center/above, east/right, 
     mid west/right, mid/above, mid east/left, 
     base west/left, base/below, base east/right, 
     south west/below left, south/below, south east/below right, 
     text/left, 10/right, 130/above}
     \draw[shift=(s.\anchor)] plot[mark=x] coordinates{(0,0)}
       node[\placement] {\scriptsize\texttt{(s.\anchor)}};
\end{tikzpicture}
\end{codeexample}
\end{shape}

\begin{shape}{strike out}
  This shape is idential to the |cross out| shape, only its foreground
  path consists of a single line from the lower left to the upper
  right.
  
\begin{codeexample}[]
Strike \tikz[baseline] \node [strike out,draw,anchor=text] {me}; out!  
\end{codeexample}

  See the |cross out| shape for the anchors.
\end{shape}




\begin{shape}{rounded rectangle}
	This shape is a rectangle which can be optionally round sides.

\begin{codeexample}[]
\begin{tikzpicture}
  \node[rounded rectangle, draw, fill=red!20]{Hallo};
\end{tikzpicture}
\end{codeexample}

	There are keys to specify how the sides are rounded (to use
	these keys in \tikzname, simply remove the \declare{|/pgf/|} path).


\begin{key}{/pgf/rounded rectangle arc length=\meta{angle} (initially 180)}

	Set the length of the arcs for the rounded ends. Recommended values 
	for	\meta{angle} are between |90| and |180|. 
	
\begin{codeexample}[]
\begin{tikzpicture}
  \matrix[row sep=5pt, every node/.style={draw, rounded rectangle}]{
    \node[rounded rectangle arc length=180] {180}; \\
    \node[rounded rectangle arc length=120] {120}; \\
    \node[rounded rectangle arc length=90]  {90};  \\};
\end{tikzpicture} 
\end{codeexample}

\end{key}

\begin{key}{/pgf/rounded rectangle west arc=\meta{arc type} (initially convex)}
	Set the style of the rounding for the left side. The permitted values
	for \meta{arc type} are |concave|, |convex|, or |none|.

\begin{codeexample}[]
\begin{tikzpicture}
  \matrix[row sep=5pt, every node/.style={draw, rounded rectangle}]{
  	\node[rounded rectangle west arc=concave] {Concave}; \\
  	\node[rounded rectangle west arc=convex]  {Convex};  \\
  	\node[rounded rectangle left arc=none]    {None};    \\};
\end{tikzpicture} 
\end{codeexample}
\end{key}

\begin{stylekey}{/pgf/rounded rectangle left arc=\meta{arc type}}
	Alternative key for specifying the west arc.
\end{stylekey}

\begin{key}{/pgf/rounded rectangle east arc=\meta{arc type} (initially convex)}
	Set the style of the rounding for the east side.
\end{key}

\begin{stylekey}{/pgf/rounded rectangle right arc=\meta{arc type}}
	Alternative key for specifying the east arc.
\end{stylekey}

	The anchors for this shape are shown below (anchor |10| is an example
	of a border angle). Note that if only one side is rounded, the 
	|center| anchor will not be the precise center of the shape.
	
\begin{codeexample}[]
\Huge
\begin{tikzpicture}
  \node[name=s,shape=rounded rectangle, shape example, inner xsep=1.5cm, inner ysep=1cm] 
  	{Rounded Rectangle\vrule width 1pt height 2cm};
  \foreach \anchor/\placement in
    {center/above, text/below,      10/above,
     mid/above,    mid west/right,  mid east/left, 
     base/below,   base west/below, base east/below, 
     north/above,  south/below, east/above, west/above, 
     north west/above left,    north east/above right,     
     south west/below left,    south east/below right}
     \draw[shift=(s.\anchor)] plot[mark=x] coordinates{(0,0)}
       node[\placement] {\scriptsize\texttt{(s.\anchor)}};
\end{tikzpicture}
\end{codeexample}

\end{shape}


\begin{shape}{chamfered rectangle}

	This shape is a rectangle with optionally chamfered corners.
	
\begin{codeexample}[]
\begin{tikzpicture}
  \node[chamfered rectangle, white, fill=red, double=red, draw, very thick]
    {\bf STOP!};
\end{tikzpicture}
\end{codeexample}

	There are \pgfname{} keys to specify how this shape is drawn (to use
	these keys in \tikzname{} simply remove the \declare{|/pgf/|} path).

\begin{key}{/pgf/chamfered rectangle angle=\meta{angle} (initially 45)}
	Set the angle \emph{from the vertical} for the chamfer.
	
\begin{codeexample}[]
\begin{tikzpicture}
  \tikzset{every node/.style={chamfered rectangle, draw}}
  \node[chamfered rectangle angle=30] {abc};
  \node[chamfered rectangle angle=60] at (1.5,0) {123};
\end{tikzpicture}
\end{codeexample}
\end{key}

\begin{key}{/pgf/chamfered rectangle xsep=\meta{length} (initially .666ex)}
	Set the distance that the chamfer extends horizontally beyond the node 
	contents (which includes the |inner sep|). 
	If \meta{length} is large, such
	that the top and bottom chamfered edges would cross, then 
	\meta{length} is ignored and the chamfered edges are drawn so that
	they meet in the middle.

\begin{codeexample}[]
\begin{tikzpicture}
  \tikzset{every node/.style={chamfered rectangle, draw}}
  \node[chamfered rectangle xsep=2pt] {def};
  \node[chamfered rectangle xsep=2cm] at (1.5,0) {456};
\end{tikzpicture}
\end{codeexample}
	
\end{key}

\begin{key}{/pgf/chamfered rectangle ysep=\meta{length} (initially .666ex)}
	Set the distance that the chamfer extends vertically beyond the node 
	contents. 
	If \meta{length} is large, such that the left and right chamfered 
	edges would cross, then \meta{length} is ignored and the chamfered 
	edges are drawn so that	they meet in the middle.
\end{key}

\begin{key}{/pgf/chamfered rectangle sep=\meta{length} (initially .666ex)}
	Set both the |xsep| and |ysep| simultaneously.
\end{key}

\begin{key}{/pgf/chamfered rectangle corners=\meta{list} (initially chamfer all)}
	Specify which corners are chamfered. The corners are identified by 
	their ``compass point'' directions (i.e. |north east|, |north west|,
	|south west|, and |south east|), and must be separated by commas (so
	if there is more than one corner in the list, it must be surrounded
	by braces). Any corners not mentioned in 
	\meta{list} are automatically not chamfered. Two additional values
	|chamfer all| and |chamfer none|, are also permitted.

\begin{codeexample}[]
\begin{tikzpicture}
  \tikzset{every node/.style={chamfered rectangle, draw}}
  \node[chamfered rectangle corners=north west] {ghi};
  \node[chamfered rectangle corners={north east, south east}] at (1.5,0) {789};
\end{tikzpicture}
\end{codeexample}
\end{key}


	The anchors for this shape are shown below (anchor |60| is an example
	of a border angle.
	
\begin{codeexample}[]
\Huge
\begin{tikzpicture}
  \node[name=s,shape=chamfered rectangle, chamfered rectangle sep=1cm,
        shape example, inner ysep=1cm, inner xsep=.75cm] 
    {Chamfered Rectangle\vrule width1pt height2cm};
  \foreach \anchor/\placement in
    {text/right, center/above,    70/above, 
     base/below, base east/left, base west/right,
     mid/right,  mid east/above,   mid west/above,  
     north/above, south/below, east/above, west/above,
     before north east/above, north east/above, after north east/above,
     before north west/above, north west/above, after north west/above,
     before south west/below, south west/below, after south west/below,
     before south east/below, south east/below, after south east/below}     
     \draw[shift=(s.\anchor)] plot[mark=x] coordinates{(0,0)}
       node[\placement] {\scriptsize\texttt{(s.\anchor)}};
\end{tikzpicture}
\end{codeexample}

\end{shape}



%%% Local Variables: 
%%% mode: latex
%%% TeX-master: "pgfmanual-pdftex-version"
%%% End: 

% Copyright 2006 by Till Tantau
%
% This file may be distributed and/or modified
%
% 1. under the LaTeX Project Public License and/or
% 2. under the GNU Free Documentation License.
%
% See the file doc/generic/pgf/licenses/LICENSE for more details.


\section{Spy Library: Magnifying Parts of Picutres}
\label{section-library-spy}

\begin{tikzlibrary}{spy}
  The package defines options for creating pictures in which some part
  of the picture is repeated in another area in a magnified way (as if
  you were looking through a spyglass, hence the name).
\end{tikzlibrary}


\subsection{Magnifying a Part of a Picture}

The idea behind the |spy| library is to make is easy to create high-density
pictures in which some important parts are repeated somewhere, but
magnified as if you were looking through a spyglass:

\begin{codeexample}[]
\begin{tikzpicture}
  [spy using outlines={circle, magnification=4, size=2cm, connect spies}]
  
  \draw [help lines] (0,0) grid (3,2);
  
  \draw [decoration=Koch curve type 1]
    decorate { decorate{ decorate{ decorate{ (0,0) -- (2,0) }}}};
  
  \spy [red] on (1.6,0.3)
             in node [left] at (3.5,-1.25);    
             
  \spy [blue, size=1cm] on (1,1)
              in node [right] at (0,-1.25);    
\end{tikzpicture}
\end{codeexample}

\begin{codeexample}[]
\begin{tikzpicture}[spy using overlays={size=12mm}]
  \draw [decoration=Koch snowflake]
    decorate { decorate{ decorate{ decorate{ (0,0) -- (2,0) }}}};
  
  \spy [green,magnification=3] on (0.6,0.1) in node at (-0.3,-1);    
  \spy [blue,magnification=5]  on (1,0.5)   in node at (1,-1);    
  \spy [red,magnification=10]  on (1.6,0.1) in node at (2.3,-1);    
\end{tikzpicture}
\end{codeexample}


Note that this magnification uses what is called a \emph{canvas
  transformation} in this manual: Everything is magnified, including
line width and text.

In order for ``spying'' to work, the picture obviously has to be drawn
several times: Once at its normal size and then again for each
``magnifying glass.'' Several keys and commands work in concert to
make this possible:
\begin{itemize}
\item You need to make \tikzname\ aware of the fact that a picture (or
  just a scope) is to be magnified. This is done by adding the special
  key |spy scope| to a |{scope}| or |{tikzpicture}| (which is also
  just a scope). Some special keys like |spy using outlines|
  implicitly set the |spy scope|.

\item Inside this scope you may then use the command |\spy|, which is
  only available inside such scopes (so there is no danger of your
  inadvertently using this command outside such a scope). This command
  has a special syntax and will (at some point) create two nodes: One
  node that shows the magnified picture (called the \emph{spy-in
    node}) and another node showing which part of the original picture
  is magnified (called the \emph{spy-on} node). The spy-in node is,
  indeed, a normal node, so it can have any shape or border that you
  like and you can apply all of \tikzname's advanced features to
  it. The only difference compared to a normal node is that instead of
  some ``text'' it contains a magnified version of the picture,
  clipped to the size of the node. 

  The |\spy| command does not create the nodes immediately. Rather,
  the creation of these nodes is postponed till the end of the
  |spy scope| in which the |\spy| command is used. This is necessary
  since in order to repeat the whole scope inside the node containing
  the magnified version, the whole picture needs to be available when
  this node is created.
\end{itemize}

A basic question any library for ``magnifying things'' has to address
is how you specify which part of the picture is to be
magnified (the spy-on node) and where this magnified part is to be
shown (the spy-in node). There are two possible ways:
\begin{enumerate}
\item You specify the size and position of the spy-on node. Then the
  size of the spy-in node is determined by the size of the spy-on node
  and the magnification factor -- you can still decide where the
  spy-in node should be placed, but not its size.
\item Alternatively, you specify the size and position of the spy-in
  node. Then, similarly to the first case, the size of the spy-on node
  is determined implicitly and you can only decide where the
  spy-on node should be placed, but not its size.
\end{enumerate}

The |spy| library uses the second method: You specify the size and
position of the spy-in nodes, the sizes of the spy-on nodes are then
computed automatically.



\subsection{Spy Scopes}

\begin{key}{/tikz/spy scope=\meta{options} (default \normalfont empty)}
  This option may be used with a |{scope}| or any environment that
  creates such a scope internally (like |{tikzpicture}|). It has the
  following effects:
  \begin{itemize}
  \item It resets a number of graphic state parameters, including the
    color, line style, and other. This is necessary for technical
    reasons. 
  \item It tells \tikzname\ that the content of the scope should be saved
    internally in a special box.
  \item It defines the command |\spy| so that it can be used inside
    the scope.
  \item At the end of the scope, the nodes belonging to the |\spy|
    commands used inside the scope are created.
  \item The \meta{options} are saved in an internal style. Each time
    |\spy| is used, these \meta{options} will be used.
  \item Three keys are defined that provide useful shortcuts:
    \begin{key}{/tikz/size=\meta{dimension}}
      Inside a |spy scope|, |size| this is a shortcut for |minimum size|.      
    \end{key}
    \begin{key}{/tikz/height=\meta{dimension}}
      Inside a |spy scope|, |height| this is a shortcut for |minimum height|.      
    \end{key}
    \begin{key}{/tikz/width=\meta{dimension}}
      Inside a |spy scope|, |width| this is a shortcut for |minimum width|.      
    \end{key}
  \end{itemize}
  It is permissible to nest |spy scopes|. In this case, all |\spy|
  commands inside the inner |spy scope| only have an effect on
  material inside the scope, whereas |\spy| commands outside the inner
  |spy scope| but inside the outer |spy scope| allow you to ``spy on
  the spy.''
  
\begin{codeexample}[]
\begin{tikzpicture}
  [spy using outlines={rectangle, red, magnification=5,
                       size=1.5cm, connect spies}]

  \begin{scope}
    [spy using outlines={circle, blue,
                         magnification=3, size=1.5cm, connect spies}]
    \draw [help lines] (0,0) grid (3,2);
  
    \draw [decoration=Koch curve type 1]
      decorate{ decorate{ decorate{ (0,0) -- (2,0) }}};
  
    \spy on (1.6,0.3) in node (zoom) [left] at (3.5,-1.25);    
  \end{scope}       

  \spy on (zoom.north west) in node [right] at (0,-1.25);    
\end{tikzpicture}
\end{codeexample}
  
\end{key}



\subsection{The Spy Command}

\begin{command}{\spy \opt{\oarg{options}} |on| \meta{coordinate}
    |in node| \meta{node options}|;|}    
  This command can only be used inside a |spy scope|. Let us start with the syntax:
  \begin{itemize}
  \item The |\spy| command is not a special case of |\path|. Rather,
    it has a small parser of its own.
  \item Following the optional \meta{options}, you must write |on|,
    followed by a coordinate. This coordinate will be the center of
    the area that is to be magnified.
  \item Following the \meta{coordinate}, you must write |in node|
    followed by some \meta{node options}. The syntax for these options is the same
    as for a normal |node| path command, such as |[left]| or
    |(foo) [red] at (bar)|. \emph{However},  \meta{node options} are
    \emph{not} followed by a curly brace. Rather, the \meta{node
      options} must directly be followed by a semicolon.
  \end{itemize}
  The effect of this command is the following: The \meta{options},
  \meta{coordinate}, and \meta{node options} are stored internally
  till the end of the current 
  |spy scope|. This means that, in particular, you can reference any node
  inside the |spy scope|, even if it is not yet defined when the
  |\spy| command is given. At the end of the current |spy scope|, two
  nodes are created, called the \emph{spy-in node} and the
  \emph{spy-on node}. 
  \begin{itemize}
  \item The \emph{spy-in node} is the node that contains a magnified
    part of the picture (the node \emph{in} which we see on what we
    spy). This node is, indeed, a normal \tikzname\ 
    node, so you can use all standard options to style this node. In
    particular, you can specify a shape or a border color or a drop
    shadow or whatever. The only thing that is special about this node
    is that instead of containing some normal text, its ``text'' is 
    the magnified picture.

    To be precise, the picture of the |spy scope| is scaled by a 
    certain factor, specified by the |lens| or |magnification| options
    discussed below, and the shifted in such a way that the
    \meta{coordinate} lies at the center of the spy-on node.
  \item The \emph{spy-on node} is a node that is centered on the
    \meta{coordinate} and whose size reflects exactly the area shown
    inside the spy-in node (the node containing \emph{on} what we
    spy).  
  \end{itemize}
  
  Let us now go over what happens in detail when the two nodes are
  created: 
  \begin{enumerate}
  \item A scope is started. Two sets of options are used with this
    scope: First, the options passed to the enclosing |spy scope| and
    then the \meta{options} (which will, thus, overrule the options of
    the |spy scope|).
  \item Then, the spy-on node is created. However, we will first
    discuss the spy-in node.
  \item The spy-in node is created after the spy-on node (and, hence,
    will cover the spy-on node in case they overlap). When this node is
    created, the \meta{node options} are used in addition to the 
    effect caused by the \meta{options} and the options of the
    |{spy scope}|. Additionally, the following style is used:
    \begin{stylekey}{/tikz/every spy in node}
      This style is used with every spy in node.  
    \end{stylekey}
    The position of the node (the |at| option) is set to the
    \meta{coordinate} by default, so that it will cover the
    to-be-magnified area. You can change this by providing the |at|
    option yourself:
\begin{codeexample}[]
\begin{tikzpicture}
  [spy using outlines={circle, magnification=3, size=1cm}]
  
  \draw [decoration=Koch curve type 1]
    decorate{ decorate{ decorate{ (0,0) -- (2,0) }}};
  
  \spy [red]  on (1.6,0.3) in node;             
  \spy [blue] on (1,1)     in node at (1,-1);    
\end{tikzpicture}
\end{codeexample}    
    No ``text'' can be specified for the node. Rather, the ``text''
    shown inside this node is the picture of the current |spy scope|,
    but canvas-transformed according to the following key:
    \begin{key}{/tikz/lens=\meta{options}}
      The \meta{options} should contain transformation commands like
      |scale| or |rotate|. These transformations are applied to the
      picture when it is shown inside the spy-on node.
    \end{key}
    Since the most common transformation is undoubtedly a simple
    scaling, there is a special style for this:
    \begin{key}{/tikz/magnification=\meta{number}}
      This has the same effect as saying
      |lens={scale=|\meta{number}|}|. 
    \end{key}
    Now, usually the size of a node is determined in such a way that
    it ``fits'' around the text of the node. For a spy-on node this is
    not a good approach since the ``text'' of this node would contain
    ``the whole picture.'' Because of this, \tikzname\ acts
    as if the ``text'' of the node has zero size. You must then use
    keys like |minimum size| to cause the node to have a certain
    size. Note that the key |size| is an abbreviation for
    |minimum size| inside a spy scope.

    You can name the spy on node in the usual ways. Additionally, the
    node is (also) always named |tikzspyinnode|. Following the spy
    scope, you can use this node like any other node:    
\begin{codeexample}[]
\begin{tikzpicture}
  \begin{scope}
    [spy using outlines={circle, magnification=3, size=2cm, connect spies}]
  
    \draw [decoration=Koch curve type 1]
      decorate{ decorate{ decorate{ (0,0) -- (2,0) }}};
    
    \spy [red] on (1.6,0.3) in node (a) [left] at (3.5,-1.25);    
             
    \spy [blue, size=1cm] on (1,1) in node (b) [right] at (0,-1.25);
  \end{scope}
  \draw [ultra thick, green!50!black] (b) -- (a.north west);
\end{tikzpicture}
\end{codeexample}

  \item Once both nodes have been created, the current value of the
    following key is used to connect them:
    \begin{key}{/tikz/spy connection path=\meta{code} (initially
        \normalfont empty)}
      The \meta{code} is executed after the spy-on and spy-in nodes
      have just been created. Inside this \meta{code}, the two nodes
      can be accessed as |tikzspyinnode| and  |tikzspyonnode|.
      For example, the key |connect spies| sets this command to
\begin{codeexample}[code only]
\draw[thin] (tikzspyonnode) -- (tikzspyinnode);        
\end{codeexample}
    \end{key}
  \end{enumerate}
  Returning to the creation of the spy-in node: This node is centered on
  \meta{coordinate} (more precisely, its anchor is set to |center| and
  the |at| option is set to \meta{coordinate}). Its size and shape are
  initially determined in the same way as the size and shape of the
  spy-on node (unless, of course, you explicitly provide a different
  shape for, say, the spy-on node locally, which is not really a good
  idea). Then, additionally, the \emph{inverted} transformation done
  by the |lens| option is applied, resulting in a node whose size and
  shape exactly corresponds to the area in the picture that is shown
  in the spy-on node.
\begin{codeexample}[]
\begin{tikzpicture}
  [spy using outlines={lens={scale=3,rotate=20}, size=2cm, connect spies}]
  
  \draw [decoration=Koch curve type 1]
    decorate{ decorate{ decorate{ (0,0) -- (2,0) }}};
    
  \spy [red] on (1.6,0.3) in node at (2.5,-1.25);    
\end{tikzpicture}
\end{codeexample}

  Like for the spy-in node, a style can be used to format the spy-on
  node:
  \begin{stylekey}{/tikz/every spy on node}
    This style is used with every spy on node.
  \end{stylekey}
  The spy-on node is named |tikzspyonnode| (but, as always, this node
  is only available after the spy scope). If you have multiple
  spy-on nodes and you would like to access all of them, you need to
  use the |name| key inside the |every spy on node| style.
  
  The |inner sep| and |outer sep| of both spy-in and spy-on nodes are
  set to |0pt|.   
\end{command}



\subsection{Predefined Spy Styles}

There are some predefined styles that make using the |spy| library
easier. The following two styles can be used instead of |spy scope|,
they pass their \meta{options} directly to |spy scope|. They
additionally setup the graphic styles to be used for the spy-in nodes
and the spy-on nodes in some special way. 

\begin{key}{/tikz/spy using outlines=\meta{options} (default
    \normalfont empty)}
  This key creates a |spy scope| in which the spy-in node is drawn,
  but not filled, using a thick line; and the spy-on node is drawn,
  but not filled, using a very thin line.

\begin{codeexample}[]
\begin{tikzpicture}
  [spy using outlines={circle, magnification=3, size=1cm, connect spies}]
  
  \draw [decoration=Koch curve type 1]
    decorate{ decorate{ decorate{ (0,0) -- (2,0) }}};
  
  \spy [red] on (1.6,0.3) in node at (3,1);    
\end{tikzpicture}
\end{codeexample}
\end{key}

\begin{key}{/tikz/spy using overlays=\meta{options} (default
    \normalfont empty)}
  This key creates a |spy scope| in which both the spy-in and spy-on
  nodes are filled, but with the fill opacity set to 20\%.

\begin{codeexample}[]
\begin{tikzpicture}
  [spy using overlays={circle, magnification=3, size=1cm, connect spies}]
  
  \draw [decoration=Koch curve type 1]
    decorate{ decorate{ decorate{ (0,0) -- (2,0) }}};
  
  \spy [green] on (1.6,0.3) in node at (3,1);    
\end{tikzpicture}
\end{codeexample}
\end{key}

The following style is useful for connecting the spy-in and the spy-on
nodes:

\begin{key}{/tikz/connect spies}
  Causes the spy-in and the spy-on nodes to be connected by a thin
  line. 
  
\begin{codeexample}[]
\begin{tikzpicture}
  [spy using overlays={circle, magnification=3, size=1cm}]
  
  \draw [decoration=Koch curve type 2]
    decorate{ decorate{ decorate{ (0,0) -- (2,0) }}};
  
  \spy [green] on (1.6,0.1) in node at (3,1);    
  \spy [red,connect spies] on (0.5,0.4) in node at (1,1.5);    
\end{tikzpicture}
\end{codeexample}
\end{key}


\subsection{Examples}

Usually, the spy-in node and the spy-on node should have the same
shape. However, you might also wish to use the |circle| shape for the
spy-on node and the |magnifying glass| shape for the spy-in node:

\begin{codeexample}[]
\tikzset{spy using mag glass/.style={
    spy scope={
      every spy on node/.style={
        circle,
        fill, fill opacity=0.2, text opacity=1},
      every spy in node/.style={
        magnifying glass, circular drop shadow,
        fill=white, draw, ultra thick, cap=round},
      #1
    }}}
\begin{tikzpicture}[spy using mag glass={magnification=3, size=1cm}]
  \draw [decoration=Koch curve type 2]
    decorate{ decorate{ decorate{ (0,0) -- (2,0) }}};
  
  \spy [green!50!black] on (1.6,0.1) in node at (2.5,-0.5);    
\end{tikzpicture}
\end{codeexample}

With the magnifying glass, you can also put it ``on top'' of the
picture itself:

\begin{codeexample}[]
\begin{tikzpicture}
  [spy scope={magnification=4, size=1cm},
   every spy in node/.style={
     magnifying glass, circular drop shadow,
     fill=white, draw, ultra thick, cap=round}]
  
  \draw [decoration=Koch curve type 2]
    decorate{ decorate{ decorate{ (0,0) -- (2,0) }}};
  
  \spy on (1.6,0.1) in node;    
\end{tikzpicture}
\end{codeexample}

%%% Local Variables: 
%%% mode: latex
%%% TeX-master: "pgfmanual-pdftex-version"
%%% End: 

% Copyright 2009 by Till Tantau
%
% This file may be distributed and/or modified
%
% 1. under the LaTeX Project Public License and/or
% 2. under the GNU Free Documentation License.
%
% See the file doc/generic/pgf/licenses/LICENSE for more details.


\section{SVG-Path Library}
\label{section-library-svg-path}

\begin{pgflibrary}{svg.path}
  This library defines a command that allows you to specify a path
  using the svg-syntax.
\end{pgflibrary}

\begin{command}{\pgfpathsvg\marg{path}}
  This command extends the current path by a \meta{path} given in the
  \textsc{svg}-path-data syntax. This syntax is described in detail in
  Section~8.3 of the \textsc{svg}-specification, Version 1.1. 

  In principle, the complete syntax is supported and the library just
  provides a parser and a mapping to basic layer commands. For
  instance, |M 0 10| is mapped to
  |\pgfpathmoveto{\pgfpoint{0pt}{10pt}}|. There, however, a few things
  to be aware of:
  \begin{itemize}
  \item The computation underlying the arc commands |A| and |a|
    are not numerically stable, which may result in quite imprecise
    arcs. B\´ezier curves, both quadratic and cubic, are not affected,
    and also not arcs spanning degrees that are multiples of
    $90^\circ$.
  \item The dimensionless units of \textsc{svg} are always interpreted
    at points (|pt|). This is a problem with paths like |M 20000 0|,
    which will raise an error message since \TeX\ cannot handle
    dimensions larger than about 16000 points.
  \item
    All coordinate and canvas transformations apply to the path in the
    usual fashion.
  \item
    The |\pgfpathsvg| command can be freely intermixed with other path
    commands. 
  \end{itemize}
\begin{codeexample}[]
\begin{pgfpicture}
  \pgfpathsvg{M 0 0 l 20 0 0 20 -20 0 q 10 0 10 10
              t 10 10 10 10 h -50 z}
  \pgfusepath{stroke}
\end{pgfpicture}
\end{codeexample}
\end{command}

%%% Local Variables: 
%%% mode: latex
%%% TeX-master: "pgfmanual-pdftex-version"
%%% End: 

% Copyright 2006 by Till Tantau
%
% This file may be distributed and/or modified
%
% 1. under the LaTeX Project Public License and/or
% 2. under the GNU Free Documentation License.
%
% See the file doc/generic/pgf/licenses/LICENSE for more details.

\section{To Path Library}

\label{library-to-paths}

\begin{tikzlibrary}{topaths}
  This library provides predefined to paths for use with the |to|
  path operation. After loading this package, you can say for instance
  |to [loop]| to add a loop to a node.

  This library is loaded automatically by \tikzname, so you do not
  need to load it yourself.
\end{tikzlibrary}


\subsection{Straight Lines}

The following style installs a to path that is simply a straight line
from the start coordinate to the target coordinate.

\begin{key}{/tikz/line to}
  Causes a straight line to be added to the path upon a |to| or an
  |edge| operation.
\begin{codeexample}[]
\tikz {\draw (0,0) to[line to] (1,0);}
\end{codeexample}
\end{key}


\subsection{Curves}

The |curve to| style causes the to path to be set to a curve. The
exact way this curve looks can be influenced via a number of options.

\begin{key}{/tikz/curve to}
  Specifies that the |to path| should be a curve. This curve will
  leave the start coordinate at a certain angle, which can be
  specified using the |out| option. It reaches the target coordinate
  also at a certain angle, which is specified using the |in|
  option. The control points of the curve are at a certain distance
  that is computed in different ways, depending on which options are
  set.

  All of the following options implictly cause the |curve to| style to
  be installed.

  \begin{key}{/tikz/out=\meta{angle}}
    The angle at which the curve leaves the start coordinate. If the
    start coordinate is a node, the start coordinate is the point on the
    border of the node at the given \meta{angle}. The control point
    will, thus, lie at a certain distance in the direction \meta{angle}
    from the start coordinate.
\begin{codeexample}[]
\begin{tikzpicture}[out=45,in=135]
  \draw (0,0) to (1,0)
        (0,0) to (2,0)
        (0,0) to (3,0);
\end{tikzpicture}
\end{codeexample}
  \end{key}
  \begin{key}{/tikz/in=\meta{angle}}
    The angle at which the curve reaches the target coordinate.
  \end{key}

  \begin{key}{/tikz/relative=\meta{true or false} (default true)}
    This option tells \tikzname\ whether the |in| and |out| angles
    should be considered absolute or relative. Absolute means that an
    |out| angle of 30$^\circ$ means that the curve leaves the start
    coordinate at an angle of 30$^\circ$ relative to the paper (unless,
    of course, further transformations have been installed). A
    \emph{relative} angle is, by comparison, measured relative to a
    straight line from the start coordinate to the target
    coordinate. Thus, a relative angle of 30$^\circ$ means that the
    curve will bend to the left from the line going straight from the
    start to the target. For the target, the relative coordinate is
    measured in the same manner, namely relative to the line going from
    the start to the target. Thus, an angle of 150$^\circ$ means that
    the curve will reach target coming slightly from the left.

\begin{codeexample}[]
\begin{tikzpicture}[out=45,in=135,relative]
  \draw (0,0) to (1,0)
              to (2,1)
              to (2,2);
\end{tikzpicture}
\end{codeexample}

\begin{codeexample}[]
\begin{tikzpicture}[out=90,in=90,relative]
  \node [circle,draw] (a) at (0,0) {a};
  \node [circle,draw] (b) at (1,1) {b};
  \node [circle,draw] (c) at (2,2) {c};

  \path (a) edge (b)
            edge (c);
\end{tikzpicture}
\end{codeexample}
  \end{key}

  \begin{key}{/tikz/bend left=\meta{angle} (default \normalfont last value)}
    This option sets |out=|\meta{angle}|,in=|$180-\meta{angle}$|,relative|. If no
    \meta{angle} is given, the last given |bend left| or |bend right|
    angle is used.  
  
\begin{codeexample}[]
\begin{tikzpicture}[shorten >=1pt,node distance=2cm,on grid]
  \node[state,initial]  (q_0)                {$q_0$};
  \node[state]          (q_1) [right=of q_0] {$q_1$};
  \node[state,accepting](q_2) [right=of q_1] {$q_2$};

  \path[->] (q_0) edge              node [above]  {0} (q_1)
                  edge [loop above] node          {1} ()
                  edge [bend left]  node [above]  {1} (q_2)
                  edge [bend right] node [below]  {0} (q_2)
            (q_1) edge              node [above]  {1} (q_2);
\end{tikzpicture}
\end{codeexample}

\begin{codeexample}[]
\begin{tikzpicture}
  \foreach \angle in {0,45,...,315}
    \node[rectangle,draw=black!50] (\angle) at (\angle:2) {\angle};

  \foreach \from/\to in {0/45,45/90,90/135,135/180,
                         180/225,225/270,270/315,315/0}
    \path (\from) edge [->,bend right=22,looseness=0.8] (\to)
                  edge [<-,bend left=22,looseness=0.8] (\to);
\end{tikzpicture}
\end{codeexample}
  \end{key}

  \begin{key}{/tikz/bend right=\meta{angle} (default \normalfont last  value)}
    Works like the |bend left| option, only the bend is to the other side.
  \end{key}

  \begin{key}{/tikz/bend angle=\meta{angle}}
    Sets the angle to be used by the |bend left| or |bend right|, but
    without actually selecting the |curve to| or the |relative|
    option. This is useful for globally specifying a |bend angle| for a
    whole picture.
  \end{key}

  \begin{key}{/tikz/looseness=\meta{number} (initially 1)}
    This number specifies how ``loose'' the curve will be. In detail,
    the following happens: \tikzname\ computes the distance between the
    start and the target coordinate (if the start and/or target
    coordinate are nodes, the distance is computed between the points on
    their border). This distance is then multiplied by a fixed factor
    and also by the factor \meta{number}. The resulting distance, let us
    call it $d$, is then used as the distance of the control points from
    the start and target coordinates.

    The fixed factor has been chosen in such a way that if \meta{number}
    is |1|, if the |in| and |out| angles differ by
    90$\circ$, then a quarter circle results:
  \begin{codeexample}[]
\tikz \draw (0,0) to [out=0,in=-90]               (1,1);
\tikz \draw (0,0) to [out=0,in=-90,looseness=0.5] (1,1);
  \end{codeexample}
  \end{key}

  \begin{key}{/tikz/out looseness=\meta{number}}
    specifies the looseness factor for the out distance only.
  \end{key}

  \begin{key}{/tikz/in looseness=\meta{number}}
    specifies the looseness factor for the in distance only.
  \end{key}
  \begin{key}{/tikz/min distance=\meta{distance}}
    If the computed distance for the start and target coordinates are
    below \meta{distance}, then \meta{distance} is used instead.
  \end{key}
  \begin{key}{/tikz/max distance=\meta{distance}}
    If the computed distance for the start and target coordinates are
    above \meta{distance}, then \meta{distance} is used instead.
  \end{key}
  \begin{key}{/tikz/out min distance=\meta{distance}}
    The mininimum distance set only for the start coordinate.
  \end{key}
  \begin{key}{/tikz/out max distance=\meta{distance}}
    The maximum distance set only for the start coordinate.
  \end{key}
  \begin{key}{/tikz/in min distance=\meta{distance}}
    The min distance set only for the target coordinate.
  \end{key}
  \begin{key}{/tikz/in max distance=\meta{distance}}
    The max distance set only for the target coordinate.
  \end{key}
  \begin{key}{/tikz/distance=\meta{distance}}
    Set the min and max distance to the same value \meta{distance}. Note
    that this causes any computed distance $d$ to be ignored and
    \meta{distance} to be used instead.
\begin{codeexample}[]
\begin{tikzpicture}[out=45,in=135,distance=1cm]
  \draw (0,0) to (1,0)
        (0,0) to (2,0)
        (0,0) to (3,0);
\end{tikzpicture}
\end{codeexample}
  \end{key}
  \begin{key}{/tikz/out distance=\meta{distance}}
    Sets the min and max out distance.
  \end{key}
  \begin{key}{/tikz/in distance=\meta{distance}}
    Sets the min and max in distance.
  \end{key}
  \begin{key}{/tikz/out control=\meta{coordinate}}
    This option causes the \meta{coordinate} to be used as the start
    control point. All computations of $d$ are ignored. You can use a
    coordinate like |+(1,0)| to specify a point relative to the start
    coordinate.
  \end{key}
  \begin{key}{/tikz/in control=\meta{coordinate}}
    This option causes the \meta{coordinate} to be used as the target
    control point.
  \end{key}
  \begin{key}{/tikz/controls=\meta{coordinate}| and |\meta{coordinate}}
    This option causes the \meta{coordinate}s to be used as control
    points. 
\begin{codeexample}[]
\tikz \draw (0,0) to [controls=+(90:1) and +(90:1)] (3,0);
\end{codeexample}
  \end{key}
\end{key}


\subsection{Loops}

\begin{key}{/tikz/loop}
  This key is similar to the |curve to| key, but differs in the
  following ways: First, the actual target coordinate is ignored and the
  start coordiante is used as the target coordinate. Thus, it is
  allowed not to provide any target coordinate, which can be useful
  with unnamed nodes. Second, the |looseness| is set to |8| and the
  |min distance| to |5mm|. These settings result in rather nice loops
  when the opening angle (difference between |in| and |out|) is
  30$^\circ$.
\begin{codeexample}[]
\begin{tikzpicture}
  \node [circle,draw] {a} edge [in=30,out=60,loop] ();    
\end{tikzpicture}
\end{codeexample}
\end{key}

\begin{stylekey}{/tikz/loop above}
  Sets the |loop| style and sets in and out angles such that
  loop is above the node. Furthermore, the |above| option is set,
  which causes a node label to be placed at the correct position. 
\begin{codeexample}[]
\begin{tikzpicture}
  \node [circle,draw] {a} edge [loop above] node {x} ();    
\end{tikzpicture}
\end{codeexample}
\end{stylekey}
\begin{stylekey}{/tikz/loop below} Works like the previous option. \end{stylekey}
\begin{stylekey}{/tikz/loop left} Works like the previous option. \end{stylekey}
\begin{stylekey}{/tikz/loop right} Works like the previous option. \end{stylekey}
\begin{stylekey}{/tikz/every loop (initially {->,shorten >=1pt})}
  This style is installed at the beginning of
  every loop.
\begin{codeexample}[]
\begin{tikzpicture}[every loop/.style={}]
  \draw (0,0) to [loop above] () to [loop right] ()
              to [loop below] () to [loop left]  ();
\end{tikzpicture}
\end{codeexample}
\end{stylekey}



%%% Local Variables: 
%%% mode: latex
%%% TeX-master: "pgfmanual-pdftex-version"
%%% End: 

% Copyright 2006 by Till Tantau
%
% This file may be distributed and/or modified
%
% 1. under the LaTeX Project Public License and/or
% 2. under the GNU Free Documentation License.
%
% See the file doc/generic/pgf/licenses/LICENSE for more details.


\section{Through Library}

\label{section-through-library}


\begin{tikzlibrary}{through}
  This library defines keys for creating shapes that go through given
  points. 
\end{tikzlibrary}


\begin{key}{/tikz/circle through=\meta{coordinate}}

  When this key is given as an option to a node, the following
  happens:
  \begin{enumerate}
  \item The |inner sep| and the |outer sep| are set to zero.
  \item The shape is set to |circle|.
  \item The |minimum size| is set such that the circle around the
    center of the node (which is specified using |at|), goes through
    \meta{coordinate}. 
  \end{enumerate}
\begin{codeexample}[]
\begin{tikzpicture}
  \draw[help lines] (0,0) grid (3,2);
  \node (a) at (2,1.5) {$a$};
  \node [draw] at (1,1) [circle through={(a)}] {$c$};
\end{tikzpicture}
\end{codeexample}
\end{key}

  

%%% Local Variables: 
%%% mode: latex
%%% TeX-master: "pgfmanual-pdftex-version"
%%% End: 

% Copyright 2003 by Till Tantau <tantau@cs.tu-berlin.de>.
%
% This program can be redistributed and/or modified under the terms
% of the LaTeX Project Public License Distributed from CTAN
% archives in directory macros/latex/base/lppl.txt.




\section{Tree Library}

\label{section-tree-library}


\begin{tikzlibrary}{trees}
  This packages defines styles to be used when drawing trees. 
\end{tikzlibrary}

\subsection{Growth Functions}

The package |pgflibrarytikztrees| defines two new growth
functions. They are installed using the following options:

\begin{itemize}
  \itemoption{grow via three points}|=one child at (|\meta{x}%
  |) and two children at (|\meta{y}|) and (|\meta{z}|)|
  This option installs a growth function that works as follows: If a
  parent node has just one child, this child is placed at \meta{x}. If
  the parent node has two children, these are placed at \meta{y} and
  \meta{z}. If the parent node has more than two children, the
  children are placed at points that are linearly extrapolated from
  the three points \meta{x}, \meta{y}, and \meta{z}. In detail, the
  position is $x + \frac{n-1}{2}(y-x) + (c-1)(z-y)$, where $n$ is the
  number of children and $c$ is the number of the current child
  (starting with~$1$).

  The net effect of all this is that if you have a certain ``linear
  arrangement'' in mind and use this option to specify the placement
  of a single child and of two children, then any number of children
  will be placed correctly.

  Here are some arrangements based on this growth function. We start
  with a simple ``above'' arrangement:
\begin{codeexample}[]
\begin{tikzpicture}[grow via three points={%
    one child at (0,1) and two children at (-.5,1) and (.5,1)}]
  \node at (0,0) {one} child;
  \node at (0,-1.5) {two} child child;
  \node at (0,-3) {three} child child child;
  \node at (0,-4.5) {four} child child child child;
\end{tikzpicture}
\end{codeexample}    

  The next arrangement places children above, but ``grows only to the
  right.'' 
\begin{codeexample}[]
\begin{tikzpicture}[grow via three points={%
    one child at (0,1) and two children at (0,1) and (1,1)}]
  \node at (0,0) {one} child;
  \node at (0,-1.5) {two} child child;
  \node at (0,-3) {three} child child child;
  \node at (0,-4.5) {four} child child child child;
\end{tikzpicture}
\end{codeexample}    

  In the final arrangement, the children are placed along a line going
  down and right.
\begin{codeexample}[]
\begin{tikzpicture}[grow via three points={%
    one child at (-1,-.5) and two children at (-1,-.5) and (0,-.75)}]
  \node at (0,0) {one} child;
  \node at (0,-1.5) {two} child child;
  \node at (0,-3) {three} child child child;
  \node at (0,-4.5) {four} child child child child;
\end{tikzpicture}
\end{codeexample}

  These examples should make it clear how you can create new styles to
  arrange your children along a line.

  \itemstyle{grow cyclic}
  This style causes the children to be arranged ``on a circle.'' For
  this, the children are placed at distance |\tikzleveldistance| from
  the parent node, but not on a straight line, but points on a
  circle. Instead of a sibling distance, there is a |sibling angle|
  that denotes the angle between two given children.
  \begin{itemize}
    \itemoption{sibling angle}|=|\meta{angle}
    Sets the angle between siblings in the |grow cyclic| style.
  \end{itemize}
  Note that this function will rotate the coordinate system of the
  children to ensure that the grandchildren will grow in the right
  direction.
\begin{codeexample}[]
\begin{tikzpicture}[grow cyclic]
  \tikzstyle{level 1}=[level distance=8mm,sibling angle=60]
  \tikzstyle{level 2}=[level distance=4mm,sibling angle=45]
  \tikzstyle{level 3}=[level distance=2mm,sibling angle=30]
  \coordinate [rotate=-90] % going down
    child foreach \x in {1,2,3}
      {child foreach \x in {1,2,3}
        {child foreach \x in {1,2,3}}};
\end{tikzpicture}
\end{codeexample}

  \itemoption{clockwise from}|=|\meta{angle}
  This option also cuases children to be arranged on a
  circle. However, the rule for placing children is simpler thatn with
  the |grow cyclic| style: The first child is placed at
  \meta{angle} at a distance of |\tikzleveldistance|. The second child
  is placed at the same distance from the parent, but at angle
  \meta{angle}${}-{}$|\tikzsiblingangle|. The third child is displaced
  by another |\tikzsiblingangle| in a clockwise fashion, and so on. 

  Note that this function will not rotate the coordinate system.
\begin{codeexample}[]
\begin{tikzpicture}
  \node {root}
  [clockwise from=30,sibling angle=30]
  child {node {$30$}}
  child {node {$0$}}
  child {node {$-30$}}
  child {node {$-60$}};
\end{tikzpicture}
\end{codeexample}
  \itemoption{counterclockwise from}|=|\meta{angle}
  Works the same way as |clockwise from|, but sibling angles are added
  instead of subtracted.
\end{itemize}

\subsection{Edges From Parent}

The following styles can be used to modify how the edges from parents
are drawn:

\begin{itemize}
  \itemstyle{edge from parent fork down}
  This style will draw a line from the parent downwards (for half the
  level distance) and then on to the child using only horizontal and
  vertical lines. 
\begin{codeexample}[]
\begin{tikzpicture}
  \node {root}
    [edge from parent fork down]
    child {node {left}}
    child {node {right}
      child[child anchor=north east] {node {child}}
      child {node {child}}
    };
\end{tikzpicture}
\end{codeexample}
  \itemstyle{edge from parent fork right}
  This style behaves similarly, only it will first draw its edge to
  the right.
\begin{codeexample}[]
\begin{tikzpicture}
  \node {root}
    [edge from parent fork right,grow=right]
    child {node {left}}
    child {node {right}
      child {node {child}}
      child {node {child}}
    };
\end{tikzpicture}
\end{codeexample}
  \itemstyle{edge from parent fork left}
  behaves similary. 
  \itemstyle{edge from parent fork up}
  behaves similary. 
\end{itemize}




%%% Local Variables: 
%%% mode: latex
%%% TeX-master: "pgfmanual-pdftex-version"
%%% End: 

% Copyright 2008 by Mark Wibrow
%
% This file may be distributed and/or modified
%
% 1. under the LaTeX Project Public License and/or
% 2. under the GNU Free Documentation License.
%
% See the file doc/generic/pgf/licenses/LICENSE for more details.

\section{Turtle Graphics Library}
\label{section-library-tutrle}


\begin{pgflibrary}{turtle}
  This little library defines some keys to create simple turtle
  graphics in the tradition of the Logo programming language. These
  commands are mostly for fun, but they can also be used for more
  ``serious'' business.
\begin{codeexample}[]
\tikz[turtle/distance=2mm]
  \draw [turtle={home,forward,right,forward,left,forward,left,forward}];
\end{codeexample}
\end{pgflibrary}

Even though the |turtle| keys looks like an option, it uses the
|insert path| option internally to produce a path.

The basic drawing model behind the turtle graphics is very simple:
There is a (virtual) \emph{turtle} that crawls around the page,
thereby extending the path. The turtle always heads in a certain
direction. When you move the turtle forward, you extend the path in
that direction; turning the turtle just changes the direction, it does
not cause anything to be drawn.

The turtle always moves relative to the last current point of the
path and you can mix normal path commands with turtle
commands. However, the direction of the turtle is managed
independently of other path commands.

\begin{key}{/tikz/turtle=\meta{keys}}
  This key executes the \meta{keys} with the current key path set to
  |/tikz/turtle|. 
\begin{codeexample}[]
\tikz[turtle/distance=2mm]
  \draw [turtle={home,fd,rt,fd,lt,fd,lt,fd}];
\end{codeexample}
\end{key}

\begin{key}{/tikz/turtle/home}
  Places the turtle at the origin and lets it head upward. 
\end{key}

\begin{key}{/tikz/turtle/forward=\meta{distance} (default \normalfont see text)}
  Makes the turtle move forward by the given \meta{distance}. If no
  \meta{distance} is specified, the current value of the following key
  is used:
  \begin{key}{/tikz/turtle/distance=\meta{distance} (initially 1cm)}
    The default distance by which the turtle advances.
  \end{key}
  ``Moving forward the turtle'' actually means that, relative to the
  current last point on the path, a point at the given \meta{distance}
  in the direction the turtle is currently heading is computed. Then,
  the operation |to[turtle/how]| is used to extend the path to this
  point. 
  \begin{stylekey}{/tikz/turtle/how (initially \normalfont empty)}
    This style can set up the |to path| used by turtles. By setting
    this style you can change the to-path: 
\begin{codeexample}[]
\tikz \draw [turtle={how/.style={bend left},home,forward,right,forward}];
\end{codeexample}    
  \end{stylekey}
\end{key}

\begin{key}{/tikz/turtle/fd}
  An abbreviation for the |forward| key.
\end{key}

\begin{key}{/tikz/turtle/back=\meta{distance} (default \normalfont see text)}
  This has the same effect as a |turtle/forward| for the negated
  \meta{distance} value.
\end{key}

\begin{key}{/tikz/turtle/bk}
  An abbreviation for the |back| key.
\end{key}

\begin{key}{/tikz/turtle/left=\meta{angle} (default 90)}
  Turns the turtle left by the given angle. 
\end{key}

\begin{key}{/tikz/turtle/lt}
  An abbreviation for the |left| key.
\end{key}

\begin{key}{/tikz/turtle/right=\meta{angle} (default 90)}
  Turns the turtle right by the given angle. 
\end{key}

\begin{key}{/tikz/turtle/rt}
  An abbreviation for the |right| key.
\end{key}

Turtle graphics are especially nice in conjunction with the |\foreach|
statement:

\begin{codeexample}[]
\tikz \filldraw [thick,blue,fill=blue!20]
  [turtle=home]
  \foreach \i in {1,...,5}
  {
    [turtle={forward,right=144}]
  };
\end{codeexample}

\endinput






\part{Data Visualization}
\label{part-dv}

{\Large \emph{by Till Tantau}}

\bigskip
\noindent

\begin{codeexample}[graphic=white]
\tikz \datavisualization [scientific axes=clean]
[
  visualize as smooth line=Gaussian,
  Gaussian={pin in data={text={$e^{-x^2}$},when=x is 1}}
]
data [format=function] {
  var x : interval [-7:7] samples 51;
  func y = exp(-\value x*\value x);
}
[
  visualize as scatter,
  legend={south east outside},
  scatter={
    style={mark=*,mark size=1.4pt},
    label in legend={text={
        $\sum_{i=1}^{10} x_i$, where $x_i \sim U(-1,1) $}}}
]
data [format=function] {
  var i : interval [0:1] samples 20;
  func y = 0;
  func x = (rand + rand + rand + rand + rand +
            rand + rand + rand + rand + rand);
};
\end{codeexample}

% Copyright 2008 by Till Tantau
%
% This file may be distributed and/or modified
%
% 1. under the LaTeX Project Public License and/or
% 2. under the GNU Free Documentation License.
%
% See the file doc/generic/pgf/licenses/LICENSE for more details.


\section{Introduction to Data Visualization}

\emph{Data visualization} is the process of converting \emph{data
  points,} which typically consist of multiple numerical values, into
a graphical representation. Examples include the well-known function
plots, but pie charts, bar diagrams, box plots, or vector fields are
also examples of data visualizations.

The data visualization subsystem of \pgfname\ takes a general, open
approach to data visualization. Like everything else in \pgfname,
there is a powerful, but not-so-easy-to-use basic layer in the data
visualization system and a less flexible, but much simpler-to-use
frontend layer. The present section gives an overview of the
basic ideas behind the data visualization system.


\subsection{Concept: Data Points}

\label{section-dv-intro-data-points}

The most important input for a data visualization is always raw
data. This data is typically present in different formats and the data
visualization subsystem provides methods for reading such formats and
also for defining new input formats. However, independently of the
input format, we may ask what kind of data the data visualization
subsystem should be able to process. For two-dimensional plots we
need lists of pairs of real numbers. For a bar plot we usually need a
list of numbers, possibly together with some colors and labels. For a
surface plot we need a matrix of triples of real numbers. For a vector
field we need even more complex data.

The data visualization subsystem makes no assumption as to
the kind of data that is being processed. Instead, the whole
``rendering pipeline'' is centered around a concept called the
\emph{data point}. Conceptually, a data point is an aribitrarily
complex record that represents one piece of data that should be
visualized. Data points are \emph{not} just coordinates in the plane
or the numerical values that need to be visualized. Rather, they
represent the basic units of the data that needs to be visualized.

Consider the following example: In an experiment we drive a car along
a road and have different measurement instruments installed. We
measure the position of the car, the time, the speed, the direction
the car is heading, the acceleration, and perhaps some further
values. A data point would now consist of a record consisting of a
timestamp together with the current position of the car (presumably
two or three numbers), the speed vector (another two or three
numbers), the acceleration (another two or three numbers), and perhaps
the label text of the current experiment. 

Data points should be ``information rich.'' They might even contain
more information than what will actually be visualized. It is the job
of the rendering pipeline to pick out the information relevant to one
particular data visualization -- another visualization of the same
data might pick different aspects of the data points, thereby
hopefully allowing new insights into the data.

Technically, there is no special data structure for data
points. Rather, when a special macro called |\pgfdatapoint| is called,
the ``totality'' of all currently set keys with the |/data point/|
prefix in the current scope forms the data point. This is both a very
general approach and quite fast since no extra data structures need to
be created. 


\subsection{Concept: Visualization Pipeline}

The \emph{visualization pipeline} is a series of actions that are
performed on the to-be-visualized data. As was just described, the
data is presented to the visualization pipeline in the form of a long
stream of  complex data points. The visualization pipeline makes
several passes over this stream of data points. During the first
pass(es), called the \emph{survey phase}, information is gathered
about the data points such as minimal and maximal values, which can be
useful for automatic fitting of the data into a given area. In the
main pass over the data, called the \emph{visualization phase}, the
data points are actually visualized, for instance in the form of lines
or points. 

Like for data points, the visualized pipeline makes no assumptions as
to what kind of visualization is desired. Indeed, one could even use
it to produce a plain-text table. This flexibility is achieved by
extensive use of objects and signals: When a data visualization
starts, a number of signals (see Section~\ref{section-signals} for an
introduction to signals) are initialized. Then, numerous
``visualization objects'' are created that listen to these
signals. These objects are all involved in processing the data
points. For instance, the job of an |interval mapper| object is to
map one attribute of a data point, such as a car's velocity, to
another, such as the $y$-axis of a plot. For each data point the
different signals are raised in a certain order and the different
visualization objects now have a chance of preparing the data point
for the actual visualization. Continuing the above example, there
might be a second |interval mapper| that takes the computed
$y$-position and applies a logarithm to it, because a log-plot was
requested. Then another mapper, this time a |polar mapper| might be
used to map everything to polar coordinates. Following this, a
|plot mark visualizer| might actually draw something at the computed
position.

The whole idea behind the rendering pipeline is that new kinds of data
visualizations can be implemented, ideally, just by adding one or two
new objects to the visualization pipeline. Furthermore, different
kinds of plots can be combined in novel ways in this manner, which is
usually very hard to do. For instance, the visualization pipeline
makes it easy to create, say, polar-semilog-box-plots. At first sight,
such new kinds of plots may seem superfluous, but data visualization
is all about gaining insights into the data from as many different
angles as possible.

Naturally, creating new classes and objects for the rendering pipeline
is not trivial, so most users will just use the existing classes,
which should, thus, be as flexible as possible. But even when one only
intends to use existing classes, it is still tricky to setup the
pipeline correctly since the ordering is obviously important and since
tricky things like axes and ticks need to be configured and taken care
of. For this reason, the frontend libraries provide
preconfigured rendering pipelines so that one can simply say that a
data visualization should look like a |line plot| with
|school book axes| or with |scientific axes|, which selects a certain
visualization pipeline appropriate for this kind of plot:
\begin{codeexample}[]
\begin{tikzpicture}[scale=.7]
  \datavisualization [school book axes, smooth line plot]
  data [format=function] {
    var x : interval [-2:2];
    func y = \value x*\value x + 1;
  };
\end{tikzpicture}
\end{codeexample}
\begin{codeexample}[]
\begin{tikzpicture}[scale=.7]
  \datavisualization [scientific axes, smooth line plot]
  data [format=function] {
    var x : interval [-2:2];
    func y = \value x*\value x + 1;
  };
\end{tikzpicture}
\end{codeexample}
One must still configure such a plot (choose styles and themes
and also specify which attributes of a data point should be used), but
on the whole the plot is quite simple to specify.



%%% Local Variables: 
%%% mode: latex
%%% TeX-master: "pgfmanual"
%%% End: 

% Copyright 2008 by Till Tantau
%
% This file may be distributed and/or modified
%
% 1. under the LaTeX Project Public License and/or
% 2. under the GNU Free Documentation License.
%
% See the file doc/generic/pgf/licenses/LICENSE for more details.


\section{Creating Data Visualizations}
\label{section-dv-main}
\label{section-dv-main-setup}

\subsection{Overview}

The present section explains how a data visualization is created in
\tikzname. For this, you need to include the |datavisualization|
library and then use the command |\datavisualization| whose syntax is
explained in the rest of the present section. This command is part of
the following library:

\begin{tikzlibrary}{datavisualization}
  This library must be loaded if you wish to use the
  |\datavisualization| command. It defines all styles needed to create
  basic data visualizations; additional, more specialized libraries
  need to be loaded for more advanced features.
\end{tikzlibrary}


In order to visualize, you basically need to do three things:
\begin{enumerate}
\item You need to select what kind of plot you would like to have (a
  ``school book plot'' or a ``scientific 2d plot'' or a ``scientific
  spherical plot'' etc.). This is done by passing an option to the
  |\datavisualization| command that selects this kind of plot.
\item You need to provide data points, which is done using the |data|
  command.
\item Additionally, you can add options that give you more
  fine-grained control over the way the visualization will look. You
  can configure the number of ticks and grid lines, where the labels
  are placed, the colors, or the fonts. Indeed, since the
  data visualization engine internally uses \tikzname-styles, you can
  have extremely fine-grained control over how a plot will look like.
\end{enumerate}

The syntax of the |\datavisualization| command is designed in such a
way that if you only need to provide very few options to create plots
that ``look good by default''.

This section is structured as follows: First, the philosophy behind
concepts like ``data points,'' ``axes,'' or ``visualizers'' is
explained. Each of these concepts is further detailed in later
section. Then, the syntax of the |\datavisualization| command is
covered. The reference sections explain which predefined plot kinds
are available.


\subsection{Concept: Data Points and Data Formats}

As explained in Section~\ref{section-dv-intro-data-points}, data
points are the basic entities that are processed by the data
visualization engine. In order to specify data points, you use the
|data| command, whose syntax is explained in more detail in
Section~\ref{section-dv-data-syntax}. The |data| command allows you to
either specify points ``inline,'' directly inside your \TeX-file; or
you can specify the name of file that contains the data points.

\medskip
\textbf{Specifying data points.}
Data points can be formatted in different ways. For instance, in the so
called \emph{comma separated values} format, there is one line for
each data point and the different attributes of a data point are
separated by commas. Another common format is to specify data points
using the so called \emph{key-value} format, where on each line the
different attributes of a data point are set using a comma-separated
list of strings of the form |attribute=value|.

Here are two examples, where similar data is given in different
formats:

    \begin{codeexample}[]
\begin{tikzpicture}
  \datavisualization [school book axes, visualize as smooth line]
    data {
      x, y
      -1.5, 2.25
      -1, 1
      -.5, .25
      0, 0
      .5, .25
      1, 1
      1.5, 2.25
    };
\end{tikzpicture}
    \end{codeexample}

    \begin{codeexample}[]
\begin{tikzpicture}
  \datavisualization [school book axes, visualize as smooth line]
    data [format=function] {
      var x : interval [-1.5:1.5] samples 7;
      func y = \value x*\value x;
    };
\end{tikzpicture}
    \end{codeexample}

In the first example, no format needed to be specified explicitly
since the default format is the one used for the data following the
|data| keyword: A list of comma-separated values, where each line
represents a data point.

\medskip
\textbf{Number accuracy.}\label{section-dv-expressions}
Data visualizations typically demand a much higher accuracy and range
of values than \TeX\ provides: \TeX\ numbers are limited to 13 bits
for the integer part and 16 bits for the fractional part. Because of
this, the data visualization engine does not use \pgfname's standard
representation of numbers and \TeX\ dimensions and is does not use the
standard parser when reading numbers in a data point. Instead, the
|fpu| library, described in Section~\ref{section-library-fpu}, is used
to handle numbers.

This use of the |fpu| library has several effects that users of the data
visualization system should be aware of:
\begin{enumerate}
\item You can use numbers like |100000000000000| or |0.00000000001| in
  a data points.
\item Since the |fpu| library does not support advanced parsing, you
  currently \emph{cannot} write things like |3+2| in a data point
  number. This will result in an error
\item However, there is a loop-hole: If a ``number'' in a data point
  starts with a parenthesis, the value between the parentheses
  \emph{is} parsed using the normal parser:
  \begin{itemize}
  \item |100000| is allowed.
  \item |2+3| yields an error.
  \item |(2+3)| is allowed and evaluates to |5|.
  \item |(100000)| yields an error since $100000$ is beyond the normal
    parser's precision.
  \end{itemize}
  The bottom line is that any normal calculations should be set inside
  round parentheses, while large numbers should not be surrounded by
  parentheses. Hopefully, in the future, these restrictions will be
  lifted.
\end{enumerate}


Section~\ref{section-dv-formats} gives an
in-depth coverage of the available data formats and explains how new
data formats can be defined.




\subsection{Concept: Axes, Ticks, and Grids}

Most plots have two or three axes: A horizontal axis usually called
the $x$-axis, a vertical axis called the $y$-axis, and possibly some
axis pointing in a sloped direction called the $z$-axis. Axes are
usually drawn as lines with \emph{ticks} indicating interesting
positions on the axes. The data visualization engine gives you
detailed control over where these ticks are rendered and how many of
them are used. Great care is taken to ensure that the position of
ticks are chosen well by default.

From the point of view of the data visualization engine, axes are a
somewhat more general concept than ``just'' lines that point ``along''
some dimension: The data visualization engine uses axes to visualize
any change of an attribute by varying the position of data points in the
plane. For instance, in a polar plot, there is an ``axis'' for the
angle and another ``axis'' for the distance if the point from the
center. Clearly these axes vary the position of data points in the
plane according to some attribute of the data points; but just as
clearly they do not point in any ``direction.''

A great benefit of this approach is that the powerful methods for
specifying and automatic inference of ``good'' positions for ticks or
grid lines apply to all sorts of situations. For instance, you can use
it to automatically put ticks and grid lines at well-chosen angles of
a polar plot.

Typically, you will not need to specify axes explicitly. Rather,
predefined styles take care of this for you:


    \begin{codeexample}[]
\begin{tikzpicture}
  \datavisualization [
    scientific axes,
    x axis={length=3cm, ticks=few},
    visualize as smooth line
  ]
    data [format=function] {
      var x : interval [-1.5:1.5] samples 7;
      func y = \value x*\value x;
    };
\end{tikzpicture}
    \end{codeexample}


    \begin{codeexample}[]
\begin{tikzpicture}
  \datavisualization [
    scientific clean axes,
    x axis={length=3cm, ticks=few},
    all axes={grid},
    visualize as smooth line
  ]
    data [format=function] {
      var x : interval [-1.5:1.5] samples 7;
      func y = \value x*\value x;
    };
\end{tikzpicture}
    \end{codeexample}

Section~\ref{section-dv-axes} explains in more detail how axes, ticks,
and grid lines can be chosen and configured.


\subsection{Concept: Visualizers}

Data points and axes specify \emph{what} is visualized and
\emph{where}. A \emph{visualizer} specifies \emph{how} they are
visualized. One of the most common visualizers is a \emph{line
  visualizer} which connects the positions of the data points in the
plane using a line. Another common visualizer is the \emph{scatter
  plot visualizer} where small marks are drawn at the positions of the
data points. More advanced visualizers include, say, box plot
visualizers or pie chart visualizers.

    \begin{codeexample}[]
\begin{tikzpicture}
  \datavisualization [
    scientific clean axes,
    x axis={length=3cm, ticks=few},
    visualize as smooth line
  ]
    data [format=function] {
      var x : interval [-1.5:1.5] samples 7;
      func y = \value x*\value x;
    };
\end{tikzpicture}
    \end{codeexample}
    \begin{codeexample}[]
\begin{tikzpicture}
  \datavisualization [
    scientific clean axes,
    x axis={length=3cm, ticks=few},
    visualize as scatter
  ]
    data [format=function] {
      var x : interval [-1.5:1.5] samples 7;
      func y = \value x*\value x;
    };
\end{tikzpicture}
    \end{codeexample}

Section~\ref{section-dv-visualizers} provides more information on
visualizers as well as reference lists.


\subsection{Concept: Style Sheets and Legends}

A single data visualizations may use more than one visualizer. For
instance, if you wish to create a plot containing several lines, a
separate visualizer is used for each line. In this case, two problems
arise:

\begin{enumerate}
\item You may wish to make it easy for the reader to differentiate
  between the different visualizers. For instance, one line should be
  black, another should be red, and another blue. Alternatively, you
  might wish one line to be solid, another to be dashed, and a third
  to be dotted.

  Specifying such styles is trickier than one might expect; experience
  shows that many plots use ill-chosen and inconsistent
  styling. For this reason, the data visualization introduces the
  notion of \emph{style sheets} for visualizers and comes with some
  well-designed predefined style sheets.
\item You may wish to add information concerning what the different
  visualizers represent. This is typically done using a legend, but it
  is even better to add labels directly inside the visualization. Both
  approaches are supported.
\end{enumerate}

An example where three functions are plotted and a legend is added is
shown below. Two style sheets are used so that \emph{both} the
coloring and the dashing is varied.

\begin{codeexample}[]
\begin{tikzpicture}[baseline]
  \datavisualization [ scientific clean axes,
                       y axis=grid,
                       visualize as smooth line/.list={sin,cos,tan},
                       visualizer style sheet=strong colors,
                       visualizer style sheet=vary dashing,
                       sin={label in legend={text=$\sin x$}},
                       cos={label in legend={text=$\cos x$}},
                       tan={label in legend={text=$\tan x$}},
                       data/format=function ]
  data [sin] {
    var x : interval [-0.5*pi:4];
    func y = sin(\value x r);
  }
  data [cos] {
    var x : interval [-0.5*pi:4];
    func y = cos(\value x r);
  }
  data [tan] {
    var x : interval [-0.3*pi:.3*pi];
    func y = tan(\value x r);
  };
\end{tikzpicture}
\end{codeexample}

Section~\ref{section-dv-style-sheets} details style sheets and
legends.


\subsection{Usage}
\label{section-dv-data-syntax}

Inside a \tikzname\ picture you can use the |\datavisualization|
command to create a data visualization. You can use this command
several times in a picture to create pictures containing multiple data
visualizations.

\begin{command}{\datavisualization\opt{\oarg{data visualization
      options}}\meta{data specification}|;|}
  This command is available only inside a |{tikzpicture}| environment.

  The \meta{data visualization options} are used to configure the data visualization,
  that is, how the data is to be depicted. The options are executed
  with the path prefix |/tikz/data visualization|. This means that
  normal \tikzname\ options like |thin| or |red| cannot be used
  here. Rather, a large number of options specific to data
  visualizations are available.

  As a minimum, you should specify at least two options: First, you
  should use an option that selects an axis system that is appropriate
  for your plot. Typical possible keys are |school book axes| or
  |scientific axes|, detailed information on them can be found in
  Section~\ref{section-dv-axes}.

  Second, you use an option to select
  \emph{how} the data should be visualized. This is done using a key
  like |visualize as line| which will, as the name suggests, visualize
  the data by connecting data points in the plane using a
  line. Similarly, |visualize as smooth cycle| will try to fit a smooth
  cycle through the data points. Detailed information on possible
  visualizers can be found in Section~\ref{section-dv-visualizers}.

  Following these options, the \meta{data specification} is used to
  provide the actual to-be-visualized data. The syntax is somewhat
  similar to commands like |\path|: The \meta{data
    specification} is a sequence of keywords followed by local options
  and parameters, terminated with a semicolon. (Indeed, like for the
  |\path| command, the \meta{data visualizers options} need not be
  specified at the beginning, but additional option surrounded by
  square brackets may be given anywhere inside the \meta{data specification}.)

  The different possible keywords inside the \meta{data specification}
  are explained in the following.
\end{command}


\begin{datavisualizationoperation}{data}{\opt{\oarg{options}}\opt{\marg{inline
        data}}}
  This command is used to specify data for the data visualization. It
  can be used several times inside a single visualization and each
  time the to-be-read data may have a different format, but the data
  will be visualized as if it have been specified inside a single |data|
  command.

  The behaviour of the |data| command depends on
  whether the \meta{inline data} is present. If it is not present, the
  \meta{options} must be used to specify a source file from which the
  data is read; if the \meta{inline data} is present no file will be
  used, instead the data should directly reside inside the \TeX-file
  and be given between the curly braces surrounding the \meta{inline
    data}.

  The \meta{options} are executed with the prefix |/pgf/data|. The
  following options are always available:
  \begin{key}{/pgf/data/read from file=\meta{filename} (initially \normalfont empty)}
    If you set the |source| attribute to a non-empty \meta{filename},
    the data will be read from this file. In this case, no
    \meta{inline data} may be present, not even empty curly braces
    should be provided.
\begin{codeexample}[code only]
\datavisualization ...
  data [read from file=file1.csv]
  data [read from file=file2.csv];
\end{codeexample}
    The other way round, if |read from file| is empty, the  data must directly
    follow as \meta{inline data}.
\begin{codeexample}[code only]
\datavisualization ...
  data {
    x, y
    1, 2
    2, 3
  };
\end{codeexample}
  \end{key}
  The second important key is |format|, which is used to specify the
  data format:
  \begin{key}{/pgf/data/format=\meta{format} (initially table)}
    Use this key to locally set the format used for parsing the
    data, see Section~\ref{section-dv-formats} for a list of
    predefined formats.

    The default format is the |table|-format, also known as
    ``comma-separated values.'' The first line contains names of
    attributes separated by commas, all following lines constitute a
    data point where the attributes are given by the comma-separated
    values in that line.
  \end{key}

  \medskip
  \textbf{Presetting attributes.}
  Normally, the inline data or the external data contains for each
  data point the values of the different attributes. However,
  sometimes you may also wish to set an attribute to a fixed value for
  all data points of a data set. Suppose, for instance, that you have
  to source files |experiment007.csv| and |experiment023.csv| and you
  would like that for all data points of the first file the attribute
  |/data point/experiment id| is set to 7 while for the data points of
  the second file they are set to 23. In this case, you can specify
  the desired settings using an absolute path inside the
  \meta{options}. The effect will be local to the current |data|
  command:
\begin{codeexample}[code only]
\datavisualization...
  data [/data point/experiment=7,  read from file=experiment007.csv]
  data [/data point/experiment=23, read from file=experiment023.csv];
\end{codeexample}

  \begin{codeexample}[]
\begin{tikzpicture}
  \datavisualization [school book axes, visualize as line]
    data [/data point/x=1] {
      y
      1
      2
    }
    data [/data point/x=2] {
      y
      2
      0.5
    };
\end{tikzpicture}
\end{codeexample}

  \medskip
  \textbf{Setting options for multiple |data| commands.}
  You may wish to generally set the format once and for all. This can
  be done by using the following key:
  \begin{stylekey}{/tikz/every data}
    This key is executed for every |data| command.
  \end{stylekey}

  Another way of passing options to multiple |data| commands is to use
  the following facility: Whenever an option with the path
  |/tikz/data visualization/data| is used, the path will be remapped
  to  |/pgf/data|. This means, in particular, that you can pass an
  option like |data/format=table| to the |\datavisualization| command
  to set the data format for all |data| commands of the data
  visualization.

  \medskip
  \textbf{Parsing inline data.}
  When you specify data inline, \TeX\ needs to read the data
  ``line-by-line,'' while \TeX\ normally largely ignores end-of-line
  characters. For this reason, the data visualization system
  temporarily changes the meaning of the end-of-line character. This
  is only possible if \TeX\ has not already processed the data in some
  other way (namely as the parameter to some macro).

  The bottom line is that you cannot use inline data when the whole
  |\datavisualization| command is passed as a parameter to some
  macro that is not setup to handle ``fragile'' code. For instance, in
  a \textsc{beamer} |frame| you need to add the |fragile| option when
  a data visualization contains inline data.

  The problem does not arise when an external data |source| is
  specified.
\end{datavisualizationoperation}



\begin{datavisualizationoperation}{data set}{\opt{\oarg{options}}\marg{name}\opt{|+=|\marg{data specifications}}}
  You can store a whole \meta{data specification} in a \emph{data
    set}. This allows you to reuse data in multiple places without
  having to write the data to an external file.

  The syntax of this command comes in the following three variants:
  \begin{itemize}
  \item |data set| \opt{\oarg{options}} \marg{name} |=| \marg{data specifications}
  \item |data set| \opt{\oarg{options}} \marg{name} |+=| \marg{data specifications}
  \item |data set| \opt{\oarg{options}} \marg{name}
  \end{itemize}
  In the first case, a new data set called \meta{name} is created (an
  existing data set of the same name will be erased) and the following
  \meta{data specifications} is stored in this data set. The data set
  will not be fed to the rendering pipeline, but it is parsed at this
  point as if it were. The defined data set is defined globally, so
  you can used it in subsequent visualizations. The \meta{options} are
  saved with the parsed \meta{data specifications}.

  In the second case, an already existing data set is extended by
  adding the \meta{data specifications} to it.

  In the third case (detected by noting that the \meta{name} is
  neither followed by an equal sign nor a plus sign), the contents of
  the previously defined data set \meta{name} is inserted. The
  \meta{options} are also executed.

  Let is now first create a data set. Note that nothing is drawn since
  the ``dummy'' data visualization is empty and used only for the
  definition of the data set.
\begin{codeexample}[]
\tikz \datavisualization data set {points} = {
  data {
    x, y
    0, 1
    1, 2
    2, 2
    5, 1
    2, 0
    1, 1
  }
};
\end{codeexample}

  We can now use this data in different plots:
\begin{codeexample}[]
\tikz \datavisualization [school book axes,      visualize as line] data set {points};
\qquad
\tikz \datavisualization [scientific clean axes, visualize as line] data set {points};
\end{codeexample}
\end{datavisualizationoperation}


\begin{datavisualizationoperation}{scope}{\opt{\oarg{options}}\marg{data
      specification}}
  Scopes can be used to nest hierarchical data sets. The
  \meta{options} will be executed with the path |/pgf/data| and will
  only apply to the data sets specified inside the \meta{data
    specification}, which may contain |data| or |scope| commands once more:
\begin{codeexample}[code only]
\datavisualization...
  scope [/data point/experiment=7]
  {
    data [read from file=experiment007-part1.csv]
    data [read from file=experiment007-part2.csv]
    data [read from file=experiment007-part3.csv]
  }
  scope [/data point/experiment=23, format=foo]
  {
    data [read from file=experiment023-part1.foo]
    data [read from file=experiment023-part2.foo]
  };
\end{codeexample}
\end{datavisualizationoperation}


\begin{datavisualizationoperation}{info}{\opt{\oarg{options}}\marg{code}}
  This command will execute normal \tikzname\ \meta{code} at the end
  of a data visualization. The \meta{options} are executed with the
  normal path |/tikz/|.

  The only difference between this command and just giving the
  \meta{code} directly following the data visualization is that inside
  the \meta{code} following an |info| command you still have access
  to the coordinate system of the data visualization. In sharp
  contrast, \tikzname\ code given after a data visualization can no
  longer access this coordinate system.


\begin{codeexample}[]
\begin{tikzpicture}[baseline]
  \datavisualization [ school book axes, visualize as line ]
  data [format=function] {
    var x : interval [-0.1*pi:2];
    func y = sin(\value x r);
  }
  info {
    \draw [red] (visualization cs: x={(.5*pi)}, y=1) circle [radius=1pt]
      node [above,font=\footnotesize] {extremal point};
  };
\end{tikzpicture}
\end{codeexample}

  As can be seen, inside a data visualization a special coordinate
  system is available:

  \begin{coordinatesystem}{visualization}
    As for other coordinate systems, the syntax is
    \declare{|(visualization cs:|\meta{list
        of attribute-value pairs}|)|}. The effect is the following:
    For each pair \meta{attribute}|=|\meta{value} in the \meta{list}
    the key |/data point/|\meta{attribute} is set to
    \meta{value}. Then, it is computed where the resulting data point
    ``would lie'' on the canvas (however, no data point is passed to
    the visualizers).
  \end{coordinatesystem}
\end{datavisualizationoperation}

\begin{datavisualizationoperation}{info'}{\opt{\oarg{options}}\marg{code}}
  This command works like |info|, only the \meta{code} will be
  executed just before the visualization is done. This allows you to
  draw things \emph{behind} the visualization.

\begin{codeexample}[]
\begin{tikzpicture}[baseline]
  \datavisualization [ school book axes, visualize as line ]
  data [format=function] {
    var x : interval [-0.1*pi:2];
    func y = sin(\value x r);
  }
  info' {
    \fill [red] (visualization cs: x={(.5*pi)}, y=1) circle [radius=2mm];
  };
\end{tikzpicture}
\end{codeexample}
\end{datavisualizationoperation}


\begin{predefinednode}{data visualization bounding box}
  This rectangle node stores a bounding box of the data visualization
  that is currently being constructed. This node can be useful inside
  |info| commands or when labels need to be added.
\end{predefinednode}

\subsection{Advanced: Executing User Code During a Data Visualization}
\label{section-dv-user-code}

The following keys can be passed to the |\datavisualization| command
and allow you to execute some code at some special time during the
data visualization process. For details of the process and on which
signals are emitted when,
see Section~\ref{section-dv-backend}.

\begin{key}{/tikz/data visualization/before survey=\meta{code}}
  The \meta{code} is passed to the |before survey| method of the data
  visualization object and then executed at the appropriate time (see
  Section~\ref{section-dv-backend} for details).

  The following commands work likewise:
\end{key}
\begin{key}{/tikz/data visualization/at start survey=\meta{code}}
\end{key}
\begin{key}{/tikz/data visualization/at end survey=\meta{code}}
\end{key}
\begin{key}{/tikz/data visualization/after survey=\meta{code}}
\end{key}
\begin{key}{/tikz/data visualization/before visualization=\meta{code}}
\end{key}
\begin{key}{/tikz/data visualization/at start visualization=\meta{code}}
\end{key}
\begin{key}{/tikz/data visualization/at end visualization=\meta{code}}
\end{key}
\begin{key}{/tikz/data visualization/after visualization=\meta{code}}
\end{key}



\subsection{Advanced: Creating New Objects}

You will need the following key only when you wish to create new
rendering pipelines from scratch -- instead of modifying an existing
pipeline as you would normally do. In the following it is assumed that
you are familiar with the concepts of Section~\ref{section-dv-backend}.

\begin{key}{/tikz/data visualization/new object=\meta{options}}
  This key servers two purposes:
  \begin{enumerate}
  \item
    This method makes it easy to create a new object as part of the
    rendering pipeline, using \meta{options} to specify arguments rather
    that directly calling |\pgfoonew|. Since you have the full power
    of the keys mechanism at your disposal, it is easy, for instance,
    to control whether or not parameters to the constructor are
    expanded or not.
  \item
    The object is not created immediately, but only just before the
    visualization starts. This allows you to specify that an object
    must be created, but the parameter values of for its constructor
    may depend on keys that are not yet set. A typical application is
    the creating of an axis object: When you say |scientific axes|,
    the |new object| command is used internally to create two
    objects representing these axes. However, keys like
    |x={length=5cm}| can only \emph{later} be used to specify the
    parameters that need to be passed to the constructor of the
    objects.
  \end{enumerate}

  The following keys may be used inside the \meta{options}:
  \begin{key}{/tikz/data visualization/class=\meta{class name}}
    The class of the to-be-created object.
  \end{key}
  \begin{key}{/tikz/data visualization/when=\meta{phase name}
      (initially before survey)}
    This key is used to specify when the object is to be created. As
    described above, the object is not created immediately, but at
    some time during the rendering process. You can specify any of the
    phases defined by the data visualization object, see
    Section~\ref{section-dv-backend} for details.
  \end{key}
  \begin{key}{/tikz/data visualization/store=\meta{key name}}
    If the \meta{key name} is not empty, once the object has been
    created, a handle to the object will be stored in \meta{key
      name}. If a handle is already stored in \meta{key name}, the
    object is not created twice.
  \end{key}
  \begin{key}{/tikz/data visualization/before creation=\meta{code}}
    This code is executed right before the object is finally
    created. It can be used to compute values that are then passed to
    the constructor.
  \end{key}
  \begin{key}{/tikz/data visualization/after creation=\meta{code}}
    This code is executed right after the object has just been
    created. A handle to the just-created object is available in
    |\tikzdvobj|.
  \end{key}
  \begin{key}{/tikz/data visualization/arg1=\meta{value}}
    The value to be passed as the first parameter to the
    constructor. Similarly, the keys |arg2| to |arg8| specify further
    parameters passed. Naturally, only as many arguments are passed as
    parameters are set. Here is an example:
\begin{codeexample}[code only]
\tikzdatavisualizationset{
  new object={
    class = example class,
    arg1  = foo,
    arg2  = \bar
  }
}
\end{codeexample}
  causes the following object creation code to be executed later on:
\begin{codeexample}[code only]
\pgfoonew \tikzdvobj=new example class(foo,\bar)
\end{codeexample}
    Note that you key mechanisms like |.expand once| to pass the value of
    a macro instead of the macro itself:
\begin{codeexample}[code only]
\tikzdatavisualizationset{
  new object={
    class              = example class,
    arg1               = foo,
    arg2/.expand once  = \bar
  }
}
\end{codeexample}
  Now, if |\bar| is set to |This \emph{is} it.| at the moment to
  object is created later on, the following object creation code is executed:
\begin{codeexample}[code only]
\pgfoonew \tikzdvobj=new example class(foo,This \emph{is} it)
\end{codeexample}
  \end{key}

  \begin{key}{/tikz/data visualization/arg1 from key=\meta{key}}
    Works like the |arg1|, only the value that is passed to the
    constructor is the current value of the specified \meta{key} at
    the moment when the object is created.
\begin{codeexample}[code only]
\tikzdatavisualizationset{
  new object={
    class              = example class,
    arg1 from key      = /tikz/some key
  }
}
\tikzset{some key/.initial=foobar}
\end{codeexample}
  causes the following to be executed:
\begin{codeexample}[code only]
\pgfoonew \tikzdvobj=new example class(foobar)
\end{codeexample}
    Naturally, the keys |arg2 from key| to |arg8 from key| are also
    provided.
  \end{key}

  \begin{key}{/tikz/data visualization/arg1 handle from key=\meta{key}}
    Works like the |arg1 from key|, only the key must store an object
    and instead of the object a handle to the object is passed to the
    constructor.
  \end{key}
\end{key}

% Copyright 2008 by Till Tantau
%
% This file may be distributed and/or modified
%
% 1. under the LaTeX Project Public License and/or
% 2. under the GNU Free Documentation License.
%
% See the file doc/generic/pgf/licenses/LICENSE for more details.


\section{Providing Data for a Data Visualization}

\subsection{Overview}

The data visualization system needs a stream of data points as
input. These data points can be directly generated by repeatedly
calling the |\pgfdatapoint| command, but usually data is available in
some special (text) format and one one would like to visualize this
data. The present section explains how data in some specific format
can be fed to the data visualization system.

This section starts with an explanation of the main concepts. Then,
the standard formats are listed in the reference section. It is also
possible to define new formats, but this an advanced concept which
requires an understanding of some of the internals of the parsing mechanism,
explained in Section~\ref{section-dv-parsing}, and the usage of a
rather low-level command, explained in Section~\ref{section-dv-declaring-formats}.


\subsection{Concepts}

For the purposes of this section, let call a \emph{data format} some
standarderized way of writing down a list of data points. A simple
example of a data format is the \textsc{csv} format (the acronym
stands for \emph{comma separated values}), where each line contains a
data point, specified by values separated by commas. A different
format is the \emph{key--value format}, where data points are
specified by lists of key--value pairs. A far more complex format is
the \textsc{pdb}-format used by the protein database to describe
molecules.

The data visualization system does not use any specific
format. Instead, whenever data is read by the data visualization
system, you must specify a format parser (or it is chosen
automatically for you). It is the job of the parser to read (parse)
the data lines and to turn them into data points, that is, to setup
appropriate subkeys of |/data point/|.

To give a concrete example, suppose a file contains the following
lines:
\begin{codeexample}[code only]
x, y, z
0, 0, 0
1, 1, 0
1, 1, 0.5
0, 1, 0.5
\end{codeexample}
This file is in the \textsc{csv}-format. This format can be read by
the |table| parser (which is called thus, rather than ``|csv|,'' since
it can also read files in which the columns are separated by, say, a
semicolon or a space). The |table| format will then read the data and
for each line of the data, except for the headline of course, it will
produce one data point. For instance, for the last data point the key
|/data point/x| will be set to |0|, the key |/data point/y| will be
set to |1|, and the key |/data point/z| will be set to |0.5|.

All parsers are basically line-oriented. This means that, normally,
each line in the input data should contain one data point. This rule
may not always apply, for instance empty lines are typically ignored
and sometimes a data point may span several lines, but deviating from
this ``one data point per line'' rule makes parsers harder to
program. 


\subsection{Reference: Standard Formats}

The following format is the default format, when no |format=...| is
specified.

\begin{dataformat}{table}
  This format is used to parse data that is formatted in the following
  manner: Basically, each line consists of \emph{values} that are
  separated by a \emph{separator} like a comma or a space. The values
  are stored in different \emph{attributes}, that is, subkeys of
  |/data point| like |/data point/x|. In order to decide which
  attribute is chosen for a give value, the headline is
  important. This is the first non-empty line of a table. It is
  formatted in the same way as normal data lines (value separated by
  the separator), but the meaning of the values is different: The
  first value in the headline is the name of the attribute where the
  first values in the following lines should go each time. Similaly,
  the second value in the headline is the name of the attribute for
  the second values in the following lines, and so on.

  A simple example is the following:
\begin{codeexample}[code only]
angle, radius
0, 1
45, 2
90, 3
135, 4
\end{codeexample}
  The headline states that the values in the first column should be
  stored in the |angle| attribute (|/data point/angle| to be precise)
  and that the values in the second column should be stored in the
  |radius| attribute. There are four data points in this data set.

  The format will tolerate too few or too many values in a line. If
  there are less values in a line than in the headline, the last
  attributes will simply be empty. If there are more values in a line
  than in the headline, the values are stored in attributes called
  |/data point/attribute |\meta{column number}, where the first value
  of a line gets \meta{column number} equal to |1| and so on.

  The |table| format can be configured using the following options:
  \begin{key}{/pgf/data/separator=\meta{character} (initially ,)}
    Use this key to change which character is used to separate values
    in the headline and in the data lines. To set the separator to a
    space, either set this key to an empty value or say
    |separator=\space|. Note that you must surround a comma by curly
    braces if you which to (re)set the separator charactor to a space.
    \begin{codeexample}[]
\begin{tikzpicture}
  \datavisualization [school book plot] 
    data [separator=\space] {
      x y
      0 0
      1 1
      2 1
      3 0
    }
    data [separator=;] {
      x; y; z
      3; 1; 0
      2; 2; 0
    };  
\end{tikzpicture}
    \end{codeexample}
  \end{key}
  \begin{key}{/pgf/data/headline=\meta{headline}}
    When this key is set to a non-empty value, the value of
    \meta{headline} is used as the headline and the first line of the
    data is treated as a normal line rather than as a headline.
    \begin{codeexample}[]
\begin{tikzpicture}
  \datavisualization [school book plot]
    data [headline={x, y}] {
      0, 0
      1, 1
      2, 1
      3, 0
    };
\end{tikzpicture}
    \end{codeexample}
  \end{key}  
\end{dataformat}


\begin{dataformat}{key value pairs}
  Each line of the data is simply passed to the |\pgfkeys| command
  with the path set to |/data point/|. This means that each data line
  will typically consist of comma-separated assignments of the form
  \meta{attribute}|=|\meta{value}.
  \begin{codeexample}[]
\begin{tikzpicture}
  \datavisualization [school book plot]
    data [format=key value pairs] {
      x=0, y=0
      x=1, y=1
      x=2, y=1
      x=3, y=0
    };
\end{tikzpicture}
  \end{codeexample}  
\end{dataformat}

\begin{dataformat}{TeX code}
  This format will simply execute each line of the data, each of which
  should contain some normal TeX code. Note that at the end of each
  line control returns to the format handler, so for instance the
  arguments of a command may not be spread over several
  lines. However, not each line needs to produce a data point.
  \begin{codeexample}[]
\begin{tikzpicture}
  \datavisualization [school book plot]
    data [format=TeX code] {
      \pgfkeys{/data point/.cd,x=0, y=0} \pgfdatapoint
      \pgfkeys{/data point/.cd,x=1, y=1} \pgfdatapoint
      \pgfkeys{/data point/x=2}          \pgfdatapoint
      \pgfkeyssetvalue{/data point/x}{3}
      \pgfkeyssetvalue{/data point/y}{0} \pgfdatapoint
    };
\end{tikzpicture}
  \end{codeexample}  
\end{dataformat}

\begin{tikzlibrary}{datavisualization.formats.functions}
  This library defines the formats described in the following, which
  allow you to specify the data points indirectly, namely via a
  to-be-evaluated function. 
  
  \begin{dataformat}{function}
    This format allows you to specify a function that is then
    evaluated in order to create the desired data points. In other
    words, the data lines do not contain the data itself, but rather
    a functional description of the data.

    The format used to specify the function works as follows: Each
    nonempty line of the data should contain at least one of either a
    \emph{variable  declaration} or a \emph{function declaration}. A
    variable declaration signals that a certain attribute will range
    over a given interval. The function declarations will then, later,
    be evaluated for values inside this interval. The syntax for a
    variable declaration is as follows:
    \begin{quote}
      |var |\declare{\meta{variable}}| : interval[|\meta{low}|:|\meta{high}|]|
      \opt{|samples |\meta{number}} \opt{|step |\meta{step}}|;|
    \end{quote}
    The |samples| and |step| options cannot be given both at the same
    time. If both are missing, |samples| is used with the value stored
    in the following key:
    \begin{key}{/pgf/data/samples=\meta{number} (initially 25)}
      Sets the number of samples to be used when no sample number is
      specified. 
    \end{key}
    The meaning of a declaration like the above is the following: The
    attribute named \meta{variable}, that is, the key
    |/data point/|\meta{variable}, will range over the interval
    $[\meta{low},\meta{high}]$. If the number of |samples| is given
    (directly or indirectly), the interval is evenly divided into
    \meta{number} many points and the attribute is set to each of
    these values. Similarly, when a \meta{step} is specified, this
    stepping is used to increase \meta{low} iteratively up to the
    largest value that is still less or equal to \meta{high}.

    You can specify more than one variable. In this case, each
    variable is varied independetly of the other variables. For
    instance, if you declare an $x$-variable to range over the
    interval $[0,1]$ in $25$ steps and you also declare a $y$-variable
    to range over the same interval, you get a total of $625$ value
    pairs.

    The variable declarations specify which (input) variables will
    take which values. It is the job of the \emph{function
      declarations} to specify how some additional attributes are to
    be computed. The syntax of a function declaration is as follows:
    \begin{quote}
      |func |\declare{\meta{attribute}}| = |\meta{expression}|;|
    \end{quote}
    The meaning of such a declaration is the following: For each
    setting of the input variables (the variables specified using the
    |var| declaration), evaluate the \meta{expression} using the
    standard mathematical parser of \tikzname. The resulting value is
    then stored in |/data point/|\meta{attribute}.

    Inside \meta{expression} you can reference data point attributes
    using the following command, which is only defined inside such an
    expression:
    \begin{command}{\value\marg{variable}}
      This expands to the current value of the key |/data point/|\meta{variable}.
    \end{command}

    There can be multiple function declarations in a single data
    specification. In this case, all of these functions will be
    evaluated for each setting of input variables.

    \begin{codeexample}[]
\begin{tikzpicture}
  \datavisualization [school book plot,smooth]
    data [format=function] {
      var x : interval [-1.5:1.5];
      
      func y = \value x * \value x;
    };
\end{tikzpicture}
    \end{codeexample}
    \begin{codeexample}[]
\begin{tikzpicture}
  \datavisualization [school book plot, smooth,
                      all axes={ticks={few},unit length=3mm}]
    data [format=function] {
      var t : interval [0:2*pi];
      
      func x = \value t * cos(\value t r);
      func y = \value t * sin(\value t r);
    };
\end{tikzpicture}
    \end{codeexample}
  \end{dataformat}
  
\end{tikzlibrary}



\subsection{Advanced: The Data Parsing Process}

\label{section-dv-parsing}

Whenever data is fed to the data visualization system, it will be
handled by the |\pgfdata| command, declared in the |datavisualization|
module. The command is both used to parse data stored in external
sources (that is, in external files or which is produced on the fly by
calling an external command) as well as data given inline. A data
format does not need to know whether data comes from a file or is
given inline, the |\pgfdata| command will take care of this.

Since \TeX\ will always read files in a line-wise fashion, data is
alwyas fed to data format parsers in such a fashion. Thus, even it
would make more sense for a format to ignore line-breaks, the parser
must still handle data given line-by-line.

Let us now have a look at how |\pgfdata| works.

\begin{command}{\pgfdata\opt{\oarg{options}\marg{inline data}}}
  This command is used to feed data to the visualization
  pipeline. This command can only be used when a data visualization
  object has been properly setup, see
  Section~\ref{section-dv-main-setup}.

  \medskip
  \textbf{Basic options.}
  The |\pgfdata| command may be followed by \meta{options}, which are
  executed with the path |/pgf/data/|. Depending
  on these options, the \meta{options} may either be followed by
  \meta{inline data} or, alternatively, no \meta{inline data} is
  present and the data is read from an external source.
  
  The first important option is \meta{source}, which governs which of these
  two alternatives applies: 
  \begin{key}{/pgf/data/source=\meta{filename} (initially \normalfont empty)}
    If you set the |source| attribute to a non-empty \meta{filename},
    the data will be read from this file. In this case, no
    \meta{inline data} may be present, not even empty curly braces
    should be provided. If |source| is empty, the  data must directly
    follow as \meta{inline data}.
\begin{codeexample}[code only]
% Data is read from two external files:
\pgfdata[format=table,source=file1.csv]
\pgfdata[format=table,source=file2.csv]
\end{codeexample}
\begin{codeexample}[code only]
% Data is given inline:
\pgfdata[format=table]
{
  x, y
  1, 2
  2, 3
}
\end{codeexample}
  \end{key}
  The second important key is |format|, which is used to specify the
  data format:
  \begin{key}{/pgf/data/format=\meta{format} (initially table)}
    Use this key to locally set the format used for parsing the
    data. The \meta{format} must be a format that has been previously
    declared using the |\pgfdeclaredataformat| command. See the
    reference section for a list of the predefined formats.
  \end{key}
  In case all your data is in a certain format, you may wish to
  generally set the above key somewhere at the beginning of your
  file. Alternatively, you can use the following style to setup the
  |format| key and possibly further keys concerning the data format:
  \begin{stylekey}{/pgf/every data}
    This style is executed by |\pgfdata| before the \meta{options} are
    parsed. 

    Note that the path of this key is just |/pgf/|, not
    |/pgf/data/|. Also note that \tikzname\ internally sets the value
    of this key up in such a way that the keys |/tikz/every data| and
    also |/tikz/data visualization/every data| are executed. The
    bottom line of this is that when using \tikzname, you should not
    set this key directly, set |/tikz/every data| instead.
  \end{stylekey}
  
  \medskip
  \textbf{Gathering of the data.}
  Once the data format and the source have been decided upon, the data
  is ``gathered.'' During this phase the data is not actually parsed
  in detail, but just gathered so that it can later be parsed during
  the visualization. There are two different ways in which the data is
  gathered:
  \begin{itemize}
  \item In case you have specified an external source, the data
    visualization object is told (by means of invoking the |add data|
    method) that it should (later) read data from  the file specified
    by the |source| key using the format specified 
    by the |format| key. The file is not read at this point, but only
    later during the actual visualization.
  \item Otherwise, namely when data is given inline, depending on
    which format is used, some catcodes get changed. This is necessary
    since \TeX's special characters are often not-so-special in a
    certain format.

    Independently of the format, the end-of-line character
    (carriage return) is made an active character.

    Finally, the \meta{inline data} is then read as a normal argument
    and the data visualization object is told that later on it should
    parse this data using the given format parser. Note that in this
    case the data visualization object must store the whole data
    internally. 
  \end{itemize}
  In both cases the ``data visualization object'' is the object stored
  in the |/pgf/data visualization/obj| key.

  \medskip
  \textbf{Parsing of the data.}
  During the actual data visualization, all code that has been added
  to the data visualization object by means of the |add data| method
  is executed several times. It is the job of this code to call the
  |\pgfdatapoint| method for all data points present in the data.

  When the |\pgfdata| method calls |add data|, the code that is passed
  to the data visualization object is just a call to internal macros
  of |\pgfdata|, which are able to parse the data stored in an
  external file or in the inlined data. Independently of where the
  data is stored, these macros always do the following:
  \begin{enumerate}
  \item The catcodes are setup according to what the data
    format requires.
  \item Format-specific startup code gets called, which can initialize
    internal 
    variables of the parsing process. (The catcode changes are not
    part of the startup code since in order to read inline data
    |\pgfdata| must be able to setup to temporarily setup the catcodes
    needed later on by the parsers, but since no reading is to be
    done, no startup code should be called at this point.)
  \item For each line of the data a format-specific code handler,
    which depends on the 
    data format, is called. This handler gets the current line as
    input and should call |\pgfdatapoint| once for each data point
    that is encoded by this line (a line might define multiple data
    points or none at all). Empty lines are handled by special
    format-specific code.
  \item At the end, format-specific end code is executed.    
  \end{enumerate}
  For an example of how this works, see the description of the
  |\pgfdeclaredataformat| command.
\end{command}


\subsection{Advanced: Defining New Formats}
\label{section-dv-declaring-formats}

In order to define a new data format you can use the following
command, which is basic layer command defined in the module
|datavisualization|:

\begin{command}{\pgfdeclaredataformat\marg{format name}\marg{catcode
      code}\marg{startup code}\marg{line arguments}\\\marg{line
      code}\marg{empty line code}\marg{end code}}
  This command defines a new data format called \meta{format name},
  which can subsequently be used in the |\pgfdata| command. (The
  \tikzname's |data| maps directly to |\pgfdata|, so the following
  applies to \tikzname\ as well.)

  As explained in the description of the |\pgfdata| command, when data
  is being parsed that is formatted according to \meta{format name},
  the following happens:
  \begin{enumerate}
  \item The \meta{catcode code} is executed. This code should just
    contain catcode changes. The \meta{catcode code} will also be
    executed when inline data is read.
  \item Next, the \meta{startup code} is executed.
  \item Next, for each non-empty line of the data, the line is passed
    to a macro whose argument list is given by \meta{line
      arguments} and whose body is given by \meta{line code}. The idea
    is that you can use \TeX's powerful pattern matching capabilities
    to parse the non-empty lines. See also the below example.
  \item Empty lines are not processed by the \meta{line code}, but
    rather by the \meta{empty line code}. Typically, empty lines can
    simply be ignored and in this case you can let this parameter be
    empty.
  \item At the end of the data, the \meta{end code} is executed.
  \end{enumerate}

  As an example, let us now define a simple data format for reading
  files formatted in the following manner: Each line should contain a
  coordinate pair as in |(1.2,3.2)|, so two numbers separated by a
  comma and surrounded by parentheses. To make things more
  interesting, suppose that the hash mark symbol can be used to
  indicate comments. Here is an example of some data given in this
  format:
\begin{codeexample}[code only]
# This is some data formatted according to the "coordinates" format
(0,0)
(0.5,0.25)
(1,1)
(1.5,2.25)
(2,4)
\end{codeexample}

  A format parser for this format could be defined as follows:
\begin{codeexample}[code only]
\pgfdeclaredataformat{coordinates}
% First comes the catcode argument. We turn the hash mark into a comment character.
{\catcode`\#=14\relax}
% Second comes the startup code. Since we do not need to setup things, we can leave
% it empty. Note that we could also set it to something like \begingroup, provided we
% put an \endgroup in the end code
{}
% Now comes the arguments for non-empty lines. Well, these should be of the form
% (#1,#2), so we specify that:
{(#1,#2)}
% Now we must do something with a line of this form. We store the #1 argument in
% /data point/x and #2 in /data point/y. Then we call \pgfdatapoint to create a data point.
{
  \pgfkeyssetvalue{/data point/x}{#1}
  \pgfkeyssetvalue{/data point/y}{#2}
  \pgfdatapoint
}
% We ignore empty lines:
{}
% And we also have no end-of-line code.
{}
\end{codeexample}
This format could now be used as follows:
\begin{codeexample}[code only]
\begin{tikzpicture}
  \datavisualization[school book plot,smooth]
  data [format=coordinates] {
    # This is some data formatted according
    # to the "coordinates" format
    (0,0)
    (0.5,0.25)
    (1,1)
    (1.5,2.25)
    (2,4)
  };
\end{tikzpicture}
\end{codeexample}
\end{command}


% Copyright 2010 by Till Tantau
%
% This file may be distributed and/or modified
%
% 1. under the LaTeX Project Public License and/or
% 2. under the GNU Free Documentation License.
%
% See the file doc/generic/pgf/licenses/LICENSE for more details.


\section{Axes}
\label{section-dv-axes}

\subsection{Overview}

To be written...


\subsection{Concepts}

\subsubsection{Axes}

\subsubsection{Mayor, Minor, and Subminor Ticks}

\subsubsection{Tick Placement Strategies}
\label{section-dv-concept-tick-placement-strategies}

Consider the following problem: The data visualization engine
determines that in a
plot the $x$-values vary between $17.4$ and $34.5$. In this case, we
certainly do not want, say, ten ticks at exactly ten evenly spaced
positions starting with $17.4$ and ending with $34.5$, because this
would yield ticks at positions like $32.6$. Ticks should be placed at
``nice'' positions like $20$, $25$, and $30$.

Determining which positions are ``nice'' is somewhat difficult. In the
above example, the positions $20$, $25$, and $30$ are certainly nice,
but only three ticks may be a bit few of them. Better might be the
tick positions $17.5$, $20$, $22.5$, through to $32.5$. However, users
might prefer even numbers over fractions like $2.5$ as the stepping.

A \emph{tick placement strategy} is a method of automatically deciding
which positions are \emph{good} for placing ticks. The data
visualization engine comes with a number of predefined strategies, see
Section~\ref{section-dv-tick-placement-strategies}, but you can also
define new ones yourself. When the data visualization is requested to
automatically determine
``good'' positions for the placement of ticks on an axis, it uses one
of several possible \emph{basic strategies}. These strategies differ
dramatically in which tick positions they will choose: For a range of
values between $5$ and $1000$, a |linear steps| strategy might place
ticks at positions $100$, $200$, through to $1000$, while an
|exponential steps| strategy would prefer the tick positions $10$,
$100$ and $1000$. The exact number and values of the tick positions
chosen by either strategy can be fine-tuned using additional options
like |step| or |about|.

Here is an example of the different stepping chosen when one varies
the tick placement strategy:

\begin{codeexample}[]
\begin{tikzpicture}
  \datavisualization [scientific axes, visualize as smooth line]
    data [format=function] {
      var x : interval [1:11];
      func y = \value x*\value x;
    };
\end{tikzpicture}
\qquad
\begin{tikzpicture}
  \datavisualization [scientific axes, visualize as smooth line,
    y axis={exponential steps},
    x axis={ticks={quarter about strategy}},
  ]
    data [format=function] {
      var x : interval [1:11];
      func y = \value x*\value x;
    };
\end{tikzpicture}
\end{codeexample}



\subsubsection{Grids}

\subsection{Usage}


\subsection{Reference: Standard Axis Systems}

In this section the axis system commonly used in data visualizations
are described.

\subsubsection{Scientific Axis Systems}

\begin{key}{/tikz/data visualization/scientific axes}
  This key installs a two-dimensional coordinate system based on the
  attributes |/data point/x| and |/data point/y|.

\begin{codeexample}[]
\begin{tikzpicture}
  \datavisualization [scientific axes, visualize as smooth line]
    data [format=function] {
      var x : interval [0:100];
      func y = sqrt(\value x);
    };
\end{tikzpicture}
\end{codeexample}

  This axis system is usually a good choice to depict ``arbitrary two
  dimensional data.'' Because the axes are automatically scaled, you
  do not need to worry about how large or small the values will
  be. The name |scientific axes| is intended to indicate that this
  axis system is often used in scientific publications.

  Note, however, that this axis system will always distort the
  relative magnitudes of the units on the two axis. If you wish the
  units on both axes to be equal, consider directly specifying the
  unit length ``by hand'':

\begin{codeexample}[]
\begin{tikzpicture}
  \datavisualization [visualize as smooth line,
                      scientific axes,
                      all axes={unit length=1cm per 10 units, ticks={few}}]
    data [format=function] {
      var x : interval [0:100];
      func y = sqrt(\value x);
    };
\end{tikzpicture}
\end{codeexample}

  The |scientific axes| have the following properties:
  \begin{itemize}
  \item The |x|-values are surveyed and the $x$-axis is then scaled
    and shifted so  that it has the length specified by the following key.
    \begin{key}{/tikz/data visualization/scientific
        axes/width=\meta{dimension} (initially 5cm)}
    \end{key}
    The minimum value is at the left end of the axis and at the canvas
    origin. The maximum value is at the right end of the axis.
  \item The |y|-values are surveyed and the $y$-axis is then scaled so
    that is has the length specified by the following key.
    \begin{key}{/tikz/data visualization/scientific
        axes/height=\meta{dimension}}
      By default, the |height| is the golden ratio times the |width|.
    \end{key}
    The minimum value is at the bottom of the axis and at the canvas
    origin. The maximum value is at the top of the axis.
  \item Lines (forming a frame) are depicted at the minimum and
    maximum values of the axes in 50\% black.
  \item Ticks are drawn `` on the outside'' of the frame so that they
    interfere as little as possible with the data.
  \item Tick labels and axis labels (if present) are drawn left and
    below.
  \end{itemize}
\end{key}

\begin{key}{/tikz/data visualization/scientific inner axes}
  This axis system works like |scientific axes|, only the ticks are on
  the ``inside'' of the frame.

\begin{codeexample}[]
\begin{tikzpicture}
  \datavisualization [scientific inner axes, visualize as smooth line]
    data [format=function] {
      var x : interval [-12:12];
      func y = \value x*\value x*\value x;
    };
\end{tikzpicture}
\end{codeexample}

  This axis system is also common in publications, but the ticks tend
  to interfere with marks if they are near to the border as can be
  seen in the following example:
\begin{codeexample}[]
\begin{tikzpicture}
  \datavisualization [scientific inner axes, scientific axes/width=3.2cm,
                      visualize as scatter/.list={a,b}]
    data [a] {
      x, y
      0, 0
      1, 1
      0.5, 0.5
      2, 1
    }
    data [b] {
      x, y
      0.05, 0
      1.5, 1
      0.5, 0.75
      2, 0.5
    };
\end{tikzpicture}
\end{codeexample}

\end{key}

\begin{key}{/tikz/data visualization/scientific clean axes}
  This axis system is another version of |scientific axes|. However, the
  axes and the ticks are completely removed from the actual data,
  making this axis system especially useful for scatter plots, but
  also for most other scientific plots.

\begin{codeexample}[]
\begin{tikzpicture}
  \datavisualization [scientific clean axes, visualize as smooth line]
    data [format=function] {
      var x : interval [-12:12];
      func y = \value x*\value x*\value x;
    };
\end{tikzpicture}
\end{codeexample}

  The distance of the axes from the actual plot is given by the
  padding of the axes.
\end{key}


For all scientific axis systems, different label placement strategies
can be specified. They are discussed in the following.


\begin{key}{/tikz/data visualization/scientific axes standard labels}
  As the name suggests, this is the standard placement strategy. The
  label of the $x$-axis is placed below the center of the $x$-axis,
  the label of the $y$-axis is rotated by $90^\circ$ and placed left
  of the center of the $y$-axis.
\begin{codeexample}[]
\begin{tikzpicture}
  \datavisualization [scientific clean axes,
                      visualize as smooth line,
                      scientific axes standard labels,
                      x axis={label=degree $d$, ticks={tick unit=${}^\circ$}},
                      y axis={label=$\sin d$}]
    data [format=function] {
      var x : interval [-10:10] samples 10;
      func y = sin(\value x);
    };
\end{tikzpicture}
\end{codeexample}
\end{key}

\begin{key}{/tikz/data visualization/scientific axes upright labels}
  Works like |scientific axes standard labels|, only the label of the
  $y$-axis is not rotated.
\begin{codeexample}[]
\begin{tikzpicture}
  \datavisualization [scientific clean axes,
                      visualize as smooth line,
                      scientific axes upright labels,
                      x axis={label=degree $d$, ticks={tick unit=${}^\circ$}},
                      y axis={label=$\cos d$,
                              ticks={style={/pgf/number format/.cd,precision=4,fixed zerofill}}}]
    data [format=function] {
      var x : interval [-10:10] samples 10;
      func y = cos(\value x);
    };
\end{tikzpicture}
\end{codeexample}
\end{key}


\begin{key}{/tikz/data visualization/scientific axes end labels}
  Places the labels at the end of the $x$- and the $y$-axis, similar
  to the axis labels of a school book axis system.
\begin{codeexample}[]
\begin{tikzpicture}
  \datavisualization [scientific clean axes,
                      visualize as smooth line,
                      scientific axes end labels,
                      x axis={label=degree $d$, ticks={tick unit=${}^\circ$}},
                      y axis={label=$\tan d$}]
    data [format=function] {
      var x : interval [-10:10] samples 10;
      func y = tan(\value x);
    };
\end{tikzpicture}
\end{codeexample}
\end{key}





\subsubsection{School Book Axis Systems}

\begin{key}{/tikz/data visualization/school book axes}
  This axis system is intended to ``look like'' the coordinate systems
  often used in school books: The axes are drawn in such a way that
  they intersect to origin. Furthermore, no automatic
  scaling is done to ensure that the lengths of units are the same in
  all directions.

  This axis system must be used with care -- it is nearly always
  necessary to specify the desired unit length by hand using the
  option |unit length|. If the magnitudes of the units on the two axes
  differ, different unit lengths typically need to be specified for
  the different axes.

  Finally, if the data is ``far removed'' from the origin, this
  axis system will also ``look bad.''

\begin{codeexample}[]
\begin{tikzpicture}
  \datavisualization [school book axes, visualize as smooth line]
    data [format=function] {
      var x : interval [-1.3:1.3];
      func y = \value x*\value x*\value x;
    };
\end{tikzpicture}
\end{codeexample}

  The stepping of the ticks is one unit by default. Using keys like
  |ticks=some| may help to give better steppings.
\end{key}


\begin{key}{/tikz/data visualization/school book axes standard labels}
  This key makes the label of the $x$-axis appear at the right end of
  this axis and it makes the label of the $y$-axis appear at the top
  of the $y$-axis.

  Currently, this is the only supported placement strategy for the
  school book axis system.
\begin{codeexample}[]
\begin{tikzpicture}
  \datavisualization [school book axes,
                      visualize as smooth line,
                      school book axes standard labels,
                      clean ticks,
                      x axis={label=$x$},
                      y axis={label=$f(x)$}]
    data [format=function] {
      var x : interval [-1:1];
      func y = \value x*\value x + 1;
    };
\end{tikzpicture}
\end{codeexample}
\end{key}





\subsubsection{Advanced: Underlying Cartesian Axis Systems}

The axis systems described in the following are typically not used
directly by the user. The systems setup \emph{directions} for several
axes in some sensible way, but they do not actually draw anything on
these axes. For instance, the |xy Cartesian| creates two axes called
|x axis| and |y axis| and makes the $x$-axis point right and the
$y$-axis point up. In contrast, an axis system like |scientific axes|
uses the axis system |xy Cartesian| internally and then proceeds to
setup a lot of keys so that the axis lines are drawn,
ticks and grid lines are drawn, and labels are placed at the correct
positions.

\begin{key}{/tikz/data visualization/xy Cartesian}
  This axis system creates two axes called |x axis| and |y axis| that
  point right and up, respectively. By default, one unit is mapped to
  one cm.

\begin{codeexample}[]
\begin{tikzpicture}
  \datavisualization [xy Cartesian, visualize as smooth line]
    data [format=function] {
      var x : interval [-1.25:1.25];
      func y = \value x*\value x*\value x;
    };
\end{tikzpicture}
\end{codeexample}


  \begin{key}{/tikz/data visualization/xy axes=\meta{options}}
    This key applies the \meta{options} both to the |x axis| and the
    |y axis|.
  \end{key}

\end{key}


\begin{key}{/tikz/data visualization/xyz Cartesian cabinet}
  This axis system works like |xy Cartesian|, only it
  \emph{additionally} creates an axis called |z axis| that points left
  and down. For this axis, one unit corresponds to $\frac{1}{2}\sin
  45^\circ\mathrm{cm}$. This is also known as a cabinet projection.

  \begin{key}{/tikz/data visualization/xyz axes=\meta{options}}
    This key applies the \meta{options} both to the |x axis| and the
    |y axis|.
  \end{key}

\end{key}


\begin{key}{/tikz/data visualization/uv Cartesian}
  This axis system works like |xy Cartesian|, but it introduces two
  axes called |u axis| and |v axis| rather than the |x axis| and the
  |y axis|. The idea is that in addition to a ``major''
  $xy$-coordinate system this is also a ``smaller'' or ``minor''
  coordinate system in use for depicting, say, small vectors with
  respect to this second coordinate system.

  \begin{key}{/tikz/data visualization/uv axes=\meta{options}}
    Applies the \meta{options} to both the |u axis| and the |y axis|.
  \end{key}

\end{key}

\begin{key}{/tikz/data visualization/uvw Cartesian cabinet}
  Like |xyz Cartesian cabinet|, but for the $uvw$-system.

  \begin{key}{/tikz/data visualization/uvw axes=\meta{options}}
    Like |xyz axes|.
  \end{key}
\end{key}



\subsection{Reference: Tick Placement Strategies}
\label{section-dv-tick-placement-strategies}

As described in \ref{section-dv-concept-tick-placement-strategies},
it is not a trivial task for the data visualization engine to
correctly automatically determine good positions for the placement of
ticks on axes. When the values on an axis range between, say, $17.4$
and $34.5$, it is somewhat unclear where ticks should be placed.


\subsubsection{Predefined Strategies}

The following strategies are always available:

\begin{key}{/tikz/data visualization/axis options/linear steps}
  This strategy placed ticks at positions that are evenly spaced by
  the current value of |step|.

  In detail, the following happens: Let $a$ be the minimum value of the
  data values along the axis and let $b$ be the maximum. Let the
  current \emph{stepping} be $s$ (the stepping is set using the |step|
  option, see below) and let the current \emph{phasing} be $p$ (set
  using the |phase|) option. Then ticks are placed all positions
  $i\cdot s + p$ that lie in the interval $[a,b]$, where $i$ ranges
  over all integers.

  The tick positions computed in the way described above are
  \emph{mayor} step positions. In addition to these, if the key
  |minor steps between steps| is set to some number $n$, then $n$ many
  minor ticks are introduced between each two mayor ticks (and also
  before and after the last mayor tick, provided the values still lie
  in the interval $[a,b]$). Note that is $n$ is $1$, then one minor tick
  will be added in the middle between any two mayor ticks. Use a value
  of $9$ (not $10$) to partition the interval between two mayor ticks
  into ten equally sized minor intervals.

\begin{codeexample}[]
\begin{tikzpicture}
  \datavisualization
    [scientific inner axes, scientific axes/width=3cm,
     x axis={ticks={step=3, minor steps between steps=2}},
     y axis={ticks={step=.36}},
     visualize as scatter]
    data {
      x, y
      17, 30
      34, 32
    };
\end{tikzpicture}
\end{codeexample}
\end{key}

\begin{key}{/tikz/data visualization/axis options/exponential steps}
  This strategy produces ticks at positions that are appropriate for
  logarithmic plots. It is automatically selected when you use the
  |logarithmic| option with an axis.

  In detail, the following happens: As for |linear steps| let numbers
  $a$, $b$, $s$, and $p$ be given. Then, mayor ticks are placed at all
  positions $10^{i\cdot s+p}$ that lie in the interval $[a,b]$ for $i
  \in \mathbb Z$.

  The minor steps are added in the same way as for |linear steps|. In
  particular, they interpolate \emph{linearly} between mayor steps.

\begin{codeexample}[]
\begin{tikzpicture}
  \datavisualization
    [scientific axes,
     x axis={logarithmic, length=2cm, ticks={step=1.5}},
     y axis={logarithmic, ticks={step=1, minor steps between steps=9}},
     visualize as scatter]
    data {
      x, y
      1, 10
      1000, 1000000
    };
\end{tikzpicture}
\end{codeexample}
\end{key}




\subsubsection{Choosing a Stepping Explicitly}

The following options are used to configure tick placement strategies
like |linear steps|. Unlike the basic choice of a placement strategy,
which is an axis option, the following should be passed to the option
|ticks| or |grid| only.. So, you would write
things like |x axis={ticks={step=2}}|, but |x axis=linear steps|.

\begin{key}{/tikz/data visualization/step=\meta{value} (initially 1)}
  The value of this key is used to determine the spacing of the major
  ticks. The key is used by the |linear steps| and |exponential steps|
  strategies described above, see the explanations there.
\end{key}

\begin{key}{/tikz/data visualization/minor steps between
    steps=\meta{number} (default 9)}
  As for |step|, see the explanation of |linear steps|.
\end{key}

\begin{key}{/tikz/data visualization/phase=\meta{value} (initially 0)}
  As for |step|, see the explanation of |linear steps|.
\end{key}



\subsubsection{Choosing a Stepping Automatically}

The |step| option gives you ``total control'' over the stepping of
ticks on an axis, but you often do not know the correct stepping in
advance. In this case, you may prefer to have a good value for |step|
being computed for you automatically.

Like the |step| key, these options are passed to the |ticks|
option. So, for instance, you would write |x axis={ticks={about=4}}|
to request about four ticks to be placed on the $x$-axis.


\begin{key}{/tikz/data visualization/about=\meta{number}}
  This key asks the data visualization to place \emph{about}
  \meta{number} many ticks on an axis. It is not guaranteed that
  \emph{exactly} \meta{number} many ticks will be used, rather the
  actual number will be the closest number of ticks to \meta{number}
  so that their stepping is still ``good''. For instance, when you say
  |about=10|, it may happen that exactly |10|, but perhaps even |13| ticks are
  actually selected, provided that these numbers of ticks lead to good
  stepping values like |5| or |2.5| rather than numbers like |3.4| or
  |7|. The method that is used to determine which steppings a deemed to
  be ``good'' depends on the current tick placement strategy.

  \medskip
  \textbf{Linear steps.}
  Let us start with |linear steps|: First, the difference between the
  maximum value $v_{\max}$ and the minimum value $v_{\min}$ on the
  axis is computed; let us call it $r$ for ``range.'' Then, $r$ is
  divided by \meta{number},
  yielding a target stepping~$s$. If $s$ is a number like $1$ or $5$
  or $10$, then this number could be used directly as the new value of
  |step|. However, $s$ will typically something strange like $0.02345$
  or $345223.76$, so $s$ must be replaced by a better value like $0.02$
  in the first case and perhaps $250000$ in the second case.

  In order to determine which number is to be used, $s$ is rewritten
  in the form $m \cdot 10^k$ with $1 \le m < 10$ and $k \in \mathbb
  Z$. For instance, $0.02345$ would be rewritten as $2.345 \cdot
  10^{-2}$ and $345223.76$ as $3.4522376 \cdot 10^5$. The next step
  is to replace the still not-so-good number $m$ like $2.345$ or
  $3.452237$ by a ``good'' value $m'$. For this, the current value of
  the |about strategy| is used:
  \begin{key}{/tikz/data visualization/about strategy=\meta{list}}
    The \meta{list} is a comma-separated sequence of pairs
    \meta{threshold}/\meta{value} like for instance |1.5/1.0| or
    |2.3/2.0|. When a good value $m'$ is sought for a given $m$, we
    iterate over the list and find the first pair
    \meta{threshold}/\meta{value} where \meta{threshold}
    exceeds~$m$. Then $m'$ is set to \meta{value}. For instance, if
    \meta{list} is |1.5/1.0,2.3/2.0,4/2.5,7/5,11/10|, which is the
    default, then for $m=3.141$ we would get $m'=2.5$ since $4 >
    3.141$, but $2.3 \le 3.141$. For $m=6.3$ we would get $m'=5$.
  \end{key}
  Once $m'$ has been determined, the stepping is set to $s' = m'
  \cdot 10^k$.

  % Define an axis type
  \tikzdatavisualizationset{
    one dimensional axis/.style={
      new Cartesian axis=axis,
      axis={
        attribute=main,
        unit vector={(0pt,1pt)},
        visualize axis={style=->},
        visualize ticks={major={tick text at low},direction axis=perpendicular},
        length=3cm
      },
      new Cartesian axis=perpendicular,
      perpendicular={
        attribute=perp,
        unit vector={(1pt,0pt)},
        include values=0,
        include values=1
      }
    }
  }

  \def\showstrategy#1{

  % Show the effect for the different strategies
    \medskip
    \begin{tikzpicture}
      \foreach \max/\about [count=\c] in {10/5,20/5,30/5,40/5,50/5,60/5,70/5,80/5,90/5,100/5,100/3,100/10}
      {
        \begin{scope}[xshift=\c pt*30]
          \datavisualization [#1,
          one dimensional axis,
          axis={
            ticks={about=\about},
            include values=0,
            include values=\max
          }
          ];

          \node at (0,-5mm) [anchor=mid] {\texttt{\about}};
        \end{scope}
      }

      \node at (30pt,-5mm) [anchor=mid east] {\texttt{about=\ \ }};
  \end{tikzpicture}
}

  The net effect of all this is that for the default strategy, the
  only valid stepping are the values $1$, $2$, $2.5$ and $5$ and every
  value obtainable by multiplying one of these values by a power of
  ten. The following example shows the effects of, first, setting
  |about=5| (corresponding to the |some| option) and then having axes
  where the minimum value is always |0| and where the maximum value
  ranges from |10| to |100| and, second, setting |about| to the values
  from |3| (corresponding to the |few| option) and to |10|
  (corresponding to the |many| option) while having the
  minimum at |0| and the maximum at |100|:

  \showstrategy{standard about strategy}

  \medskip
  \textbf{Exponential steps.}
  For |exponential steps| the strategy for determining a good stepping
  value is similar to |linear steps|, but with the following
  differences:
  \begin{itemize}
  \item Naturally, since the stepping value refers to the exponent,
    the whole computation of a good stepping value needs to be done
    ``in the exponent.'' Mathematically spoken, instead of considering
    the difference $r = v_{\max} - v_{\min}$, we consider the difference $r =
    \log v_{\max} - \log v_{\min}$. With this difference, we still
    compute $s = r / \meta{number}$ and let $s = m \cdot 10^k$ with $1
    \le m < 10$.
  \item It makes no longer sense to use values like $2.5$ for $m'$
    since this would yield a fractional exponent. Indeed, the only
    sensible values for $m'$ seem to be $1$, $3$, $6$, and
    $10$. Because of this, the |about strategy| is ignored and one of
    these values or a multiple of one of them by a power of ten is
    used.
  \end{itemize}

  The following example shows the chosen steppings for a maximum
  varying from $10^1$ to $10^5$ and from $10^{10}$ to $10^{50}$ as
  well as for $10^{100}$ for |about=3|:

  \medskip
  \begin{tikzpicture}
    \foreach \max [count=\c] in {1,...,5,10,20,...,50,100}
      {
        \begin{scope}[xshift=\c pt*40]
          \datavisualization [
          one dimensional axis,
          axis={
            logarithmic,
            ticks={about=3},
            include values=1,
            include values=1e\max
          }
          ];
        \end{scope}
      }
  \end{tikzpicture}


  \medskip
  \textbf{Alternative strategies.}

  In addition to the standard |about strategy|, there are some
  additional strategies that you might wish to use instead:

  \begin{key}{/tikz/data visualization/standard about
      strategy}
    Permissible values for $m'$ are: $1$, $2$, $2.5$, and~$5$. This
    strategy is the default strategy.
  \end{key}

  \begin{key}{/tikz/data visualization/euro about strategy}
    Permissible values for $m'$ are: $1$, $2$, and~$5$. These are the
    same values as for the Euro coins, hence the
    name.

    \showstrategy{euro about strategy}
  \end{key}

  \begin{key}{/tikz/data visualization/half about strategy}
    Permissible values for $m'$: $1$ and $5$. Use this
    strategy if only powers of $10$ or halves thereof seem logical.

    \showstrategy{half about strategy}
  \end{key}

  \begin{key}{/tikz/data visualization/quarter about strategy}
    Permissible values for $m'$ are: $1$, $2.5$, and $5$.

    \showstrategy{quarter about strategy}
  \end{key}

  \begin{key}{/tikz/data visualization/int about strategy}
    Permissible values for $m'$ are: $1$, $2$, $3$, $4$, and $5$.

    \showstrategy{int about strategy}
  \end{key}
\end{key}

\begin{key}{/tikz/data visualization/many}
  This is an abbreviation for |about=10|.
\end{key}

\begin{key}{/tikz/data visualization/some}
  This is an abbreviation for |about=5|.
\end{key}

\begin{key}{/tikz/data visualization/few}
  This is an abbreviation for |about=3|.
\end{key}

\begin{key}{/tikz/data visualization/none}
  Switches off the automatic step computation. Unless you use |step=|
  explicitly to set a stepping, no ticks will be (automatically)
  added.
\end{key}




\subsubsection{Advanced: Defining New Placing Strategies}

\begin{key}{/tikz/data visualization/axis options/tick placement strategy=\meta{macro}}
  This key can be used to install a so-called \emph{tick placement
    strategy}. Whenever |visualize ticks| is used to request some
  ticks to be visualized, it is checked whether some automatic ticks
  should be created. This is the case when the following key is set:
  \begin{key}{/tikz/data visualization/compute step=\meta{code}}
    The \meta{code} should compute a suitable value for the stepping
    to be used by the \meta{macro} in the tick placement strategy.

    For instance, the |step| key sets |compute step| to
    |\def\tikz@lib@dv@step{#1}|. Thus, when you say |step=5|, then the
    desired stepping of |5| is communicated to the \meta{macro} via the
    macro |\tikz@lib@dv@step|.
  \end{key}

  Provided |compute step| is set to some nonempty value, upon
  visualization of ticks the \meta{macro} is executed. Typically,
  \meta{macro} will first call the \meta{code} stored in the key
  |compute step|. Then, it should implement some strategy then uses
  the value of the computed or desired stepping to create appropriate
  |at| commands. To be precise, it should set the keys |major|,
  |minor|, and/or |subminor| with some appropriate |at| values.

  Inside the call of \meta{macro}, the macro |\tikzdvaxis| will have
  been set to the name of the axis for which default ticks need to be
  computed. This allows you to access the minimum and the maximum
  value stored in the |scaling mapper| of that axis.
  \begin{codeexample}[]
\def\silly{
  \tikzdatavisualizationset{major={at={2,3,5,7,11,13}}}
}
\begin{tikzpicture}
  \datavisualization [
    scientific axes, visualize as scatter,
    x axis={tick placement strategy=\silly}
    ]
    data {
      x, y
      0, 0
      15, 15
    };
\end{tikzpicture}
\end{codeexample}
\end{key}

\subsection{Advanced: Creating New Axes}

\subsection{Advanced: Creating New Axis Systems}

% Copyright 2010 by Till Tantau
%
% This file may be distributed and/or modified
%
% 1. under the LaTeX Project Public License and/or
% 2. under the GNU Free Documentation License.
%
% See the file doc/generic/pgf/licenses/LICENSE for more details.


\section{Visualizers}
\label{section-dv-visualizers}

\subsection{Overview}

In a data visualization a long stream of data points is
\emph{visualized} using \emph{visualizers}. Recall that it is the job
of the axis systems as described in Section~\ref{section-dv-axes} to
determine \emph{where} data points are visualized. It is the job of
the visualizers to determine \emph{how} they are visualized.

The most basic and common visualizer is the \emph{line visualizer}. It
simply connects subsequent data points by straight lines to indicate
either that the points on these lines interpolate between the real
data points or the straight lines are used to indicate the order in
which the data points appear. A different, more ``conservative''
visualizer is the \emph{scatter visualizer} or \emph{mark visualizer},
which just places a small mark at each data point. Such a visualizer
does not imply any interpolation or ordering between the data points.

Visualizers may, however, also be more complicated. For instance, a
visualizer used for a box plot could visualize a data point as a box
with a median value, standard deviation, outliers, and other
information; a rectangle visualizer might visualize data points as
larger areas; a projection visualizer might visualize the projection
of data points onto different axes; and so.

Creating a new visualizer is not quite trivial since a new \pgfname\
class needs to be implemented. Fortunately, using visualizers is much
simpler: For each kind of visualizer there is a key that allows you to
create such a visualizer. You can then use further keys to configure
the visualizer and to connect it to the data.

In a data visualization multiple visualizers may exist at the same
time. This happens in different situations:
\begin{itemize}
\item A data visualization may contain several independent data sets
  that are to be visualized. There might be a line plot, for which a
  line visualizer is used, and also a scatter plot, for which a
  scatter visualizer would be used.

  In this case, for each data point only one visualizer will do
  anything. To achieve this, each data point has an attribute called
  |visualizer| which tells the visualizer objects whether they should
  ``react'' to the data point or not.
\item A single data point might be visualized several times. For
  instance, a scatter visualizer might draw a mark at the data point's
  position on the page and a projection visualizer might draw,
  additionally, a mark at the projected position.
\end{itemize}


\subsection{Usage}

\subsubsection{Using a Single Visualizer}

The simplest scenario for using visualizers are data visualizations in
which there is only a single data set that is visualized in one
style. In this case, all that needs to be done in order to choose a
visualizer is use one of the options starting with |visualize as ...|
together with the |\datavisualization| command:

\begin{codeexample}[]
% Define a data set:  
\tikz \datavisualization data set {example} = {
data {
  x, y
  0, 0
  0.5, 2
  1, 2
  1.5, 1.5
  2, 0.5
}};
\tikz \datavisualization [school book axes, visualize as line]        data set {example};
\qquad 
\tikz \datavisualization [school book axes, visualize as smooth line] data set {example};
\qquad 
\tikz \datavisualization [school book axes, visualize as scatter]     data set {example};
\end{codeexample}

Methods for styling visualizers are discussed in Section~\ref{section-dv-visualizer-styling}.


\subsubsection{Using Multiple Visualizers}

A data visualization may contain multiple data sets and for each data
set we might wish to use a different visualizer. In this case, we need
some way of telling the data visualization engine to which visualizer
should be used with the different data points.

To solve this problem, you can \emph{name} a visualizer. The
visualizer's name can then both be used to configure the visualizer
and also to indicate that data points ``belong'' to the visualizer.

Naming a visualizer is quite simple: The |visualize as ...| keys
actually take a single parameter, which is the name of the
visualizer. For instance, the following code creates three
visualizers, named |sin|, |cos|, and |tan|:

\begin{codeexample}[code only]
visualize as line=sin,
visualize as line=cos,
visualize as scatter=tan
\end{codeexample}

(When you just say |visualize as line| without providing a name, the
name |line| is chosen as a default, for |visualize as scatter| the
name |scatter| is the default and so.)

In order to indicate which data points should be visualized by which
of these visualizers, the following key is important:

\begin{key}{/data point/visualizer}
  A visualizer will only act on a data point when its name matches the
  value of this key. Initially, it is set to the last visualizer
  created, so if there is only one, there is no need to set or worry
  about this key.
\end{key}

Since the |visualizer| key has the path prefix |/data point|, it can
be set like any other attribute of a data key:

\begin{codeexample}[width=7cm]
\tikz \datavisualization
 [scientific clean axes,
  visualize as line=sin,
  visualize as line=cos,
  visualize as scatter=tan]
data {
  x, y, visualizer
  0, 0, sin
  1, 1, sin
  2, 0, sin
  3, -1, sin
  4, 0, sin
  0, 1, cos
  1, 0, cos
  0, 0, tan
  1, 1, tan
  2, 2, tan
  3, 4, tan
  2, -1, cos
  3, 0, cos
  4, 1, cos
};
\end{codeexample}

As can be seen, the data points with the same |visualizer| attribute
do not need to be consecutive.

The above method of specifying the visualizer works nicely, but in
most cases it would be more natural to keep the |visualizer| attribute
out of the table. This is easy to achieve by using multiple |data|
blocks and by simply adding the options like
|[/data point/visualizer=sin]| to the |data| blocks:

\begin{codeexample}[width=7cm]
\tikz \datavisualization
 [scientific clean axes,
  visualize as line=sin,
  visualize as line=cos]
data [/data point/visualizer=sin] {
  x, y
  0, 0
  1, 1
  2, 0
  3, -1
  4, 0
}
data [/data point/visualizer=cos] {
  x, y
  0, 1
  1, 0
  2, -1
  3, 0
  4, 1
};
\end{codeexample}

It turns out that there is an even simpler way of achieving the above:
You can simply pass the name of the visualizer as an option to the
|data| block command and it will setup the |visualizer| key
correctly. This works because when a visualizer is named using
|visualize as ...| in addition to creating a visualizer object, the
following key is also setup: 
\begin{key}{/pgf/data/\meta{visualizer name}}
  This key is a shorthand for
  |/data point/visualizer=|\meta{visualizer name}.  
\end{key}

\begin{codeexample}[width=7cm]
\tikz \datavisualization
 [scientific clean axes,
  visualize as line=sin,
  visualize as line=cos]
data [sin] {
  x, y
  0, 0
  1, 1
  2, 0
  3, -1
  4, 0
}
data [cos] {
  x, y
  0, 1
  1, 0
  2, -1
  3, 0
  4, 1
};
\end{codeexample}

When you need to visualize several similar things in a single plot
(like ten lines that all get visualized by |visualize as line|), it is
somewhat cumbersome having to write this ten times. In this case you
can shorten your code by making use of the |.list| key handler: When
you add it to a key, the ``value'' passed to the key is parsed as a
list of values. The key is then executed once for each of these
values:

\begin{codeexample}[width=7cm]
\tikz \datavisualization
 [scientific clean axes,
  visualize as line/.list={sin, cos, tan}]
data [sin, format=function] {
  var x : interval[0:3*pi];
  func y = sin(\value x r);
}
data [cos, format=function] {
  var x : interval[0:3*pi];
  func y = cos(\value x r);
}
data [tan, format=function] {
  var x : interval[0:pi/2.2];
  func y = tan(\value x r);
};
\end{codeexample}



\subsubsection{Styling a Visualizer}
\label{section-dv-visualizer-styling}

In order to style a visualizer that has been created using for
instance |visualize as line=|\meta{visualizer name}, you can use the
following key: 

\begin{key}{/tikz/data visualization/\meta{visualizer
      name}=\meta{options}}
  For each visualizer, a key of the same name is created with the path
  prefix |/tikz/data visualization|. This key takes the \meta{options}
  and executes them with the path prefix
\begin{codeexample}[code only]
/tikz/data visualization/visualizer options/  
\end{codeexample}
  These options are then used to configure the appearance of the
  current visualizer. (This is quite similar to the way options are
  passed to an axis in order to configure the axis.)
  Possible options include |style|, but also |label in legend| and
  |label in data|. The latter two options are discussed in
  Section~\ref{section-dv-labels-in}, the first option below.

\begin{codeexample}[width=7cm]
\tikz \datavisualization
 [scientific clean axes,
  visualize as smooth line/.list={sin, cos},
  sin={style=red},
  cos={style=blue}]
data [sin, format=function] {
  var x : interval[0:3*pi];
  func y = sin(\value x r);
}
data [cos, format=function] {
  var x : interval[0:3*pi];
  func y = cos(\value x r);
};
\end{codeexample}
  
  (Note that the key |/pgf/data/|\meta{visualizer name} is also
  created for each visualizer, but the purpose of that key is to
  change the |/data point/visualizer| attribute, whereas the purpose
  of the present key is to configure the visualizer.)
\end{key}

\begin{key}{/tikz/data visualization/visualizer
    options/style=\meta{options}}
  The \meta{options} given to this key should be normal \tikzname\
  options. They will be executed when the visualizer is used.

\begin{codeexample}[width=7cm]
\tikz \datavisualization
 [scientific clean axes,
  visualize as smooth line=sin,
  sin={style={red, densely dotted}},
  visualize as smooth line=cos,
  cos={style={mark=x}},
]
data [sin, format=function] {
  var x : interval[0:3*pi];
  func y = sin(\value x r);
}
data [cos, format=function] {
  var x : interval[0:3*pi];
  func y = cos(\value x r);
};
\end{codeexample}
\end{key}

In addition to the options passed to a visualizer via |style|, the
following also gets executed when a visualizer is used:

\begin{stylekey}{/tikz/data visualization/every visualizer}
  This style is used with every visualizer. Note that it should
  contain normal \tikzname\ keys.

\begin{codeexample}[width=7cm]
\tikz \datavisualization
 [scientific clean axes,
  every visualizer/.style={dashed},
  visualize as smooth line]
data [format=function] {
  var x : interval[0:3*pi];
  func y = sin(\value x r);
};
\end{codeexample}
\end{stylekey}


\subsection{Reference: Basic Visualizers}

\begin{key}{/tikz/data visualizers/visualize as line=\meta{visualizer
      name} (default line)}
  Creates a new visualizer named \meta{visualizer name}. Basically, 
  this visualizer connects all data points for which the
  |/data point/visualizer| attribute equals \meta{visualizer name} by
  a line that is styled by the visualizer's style.

  In more detail, the following happens:
  \begin{enumerate}
  \item A new object is created (of class |plot handler visualizer|)
    that is configured to collect the canvas positions of all data
    points whose |visualizer| attribute equals \meta{visualizer name}.
  \item During the end of the data visualization, \pgfname's plotting
    mechanism (see Section~\ref{section-plots}) is used to plot the
    stream of recorded data points.

    This means that, in principle, all of the plot handlers available
    in \tikzname\ could be used for the visualization (such as the
    |smooth| handler). However, some plot handlers such as, say, the
    |xcomb| are unsuitable as plot handlers since they do not support
    the advanced axis handling done by the data visualization
    engine. Because of this (and also for other reasons), you cannot
    set the plot handler directly, but must use one of the options
    described in a moment. 
  \item Additionally, plot marks can be drawn at the collected data
    points. Here, all of the options available to \tikzname\ for
    drawing plot marks are available.
  \end{enumerate}
  
  The following keys can be used for changing the plot handler (the
  way the ``line'' is rendered).
  
  \begin{key}{/tikz/data visualization/visualizer options/straight line}
    Causes the data points to be connected by straight lines.
\begin{codeexample}[]
\tikz [scale=.55] \datavisualization
 [scientific clean axes, all axes={ticks=few},
  visualize as smooth line=my data,  my data={straight line}]
data [format=function] {
  var t : interval [0:4] samples 5;
  func x = cos(\value t r);
  func y = sin(\value t r);
};
\end{codeexample}
  \end{key}

  \begin{key}{/tikz/data visualization/visualizer options/straight cycle}
    Causes the data points to be connected by a polygon.
\begin{codeexample}[]
\tikz [scale=.55] \datavisualization
 [scientific clean axes, all axes={ticks=few},
  visualize as smooth line=my data,  my data={straight cycle}]
data [format=function] {
  var t : interval [0:4] samples 5;
  func x = cos(\value t r);
  func y = sin(\value t r);
};
\end{codeexample}
 \end{key}
 
 \begin{key}{/tikz/data visualization/visualizer options/polygon}
   This is an alias for |straight cycle|.
 \end{key}
 
 \begin{key}{/tikz/data visualization/visualizer options/smooth line}
   Causes the data points to be connected by a line that is smoothed
   at the joins:
\begin{codeexample}[]
\tikz [scale=.55] \datavisualization
 [scientific clean axes, all axes={ticks=few},
  visualize as smooth line=my data,  my data={smooth line}]
data [format=function] {
  var t : interval [0:4] samples 5;
  func x = cos(\value t r);
  func y = sin(\value t r);
};
\end{codeexample}
 \end{key}
 
 \begin{key}{/tikz/data visualization/visualizer options/smooth cycle}
   Causes the data points to be connected by a circular line that is
   smoothed at the joins:
\begin{codeexample}[]
\tikz [scale=.55] \datavisualization
 [scientific clean axes, all axes={ticks=few},
  visualize as smooth line=my data,  my data={smooth cycle}]
data [format=function] {
  var t : interval [0:4] samples 5;
  func x = cos(\value t r);
  func y = sin(\value t r);
};
\end{codeexample}
 \end{key}
 
 \begin{key}{/tikz/data visualization/visualizer options/gap line}
   This key causes the data points to be connected by lines that ``do
   not quite touch'' the data points. This is implemented by using the
   |\pgfplothandlergaplineto|, see Section~\ref{section-plot-gapped}. 
\begin{codeexample}[]
\tikz [scale=.55] \datavisualization
 [scientific clean axes, all axes={ticks=few},
  visualize as smooth line=my data,  my data={gap line}]
data [format=function] {
  var t : interval [0:4] samples 5;
  func x = cos(\value t r);
  func y = sin(\value t r);
};
\end{codeexample}
 \end{key}
 
 \begin{key}{/tikz/data visualization/visualizer options/gap cycle}
   Like |gapped line|, only with a cycle:
\begin{codeexample}[]
\tikz [scale=.55] \datavisualization
 [scientific clean axes, all axes={ticks=few},
  visualize as smooth line=my data,  my data={gap cycle}]
data [format=function] {
  var t : interval [0:4] samples 5;
  func x = cos(\value t r);
  func y = sin(\value t r);
};
\end{codeexample}
 \end{key}
 
 \begin{key}{/tikz/data visualization/visualizer options/no lines}
   Suppresses the line. This option only makes sense when the |mark|
   option is used.
\begin{codeexample}[]
\tikz [scale=.55] \datavisualization
 [scientific clean axes, all axes={ticks=few},
  visualize as smooth line=my data,  my data={no lines, style={mark=x}}]
data [format=function] {
  var t : interval [0:4] samples 5;
  func x = cos(\value t r);
  func y = sin(\value t r);
};
\end{codeexample}
 \end{key}

 As indicated earlier, marks will be drawn at the data points when the
 |mark| option is used. All options offered by \tikzname\ for
 configuring marks are available such as |mark repeat|: 
\begin{codeexample}[width=7cm]
\tikz \datavisualization
 [scientific clean axes, 
  visualize as line=my data,
  my data={style={mark=x, mark repeat=3}}]
data [format=function] {
  var x : interval [0:pi] samples 10;
  func y = sin(\value x r);
};
\end{codeexample} 
\end{key}


\begin{key}{/tikz/data visualizers/visualize as smooth line=\meta{visualizer
      name} (default line)}
  A shorthand |visualize as line=|\meta{visualizer name}
  followed \meta{visualizer name}|=smooth line|.
\end{key}

\begin{key}{/tikz/data visualizers/visualize as scatter=\meta{visualizer
      name} (default scatter)}
  A shorthand  |visualize as line=|\meta{visualizer name}
  followed  \meta{visualizer name}|=no lines| and setting
  the |style| of the visualizer so that is will use |mark=x| (plus
  some size adjustments) to draw marks at the data points. 
\begin{codeexample}[width=7cm]
\tikz \datavisualization
 [scientific clean axes, 
  visualize as scatter]
data [format=function] {
  var x : interval [0:pi] samples 10;
  func y = sin(\value x r);
};
\end{codeexample} 
\end{key}


\subsection{Advanced: Creating New Visualizers}

To be written...
% Copyright 2010 by Till Tantau
%
% This file may be distributed and/or modified
%
% 1. under the LaTeX Project Public License and/or
% 2. under the GNU Free Documentation License.
%
% See the file doc/generic/pgf/licenses/LICENSE for more details.


\section{Style Sheets and Legends}
\label{section-dv-style-sheets}

\subsection{Overview}

In many data visualizations, different sets of data need to be
visualized in a single visualization. For instance, in a plot there
might be a line for the sine of~$x$ and another line for the cosine
of~$x$; in another visualization there might be a set of points
representing data from a first experiment and another set of points
representing data from a second experiment; and so on. In order to
indicate to which data set a data point belongs, one might plot the
curve of the sine in, say, black, and the curve of the cosine in red;
we might plot the data from the fist experiment using stars and the
data from the second experiment using circles; and so on. Finally, at
some place in the visualization -- either inside the data or in a
legend next to it -- the meaning of the colors or symbols need to be
explained.

Just as you would like \tikzname\ to map the data points automatically
onto the axes, you will also typically wish \tikzname\ to choose for
instance the coloring of the lines automatically for you. This is done
using \emph{style sheets}. There are at least two good reasons why you
should prefer style sheets over configuring the styling of each
visualizer ``by hand'' using the |style| key:
\begin{enumerate}
\item It is far more convenient to just say
  |style sheet=strong colors| than having to individually
  picking the different colors.
\item The style sheets were chosen and constructed rather
  carefully.

  For instance, the |strong colors| style sheet does not
  pick colors like pure green or pure yellow, which have very low
  contrast with respect to a white background and which often lead to
  unintelligible graphics. Instead, opposing primary colors with
  maximum contrast on a white background were picked that are visually
  quite pleasing.

  Similarly, the different dashing style sheets are
  constructed in such a way that there are only few and small gaps in
  the dashing so that no data points get lost because the dashes are
  spaced too far apart. Also dashing patterns were chosen that have a
  maximum optical difference.

  As a final example, style sheets for
  plot marks are constructed in such a way that even when two plot
  marks lie directly on top of each other, they are still easily
  distinguishable. 
\end{enumerate}
The bottom line is that whenever possible, you should use one of the
predefined style sheets rather than picking colors or dashings at
random.

\subsection{Concepts: Style Sheets}

A \emph{style sheet} is a predefined list of styles such as a list of
colors, a list of dashing pattern, a list of plot marks, or a
combinations thereof. A style sheet can be \emph{attached} to a data
point attribute. Then, the value of this attribute is used with data
points to choose which style in the list should be chosen to visualize
the data point.

In most cases, there is just one attribute to which style sheets get
attached: the |/data point/visualizer| attribute. The effect of
attaching a style sheet to this attribute is that each visualizer is
styled differently.

For the following examples, let us first define a simple data set:
\begin{codeexample}[]
\tikz \datavisualization data group {function classes} = {
  data [set=log, format=function] {
    var x : interval [0.2:2.5];
    func y = ln(\value x);
  }
  data [set=lin, format=function] {
    var x : interval [-2:2.5];
    func y = 0.5*\value x;
  }
  data [set=squared, format=function] {
    var x : interval [-1.5:1.5];
    func y = \value x*\value x;
  }
  data [set=exp, format=function] {
    var x : interval [-2.5:1];
    func y = exp(\value x);
  }
};
\end{codeexample}

\begin{codeexample}[width=6cm]
\tikz \datavisualization [
  school book axes, all axes={unit length=7.5mm},
  visualize as smooth line/.list={log, lin, squared, exp},
  style sheet=strong colors]
data group {function classes};
\end{codeexample}

\begin{codeexample}[width=6cm]
\tikz \datavisualization [
  school book axes, all axes={unit length=7.5mm},
  visualize as smooth line/.list={log, lin, squared, exp},
  style sheet=vary dashing]
data group {function classes};
\end{codeexample}



\subsection{Concepts: Legends}
\label{section-dv-labels-in}

A \emph{legend} is a box that is next to a data visualization (or
inside it at some otherwise empty position) that contains a textual
explanation of the different colors or styles used in a data
visualization.

Just as it is difficult to get colors and dashing patterns right ``by
hand,'' it is also difficult to get a legend right. For instance, when
a small line is shown in the legend that represents the actual line in
the data visualization, if the line is too short and the dashing is
too large, it may be impossible to discern which dashing is actually
meant. Similarly, when plot marks are shown on such a short line,
using a simple straight line may make it hard to read the plot marks
correctly.

The data visualization engine makes some effort to make it easy to
create high-quality legends. Additionally, it also offers ways of
easily adding labels for visualizers directly inside the data
visualization, which is even better than adding a legend, in general.

\begin{codeexample}[width=7cm]
\tikz \datavisualization [
  school book axes, all axes={unit length=7.5mm},
  x axis={label=$x$},
  visualize as smooth line/.list={log, lin, squared, exp},
  log=    {label in legend={text=$\log x$}},
  lin=    {label in legend={text=$x/2$}},
  squared={label in legend={text=$x^2$}},
  exp=    {label in legend={text=$e^x$}},
  style sheet=vary dashing]
data group {function classes};
\end{codeexample}


\begin{codeexample}[width=6.3cm]
\tikz \datavisualization [
  school book axes,
  x axis={label=$x$},
  visualize as smooth line/.list={log, lin, squared, exp},
  every data set label/.append style={text colored},
  log=    {pin in data={text'=$\log x$, when=y is -1}},
  lin=    {pin in data={text=$x/2$, when=x is 2,
                        pin length=1ex}},
  squared={pin in data={text=$x^2$, when=x is 1.1,
                        pin angle=230}},
  exp=    {label in data={text=$e^x$, when=x is -2}},
  style sheet=vary hue]
data group {function classes};
\end{codeexample}


\subsection{Usage: Style Sheets}

\subsubsection{Picking a Style Sheet}

To use a style sheet, you need to \emph{attach} it to an
attribute. You can attach multiple style sheets to an attribute and
in this case all of these style sheets can influence the appearance of
the data points.

Most of the time, you will attach a style sheet to the |set|
attribute. This has the effect that each different data set inside the
same visualization is rendered in a different way. Since this use of
style sheets is the most common, there is a special, easy-to-remember
option for this:

\begin{key}{/tikz/data visualization/style sheet=\meta{style sheet}}
  Adds the \meta{style sheet} to the list of style sheets attached to
  the |set| attribute.
\begin{codeexample}[width=6cm]
\tikz \datavisualization [
  school book axes, all axes={unit length=7.5mm},
  visualize as smooth line/.list={log, lin, squared, exp},
  style sheet=vary thickness and dashing,
  style sheet=vary hue]
data group {function classes};
\end{codeexample}
\end{key}

While the |style sheet| key will attach a style sheet only to the
|set| attribute, the following key handler can be used to attach a
style sheet to an arbitrary attribute:


\begin{handler}{{.style sheet}=\meta{style sheet}}
  Inside a data visualization you can use this key handler together
  with an attribute, that is, with a key having the path prefix
  |/data point|. For instance, in order to attach the \meta{style
    sheet} |strong colors| to the attribute |set|, you could write
\begin{codeexample}[code only]
/data point/set/.style sheet=strong colors    
\end{codeexample}
  Indeed, the |style sheet| key is just a shorthand for the above.

  The effect of attaching a style sheet is the following:
  \begin{itemize}
  \item A new object is created that will monitor the attribute.
  \item Each time a special \emph{styling key} is emitted by the data
    visualization engine, this object will inspect the current value
    of the attribute to which it is attached.
  \item Depending on this value, one of the styles stored in the style
    sheet is chosen (how this works, exactly, will be explained in a
    moment).
  \item The chosen style is then locally applied.
  \end{itemize}
  
  In reality, things are a bit more complicated: If the attribute of
  the data point happens to have a subkey named in the same way as the
  value, then the value of is this subkey is used instead of the value
  itself. This allows you to ``rename'' a value.
  
  In a sense, a style sheet behaves much like a visualizer (see
  Section~\ref{section-dv-visualizers}): In accordance with the value
  of a certain attribute, the appearance of data points
  change. However, there are a few differences: First, the styling of
  a data point needs to be triggered explicitly and this triggering is
  not necessarily done for each data point individually, but only for
  a whole visualizer. Second, styles can be computed even when no data
  point is present. This is useful for instance in a legend since,
  here, a visual representation of a visualizer needs to be created
  independently of the actual data points.
\end{handler}

\subsubsection{Creating a New Style Sheet}

Creating a style sheet works as follows: For each
possible value that an attribute can attain we must specify a
style. This is done by creating a style key for each such possible
value with a special path prefix and setting this style key to the
desired value. The special path prefix is
|/pgf/data visualization/style sheets| followed by the name of the
style sheet.

As an example, suppose we wish to create a style sheet |test| that makes
styled data points |red| when the attribute has value |foo| and
|green| when the attribute has value |bar| and |dashed, blue| when the
attribute is |foobar|. We could then write
\begin{codeexample}[code only]
/pgf/data visualization/style sheets/test/foo/.style={red},    
/pgf/data visualization/style sheets/test/bar/.style={green},    
/pgf/data visualization/style sheets/test/foobar/.style={dashed, blue},    
\end{codeexample}

We could then attach this style sheet to the attribute |code| as
follows:
\begin{codeexample}[code only]
/data point/code/.style sheet=test
\end{codeexample}

Then, when |/data point/code=foobar| holds when the styling signal is
raised, the stying |dashed, blue| will get executed.

A natural question arises concerning the situation that the value of
the attribute is not defined as a subkey of the style sheet. In this
case, a special key gets executed:

\begin{stylekey}{/pgf/data visualization/style sheets/\meta{style
      sheet}/default style=\meta{value}}
  This key gets during styling whenever
  |/pgf/data visualization/style sheet/|\meta{style
    sheet}|/|\meta{value} is not defined. 
\end{stylekey}

Let us put all of this together in a real-life example. Suppose we
wish to create a style sheet that makes the first data set |green|, the
second |yellow| and the third one |red|. Further data sets should be,
say, |black|. The attribute that we intend to style is the |set|
attribute. For the moment, we assume that the data sets will be named
|1|, |2|, |3|, and so on (instead of, say, |experiment 1| or |sin| or
something more readable -- we will get rid of this restriction in a
minute).

We would now write:

\begin{codeexample}[]
\pgfkeys{
  /pgf/data visualization/style sheets/traffic light/.cd,
  % All these styles have the above prefix.
  1/.style={green!50!black},
  2/.style={yellow!90!black},
  3/.style={red!80!black},
  default style/.style={black}
}
\tikz \datavisualization [
  school book axes,
  visualize as line=1,
  visualize as line=2,
  visualize as line=3,
  style sheet=traffic light]
data point [x=0, y=0, set=1]
data point [x=2, y=2, set=1]
data point [x=0, y=1, set=2]
data point [x=2, y=1, set=2]
data point [x=0.5, y=1.5, set=3]
data point [x=2.25, y=1.75, set=3];
\end{codeexample}

In the above example, we have to name the visualizers |1|, |2|, |3|
and so one since the value of the |set| attribute is used both assign
data points to visualizers and also pick a style sheet. However, it
would be much nicer if we could name any way we want. To achieve this,
we use the special rule for style sheets that says that if there is a
subkey of an attribute whose name is the same name as the value, then
the value of this key is used instead. This slightly intimidating
definition is much easier to understand when we have a look at an
example:

\pgfkeys{
  /pgf/data visualization/style sheets/traffic light/.cd,
  % All these styles have the above prefix.
  1/.style={green!50!black},
  2/.style={yellow!90!black},
  3/.style={red!80!black},
  default style/.style={black}
}

\begin{codeexample}[]
% Definition of traffic light keys as above  
\begin{tikzpicture}
  \datavisualization data group {lines} = {  
    data point [x=0, y=0,       set=normal]
    data point [x=2, y=2,       set=normal]
    data point [x=0, y=1,       set=heated]
    data point [x=2, y=1,       set=heated]
    data point [x=0.5, y=1.5,   set=critical]
    data point [x=2.25, y=1.75, set=critical]
  };
  \datavisualization [
    school book axes,
    visualize as line=normal,
    visualize as line=heated,
    visualize as line=critical,
    /data point/set/normal/.initial=1,
    /data point/set/heated/.initial=2,
    /data point/set/critical/.initial=3,
    style sheet=traffic light]
  data group {lines};
\end{tikzpicture}
\end{codeexample}

Now, it is a bit bothersome that we have to set all these
|/data point/set/...| keys by hand. It turns out that this is not
necessary: Each time a visualizer is created, a subkey of
|/data point/set| with the name of the visualizer is created
automatically and a number is stored that is increased for each new
visualizer in a data visualization. This means that the three lines
starting with |/data point| are inserted automatically for you, so
they can be left out. However, you would need them for instance when
you would like several different data sets to use the same styling:


\begin{codeexample}[]
% Definition of traffic light keys as above  
\tikz \datavisualization [
  school book axes,
  visualize as line=normal,
  visualize as line=heated,
  visualize as line=critical,
  /data point/set/critical/.initial=1, % same styling as first set
  style sheet=traffic light]
data group {lines};
\end{codeexample}

We can a command that slightly simplifies the definition of style
sheets:

\begin{command}{\pgfdvdeclarestylesheet\marg{name}\marg{keys}}
  This command executes the \meta{keys} with the path prefix
  |/pgf/data visualization/style sheets/|\penalty0\meta{name}. The above
  definition of the traffic light style sheet could be rewritten as
  follows:
\begin{codeexample}[code only]
\pgfdvdeclarestylesheet{traffic light}{
  1/.style={green!50!black},
  2/.style={yellow!90!black},
  3/.style={red!80!black},
  default style/.style={black}
}
\end{codeexample}
\end{command}

As a final example, let us create a style sheet that changes the
dashing pattern according to the value of the attribute. We do not
need to define an large number of styles in this case, but can use the
|default style| key to ``calculate'' the correct dashing.

\begin{codeexample}[]
\pgfdvdeclarestylesheet{my dashings}{
  default style/.style={dash pattern={on #1pt off 1pt}}
}
\tikz \datavisualization [
  school book axes,
  visualize as line=normal,
  visualize as line=heated,
  visualize as line=critical,
  style sheet=my dashings]
data group {lines};
\end{codeexample}

\subsubsection{Creating a New Color Style Sheet}

Creating a style sheet that varies colors according to an attribute
works the same way as creating a normal style sheet: Subkeys lies |1|,
|2|, and so on use the |style| attribute to setup a color. However,
instead of using the |color| attribute to set the color, you should
use the |visualizer color| key to set the color:

\begin{key}{/tikz/visualizer color=\meta{color}}
  This key is used to set the color |visualizer color| to
  \meta{color}. This color is used by visualizers to color the data
  they visualize, rather than the current ``standard color.'' The
  reason for not using the normal current color is simply that it
  makes many internals of the data visualization engine a bit
  simpler. 
\begin{codeexample}[]
\pgfdvdeclarestylesheet{my colors}
{
  default style/.style={visualizer color=black},
  1/.style={visualizer color=black},
  2/.style={visualizer color=red!80!black},
  3/.style={visualizer color=blue!80!black},
}
\tikz \datavisualization [
  school book axes,
  visualize as line=normal,
  visualize as line=heated,
  visualize as line=critical,
  style sheet=my colors]
data group {lines};
\end{codeexample}
\end{key}

There is an additional command that makes it easy to define a style
sheet based on a \emph{color series}. Color series are a concept from
the |xcolor| package: The idea is that we start with a certain color
for the first data set and then add a certain ``color offset'' for
each next data point. Please consult the documentation of the |xcolor|
package for details.

\begin{command}{\tikzdvdeclarestylesheetcolorseries\marg{name}\marg{color
      model}\marg{initial color}\marg{step}}
  This command creates a new style sheet using
  |\pgfdvdeclarestylesheet|. This style sheet will only have a default
  style setup that maps numbers to the color in the color series
  starting with \meta{initial color} and having a stepping of
  \meta{step}. Note that when the value of the attribute is |1|, which
  it is the first data set, the \emph{second} color in the color
  series is used (since counting starts at |0| for color
  series). Thus, in general, you need to start the \meta{initial
    color} ``one early.''
\begin{codeexample}[]
\tikzdvdeclarestylesheetcolorseries{greens}{hsb}{0.3,1.3,0.8}{0,-.4,-.1}
\tikz \datavisualization [
  school book axes,
  visualize as line=normal,
  visualize as line=heated,
  visualize as line=critical,
  style sheet=greens]
data group {lines};
\end{codeexample}

\end{command}




\subsection{Reference: Style Sheets for Lines}

The following style sheets can be applied to visualizations that use
the |visualize as line| and related keys. For the examples, the
following style and data set are used:

\begin{codeexample}[code only]
\tikzdatavisualizationset {
  example visualization/.style={
    scientific axes=clean,
    y axis={ticks={style={
          /pgf/number format/fixed,
          /pgf/number format/fixed zerofill,
          /pgf/number format/precision=2}}},
    x axis={ticks={tick suffix=${}^\circ$}},
    1={label in legend={text=$\frac{1}{6}\sin 11x$}},
    2={label in legend={text=$\frac{1}{7}\sin 12x$}},
    3={label in legend={text=$\frac{1}{8}\sin 13x$}},
    4={label in legend={text=$\frac{1}{9}\sin 14x$}},
    5={label in legend={text=$\frac{1}{10}\sin 15x$}},
    6={label in legend={text=$\frac{1}{11}\sin 16x$}},
    7={label in legend={text=$\frac{1}{12}\sin 17x$}},
    8={label in legend={text=$\frac{1}{13}\sin 18x$}}
  }
}  
\end{codeexample}
\tikzdatavisualizationset {
  example visualization/.style={
    scientific axes=clean,
    y axis={ticks={style={
          /pgf/number format/fixed,
          /pgf/number format/fixed zerofill,
          /pgf/number format/precision=2}}},
    x axis={ticks={tick suffix=${}^\circ$}},
    1={label in legend={text=$\frac{1}{6}\sin 11x$}},
    2={label in legend={text=$\frac{1}{7}\sin 12x$}},
    3={label in legend={text=$\frac{1}{8}\sin 13x$}},
    4={label in legend={text=$\frac{1}{9}\sin 14x$}},
    5={label in legend={text=$\frac{1}{10}\sin 15x$}},
    6={label in legend={text=$\frac{1}{11}\sin 16x$}},
    7={label in legend={text=$\frac{1}{12}\sin 17x$}},
    8={label in legend={text=$\frac{1}{13}\sin 18x$}}
  }
}  

\begin{codeexample}[code only]
\tikz \datavisualization data group {sin functions} = {
  data [format=function] {
    var set : {1,...,8};
    var x : interval [0:50];
    func y = sin(\value x * (\value{set}+10))/(\value{set}+5);
  }
};  
\end{codeexample}
\tikz \datavisualization data group {sin functions} = {
  data [format=function] {
    var set : {1,...,8};
    var x : interval [0:50];
    func y = sin(\value x * (\value{set}+10))/(\value{set}+5);
  }
};  

\begin{stylesheet}{vary thickness}
  This style varies the thickness of lines. It should be used only
  when there are only two or three lines, and even then it is not
  particularly pleasing visually.
\begin{codeexample}[width=10cm]
\tikz \datavisualization [
  visualize as smooth line/.list=
    {1,2,3,4,5,6,7,8},
  example visualization,
  style sheet=vary thickness]
data group {sin functions};
\end{codeexample}
\end{stylesheet}


\begin{stylesheet}{vary dashing}
  This style varies the dashing of lines. Although it is not
  particularly pleasing visually and although visualizations using
  this style sheet tend to look ``excited'' (but not necessarily
  ``exciting''), this style sheet is often the best choice when the
  visualization is to be printed in black and white.
\begin{codeexample}[width=10cm]
\tikz \datavisualization [
  visualize as smooth line/.list=
    {1,2,3,4,5,6,7,8},
  example visualization,
  style sheet=vary dashing]
data group {sin functions};
\end{codeexample}
  As can be seen, there are only seven distinct dashing patterns. The
  eighth and further lines will use a solid line once more. You will
  then have to specify the dashing ``by hand'' using the |style|
  option together with the visualizer.
\end{stylesheet}

\begin{stylesheet}{vary dashing and thickness}
  This style alternates between varying the thickness and the dashing
  of lines. The 
  difference to just using both the |vary thickness| and
  |vary dashing| is that too thick lines are avoided. Instead, this
  style creates clearly distinguishable line styles for many lines (up
  to 14) with a minimum of visual clutter. This style is the most
  useful for visualizations when many different lines (ten or more)
  should be printed in black and white.
\begin{codeexample}[width=10cm]
\tikz \datavisualization [
  visualize as smooth line/.list=
    {1,2,3,4,5,6,7,8},
  example visualization,
  style sheet=vary thickness
              and dashing]
data group {sin functions};
\end{codeexample}
  For comparison, here is the must-less-than-satisfactory result of
  combining the two independent style sheets:
\begin{codeexample}[width=10cm]
\tikz \datavisualization [
  visualize as smooth line/.list=
    {1,2,3,4,5,6,7,8},
  example visualization,
  style sheet=vary thickness,
  style sheet=vary dashing]
data group {sin functions};
\end{codeexample}
\end{stylesheet}


\subsection{Reference: Style Sheets for Scatter Plots}

The following style sheets can be used both for scatter plots and also
with lines. In the latter case, the marks are added to the lines.

\begin{stylesheet}{cross marks}
  This style uses different crosses to distinguish between the data
  points of different data sets. The crosses were chosen in such a way
  that when two different cross marks lie at the same coordinate,
  their overall shape allows one to still uniquely determine which
  marks are on top of each other.

  This style supports only up to six different data sets.
\begin{codeexample}[width=10cm]
\tikz \datavisualization [
  visualize as scatter/.list=
    {1,2,3,4,5,6,7,8},
  example visualization,
  style sheet=cross marks]
data group {sin functions};
\end{codeexample}
\begin{codeexample}[width=10cm]
\tikz \datavisualization [
  visualize as smooth line/.list=
    {1,2,3,4,5,6,7,8},
  example visualization,
  style sheet=cross marks]
data group {sin functions};
\end{codeexample}
\end{stylesheet}


\subsection{Reference: Color Style Sheets}

Color style sheets are very useful for creating visually pleasing data
visualizations that contain multiple data sets. However, there are two
things to keep in mind:

\begin{itemize}
\item At some point, every data visualization is printed or photo
  copied in black and white by someone. In this case, data sets can
  often no longer be distinguished.
\item A few people are color blind. They will not be able to
  distinguish between red and green lines (and some people are not
  even able to distinguish colors at all).
\end{itemize}

For these reasons, if there is any chance that the data visualization
will be printed in black and white at some point, consider combining
color style sheets with style sheets like |vary dashing| to make data
sets distinguishable in all situations.


\begin{stylesheet}{strong colors}
  This style sheets uses pure primary colors that can very easily be
  distinguished. Although not as visually pleasing as the |vary hue|
  style sheet, the visualizations are easier to read when this style
  sheet is used. Up to six different data sets are supported.
\begin{codeexample}[width=10cm]
\tikz \datavisualization [
  visualize as smooth line/.list=
    {1,2,3,4,5,6,7,8},
  example visualization,
  style sheet=strong colors]
data group {sin functions};
\end{codeexample}
\begin{codeexample}[width=10cm]
\tikz \datavisualization [
  visualize as smooth line/.list=
    {1,2,3,4,5,6,7,8},
  example visualization,
  style sheet=strong colors,
  style sheet=vary dashing]
data group {sin functions};
\end{codeexample}
\end{stylesheet}


Unlike |strong colors|, the following style sheets support, in
principle, an unlimited number of data set. In practice, as always,
more than four or five data sets lead to nearly indistinguishable data
sets.

\begin{stylesheet}{vary hue}
  This style uses a different hue for each data set. 
\begin{codeexample}[width=10cm]
\tikz \datavisualization [
  visualize as smooth line/.list=
    {1,2,3,4,5,6,7,8},
  example visualization,
  style sheet=vary hue]
data group {sin functions};
\end{codeexample}
\end{stylesheet}

\begin{stylesheet}{shades of blue}
  As the name suggests, different shades of blue are used for different
  data sets.
\begin{codeexample}[width=10cm]
\tikz \datavisualization [
  visualize as smooth line/.list=
    {1,2,3,4,5,6,7,8},
  example visualization,
  style sheet=shades of blue]
data group {sin functions};
\end{codeexample}
\end{stylesheet}


\begin{stylesheet}{shades of red}
\begin{codeexample}[width=10cm]
\tikz \datavisualization [
  visualize as smooth line/.list=
    {1,2,3,4,5,6,7,8},
  example visualization,
  style sheet=shades of red]
data group {sin functions};
\end{codeexample}
\end{stylesheet}


\begin{stylesheet}{gray scale}
  For once, this style sheet can also be used when the visualization
  is printed in black and white.
\begin{codeexample}[width=10cm]
\tikz \datavisualization [
  visualize as smooth line/.list=
    {1,2,3,4,5,6,7,8},
  example visualization,
  style sheet=gray scale]
data group {sin functions};
\end{codeexample}
\end{stylesheet}


\subsection{Usage: Labeling Data Sets Inside the Visualization}

In a visualization that contains multiple data sets, it is often
necessary to clearly point out which line or mark type corresponds to
which data set. This can be done in the main text via a sentence like
``the normal data (black) lies clearly below the critical values
(red),'' but it often a good idea to indicate data sets ideally
directly inside the data visualization or directly next to it in a
so-called legend.

The data visualization engine has direct support both for indicating
data sets directly inside the visualization and also for indicating
them in a legend.

The ``best'' way of indicating where a data set lies or which color is
used for it is to put a label directly inside the data
visualization. The reason this is the ``best'' way is that people do
not have to match the legend entries against the data, let alone
having to look up the meaning of line styles somewhere in the
text. However, adding a label directly inside the visualization is
also the most tricky way of indicating data sets since it is hard to
compute good positions for the labels automatically and since there
needs to be some empty space where the label can be put.

\subsubsection{Placing a Label Next to a Data Set}

The following key is used to create a label inside the data
visualization for a data set:

\begin{key}{/tikz/data visualization/visualizer options/label in data=\meta{options}}
  This key is passed to a visualizer that has previously been created
  using keys starting |visualize as ...|. It will create a label
  inside the data visualization ``next'' to the visualizer (the
  details are explained in a moment). You can use this key multiple
  times with a visualizer to create multiple labels at different
  points with different texts.

  The \meta{options} determine which text is shown and where it is
  shown. They are executed with the following path prefix:
\begin{codeexample}[code only]
/tikz/data visualization/visualizer label options
\end{codeexample}

  In order to configure which text is shown and where, use the
  following keys inside the \meta{options}:
  
  \begin{key}{/tikz/data visualization/visualizer label options/text=\meta{text}}
    This is the text that will be displayed next to the data. It will
    be to the ``left'' of the data, see the description below.
  \end{key}
  \begin{key}{/tikz/data visualization/visualizer label options/text'=\meta{text}}
    Like |text|, only the text will be to the ``right'' of the data.
  \end{key}
  
  The following keys are used to configure where the label will be
  shown. They use different strategies to specify one data point where
  the label will be anchored. The coordinate of this data point will
  be stored in |(label| |visualizer| |coordinate)|. Independently of
  the strategy, once the data point has been chosen, the coordinate of
  the next data point is stored in |(label| |visualizer|
  |coordinate')|. Then, a (conceptual) line is created from the first
  coordinate to the second and a node is placed at the beginning of
  this line to its ``left'' or, for the |text'| option, on its
  ``right.'' More precisely, an automatic anchor is computed for a
  node placed implicitly on this line using the |auto| option or, for
  the |text'| option, using |auto,swap|.

  The node placed at the position computed in this way will have the
  \meta{text} set by the |text| or |text'| option and its styling is
  determined by the current |node style|.
  
  Let us now have a look at the different ways of determining the data
  point at which the label in anchored:
  \begin{key}{/tikz/data visualization/visualizer label
      options/when=\meta{attribute}| is|\meta{number}}
    This key causes the value of the \meta{attribute} to be monitored
    in the stream of data points. The chosen is data point is the
    first data point where the \meta{attribute} is at least
    \meta{number} (if this never happens, the last data point is used).
\begin{codeexample}[width=6.3cm]
\tikz \datavisualization [
  school book axes,
  x axis={label=$x$},
  visualize as smooth line/.list={log, lin, squared, exp},
  log=    {label in data={text'=$\log x$, when=y is -1,
                          text colored}},
  lin=    {label in data={text=$x/2$,     when=x is 2}},
  squared={label in data={text=$x^2$,     when=x is 1.1}},
  exp=    {label in data={text=$e^x$,     when=x is -2,
                          text colored}},
  style sheet=vary hue]
data group {function classes};
\end{codeexample}
  \end{key}
  \begin{key}{/tikz/data visualization/visualizer label
      options/index=\meta{number}}
    This key chooses the \meta{number}th data point belonging to the
    visualizer's data set.
\begin{codeexample}[width=6.3cm]
\tikz \datavisualization [
  school book axes,
  x axis={label=$x$},
  visualize as smooth line/.list={exp},
  exp=    {label in data={text=$5$, index=5},
           label in data={text=$10$, index=10},
           label in data={text=$20$, index=20},
           style={mark=x}},
  style sheet=vary hue]
data group {function classes};
\end{codeexample}
  \end{key}
  \begin{key}{/tikz/data visualization/visualizer label options/pos=\meta{fraction}}
    This key chooses the first data point belonging to the data set
    whose index is at least \meta{fraction} times the number of all
    data points in the data set.
\begin{codeexample}[width=6.3cm]
\tikz \datavisualization [
  school book axes,
  x axis={label=$x$},
  visualize as smooth line=exp,
  exp=    {label in data={text=$.2$, pos=0.2},
           label in data={text=$.5$, pos=0.5},
           label in data={text=$.95$, pos=0.95},
           style={mark=x}},
  style sheet=vary hue]
data group {function classes};
\end{codeexample}
  \end{key}
  \begin{key}{/tikz/data visualization/visualizer label options/auto}
    This key is executed automatically by default. It works like the
    |pos| option, where the \meta{fraction} is set to $(\meta{data set's
      index}-1/2)/\meta{number of data sets}$. For instance, when
    there are $10$ data sets, the fraction for the first one will be
    $5\%$, the fraction for the second will be $15\%$, for the third
    it will be $25\%$, ending with $95\%$ for the last one.

    The net effect of all this is that when there are several lines,
    labels will be placed at different positions along the lines with
    hopefully only little overlap.
\begin{codeexample}[width=6.3cm]
\tikz \datavisualization [
  scientific axes=clean,
  visualize as smooth line/.list={linear, squared, cubed},
  linear ={label in data={text=$2x$}},
  squared={label in data={text=$x^2$}},
  cubed  ={label in data={text=$x^3$}}]
data [set=linear, format=function] {
  var x : interval [0:1.5];
  func y = 2*\value x;
}
data [set=squared, format=function] {
  var x : interval [0:1.5];
  func y = \value x * \value x;
}
data [set=cubed, format=function] {
  var x : interval [0:1.5];
  func y = \value x * \value x * \value x;
};
\end{codeexample}
    As can be seen in the example, the result is not always
    satisfactory. In this case, the |pin in data| option might be
    preferable, see below.
  \end{key}
  
  The following keys allow you to style labels.

  \begin{key}{/tikz/data visualization/visualizer label
      options/node style=\meta{options}}
    Just passes the options to |/tikz/data visualization/node style|.
  \end{key}
  \begin{key}{/tikz/data visualization/visualizer label
      options/text colored}
    Causes the |node style| to set the text color to
    |visualizer color|. The effect of this is that the label's text
    will have the same color as the data set to which it is attached.
  \end{key}
  
  \begin{stylekey}{/tikz/data visualization/every data set label}
    This style is executed with every label that represents a
    data set. Inside this style, use |node style| to change the
    appearance of nodes. This style has a default definition, usually
    you should just append things to this style.

\begin{codeexample}[width=6.3cm]
\tikz \datavisualization [
  school book axes,
  x axis={label=$x$},
  visualize as smooth line/.list={log, lin, squared, exp},
  every data set label/.append style={text colored},
  log=    {label in data={text'=$\log x$, when=y is -1}},
  lin=    {label in data={text=$x/2$,
                    node style=sloped,    when=x is 2}},
  squared={label in data={text=$x^2$,     when=x is 1.1}},
  exp=    {label in data={text=$e^x$,
                    node style=sloped,    when=x is -2}},
  style sheet=vary hue]
data group {function classes};
\end{codeexample}
  \end{stylekey}
  
  \begin{stylekey}{/tikz/data visualization/every label in data}
    Like |every data set label|, this key is also executed with
    labels. However, this key is executed after the style sheets have
    been executed, giving you a chance to overrule their styling.
  \end{stylekey}
\end{key}

\subsubsection{Connecting a Label to a Data Set via a Pin} 

\begin{key}{/tikz/data visualization/visualizer options/pin in data=\meta{options}}
  This key is a variant of the |label in data| key and takes the same
  options, plus two additional ones. The difference to |label in data|
  is that the label node is shown a bit removed from the data set, but
  connected to it via a small line (this is like the difference
  between the |label| and |pin| options).
\begin{codeexample}[width=6.3cm]
\tikz \datavisualization [
  scientific axes=clean,
  visualize as smooth line/.list={linear, squared, cubed},
  linear ={pin in data={text=$2x$}},
  squared={pin in data={text=$x^2$}},
  cubed  ={pin in data={text=$x^3$}}]
data [set=linear, format=function] {
  var x : interval [0:1.5];
  func y = \value x;
}
data [set=squared, format=function] {
  var x : interval [0:1.5];
  func y = \value x * \value x;
}
data [set=cubed, format=function] {
  var x : interval [0:1.5];
  func y = \value x * \value x * \value x;
};
\end{codeexample}
  The following keys can be used additionally:
  \begin{key}{/tikz/data visualization/visualizer label options/pin angle=\meta{angle}}
    The position of the label of a |pin in data| is mainly computed in
    the same way as for a |label in data|. However, once the position
    has been computed, the label is shifted as follows:
    \begin{itemize}
    \item When an \meta{angle} is specified using the present key, the
      shift is by the current value of |pin length| in the direction
      of \meta{angle}.
    \item When \meta{angle} is empty (which is the default), then the
      shift is also by the current value of |pin length|, but now in
      the direction that is orthogonal and to the left of the line
      between the coordinate of the data point and the coordinate of
      the next data point. When |text'| is used, the direction is to
      the right instead of the left.
    \end{itemize}
  \end{key}
  
  \begin{key}{/tikz/data visualization/visualizer label options/pin length=\meta{dimension}}
    See the description of |pin angle|.
  \end{key}  

\begin{codeexample}[width=6.3cm]
\tikz \datavisualization [
  school book axes,
  x axis={label=$x$},
  visualize as smooth line/.list={log, lin, squared, exp},
  every data set label/.append style={text colored},
  log=    {pin in data={text'=$\log x$, when=y is -1}},
  lin=    {pin in data={text=$x/2$, when=x is 2,
                        pin length=1ex}},
  squared={pin in data={text=$x^2$, when=x is 1.1,
                        pin angle=230}},
  exp=    {label in data={text=$e^x$, when=x is -2}},
  style sheet=vary hue]
data group {function classes};
\end{codeexample}
\end{key}



\subsection{Usage: Labeling Data Sets Inside a Legend}

The ``classical'' way of indicating the style used for the different
data sets inside a visualization is a \emph{legend}. It is a
description next to or even inside the visualization that contains one
line for each data set and displays an iconographic version of the
data set next to some text labeling the data set. Note, however, that
even though legend are quite common, also
consider using a |label in data| or a |pin in data| instead.

Creating a high-quality legend is by no means simple. A legend should
not distract the reader, so aggressive borders should definitively be
avoided. A legend should make it easy to match the actual
styling of a data set (like, say, using a red, dashed line) to
the ``iconographic'' representation of this styling. An example of
what can go wrong here is using short lines to represent lines dashed
in different way where the lines are so short that the differences in
the dashing cannot be discerned. Another example is showing straight
lines with plot marks on them where the plot marks are obscured by the
horizontal line itself, while the plot marks are clearly visible in
the actual visualization since no horizontal lines occur.

The data visualization engine comes with a large set of options for
creating and placing high-quality legends next or inside data
visualizations. 

\subsubsection{Creating Legends and Legend Entries}

A data visualization can be accompanied by one or more legends. In
order to create a legend, the following key can be used (although, in
practice, you will usually use the |legend| key instead, see below):

\begin{key}{/tikz/data visualization/new legend=\meta{legend name}
    (default main legend)}
  This key is used to create a new legend named \meta{legend name}. The
  legend is empty by default and further options are needed to add
  entries to it. When the key is called a second time for the same
  \meta{legend name} nothing happens.

  When a legend is created, a new key is created that can
  subsequently be used to configure the legend:
  \begin{key}{/tikz/data visualization/\meta{legend name}=\meta{options}}
    When this key is used, the \meta{options} are executed with the
    path prefix
\begin{codeexample}[code only]
/tikz/data visualization/legend options
\end{codeexample}
    The different keys with this path prefix allow you to change the
    position where the legend is shown and how it is organised (for
    instance, whether legend entries are shown in a row or in a column
    or in a square).

    The different possible keys will be explained in the course of
    this section.
  \end{key}
  
  In the end, the legend is just a \tikzname\ node, a |matrix| node,
  to be precise. The following key is used to style this node:
  
  
  \begin{key}{/tikz/data visualization/legend options/matrix node style=\meta{options}}
    Adds the \meta{options} to the list of options that will be
    executed when the legend's node is created.
\begin{codeexample}[width=8cm]
\tikz \datavisualization [
  scientific axes,
  visualize as smooth line/.list=
    {log, lin, squared, exp},
  legend={matrix node style={fill=black!25}},
  log=    {label in legend={text=$\log x$}},
  lin=    {label in legend={text=$x/2$}},
  squared={label in legend={text=$x^2$}},
  exp=    {label in legend={text=$e^x$}},
  style sheet=vary dashing]
data group {function classes};
\end{codeexample}      
  \end{key}

  
  The following style allows you to configure the default appearance
  of every newly created legend:
  \begin{stylekey}{/tikz/data visualization/legend options/every new legend}
    This key defaults to |east outside, label style=text right|. This means
    that by default a legend is placed to the right of the data
    visualization and that in the individual legend entries the text
    is to the right of the data set visualization. 
  \end{stylekey}

\begin{codeexample}[width=6cm]
\tikz \datavisualization [
  scientific axes, x axis={label=$x$},
  visualize as smooth line/.list={log, lin, squared, exp},
  new legend={upper legend},
  new legend={lower legend},
  upper legend=above,
  lower legend=below,
  log=    {label in legend={text=$\log x$, legend=upper legend}},
  lin=    {label in legend={text=$x/2$, legend=upper legend}},
  squared={label in legend={text=$x^2$, legend=lower legend}},
  exp=    {label in legend={text=$e^x$, legend=lower legend}},
  style sheet=vary dashing]
data group {function classes};
\end{codeexample}  
\end{key}

\begin{key}{/tikz/data visualization/legend=\meta{options}}
  This is a shorthand for |new legend=main legend, main legend=|\meta{options}.
  In other words, this key creates a new |main legend| and immediately
  passes the configuration \meta{options} to this legend.

\begin{codeexample}[width=7cm]
\tikz \datavisualization [
  scientific axes, x axis={label=$x$},
  visualize as smooth line/.list={log, lin, squared, exp},
  legend=below,
  log=    {label in legend={text=$\log x$}},
  lin=    {label in legend={text=$x/2$}},
  squared={label in legend={text=$x^2$}},
  exp=    {label in legend={text=$e^x$}},
  style sheet=vary dashing]
data group {function classes};
\end{codeexample}
\end{key}
  
As pointed out above, a legend is empty by default. In particular,
the different data sets are not automatically inserted into the
legend. Instead, the key |label in legend| must be used together
with a data set:

\begin{key}{/tikz/data visualization/visualizer options/label in legend=\meta{options}}
  This key is passed to a data set, similar to options like
  |pin in data| or |smooth line|. The \meta{options} are used to
  configure the following:
  \begin{itemize}
  \item The legend in which the data set should be visualized.
  \item The text that is to be shown in the legend for the data set.
  \item The appearance of the legend entries.
  \end{itemize}
  In detail, the \meta{options} are executed with the path prefix
\begin{codeexample}[code only]
/tikz/data visualization/legend entry options
\end{codeexample}
  To configure in which legend the label should appear, use the
  following key:
  \begin{key}{/tikz/data visualization/legend entry
      options/legend=\meta{name} (initially main legend)}
    Set this key to the name of a legend that has previously been
    created using |new legend|. The label will then be shown in this
    legend. 

    In most cases, there is only one legend (namely |main legend|) and
    there is no need to set this key since it defaults to the main
    legend.

    Also note that the legend \meta{name} is automatically created if
    it nodes not yet exist.
  \end{key}

  \begin{key}{/tikz/data visualization/legend entry options/text=\meta{text}}
    Use this key to setup the \meta{text} that is shown as the label
    of the data set. 

\begin{codeexample}[width=8cm]
\tikz \datavisualization [
  scientific axes, x axis={label=$x$},
  visualize as smooth line/.list=
    {log, lin, squared, exp},
  log=    {label in legend={text=$\log x$}},
  lin=    {label in legend={text=$x/2$}},
  squared={pin in data    ={text=$x^2$, pos=0.1}},
  exp=    {label in data  ={text=$e^x$}},
  style sheet=vary dashing]
data group {function classes};
\end{codeexample}  
  \end{key}
  
  In addition to the two keys described above, there are further
  keys that are described in
  Section~\ref{section-dv-label-legend-entry-options}.
\end{key}


\subsubsection{Rows and Columns of Legend Entries}

In a legend, the different legend entries are arranged in a matrix,
which typically has only one row or one column. For the impatient
reader: Say |rows=1| to get everything in a row, say |columns=1| to
get everything in a single column, and skip the rest of this section.

The more patient reader will appreciate that when there are very many
different data sets in a single visualization, it may be 
necessary to use more than one row or column inside the legend.
\tikzname\ comes with a rather powerful mechanism for distributing the
multiple legend entries over the matrix. 

The first thing to decide is in which ``direction'' the entries should
be inserted into the matrix. Suppose we have a $3 \times 3$ matrix and
our entries are $a$, $b$, $c$, and so on. Then, one might place the
$a$ in the upper left corner of the matrix, $b$ in the upper middle
position, $c$ in the upper right position, and $d$ in the middle left
position. This is a ``first right, then down'' strategy. A different
strategy might be to place the $a$ in the upper left corner, but $b$
in the middle left position, $c$ in the lower left position, and $d$
then in the upper middle position. This is a ``first down, then
right'' strategy. In certain situations it might even make sense to
place $a$ in the lower right corner and then go ``first up, then
left''.

All of these strategies are supported by the |legend| command. You can
configure which strategy is used using the following keys:

\tikzdatavisualizationset {
  legend example/.style={
    scientific axes, all axes={length=1cm, ticks=none},
    1={label in legend={text=1}},
    2={label in legend={text=2}},
    3={label in legend={text=3}},
    4={label in legend={text=4}},
    5={label in legend={text=5}},
    6={label in legend={text=6}},
    7={label in legend={text=7}},
    8={label in legend={text=8}}
  }
}  


\begin{key}{/tikz/data visualization/legend options/down then right}
  Causes the legend entries to fill the legend matrix first downward
  and, once a column is full, the next column is begun to the right of
  the previous one. This is the default.
\begin{codeexample}[width=6cm]
\tikz \datavisualization [
  visualize as smooth line/.list={1,2,3,4,5,6,7,8},
  legend example, style sheet=vary hue,
  main legend={down then right, columns=3}]
data group {sin functions};
\end{codeexample}
  In the example, the |legend example| is the following style:
\begin{codeexample}[code only]
\tikzdatavisualizationset {
  legend example/.style={
    scientific axes, all axes={length=1cm, ticks=none},
    1={label in legend={text=1}},
    2={label in legend={text=2}},
    3={label in legend={text=3}},
    4={label in legend={text=4}},
    5={label in legend={text=5}},
    6={label in legend={text=6}},
    7={label in legend={text=7}},
    8={label in legend={text=8}}
  }
}  
\end{codeexample}
\end{key}

\begin{key}{/tikz/data visualization/legend options/down then left}
\begin{codeexample}[width=6cm]
\tikz \datavisualization [
  visualize as smooth line/.list={1,2,3,4,5,6,7,8},
  legend example, style sheet=vary hue,
  main legend={down then left, columns=3}]
data group {sin functions};
\end{codeexample}
\end{key}

\begin{key}{/tikz/data visualization/legend options/up then right}
\begin{codeexample}[width=6cm]
\tikz \datavisualization [
  visualize as smooth line/.list={1,2,3,4,5,6,7,8},
  legend example, style sheet=vary hue,
  main legend={up then right, columns=3}]
data group {sin functions};
\end{codeexample}
\end{key}

\begin{key}{/tikz/data visualization/legend options/up then left}
\begin{codeexample}[width=6cm]
\tikz \datavisualization [
  visualize as smooth line/.list={1,2,3,4,5,6,7,8},
  legend example, style sheet=vary hue,
  main legend={up then left, columns=3}]
data group {sin functions};
\end{codeexample}
\end{key}


\begin{key}{/tikz/data visualization/legend options/left then up}
\begin{codeexample}[width=6cm]
\tikz \datavisualization [
  visualize as smooth line/.list={1,2,3,4,5,6,7,8},
  legend example, style sheet=vary hue,
  main legend={left then up, columns=3}]
data group {sin functions};
\end{codeexample}
\end{key}

\begin{key}{/tikz/data visualization/legend options/left then down}
\begin{codeexample}[width=6cm]
\tikz \datavisualization [
  visualize as smooth line/.list={1,2,3,4,5,6,7,8},
  legend example, style sheet=vary hue,
  main legend={left then down, columns=3}]
data group {sin functions};
\end{codeexample}
\end{key}

\begin{key}{/tikz/data visualization/legend options/right then up}
\begin{codeexample}[width=6cm]
\tikz \datavisualization [
  visualize as smooth line/.list={1,2,3,4,5,6,7,8},
  legend example, style sheet=vary hue,
  main legend={right then up, columns=3}]
data group {sin functions};
\end{codeexample}
\end{key}

\begin{key}{/tikz/data visualization/legend options/right then down}
\begin{codeexample}[width=6cm]
\tikz \datavisualization [
  visualize as smooth line/.list={1,2,3,4,5,6,7,8},
  legend example, style sheet=vary hue,
  main legend={right then down, columns=3}]
data group {sin functions};
\end{codeexample}
\end{key}


Having configured the directions in which the matrix is being filled,
you must next setup the number of rows or columns that are to be
shown. There are actually two different ways of doing so. The first
way is to specify a maximum number of rows or columns. For instance,
you might specify that there should be at most ten rows to a column
and when there are more, a new column should be begun. This is
achieved using the following keys:

\begin{key}{/tikz/data visualization/legend options/max rows=\meta{number}}
  As the legend matrix is being filled, whenever the number of rows in
  the current column would exceed \meta{number}, a new column is
  started.
\begin{codeexample}[width=7cm]
\tikz \datavisualization [
  visualize as smooth line/.list={1,2,3,4,5,6,7,8},
  legend example, style sheet=vary hue,
  main legend={max rows=3}]
data group {sin functions};
\end{codeexample}  
\begin{codeexample}[width=7cm]
\tikz \datavisualization [
  visualize as smooth line/.list={1,2,3,4,5,6,7,8},
  legend example, style sheet=vary hue,
  main legend={max rows=4}]
data group {sin functions};
\end{codeexample}  
\begin{codeexample}[width=7cm]
\tikz \datavisualization [
  visualize as smooth line/.list={1,2,3,4,5,6,7,8},
  legend example, style sheet=vary hue,
  main legend={max rows=5}]
data group {sin functions};
\end{codeexample}  
\end{key}


\begin{key}{/tikz/data visualization/legend options/max columns=\meta{number}}
  This key works like |max rows|, only now the number of columns is
  monitored. Note that this strategy only really makes sense when the
  when you use this key with a strategy that first goes left or right
  and then up or down.
\begin{codeexample}[width=7cm]
\tikz \datavisualization [
  visualize as smooth line/.list={1,2,3,4,5,6,7,8},
  legend example, style sheet=vary hue,
  main legend={right then down, max columns=2}]
data group {sin functions};
\end{codeexample}  
\begin{codeexample}[width=7cm]
\tikz \datavisualization [
  visualize as smooth line/.list={1,2,3,4,5,6,7,8},
  legend example, style sheet=vary hue,
  main legend={right then down,max columns=3}]
data group {sin functions};
\end{codeexample}  
\begin{codeexample}[width=7cm]
\tikz \datavisualization [
  visualize as smooth line/.list={1,2,3,4,5,6,7,8},
  legend example, style sheet=vary hue,
  main legend={right then down,max columns=4}]
data group {sin functions};
\end{codeexample}  
\end{key}


The second way of specifying the number of entries in a row or column
is to specify an ``ideal number of rows or columns.'' The idea is as
follows: Suppose that we use the standard strategy and would like to
have everything in two columns. Then if there are eight entries, the
first four should go to the first column, while the next four should
go to the second column. If we have 20 entries, the first ten should
go the first column and the next ten to the second, and so on. So, in
general, the objective is to distribute the entries evenly so the this
``ideal number of columns'' is reached. Only when there are too few
entries to achieve this or when the number of entries per column would
exceed the |max rows| value, will the number of columns deviate from
this ideal value.



\begin{key}{/tikz/data visualization/legend options/ideal number of columns=\meta{number}}
  Specifies, that the entries should be split into \meta{number}
  different columns, whenever possible. However, when there would be
  more than the |max rows| value of rows per column, more columns than
  the ideal number are created.
\begin{codeexample}[width=7cm]
\tikz \datavisualization [
  visualize as smooth line/.list={1,2,3,4,5,6,7,8},
  legend example, style sheet=vary hue,
  main legend={ideal number of columns=2}]
data group {sin functions};
\end{codeexample}  
\begin{codeexample}[width=7cm]
\tikz \datavisualization [
  visualize as smooth line/.list={1,2,3,4,5,6,7,8},
  legend example, style sheet=vary hue,
  main legend={ideal number of columns=4}]
data group {sin functions};
\end{codeexample}  
\begin{codeexample}[width=7cm]
\tikz \datavisualization [
  visualize as smooth line/.list={1,2,3,4,5,6,7,8},
  legend example, style sheet=vary hue,
  main legend={max rows=3,ideal number of columns=2}]
data group {sin functions};
\end{codeexample}  
\end{key}

\begin{key}{/tikz/data visualization/legend
    options/rows=\meta{number}}
  Shorthand for |ideal number of rows=|\meta{number}.
\end{key}


\begin{key}{/tikz/data visualization/legend options/ideal number of rows=\meta{number}}
  Works like |ideal number of columns|.
\begin{codeexample}[width=7cm]
\tikz \datavisualization [
  visualize as smooth line/.list={1,2,3,4,5,6,7,8},
  legend example, style sheet=vary hue,
  main legend={ideal number of rows=2}]
data group {sin functions};
\end{codeexample}  
\begin{codeexample}[width=7cm]
\tikz \datavisualization [
  visualize as smooth line/.list={1,2,3,4,5,6,7,8},
  legend example, style sheet=vary hue,
  main legend={ideal number of rows=4}]
data group {sin functions};
\end{codeexample}  
\begin{codeexample}[width=7cm]
\tikz \datavisualization [
  visualize as smooth line/.list={1,2,3,4,5,6,7,8},
  legend example, style sheet=vary hue,
  main legend={max columns=3,ideal number of rows=2}]
data group {sin functions};
\end{codeexample}  
\end{key}

\begin{key}{/tikz/data visualization/legend
    options/columns=\meta{number}}
  Shorthand for |ideal number of columns=|\meta{number}.
\end{key}


\subsubsection{Legend Placement: The General Mechanism}

A legend can either be placed next to the data visualization or inside
the data visualization at some place where there are no data
entries. Both approached have advantages: Placing the legend next to
the visualization minimises the ``cluttering'' by keeping all the
extra information apart from the actual data, while placing the legend
inside the visualization minimises the distance between the data sets
and their explanations, making it easier for the eye to connect them.

For both approaches there are options that make the placement easier,
see Sections \ref{section-dv-legend-outside}
and~\ref{section-dv-legend-inside}, but these options internally just
map to the following two options:

\begin{key}{/tikz/data visualization/legend
    options/anchor=\meta{anchor}}
  The whole legend is a \tikzname-matrix internally. Thus,
  in particular, it is stored in a node, which has anchors. Like for
  any other node, when the node is shown, the node is shifted in such
  a way that the \meta{anchor} of the node lies at the current |at|
  position. 
\end{key}

\begin{key}{/tikz/data visualization/legend
    options/at=\meta{coordinate}}
  Configures the \meta{coordinate} at which the \meta{anchor} of the
  legend's node should lie.

  It may seem hard to predict a good \meta{coordinate} for a legend
  since, depending of the size of the axis, different positions need
  to the chosen for the legend. However, it turns out that one
  can often use the coordinates of the special nodes
  |data bounding box| and |data visualization bounding box|,
  documented in Section~\ref{section-dv-bounding-box}.
  
  As an example, let us put a legend to the right of the
  visualization, but so that the first entry starts at the top of the
  visualization: 
\begin{codeexample}[width=8cm]
\tikz \datavisualization [
  scientific axes, x axis={label=$x$},
  visualize as smooth line/.list=
    {log, lin, squared, exp},
  legend={anchor=north west, at=
    (data visualization bounding box.north east)},
  log=    {label in legend={text=$\log x$}},
  lin=    {label in legend={text=$x/2$}},
  squared={label in legend={text=$x^2$}},
  exp=    {label in legend={text=$e^x$}},
  style sheet=vary dashing]
data group {function classes};
\end{codeexample}
  As can be seen, a bit of an additional shift might have been in
  order, but the result is otherwise quite satisfactory.
\end{key}


\subsubsection{Legend Placement: Outside to the Data Visualization}
\label{section-dv-legend-outside}

The following keys make it easy to place a legend outside the data
visualization. 

\begin{key}{/tikz/data visualization/legend options/east outside}
  Placing the legend to the right of the data visualization is the default:
\begin{codeexample}[width=8cm]
\tikz \datavisualization [
  scientific axes, 
  visualize as smooth line/.list=
    {log, lin, squared, exp},
  legend=east outside,
  log=    {label in legend={text=$\log x$}},
  lin=    {label in legend={text=$x/2$}},
  squared={label in legend={text=$x^2$}},
  exp=    {label in legend={text=$e^x$}},
  style sheet=strong colors]
data group {function classes};
\end{codeexample}  

  \begin{key}{/tikz/data visualization/legend options/right}
    This is an easier-to-remember alias.
  \end{key}    
\end{key}

\begin{key}{/tikz/data visualization/legend options/north east outside}
  A variant, where the legend is to the right, but aligned with the
  northern end of the data visualization:
\begin{codeexample}[width=8cm]
\tikz \datavisualization [
  scientific axes, 
  visualize as smooth line/.list=
    {log, lin, squared, exp},
  legend=north east outside,
  log=    {label in legend={text=$\log x$}},
  lin=    {label in legend={text=$x/2$}},
  squared={label in legend={text=$x^2$}},
  exp=    {label in legend={text=$e^x$}},
  style sheet=strong colors]
data group {function classes};
\end{codeexample}  
\end{key}

\begin{key}{/tikz/data visualization/legend options/south east outside}
\begin{codeexample}[width=8cm]
\tikz \datavisualization [
  scientific axes, 
  visualize as smooth line/.list=
    {log, lin, squared, exp},
  legend=south east outside,
  log=    {label in legend={text=$\log x$}},
  lin=    {label in legend={text=$x/2$}},
  squared={label in legend={text=$x^2$}},
  exp=    {label in legend={text=$e^x$}},
  style sheet=strong colors]
data group {function classes};
\end{codeexample}  
\end{key}

\begin{key}{/tikz/data visualization/legend options/west outside}
  The legend is placed left. Note that the text also swaps its
  position. 
\begin{codeexample}[width=8cm]
\tikz \datavisualization [
  scientific axes, 
  visualize as smooth line/.list=
    {log, lin, squared, exp},
  legend=west outside,
  log=    {label in legend={text=$\log x$}},
  lin=    {label in legend={text=$x/2$}},
  squared={label in legend={text=$x^2$}},
  exp=    {label in legend={text=$e^x$}},
  style sheet=strong colors]
data group {function classes};
\end{codeexample}  
  \begin{key}{/tikz/data visualization/legend options/left}
    This is an easier-to-remember alias.
  \end{key}    
\end{key}

\begin{key}{/tikz/data visualization/legend options/north west outside}
\begin{codeexample}[width=8cm]
\tikz \datavisualization [
  scientific axes, 
  visualize as smooth line/.list=
    {log, lin, squared, exp},
  legend=north west outside,
  log=    {label in legend={text=$\log x$}},
  lin=    {label in legend={text=$x/2$}},
  squared={label in legend={text=$x^2$}},
  exp=    {label in legend={text=$e^x$}},
  style sheet=strong colors]
data group {function classes};
\end{codeexample}  
\end{key}

\begin{key}{/tikz/data visualization/legend options/south west outside}
\begin{codeexample}[width=8cm]
\tikz \datavisualization [
  scientific axes, 
  visualize as smooth line/.list=
    {log, lin, squared, exp},
  legend=south west outside,
  log=    {label in legend={text=$\log x$}},
  lin=    {label in legend={text=$x/2$}},
  squared={label in legend={text=$x^2$}},
  exp=    {label in legend={text=$e^x$}},
  style sheet=strong colors]
data group {function classes};
\end{codeexample}  
\end{key}


\begin{key}{/tikz/data visualization/legend options/north outside}
  The legend is placed above the data. Note that the legend entries
  now for a row rather than a column.
\begin{codeexample}[width=8cm]
\tikz \datavisualization [
  scientific axes, 
  visualize as smooth line/.list=
    {log, lin, squared, exp},
  legend=north outside,
  log=    {label in legend={text=$\log x$}},
  lin=    {label in legend={text=$x/2$}},
  squared={label in legend={text=$x^2$}},
  exp=    {label in legend={text=$e^x$}},
  style sheet=strong colors]
data group {function classes};
\end{codeexample}  
  \begin{key}{/tikz/data visualization/legend options/above}
    This is an easier-to-remember alias.
  \end{key}    
\end{key}

\begin{key}{/tikz/data visualization/legend options/south outside}
\begin{codeexample}[width=8cm]
\tikz \datavisualization [
  scientific axes, 
  visualize as smooth line/.list=
    {log, lin, squared, exp},
  legend=south outside,
  log=    {label in legend={text=$\log x$}},
  lin=    {label in legend={text=$x/2$}},
  squared={label in legend={text=$x^2$}},
  exp=    {label in legend={text=$e^x$}},
  style sheet=strong colors]
data group {function classes};
\end{codeexample}  
  \begin{key}{/tikz/data visualization/legend options/below}
    This is an easier-to-remember alias.
  \end{key}    
\end{key}



\subsubsection{Legend Placement: Inside to the Data Visualization}
\label{section-dv-legend-inside}

There are two sets of options for placing a legend directly inside a
data visualization: First, there are options for placing it inside,
but next to some part of the border. Second, there are options for
positioning it relative to a coordinate given by a certain data point.



\begin{key}{/tikz/data visualization/legend options/south east inside}
  Puts the legend in the upper right corner of the data.
\begin{codeexample}[width=8cm]
\tikz \datavisualization [
  scientific axes, 
  visualize as smooth line/.list=
    {log, lin},
  legend=south east inside,
  log=    {label in legend={text=$\log x$}},
  lin=    {label in legend={text=$x/2$}},
  style sheet=strong colors]
data group {function classes};
\end{codeexample}  

  Note that the text is now a little smaller since there tends to be
  much less space inside the data visualization than next to it. Also,
  the legend's node is filled in white by default to ensures that the
  legend is clearly legible even in the presence of, say, a grid or
  data points behind it. This behaviour is triggered by the following
  style key:

  \begin{stylekey}{/tikz/data visualization/legend options/every legend inside}
    Executed the keys |opaque| by default and sets the  text size to
    the size of footnotes.
  \end{stylekey}
\end{key}

In order to further configure the default appearance of an inner
legend, the following keys might be useful:

\begin{key}{/tikz/data visualization/legend
    options/opaque=\meta{color} (default white)}
  When this key is used, the legend's node will be filled with the
  \meta{color} and its corners will be rounded. Additionally, the
  inner and outer separations will be set to sensible values.  
\end{key}
\begin{key}{/tikz/data visualization/legend
    options/transparent}
  Sets the filling of the legend node to |none|.
\end{key}

The following keys work much the same way as |south east inside|:

\begin{key}{/tikz/data visualization/legend options/east inside}
\end{key}
\begin{key}{/tikz/data visualization/legend options/north east inside}
\end{key}
\begin{key}{/tikz/data visualization/legend options/south west inside}
\end{key}
\begin{key}{/tikz/data visualization/legend options/west inside}
\end{key}
\begin{key}{/tikz/data visualization/legend options/north west inside}
\end{key}

The keys |south inside| and |north inside| are a bit different: They use a row
rather than a column for the legend entries:

\begin{key}{/tikz/data visualization/legend options/south inside}
  Puts the legend in the upper right corner of the data. Note that the
  text is now a little smaller since there tends to be much less space
  inside the data visualization than next to it.
\begin{codeexample}[width=8cm]
\tikz \datavisualization [
  scientific axes, 
  visualize as smooth line/.list={log, lin},
  legend=south inside,
  log=    {label in legend={text=$\log x$}},
  lin=    {label in legend={text=$x/2$}},
  style sheet=strong colors]
data group {function classes};
\end{codeexample}  
\end{key}

\begin{key}{/tikz/data visualization/legend options/north inside}
  As above.
\end{key}

The above keys do not always give you as fine a control as you may
need over the placement of the legend. In such cases, the following
keys may help (or you can revert to directly setting the |at| and the
|anchor| keys):

\begin{key}{/tikz/data visualization/legend options/at
    values=\meta{data point}}
  This key allows you to specify the desired center of the legend in
  terms of a data point. The \meta{data point} should be a list of
  comma-separated key--value pairs that specify a data point. The
  legend will then be centered at this data point.
\begin{codeexample}[width=6cm]
\tikz \datavisualization [
  scientific axes, 
  visualize as smooth line/.list={log, lin},
  legend={at values={x=-1, y=2}},
  log=    {label in legend={text=$\log x$}},
  lin=    {label in legend={text=$x/2$}},
  style sheet=strong colors]
data group {function classes};
\end{codeexample}    
\end{key}

\begin{key}{/tikz/data visualization/legend options/right
    of=\meta{data point}} 
  Works like |at values|, but the anchor is set to |west|:
\begin{codeexample}[width=6cm]
\tikz \datavisualization [
  scientific axes, 
  visualize as smooth line/.list={log, lin},
  legend={right of={x=-1, y=2}},
  log=    {label in legend={text=$\log x$}},
  lin=    {label in legend={text=$x/2$}},
  style sheet=strong colors]
data group {function classes};
\end{codeexample}    
\end{key}

The following keys work similarly:
\begin{key}{/tikz/data visualization/legend options/above right of=\meta{data point}}   
\end{key}
\begin{key}{/tikz/data visualization/legend options/above of=\meta{data point}} 
\end{key}
\begin{key}{/tikz/data visualization/legend options/above left of=\meta{data point}} 
\end{key}
\begin{key}{/tikz/data visualization/legend options/left of=\meta{data point}} 
\end{key}
\begin{key}{/tikz/data visualization/legend options/below left of=\meta{data point}} 
\end{key}
\begin{key}{/tikz/data visualization/legend options/below of=\meta{data point}} 
\end{key}
\begin{key}{/tikz/data visualization/legend options/below right of=\meta{data point}} 
\end{key}




\subsubsection{Legend Entries: General Styling}

\label{section-dv-label-legend-entry-options}

The entries in a legend can be styled in several ways:

\begin{itemize}
\item
  You can configure the styling of the text node.
\item
  You can configure the relative placement of the text node and the
  little picture depicting the data set's styling.
\item
  You can configure how the data set's styling is depicted.
\end{itemize}

Before we have look at how each of these are configured, in detail,
let us first have a look at the keys that allow us to save a set of
such styles:

\begin{stylekey}{/tikz/data visualization/every label in legend}
  This key is executed with every label in a legend. However, the
  options stored in this style are executed with the path prefix
  |/tikz/data visualization/legend entry options|. Thus, this key can
  use keys like |node style| to configure the styling of all text
  nodes: 
\begin{codeexample}[width=8cm]
\tikz \datavisualization [
  scientific axes,
  every label in legend/.style={node style=
    {fill=red!30}},
  visualize as smooth line/.list=
    {log, lin, squared, exp},
  legend=north east outside,
  log=    {label in legend={text=$\log x$}},
  lin=    {label in legend={text=$x/2$,
      node style={circle, draw=red}}},
  squared={label in legend={text=$x^2$}},
  exp=    {label in legend={text=$e^x$}},
  style sheet=strong colors]
data group {function classes};
\end{codeexample}
\end{stylekey}

\begin{key}{/tikz/data visualization/legend options/label style=\meta{options}}
  This key can be used with a legend. It will simply add the
  \meta{options} to the |every label in legend| style for the given
  legend. 
\begin{codeexample}[width=8cm]
\tikz \datavisualization [
  scientific axes,
  visualize as smooth line/.list=
    {log, lin, squared, exp},
  legend={label style={node style=draw}},
  log=    {label in legend={text=$\log x$}},
  lin=    {label in legend={text=$x/2$,
      node style={circle, draw=red}}},
  squared={label in legend={text=$x^2$}},
  exp=    {label in legend={text=$e^x$}},
  style sheet=strong colors]
data group {function classes};
\end{codeexample}
\end{key}


\subsubsection{Legend Entries: Styling the Text Node}

The appearance of the text nodes is easy to configure. 

\begin{key}{/tikz/data visualization/legend entry options/node style=\meta{options}}
  This key adds \meta{options} to the styling of the text nodes of the
  label. 
\begin{codeexample}[width=8cm]
\tikz \datavisualization [
  scientific axes, 
  visualize as smooth line/.list=
    {log, lin, squared, exp},
  legend=north east outside,
  log=    {label in legend={text=$\log x$}},
  lin=    {label in legend={text=$x/2$,
      node style={circle, draw=red}}},
  squared={label in legend={text=$x^2$}},
  exp=    {label in legend={text=$e^x$}},
  style sheet=strong colors]
data group {function classes};
\end{codeexample}
\end{key}

\begin{key}{/tikz/data visualization/legend entry options/text colored}
  Causes the |node style| to set the text color to
  |visualizer color|. The effect of this is that the label's text
  will have the same color as the data set to which it is attached.
\begin{codeexample}[width=8cm]
\tikz \datavisualization [
  scientific axes, 
  visualize as smooth line/.list=
    {log, lin, squared, exp},
  legend={label style=text colored},
  log=    {label in legend={text=$\log x$}},
  lin=    {label in legend={text=$x/2$}},
  squared={label in legend={text=$x^2$}},
  exp=    {label in legend={text=$e^x$}},
  style sheet=strong colors]
data group {function classes};
\end{codeexample}
\end{key}


\subsubsection{Legend Entries: Text Placement}

Three keys govern where the text will be placed relative to the data
set style visualization.

\begin{key}{/tikz/data visualization/legend entry options/text right}
  Placed the text node to the right of the data set style
  visualization. This is the default for most, but not all, legends.
\end{key}
\begin{key}{/tikz/data visualization/legend entry options/text left}
  Placed the text node to the left of the data set style
  visualization. 
\begin{codeexample}[width=8cm]
\tikz \datavisualization [
  scientific axes, 
  visualize as smooth line/.list=
    {log, lin, squared, exp},
  legend={label style=text left},
  log=    {label in legend={text=$\log x$}},
  lin=    {label in legend={text=$x/2$}},
  squared={label in legend={text=$x^2$}},
  exp=    {label in legend={text=$e^x$}},
  style sheet=strong colors]
data group {function classes};
\end{codeexample}
\end{key}
\begin{key}{/tikz/data visualization/legend entry options/text only}
  Shows only the text nodes and no data set style visualization at
  all. This options only makes sense in conjunction with the
  |text colored| options, which is why this options is also selected
  implicitly. 
\begin{codeexample}[width=8cm]
\tikz \datavisualization [
  scientific axes, 
  visualize as smooth line/.list=
    {log, lin, squared, exp},
  legend={south east inside, rows=2,
          label style=text only},
  log=    {label in legend={text=$\log x$}},
  lin=    {label in legend={text=$x/2$}},
  squared={label in legend={text=$x^2$}},
  exp=    {label in legend={text=$e^x$}},
  style sheet=strong colors]
data group {function classes};
\end{codeexample}
\end{key}





\subsubsection{Advanced: Labels in Legends and Their Visualizers}

\label{section-dv-legend-entries}

The following explanations are important only for you if you intend to
create a new visualizer and an accompanying label in legend
visualizer; otherwise you can safely proceed with the next section.

A legend entry consists not only of some explaining text, but, even
more importantly, of a visual representation of the style used for the
data points, created by a \emph{label in legend visualizer}. For
instance, when data points are visualized as lines in 
different colors, the legend entry for the first line might consist of
the text ``first experiment'' and a short line in black and the second
entry might consist of ``failed experiment'' and a short line in red
-- assuming, of course, that the style sheet makes the first line
black and the second line blue. As another example, when data sets are
visualized as clouds of plot marks, the texts in the legend would be
accompanied by the plot marks used to visualize the data sets.

For every visualizer, the \emph{label in legend visualizer} creates an
appropriate visualization of the data set's styling. There may be more
than one possible such label in legend visualizer that is appropriate,
in which case options are used to choose between them.

Let us start with the key for creating a new legend entry. This key
gets called for instance by |label in legend|:

\begin{key}{/tikz/data visualization/new legend entry=\meta{options}}
  This key will add a new entry to the legend that is identified by
  the \meta{options}. For this, the \meta{options} are executed once
  with the path prefix |/tikz/data visualization/legend entry options|
  and the resulting setting of the |legend| key is used to pick which
  legend the new entry should belong to. Then, the \meta{options} are
  stored away for the time being.

  Later, when the legend is created, the \meta{options} get executed
  once more. This time, however, the |legend| key is no longer
  important. Instead, the \meta{options} that setup keys like
  |text| or |visualizer in legend| now play a role.

  In detail, the following happens:
  \begin{itemize}
  \item For the legend entry, a little cell picture is created in the
    matrix of the legend (see Section~\ref{section-tikz-cell-pictures}
    for details on cell pictures).
  \item Inside this picture, a node is created whose text is taken
    from the key
\begin{codeexample}[code only]
/tikz/data visualization/legend entry options/text      
\end{codeexample}
  \item Also inside the picture, the code stored in the following key
    gets executed:
    \begin{key}{/tikz/data visualization/legend entry options/visualizer in legend}
      Set this key to some code that paints something in the cell
      picture. Typically, this will be a visual representation of the
      data set styling, but it could also be something different.
\begin{codeexample}[width=6cm]
\tikz \datavisualization [
  school book axes, visualize as line/.list={a,b},
  style sheet=vary dashing,
  a={label in legend={text=a}},
  new legend entry={
    text=spacer,
    visualizer in legend={\draw[solid] (0,0) circle[radius=2pt];}
  },
  b={label in legend={text=b}}]
data point [x=-1, y=-1, set=a]   data point [x=1, y=0, set=a]
data point [x=-1, y=1,  set=b]   data point [x=1, y=0.5, set=b];
\end{codeexample}
    \end{key}
  \end{itemize}
  The following styles are applied in the following order before the
  cell picture is filled:
  \begin{enumerate}
  \item |/tikz/data visualization/every data set label| with path
    |/tikz/data visualization|
  \item |/tikz/data visualization/every label in legend| with path\\
    |/tikz/data visualization/legend entry options|.
  \item The \meta{options}.
  \item The code in the following key:
    \begin{key}{/tikz/data visualization/legend entry options/setup}
      Some code to be executed at this point. Mostly, it is used to
      setup attributes for style sheets.
    \end{key}
  \item A styling signal is emitted.
  \item Only for the node: The current value of |node style|.
  \item Only for the visualizer in legend: The styling that has been
    accumulated by calls to the following key:
    \begin{stylekey}{/tikz/data visualization/legend entry
        options/visualizer in legend style=\\\marg{options}}
      Calls to this key accumulate \meta{options} that will be
      executed with the path prefix |/tikz| at this point.
    \end{stylekey}
  \end{enumerate}
\end{key}

As indicated earlier, the |new legend entry| key is called by the
|label in legend=|\meta{options} key internally. In this case, the
following extra \meta{extra options} are passed to |new legend entry|
key:
\begin{itemize}
\item The styling of the visualizer.
\item The |/tikz/data visualization/every label in legend| style.
\item The |/tikz/every label| style with path |/tikz|.
\item Setting |setup| to |/data point/set=|\meta{name of the visualizer}.
\item The value of the |label legend options| that are stored in the
  visualizer. These options can be changed using the following key:
  \begin{key}{/tikz/data visualization/visualizer options/label in
      legend options=\meta{options}}
    Use this key with a visualizer to configure the label in legend
    options. Typically, this key is used only internally by a
    visualizer upon its creating to set the \meta{options} to setup
    the |visualizer in legend| key.
  \end{key}
\end{itemize}


\subsubsection{Reference: Label in Legend Visualizers for Lines and Scatter Plots}

Visualizers like |visualize as line| or |visualize as smooth line|
use a label in legend visualizer that draws a short line to represent
the data set inside the legend. However, this line needs not be a
simple straight line, but can be a little curve or a small circle --
indeed, even the default line is not a simple straight line but rather
a small zig-zag curve. To configure this line, the two keys
are used, although you will only rarely use them directly, but
rather use one of the predefined styles mentioned later on.

Before we go into the glorious details of all of these keys, let us
first have a look at the keys you are most likely to use in practice:
The keys for globally reconfiguring the default label in legend
visualizers:
\begin{stylekey}{/tikz/data visualization/legend entry options/default
    label in legend path}
  This style is set, by default, to |zig zag label in legend line|. It
  is installed by the styles |straight line|, |smooth line|, and
  |gap line|, so changing this style will change the appearance of lines in
  legends. The main other sensible option for this key is
  |straight label in legend line|.
\begin{codeexample}[width=5cm]
\tikz \datavisualization [
  school book axes, visualize as line/.list={a,b},
  style sheet=vary dashing,
  a={label in legend={text=a}},  b={label in legend={text=b}}]
data point [x=-1, y=-1, set=a]   data point [x=1, y=0, set=a]
data point [x=-1, y=1,  set=b]   data point [x=1, y=0.5, set=b];
\end{codeexample}
\begin{codeexample}[width=5cm]
\tikz \datavisualization [
  school book axes, visualize as line/.list={a,b},
  legend entry options/default label in legend path/.style=
    straight label in legend line,
  style sheet=vary dashing,
  a={label in legend={text=a}},  b={label in legend={text=b}}]
data point [x=-1, y=-1, set=a]   data point [x=1, y=0, set=a]
data point [x=-1, y=1,  set=b]   data point [x=1, y=0.5, set=b];
\end{codeexample}
\end{stylekey}
\begin{stylekey}{/tikz/data visualization/legend entry options/default
    label in legend closed path}
  This style is executed by |smooth cycle| and |straight cycle|. There
  are (currently) no other predefined sets of coordinates that can be
  used instead of the default value |circular label in legend line|.
\end{stylekey}

\begin{stylekey}{/tikz/data visualization/legend entry options/default
    label in legend mark}
  This style is executed by |no lines| and, implicitly, by scatter
  plots. The default is to use
  |label in legend line one mark|. Another possible value is
  |label in legend line three marks|.
\begin{codeexample}[width=5cm]
\tikz \datavisualization [
  visualize as scatter/.list={a,b,c}, 
  style sheet=cross marks,
  legend entry options/default label in legend mark/.style=
    label in legend three marks,
  a={label in legend={text=example a}},
  b={label in legend={text=example b}},
  c={label in legend={text=example c}}];
\end{codeexample}
\end{stylekey}

\begin{key}{/tikz/data visualization/legend entry options/label in
    legend line coordinates=\\\marg{list of coordinates}}
  This key takes a \meta{list of coordinates}, which are
  \tikzname-coordinates separated by commas like |(0,0),|\penalty0|(1,1)|. The
  effect of setting the key is the following: The label in legend
  visualizer used by, for instance, |visualize as line| will draw a
  path going through these points. When the line is drawn, the exact
  same style will be used as was used for the data set. For instance,
  if the |smooth line| key was used and also the |style=red| key, the
  line through the \meta{list of coordinates} will also be red and
  smooth. When the |straight cycle| key was used, the coordinates will
  also be connected by a cycle, and so on.

  When the line connecting the \meta{list of coordinates} is drawn,
  the coordinate system will have been shifted and transformed in such
  a way that |(0,0)| lies to the left of the text and at half the
  height of the character ``x''. This means that the right-most-point
  in the list should usually be |(0,0)| and all other $x$-coordinates
  should usually be negative. When the |text left| options is used,
  the coordinate system will have been flipped, so the \meta{list of
    coordinates} is independent of whether the text is to the right or
  to the left of the line.

  Let us now have a look at a first, simple example. We create a
  legend entry that is just a straight line, so it should start
  somewhere to the left of the origin at height $0$ and go to the
  origin:
\begin{codeexample}[width=5cm]
\tikz \datavisualization [
  school book axes, visualize as line/.list={a,b},
  style sheet=vary dashing,
  a={label in legend={text=a,
      label in legend line coordinates={(-1em,0), (0,0)}}},
  b={label in legend={text=b,
      label in legend line coordinates={(-2em,0), (0,0)}}}]
data point [x=-1, y=-1, set=a]   data point [x=1, y=0, set=a]
data point [x=-1, y=1,  set=b]   data point [x=1, y=0.5, set=b];
\end{codeexample}

  Now let us make this a bit more fancy and useful by using shifted lines:
\begin{codeexample}[width=5cm]
\tikz \datavisualization [
  school book axes, visualize as line/.list={a,b},
  legend={up then right}, style sheet=vary dashing,
  a={label in legend={text=a,
      label in legend line coordinates={(-2em,-.25ex), (0,0)}}},
  b={label in legend={text=b,
      label in legend line coordinates={(-2em,.25ex), (0,0)}}}]
data point [x=-1, y=-1, set=a]   data point [x=1, y=0, set=a]
data point [x=-1, y=1,  set=b]   data point [x=1, y=0.5, set=b];
\end{codeexample}

  In the final example, we use a little ``hat'' to represent lines:
\begin{codeexample}[width=5cm]
\tikz \datavisualization [
  school book axes, visualize as line/.list={a,b},
  legend={up then right}, style sheet=vary dashing,
  a={label in legend={text=a,
      label in legend line coordinates={
        (-2em,-.2ex), (-1em,.2ex), (0,-.2ex)}}},
  b={label in legend={text=b,
      label in legend line coordinates={
        (-2em,-.2ex), (-1em,.2ex), (0,-.2ex)}}}]
data point [x=-1, y=-1, set=a]   data point [x=1, y=0, set=a]
data point [x=-1, y=1,  set=b]   data point [x=1, y=0.5, set=b];
\end{codeexample}
\end{key}

\begin{key}{/tikz/data visualization/legend entry options/label in
    legend mark coordinates=\\\marg{list of coordinates}}
  This key is similar to |label in legend line coordinates|, but now
  the \meta{list of coordinates} is used as the positions where plot
  marks are shown. Naturally, plot marks are only shown there if they
  are also shown by the visualizer in the actual data -- just like the
  line through the coordinates of the previous key is only shown when
  there is a line.

  The \meta{list of coordinates} may be the same as the one used for
  lines, but usually it is not. In general, it is better to have marks
  for instance not at the ends of the line.
\begin{codeexample}[width=5cm]
\tikz \datavisualization [
  school book axes, visualize as line/.list={a,b},
  legend={up then right},
  style sheet=vary dashing,
  style sheet=cross marks,
  a={label in legend={text=a,
      label in legend line coordinates={
        (-2em,-.2ex), (-1em,.2ex), (0,-.2ex)},
      label in legend mark coordinates={
        (-1em,.2ex)}}},
  b={label in legend={text=b,
      label in legend line coordinates={
        (-2em,-.2ex), (-1em,.2ex), (0,-.2ex)},
      label in legend mark coordinates={
        (-2em,-.2ex), (0,-.2ex)}}}]
data point [x=-1, y=-1, set=a]   data point [x=1, y=0, set=a]
data point [x=-1, y=1,  set=b]   data point [x=1, y=0.5, set=b];
\end{codeexample}
\end{key}



Naturally, you typically will not give coordinates explicitly for each
label, but use one of the following styles:

\begin{key}{/tikz/data visualization/legend entry options/straight label in legend line}
  Just gives a straight line and two plot marks.
\begin{codeexample}[width=5cm]
\tikz \datavisualization [visualize as line,    
  line={style={mark=x}, label in legend={text=example, 
    straight label in legend line}}];
\end{codeexample}
  This style might seem like a good idea to use in general, but it
  does have a huge drawback: Some commonly used plot marks will be impossible to
  distinguish -- even though there is no problem distinguishing them
  in a graph.
\begin{codeexample}[width=5cm]
\tikz \datavisualization [visualize as line/.list={a,b,c},    
  legend entry options/default label in legend path/.style=
    straight label in legend line,
  a={style={mark=+}, label in legend={text=bad example a}},
  b={style={mark=-}, label in legend={text=bad example b}},
  c={style={mark=|}, label in legend={text=bad example c}}];
\end{codeexample}
  For this reason, this option is not the default, but rather the next one.
\end{key}

\begin{key}{/tikz/data visualization/legend entry options/zig zag label in legend line}
  Uses a small up-down-up line as the label in legend visualizer. The
  two plot marks are at the extremal points of the line. It works
  pretty well in almost all situations and is the default.
\begin{codeexample}[width=5cm]
\tikz \datavisualization [
  visualize as line=a,
  visualize as smooth line/.list={b,c},    
  a={style={mark=+}, label in legend={text=better example a}},
  b={style={mark=-}, label in legend={text=better example b}},
  c={style={mark=|}, label in legend={text=better example c}}];
\end{codeexample}
  Even though the above example shows that the marks are easier to
  distinguish than with a straight line, the chosen marks are still
  not optimal. This is the reason that the |cross marks| style uses
  different crosses:
\begin{codeexample}[width=5cm]
\tikz \datavisualization [
  visualize as line/.list={a,b},
  visualize as smooth line=c, 
  style sheet=cross marks,
  a={label in legend={text=good example a}},
  b={label in legend={text=good example b}},
  c={gap line, label in legend={text=good example c}}];
\end{codeexample}
\end{key}


\begin{key}{/tikz/data visualization/legend entry options/circular label in legend line}
  This style is especially tailored to represent lines that are
  closed. It is automatically selected for instance by the |polygon|
  or the |smooth cycle| styles.
\begin{codeexample}[width=7cm]
\tikz \datavisualization [
  scientific axes={clean}, all axes={length=3cm},
  visualize as line/.list={a,b,c},
  a={polygon}, b={smooth cycle}, 
  style sheet=cross marks,
  a={label in legend={text=polygon}},
  b={label in legend={text=circle}},
  c={label in legend={text=line}}]
data [format=function, set=a] {
  var t : {0,72,...,359};
  func x = cos(\value t);
  func y = sin(\value t);
}
data [format=function, set=b] {
  var t : [0:2*pi];
  func x = .8*cos(\value t r);
  func y = .8*sin(\value t r);
}
data point [x=-1, y=0.5, set=c]
data point [x=1,  y=0.25, set=c];
\end{codeexample}
\end{key}


\begin{key}{/tikz/data visualization/legend entry options/gap circular label in legend line}
  This style is especially tailored to for the |gap cycle| style and
  automatically selected by it:
\begin{codeexample}[width=7cm]
\tikz \datavisualization [
  scientific axes={clean}, all axes={length=3cm},
  visualize as line/.list={a,b,c},
  a={gap cycle}, b={smooth cycle}, c={gap line}, 
  a={style={mark=*, mark size=0.5pt},
     label in legend={text=polygon}},
  b={label in legend={text=circle}},
  c={style={mark=*, mark size=0.5pt, mark options=red},
     label in legend={text=line}}]
data [format=function, set=a] {
  var t : {0,72,...,359};
  func x = cos(\value t);
  func y = sin(\value t);
}
data [format=function, set=b] {
  var t : [0:352];
  func x = .8*cos(\value t);
  func y = .8*sin(\value t);
}
data point [x=-1, y=0.5, set=c]
data point [x=1,  y=0.25, set=c];
\end{codeexample}
\end{key}



\begin{key}{/tikz/data visualization/legend entry options/label in legend one mark}
  To be used with scatter plots, since no line is drawn. Just displays
  a single mark (this is the default with a scatter plot or when the
  |no line| is selected.
\begin{codeexample}[width=5cm]
\tikz \datavisualization [visualize as scatter/.list={a,b,c},
   style sheet=cross marks,
  a={label in legend={text=example a}},
  b={label in legend={text=example b}},
  c={label in legend={text=example c}}];
\end{codeexample}
\end{key}

\begin{key}{/tikz/data visualization/legend entry options/label in legend three marks}
  An alternative to the previous style, where several marks are
  shown. 
\begin{codeexample}[width=5cm]
\tikz \datavisualization [visualize as scatter/.list={a,b,c},
  style sheet=cross marks,
  a={label in legend={text=example a, label in legend three marks}},
  b={label in legend={text=example b, label in legend three marks}},
  c={label in legend={text=example c, label in legend three marks}}];
\end{codeexample}
\end{key}




% Copyright 2010 by Till Tantau
%
% This file may be distributed and/or modified
%
% 1. under the LaTeX Project Public License and/or
% 2. under the GNU Free Documentation License.
%
% See the file doc/generic/pgf/licenses/LICENSE for more details.




\section{Polar Axes}

\label{section-dv-polar}


\subsection{Overview}

\begin{tikzlibrary}{datavisualization.polar}
  This library contains keys that allow you to create plots in a polar
  axis system is used. 
\end{tikzlibrary}

In a \emph{polar axis system} two attributes are visualized by
displacing a data point as follows: One attribute is used to compute a
an angle (a direction) while a second attribute is used as a radius (a
distance). The angle can be measured in degrees, radians, or can be
scaled arbitrarily.


\begin{codeexample}[width=8.5cm]
\tikz \datavisualization [
  scientific polar axes={0 to pi, clean},
  all axes=grid,
  style sheet=vary hue,
  legend=below
  ]
  [visualize as smooth line=sin,
   sin={label in legend={text=$1+\sin \alpha$}}]
  data [format=function] {
    var  angle : interval [0:pi];
    func radius = sin(\value{angle}r) + 1;
  }
  [visualize as smooth line=cos,
   cos={label in legend={text=$1+\cos\alpha$}}]
  data [format=function] {
    var  angle : interval [0:pi];
    func radius = cos(\value{angle}r) + 1;
  };
\end{codeexample}

Most of the time, in order to create a polar axis system, you will
just use the |scientific polar axes| key, which takes a number of
options that allow you to configure the axis system in greater
detail. This key is documented in
Section~\ref{section-dv-sci-polar-axes}. Internally, this key uses
more low level keys which are documented in the en suite sections.

It is worthwhile to note that the axes of a polar axis system are,
still, normal axes of the data visualization system. In particular,
all the configurations possible for, say, Cartesian axes also apply to
the ``angle axis'' and the ``radius axis'' of a polar axis system. For
instance, you can could make both axes logarithmic or style their
ticks:

\begin{codeexample}[]
\tikz[baseline] \datavisualization [
  scientific axes={clean},
  x axis={attribute=angle, ticks={minor steps between steps=4}},
  y axis={attribute=radius, ticks={some, style=red!80!black}},
  all axes=grid,
  visualize as smooth line=sin]
  data [format=function] {
    var t : interval [-3:3];
    func angle = exp(\value t);
    func radius = \value{t}*\value{t};
  };
\qquad
\tikz[baseline] \datavisualization [
  scientific polar axes={right half clockwise, clean},
  angle axis={logarithmic,
    ticks={
      minor steps between steps=8,
      major also at/.list={2,3,4,5,15,20}}},
  radius axis={ticks={some, style=red!80!black}},
  all axes=grid,
  visualize as smooth line=sin]
  data [format=function] {
    var t : interval [-3:3];
    func angle = exp(\value t);
    func radius = \value{t}*\value{t};
  }; 
\end{codeexample}


\subsection{Scientific Polar Axis System}
\label{section-dv-sci-polar-axes}

\begin{key}{/tikz/data visualization/scientific polar axes=\meta{options}}
  This key installs a polar axis system that can be used in a
  ``scientific'' publication. Two axes are created called the
  |angle axis| and the |radius axis|. Unlike ``normal'' Cartesian
  axes, these axes do not point in a specific direction. Rather, the
  |radius axis| is used to map the values of one attribute to a
  distance from the origin while the |angle axis| is used to map the
  values of another attribute to a rotation angle.
  
  The \meta{options} will be executed with the path prefix
\begin{codeexample}[code only]
/tikz/data visualization/scientific polar axes    
\end{codeexample}
  The permissible keys are documented in the later subsections of this
  section. 

  Let us start with the configuration of the radius axis since it is
  easier. Firstly, you should specify which attribute is linked to the
  radius. The default is |radius|, but you will typically wish to
  change this. As with any other axis, the |attribute| key is used to
  configure the axis, see Section~\ref{section-dv-axis-attribute} for
  details. You can also apply all other configurations to the radius
  axis like, say, |unit length| or |length| or |style|. Note, however,
  that the |logarithmic| key will not work with the radius axis for a
  |scientific polar axes| system since the attribute value zero is
  always placed at the center -- and for a logarithmic plot the value
  |0| cannot be mapped.
\begin{codeexample}[width=8.8cm]
\tikz \datavisualization [
  scientific polar axes,
  radius axis={
    attribute=distance,
    ticks={step=5000},
    padding=1.5em,
    length=3cm,
    grid
  },
  visualize as smooth line]
data [format=function] {
  var  angle : interval [0:100];
  func distance = \value{angle}*\value{angle};
};
\end{codeexample}

  For the |angle axis|, you can also specify an attribute using the
  |attribute| key. However, for this axis the mapping of a value to an
  actual angle is a complicated process involving many considerations
  of how the polar axis system should be visualized. For this reason,
  there are a large number of predefined such mappings documented in
  Section~\ref{section-dv-angle-ranges}.

  Finally, as for a |scientific plot|, you can configure where the
  ticks should be shown using the keys |inner ticks|, |outer ticks|,
  and |clean|, documented below.
\end{key}


\subsubsection{Tick Placements}

\begin{key}{/tikz/data visualization/scientific polar axes/outer ticks}
  This key, which is the default, causes ticks to be drawn
  ``outside'' the outer ``ring'' of the polar axes:
\begin{codeexample}[width=8.8cm]
\tikz \datavisualization [
  scientific polar axes={outer ticks, 0 to 180},
  visualize as smooth line]
data [format=function] {
  var  angle : interval [0:100];
  func radius = \value{angle};
};
\end{codeexample}
\end{key}

\begin{key}{/tikz/data visualization/scientific polar axes/inner ticks}
  This key causes the ticks to be ``turned to the inside.'' I do not
  recommend using this key.
\begin{codeexample}[width=8.8cm]
\tikz \datavisualization [
  scientific polar axes={inner ticks, 0 to 180},
  visualize as smooth line]
data [format=function] {
  var  angle : interval [0:100];
  func radius = \value{angle};
};
\end{codeexample}
\end{key}

\begin{key}{/tikz/data visualization/scientific polar axes/clean}
  This key separates the area where the data is shown from the area
  where the ticks are shown. Usually, this is the best choice for the
  tick placement since it avoids a collision of data and
  explanations. 
\begin{codeexample}[width=8.8cm]
\tikz \datavisualization [
  scientific polar axes={clean, 0 to 180},
  visualize as smooth line]
data [format=function] {
  var  angle : interval [0:100];
  func radius = \value{angle};
};
\end{codeexample}  
\end{key}


\subsubsection{Angle Ranges}
\label{section-dv-angle-ranges}

Suppose you create a polar plot in which the radius values vary
between, say, $567$ and $1234$. Then the normal axis scaling mechanisms 
can be used to compute a good scaling for the ``radius axis'': Place
the value $1234$ at a distance of , say, $5\,\mathrm{cm}$ from the
origin and place the value $0$ at the origin. Now, by comparison,
suppose that the values of the angle axis's attribute ranged between,
say, $10$ and $75.7$. In this case, we may wish the angles to be
scaled so that the minimum value is horizontal and the maximum value is
vertical. But we may also wish the a value of $0$ is horizontal and a
value of $90$ is vertical. 

Since it is unclear which interpretation is the right one, you have to
use an option to select which should happen. The applicable options
fall into three categories:

\begin{itemize}
\item Options that request the scaling to be done in such a way that
  the attribute is interpreted as a value in degrees and such that the
  minimum and maximum of the depicted range is a multiple of
  $90^\circ$. For instance, the option |0 to 180| causes the angle
  axis to range from $0^\circ$ to $180^\circ$, independently of the
  actual range of the values.
\item Options that work as above, but use radians rather than
  degrees. An example is the option |0 to pi|.
\item Options that map the minimum value in the data to a horizontal
  or vertical line and the maximum value to another such line. This is
  useful when the values neither directly correspond to degrees or
  radians. In this case, the angle axis may also be a logarithmic
  axis. 
\end{itemize}

In addition to the above categories, all of the option documented in
the following implicitly also select quadrants that are used to depict
the data. For instance, the |0 to 90| key and also the |0 to pi half|
key setup the polar axis system in such a way that only first (upper
right) quadrant is used. No check is done whether the data fill
actually lie in this quadrant -- if it does not, the data will ``bleed
outside'' the range. Naturally, with a key like |0 to 360| or
|0 to 2pi| this cannot happen.

In order to save some space in this manual, in the following the
different possible keys are only given in a table together with a
small example for each key. The examples were created using the
following code:

\begin{codeexample}[]
\tikz \datavisualization [
  scientific polar axes={
    clean,
    0 to 90  % the option 
  },
  angle axis={ticks={step=30}},
  radius axis={length=1cm, ticks={step=1}},
  visualize as scatter]
data point [angle=20, radius=0.5]
data point [angle=30, radius=1]
data point [angle=40, radius=1.5];
\end{codeexample}

For the options on radians, the angle values have been replaced by
|0.2|, |0.3|, and |0.4| and the stepping has been changed by setting
|step=(pi/6)|. For the quadrant options, no stepping is set at all (it
is computed automatically).

\def\polarexample#1#2#3#4#5{%
  \texttt{#1}% 
  \indexkey{/tikz/data visualization/scientific polar axes/#1}&
  \tikz [baseline]{\path(-2.25cm,0)(2.25cm,0); \datavisualization [
    scientific polar axes={clean, #1},
    angle axis={ticks={#2}},
    radius axis={length=1cm, ticks={step=1}},
    visualize as scatter
    ]
    data point [angle=#3, radius=0.5]
    data point [angle=#4, radius=1]
    data point [angle=#5, radius=1.5];
    \path ([yshift=-1em]current bounding box.south);
  }&
  \tikz [baseline]{\path(-2.25cm,0)(2.25cm,0); \datavisualization [
    scientific polar axes={outer ticks, #1},
    angle axis={ticks={#2}},
    radius axis={length=1cm, ticks={step=1}},
    visualize as scatter
    ]
    data point [angle=#3, radius=0.5]
    data point [angle=#4, radius=1]
    data point [angle=#5, radius=1.5];
    \path ([yshift=-1em]current bounding box.south);
  }
  \\
}

\begin{tabular}{lcc}
  \emph{Option} & \emph{With clean ticks} & \emph{With outer ticks} \\
  \polarexample{0 to 90}{step=30}{20}{30}{40}
  \polarexample{-90 to 0}{step=30}{20}{30}{40}
  \polarexample{0 to 180}{step=30}{20}{30}{40}
  \polarexample{-90 to 90}{step=30}{20}{30}{40}
  \polarexample{0 to 360}{step=30}{20}{30}{40}
  \polarexample{-180 to 180}{step=30}{20}{30}{40}
\end{tabular}

\begin{tabular}{lcc}
  \emph{Option} & \emph{With clean ticks} & \emph{With outer ticks} \\
  \polarexample{0 to pi half}{step=(pi/6)}{0.2}{0.3}{0.4}
  \polarexample{-pi half to 0}{step=(pi/6)}{0.2}{0.3}{0.4}
  \polarexample{0 to pi}{step=(pi/6)}{0.2}{0.3}{0.4}
  \polarexample{-pi half to pi half}{step=(pi/6)}{0.2}{0.3}{0.4}
  \polarexample{0 to 2pi}{step=(pi/6)}{0.2}{0.3}{0.4}
  \polarexample{-pi to pi}{step=(pi/6)}{0.2}{0.3}{0.4}
\end{tabular}

\begin{tabular}{lcc}
  \emph{Option} & \emph{With clean ticks} & \emph{With outer ticks} \\
  \polarexample{quadrant}{}{20}{30}{40}
  \polarexample{quadrant clockwise}{}{20}{30}{40}
  \polarexample{fourth quadrant}{}{20}{30}{40}
  \polarexample{fourth quadrant clockwise}{}{20}{30}{40}
  \polarexample{upper half}{}{20}{30}{40}
  \polarexample{upper half clockwise}{}{20}{30}{40}
  \polarexample{lower half}{}{20}{30}{40}
  \polarexample{lower half clockwise}{}{20}{30}{40}
\end{tabular}

\begin{tabular}{lcc}
  \emph{Option} & \emph{With clean ticks} & \emph{With outer ticks} \\
  \polarexample{left half}{}{20}{30}{40}
  \polarexample{left half clockwise}{}{20}{30}{40}
  \polarexample{right half}{}{20}{30}{40}
  \polarexample{right half clockwise}{}{20}{30}{40}
\end{tabular}



\subsection{Advanced: Creating a New Polar Axis System}

\begin{key}{/tikz/data visualization/new polar axes=|\char`\{|\meta{angle axis
      name}|\char`\}||\char`\{|\meta{radius axis name}|\char`\}|}
  This key actually creates two axes, whose names are give as
  parameters: An \emph{angle axis} and a \emph{radius axis}. These two
  axes work in concert in the following way: Suppose a data point has two
  attributes called |angle| and |radius| (these attribute names can be
  changed by changing the |attribute| of the \meta{angle axis name} or
  the \meta{radius axis name}, respectively). These two attributes are
  then scaled as usual, resulting in two ``reasonable'' values $a$
  (for the angle) and $r$ (for the radius). Then, the data point gets
  visualized (in principle, details will follow) at a position on the
  page that is at a distance of $r$ from the origin and at an angle
  of~$a$.  
\begin{codeexample}[]
\tikz \datavisualization
    [new polar axes={angle axis}{radius axis},
     radius axis={length=2cm},
     visualize as scatter]
  data [format=named] {
    angle={0,20,...,160}, radius={0,...,5}
  };
\end{codeexample}
  In detail, the \meta{angle axis} keeps track of two vectors $v_0$
  and $v_{90}$, each of which will usually have unit length (length
  |1pt|) and which point in two different directions. Given a radius
  $r$ (measured in \TeX\ |pt|s, so if the 
  radius attribute |10pt|, then $r$ would be $10$) and an angle $a$,
  let $s$ be the sine of $a$ and let $c$ be the cosine
  of $a$, where $a$ is a number is degrees (so $s$
  would be $1$ for $a = 90$). Then, the current page position is
  shifted by $c \cdot r$ times $v_0$ and, additionally, by $s \cdot r$
  times $v_{90}$. This means that in the
  ``polar coordinate system'' $v_0$ is the unit vector along the
  ``$0^\circ$-axis'' and $v_{90}$ is the unit vector along
  ``$90^\circ$-axis''. The values of $v_0$ and $v_{90}$ can be changed
  using the following key on the \meta{angle axis}:
  \begin{key}{/tikz/data visualization/axis options/unit
      vectors=|\char`\{|\meta{unit vector 0
        degrees}|\char`\}\char`\{|\meta{unit vector 90
        degrees}|\char`\}| (initially {\char`\{(1pt,0pt)\char`\}\char`\{(0pt,1pt)\char`\}})}
    Both the \meta{unit vector 0 degrees} and the \meta{unit vector 90
      degrees} are \tikzname\ coordinates:
\begin{codeexample}[]
\tikz \datavisualization
    [new polar axes={angle axis}{radius axis},
     radius axis={unit length=1cm},
     angle axis={unit vectors={(10:1pt)}{(60:1pt)}},
     visualize as scatter]
  data [format=named] {
    angle={0,90}, radius={0.25,0.5,...,2}
  };
\end{codeexample}    
  \end{key}
\end{key}

Once created, the |angle axis| can be scaled conveniently using the
following keys:

\begin{key}{/tikz/data visualization/axis options/degrees}
  When this key is passed to the angle axis of a polar axis system, it
  sets up the scaling so that a value of |360| on this axis
  corresponds to a complete circle.
\begin{codeexample}[]
\tikz \datavisualization
    [new polar axes={angle axis}{radius axis},
     radius axis={unit length=1cm},
     angle axis={degrees},
     visualize as scatter]
  data [format=named] {
    angle={10,90}, radius={0.25,0.5,...,2}
  };
\end{codeexample}    
\end{key}

\begin{key}{/tikz/data visualization/axis options/radians}
  In contrast to |degrees|, this option sets up things so that a value
  of |2*pi| on this axis corresponds to a complete circle.
\begin{codeexample}[]
\tikz \datavisualization
    [new polar axes={angle axis}{radius axis},
     radius axis={unit length=1cm},
     angle axis={radians},
     visualize as scatter]
  data [format=named] {
    angle={0,1.5}, radius={0.25,0.5,...,2}
  };
\end{codeexample}
\end{key}




% Copyright 2008 by Till Tantau
%
% This file may be distributed and/or modified
%
% 1. under the LaTeX Project Public License and/or
% 2. under the GNU Free Documentation License.
%
% See the file doc/generic/pgf/licenses/LICENSE for more details.


\section{The Data Visualization Backend}
\label{section-dv-backend}

\subsection{Overview}

The present section explains the mechanisms behind the data
visualization engine. 

\subsection{The Rendering Pipeline}

\subsection{Usage}




\part{Utilities}
\label{part-utilities}

{\Large \emph{by Till Tantau}}


\bigskip
\noindent
The utility packages are not directly involved in creating graphics,
but you may find them useful nonetheless. All of them either directly
depend on \pgfname\ or they are designed to work well together with
\pgfname\ even though they can be used in a stand-alone way.

\vskip2cm
\medskip
\noindent
\begin{codeexample}[graphic=white]
\begin{tikzpicture}[scale=2]
  \shade[top color=blue,bottom color=gray!50] (0,0) parabola (1.5,2.25) |- (0,0);
  \draw (1.05cm,2pt) node[above] {$\displaystyle\int_0^{3/2} \!\!x^2\mathrm{d}x$};

  \draw[help lines] (0,0) grid (3.9,3.9)
       [step=0.25cm]      (1,2) grid +(1,1);

  \draw[->] (-0.2,0) -- (4,0) node[right] {$x$};
  \draw[->] (0,-0.2) -- (0,4) node[above] {$f(x)$};

  \foreach \x/\xtext in {1/1, 1.5/1\frac{1}{2}, 2/2, 3/3}
    \draw[shift={(\x,0)}] (0pt,2pt) -- (0pt,-2pt) node[below] {$\xtext$};

  \foreach \y/\ytext in {1/1, 2/2, 2.25/2\frac{1}{4}, 3/3}
    \draw[shift={(0,\y)}] (2pt,0pt) -- (-2pt,0pt) node[left] {$\ytext$};

  \draw (-.5,.25) parabola bend (0,0) (2,4) node[below right] {$x^2$};
\end{tikzpicture}
\end{codeexample}

% Copyright 2006 by Till Tantau
%
% This file may be distributed and/or modified
%
% 1. under the LaTeX Project Public License and/or
% 2. under the GNU Free Documentation License.
%
% See the file doc/generic/pgf/licenses/LICENSE for more details.


\section{Key Management}
\label{section-keys}

This section describes the package |pgfkeys|. It is loaded
automatically by both \pgfname\ and \tikzname.

\begin{package}{pgfkeys}
  This package can be used independently of \pgfname. Note that no
  other package of \pgfname\ needs to be loaded (so neither the
  emulation layer nor the system layer is needed). The Con\TeX t
  abbreviation is |pgfkey| if |pgfmod| is not loaded.
\end{package}



\subsection{Introduction}

\subsubsection{Comparison to Other Packages}

The |pgfkeys| package defines a key--value management system that is in
some sense similar to the more light-weight |keyval| system and the
improved |xkeyval| system. However, |pgfkeys| uses a slightly
different philosophy than these systems and it will coexist peacefully
with both of them.

The main differences between |pgfkeys| and |xkeyval| are the
following:

\begin{itemize}
\item |pgfkeys| organizes keys in a tree, while |keyval| and |xkeyval|
  use families. In |pgfkeys| the families correspond to the root
  entries of the key tree.
\item For efficiency reasons, |pgfkeys| does not directly support
  setting keys drawn from multiple families as |xkeyval| does. This
  can be emulated if necessary, but it will be slower than |xkeyval|'s
  native support.
\item |pgfkeys| has no save-stack impact (you will have to read the
  \TeX Book very carefully to appreciate this).
\item |pgfkeys| is slightly slower than |keyval|, but not much.
\item |pgfkeys| supports styles. This means that keys can just stand
  for other keys (which can stand for other keys in turn or which can
  also just execute some code). \tikzname\ uses this mechanism heavily.
\item |pgfkeys| supports multi-argument key code. This can, however,
  be emulated in |keyval|.
\item |pgfkeys| supports handlers. These are call-backs that are
  called when a key is not known. They are very flexible, in fact even
  defining keys in different ways is handled by, well, handlers.
\end{itemize}


\subsubsection{Quick Guide to Using the Key Mechanism}

The following quick guide to \pgfname's key mechanism only treats the
most commonly used features. For an in-depth discussion of what is
going on, please consult the remainder of this section.

Keys are organized in a large tree that is reminiscent of the Unix
file tree. A typical key might be, say, |/tikz/coordinate system/x|
or just |/x|. Again as in Unix, when you specify keys you can provide
the complete path of the key, but you usually just provide the name of
the key (corresponding to the file name without any path) and the path
is added automatically.

Typically (but not necessarily) some code is associated with a key. To
execute this code, you use the |\pgfkeys| command. This command takes
a list of so-called key--value pairs. Each pair is of the form
\meta{key}|=|\meta{value}. For each pair the |\pgfkeys| command will
execute the code stored for the \meta{key} with its parameter set to
\meta{value}.

Here is a typical example of how the |\pgfkeys| command is used:
\begin{codeexample}[code only]
\pgfkeys{/my key=hallo,/your keys/main key=something\strange,
         key name without path=something else}
\end{codeexample}

Now, to set the code that is stored in a key you do not need to learn
a new command. Rather, the |\pgfkeys| command can also be used to set
the code of a key. This is done using so-called \emph{handlers}. They
look like keys whose names look like ``hidden files in Unix'' since
they start with a dot. The handler for setting the code of a key is
appropriately called |.code| and it is used as follows:
\begin{codeexample}[]
\pgfkeys{/my key/.code=The value is '#1'.}
\pgfkeys{/my key=hi!}
\end{codeexample}
As you can see, in the first line we defined the code for the key
|/my key|. In the second line we executed this code with the parameter
set to |hi!|.

There are numerous handlers for defining a key. For instance, we can
also define a key whose value actually consists of more than one
parameter. 
\begin{codeexample}[]
\pgfkeys{/my key/.code 2 args=The values are '#1' and '#2'.}
\pgfkeys{/my key={a1}{a2}}
\end{codeexample}

We often want to have keys where the code is called with some default
value if the user does not provide a value. Not surprisingly, this is
also done using a handler, this time called |.default|.
\begin{codeexample}[]
\pgfkeys{/my key/.code=(#1)}
\pgfkeys{/my key/.default=hello}
\pgfkeys{/my key=hallo,/my key}
\end{codeexample}

The other way round, it is also possible to specify that a value
\emph{must} be specified, using a handler called
|.value required|. Finally, you can also require that no value
\emph{may} be specified using |.value forbidden|.

All keys for a package like, say, \tikzname\ start with the path
|/tikz|. We obviously do not like to write this path down every
time we use a key (so we do not have to write things like
|\draw[/tikz/line width=1cm]|). What we need is to somehow ``change
the default path to a specific location.'' This is done using the
handler |.cd| (for ``change directory''). Once this handler has been
used on a key, all subsequent keys {\itshape in the current call of
  |\pgfkeys| only} are automatically prefixed with this path, if
necessary.

Here is an example:
\begin{codeexample}[code only]
\pgfkeys{/tikz/.cd,line width=1cm,cap=round}
\end{codeexample}
This makes it easy to define commands like |\tikzset|, which could be
defined as follows (the actual definition is a bit faster, but the
effect is the same):
\begin{codeexample}[code only]
\def\tikzset#1{\pgfkeys{/tikz/.cd,#1}}
\end{codeexample}

When a key is handled, instead of executing some code, the key can
also cause further keys to be executed. Such keys will be called
\emph{styles}. A style is, in essence, just a key list that should be
executed whenever the style is executed. Here is an example:
\begin{codeexample}[]
\pgfkeys{/a/.code=(a:#1)}
\pgfkeys{/b/.code=(b:#1)}
\pgfkeys{/my style/.style={/a=foo,/b=bar,/a=#1}}
\pgfkeys{/my style=wow}
\end{codeexample}
As the above example shows, style can also be parametrized, just like
the normal code keys.

As a typical use of styles, suppose we wish to setup the key |/tikz|
so that it will change the default path to |/tikz|. This can be
achieved as follows:
\begin{codeexample}[code only]
\pgfkeys{/tikz/.style=/tikz/.cd}
\pgfkeys{tikz,line width=1cm,draw=red}
\end{codeexample}

Note that when |\pgfkeys| is executed, the default path is set
to~|/|. This means that the first |tikz| will be completed to
|/tikz|. Then |/tikz| is a style and, thus, replaced by |/tikz/.cd|,
which changes the default path to |/tikz|. Thus, the |line width| is
correctly prefixed with |/tikz|.

\subsection{The Key Tree}

The |pgfkeys| package organizes keys in a so-called \emph{key
  tree}. This tree will be familiar to anyone who has used a Unix
operating system: A key is addressed by a path, which consists of
different parts separated by slashes. A typical key might be
|/tikz/line width| or just |/tikz| or something more complicated like
|/tikz/cs/x/.store in|.

Let us fix some further terminology: Given a key like |/a/b/c|, we
call the part leading up the last slash (|/a/b|) the \emph{path} of
the key. We call everything after the last slash (|c|) the \emph{name}
of the key (in a file system this would be the file name). 

We do not always wish to specify keys completely. Instead, we usually
specify only part of a key (typically only the name) and the
\emph{default path} is then added to the key at the front. So, when
the default path is |/tikz| and you 
refer to the (partial) key |line width|, the actual key that is used
is |/tikz/line width|. There is a simple rule for deciding whether a
key is a partial key or a full key: If it starts with a slash, then it
is a full key and it is not modified; if it does not start with
a slash, then the default path is automatically prefixed.

Note that the default path is not the same as a search path. In
particular, the default path is just a single path. When a partial key
is given, only this single default path is prefixed; |pgfkeys| does
not try to lookup the key in different parts of a search path. It is,
however, possible to emulate search paths, but a much more
complicated mechanism must be used.

When you set keys (to be explained in a moment), you can freely mix
partial and full keys and you can change the default path. This makes
it possible to temporarily use keys from another part of the key tree
(this turns out to be a very useful feature).

Each key (may) store some \emph{tokens} and there exist commands,
described below, for setting, getting, and changing the tokens stored
in a key. However, you will only very seldom use these commands
directly. Rather, the standard way of using keys is the |\pgfkeys|
command or some command that uses it internally like, say,
|\tikzset|. So, you may wish to skip the following commands and
continue with the next subsection.

\begin{command}{\pgfkeyssetvalue\marg{full key}\marg{token text}}
  Stores the \meta{token text} in the \meta{full key}. The \meta{full key}
  may not be a partial key, so no default-path-adding is done. The
  \meta{token text} can be arbitrary tokens and may even contain things
  like |#| or unbalanced \TeX-ifs.
\begin{codeexample}[]
\pgfkeyssetvalue{/my family/my key}{Hello, world!}
\pgfkeysvalueof{/my family/my key}  
\end{codeexample}

  The setting of a key is always local to the current \TeX\ group.
\end{command}

\begin{command}{\pgfkeyslet\marg{full key}\marg{macro}}
  Performs a |\let| statement so the the \meta{full key} pionts to the
  contents of \meta{macro}.
\begin{codeexample}[]
\def\helloworld{Hello, world!}
\pgfkeyslet{/my family/my key}{\helloworld}
\pgfkeysvalueof{/my family/my key}  
\end{codeexample}
  You should never let a key be equal to |\relax|. Such a key may or
  may not be indistinguishable from an undefined key.
\end{command}

\begin{command}{\pgfkeysgetvalue\marg{full key}\marg{macro}}
  Retrieves the tokens stored in the \meta{full key} and lets
  \meta{macro} be equal to these tokens. If the key has
  not been set, the \meta{macro} will be equal to |\relax|. 
\begin{codeexample}[]
\pgfkeyssetvalue{/my family/my key}{Hello, world!}
\pgfkeysgetvalue{/my family/my key}{\helloworld}
\helloworld
\end{codeexample}
\end{command}

\begin{command}{\pgfkeysvalueof\marg{full key}}
  Inserts the value stored in \meta{full key} at the current position
  into the text.

\begin{codeexample}[]
\pgfkeyssetvalue{/my family/my key}{Hello, world!}
\pgfkeysvalueof{/my family/my key}
\end{codeexample}
\end{command}


\begin{command}{\pgfkeysifdefined\marg{full key}\marg{if}\marg{else}}
  Checks whether this key was previously set using either
  |\pgfkeyssetvalue| or |\pgfkeyslet|. If so, the code in \meta{if} is
  executed, otherwise the code in \meta{else}.

  This command will use e\TeX's |\ifcsname| command, if available, for
  efficiency. This means, however, that it may behave differently for
  \TeX\ and for e\TeX\ when you set keys to |\relax|. For this reason
  you should not do so. 
\begin{codeexample}[]
\pgfkeyssetvalue{/my family/my key}{Hello, world!}
\pgfkeysifdefined{/my family/my key}{yes}{no}
\end{codeexample}
\end{command}


\subsection{Setting Keys}

Settings keys is done using a powerful command called |\pgfkeys|. This
command takes a list of so-called \emph{key--value pairs}. These are
pairs of the form \meta{key}|=|\meta{value}. The principle idea is the
following: For each pair in the list, some \emph{action} is
taken. This action can be one of the following:

\begin{enumerate}
\item A command is executed whose argument(s) are \meta{value}. This
  command is stored in a special subkey of \meta{key}.
\item The \meta{value} is stored in the \meta{key} itself.
\item If the key's name (the part after the last slahs) is a known
  \emph{handler}, then this handler will take care of the key.
\item If the key is totally unknown, one of several possible
  \emph{unknown key handlers} is called. 
\end{enumerate}

Addtionally, if the \meta{value} is missing, a default value may or
may not be substituted. Before we plunge into all the details,
let us have a quick look at the command itself.

\begin{command}{\pgfkeys\marg{key list}}
  The \meta{key list} should be a list of key--value pairs, separated
  by commas. A key--value pair can have the following two forms:
  \meta{key}|=|\meta{value} or just \meta{key}. Any spaces around the
  \meta{key} or around the \meta{value} are removed. It is permissible
  to surround both the \meta{key} or the \meta{value} in curly braces,
  which are also removed. Especially putting the \meta{value} in curly
  braces needs to be done quite often, namely whenever the \meta{value}
  contains an equal-sign or a comma.

  The key--value pairs in the list are handled in the order they
  appear. How this handling is done, exactly, is described in the rest
  of this section.

  If a \meta{key} is a partial key, the current value of the default
  path is prefixed to the \meta{key} and this ``upgraded'' key is
  then used. The default path is just the root path |/| when the first
  key is handled, but it may change later on. At the end of the
  command, the default path is reset to the value it had before this
  command was executed. 
  
  Calls of this command may be nested. Thus, it is permissible to call
  |\pgfkeys| inside the code that is executed for a key. Since the
  default path is restored after a call of |\pgfkeys|, the default
  path will not change when you call |\pgfkeys| while executing code
  for a key (which is exactly what you want).
\end{command}

\begin{command}{\pgfqkeys\marg{default path}\marg{key list}}
  This command has the same effect as |\pgfkeys{|\meta{default
      path}|/.cd,|\meta{key list}|}|, it is only marginally
  quicker. This command should not be used in user code, but rather in
  commands like |\tikzset| or |\pgfset| that get called very often.   
\end{command}

\begin{command}{\pgfkeysalso\marg{key list}}
  This command has execatly the same effect as |\pgfkeys|, only the
  default path is not modified before or after the keys are being
  set. This command is mainly intended to be called by the code that
  is being processed for a key.
\end{command}

\begin{command}{\pgfqkeysalso\marg{default path}\marg{key list}}
  This command has the same effect as |\pgfkeysalso{|\meta{default
      path}|/.cd,|\meta{key list}|}|, it is only quicker. Changing the
  default path inside a |\pgfkeyalso| is dangerous, so use with
  care. A rather safe place to call this command is at the beginning
  of a \TeX\ group.
\end{command}


\subsubsection{Default Arguments}

The arguments of the |\pgfkeys| command can either be of the form
\meta{key}|=|\meta{value} or of the form \meta{key} with the
value-part missing. In the second case, the |\pgfkeys| will try to
provide a \emph{default value} for the \meta{value}. If such a default
value is defined, it will be used as if you had written
\meta{key}|=|\meta{default value}.

In the following, the details of how default values are determined is
described; however, you should normally use the handlers |.default|
and |.value required| as described in
Section~\ref{section-default-handlers} and you can may wish to skip
the following details.

When |\pgfkeys| encounters a \meta{key} without an equal-sign, the
following happens:
\begin{enumerate}
\item The input is replaced by \meta{key}|=\pgfkeysnovalue|. In
  particular, the commands |\pgfkeys{my key}| and
  |\pgfkeys{my key=\pgfkeysnovalue}| have exactly the same effect and
  you can ``simulate'' a missing value by providing the value
  |\pgfkeysnovalue|, which is sometimes useful. 
\item If the \meta{value} is |\pgfkeysnovalue|, then it is checked
  whether the subkey \meta{key}|/.@def| exists. For instance, if you
  write |\pgfkeys{/my key}|, then it is checked whether the key
  |/my key/.@def| exists.
\item If the key \meta{key}|/.@def| exists, then the tokens stored in
  this key are used as \meta{value}.
\item If the key does not exist, then |\pgfkeysnovalue| is used as the
  \meta{value}.
\item At the end, if the \meta{value} is now equal to
  |\pgfkeysvaluerequired|, then the code  (or something fairly equivalent)
  |\pgfkeys{/errors/value required=|\meta{key}|{}}|
  is executed. Thus, by changing this key you can change the error
  message that is printed or you can handle the missing value in some
  other way.
\end{enumerate}



\subsubsection{Keys That Execute Commands}
\label{section-key-code}

After the transformation process described in the previous subsection,
we arrive at a key of the form \meta{key}=\meta{value}, where
\meta{key} is a full key. Different things can now happen, but always
the macro |\pgfkeyscurrentkey| will have been setup to expand to the
text of the \meta{key} that is currently being processed.

The first things that is tested is whether the key \meta{key}|/.@cmd|
exists. If this is the case, then it is assumed that this key stores
the code of a macro and this macro is executed. The argument of this
macro is \meta{value} directly followed by |\pgfeov|, which stands for
``end of value.'' The \meta{value} is not surrounded by braces. After
this code has been executed, |\pgfkeys| continues with the next key in
the \meta{key list}.

It may seem quite peculiar that the macro stored in the key
\meta{key}|/.@cmd| is not simply executed with the argument
|{|\meta{value}|}|. However, the approach taken in the |pgfkeys|
packages allows for more flexibility. For instance, assume that you
have a key that expects a \meta{value} of the form
``\meta{text}|+|\meta{more text}'' and wishes to store \meta{text} and
\meta{more text} in two different macros. This can be achieved as
follows:
\begin{codeexample}[]
\def\mystore#1+#2\pgfeov{\def\a{#1}\def\b{#2}}
\pgfkeyslet{/my key/.@cmd}{\mystore}
\pgfkeys{/my key=hello+world}

|\a| is \a, |\b| is \b.
\end{codeexample}

Naturally, defining the code to be stored in a key in the above manner
is too awkward. The following commands simplify things a bit, but the
usual manner of setting up code for a key is to use one of the
handlers described in Section~\ref{section-code-handlers}.

\begin{command}{\pgfkeysdef\marg{key}\marg{code}}
  This command temporarily defines a \TeX-macro with the argument list
  |#1\pgfeov| and then lets \meta{key}|/.@cmd| be equal to this
  macro. The net effect of all this is that you have then setup code
  for the key \meta{key} so that when you write
  |\pgfkeys{|\meta{key}|=|\meta{value}|}|, then the \meta{code} is
  executed with all occurrences of |#1| in \meta{code} being replaced
  by \meta{value}. (This behaviour is quite similar to the
  |\define@key| command of |keyval| and |xkeyval|).

\begin{codeexample}[]
\pgfkeysdef{/my key}{#1, #1.}
\pgfkeys{/my key=hello}
\end{codeexample}
\end{command}

\begin{command}{\pgfkeysedef\marg{key}\marg{code}}
  This command works like |\pgfkeysdef|, but it uses |\edef| rather
  than |\def| when defining the key macro. If you do not know the
  difference between the two, then you will not need this command;
  and if you know the difference, then you will know when you need this
  command.
\end{command}

\begin{command}{\pgfkeysdeargs\marg{key}\marg{argument pattern}\marg{code}}
  This command works like |\pgfkeysdef|, but it allows you to provide
  an arbitrary \meta{argument pattern} rather than just the simple
  single argument of |\pgfkeysdef|. 

\begin{codeexample}[]
\pgfkeysdefargs{/my key}{#1+#2}{\def\a{#1}\def\b{#2}}
\pgfkeys{/my key=hello+world}

|\a| is \a, |\b| is \b.
\end{codeexample}
\end{command}

\begin{command}{\pgfkeysedefargs\marg{key}\marg{argument pattern}\marg{code}}
  The |\edef| version of |\pgfkeysdefargs|.
\end{command}


\subsubsection{Keys That Store Values}

Let us continue with what happens when |\pgfkeys| processes the
current key and  the subkey \meta{key}|/.@cmd| is not defined. Then
it is checked whether the \meta{key} itself exists (has been
previously assigned a value using, for instance,
|\pgfkeyssetvalue|). In this case, the tokens stored in \meta{key} are
replaced by \meta{value} and |\pgfkeys| proceeds with the next key in
the \meta{key list}. 


\subsubsection{Keys That Are Handled}
\label{section-key-handlers}

If neither the \meta{key} itself nor the subkey \meta{key}|/.@cmd| are
defined, then the \meta{key} cannot be processed ``all by itself.''
Rather, a \meta{handler} is needed for this key. Most of the power of
|pgfkeys| comes from the proper use of such handlers.

Recall that the \meta{key} is always a full key (if it was not
originally, it has already been upgraded at this point to a full
key). It decomposed into  two parts:

\begin{enumerate}
\item The \meta{path} of \meta{key} (everything
  before the last slash) is stored in the macro |\pgfkeyscurrentpath|.
\item The \meta{name} of \meta{key} (everything
  after the last slash) is stored in the macro |\pgfkeyscurrentname|.

  It is recommended (but not necessary) that the name of a handler
  starts with a dot (but not with |.@|), so that they are easy to
  detect for the reader.  
\end{enumerate}

(For efficiency reasons, these two macros are only setup at this point;
so when code is executed for a key in the ``usual'' manner then these
macros are not setup.)

The |\pgfkeys| command now checks whether the key
|/handlers/|\meta{name}|/.@cmd| exists. If so, it should store a command
and this command is executed exactly in the same manner as described
in Section~\ref{section-key-code}.
Thus, this code gets the \meta{value} that was originally intended for
\meta{key} as its argument, followed by |\pgfeov|.
It is the job of the handlers to so something useful with the
\meta{value}.

For an example, let us write a handler that will output the value
stored in a key to the log file. We call this handler
|.print to log|. The idea is that when someone tries to use the key
|/my key/.print to log|, then this key will not be defined and the
handler gets executed. The handler will then have access to the
path-part of the key, which is |/my key|, via the macro
|\pgfkeyscurrentpath|. It can then lookup which value is stored in
this key and print it.

\begin{codeexample}[code only]
\pgfkeysdef{/handlers/.print to log}
{%
  \pgfkeysgetvalue{\pgfkeyscurrentpath}{\temp}
  \writetolog{\temp}
}
\pgfkeyssetvalue{/my key}{Hi!}
...
\pgfkeys{/my key/.print to log}
\end{codeexample}
The above code will print |Hi!| in the log, provided the macro
|\writetolog| is setup appropriately.

For a more interesting handler, let us program a handler that will
setup a key so that when the key is used some code is executed. This
code is given as \meta{value}. All the handler must do is to call
|\pgfkeysdef| for the path of the key (which misses the handler's
name) and assign the parameter value to it.
\begin{codeexample}[]
\pgfkeysdef{/handlers/.my code}{\pgfkeysdef{\pgfkeyscurrentpath}{#1}}
\pgfkeys{/my key/.my code=(#1)}
\pgfkeys{/my key=hallo}
\end{codeexample}


\subsubsection{Keys That Are Unknown}

For some keys, neither the key is defined nor its |.@cmd| subkey nor
is a handler defined for this key. In this case, it is checked whether
the key \meta{current path}|/.unknown/.@cmd| exists. Thus, when you try to
use the key |/tikz/strange|, then it is checked whether
|/tikz/.unknown/.@cmd| exists. If this key exists (which it does), it is
executed. This code can then try to make sense of the key. For
instance, the handler for \tikzname\ will try to interpret the key's
name as a color or as an arrow specification or as a \pgfname\
option.

You can setup unknown key handlers for your own keys by simply setting
the code of the key \meta{my path prefix}|/.unknown|. This also allows
you to setup ``search paths.'' The idea is that you would like keys to
be searched not only in a single default path, but in
several. Suppose, for instance, that you would like keys to be
searched 
for in |/a|, |/b|, and |/b/c|. We setup a key |/my search path| for
this:
\begin{codeexample}[code only]
\pgfkeys{/my search path/.unknown/.code=
  {%
    \let\searchname=\pgfkeyscurrentname%
    \pgfkeysalso{%
      /a/\searchname/.try=#1,
      /b/\searchname/.retry=#1,
      /b/c/\searchname/.retry=#1%
    }%
  }%
}
\pgfkeys{/my search path/.cd,foo,bar}
\end{codeexample}
In the above code, |foo| and |bar| will be searched for in the three
directories  |/a|, |/b|, and |/b/c|. 

If the key \meta{current path}|/.unknown/.@cmd| does not exist, the
handler |/handlers/.unknown| is invoked instead, which is always
defined and which prints an error message by default.

\subsection{Key Handlers}

We now describe which key handlers are defined by default. You can
also define new ones as described in Section~\ref{section-key-handlers}.


\subsubsection{Handlers for Path Management}

\begin{handler}{{.cd}}
  This handler causes the default path to be set to \meta{key}. Note that
  the default path is reset at the beginning of each call to
  |\pgfkeys| to be equal to~|/|.

  \example |\pgfkeys{/tikz/.cd,...}|
\end{handler}

\begin{handler}{{.is family}}
  This handler sets up things such that when \meta{key} is executed, then
  the current path is set to \meta{key}. A typical use is the following:
\begin{codeexample}[code only]
\pgfkeys{/tikz/.is family}
\pgfkeys{tikz,line width=1cm}  
\end{codeexample}
  The effect of this handler is the same as if you had written
  \meta{key}|/.style=|\meta{key}|/.cd|, only the code produced by the
  |.is family| handler is quicker.
\end{handler}


\subsubsection{Setting Defaults}
\label{section-default-handlers}

\begin{handler}{{.default}|=|\meta{value}}
  Sets the default value of \meta{key} to \meta{value}. This means
  that whenever no value is provided in a call to |\pgfkeys|, then
  this \meta{value} will be used instead.

  \example |\pgfkeys{/width/.default=1cm}|
\end{handler}

\begin{handler}{{.value required}}
  This handler causes the error message key |/erros/value required| to
  be issued whenever the \meta{key} is used without a value.

  \example |\pgfkeys{/width/.value required}|
\end{handler}

\begin{handler}{{.value forbidden}}
  This handler causes the error message key |/erros/value forbidden|
  to be issued whenever the \meta{key} is used with a value.

  This handler works be adding code to the code of the key. This means
  that you have to define the key first before you can use this
  handler. 
\begin{codeexample}[code only]
\pgfkeys{/my key/.code=I do not want an argument!}
\pgfkeys{/my key/.value forbidden}

\pgfkeys{/my key}     % Ok
\pgfkeys{/my key=foo} % Error
\end{codeexample}
\end{handler}


\subsubsection{Defining Key Codes}
\label{section-code-handlers}

A number of handlers exist for defining the code of keys.

\begin{handler}{{.code}|=|\meta{code}}
  This handler executes |\pgfkeysdef| with the parameters \meta{key}
  and \meta{code}. This means that, afterwards, whenever the
  \meta{key} is used, the \meta{code} gets executed. More precisely,
  when \meta{key}|=|\meta{value} is encountered in a key list,
  \meta{code} is executed with any occurrence of |#1| replaced by
  \meta{value}. As always, if no \meta{value} is given, the default
  value is used, if defined, or the special value |\pgfkeysnovalue|.

  It is permissible that \meta{code} calls the command |\pgfkeys|. It
  is also permissible the \meta{code} calls the command
  |\pgfkeysalso|, which is useful for styles, see below.

\begin{codeexample}[code only]
\pgfkeys{/par indent/.code={\parindent=#1},/par indent/.default=2em}
\pgfkeys{/par indent=1cm}
...
\pgfkeys{/par indent}
\end{codeexample}
\end{handler}

\begin{handler}{{.ecode}|=|\meta{code}}
  This handler works like |.code|, only the command |\pgfkeysedef| is
  used. 
\end{handler}


\begin{handler}{{.code 2 args}|=|\meta{code}}
  This handler works like |.code|, only two arguments rather than one
  are expected when the \meta{code} is executed. This means that when
  \meta{key}|=|\meta{value} is encountered in a key list, the
  \meta{value} should consist of two arguments. For instance,
  \meta{value} could be |{first}{second}|. Then \meta{code} is
  executed with any occurrence of |#1| replaced |first| and any
  occurrence of |#2| replaced by |second|.

  Because of the special way the \meta{value} is parsed, if you set
  \meta{value} to, for instance, |first| (without any braces), then
  |#1| will be set to |f| and |#2| will be set to |irst|. 

\begin{codeexample}[code only]
\pgfkeys{/page size/.code 2 args={\paperheight=#2\paperwidth=#1}}
\pgfkeys{/page size={30cm}{20cm}}
\end{codeexample}
\end{handler}

\begin{handler}{{.ecode 2 args}|=|\meta{code}}
  This handler works like |.code 2 args|, only an |\edef| is used
  rather than a |\def| to define the macro.
\end{handler}



\begin{handler}{{.code args}|=|\marg{argument pattern}\marg{code}}
  This handler also works like |.code|, but you can now specify an
  arbitrary \meta{argument pattern}. Such a pattern is a usual \TeX\
  macro pattern. For instance, suppose \meta{argument pattern} is
  |(#1/#2)| and \meta{key}|=|\meta{value} is encountered in a
  key list with \meta{value} being |(first/second)|. Then \meta{code}
  is executed with any occurrence of |#1| replaced |first| and any
  occurrence of |#2| replaced by |second|. So, the actual \meta{value}
  is matched against the \meta{argument pattern} in the standard \TeX\
  way. 

\begin{codeexample}[code only]
\pgfkeys{/page size/.code args={#1 and #2}{\paperheight=#2\paperwidth=#1}}
\pgfkeys{/page size=30cm and 20cm}
\end{codeexample}
\end{handler}

\begin{handler}{{.ecode args}|=|\marg{argument pattern}\marg{code}}
  This handler works like |.code args|, only an |\edef| is used
  rather than a |\def| to define the macro.
\end{handler}


There are also handlers for modifying existing keys.

\begin{handler}{{.add code}|=|\marg{prefix code}\marg{append code}}
  This handler adds code to an existing key. The \meta{prefix code} is
  added to the code stored in \meta{key}|/.@cmd| at the beginning, the
  \meta{append code} is added to this code at the end. Either can be
  empty. The argument list of \meta{code} cannot be changed using this
  handler. Note that both \meta{prefix code} and \meta{append code}
  may contain parameters like |#2|. 
  
\begin{codeexample}[code only]
\pgfkeys{/par indent/.code={\parindent=#1}}
\newdimen\myparindent  
\pgfkeys{/par indent/.add code={}{\myparindent=#1}}
...
\pgfkeys{/par indent=1cm} % This will set both \parindent and
                          % \myparindent to 1cm
\end{codeexample}
\end{handler}

\begin{handler}{{.prefix code}|=|\meta{prefix code}}
  This handler is a shortcut for \meta{key}|/.add code={|\meta{prefix
      code}|}{}|. That is, this handler adds the \meta{prefix code} at
  the beginning of the code stored in \meta{key}|/.@cmd|.
\end{handler}

\begin{handler}{{.append code}|=|\meta{append code}}
  This handler is a shortcut for \meta{key}|/.add code={}{|\meta{append
      code}|}{}|.
\end{handler}


\subsubsection{Defining Styles}

The following handlers allow you to define \emph{styles}. A style is a
key list that is processed whenever the style is given as a key in a
key list. Thus, a style ``stands for'' a certain key value
list. Styles can be parametrized just like normal code.

\begin{handler}{{.style}|=|\meta{key list}}
  This handler set things up so that whenever \meta{key}|=|\meta{value} is
  encountered in a key list, then the \meta{key list}, with every
  occurrence of |#1| replaced by \meta{value}, is processed
  instead. As always, if no \meta{value} is given, the default
  value is used, if defined, or the special value |\pgfkeysnovalue|.

  You can achieve the same effect by writing
  \meta{key}|/.code=\pgfkeysalso{|\meta{key list}|}|. This means, in
  particular, that the code of a key could also first execute some
  normal code and only then process some further keys. 

\begin{codeexample}[code only]
\pgfkeys{/par indent/.code={\parindent=#1}}
\pgfkeys{/no indent/.style={/par indent=0pt}}
\pgfkeys{/normal indent/.style={/par indent=2em}}
\pgfkeys{/no indent}
...
\pgfkeys{/normal indent}
\end{codeexample}
  The following example shows a parametrized style ``in action''.
\begin{codeexample}[]
\begin{tikzpicture}[outline/.style={draw=#1,fill=#1!20}]
  \node [outline=red]            {red box};
  \node [outline=blue] at (0,-1) {blue box};
\end{tikzpicture}
\end{codeexample}
\end{handler}

\begin{handler}{{.estyle}|=|\meta{key list}}
  This handler works like |.style|, only the \meta{code} is set using
  |\edef| rather than |\def|. Thus, all macros in the \meta{code} are
  expanded prior to saving the style.
\end{handler}

For styles the corresponding handlers as for normal code exist:

\begin{handler}{{.style 2 args}|=|\meta{key list}}
  This handler works like |.code 2 args|, only for styles. Thus, the
  \meta{key list} may contain occurrences of both |#1| and |#2| and
  when the style is used, two parameters must be given as
  \meta{value}. 
\begin{codeexample}[code only]
\pgfkeys{/paper height/.code={\paperheight=#1},/paper width/.code={\paperwidth=#1}}
\pgfkeys{/page size/.style 2 args={/paper height=#1,/paper width=#2}}
\pgfkeys{/page size={30cm}{20cm}}
\end{codeexample}
\end{handler}

\begin{handler}{{.estyle 2 args}|=|\meta{key list}}
  This handler works like |.style 2 args|, only an |\edef| is used
  rather than a |\def| to define the macro.
\end{handler}

\begin{handler}{{.style args}|=|\marg{argument pattern}\marg{key list}}
  This handler works like |.code args|, only for styles.
\end{handler}

\begin{handler}{{.estyle args}|=|\marg{argument pattern}\marg{code}}
  This handler works like |.ecode args|, only for styles.
\end{handler}

\begin{handler}{{.add style}|=|\marg{prefix key list}\marg{append key list}}
  This handler works like |.add code|, only for styles. However, it is
  permissible to add styles to keys that have previously been set
  using  |.code|. (It is also permissible to add normal \meta{code} to
  a key that has previously been set using |.style|). When you add a
  style to a key that was previously set using |.code|, the following
  happens: When \meta{key} is processed, the \meta{prefix key list}
  will be processed first, then the \meta{code} that was previously
  stored in \meta{key}|/.@cmd|, and then the keys in \meta{append key
    list} are processed.
\begin{codeexample}[code only]
\pgfkeys{/par indent/.code={\parindent=#1}}
\pgfkeys{/par indent/.add style={}{/my key=#1}}
...
\pgfkeys{/par indent=1cm} % This will set \parindent and
                          % then execute /my key=#1
\end{codeexample}
\end{handler}

\begin{handler}{{.prefix style}|=|\meta{prefix key list}}
  Works like |.add style|, but only for the prefix key list.
\end{handler}

\begin{handler}{{.append style}|=|\meta{append key list}}
  Works like |.add style|, but only for the append key list.
\end{handler}


\subsubsection{Defining Value-, Macro-, If- and Choice-Keys}

For some keys, the code that should be executed for them is rather
``specialized.'' For instance, it happens often that the code for a
key just sets a certain \TeX-if to true or false. For these case
predefine handlers make it easier to install the necessary code.

However, we start with some handlers that are used to manage the value
that is directly stored in a key.

\begin{handler}{{.initial}|=|\meta{value}}
  This handler sets the value of \meta{key} to \meta{value}. Note that
  no subkeys are involved. After this handler has been used, by the
  rules governing keys, you can subsequently change the value of the
  \meta{key} by just writing \meta{key}|=|\meta{value}. Thus, this
  handler is used to set the initial value of key.

\begin{codeexample}[code only]
\pgfkeys{/my key/.initial=red}
% "/my key" now stores the value "red"
\pgfkeys{/my key=blue}
% "/my key" now stores the value "blue"
\end{codeexample}

  Note that in the after the example, writing |\pgfkeys{/my key}| will not
  have the effect you might expect (namely that |blue| is inserted
  into the main text). Rather, |/my key| will be promoted to
  |/my key=\pgfkeysnovalue| and, thus, |\pgfkeysnovalue| will be
  stored in |/my key|.

  To retrieve the value stored in a key, the handler |.get| is used.
\end{handler}

\begin{handler}{{.get}|=|\meta{macro}}
  Executes a |\let| command so that \meta{macro} contains the contents
  stored in \meta{key}.  

\begin{codeexample}[]
\pgfkeys{/my key/.initial=red}
\pgfkeys{/my key=blue}
\pgfkeys{/my key/.get=\mymacro}
\mymacro
\end{codeexample}
\end{handler}

\begin{handler}{{.add}|=|\marg{prefix value}\marg{append value}}
  Adds the \meta{prefix value} and the beginning and the \meta{append
    value} at the end of the value stored in \meta{key}.
\end{handler}

The next handler is useful for the common situation where
\meta{key}|=|\meta{value} should cause the \meta{value} to be stored
in some macro. Note that, typically, you could just as well store the
value in the key itself.

\begin{handler}{{.store in}|=|\meta{macro}}
  This handler has the following effect: When you write
  \meta{key}|=|\meta{value}, the code
  |\def|\meta{macro}|{|\meta{value}|}| is executed. Thus, the given
  value is ``stored'' in the \meta{macro}.  
\begin{codeexample}[]
\pgfkeys{/text/.store in=\mytext}
\def\a{world} 
\pgfkeys{/text=Hello \a!}
\def\a{Gruffalo}
\mytext
\end{codeexample}
\end{handler}

\begin{handler}{{.estore in}|=|\meta{macro}}
  This handler is similar to |.store in|, only the code 
  |\edef|\meta{macro}|{|\meta{value}|}| is used. Thus, the
  macro-expanded version of \meta{value} is stored in the
  \meta{macro}. 
\begin{codeexample}[]
\pgfkeys{/text/.estore in=\mytext}
\def\a{world} 
\pgfkeys{/text=Hello \a!}
\def\a{Gruffalo}
\mytext
\end{codeexample}
\end{handler}

In another common situation a key is used to set a \TeX-if to true or
false. 

\begin{handler}{{.is if}|=|\meta{\TeX-if name}}
  This handler has the following effect: When you write
  \meta{key}|=|\meta{value}, it is first checked that \meta{value} is
  |true| or |false| (the default is |true| if no \meta{value} is
  given). If this is not the case, the error key
  |/errors/boolean expected| is executed. Otherwise, 
  the code |\|\meta{\TeX-if name}\meta{value} is executed, which sets
  the \TeX-if accordingly.
\begin{codeexample}[]
\newif\iftheworldisflat    
\pgfkeys{/flat world/.is if=theworldisflat}
\pgfkeys{/flat world=false}
\iftheworldisflat
  Flat
\else
  Round?
\fi
\end{codeexample}
\end{handler}

The next handler deals with the problem when a
\meta{key}|=|\meta{value} makes sense only for a small set of possible
\meta{value}s. For instance, the line cap can only be |rounded| or
|rect| or |butt|, but nothing else. For this situation the following
handler is useful.

\begin{handler}{{.is choice}}
  This handler set things up so that writing \meta{key}|=|\meta{value}
  will cause the subkey \meta{key}|/|\meta{value} to be executed. So,
  each of the different possible choices should be given by a subkey
  of \meta{key}.
\begin{codeexample}[code only]
\pgfkeys{/line cap/.is choice}
\pgfkeys{/line cap/round/.style={\pgfsetbuttcap}}
\pgfkeys{/line cap/butt/.style={\pgfsetroundcap}}
\pgfkeys{/line cap/rect/.style={\pgfsetrectcap}}
\pgfkeys{/line cap/rectangle/.style={/line cap=rect}}
...
\draw [/line cap=butt] ...
\end{codeexample}
  If the subkey \meta{key}|/|\meta{value} does not exist, the error
  key |/errors/unknown choice value| is executed.
\end{handler}


\subsubsection{Expanding Values}

When you write \meta{key}|=|\meta{value}, you usually wish to use the
\meta{value} ``as is.'' Indeed, great care is taken to ensure that you
can even use things like |#1| or unbalanced \TeX-ifs inside
\meta{value}. However, sometimes you want the \meta{value} to be
expanded before it is used. For instance, \meta{value} might be a
macro name like |\mymacro| and you do not want |\mymacro| to be used
as the macro, but rather the \emph{contents} of |\mymacro|. Thus,
instead of using \meta{value} you wish to use whatever \meta{value}
expands to. Instead of using some fancy |\expandafter| hackery, you
can use the following handlers:

\begin{handler}{{.expand once}|=|\meta{value}}
  This handler expands \meta{value} once (more precisely, it executes
  an |\expandafter| command on the first token of \meta{value}) and
  then process the resulting \meta{result} as if you had written
  \meta{key}|=|\meta{result}. Note that if \meta{key} contains a
  handler itself, this handler will be called normally.
\begin{codeexample}[]
\def\a{bottom}
\def\b{\a}
\def\c{\b}

\pgfkeys{/key1/.initial=\c}
\pgfkeys{/key2/.initial/.expand once=\c}
\pgfkeys{/key3/.initial/.expand twice=\c}
\pgfkeys{/key4/.initial/.expanded=\c}

\def\a{{\ttfamily\string\a}}
\def\b{{\ttfamily\string\b}}
\def\c{{\ttfamily\string\c}}

\begin{tabular}{ll}
Key 1:& \pgfkeys{/key1} \\
Key 2:& \pgfkeys{/key2} \\
Key 3:& \pgfkeys{/key3} \\
Key 4:& \pgfkeys{/key4}
\end{tabular}
\end{codeexample}
\end{handler}

\begin{handler}{{.expand twice}|=|\meta{value}}
  This handler works like saying \meta{key}|/.expand once/.expand once=|\meta{value}.
\end{handler}

\begin{handler}{{.expanded}|=|\meta{value}}
  This handler will completely expand \meta{value} (using |\edef|)
  before processing \meta{key}|=|\meta{result}.
\end{handler}


\subsubsection{Handlers for Testing Keys}

\begin{handler}{{.try}|=|\meta{value}}
  This handler causes the same things to be done as if
  \meta{key}|=|\meta{value} had been written instead. However, if
  neither \meta{key}|/.@cmd| nor the key itself is defined, no
  handlers will be called. Instead, 
  the execution of the key just stops. Thus, this handler will ``try''
  to use the key, but no further action is taken when the key is not
  defined.

  The \TeX-if |\||ifpgfkeyssuccess| will be set according to whether
  the \meta{key} was successfully executed or not. 
\begin{codeexample}[]
\pgfkeys{/a/.code=(a:#1)}
\pgfkeys{/b/.code=(b:#1)}
\pgfkeys{/x/.try=hmm,/a/.try=hallo,/b/.try=welt}
\end{codeexample}
\end{handler}

\begin{handler}{{.retry}|=|\meta{value}}
  This handler works just like |.try|, only it will not do anything if
  |\||ifpgfkeyssuccess| is false. Thus, this handler will only retry
  to set a key if ``the last attempt failed''. 
\begin{codeexample}[]
\pgfkeys{/a/.code=(a:#1)}
\pgfkeys{/b/.code=(b:#1)}
\pgfkeys{/x/.try=hmm,/a/.retry=hallo,/b/.retry=welt}
\end{codeexample}
\end{handler}


\subsubsection{Handlers for Key Inspection}

\begin{handler}{{.show value}}
  This handler executes a |\show| command on the value stored in
  \meta{key}. This is useful mostly for debugging.

  \example |\pgfkeys{/my/obscure key/.show value}|
\end{handler}

\begin{handler}{{.show code}}
  This handler executes a |\show| command on the code stored in
  \meta{key}|/.@cmd|. This is useful mostly for debugging.

  \example |\pgfkeys{/my/obscure key/.show code}|
\end{handler}

The following key is not a handler, but it also commonly used for
inspecting things:
\begin{key}{/utils/exec=\meta{code}}
  This key will simply execute the given \meta{code}. 

  \example |\pgfkeys{some key=some value,/utils/exec=\show\hallo,obscure key=obscure}|
\end{key}


\subsection{Error Keys}

In certain situations errors can occur, like using an undefined
key. In these situations error keys are executed. They should store a
macro that gets two arguments: The first is the offending key
(possibly only after macro expansion), the second is the value that
was passed as a parameter (also possibly only after macro expansion).

Currently, error keys are simply executed. In the future it might be a
good idea to have different subkeys that are executed depending on the
language currently set so that users get a localized error message.

\begin{key}{/errors/value required=\marg{offending key}\marg{value}}
  This key is executed whenever an \meta{offending key} is used
  without a value when a value is actually required. 
\end{key}

\begin{key}{/errors/value forbidden=\marg{offending key}\marg{value}}
  This key is executed whenever a key is used with a value when a
  value is actually forbidden.
\end{key}

\begin{key}{/errors/boolean expected=\marg{offending key}\marg{value}}
  This key is executed whenever a key setup using |.is if| gets called
  with a \meta{value} other than |true| or |false|.
\end{key}

\begin{key}{/errors/unknown choice value=\marg{offending key}\marg{value}}
  This key is executed whenever a choice is used as a \meta{value} for
  a key setup using the |.is choice| handler that is not defined.
\end{key}

\begin{key}{/errors/unknown key=\marg{offending key}\marg{value}}
  This key is executed whenever a key is unknown and no specific
  |.unknown| handler is found.
\end{key}

% Copyright 2006 by Till Tantau
%
% This file may be distributed and/or modified
%
% 1. under the LaTeX Project Public License and/or
% 2. under the GNU Free Documentation License.
%
% See the file doc/generic/pgf/licenses/LICENSE for more details.


\section{Repeating Things: The Foreach Statement}
\label{section-foreach}

This section describes the package |pgffor|, which is loaded
automatically by \tikzname, but not by \pgfname:

\begin{package}{pgffor}
  This package can be used independently of \pgfname, but works
  particularly well together with \pgfname\ and \tikzname. It defines
  two new commands: |\foreach| and |\breakforeach|.
\end{package}

\begin{command}{\foreach| |\meta{variables}| |\opt{{\ttfamily[}\meta{options}{\ttfamily]}}% 
	| in |\meta{list}  \meta{commands}}
  The syntax of this command is a bit complicated, so let us go
  through it step-by-step.

  In the easiest case, \meta{variables} is a single \TeX-command like
  |\x| or |\point|. (If you want to have some fun, you can also use
  active characters. If you do not know what active characters are,
  you are blessed.)

	Still in the easiest case, \meta{options} will be omitted. The keys
	for customising this command will be discussed below.
	
  Again, in the easiest case, \meta{list} is either a comma-separated
  list of values surrounded by curly braces or it is the name of a
  macro that contain such a list of values. Anything can be used as a
  value, but numbers are most likely.

  Finally, in the easiest case, \meta{commands} is some \TeX-text in
  curly braces.

  With all these assumptions, the |\foreach| statement will execute
  the \meta{commands} repeatedly, once for every element of the
  \meta{list}. Each time the \meta{commands} are executed, the
  \meta{variable} will be set to the current value of the list
  item.

\begin{codeexample}[]
\foreach \x in {1,2,3,0} {[\x]}
\end{codeexample}

\begin{codeexample}[]
\def\mylist{1,2,3,0}
\foreach \x in \mylist {[\x]}
\end{codeexample}

  Note that in each execution of \meta{commands} the
  \meta{commands} are put in a \TeX\ group. This means that
  \emph{local changes to counters inside \meta{commands} do not
    persist till the next iteration}. For instance, if you add 1 to a
  counter inside \meta{commands} locally, then in the next iteration
  the counter will have the same value it had at the beginning of the
  first iteration. You have to add |\global| if you wish changes to
  persist from iteration to iteration.

  \medskip
  \textbf{Syntax for the commands.}
  Let us move on to a more complicated setting. The first
  complication occurs when the \meta{commands} are not some text in
  curly braces. If the |\foreach| statement does not encounter an
  opening brace, it will instead scan everything up to the next
  semicolon and use this as \meta{commands}. This is most useful in
  situations like the following:

\begin{codeexample}[]
\tikz
  \foreach \x in {0,1,2,3}
    \draw (\x,0) circle (0.2cm);
\end{codeexample}

  However, the ``reading till the next semicolon'' is not the whole
  truth. There is another rule: If a |\foreach| statement is directly
  followed by another |\foreach| statement, this second foreach
  statement is collected as \meta{commands}. This allows you to write
  the following:

\begin{codeexample}[]
\begin{tikzpicture}
  \foreach \x in {0,1,2,3}
    \foreach \y in {0,1,2,3}
      {
        \draw (\x,\y) circle (0.2cm);
        \fill (\x,\y) circle (0.1cm);
      }
\end{tikzpicture}
\end{codeexample}

  \medskip
  \textbf{The dots notation.}
  The second complication concerns the \meta{list}. If this
  \meta{list} contains the list item ``|...|'', this list item is replaced
  by the ``missing values.'' More precisely, the following happens:

  Normally, when a list item |...| is encountered, there should
  already have been \emph{two} list items before it, which where
  numbers. Examples of \emph{numbers} are |1|, |-10|, or
  |-0.24|. Let us call these numbers $x$ and $y$ and let $d := y-x$ be
  their difference. Next, there should also be one number following
  the three dots, let us call this number~$z$.

  In this situation, the part of the list reading
  ``$x$|,|$y$|,...,|$z$'' is replaced by ``$x$, $x+d$, $x+2d$, $x+3d$,
  \dots, $x+md$,'' where the last dots are semantic dots, not
  syntactic dots. The value $m$ is the largest number such that $x +
  md \le z$ if $d$ is positive or such that $x+md \ge z$ if $d$ is
  negative. 

  Perhaps it is best to explain this by some examples:  The following
  \meta{list} have the same effects:

  |\foreach \x in {1,2,...,6} {\x, }| yields \foreach \x in {1,2,...,6} {\x, }

  |\foreach \x in {1,2,3,...,6} {\x, }| yields \foreach \x in {1,2,3,...,6} {\x, }

  |\foreach \x in {1,3,...,11} {\x, }| yields \foreach \x in {1,3,...,11} {\x, }

  |\foreach \x in {1,3,...,10} {\x, }| yields \foreach \x in {1,3,...,10} {\x, }

  |\foreach \x in {0,0.1,...,0.5} {\x, }| yields \foreach \x in {0,0.1,...,0.5} {\x, }

  |\foreach \x in {a,b,9,8,...,1,2,2.125,...,2.5} {\x, }| yields \foreach \x in {a,b,9,8,...,1,2,2.125,...,2.5} {\x, }

  As can be seen, for fractional steps that are not multiples of
  $2^{-n}$ for some small $n$, rounding errors can occur pretty
  easily. Thus, in the second last case, |0.5| should probably be
  replaced by |0.501| for robustness.
  
  There is another special case for the |...| statement: If the
  |...| is used right after the first item in the list, that is, if
  there is an $x$, but no $y$, the difference $d$ obviously cannot be
  computed and is set to $1$ if the number $z$ following the dots is
  larger than $x$ and is set to $-1$ if $z$ is smaller:

  |\foreach \x in {1,...,6} {\x, }| yields \foreach \x in {1,...,6} {\x, }

  |\foreach \x in {9,...,3.5} {\x, }| yields \foreach \x in {9,...,3.5} {\x, }

	
	There is a yet a further special case for the |...| statement, in that
	it can indicate an alphabetic character sequence:
	
	|\foreach \x in {a,...,m} {\x, }| yields \foreach \x in {a,...,m} {\x, }
	
	|\foreach \x in {Z,X,...,M} {\x, }| yields \foreach \x in {Z,X,...,M} {\x, }
	
	A final special case for the |...| statement is contextual replacement.
	If the |...| is used in some context, for example, |sin(...)|, this 
	context will be interpreted correctly, provided that the list items 
	prior to the |...| statement have \emph{exactly} the same pattern,
	except that, instead of dots, they have a number or a character:
	
	|\foreach \x in {2^1,2^...,2^7} {$\x$, }| yields \foreach \x in {2^1,2^...,2^7} {$\x$, }
	
	|\foreach \x in {0\pi,0.5\pi,...\pi,3\pi} {$\x$, }| yields \foreach \x in {0\pi,0.5\pi,...\pi,3\pi} {$\x$, }
	
	|\foreach \x in {A_1,..._1,H_1} {$\x$, }| yields \foreach \x in {A_1,..._1,H_1} {$\x$, }
	
  \textbf{Special handling of pairs.}
  Different list items are separated by commas. However, this causes a
  problem when the list items contain commas themselves as pairs like
  |(0,1)| do. In this case, you should put the items containing commas
  in braces as in |{(0,1)}|. However, since pairs are such a natural
  and useful case, they get a special treatment by the |\foreach|
  statement. When a list item starts with a |(| everything up to the
  next |)| is made part of the item. Thus, we can write things like
  the following:

\begin{codeexample}[]
\tikz
  \foreach \position in {(0,0), (1,1), (2,0), (3,1)}
    \draw \position rectangle +(.25,.5);
\end{codeexample}
  
  \medskip
  \textbf{Using the foreach-statement inside paths.}
  \tikzname\ allows you to use a |\foreach| statement inside a path
  construction. In such a case, the \meta{commands} must be path
  construction commands. Here are two examples:

\begin{codeexample}[]
\tikz
  \draw (0,0)
    \foreach \x in {1,...,3}
      { -- (\x,1) -- (\x,0) }
    ;
\end{codeexample}

\begin{codeexample}[]
\tikz \draw \foreach \p in {1,...,3} {(\p,1)--(\p,3) (1,\p)--(3,\p)};
\end{codeexample}
    
  \medskip
  \textbf{Multiple variables.}
  You will often wish to iterate over two variables at the same
  time. Since you can nest |\foreach| loops, this is normally
  straight-forward. However, you sometimes wish variables to
  iterate ``simultaneously.'' For example, we might be given a list of
  edges that connect two coordinates and might wish to iterate over
  these edges. While doing so, we would like the source and target of
  the edges to be set to two different variables.

  To achieve this, you can use the following syntax: The
  \meta{variables} may not only be a single \TeX-variable. Instead, it
  can also be a list of variables separated by slashes (|/|). In this
  case the list items can also be lists of values separated by
  slashes.

  Assuming that the \meta{variables} and the list items are lists of
  values, each time the \meta{commands} are executed, each of the
  variables in \meta{variables} is set to one part of the list making
  up the current list item. Here is an example to clarify this:

  \example |\foreach \x / \y in {1/2,a/b} {``\x\ and \y''}| yields
  \foreach \x / \y in {1/2,a/b} {``\x\ and \y''}.

  If some entry in the \meta{list} does not have ``enough'' slashes,
  the last entry will be repeated. Here is an example:
\begin{codeexample}[]
\begin{tikzpicture}
  \foreach \x/\xtext in {0,...,3,2.72 / e}
    \draw (\x,0) node{$\xtext$};
\end{tikzpicture}
\end{codeexample}
  
  Here are more useful examples:
\begin{codeexample}[]
\begin{tikzpicture}
  % Define some coordinates:
  \path[nodes={circle,fill=examplefill,draw}]
    (0,0)    node(a) {a}
    (2,0.55) node(b) {b}
    (1,1.5)  node(c) {c}
    (2,1.75) node(d) {d};

  % Draw some connections:
  \foreach \source/\target in {a/b, b/c, c/a, c/d}
    \draw (\source) .. controls +(.75cm,0pt) and +(-.75cm,0pt)..(\target);  
\end{tikzpicture}
\end{codeexample}

\begin{codeexample}[]
\begin{tikzpicture}
  % Let's draw circles at interesting points:
  \foreach \x / \y / \diameter in {0 / 0 / 2mm, 1 / 1 / 3mm, 2 / 0 / 1mm}
    \draw (\x,\y) circle (\diameter);
    
  % Same effect
  \foreach \center/\diameter in {{(0,0)/2mm}, {(1,1)/3mm}, {(2,0)/1mm}}
    \draw[yshift=2.5cm] \center circle (\diameter);
\end{tikzpicture}
\end{codeexample}

\begin{codeexample}[]
\begin{tikzpicture}[line cap=round,line width=3pt]
  \filldraw [fill=examplefill] (0,0) circle (2cm);

  \foreach \angle / \label in
    {0/3, 30/2, 60/1, 90/12, 120/11, 150/10, 180/9,
     210/8, 240/7, 270/6, 300/5, 330/4}
  {
    \draw[line width=1pt] (\angle:1.8cm) -- (\angle:2cm);
    \draw (\angle:1.4cm) node{\textsf{\label}};
  }
  
  \foreach \angle in {0,90,180,270}
    \draw[line width=2pt] (\angle:1.6cm) -- (\angle:2cm);

  \draw (0,0) -- (120:0.8cm); % hour
  \draw (0,0) -- (90:1cm);    % minute
\end{tikzpicture}%
\end{codeexample}

\begin{codeexample}[]
\tikz[shading=ball]
  \foreach \x / \cola in {0/red,1/green,2/blue,3/yellow}
    \foreach \y / \colb in {0/red,1/green,2/blue,3/yellow}
      \shade[ball color=\cola!50!\colb] (\x,\y) circle (0.4cm);
\end{codeexample}

	\medskip
  \textbf{Options to customise the foreach-statement.}
  
  The keys described below can be used in the \meta{options} argument
  to the |\foreach| command. They all have the path |/pgf/foreach/|,
  however, the path is set automatically when \meta{options} are
  parsed, so it does not have to explicitly stated.
  
\begin{key}{/pgf/foreach/var=\meta{variable}}
  This key provides an alternative way to specify variables:
  |\foreach [var=\x,var=\y]| is the same as |\foreach \x/\y|.
  If used, this key should be used before the other keys. 
\end{key}

\begin{key}{/pgf/foreach/evaluate=\meta{variable}| |\opt{|as |\meta{macro}| using |\meta{formula}| (trim integers)|}}
  By default list items are not evaluated: |1+2|, yields |1+2|, 
  not |3|. This key allows a variable to be evaulated using the 
  mathematical engine. The variable must have been specified either
  using the |var| key or in the \meta{variables} argument of the 
  |foreach| command. 
  By default, the result of the evaluation will be stored in
  \meta{variable}. However, the optional |as |\meta{macro} statement 
  can be used to store the result in \meta{macro}.

\begin{codeexample}[]
\foreach \x [evaluate=\x] in {2^0,2^...,2^8}{$\x$, }
\end{codeexample}

\begin{codeexample}[]
\foreach \x [evaluate=\x as \xeval] in {2^0,2^...,2^8}{$\x=\xeval$, }
\end{codeexample}

	The optional |using |\meta{formula} statement means an evaluation 
	does not have to be explicitly stated for each item in \meta{list}. 
	The	\meta{formula} should contain at least one reference to 
	\meta{variable}.

\begin{codeexample}[]
\tikz\foreach \x [evaluate=\x as \shade using \x*10] in {0,1,...,10}
  \node [fill=red!\shade!yellow, minimum size=0.65cm] at (\x,0) {\x};
\end{codeexample}
	
  The |(trim integers)| statement (which is also optional),
  will remove the decimal point and trailing zero, when the evaluation
  is an integer:
  
\begin{codeexample}[]
\foreach \x [evaluate=\x (trim integers)] in {3^0,3^...,3^8}{$\x$, }
\end{codeexample}
   
\end{key}

\begin{key}{/pgf/foreach/remember=\meta{variable}| as |\meta{macro}| |\opt{|(initially |\meta{value}|)|}}
	This key allows the item value stored in \meta{variable} to be
	remembered during the next iteration, stored in \meta{macro}.
	If a variable is evaluated, the result of this evaluation is 
	remembered.	
	By default the value of \meta{variable} is zero for the first
	iteration, however, the optional |(initially |\meta{value}|)| 
	statement, allows the \meta{macro} to be initially defined
	as \meta{value}. 
	
\begin{codeexample}[]
\foreach \x [remember=\x as \lastx (initially A)] in {B,...,H}{$\overrightarrow{\lastx\x}$, }
\end{codeexample}
\end{key}

\begin{key}{/pgf/foreach/count=\meta{macro}| |\opt{|from |\meta{value}}}
  This key allows \meta{macro} to hold the position in the list of 
  the current list. The optional |from |\meta{value} statement allows 
  the counting to begin from \meta{value}. 
  
\begin{codeexample}[]
\tikz[x=0.75cm,y=0.75cm]
  \foreach \x [count=\xi] in {a,...,e}
    \foreach \y [count=\yi] in {\x,...,e}
      \node [draw, top color=white, bottom color=blue!50, minimum size=0.666cm] 
        at (\xi,-\yi) {$\mathstrut\x\y$};
\end{codeexample}
\end{key}

\end{command}


\begin{command}{\breakforeach}
  If this command is given inside a |\foreach| command, no further
  executions of the \meta{commands} will occur. However, the current
  execution of the \meta{commands} is continued normally, so it is
  probably best to use this command only at the end of a |\foreach|
  command. 

\begin{codeexample}[]
\begin{tikzpicture}
  \foreach \x in {1,...,4}
    \foreach \y in {1,...,4}
    {
      \fill[red!50] (\x,\y) ellipse (3pt and 6pt);

      \ifnum \x<\y
        \breakforeach
      \fi
    }      
\end{tikzpicture}
\end{codeexample}
  
\end{command}



% Copyright 2006 by Till Tantau
%
% This file may be distributed and/or modified
%
% 1. under the LaTeX Project Public License and/or
% 2. under the GNU Free Documentation License.
%
% See the file doc/generic/pgf/licenses/LICENSE for more details.


\section{Date and Calendar Utility Macros}
\label{section-calendar}

This section describes the package |pgfcalendar|.

\begin{package}{pgfcalendar}
  This package can be used independently of \pgfname. It has two
  purposes:
  \begin{enumerate}
  \item It provides functions for working with dates. Most noticably,
    it can convert a date in ISO-standard format (like 1975-12-26) to
    a so-called Julian day number, which is defined in Wikipedia as
    follows:  ``The Julian day or Julian day number is the
    (integer) number of days that have elapsed since the initial epoch
    at noon Universal Time (UT) Monday, January 1, 4713 BC in the
    proleptic Julian calendar.'' The package also provides a function
    for converting a Julian day number to an ISO-format date.

    Julian day numbers make it very easy to work with days. For
    example, the date ten days in the future of 2008-02-20 can
    be computed by converting this date to a Julian day number, adding
    10, and then converting it back. Also, the day of week of a given
    date can be computed by taking the Julian day number modulo~7.
  \item It provides a macro for typesetting a calendar. This macro
    is highly configurable and flexible (for example, it can produce
    both plain text calendars and also complicated \tikzname-based
    calendars), but most users will not use the macro directly. It is
    the job of a frontend to provide useful configruations for
    typesetting calendars based on this command.
  \end{enumerate}
\end{package}


\subsection{Handling Dates}

\subsubsection{Conversions Between Date Types}

\begin{command}{\pgfcalendardatetojulian\marg{date}\marg{counter}}
  This macro converts a date in a format to be described in a moment
  to the Julian day number in the Gregorian calendar. The \meta{date}
  should expand to a string of the following form:
  \begin{enumerate}\label{calendar-date-format}
  \item It should start with a number representing the year. Use
    |\year| for the current year, that is, the year the file is being
    typeset.
  \item The year must be followed by a hyphen.
  \item Next should come a number representing the month. Use |\month|
    for the current month. You can, but need not, use leading
    zeros. For example, |02| represents February, just like |2|.
  \item The month must also be followed by a hyphen.
  \item Next you must either provide a day of month (again, a number
    and, again, |\day| yields the current day of month) or the keyword
    |last|. This keyword refers to the last day of the month, which is
    automatically computed (and which is a bit tricky to compute,
    especially for February).
  \item Optionally, you can next provide a plus sign followed by
    positive or negative number. This number of days will be added to
    the computed date.
  \end{enumerate}

  Here are some examples:
  \begin{itemize}
  \item |2006-01-01| refers to the first day of 2006.
  \item |2006-02-last| refers to February 28, 2006.
  \item |\year-\month-\day| refers to today.
  \item |2006-01-01+2| refser to January 3, 2006.
  \item |\year-\month-\day+1| refers to tomorrow.
  \item |\year-\month-\day+-1| refers to yesterday.
  \end{itemize}
  
  The conversion method is taken from the English Wikipedia entry on
  Julian days. 

  \newcount\mycount
  \example |\pgfcalendardatetojulian{2007-01-14}{\mycount}| sets
  |\mycount| to
  \pgfcalendardatetojulian{2007-01-14}{\mycount}\the\mycount.

  
\end{command}

\begin{command}{\pgfcalendarjuliantodate\marg{Julian day}\marg{year
      macro}\marg{month macro}\marg{day macro}}
  This command converts a Julian day number to an ISO-date. The
  \meta{Julian day} must be a number or \TeX\ counter, the \meta{year macro},
  \meta{month macro} and \meta{day macro} must be \TeX\ macro
  names. They will be set to numbers representing the year, month, and
  day of the given Julian day in the Gregorian calendar.

  The \meta{year macro} will be assigned the year without leading
  zeros. Note that this macro will produce year 0 (as opposed to other
  calendars, where year 0 does not exist). However, if you really need
  calendars for before the year 1, it is expected that you know what
  you are doing anyway.

  The \meta{month macro} gets assigned a two-digit number representing
  the month (with a leading zero, if necessary). Thus, the macro is
  set to |01| for January.

  The \meta{day macro} gets assigned a two-digit number representing
  the day of the month (again with a possible leading zero).

  To convert a Julian day number to an ISO-date you use code like the
  following:
\begin{verbatim}
\pgfcalendardatetojulian{2454115}{\myyear}{\mymonth}{\myday}
\edef\isodate{\myyear-\mymonth-\myday}
\end{verbatim}
  The above code sets |\isodate| to
  \pgfcalendarjuliantodate{2454115}{\myyear}{\mymonth}{\myday}%
  \edef\isodate{\myyear-\mymonth-\myday}\texttt{\isodate}.
\end{command}


\begin{command}{\pgfcalendarjuliantoweekday\marg{Julian day}\marg{week day counter}}
  This command converts a Julian day to a week day by computing the
  day modulo 7. The \meta{week day counter} must be a \TeX\
  counter. It will be set to 0 for a Monday, to 1 for a Tuesday, and
  so on.

  \example |\pgfcalendarjuliantoweekday{2454115}{\mycount}| sets
  |\mycount| to
  \pgfcalendarjuliantoweekday{2454115}{\mycount}\the\mycount. 
\end{command}


\subsubsection{Checking Dates}


\begin{command}{\pgfcalendarifdate\marg{date}\marg{tests}\marg{code}\marg{else code}}
  \label{pgfcalendarifdate}
  This command is used to execute code based on properties of
  \meta{date}. The \meta{date} must be a date in ISO-format. For
  this date, the \meta{tests} are checked (to be detailed later)
  and if one of the tests applied, the \meta{code} is
  executed. If none of the tests applies, the \meta{else code} is
  executed.

  \example |\pgfcalendarifdate{2007-02-07}{Wednesday}{Is a Wednesday}{Is not a Wednesday}|
  yields \texttt{\pgfcalendarifdate{2007-02-07}{Wednesday}{Is a
      Wednesday}{Is not a Wednesday}}.

  The \meta{tests} is a comma-separated list of key-value
  pairs. The following are defined by default:
  \begin{itemize}
  \itemcalendaroption{all} This test is passed by all dates.
  \itemcalendaroption{Monday} This test is passed by all dates that
  are Mondays.
  \itemcalendaroption{Tuesday} as above.
  \itemcalendaroption{Wednesday} as above.
  \itemcalendaroption{Thursday} as above.
  \itemcalendaroption{Friday} as above.
  \itemcalendaroption{Saturday} as above.
  \itemcalendaroption{Sunday} as above.
  \itemcalendaroption{workday} Passed by Mondays, Tuesdays,
  Wednesdays, Thursdays, and Fridays.
  \itemcalendaroption{weekend} Passed Saturdays and Sundays. 
  \itemcalendaroption{equals}|=|\meta{reference} The \meta{reference}
  can be in one of two forms: Either, it is a full ISO format date
  like |2007-01-01| or the year may be missing as in |12-31|. In the
  first case, the test is passed if \meta{date} is the same as
  \meta{reference}. In the second case, the test is passed if the
  month and day part of \meta{date} is the same as \meta{reference}.

  For example, the test |equals=2007-01-10| will only be passed by this
  particular date. The test |equals=05-01| will be passed by every first
  of May on any year.
  \itemcalendaroption{at least}|=|\meta{reference} This test works
  similarly to the |equals| test, only it is checked whether
  \meta{date} is equal to \meta{reference} or to any later
  date. Again, the \meta{reference} can be a full date like
  |2007-01-01| or a short version like |07-01|. For example,
  |at least=07-01| is true for every day in the second half of any
  year.
  \itemcalendaroption{at most}|=|\meta{reference} as above.
  \itemcalendaroption{between}|=|\meta{start reference}| and |\meta{end
    refernce} This test checks whether the current date lies between
  the two given reference dates. Both full and short version may be
  given.

  For example |between=2007-01-01 and 2007-02-28| is true for the days
  in January and February of 2007.

  For another example, |between=05-01 and 05-07| is true for the
  days of the first week of May of any year.
  \itemcalendaroption{day of month}|=|\meta{number} Passed by the day
  of month of the \meta{date} is \meta{number}. For example, the test
  |day of month=1| is passed by every first of every month.
  \itemcalendaroption{end of month}\opt{|=|\meta{number}} Passed by
  the day of month of the \meta{date} that is \meta{number} from the
  end of the month. For example, the test |end of month=1| is passed
  by the last day of every month, the test |end of month=2| is passed
  by the second last day of every month. If \meta{number} is omitted,
  it is assumed to be |1|.
  \end{itemize}

  In addition to the above checks, you can also define new checks. To
  do so, you must add a new key to the key-value group |pgfcalendar|
  using |\define@key|. The job of the code of this new key is to
  possibly set the \TeX-if |\ifpgfcalendarmatches| to true (if it is
  already true, no action should be taken) to indicate that the
  \meta{date} passes the test setup by this new key.

  In order to perform the test, the key code needs to know the date
  that should be checked. The date is available through a macro, but a
  whole bunch of additional information about this date is also
  available through the following macros:
  \begin{itemize}
  \item |\pgfcalendarifdatejulian|
    is the Julian day number of the \meta{date} to be checked.
  \item |\pgfcalendarifdateweekday|
    is the weekday of the \meta{date} to be checked.
  \item |\pgfcalendarifdateyear|
    is the year of the \meta{date} to be checked.
  \item |\pgfcalendarifdatemonth|
    is the month of the \meta{date} to be checked.
  \item |\pgfcalendarifdateday|
    is the day of month of the \meta{date} to be checked.
  \end{itemize}

  For example, let us define a new key that checks whether the
  \meta{date} is a Workers day (first of May). This can be done as
  follows:
\begin{verbatim}
\define@key{pgfcalendar}{workers day}[]
{
  \ifnum\pgfcalendarifdatemonth=5\relax
    \ifnum\pgfcalendarifdateday=1\relax
      \pgfcalendarmatchestrue
    \fi
  \fi
}
\end{verbatim}
\end{command}


\subsubsection{Typesetting Dates}




\begin{command}{\pgfcalendarweekdayname\marg{week day number}}
  This command expands to a textual representation of the day of week,
  given by the \meta{week day number}. Thus,
  |\pgfcalendarweekdayname{0}| expands to |Monday| if the current
  language is English and to |Montag| if the current language is
  German, and so on. See Section~\ref{section-calendar-locale} for
  more details on translations.

  \example |\pgfcalendarweekdayname{2}| yields
  \texttt{\pgfcalendarweekdayname{2}}. 
\end{command}


\begin{command}{\pgfcalendarweekdayshortname\marg{week day number}}
  This command works similiarly to the previous command, only an
  abbreviated version of the week day is produced.

  \example |\pgfcalendarweekdayshortname{2}| yields
  \texttt{\pgfcalendarweekdayshortname{2}}. 
\end{command}


\begin{command}{\pgfcalendarmonthname\marg{month number}}
  This command expands to a textual representation of the month, which
  is given by the \meta{month number}.

  \example |\pgfcalendarmonthname{12}| yields
  \texttt{\pgfcalendarmonthname{12}}. 
\end{command}


\begin{command}{\pgfcalendarmonthshortname\marg{month number}}
  As above, only an abbreviated version is produced.

  \example |\pgfcalendarmonthshortname{12}| yields
  \texttt{\pgfcalendarmonthshortname{12}}.   
\end{command}



\subsubsection{Localization}

\label{section-calendar-locale}
All textual representations of week days or months (like ``Monday'' or
``February'') are wrapped with |\translate| commands from the
|translator| package (it this package is not loaded, no translation
takes place). Furthermore, the |pgfcalendar| package will try to load
the |translator-months-dictionary|, if the |translator| package is
loaded.

The net effect of all this is that all dates will be translated to the
current language setup in the |translator| package. See the
documentation of this package for more details.



\subsection{Typesetting Calendars}

\begin{command}{\pgfcalendar\marg{prefix}\marg{start date}\marg{end
      date}\marg{rendering code}}
  This command can be used to typeset a calendar. It is a very general
  command, the actual work has to be done by giving clever
  implementations of \meta{rendering code}. Note that this macro need
  \emph{not} be called inside a |{pgfpicture}| environment (even
  though it typically will be) and you can use it to typeset calendars
  in normal \TeX\ or using packages other than \pgfname.

  \medskip
  \textbf{Basic typesetting process.}
  A calendar is typeset as follows: The \meta{start date} and
  \meta{end date} specify a range of dates. For each date in this 
  range the \meta{rendering code} is executed with certain macros
  setup to yield information about the \emph{current date}
  (the current date in the enumeration of dates of the
  range). Typically, the \meta{rendering code} places nodes inside a
  picture, but it can do other things as well. Note that it is also
  the job of the \meta{rendering code} to position the calendar
  correctly. 

  The different calls of the \meta{rending code} are not
  surrounded by \TeX\ groups (though you can do so yourself, of
  course). This means that settings can accumulate between different
  calls, which is often desirable and useful.

  \medskip
  \textbf{Information about the current date.}
  Inside the \meta{rendering code}, different macros can be access:

  \begin{itemize}
  \item |\pgfcalendarprefix|
    The \meta{prefix} parameter. This prefix is recomended for nodes
    inside the calendar, but you have to use it yourself explicitly.
  \item |\pgfcalendarbeginiso|
    The \meta{start date} of range being typeset in ISO format (like 2006-01-10).
  \item |\pgfcalendarbeginjulian|
    Julian day number of \meta{start date}.
  \item |\pgfcalendarendiso|
    The \meta{end date} of range being typeset in ISO format.
  \item |\pgfcalendarendjulian|
    Julian day number of \meta{end date}.
  \item |\pgfcalendarcurrentjulian| This \TeX\ count holds the 
    Julian day number of day currently begin rendered.
  \item |\pgfcalendarcurrentweekday| The weekday
    (a number with zero representing Monday) of the current date.
  \item |\pgfcalendarcurrentyear| The year of the current date.
  \item |\pgfcalendarcurrentmonth| The month of the current date
    (always two digits with a leading zero, if necessary).
  \item |\pgfcalendarcurrentday| The day of month of the current date
    (alwyas two digits).
  \end{itemize}

  \medskip
  {\bfseries The |\ifdate| command.}
  Inside the |\pgfcalendar| the macro |\ifdate| is available
  locally:
  \begin{command}{\ifdate\marg{tests}\marg{code}\marg{else code}}
    This command has the same effect as calling |\pgfcalendarifdate|
    for the current date.
  \end{command}

  \medskip
  \textbf{Examples.}
  In a first example, let us create a very simple calendar: It just
  lists the dates in a certain range.
\begin{codeexample}[vbox,ignorespaces]
\pgfcalendar{cal}{2007-01-20}{2007-02-10}{\pgfcalendarcurrentday\ }    
\end{codeexample}
  Let us now make this a little more interesting: Let us add a line
  break after each Sunday.
\begin{codeexample}[vbox,ignorespaces]
\pgfcalendar{cal}{2007-01-20}{2007-02-10}
{
  \pgfcalendarcurrentday\
  \ifdate{Sunday}{\par}{}
}    
\end{codeexample}
  We now want to have all Mondays to be aligned on a column. For this,
  different approaches work. Here is one based positioning each day
  horizontally using a skip.
\begin{codeexample}[vbox,ignorespaces]
\pgfcalendar{cal}{2007-01-20}{2007-02-10}
{%
  \leavevmode%
  \hbox to0pt{\hskip\pgfcalendarcurrentweekday cm\pgfcalendarcurrentday\hss}%
  \ifdate{Sunday}{\par}{}%
}    
\end{codeexample}
  Let us now typeset two complete months.
\begin{codeexample}[vbox,ignorespaces]
\pgfcalendar{cal}{2007-01-01}{2007-02-28}{%
  \ifdate{day of month=1}{
    \par\bigskip\hbox to7.5cm{\itshape\hss\pgfcalendarshorthand{J}\hss}\par
  }{}%
  \leavevmode%
  {%
    \ifdate{weekend}{\color{black!50}}{\color{black}}%
    \hbox to0pt{%
      \hskip\pgfcalendarcurrentweekday cm%
      \hbox to1cm{\hss\pgfcalendarshorthand{d}}\hss%
    }%
  }%
  \ifdate{Sunday}{\par}{}%
}    
\end{codeexample}
  For our final example, we use a |{tikzpicture}|. 
\begin{codeexample}[vbox,ignorespaces]
\begin{tikzpicture}    
  \pgfcalendar{cal}{2007-01-20}{2007-02-10}{%
    \ifdate{workday}{\tikzstyle{filling}=[fill=blue!20]}{\tikzstyle{filling}=[fill=red!20]}
    \node (\pgfcalendarsuggestedname) at (\pgfcalendarcurrentweekday,0)
      [anchor=base,circle,filling] {\pgfcalendarcurrentday};
    \ifdate{Sunday}{\pgftransformyshift{-3em}}{}%
  }
  \draw (cal-2007-01-21) -- (cal-2007-02-03);
\end{tikzpicture}
\end{codeexample}
\end{command}


\begin{command}{\pgfcalendarshorthand\marg{kind}\marg{representation}}
  \label{pgfcalendarshorthand}
  This command can be used inside a |\pgfcalendar|, where it will
  expand to a representation of the current day, month, year or day of
  week, depending on whether \meta{kind} is |d|, |m|, |y| or |w|. The
  \meta{representation} can be one of the following: |-|, |=|, |0|,
  |.|, and |t|. They have the following meanings:
  \begin{itemize}
  \item The minus sign selects the shortest numerical representation
    possible (no leading zeros).
  \item The equal sign also selects the shortest numerical
    representation, but a space is added to single digit days and
    months (thereby ensuring that they have the same length as other
    days).
  \item The zero digit selects a two-digit numerical representation
    for days and months. For years it is allowed, but has no effect.
  \item The letter |t| selects a textual representation.
  \item The dot selects an abbreviated textual representation.
  \end{itemize}
  Normally, you should say |\let\%=\pgfcalendarshorthand| locally, so
  that you can write |\%wt| instead of the much more cumbersome
  |\pgfcalendarshorthand{w}{t}|. 

\begin{codeexample}[]
\let\%=\pgfcalendarshorthand
\pgfcalendar{cal}{2007-01-20}{2007-01-20}
{ ISO form: \%y0-\%m0-\%d0, long form: \%wt, \%mt \%d-, \%y0}    
\end{codeexample}
\end{command}


\begin{command}{\pgfcalendarsuggestedname}
  This macro expands to a suggested name for nodes representing days
  in a calendar. In detail, it expands to the \meta{prefix} of the
  calendar, followed by a hyphen, followed by the ISO format version
  of the date. Thus, when the date |2007-01-01| is typeset in a
  calendar for the prefix |mycal|, the macro expands to
  |mycal-2007-01-01|. 
\end{command}



% Copyright 2003 by Till Tantau <tantau@cs.tu-berlin.de>.
%
% This program can be redistributed and/or modified under the terms
% of the LaTeX Project Public License Distributed from CTAN
% archives in directory macros/latex/base/lppl.txt.


\section{Page Management}

This section describes the |pgfpages| packages. Although this package
is not concerned with creating pictures, its implementation relies so
heavily on \pgfname\ that it is documented here. Currently, |pgfpages|
only works with \LaTeX, but if you are adventurous, feel free to hack
the code so that it also works with plain \TeX.

The aim of |pgfpages| is to provide a flexible way of putting multiple
pages on a single page \emph{inside \TeX}. Thus, |pgfpages| is quite
different from useful tools like |psnup| or |pdfnup| insofar as it
creates its output in a single pass. Furthermore, it works uniformly
with both |latex| and |pdflatex|, making it easy to put multiple pages
on a single page without any fuss.

A word of warning: \emph{using |pgfpages| will destroy
  hyperlinks}. Actually, the hyperlinks are not destroyed, only they
will appear at totally wrong positions on the final output. This is
due to a fundamental flaw in the \pdf\ specification: In \pdf\ the
bounding rectangle of a hyperlink is given in ``absolute
page coordinates'' and translations or rotations do not affect
them. Thus, the transformations applied by |pgfpages| to put the pages
where you want them are (cannot, even) be applied to the coordinates
of hyperlinks. It is unlikely that this will change in the foreseeable
future.


\subsection{Basic Usage}

The internals of |pgfpages| are complex since the package can do all
sorts of interesting tricks. For this reason, so-called \emph{layouts}
are predefined that setup all option in appropriate ways.

You use a layout as follows:
\begin{codeexample}[code only]
\documentclass{article}

\usepackage{pgfpages}
\pgfpagesuselayout{2 on 1}[a4paper,landscape,border shrink=5mm]

\begin{document}
This text is shown on the left.
\clearpage
This text is shown on the right.
\end{document}
\end{codeexample}

The layout |2 on 1| puts two pages on a single page. The option
|a4paper| tells |pgfpages| that the \emph{resulting} page (called the
\emph{physical} page in the following) should be |a4paper| and it
should be landscape (which is quite logical since putting two portrait
pages next to each other gives a landscape page). Normally, the
\emph{logical} pages, that is, the pages that \TeX\ ``thinks'' that it
is typesetting, will have the same sizes, but this need not be the
case. |pgfpages| will automatically scale down the logical pages such
that two logical pages fit next to each other inside a DIN A4 page.

The |border shrink| tells |pgfpages| that it should add an additional
5mm to the shrinking such that a 5mm-wide border is shown around the
resulting logical pages.

As a second example, let us put two pages produced by the
\textsc{beamer} class on a single page:

\begin{codeexample}[code only]
\documentclass{beamer}

\usepackage{pgfpages}
\pgfpagesuselayout{2 on 1}[a4paper,border shrink=5mm]

\begin{document}
\begin{frame}
  This text is shown at the top.
\end{frame}
\begin{frame}
  This text is shown at the bottom.
\end{frame}
\end{document}
\end{codeexample}

Note that we do not use the |landscape| option since \textsc{beamer}'s
logical pages are already in landscape mode and putting two landscape
pages on top of each other results in a portrait page. However, if you
had used the |4 on 1| layout, you would have had to add |landscape|
once more, using the |8 on 1| you must not, using |16 on 1| you need
it yet again. And, no, there is no |32 on 1| layout.

Another word of caution: \emph{using |pgfpages| will produce wrong
  page numbers in the |.aux| file}. The reason is that \TeX\
instantiates the page numbers when writing an |.aux| file only when
the physical page is shipped out. Fortunately, this problem is easy to
fix: First, typeset our file normally without using the
|\pgfpagesuselayout| command (just put the comment marker |%| before it)
Then, rerun \TeX\ with the |\pgfpagesuselayout| command included and add
the command |\nofiles|. This command ensures that the |.aux| file is
not modified, which is exactly what you want. So, to typeset the above
example, you should actually first \TeX\ the following file:

\begin{codeexample}[code only]
\documentclass{article}

\usepackage{pgfpages}
%%\pgfpagesuselayout{2 on 1}[a4paper,landscape,border shrink=5mm]
%%\nofiles

\begin{document}
This text is shown on the left.
\clearpage
This text is shown on the right.
\end{document}
\end{codeexample}
and then typeset
\begin{codeexample}[code only]
\documentclass{article}

\usepackage{pgfpages}
\pgfpagesuselayout{2 on 1}[a4paper,landscape,border shrink=5mm]
\nofiles

\begin{document}
This text is shown on the left.
\clearpage
This text is shown on the right.
\end{document}
\end{codeexample}

The final basic example is the |resize to| layout (it works a bit like
a hypothetical |1 on 1| layout). This layout resizes the logical page
such that is fits the specified physical size. Since this does not
change the page numbering, you need not worry about the |.aux| files
with this layout. For example, adding the following lines will ensure
that the physical output will fit on DIN A4 paper:
\begin{codeexample}[code only]
\usepackage{pgfpages}
\pgfpagesuselayout{resize to}[a4paper]
\end{codeexample}

This can be very useful when you have to handle lots of papers that
are typeset for, say, letter paper and you have an A4 printer or the
other way round. For example, the following article will be fit for
printing on letter paper:
\begin{codeexample}[code only]
\documentclass[a4paper]{article}
%% a4 is currently the logical size and also the physical size

\usepackage{pgfpages}
\pgfpagesuselayout{resize to}[letterpaper]
%% a4 is still the logical size, but letter is the physical one

\begin{document}
  \title{My Great Article}
...
\end{document}
\end{codeexample}



\subsection{The Predefined Layouts}

This section explains the predefined layouts in more detail. You
select a layout using the following command:
\begin{command}{\pgfpagesuselayout\marg{layout}\oarg{options}}
  Installs the specified \meta{layout} with the given \meta{options}
  set. The predefined layouts and their permissible options are
  explained below.

  If this function is called multiple times, only the last call
  ``wins.'' You can thereby overwrite any previous settings. In
  particular, layouts \emph{do not} accumulate.

  \example |\pgfpagesuselayout{resize to}[a4paper]|
\end{command}

\begin{pgflayout}{resize to}
  This layout is used to resize every logical page to a specified
  physical size. To determine the target size, the following options
  may be given:
  \begin{itemize}
  \item
    \declare{|physical paper height=|\meta{size}} sets the
    height of the physical pape size to \meta{size}.
  \item
    \declare{|physical paper width=|\meta{size}} sets the
    width of the physical pape size to \meta{size}.
  \item
    \declare{|a0paper|} sets the physical page size to DIN A0 paper.
  \item
    \declare{|a1paper|} sets the physical page size to DIN A1 paper.
  \item
    \declare{|a2paper|} sets the physical page size to DIN A2 paper.
  \item
    \declare{|a3paper|} sets the physical page size to DIN A3 paper.
  \item
    \declare{|a4paper|} sets the physical page size to DIN A4 paper.
  \item
    \declare{|a5paper|} sets the physical page size to DIN A5 paper.
  \item
    \declare{|a6paper|} sets the physical page size to DIN A6 paper.
  \item
    \declare{|letterpaper|} sets the physical page size to the
    American letter paper size.
  \item
    \declare{|legalpaper|} sets the physical page size to the
    American legal paper size.
  \item
    \declare{|executivepaper|} sets the physical page size to the
    American executive paper size.
  \item
    \declare{|landscape|} swaps the height and the width of the
    physical paper.
  \item
    \declare{|border shrink=|\meta{size}} additionally reduces the
    size of the logical page on the physical page by \meta{size}.
  \end{itemize}
\end{pgflayout}

\begin{pgflayout}{2 on 1}
  Puts two logical pages alongside each other on each physical page if
  the logical height is larger than the logical width (logical pages
  are in portrait mode). Otherwise, two
  logical pages are put on top of each other (logical pages are in
  landscape mode). When using this layout, it is advisable to use the
  |\nofiles| command, but this is not done automatically.

  The same \meta{options} as for the |resize to| layout an be used,
  plus the following option:
  \begin{itemize}
  \item
    \declare{|odd numbered pages right|}
    places the first page on the right.
  \end{itemize}
\end{pgflayout}


\begin{pgflayout}{4 on 1}
  Puts four logical pages on a single physical page.
  The same \meta{options} as for the |resize to| layout an be used.
\end{pgflayout}

\begin{pgflayout}{8 on 1}
  Puts eight logical pages on a single physical page. As for |2 on 1|,
  the orientation depends on whether the logical pages are in
  landscape mode or in portrait mode.
\end{pgflayout}

\begin{pgflayout}{16 on 1}
  This is for the \textsc{ceo}.
\end{pgflayout}

\begin{pgflayout}{rounded corners}
  \label{layout-rounded-corners}
  This layout adds ``rounded corners'' to every page, which,
  supposedly, looks nicer during presentations with projectors
  (personally, I doubt this). This is done by (possibly) resizing the
  page to the physical page size. Then four black rectangles are
  drawn in each corner. Next, a clipping region is set up that
  contains all of the logical page except for little rounded
  corners. Finally, the logical page is draw, clipped against the
  clipping region. 

  Note that every logical page should fill its background for this to
  work.

  In addition to the \meta{options} that can be given to |resize to|
  the following options may be given.
  \begin{itemize}
    \item \declare{|corner width=|\meta{size}} specifies the size of
    the corner.
  \end{itemize}

  \begin{codeexample}[code only]
\documentclass{beamer}
\usepackage{pgfpages}
\pgfpagesuselayout{rounded corners}[corner width=5pt]
\begin{document}
...
\end{document}
\end{codeexample}
\end{pgflayout}

\begin{pgflayout}{two screens with lagging second}
  This layout puts two logical pages alongside each other. The second
  page always shows what the main
  page showed on the previous physical page. Thus, the second page
  ``lags behind'' the main page. This can be useful when you have to
  projectors attached to your computer and can show different parts of
  a physical page on different projectors.

  The following \meta{options} may be given:
  \begin{itemize}
  \item \declare{|second right|} puts the second page right of the
    main page. This will make the physical pages twice as wide
    as the logical pages, but it will retain the height.
  \item \declare{|second left|} puts the second page left,
    otherwise it behave the same as |second right|.
  \item \declare{|second bottom|} puts the second page below the main
    page. This make the physical pages twice as high as the logical
    ones.
  \item \declare{|second top|} works like |second bottom|.      
  \end{itemize}
\end{pgflayout}

\begin{pgflayout}{two screens with optional second}
  This layout works similarly to
  |two screens with lagging second|. The difference is that the
  contents of the second screen only changes when one of the commands
  |\pgfshipoutlogicalpage{2}|\marg{box} or
  |\pgfcurrentpagewillbelogicalpage{2}| is called. The first puts the
  given \meta{box} on the second page. The second specifies that the
  current page should be put there, once it is finished.

  The same options as for |two screens with lagging second| may be
  given. 
\end{pgflayout}



You can define your own predefined layouts using the following
command:

\begin{command}{\pgfpagesdeclarelayout\marg{layout}\marg{before
      actions}\marg{after actions}}
  This command predefines a \meta{layout} that can later be installed
  using the |\pgfpagesuselayout| command.

  When |\pgfpagesuselayout|\marg{layout}\oarg{options} is called, the
  following happens: First, the \meta{before actions} are
  executed. They can be used, for example, to setup default values for
  keys. Next, |\setkeys{pgfpagesuselayoutoption}|\marg{options} is
  executed. Finally, the \meta{after actions} are executed.

  Here is an example:
\begin{codeexample}[code only]
\pgfpagesdeclarelayout{resize to}
{
  \def\pgfpageoptionborder{0pt}
}
{
  \pgfpagesphysicalpageoptions
  {%
    logical pages=1,%
    physical height=\pgfpageoptionheight,%
    physical width=\pgfpageoptionwidth%
  }
  \pgfpageslogicalpageoptions{1}
  {%
    resized width=\pgfphysicalwidth,%
    resized height=\pgfphysicalheight,%
    border shrink=\pgfpageoptionborder,%
    center=\pgfpoint{.5\pgfphysicalwidth}{.5\pgfphysicalheight}%
  }%
}
\end{codeexample}
\end{command}




\subsection{Defining a Layout}

If none of the predefined layouts meets your problem or if you wish to
modify them, you can create layouts from scratch. This section
explains how this is done.

Basically, |pgfpages| hooks into \TeX's |\shipout| function. This
function is called whenever \TeX\ has completed typesetting a page and
wishes to send this page to the |.dvi| or |.pdf| file. The |pgfpages|
package redefines this command. Instead of sending the page to the output
file, |pgfpages| stores it in an internal box and then acts as if the
page had been output. When \TeX\ tries to output the next page using
|\shipout|, this call is once more intercepted and the page is stored
in another box. These boxes are called \emph{logical pages}.

At some point, enough logical pages have been accumulated such that a
\emph{physical page} can be output. When this happens, |pgfpages|
possibly scales, rotates, and translates the logical pages (and
possibly even does further modifications) and then puts them at
certain positions of the \emph{physical} page. Once this page is fully
assembled, the ``real'' or ``original'' |\shipout| is called to
send the physical page to the output file.

In reality, things are slightly more complicated. First, once a
physical page has been shipped out, the logical pages are usually
voided, but this need not be the case. Instead, it is possible that
certain logical page just retain their contents after the physical
page has been shipped out and these pages need not be filled once more
before a physical shipout can occur. However, the contents of these
logical pages can still be changed using special commands. It is also
possible that after a shipout certain logical pages are filled with
the contents of \emph{other} logical pages.

A \emph{layout} defines for each logical page where it will go on the
physical page and which further modifications should be done. The
following two commands are used to define the layout:

\begin{command}{\pgfpagesphysicalpageoptions\marg{options}}
  This command sets the characteristic of the ``physical'' page. For
  example, it is used to specify how many logical pages there are and
  how many logical pages must be accumulated before a physical page is
  shipped out. How each individual logical page is typeset is
  specified using the command |\pgfpageslogicalpageoptions|, described
  later.

  \example A layout for putting two portrait pages on a single
  landscape page:
\begin{codeexample}[code only]
\pgfpagesphysicalpageoptions
{%
  logical pages=2,%
  physical height=\paperwidth,%
  physical width=\paperheight,%
}

\pgfpageslogicalpageoptions{1}
{%
  resized width=.5\pgfphysicalwidth,%
  resized height=\pgfphysicalheight,%
  center=\pgfpoint{.25\pgfphysicalwidth}{.5\pgfphysicalheight}%
}%
\pgfpageslogicalpageoptions{2}
{%
  resized width=.5\pgfphysicalwidth,%
  resized height=\pgfphysicalheight,%
  center=\pgfpoint{.75\pgfphysicalwidth}{.5\pgfphysicalheight}%
}%
\end{codeexample}

  The following \meta{options} may be set:
  \begin{itemize}
    \item \declare{|logical pages=|\meta{logical pages}} specified how many
    logical pages there are, in total. These are numbered 1 to
    \meta{logical pages}.
    \item \declare{|first logical shipout=|\meta{first}}. See the the
      next option. By default, \meta{first} is 1.
    \item \declare{|last logical shipout=|\meta{last}}. Together
    with the previous option, these two options define an interval of
    pages inside the range 1 to \meta{logical pages}. Only this range
    is used to store the pages that are shipped out by \TeX. This
    means that after a physical shipout has just occured (or at the
    beginning), the first time \TeX\ wishes to perform a shipout, the
    page to be shipped out is stored in logical page \meta{first}. The
    next time \TeX\ performs a shipout, the page is stored in logical
    page $\meta{first} +1$ and so on, until the logical page
    \meta{last} is also filled. Once this happens, a physical shipout
    occurs and the process starts once more.

    Note that logical pages that lie outside the interval between
    \meta{first} and \meta{last} are filled only indirectly or when
    special commands are used.

    By default, \meta{last} equals \meta{logical pages}.
  \item \declare{|current logical shipout=|\meta{current}} changes
    an internal counter such that \TeX's next logical shipout will be
    stored in logical page \meta{current}.

    This option can be used to ``warp'' the logical page filling
    mechanism to a certain page. You can both skip logical pages and
    overwrite already filled logical pages. After the logical page
    \meta{current} has been filled, the internal counter is
    incremented normally as if the logical page \meta{current} had
    been ``reached'' normally. If you specify a \meta{current} larger
    to \meta{last}, a physical shipout will occur after the logical
    page \meta{current} has been filled.
  \item
    \declare{|physical height=|\meta{height}}
    specifies the height of the physical pages. This height is
    typically different from the normal  |\paperheight|, which is used
    by \TeX\ for its typesetting and page breaking purposes.
  \item
    \declare{|physical width=|\meta{width}}
    specifies the physical width.
  \end{itemize}
\end{command}


\begin{command}{\pgfpageslogicalpageoptions\marg{logical page number}\marg{options}}
  This command is used to specify where the logical page number
  \meta{logical page number} will be placed on the physical page. In
  addition, this command can be used to install additional ``code'' to
  be executed when this page is put on the physical page.

  The number \meta{logical page number} should be between 1 and
  \meta{logical pages}, which has previously been installed using the
  |\pgfpagesphysicalpageoptions| command.

  The following \meta{options} may be given:
  \begin{itemize}
  \item
    \declare{|center=|\meta{pgf point}}
    specifies the center of the logical page inside the physical page
    as a \pgfname-point. The origin of the coordinate system of the
    physical page is at the \emph{lower} left corner.

\begin{codeexample}[code only]
\pgfpageslogicalpageoptions{1}
{% center logical page on middle of left side
  center=\pgfpoint{.25\pgfphysicalwidth}{.5\pgfphysicalheight}%
  resized width=.5\pgfphysicalwidth,%
  resized height=\pgfphysicalheight,%
}
\end{codeexample}

  \item
    \declare{|resized width=|\meta{size}}
    specifies the width that the logical page should have \emph{at
    most} on the physical page. To achieve this width, the pages is
    scaled down appropriately \emph{or more}. The ``or more'' part
    can happen if the |resize height| option is also used. In this
    case, the scaling is chosen such that both the specified height
    and width are met. The aspect ratio of a logical page is not
    modified.
  \item
    \declare{|resized height=|\meta{height}}
    specifies the maximum height of the logical page.
  \item
    \declare{|original width=|\meta{width}}
    specifies the width the \TeX\ ``thinks'' that the logical page
    has. This width is |\paperwidth| at the point of invocation, by
    default. Note that setting this width to something different from
    |\paperwidth| does \emph{not} change the |\pagewidth| during
    \TeX's typesetting. You have to do that yourself.

    You need this option only for special logical pages that have
    a height or width different from the normal one and for which you
    will (later on) set these sizes yourself.
  \item
    \declare{|original height=|\meta{height}}
    works like |original width|.
  \item
    \declare{|scale=|\meta{factor}}
    scales the page by at least the given \meta{factor}. A
    \meta{factor} of |0.5| will half the size of the page, a factor or
    |2| will double the size. ``At least'' means that if options like
    |resize height| are given and if the scaling required to meet that
    option is less than \meta{factor}, that other scaling is used
    instead. 
  \item
    \declare{|xscale=|\meta{factor}}
    scales the logical page along the $x$-axis by the given
    \meta{factor}. This scaling is done independently of any other
    scaling. Mostly, this option is useful for a factor of |-1|, which
    flips the page along the $y$-axis. The aspect ratio is not kept.
  \item
    \declare{|yscale=|\meta{factor}}
    works like |xscale|, only for the $y$-axis.
  \item
    \declare{|rotation=|\meta{degree}}
    rotates the page by \meta{degree} around its center. Use a degree
    of |90| or |-90| to go from portrait to landscape and back. The
    rotation need not be a multiple of |90|.
  \item
    \declare{|copy from=|\meta{logical page number}}.
    Normally, after a physical shipout has occured, all logical pages
    are voided in a loop. However, if this option is given, the
    current logical page is filled with the contents of the old
    logical page number \meta{logical page number}.

    \example Have logical page 2 retain its contents:
\begin{codeexample}[code only]
\pgfpageslogicalpageoptions{2}{copy from=2}
\end{codeexample}

    \example Let logical page 2 show what logical page 1 showed on the
    just-shipped-out physical page:
\begin{codeexample}[code only]
\pgfpageslogicalpageoptions{2}{copy from=1}
\end{codeexample}
  \item
    \declare{|border shrink|=\meta{size}}
    specifies an addition reduction of the size to which the page is
    page is scaled down.
  \item
    \declare{|border code|=\meta{code}}.
    When this option is given, the \meta{code} is executed before the
    page box is inserted with a path preinstalled that is a rectangle
    around the current logical page. Thus, setting \meta{code} to
    |\pgfstroke| draws a rectangle around the logical page. Setting
    \meta{code} to |\pgfsetlinewidth{3pt}\pgfstroke| results in a
    thick (ugly) frame. Adding dashes and filling can result in
    arbitrarily funky and distracting borders.

    You can also call |\pgfdiscardpath| and add your own path
    construction code (for example to paint a rectangle with rounded
    corners). The coordinate system is  setup in such a way that a
    rectangle starting at the origin and having the height and width
    of \TeX-box 0 will result in a rectangle filling exactly the
    logical page currently being put on the physical page. The logical
    page is inserted \emph{after} these commands have been executed.

    \example Add a rectangle around the page:
\begin{codeexample}[code only]
\pgfpageslogicalpageoptions{1}{border code=\pgfstroke}
\end{codeexample}
  \item
    \declare{|corner width|=\meta{size}}
    adds black ``rounded corners'' to the page. See the description of
    the predefined layout |rounded corners| on
    page~\pageref{layout-rounded-corners}. 
  \end{itemize}
\end{command}




\subsection{Creating Logical Pages}

Logical pages are created whenever a \TeX\ thinks that a page is full
and performs a |\shipout| command. This will cause |pgfpages| to store
the box that was supposed to be shipped out internally until enough
logical pages have been collected such that a physical shipout can
occur.

Normally, whenever a logical shipout occurs that current page is
stored in logical page number \meta{current logical page}. This
counter is then incremented, until it is larger than \meta{last
  logical shipout}. You can, however, directly change the value of
\meta{current logical page} by calling |\pgfpagesphysicalpageoptions|.

Another way to set the contents of a logical page is to use the
following command:

\begin{command}{\pgfpagesshipoutlogicalpage\marg{number}\meta{box}}
  This command sets to logical page \meta{number} to \meta{box}. The
  \meta{box} should be the code of a \TeX\ box command. This command
  does not influence the counter \meta{current logical page} and does
  not cause a physical shipout.

\begin{codeexample}[code only]
\pgfpagesshipoutlogicalpage{0}\vbox{Hi!}
\end{codeexample}

  This command can be used to set the contents of logical pages that
  are normally not filled.
\end{command}

The final way of setting a logical page is using the following
command: 

\begin{command}{\pgfpagescurrentpagewillbelogicalpage\marg{number}}
  When the current \TeX\ page has been typeset, it will be become the given
  logical page \meta{number}. This command ``interrupts'' the normal
  order of logical pages, that is, it behaves like the previous
  command and does not update the  \meta{current logical page}
  counter. 

\begin{codeexample}[code only]
\pgfpagesuselayout{two screens with optional second}
...
Text for main page.
\clearpage

\pgfpagescurrentpagewillbelogicalpage{2}
Text that goes to second page
\clearpage

Text for main page.
\end{codeexample}
\end{command}

%%% Local Variables: 
%%% mode: latex
%%% TeX-master: "pgfmanual"
%%% End: 

% Copyright 2003 by Till Tantau <tantau@cs.tu-berlin.de>.
%
% This program can be redistributed and/or modified under the terms
% of the LaTeX Project Public License Distributed from CTAN
% archives in directory macros/latex/base/lppl.txt.

\section{Extended Color Support}

This section documents the package \texttt{xxcolor}, which is
currently distributed as part of \pgfname. This package extends the
\texttt{xcolor} package, written by Uwe Kern, which in turn extends
the \texttt{color} package. I hope that the commands in
\texttt{xxcolor} will some day migrate to \texttt{xcolor}, such that
this package becomes superfluous.

The main aim of the \texttt{xxcolor} package is to provide an
environment inside which all colors are ``washed out'' or ``dimmed.''
This is useful in numerous situations and must typically be achieved
in a roundabout manner if such an environment is not available.

\begin{environment}{{colormixin}\marg{mix-in specification}}
  The mix-in specification is applied to all colors inside
  the environment. At the beginning of the environment, the mix-in is
  applied to the current color, i.\,e., the color that was in effect
  before the environment started. A mix-in specification is a number
  between 0 and 100 followed by an exclamation mark and a color
  name. When a |\color| command is 
  encountered inside a mix-in environment, the number states what
  percentage of the desired color should be used. The rest is
  ``filled up'' with the color given in the mix-in
  specification. Thus, a mix-in specification like |90!blue|
  will mix in 10\% of blue into everything, whereas |25!white| will
  make everything nearly white.

\begin{codeexample}[width=4cm]
\begin{minipage}{3.5cm}\raggedright
\color{red}Red text,%
\begin{colormixin}{25!white}
  washed-out red text,
  \color{blue} washed-out blue text,
  \begin{colormixin}{25!black}
    dark washed-out blue text,
    \color{green} dark washed-out green text,%
  \end{colormixin}
  back to washed-out blue text,%
\end{colormixin}
and back to red.
\end{minipage}%
\end{codeexample}
\end{environment}

Note that the environment only changes colors that have been installed
using the standard \LaTeX\ |\color| command. In particular,
the colors in images are not changed. There is, however, some support
offered by the commands |\pgfuseimage| and
|\pgfuseshading|. If the first command is invoked 
inside a |colormixin| environment with the parameter, say,
|50!black| on an image with the name |foo|, the command
will first check whether there is also a defined image with the name
|foo.!50!black|. If so, this image is used instead. This allows
you to provide a different image for this case. If you nest
|colormixin| environments, the different mix-ins are all appended. For
example, inside the inner environment of 
the above example, |\pgfuseimage{foo}| would first check whether
there exists an image named |foo.!50!white!25!black|.

\begin{command}{\colorcurrentmixin}
  Expands to the current accumulated mix-in. Each nesting of a
  |colormixin| adds a mix-in to this list.
\begin{codeexample}[]
\begin{minipage}{\linewidth-6pt}\raggedright
\begin{colormixin}{75!white}
  \colorcurrentmixin\ should be ``!75!white''\par
  \begin{colormixin}{75!black}
    \colorcurrentmixin\ should be ``!75!black!75!white''\par
    \begin{colormixin}{50!white}
      \colorcurrentmixin\ should be ``!50!white!75!black!75!white''\par
    \end{colormixin}
  \end{colormixin}
\end{colormixin}
\end{minipage}
\end{codeexample}
\end{command}





% Copyright 2008 by Till Tantau
%
% This file may be distributed and/or modified
%
% 1. under the LaTeX Project Public License and/or
% 2. under the GNU Free Documentation License.
%
% See the file doc/generic/pgf/licenses/LICENSE for more details.



\section{Parser Module}

\label{section-module-parser}

\begin{pgfmodule}{parser}
  This module defines some commands for creating a simple
  letter-by-letter parser.
\end{pgfmodule}

This module provides commands for defining a parser that scans some
given text letter-by-letter. For each letter, some code is executed
and, possible, a state-switch occurs. The parsing process ends when a
final state has been reached.

\begin{command}{\pgfparserparse\marg{parser name}\meta{text}}
  This command is used to parse the \meta{text} using the (previously
  defined) parser named \meta{parser name}.

  The \meta{text} is not contained in curly braces, rather it is all
  the text that follows. The end of the text is determined implicitly,
  namely when the final state of the parser has been reached.

  The parser works as follows: At any moment, it is in a certain
  \emph{state}, initially this state is called |initial|. Then, the
  first letter of the \meta{text} is examined (using the |\futurlet|
  command). For each possible state and each possible letter, some
  action code is stored in the parser in a table. This code is then
  executed. This code may, but need not, trigger a \emph{state
    switch}, causing a new state to be set. The parser then moves on
  to the next character of the text and repeats the whole
  procedure, unless it is in the state |final|, which causes the
  parsing process to stop immediately.

  In the following example, the parser counts the number of |a|'s
  in the \text{text}, ignoring any |b|'s. The \meta{text} ends with
  the first~|c|.
\begin{codeexample}[]
\newcount\mycount
\pgfparserdef{myparser}{initial}{the letter a}
{\advance\mycount by 1\relax}
\pgfparserdef{myparser}{initial}{the letter b}
{} % do nothing
\pgfparserdef{myparser}{initial}{the letter c}
{\pgfparserswitch{final}}% done!

\pgfparserparse{myparser}aabaabababbbbbabaabcccc
There are \the\mycount\ a's.
\end{codeexample}
\end{command}

\begin{command}{\pgfparserdef\marg{parser name}\marg{state}\marg{symbol meaning}\marg{action}}
  This command should be used repeatedly to define a parser named
  \meta{parser name}. With a call to this command you specify that the
  \meta{parser name} should do the following: When it is in state
  \meta{state} and reads the letter \meta{symbol meaning}, perform the
  code stored in \meta{action}.

  The \meta{symbol meaning} must be the text that results from
  applying the \TeX\ command |\meaning| to the given character. For
  instance, |\meaning a| yields |the letter a|, while |\meaning 1|
  yields |the character 1|. A space yields |blank space|.

  Inside the \meta{action} you can perform almost any kind of
  code. This code will not be surrounded by a scope, so its effect
  persist after the parsing is done. However, each time after the
  \meta{action} is executed, control goes back to the parser. You
  should not launch a parser inside the \meta{action} code, unless you
  put it in a scope.

  When you set the \meta{state} to |all|, the state \meta{action} is
  performed in all states as a fallback, whenever \meta{symbol
    meaning} is encountered. This means that when you do not specify
  anything explicitly for a state and a letter, but you do specify
  something for |all| and this letter, then the specified
  \meta{action} will be used.

  When the parser encounters a letter for which nothing is specified
  in the current state (neither directly nor indirectly via |all|), an
  error occurs.
\end{command}

\begin{command}{\pgfparserswitch\marg{state}}
  This command can be called inside the action code of a parser to
  cause a state switch to \meta{state}.
\end{command}



\part{Mathematical and Object-Oriented Engines}

{\Large \emph{by Mark Wibrow and Till Tantau}}


\bigskip
\noindent
\pgfname\ comes with two useful engines: One for doing mathematics,
one for doing object-oriented programming. Both engines can be used
independently of the main \pgfname.

The job of the mathematical
engine is to support mathematical operations like addition,
subtraction, multiplication and division, using both integers and
non-integers, but also functions such as square-roots, sine, cosine,
and generate pseudo-random numbers.
Mostly, you will use the mathematical facilities of \pgfname\
indirectly, namely when you write a coordinate like |(5cm*3,6cm/4)|,
but the mathematical engine can also be used independently of
\pgfname\ and \tikzname.

The job of the object-oriented engine is to support simple
object-oriented programming in \TeX. It allows the definition of
\emph{classes} (without inheritance), \emph{methods},
\emph{attributes} and \emph{objects}.

\vskip1cm
\begin{codeexample}[graphic=white]
\pgfmathsetseed{1}
\foreach \col in {black,red,green,blue}
{
  \begin{tikzpicture}[x=10pt,y=10pt,ultra thick,baseline,line cap=round]
    \coordinate (current point) at (0,0);
    \coordinate (old velocity) at (0,0);
    \coordinate (new velocity) at (rand,rand);

    \foreach \i in {0,1,...,100}
    {
      \draw[\col!\i] (current point)
      .. controls ++([scale=-1]old velocity) and
                  ++(new velocity) .. ++(rand,rand)
         coordinate (current point);
      \coordinate (old velocity) at (new velocity);
      \coordinate (new velocity) at (rand,rand);
    }
  \end{tikzpicture}
}
\end{codeexample}

% Copyright 2007 by Mark Wibrow and Till Tantau
%
% This file may be distributed and/or modified
%
% 1. under the LaTeX Project Public License and/or
% 2. under the GNU Free Documentation License.
%
% See the file doc/generic/pgf/licenses/LICENSE for more details.



\section{Design Principles}

\pgfname{} needs to perform many computations while typesetting a
picture. For this, \pgfname\ relies on a mathematical engine, which
can also be used independently of \pgfname, but which is distributed
as part of the \pgfname\ package nevertheless. Basically, the engine
provides a parsing mechanism similar to the \calcname{} package so
that expressions like |2*3cm+5cm| can be parsed; but the \pgfname\
engine is more powerful and can be extended and enhanced. 

\pgfname{} provides enhanced functionality, which permits the parsing
of mathematical operations involving integers and non-integers 
with or without units. Furthermore, various functions, including
trigonometric functions and random number generators can also be 
parsed (see Section~\ref{pgfmath-parsing}). 
The \calcname{} macros |\setlength| and friends have \pgfname{} versions 
which can parse these operations and functions 
(see Section~\ref{pgfmath-registers}). Additionally, each operation
and function has an independent \pgfname{} command associated with it
(see Section~\ref{pgfmath-commands}), and can be 
accessed outside the parser.

The mathematical engine of \pgfname\ is implicitly used whenever you
specify a number or dimension in a higher-level macro. For instance,
you can write |\pgfpoint{2cm+4cm/2}{3cm*sin(30)}| or
suchlike. However, the mathematical engine can also be used
independently of the \pgfname\ core, that is, you can also just load
it to get access to a mathematical parser.


\subsection{Loading the Mathematical Engine}

The mathematical engine of \pgfname\ is loaded automatically by
\pgfname, but if you wish to use the mathematical engine but you do
not need \pgfname\ itself, you can load the following package:

\begin{package}{pgfmath}
	This command will load the mathematical engine of \pgfname, but not 
	\pgfname{} itself. It defines commands like |\pgfmathparse|.
\end{package}


\subsection{Layers of the Mathematical Engine}

Like \pgfname\ itself, the mathematical engine is also structured into
different layers:

\begin{enumerate}
\item 
	The top layer, which you will typically use directly, provides
  the command |\pgfmathparse|. This command parses a mathematical
  expression and evaluates it.

  Additionally, the top layer also defines some additional functions
  similar to the macros of the |calc| package for setting dimensions
  and counters. These macros are just wrappers around the
  |\pgfmathparse| macro.
  
\item 
	The calculation layer provides macros for performing one
  specific computation like computing a reciprocal or a
  multiplication. The parser uses these macros for the actual
  computation.
  
\item 
	The implementation layer provides the actual implementations of
  the computations. These can be changed (and possibly be made more
  efficient) without affecting the higher layers.
\end{enumerate}



\subsection{Efficiency and Accuracy of the Mathematical Engine}

Currently, the mathematical algorithms are all implemented in \TeX.
This poses some intriguing programming challenges as \TeX{} is a
language for typesetting, rather than for general mathematics,
and as with any programming language, there is a trade-off between 
accuracy and efficiency. 
If you find the level of accuracy insufficient for your
purposes, you will have to replace the algorithms in the
implementation layer.

All the fancy mathematical ``bells-and-whistles'' that the parser 
provides, come with an additional processing cost, and in some
instances, such as simply setting a length to |1cm|, with no other
operations involved, the additional processing time is undesirable. 
To overcome this, the following feature is implemented: when no
mathematical operations are required, an expression
can be preceded by |+|. This will bypass the parsing process and the 
assignment will be orders of magnitude faster. This feature 
\emph{only} works with the macros for setting registers described in
Section~\ref{pgfmath-registers}.

\begin{codeexample}[code only]
\pgfmathsetlength\mydimen{1cm}  % parsed     : slower.
\pgfmathsetlength\mydimen{+1cm} % not parsed : much faster.
\end{codeexample}


% Copyright 2007 by Mark Wibrow
%
% This file may be distributed and/or modified
%
% 1. under the LaTeX Project Public License and/or
% 2. under the GNU Free Documentation License.
%
% See the file doc/generic/pgf/licenses/LICENSE for more details.
%

\section{Mathematical Expressions}

\label{pgfmath-syntax}

The easiest way of using \pgfname's mathematical engine is to provide
a mathematical expression given in familiar infix notation, for
example, |1cm+4*2cm/5.5| or |2*3+3*sin(30)|. This expression can be
parsed by the mathematical engine and the result can be placed in a
dimension register, a counter, or a macro.

It should be noted that all
calculations must not exceed $\pm16383.99999$ at \emph{any} point,
because the underlying computations rely on \TeX{} dimensions. This
means that many of the underlying computations are necessarily
approximate and, in addition, not very fast. \TeX{} is,
after all, a typesetting language and not ideally
suited to relatively advanced mathematical operations. However, it
is possible to change the computations as described in
Section~\ref{pgfmath-reimplement}.

In the present section, the high-level macros for parsing an
expression are explained first, then the syntax for expression is
explained.


\subsection{Parsing Expressions}

\label{pgfmath-registers}

\label{pgfmath-parsing}

\subsubsection{Commands}

The basic command for invoking the parser of \pgfname's mathematical
engine is the following:

\begin{command}{\pgfmathparse\marg{expression}}
  This macro parses \meta{expression} and returns the result without
  units in  the macro |\pgfmathresult|.

  \example |\pgfmathparse{2pt+3.5pt}| will set |\pgfmathresult| to the
  text |5.5|.

  In the following, the special properties of this command are
  explained. The exact syntax of mathematical expressions is explained
  in Sections \ref{pgfmath-operators} and~\ref{pgfmath-functions}.

  \begin{itemize}
  \item
    The result stored in the macro |\pgfmathresult| is a decimal
    \emph{without units}. This is true regardless of whether the
    \meta{expression} contains any unit specification. All numbers
    with units are converted to points first. See
    Section~\ref{pgfmath-units} for details on units.
  \item
    The parser will recognize \TeX{} registers and box dimensions,
    so |\mydimen|, |0.5\mydimen|, |\wd\mybox|, |0.5\dp\mybox|,
    |\mycount\mydimen| and so on can be parsed.
    
  \item
    The $\varepsilon$-TeX\ extensions |\dimexpr|, |\numexpr|, |\glueexpr|, and
    |\muexpr| are recognized and evaluated. The values they
    result in will be used in the further evaluation, as if you had
    put |\the| before them. 
    
  \item
    Parenthesis can be used to change the order of the evaluation.
    
  \item
    Various functions are recognized, so it is possible to parse
    |sin(.5*pi r)*60|, which means ``the sine of $0.5$ times $\pi$
    radians, multiplied by 60''. The argument of functions can
    be any expression.
    
  \item
    Scientific notation in the form |1.234e+4| is recognized (but
    the restriction on the range of values still applies). The exponent
    symbol can be upper or lower case (i.e., |E| or |e|).
    
  \item
    An integer with a zero-prefix (excluding, of course zero itself),
    is interpreted as an octal number and is automatically converted
    to base 10.
    
  \item
    An integer with prefix |0x| or |0X| is interpreted as a hexadecimal
    number and is automatically converted to base 10. Alphabetic digits
    can be in uppercase or lowercase.
    
  \item
    An integer with prefix |0b| or |0B| is interpreted as a binary
    number and is automatically converted to base 10.
    
  \item
    An expression (or part of an expression) surrounded with double
    quotes (i.e., the character |"|) will not be evaluated.
    Obviously this should be used with great care.
    
  \end{itemize}

\end{command}



\begin{command}{\pgfmathqparse\marg{expression}}
  This macro is similar to |\pgfmathparse|: it parses
  \meta{expression} and returns the result in the macro
  |\pgfmathresult|. It differs in two respects. Firstly,
  |\pgfmathqparse| does not parse functions, scientific
  notation, the prefixes for binary octal, or hexadecimal numbers,
  nor does it accept the special use of |"|, |?| or |:| characters.
  Secondly, numbers in \meta{expression} \emph{must}
  specify a \TeX{} unit (except in such instances as |0.5\pgf@x|),
  which greatly simplifies the problem of parsing real numbers.
  As a result of these restrictions |\pgfmathqparse|
  is about twice as fast as |\pgfmathparse|. Note that the result
  will still be a number without units.	
\end{command}

\begin{command}{\pgfmathpostparse}

  At the end of the parse this command is executed, allowing some
  custom action to be performed on the result of the parse. When this
  command is executed, the macro |\pgfmathresult| will hold the result
  of the parse (as always, without units). The result of the custom
  action should be used to redefine |\pgfmathresult| appropriately.
  By default, this command is equivalent to |\relax|. This differs
  from previous versions, where, if the parsed expression contained
  no units, the result of the parse was scaled according to the value
  in |\pgfmathresultunitscale| (which by default was |1|).

  This scaling can be  turned on again using:
  |\let\pgfmathpostparse=\pgfmathscaleresult|.
  Note, however that by scaling the result, the base conversion
  functions will not work, and the |"| character should not be
  used to quote parts of an expression.

\end{command}

Instead of the |\pgfmathparse| macro you can also use wrapper commands,
whose usage is very similar to their cousins in the \calcname{}
package. The only difference is that the expressions can be any
expression that is handled by |\pgfmathparse|.
For all of the following commands, if \meta{expression} starts with
|+|, no parsing is done and a simple assignment or increment is done
using normal \TeX\ assignments or increments. This will be orders of
magnitude faster than calling the parser.

The effect of the following commands is always local to the current
\TeX\ scope.

\begin{command}{\pgfmathsetlength\marg{register}\marg{expression}}
  Basically, this command sets the length of the \TeX{}
  \meta{register} to the value specified by
  \meta{expression}. However, there is some fine print:

  First, in case \meta{expression} starts with a |+|, a simple \TeX\
  assignment is done. In particular, \meta{register} can be a glue
  register and \meta{expression} be something like |+1pt plus 1fil|
  and the \meta{register} will be assigned the expected value.

  Second, when the \meta{expression} does not start with |+|, it is
  first parsed using |\pgfmathparse|, resulting in a (dimensionless)
  value |\pgfmathresult|. Now, if the parser encountered the unit |mu|
  somewhere in the expression, it assumes that \meta{register} is a
  |\muskip| register and will try to assign to \meta{register} the
  value |\pgfmathresult| followed by |mu|. Otherwise, in case |mu| was
  not encountered, it is assumed that \meta{register} is a dimension
  register or a glue register and we assign |\pgfmathresult| followed
  by |pt| to it.

  The net effect of the above is that you can write things like
\begin{codeexample}[]
  \muskipdef\mymuskip=0
  \pgfmathsetlength{\mymuskip}{1mu+3*4mu} \the\mymuskip 
\end{codeexample}
\begin{codeexample}[]  
  \dimendef\mydimen=0  
  \pgfmathsetlength{\mydimen}{1pt+3*4pt}  \the\mydimen
\end{codeexample}
\begin{codeexample}[]  
  \skipdef\myskip=0  
  \pgfmathsetlength{\myskip}{1pt+3*4pt}  \the\myskip
\end{codeexample}

  One thing that will \emph{not} work is
  |\pgfmathsetlength{\myskip}{1pt plus 1fil}| since the parser does
  not support fill's. You can, however, use the |+| notation in this
  case: 
\begin{codeexample}[]  
  \skipdef\myskip=0  
  \pgfmathsetlength{\myskip}{+1pt plus 1fil}  \the\myskip
\end{codeexample}
\end{command}

\begin{command}{\pgfmathaddtolength\marg{register}\marg{expression}}
  Adds the value of \meta{expression} to the \TeX{}
  \meta{register}. All of the special consideration mentioned for
  |\pgfmathsetlength| also apply here in the same way.
\end{command}

\begin{command}{\pgfmathsetcount\marg{count register}\marg{expression}}
  Sets the value of the \TeX{} \meta{count register}, to the
  \emph{truncated} value specified by \meta{expression}.
\end{command}

\begin{command}{\pgfmathaddtocount\marg{count register}\marg{expression}}
  Adds the \emph{truncated} value  of \meta{expression} to the \TeX{}
  \meta{count register}.
\end{command}

\begin{command}{\pgfmathsetcounter\marg{counter}\marg{expression}}
  Sets the value of the \LaTeX{} \meta{counter} to the \emph{truncated}
  value specified by \meta{expression}.
\end{command}

\begin{command}{\pgfmathaddtocounter\marg{counter}\marg{expression}}
  Adds the \emph{truncated} value  of \meta{expression} to
  \meta{counter}.
\end{command}

\begin{command}{\pgfmathsetmacro\marg{macro}\marg{expression}}
  Defines \meta{macro} as the  value of \meta{expression}. The result
  is a decimal without units.
\end{command}

\begin{command}{\pgfmathsetlengthmacro\marg{macro}\marg{expression}}
  Defines \meta{macro} as the value of \meta{expression}
  \LaTeX{} \emph{in points}.
\end{command}

\begin{command}{\pgfmathtruncatemacro\marg{macro}\marg{expression}}
  Defines \meta{macro} as the truncated value of \meta{expression}.
\end{command}


\subsubsection{Considerations Concerning Units}
\label{pgfmath-units}

As was explained earlier, the parser commands like |\pgfmathparse|
will always return a result without units in it and all dimensions
that have a unit like |10pt| or |1in| will first be converted to \TeX\
points (|pt|) and, then, the unit is dropped.

Sometimes it is useful, nevertheless, to find out whether an
expression or not. For this, you can use the following commands:

{\let\ifpgfmathunitsdeclared\relax
  \begin{command}{\ifpgfmathunitsdeclared}
    After a call  of |\pgfmathparse| this if will be true exactly if
    some unit was encountered in the expression. It is always set
    globally in each call.
    
    Note that \emph{any} ``mentioning'' of a unit inside an
    expression will set this \TeX-if to true. In particular, even an
    expressionlike |2pt/1pt|, which arguably should be considered
    ``scalar'' or ``unit-free'' will still have this \TeX-if set to
    true. However, see the |scalar| function for a way to change
    this. 
  \end{command}
}

\begin{math-function}{scalar(\mvar{value})}
  \mathcommand
  
  This function is the identity function on its input, but it will
  reset the \TeX-if |\ifpgfmathunitsdeclared|. Thus, it can be used to
  indicate that the given \meta{value} should be considered as a
  ``scalar'' even when it contains units; but note that it will work
  even when the \meta{value} is a string or something else. The only
  effect of this function is to clear the unit declaration.

\begin{codeexample}[]
\pgfmathparse{scalar(1pt/2pt)} \pgfmathresult\
\ifpgfmathunitsdeclared with \else without \fi unit
\end{codeexample}

  Note, however, that this command (currently) really just clears the
  \TeX-if as the input is scanned from left-to-right. Thus, even if
  there is a use of a unit before the |scalar| function is used, the
  \TeX-if will be cleared:

\begin{codeexample}[]
\pgfmathparse{1pt+scalar(1pt)} \pgfmathresult\
\ifpgfmathunitsdeclared with \else without \fi unit
\end{codeexample}

  The other way round, a use of a unit after the |scalar| function
  will set the units once more.
\begin{codeexample}[]
\pgfmathparse{scalar(1pt)+1pt} \pgfmathresult\
\ifpgfmathunitsdeclared with \else without \fi unit
\end{codeexample}

  For these reasons, you should use the function only on the outermost
  level of an expression.
  
  A typical use of this function is the following:
\begin{codeexample}[]  
\tikz{
  \coordinate["$A$"]       (A) at (2,2);%$
  \coordinate["$B$" below] (B) at (0,0);
  \coordinate["$C$" below] (C) at (3,0);
  \draw (A) -- (B) -- (C) -- cycle;
  \path
    let \p1 =($(A)-(B)$), \p2 =($(A)-(C)$),
        \n1 = {veclen(\x1,\y1)}, \n2 = {veclen(\x2,\y2)}
    in  coordinate ["$D$" below] (D) at ($ (B)!scalar(\n1/(\n1+\n2))!(C) $);
  \draw (A) -- (D);
}
\end{codeexample}
\end{math-function}

A special kind of units are \TeX's ``math units'' (|mu|). It will
be treated as if |pt| had been used, but you can
check whether an expression contained a math unit using the
following: 
{\let\ifpgfmathmathunitsdeclared\relax
  \begin{command}{\ifpgfmathmathunitsdeclared}
    This \TeX-if is similar to |\ifpgfmathunitsdeclared|, but it
    is only set when the unit |mu| is encountered at least
    once. In this case, |\ifpgfmathunitsdeclared| will \emph{also}
    be set to true. The |scalar| function has no effect on this \TeX-if.
  \end{command}
}
  
\subsection{Syntax for Mathematical Expressions: Operators}

The syntax for the expressions recognized by |\pgfmathparse| and
friends is rather straightforward. Let us start with the operators.

\label{pgfmath-operators}

The following operators (presented in the context in which they are used)
are recognized:

\begin{math-operator}{+}{infix}{add}
 Adds \mvar{x} to \mvar{y}.
\end{math-operator}

\begin{math-operator}{-}{infix}{subtract}
  Subtracts \mvar{y} from \mvar{x}.
\end{math-operator}

\begin{math-operator}{-}{prefix}{neg}
  Reverses the sign of \mvar{x}.
\end{math-operator}

\begin{math-operator}{*}{infix}{multiply}
  Multiples \mvar{x} by \mvar{y}.
\end{math-operator}

\begin{math-operator}{/}{infix}{divide}
  Divides \mvar{x} by \mvar{y}. An error will result if \mvar{y} is 0,
  or if the result of the division is too big for the mathematical
  engine. Please remember when using this command that accurate (and
  reasonably quick) division of real numbers that are not integers
  is particularly tricky in \TeX.
\end{math-operator}

\begin{math-operator}{\char`\^}{infix}{pow}
  Raises \mvar{x} to the power \mvar{y}.
\end{math-operator}

\begin{math-operator}{\protect\exclamationmarktext}{postfix}{factorial}
  Calculates the factorial of \mvar{x}.
\end{math-operator}

\begin{math-operator}{r}{postfix}{deg}
  Converts \mvar{x} to degrees (\mvar{x} is assumed to be in radians).
  This operator has the same precedence as multiplication.
\end{math-operator}

\begin{math-operators}{?}{:}{conditional}{ifthenelse}

  |?| and |:| are special operators which can be used as a shorthand
  for |if| \mvar{x} |then| \mvar{y} |else| \mvar{z} inside the parser.
  The expression \mvar{x} is taken to be true if it evaluates to any
  non-zero value.

\end{math-operators}

\begin{math-operator}{==}{infix}{equal}
  Returns |1| if \mvar{x}$=$\mvar{y}, |0| otherwise.
\end{math-operator}

\begin{math-operator}{>}{infix}{greater}
  Returns |1| if \mvar{x}$>$\mvar{y}, |0| otherwise.
\end{math-operator}

\begin{math-operator}{<}{infix}{less}
  Returns |1| if \mvar{x}$<$\mvar{y}, |0| otherwise.
\end{math-operator}

\begin{math-operator}{\protect\exclamationmarktext=}{infix}{notequal}
  Returns |1| if \mvar{x}$\neq$\mvar{y}, |0| otherwise.
\end{math-operator}

\begin{math-operator}{>=}{infix}{notless}
  Returns |1| if \mvar{x}$\geq$\mvar{y}, |0| otherwise.
\end{math-operator}

\begin{math-operator}{<=}{infix}{notgreater}
  Returns |1| if \mvar{x}$\leq$\mvar{y}, |0| otherwise.
\end{math-operator}

\begin{math-operator}{{\char`\&}{\char`\&}}{infix}{and}
  Returns |1| if both \mvar{x} and \mvar{y} evaluate to some
  non-zero value. Both arguments are evaluated.
\end{math-operator}



{
 \catcode`\|=12
\begin{math-operator}[no index]{||}{infix}{or}
	\index{*pgfmanualvbarvbarr@\protect\texttt{\protect\pgfmanualvbarvbar} math operator}%
  \index{Math operators!*pgfmanualvbarvbar@\protect\texttt{\protect\pgfmanualvbarvbar}}%
  Returns {\tt 1} if either \mvar{x} or \mvar{y} evaluate to some
  non-zero value.
\end{math-operator}
}

\begin{math-operator}{\protect\exclamationmarktext}{prefix}{not}
  Returns |1| if \mvar{x} evaluates to zero, |0| otherwise.
\end{math-operator}


\begin{math-operators}{(}{)}{group}{}

These operators act in the usual way, that is, to control the order
in which operators are executed, for example, |(1+2)*3|. This
includes the grouping of arguments for functions, for example,
|sin(30*10)| or |mod(72,3)| (the comma character is also treated
as an operator).

Parentheses for functions with one argument are not always
necessary, |sin 30| (note the space) is the same as |sin(30)|.
However, functions have the highest precedence so, |sin 30*10|
is the same as |sin(30)*10|.

\end{math-operators}


\begin{math-operators}{\char`\{}{\char`\}}{array}{}

These operators are used to process array-like structures (within an
expression these characters do not act like \TeX{} grouping tokens).
The \meta{array specification} consists of comma separated elements,
for example, |{1, 2, 3, 4, 5}|. Each element in the array will be
evaluated as it is parsed, so expressions can be used.
In addition, an element of an array can be an array itself,
allowing multiple dimension arrays to be simulated:
|{1, {2,3}, {4,5}, 6}|.
When storing an array in a macro, do not forget the surrounding
braces: |\def\myarray{{1,2,3}}| not |\def\myarray{1,2,3}|.

\begin{codeexample}[]
\def\myarray{{1,"two",2+1,"IV","cinq","sechs",sin(\i*5)*14}}
\foreach \i in  {0,...,6}{\pgfmathparse{\myarray[\i]}\pgfmathresult, }
\end{codeexample}

\end{math-operators}

\begin{math-operators}{\char`\[}{\char`\]}{array access}{array}

|[| and |]| are two operators used in one particular circumstance: to
access an array (specified using the |{| and |}| operators)
using the index \mvar{x}. Indexing starts from zero,
so, if the index is greater than, or equal to, the number of values in
the array, an error will occur, and zero will be returned.

\begin{codeexample}[]
\def\myarray{{7,-3,4,-9,11}}
\pgfmathparse{\myarray[3]} \pgfmathresult
\end{codeexample}

If the array is defined to have multiple dimensions, then the array
access operators can be immediately repeated.

\begin{codeexample}[]
\def\print#1{\pgfmathparse{#1}\pgfmathresult}
\def\identitymatrix{{{1,0,0},{0,1,0},{0,0,1}}}
\tikz[x=0.5cm,y=0.5cm]\foreach \i in {0,1,2} \foreach \j in {0,1,2}
  \node at (\j,-\i) [anchor=base] {\print{\identitymatrix[\i][\j]}};
\end{codeexample}

\end{math-operators}


\begin{math-operators}{\char`\"}{\char`\"}{group}{}

These operators are used to quote \mvar{x}. However, as every
expression is expanded with |\edef| before it is parsed, macros
(e.g., font commands like |\tt| or |\Huge|) may need to be
``protected'' from this expansion (e.g., |\noexpand\Huge|). Ideally,
you should avoid such macros anyway.
Obviously, these operators should be used with great care as further
calculations are unlikely to be possible with the result.

\begin{codeexample}[]
\def\x{5}
\foreach \y in {0,10}{
  \pgfmathparse{\x > \y ? "\noexpand\Large Bigger" : "\noexpand\tiny smaller"}
  \x\ is \pgfmathresult\ than \y.
}
\end{codeexample}

\end{math-operators}




\subsection{Syntax for Mathematical Expressions: Functions}

\label{pgfmath-functions}

The following functions are recognized:

\medskip
\def\mathlink#1{\hyperlink{math:#1}{\tt#1}}
\begin{tikzpicture}
\foreach \f [count=\i from 0] in
{abs,acos,add,and,array,asin,atan,atan2,bin,ceil,cos,
 cosec,cosh,cot,deg,depth,div,divide,e,equal,factorial, false,
 floor,frac,gcd,greater,height,hex,Hex,int,ifthenelse,iseven,isodd,isprime,
 less,ln,log10,log2,max,min,mod,Mod,multiply,
 neg,not,notequal,notgreater,notless,
 oct,or,pi,pow,rad,rand,random,real,rnd,round,
 scalar,sec,sin,sinh,sqrt,subtract,tan,tanh,true, veclen,width}
\node [anchor=base west] at ({int(\i/12)*2.5cm},{-mod(\i,12)*1.1*\baselineskip}) {\mathlink{\f}};
\end{tikzpicture}
\bigskip

Each function has a \pgfname{} command associated with it (which is
also shown with the function below). In general, the command
is simply the name of the function prefixed with |\pgfmath|, for
example, |\pgfmathadd|, but there are some notable exceptions.

\subsubsection{Basic arithmetic functions}

\label{pgfmath-functions-basic}

\begin{math-function}{add(\mvar{x},\mvar{y})}
\mathcommand

  Adds $x$ and $y$.

\begin{codeexample}[]
\pgfmathparse{add(75,6)} \pgfmathresult
\end{codeexample}
\end{math-function}

\begin{math-function}{subtract(\mvar{x},\mvar{y})}
\mathcommand

  Subtract $x$ from $y$.

\begin{codeexample}[]
\pgfmathparse{subtract(75,6)} \pgfmathresult
\end{codeexample}
\end{math-function}

\begin{math-function}{neg(\mvar{x})}
\mathcommand

	This returns $-\mvar{x}$.
	
\begin{codeexample}[]
\pgfmathparse{neg(50)} \pgfmathresult
\end{codeexample}

\end{math-function}

\begin{math-function}{multiply(\mvar{x},\mvar{y})}
\mathcommand

  Multiply $x$ by $y$.

\begin{codeexample}[]
\pgfmathparse{multiply(75,6)} \pgfmathresult
\end{codeexample}
\end{math-function}

\begin{math-function}{divide(\mvar{x},\mvar{y})}
\mathcommand

  Divide $x$ by $y$.

\begin{codeexample}[]
\pgfmathparse{divide(75,6)} \pgfmathresult
\end{codeexample}
\end{math-function}

\begin{math-function}{div(\mvar{x},\mvar{y})}
\mathcommand

  Divide $x$ by $y$ and round to the nearest integer

\begin{codeexample}[]
\pgfmathparse{div(75,9)} \pgfmathresult
\end{codeexample}
\end{math-function}

\begin{math-function}{factorial(\mvar{x})}
\mathcommand

  Return \mvar{x}!.

\begin{codeexample}[]
\pgfmathparse{factorial(5)} \pgfmathresult
\end{codeexample}

\end{math-function}

\begin{math-function}{sqrt(\mvar{x})}
\mathcommand

 Calculates $\sqrt{\textrm{\mvar{x}}}$.

\begin{codeexample}[]
\pgfmathparse{sqrt(10)} \pgfmathresult
\end{codeexample}

\begin{codeexample}[]
\pgfmathparse{sqrt(8765.432)}  \pgfmathresult
\end{codeexample}

\end{math-function}

\begin{math-function}{pow(\mvar{x},\mvar{y})}
\mathcommand

 Raises \mvar{x} to the power \mvar{y}. For greatest accuracy,
 \mvar{y} should be an integer. If \mvar{y} is not an integer,
 the actual calculation will be an approximation of $e^{y\text{ln}(x)}$.

\begin{codeexample}[]
\pgfmathparse{pow(2,7)} \pgfmathresult
\end{codeexample}

\end{math-function}


\begin{math-function}{e}
\mathcommand

  Returns the value 2.718281828.
{
\catcode`\^=7

\begin{codeexample}[]
\pgfmathparse{(e^2-e^-2)/2} \pgfmathresult
\end{codeexample}

}
\end{math-function}

\begin{math-function}{exp(\mvar{x})}
\mathcommand

{
\catcode`\^=7

	Maclaurin series for $e^x$.
}	
\begin{codeexample}[]
\pgfmathparse{exp(1)} \pgfmathresult
\end{codeexample}

\begin{codeexample}[]
\pgfmathparse{exp(2.34)} \pgfmathresult
\end{codeexample}

\end{math-function}


\begin{math-function}{ln(\mvar{x})}
\mathcommand

{
\catcode`\^=7

	An approximation for $\ln(\textrm{\mvar{x}})$.
	This uses an algorithm of Rouben Rostamian, and coefficients
	suggested by Alain Matthes.
}	
\begin{codeexample}[]
\pgfmathparse{ln(10)} \pgfmathresult
\end{codeexample}

\begin{codeexample}[]
\pgfmathparse{ln(exp(5))} \pgfmathresult
\end{codeexample}

\end{math-function}

\begin{math-function}{log10(\mvar{x})}
\mathcommand[logten(\mvar{x})]

	An approximation for $\log_{10}(\textrm{\mvar{x}})$.

\begin{codeexample}[]
\pgfmathparse{log10(100)} \pgfmathresult
\end{codeexample}

\end{math-function}

\begin{math-function}{log2(\mvar{x})}
\mathcommand[logtwo(\mvar{x})]

	An approximation for $\log_2(\textrm{\mvar{x}})$.

\begin{codeexample}[]
\pgfmathparse{log2(128)} \pgfmathresult
\end{codeexample}

\end{math-function}

\begin{math-function}{abs(\mvar{x})}
\mathcommand

	Evaluates the absolute value of $x$.
	
\begin{codeexample}[]
\pgfmathparse{abs(-5)} \pgfmathresult
\end{codeexample}

\begin{codeexample}[]
\pgfmathparse{-abs(4*-3)} \pgfmathresult
\end{codeexample}
\end{math-function}

\begin{math-function}{mod(\mvar{x},\mvar{y})}
\mathcommand

	This evaluates \mvar{x} modulo \mvar{y}, using truncated division.
	The sign of the result is the same as the sign of
	$\frac{\textrm{\mvar{x}}}{\textrm{\mvar{y}}}$.

\begin{codeexample}[]
\pgfmathparse{mod(20,6)} \pgfmathresult
\end{codeexample}

\begin{codeexample}[]
\pgfmathparse{mod(-100,30)} \pgfmathresult
\end{codeexample}

\end{math-function}

\begin{math-function}{Mod(\mvar{x},\mvar{y})}
\mathcommand

	This evaluates \mvar{x} modulo \mvar{y}, using floored division.
	The sign of the result is never negative.

\begin{codeexample}[]
\pgfmathparse{Mod(-100,30)} \pgfmathresult
\end{codeexample}

\end{math-function}





\subsubsection{Rounding functions}

\label{pgfmath-functions-rounding}

\begin{math-function}{round(\mvar{x})}
\mathcommand

	Rounds \mvar{x} to the nearest integer. It uses ``asymmetric half-up''
	rounding. So |1.5| is rounded to |2|, but |-1.5| is rounded to |-2|
	(\emph{not} |-1|).

\begin{codeexample}[]
\pgfmathparse{round(32.5/17)} \pgfmathresult
\end{codeexample}

\begin{codeexample}[]
\pgfmathparse{round(398/12)} \pgfmathresult
\end{codeexample}

\end{math-function}

\begin{math-function}{floor(\mvar{x})}
\mathcommand

	Rounds \mvar{x} down to the nearest integer.
	
\begin{codeexample}[]
\pgfmathparse{floor(32.5/17)} \pgfmathresult
\end{codeexample}

\begin{codeexample}[]
\pgfmathparse{floor(398/12)} \pgfmathresult
\end{codeexample}

\begin{codeexample}[]
\pgfmathparse{floor(-398/12)} \pgfmathresult
\end{codeexample}
\end{math-function}

\begin{math-function}{ceil(\mvar{x})}
\mathcommand

	Rounds \mvar{x} up to the nearest integer.

\begin{codeexample}[]
\pgfmathparse{ceil(32.5/17)} \pgfmathresult
\end{codeexample}

\begin{codeexample}[]
\pgfmathparse{ceil(398/12)} \pgfmathresult
\end{codeexample}

\begin{codeexample}[]
\pgfmathparse{ceil(-398/12)} \pgfmathresult
\end{codeexample}

\end{math-function}

\begin{math-function}{int(\mvar{x})}
\mathcommand

	Returns the integer part of \mvar{x}.

\begin{codeexample}[]
\pgfmathparse{int(32.5/17)} \pgfmathresult
\end{codeexample}

\end{math-function}

\begin{math-function}{frac(\mvar{x})}
\mathcommand

	Returns the fractional part of \mvar{x}.

\begin{codeexample}[]
\pgfmathparse{frac(32.5/17)} \pgfmathresult
\end{codeexample}

\end{math-function}

\begin{math-function}{real(\mvar{x})}
\mathcommand

	Ensures \mvar{x} contains a decimal point.

\begin{codeexample}[]
\pgfmathparse{real(4)} \pgfmathresult
\end{codeexample}

\end{math-function}


\subsubsection{Integer arithmetics functions}

\label{pgfmath-functions-integerarithmetics}

\begin{math-function}{gcd(\mvar{x},\mvar{y})}
\mathcommand

  Computes the greatest common divider of \mvar{x} and \mvar{y}. 

\begin{codeexample}[]
\pgfmathparse{gcd(42,56)} \pgfmathresult
\end{codeexample}

\end{math-function}

\begin{math-function}{isodd(\mvar{x})}
\mathcommand

  Returns |1| if the integer part of \mvar{x} is odd. Otherwise, returns |0|.

\begin{codeexample}[]
\pgfmathparse{isodd(2)} \pgfmathresult, 
\pgfmathparse{isodd(3)} \pgfmathresult
\end{codeexample}

\end{math-function}

\begin{math-function}{iseven(\mvar{x})}
\mathcommand

  Returns |1| if the integer part of \mvar{x} is even. Otherwise, returns |0|.

\begin{codeexample}[]
\pgfmathparse{iseven(2)} \pgfmathresult, 
\pgfmathparse{iseven(3)} \pgfmathresult
\end{codeexample}

\end{math-function}

\begin{math-function}{isprime(\mvar{x})}
\mathcommand

  Returns |1| if the integer part of \mvar{x} is prime. Otherwise, returns |0|.

\begin{codeexample}[]
\pgfmathparse{isprime(1)} \pgfmathresult, 
\pgfmathparse{isprime(2)} \pgfmathresult,
\pgfmathparse{isprime(31)} \pgfmathresult,
\pgfmathparse{isprime(64)} \pgfmathresult
\end{codeexample}

\end{math-function}


\subsubsection{Trigonometric functions}

\label{pgfmath-functions-trigonometric}

\begin{math-function}{pi}
\mathcommand

	Returns the value $\pi=3.141592654$.
	
\begin{codeexample}[]
\pgfmathparse{pi} \pgfmathresult
\end{codeexample}

\begin{codeexample}[]
\pgfmathparse{pi r} \pgfmathresult
\end{codeexample}

\end{math-function}

\begin{math-function}{rad(\mvar{x})}
\mathcommand

	Convert \mvar{x} to radians. \mvar{x} is assumed to be in degrees.
	
\begin{codeexample}[]
\pgfmathparse{rad(90)} \pgfmathresult
\end{codeexample}

\end{math-function}

\begin{math-function}{deg(\mvar{x})}
\mathcommand

	Convert \mvar{x} to degrees. \mvar{x} is assumed to be in radians.
	
\begin{codeexample}[]
\pgfmathparse{deg(3*pi/2)} \pgfmathresult
\end{codeexample}

\end{math-function}

\begin{math-function}{sin(\mvar{x})}
\mathcommand

	Sine of \mvar{x}. By employing the |r| operator, \mvar{x} can be in
	radians.
	
\begin{codeexample}[]
\pgfmathparse{sin(60)} \pgfmathresult
\end{codeexample}

\begin{codeexample}[]
\pgfmathparse{sin(pi/3 r)} \pgfmathresult
\end{codeexample}

\end{math-function}

\begin{math-function}{cos(\mvar{x})}
\mathcommand

	Cosine of \mvar{x}. By employing the |r| operator, \mvar{x} can be in
	radians.

\begin{codeexample}[]
\pgfmathparse{cos(60)} \pgfmathresult
\end{codeexample}

\begin{codeexample}[]
\pgfmathparse{cos(pi/3 r)} \pgfmathresult
\end{codeexample}

\end{math-function}

\begin{math-function}{tan(\mvar{x})}
\mathcommand

	Tangent of \mvar{x}. By employing the |r| operator, \mvar{x} can be in
	radians.
	
\begin{codeexample}[]
\pgfmathparse{tan(45)} \pgfmathresult
\end{codeexample}

\begin{codeexample}[]
\pgfmathparse{tan(2*pi/8 r)} \pgfmathresult
\end{codeexample}

\end{math-function}


\begin{math-function}{sec(\mvar{x})}
\mathcommand

	Secant of \mvar{x}. By employing the |r| operator, \mvar{x} can be in
	radians.

\begin{codeexample}[]
\pgfmathparse{sec(45)} \pgfmathresult
\end{codeexample}

\end{math-function}

\begin{math-function}{cosec(\mvar{x})}
\mathcommand

	Cosecant of \mvar{x}. By employing the |r| operator, \mvar{x} can be in
	radians.
	
\begin{codeexample}[]
\pgfmathparse{cosec(30)} \pgfmathresult
\end{codeexample}

\end{math-function}

\begin{math-function}{cot(\mvar{x})}
\mathcommand

	Cotangent of \mvar{x}. By employing the |r| operator, \mvar{x} can be in
	radians.
	
\begin{codeexample}[]
\pgfmathparse{cot(15)} \pgfmathresult
\end{codeexample}

\end{math-function}

\begin{math-function}{asin(\mvar{x})}
\mathcommand

	Arcsine of \mvar{x}. The result is in degrees and in the range $\pm90^\circ$.

\begin{codeexample}[]
\pgfmathparse{asin(0.7071)} \pgfmathresult
\end{codeexample}

\end{math-function}

\begin{math-function}{acos(\mvar{x})}
\mathcommand

	Arccosine of \mvar{x} in degrees. The result is in the range $[0^\circ,180^\circ]$.

\begin{codeexample}[]
\pgfmathparse{acos(0.5)} \pgfmathresult
\end{codeexample}

\end{math-function}

\begin{math-function}{atan(\mvar{x})}
\mathcommand

	Arctangent of $x$ in degrees.

\begin{codeexample}[]
\pgfmathparse{atan(1)} \pgfmathresult
\end{codeexample}

\end{math-function}

\begin{math-function}{atan2(\mvar{y},\mvar{x})}
\mathcommand[atantwo(\mvar{y},\mvar{x})]

	Arctangent of $y\div x$ in degrees. This also takes into account the
	quadrants.

\begin{codeexample}[]
\pgfmathparse{atan2(-4,3)} \pgfmathresult
\end{codeexample}

\end{math-function}

\subsubsection{Comparison and logical functions}

\label{pgfmath-functions-comparison}

\begin{math-function}{equal(\mvar{x},\mvar{y})}
\mathcommand

	This returns |1| if $\mvar{x}=\mvar{y}$ and |0| otherwise.
	
\begin{codeexample}[]
\pgfmathparse{equal(20,20)} \pgfmathresult
\end{codeexample}

\end{math-function}

\begin{math-function}{greater(\mvar{x},\mvar{y})}
\mathcommand

	This returns |1| if $\mvar{x}>\mvar{y}$ and |0| otherwise.
	
\begin{codeexample}[]
\pgfmathparse{greater(20,25)} \pgfmathresult
\end{codeexample}

\end{math-function}

\begin{math-function}{less(\mvar{x},\mvar{y})}
\mathcommand

	This returns |1| if $\mvar{x}<\mvar{y}$ and |0| otherwise.
	
\begin{codeexample}[]
\pgfmathparse{greater(20,25)} \pgfmathresult
\end{codeexample}
\end{math-function}

\begin{math-function}{notequal(\mvar{x},\mvar{y})}
\mathcommand

	This returns |0| if $\mvar{x}=\mvar{y}$ and |1| otherwise.
	
\begin{codeexample}[]
\pgfmathparse{notequal(20,25)} \pgfmathresult
\end{codeexample}

\end{math-function}

\begin{math-function}{notgreater(\mvar{x},\mvar{y})}
\mathcommand

	This returns |1| if $\mvar{x}\leq\mvar{y}$ and |0| otherwise.
	
\begin{codeexample}[]
\pgfmathparse{notgreater(20,25)} \pgfmathresult
\end{codeexample}
\end{math-function}

\begin{math-function}{notless(\mvar{x},\mvar{y})}
\mathcommand

	This returns |1| if $\mvar{x}\geq\mvar{y}$ and |0| otherwise.
	
\begin{codeexample}[]
\pgfmathparse{notless(20,25)} \pgfmathresult
\end{codeexample}

\end{math-function}

\begin{math-function}{and(\mvar{x},\mvar{y})}
\mathcommand

	This returns |1| if \mvar{x} and \mvar{y} both evaluate to
	non-zero values. Otherwise |0| is returned.
	
\begin{codeexample}[]
\pgfmathparse{and(5>4,6>7)} \pgfmathresult
\end{codeexample}

\end{math-function}

\begin{math-function}{or(\mvar{x},\mvar{y})}
\mathcommand

	This returns |1| if either \mvar{x} or \mvar{y} evaluate to
	non-zero values. Otherwise |0| is returned.
	
\begin{codeexample}[]
\pgfmathparse{and(5>4,6>7)} \pgfmathresult
\end{codeexample}

\end{math-function}

\begin{math-function}{not(\mvar{x})}
\mathcommand

	This returns |1| if $\mvar{x}=0$, otherwise |0|.
	
\begin{codeexample}[]
\pgfmathparse{not(true)} \pgfmathresult
\end{codeexample}

\end{math-function}


\begin{math-function}{ifthenelse(\mvar{x},\mvar{y},\mvar{z})}
\mathcommand

	This returns \mvar{y} if \mvar{x} evaluates to some non-zero value,
	otherwise \mvar{z} is returned.
	
\begin{codeexample}[]
\pgfmathparse{ifthenelse(5==4,"yes","no")} \pgfmathresult
\end{codeexample}

\end{math-function}

\begin{math-function}{true}
\mathcommand

	This evaluates to |1|.
	
\begin{codeexample}[]
\pgfmathparse{true ? "yes" : "no"} \pgfmathresult
\end{codeexample}

\end{math-function}

\begin{math-function}{false}
\mathcommand

	This evaluates to |0|.
	
\begin{codeexample}[]
\pgfmathparse{false ? "yes" : "no"} \pgfmathresult
\end{codeexample}

\end{math-function}



\subsubsection{Pseudo-random functions}

\label{pgfmath-functions-random}

\begin{math-function}{rnd}
\mathcommand

	Generates a pseudo-random number between $0$ and $1$ with a uniform distribution.

\begin{codeexample}[]
\foreach \x in {1,...,10}{\pgfmathparse{rnd}\pgfmathresult, }
\end{codeexample}

\end{math-function}

\begin{math-function}{rand}
\mathcommand

	Generates a pseudo-random number between $-1$ and $1$ with a uniform distribution.

\begin{codeexample}[]
\foreach \x in {1,...,10}{\pgfmathparse{rand}\pgfmathresult, }
\end{codeexample}

\end{math-function}

\begin{math-function}{random(\opt{\mvar{x},\mvar{y}})}
\mathcommand
  This function takes zero, one or two arguments. If there are zero
  arguments, a uniform random number between $0$ and $1$ is generated. If there is
  one argument \mvar{x}, a random integer between $1$ and \mvar{x} is
  generated. Finally, if there are two arguments, a random integer
  between \mvar{x} and \mvar{y} is generated. If there are no
  arguments, the \pgfname{} command should be called as follows:
  |\pgfmathrandom{}|.

\begin{codeexample}[]
\foreach \x in {1,...,10}{\pgfmathparse{random()}\pgfmathresult, }
\end{codeexample}

\begin{codeexample}[]
\foreach \x in {1,...,10}{\pgfmathparse{random(100)}\pgfmathresult, }
\end{codeexample}

\begin{codeexample}[]
\foreach \x in {1,...,10}{\pgfmathparse{random(232,762)}\pgfmathresult, }
\end{codeexample}
\end{math-function}

\subsubsection{Base conversion functions}

\label{pgfmath-functions-base}

\begin{math-function}{hex(\mvar{x})}
\mathcommand

  Convert \mvar{x}{} (assumed to be an integer in base 10) to a
  hexadecimal representation, using lower case alphabetic digits.
	No further calculation will be possible with the result.
	
\begin{codeexample}[]
\pgfmathparse{hex(65535)} \pgfmathresult
\end{codeexample}
\end{math-function}

\begin{math-function}{Hex(\mvar{x})}
\mathcommand

  Convert \mvar{x}{} (assumed to be an integer in base 10) to a
  hexadecimal representation, using upper case alphabetic digits.
  No further calculation will be possible with the result.

\begin{codeexample}[]
\pgfmathparse{Hex(65535)} \pgfmathresult
\end{codeexample}
\end{math-function}

\begin{math-function}{oct(\mvar{x})}
\mathcommand

  Convert \mvar{x}{} (assumed to be an integer in base 10) to an
  octal representation.
  No further calculation should be attempted with the result, as
  the parser can only process numbers converted to base 10.
\begin{codeexample}[]
\pgfmathparse{oct(63)} \pgfmathresult
\end{codeexample}
\end{math-function}

\begin{math-function}{bin(\mvar{x})}
\mathcommand

  Convert \mvar{x}{} (assumed to be an integer in base 10) to a
  binary representation.
  No further calculation should be attempted with the result, as
  the parser can only process numbers converted to base 10.

\begin{codeexample}[]
\pgfmathparse{bin(185)} \pgfmathresult
\end{codeexample}
\end{math-function}

\subsubsection{Miscellaneous functions}

\label{pgfmath-functions-misc}

\begin{math-function}{min(\mvar{x$_1$},\mvar{x$_2$},\ldots,\mvar{x$_n$})}
\mathcommand[min({\mvar{x$_1$},\mvar{x$_2$},\ldots},{\ldots,\mvar{x$_{n-1}$},\mvar{x$_n$}})]

  Return the minimum value from \mvar{x$_1$}\ldots\mvar{x$_n$}.
  For historical reasons, the command |\pgfmathmin| takes two
  arguments, but each of these can contain an arbitrary number
  of comma separated values.

\begin{codeexample}[]
\pgfmathparse{min(3,4,-2,250,-8,100)} \pgfmathresult
\end{codeexample}

\end{math-function}


\begin{math-function}{max(\mvar{x$_1$},\mvar{x$_2$},\ldots,\mvar{x$_n$})}
\mathcommand[max({\mvar{x$_1$},\mvar{x$_2$},\ldots},{\ldots,\mvar{x$_{n-1}$},\mvar{x$_n$}})]

  Return the maximum value from \mvar{x$_1$}\ldots\mvar{x$_n$}.
  Again, for historical reasons, the command |\pgfmathmax| takes two
  arguments, but each of these can contain an arbitrary number
  of comma separated values.

\begin{codeexample}[]
\pgfmathparse{max(3,4,-2,250,-8,100)} \pgfmathresult
\end{codeexample}

\end{math-function}


\begin{math-function}{veclen(\mvar{x},\mvar{y})}
\mathcommand

 Calculates $\sqrt{\left(\textrm{\mvar{x}}^2+\textrm{\mvar{y}}^2\right)}$.
 This uses a polynomial approximation, based on ideas of Rouben Rostamian
\begin{codeexample}[]
\pgfmathparse{veclen(12,5)} \pgfmathresult
\end{codeexample}

\end{math-function}





\begin{math-function}{array(\mvar{x},\mvar{y})}
\mathcommand

	This accesses the array \mvar{x} at the index \mvar{y}. The
	array must begin and end with braces (e.g., |{1,2,3,4}|) and
	array indexing starts at |0|.
	
\begin{codeexample}[]
\pgfmathparse{array({9,13,17,21},2)} \pgfmathresult
\end{codeexample}

\end{math-function}


The following hyperbolic functions were adapted from code
suggested by Martin Heller:

\begin{math-function}{sinh(\mvar{x})}
\mathcommand

	The hyperbolic sine of \mvar{x}%
	
\begin{codeexample}[]
\pgfmathparse{sinh(0.5)} \pgfmathresult
\end{codeexample}

\end{math-function}

\begin{math-function}{cosh(\mvar{x})}
\mathcommand

	The hyperbolic cosine of \mvar{x}%
	
\begin{codeexample}[]
\pgfmathparse{cosh(0.5)} \pgfmathresult
\end{codeexample}

\end{math-function}

\begin{math-function}{tanh(\mvar{x})}
\mathcommand

	The hyperbolic tangent of \mvar{x}%
	
\begin{codeexample}[]
\pgfmathparse{tanh(0.5)} \pgfmathresult
\end{codeexample}

\end{math-function}

\begin{math-function}{width("\mvar{x}")}
\mathcommand

  Return the width of a \TeX{} (horizontal) box containing \mvar{x}.
  The quote characters are necessary to prevent \mvar{x}{} from being
  parsed.
  It is important to remember that any expression is expanded with
  |\edef| before being parsed, so any macros (e.g., font commands
  like |\tt| or |\Huge|) will need to be ``protected'' (e.g.,
  |\noexpand\Huge| is usually sufficient).

\begin{codeexample}[]
\pgfmathparse{width("Some Lovely Text")} \pgfmathresult
\end{codeexample}

	Note that results of this method are provided in points.
\end{math-function}

\begin{math-function}{height("\mvar{x}")}
\mathcommand

  Return the height of a box containing \mvar{x}.

\begin{codeexample}[]
\pgfmathparse{height("Some Lovely Text")} \pgfmathresult
\end{codeexample}
\end{math-function}

\begin{math-function}{depth("\mvar{x}")}
\mathcommand

  Returns the depth of a box containing \mvar{x}.

\begin{codeexample}[]
\pgfmathparse{depth("Some Lovely Text")} \pgfmathresult
\end{codeexample}
\end{math-function}

% Copyright 2007 by Mark Wibrow
%
% This file may be distributed and/or modified
%
% 1. under the LaTeX Project Public License and/or
% 2. under the GNU Free Documentation License.
%
% See the file doc/generic/pgf/licenses/LICENSE for more details.

\section{Additional Mathematical Commands}

\label{pgfmath-commands}

Instead of parsing and evaluating complex expressions, you can also
use the mathematical engine to evaluate a single mathematical
operation. The macros used for many of these computations are listed 
above in Section~\ref{pgfmath-functions}. \pgfname{} also provides 
some additional commands which are shown below:

\subsection{Basic arithmetic functions}

\label{pgfmath-commands-basic}

In addition to the commands described in 
Section~\ref{pgfmath-functions-basic}, the following command is
provided:

\begin{command}{\pgfmathreciprocal\marg{x}}         
	Defines |\pgfmathresult| as $1\div\meta{x}$. This provides 
	greatest accuracy when \mvar{x} is small.                  
\end{command}

\subsection{Comparison and logical functions}

In addition to the commands described in 
Section~\ref{pgfmath-functions-comparison}, 
the following command was provided by Christian Feuers\"anger:

\begin{command}{\pgfmathapproxequalto\marg{x}\marg{y}}       
	Defines |\pgfmathresult| 1.0 if $ \rvert \meta{x} - \meta{y} \lvert < 0.0001$, but 0.0 otherwise.                    
	As a side-effect, the global boolean |\ifpgfmathcomparison| will be set accordingly.
\end{command}

\subsection{Pseudo-Random Numbers}

\label{pgfmath-random}

In addition to the commands described in 
Section~\ref{pgfmath-functions-random}, 
the following commands are provided:

\begin{command}{\pgfmathgeneratepseudorandomnumber}
	Defines |\pgfmathresult| as a pseudo-random integer between 1 and 
	$2^{31}-1$. This uses a linear congruency generator, based on ideas
	of Erich Janka.
\end{command}

\begin{command}{\pgfmathrandominteger\marg{macro}\marg{minimum}\marg{maximum}}
	This defines \meta{macro} as a pseudo-randomly generated integer from 
	the range \meta{minimum} to \marg{maximum} (inclusive).
	
\begin{codeexample}[]
\begin{pgfpicture}
   \foreach \x in {1,...,50}{
      \pgfmathrandominteger{\a}{1}{50}
      \pgfmathrandominteger{\b}{1}{50}
      \pgfpathcircle{\pgfpoint{+\a pt}{+\b pt}}{+2pt}
      \color{blue!40!white}
      \pgfsetstrokecolor{blue!80!black}
      \pgfusepath{stroke, fill}
   }	  
\end{pgfpicture}
\end{codeexample}
\end{command}

\begin{command}{\pgfmathdeclarerandomlist\marg{list name}\{\marg{item-1}\marg{item 2}...\}}
	This creates a list of items with the name \meta{list name}.
\end{command}

\begin{command}{\pgfmathrandomitem\marg{macro}\marg{list name}}
	Select an item from a random list \meta{list name}. The
	selected item is placed in \meta{macro}.
\end{command}

\begin{codeexample}[]
\begin{pgfpicture}
   \pgfmathdeclarerandomlist{color}{{red}{blue}{green}{yellow}{white}}
   \foreach \a in {1,...,50}{
      \pgfmathrandominteger{\x}{1}{85}
      \pgfmathrandominteger{\y}{1}{85}
      \pgfmathrandominteger{\r}{5}{10}
      \pgfmathrandomitem{\c}{color}
      \pgfpathcircle{\pgfpoint{+\x pt}{+\y pt}}{+\r pt}
      \color{\c!40!white}
      \pgfsetstrokecolor{\c!80!black}
      \pgfusepath{stroke, fill}
   }	  
\end{pgfpicture}
\end{codeexample}

\begin{command}{\pgfmathsetseed\marg{integer}}
  Explicitly sets the seed for the pseudo-random number generator. By
  default it is set to the value of |\time|$\times$|\year|.
\end{command}


      
\subsection{Base Conversion}
	
\label{pgfmath-bases}

\pgfname{} provides limited support for conversion between 
\emph{representations} of numbers. Currently the numbers must be
positive integers in the range $0$ to $2^{31}-1$, and the bases in the
range $2$ to $36$. All digits representing numbers greater than 9 (in
base ten), are alphabetic, but may be upper or lower case. 

In addition to the commands described in 
Section~\ref{pgfmath-functions-base}, 
the following commands are provided:

\begin{command}{\pgfmathbasetodec\marg{macro}\marg{number}\marg{base}}
	Defines \meta{macro} as the result of converting \meta{number} from
	base \meta{base} to base 10. Alphabetic digits can be upper or lower
	case.

\medskip{\def\medskip{}

\begin{codeexample}[]
\pgfmathbasetodec\mynumber{107f}{16} \mynumber
\end{codeexample}
	
	\noindent Note that, as usual in \TeX, the braces around an argument can be omitted if the argument is just a single token (a macro name is a single token).


\begin{codeexample}[]
\pgfmathbasetodec\mynumber{33FC}{20} \mynumber
\end{codeexample}

}\medskip

\end{command}

\begin{command}{\pgfmathdectobase\marg{macro}\marg{number}\marg{base}}
	Defines \meta{macro} as the result of converting \meta{number} from
	base 10 to base \meta{base}. Any resulting alphabetic digits are in
	\emph{lower case}.
	
\begin{codeexample}[]
\pgfmathdectobase\mynumber{65535}{16} \mynumber
\end{codeexample}

\end{command}

\begin{command}{\pgfmathdectoBase\marg{macro}\marg{number}\marg{base}}
	Defines \meta{macro} as the result of converting \meta{number} from
	base 10 to base \meta{base}. Any resulting alphabetic digits are in
	\emph{upper case}.
	
\begin{codeexample}[]
\pgfmathdectoBase\mynumber{65535}{16} \mynumber
\end{codeexample}

\end{command}

\begin{command}{\pgfmathbasetobase\marg{macro}\marg{number}\marg{base-1}\marg{base-2}}
	Defines \meta{macro} as the result of converting \meta{number} from
	base \meta{base-1} to base \meta{base-2}. Alphabetic digits in 
	\meta{number} can be upper or lower case, but any resulting 
	alphabetic digits are in \emph{lower case}.
	
\begin{codeexample}[]
\pgfmathbasetobase\mynumber{11011011}{2}{16} \mynumber
\end{codeexample}

\end{command}

\begin{command}{\pgfmathbasetoBase\marg{macro}\marg{number}\marg{base-1}\marg{base-2}}
	Defines \meta{macro} as the result of converting \meta{number} from
	base \meta{base-1} to base \meta{base-2}. Alphabetic digits in 
	\meta{number} can be upper or lower case, but any resulting 
	alphabetic digits are in \emph{upper case}.
	
\begin{codeexample}[]
\pgfmathbasetoBase\mynumber{121212}{3}{12} \mynumber
\end{codeexample}

\end{command}


\begin{command}{\pgfmathsetbasenumberlength\marg{integer}}
	Sets the number of digits in the result of a base conversion to 
	\meta{integer}. If the result of a conversion has less digits
	than this number, it is prefixed with zeros.

\begin{codeexample}[]
\pgfmathsetbasenumberlength{8}
\pgfmathdectobase\mynumber{15}{2} \mynumber
\end{codeexample}
\end{command}




\subsection{Angle Computations}

Unlike the rest of the math engine, which is a ``standalone'' package,
the following commands only work in conjunction with the core of
\pgfname.



\begin{command}{\pgfmathanglebetweenpoints\marg{p}\marg{q}}
  Returns the angle of a line from \meta{p} to \meta{q} relative to a
  line going straight right from \meta{p}.
  
\begin{codeexample}[]
\pgfmathanglebetweenpoints{\pgfpoint{1cm}{3cm}}{\pgfpoint{2cm}{4cm}}
\pgfmathresult
\end{codeexample}
\end{command}

\begin{command}{\pgfmathanglebetweenlines\marg{$p_1$}\marg{$q_1$}\marg{$p_2$}\marg{$q_2$}}
  Returns the clockwise angle between a line going through $p_1$ and
  $q_1$ and a line going through $p_2$ and $q_2$.
  
\begin{codeexample}[]
\pgfmathanglebetweenlines{\pgfpoint{1cm}{3cm}}{\pgfpoint{2cm}{4cm}}
                         {\pgfpoint{0cm}{1cm}}{\pgfpoint{1cm}{0cm}}  
\pgfmathresult
\end{codeexample}
\end{command}

% Copyright 2007 by Mark Wibrow
%
% This file may be distributed and/or modified
%
% 1. under the LaTeX Project Public License and/or
% 2. under the GNU Free Documentation License.
%
% See the file doc/generic/pgf/licenses/LICENSE for more details.


\section[Reimplementing the Computations of the Mathematical Engine]
  {Reimplementing the Computations of the\\ Mathematical Engine}

\label{pgfmath-reimplement}

Perhaps you are not satisfied with the Newton-Raphson approximation to
square-roots. Perhaps you have a fantastically more accurate
and efficient way of calculating the sine or cosine of angles. Perhaps
 you would like the library to interface with a package such as |fp| 
 for fixed-point arithmetic (but note that |fp| is \emph{very} slow).
In these case you will want to replace the current implementations of
the computations done by the mathematical engine by your own code. 

The mathematical engine was designed with such a replacement in
mind. For this reason, the operations and functions like |\pgfmathadd|
are implemented in the following manner: 

\begin{itemize}
\item |\pgfmath|\meta{function name} 

  This macro is the ``public'' interface for the function
  \meta{function name}. All arguments passed to this macro are 
  evaluated using |\pgfmathparse| and then passed on to the following
  function:
  
\item |\pgfmath|\meta{function name}|@|
  
  This macro is the ``non-public'' implementation of the functions 
  algorithm (but note that, for speed, the parser calls this macro 
  rather than the ``public'' one). Arguments passed to this macro 
  are expected to be numbers \emph{without units}. This is the macro 
  which should be rewritten with your prize-winning new algorithm.
	
\end{itemize}

The effect of |\pgfmath|\meta{function name}|@| should be to set the
macro |\pgfmathresult| to the correct value (namely to the result of
the computation without units). Furthermore, the function should have
no other side effects, that is, it should not change any global
values. One way to achieve this is to use the following code:

\begin{codeexample}[code only]
\def\pgfmath...@#1#2...{%
   \begingroup%
      ... code for algorithm XXX ...
      \pgfmath@returnone\pgfmath@x%
   \endgroup%
}
\end{codeexample}


The macro |\pgfmath@returnone<macro>| uses some |\aftergroup| magic to
save result of the algorithm, by defining |\pgfmathresult| as the 
expansion of |<macro>| \emph{without units}. |<macro>| can be a macro
containing a number (with or without units), or a dimension or count
(or possibly a skip) register. By performing the algorithm within a
\TeX{} group, \pgfname{} registers such as |\pgf@x|, |\pgf@y| and 
|\c@pgf@counta|, |\c@pgfcountb|, and so forth can be used at will.
Note that current the implementation uses |\pgfmath@x|, |\pgfmath@y|, 
and |\c@pgfmath@counta|, |\c@pgfmath@countb| throughout, so for 
consistency these should be employed. Whilst they are currently |\let|
to their \pgfname{} equivalents (see |pgfmathutil.code.tex|), this 
could change (as could the \pgfname{} registers), so keeping things
consistent is probably a good idea.
\section{Number Printing}

\label{pgfmath-numberprinting}%

{\emph{An extension by Christian Feuers\"anger}}

\medskip
\noindent
\pgfname\ supports number printing in different styles and rounds to arbitrary precision.

\begin{command}{\pgfmathprintnumber\marg{x}}
Generates pretty-printed output for the (real) number \marg{x}. The input number \marg{x} is parsed using |\pgfmathfloatparsenumber| which allows arbitrary precision.

Numbers are typeset in math mode using the current set of number printing options, see below. Optional arguments can also be provided using |\pgfmathprintnumber[|\meta{options}|]|\marg{x}.
\end{command}

\begin{command}{\pgfmathprintnumberto\marg{x}\marg{\textbackslash macro}}
	Returns the resulting number into \marg{\textbackslash macro} instead of typesetting it directly.	
\end{command}

\begin{key}{/pgf/number format/fixed}
Configures |\pgfmathprintnumber| to round the number to a fixed number of digits after the period, discarding any trailing zeros.

\begin{codeexample}[]
\pgfkeys{/pgf/number format/.cd,fixed,precision=2}
\pgfmathprintnumber{4.568}\hspace{1em}
\pgfmathprintnumber{5e-04}\hspace{1em}
\pgfmathprintnumber{0.1}\hspace{1em}
\pgfmathprintnumber{24415.98123}\hspace{1em}
\pgfmathprintnumber{123456.12345}
\end{codeexample}

See section~\ref{sec:number:styles} for how to change the appearance.
\end{key}

\begin{key}{/pgf/number format/fixed zerofill=\marg{boolean}  (default true)}
Enables or disables zero filling for any number drawn in fixed point format.

\begin{codeexample}[]
\pgfkeys{/pgf/number format/.cd,fixed,fixed zerofill,precision=2}
\pgfmathprintnumber{4.568}\hspace{1em}
\pgfmathprintnumber{5e-04}\hspace{1em}
\pgfmathprintnumber{0.1}\hspace{1em}
\pgfmathprintnumber{24415.98123}\hspace{1em}
\pgfmathprintnumber{123456.12345}
\end{codeexample}
This key affects numbers drawn with |fixed| or |std| styles (the latter only if no scientific format is choosen).
\begin{codeexample}[]
\pgfkeys{/pgf/number format/.cd,std,fixed zerofill,precision=2}
\pgfmathprintnumber{4.568}\hspace{1em}
\pgfmathprintnumber{5e-05}\hspace{1em}
\pgfmathprintnumber{1}\hspace{1em}
\pgfmathprintnumber{123456.12345}
\end{codeexample}

See section~\ref{sec:number:styles} for how to change the appearance.
\end{key}

\begin{key}{/pgf/number format/sci}
Configures |\pgfmathprintnumber| to display numbers in scientific format, that means sign, mantisse and exponent (basis~$10$). The mantisse is rounded to the desired |precision| (or |sci precision|, see below).

\begin{codeexample}[]
\pgfkeys{/pgf/number format/.cd,sci,precision=2}
\pgfmathprintnumber{4.568}\hspace{1em}
\pgfmathprintnumber{5e-04}\hspace{1em}
\pgfmathprintnumber{0.1}\hspace{1em}
\pgfmathprintnumber{24415.98123}\hspace{1em}
\pgfmathprintnumber{123456.12345}
\end{codeexample}

See section~\ref{sec:number:styles} for how to change the exponential display style.
\end{key}

\begin{key}{/pgf/number format/sci zerofill=\marg{boolean}  (default true)}
Enables or disables zero filling for any number drawn in scientific format.

\begin{codeexample}[]
\pgfkeys{/pgf/number format/.cd,sci,sci zerofill,precision=2}
\pgfmathprintnumber{4.568}\hspace{1em}
\pgfmathprintnumber{5e-04}\hspace{1em}
\pgfmathprintnumber{0.1}\hspace{1em}
\pgfmathprintnumber{24415.98123}\hspace{1em}
\pgfmathprintnumber{123456.12345}
\end{codeexample}
As with |fixed zerofill|, this option does only affect numbers drawn in |sci| format (or |std| if the scientific format is chosen).

See section~\ref{sec:number:styles} for how to change the exponential display style.
\end{key}

\begin{stylekey}{/pgf/number format/zerofill=\marg{boolean} (default true)}
	Sets both, |fixed zerofill| and |sci zerofill| at once.
\end{stylekey}

\begin{keylist}{/pgf/number format/std,%
	/pgf/number format/std=\meta{lower e},
	/pgf/number format/std=\meta{lower e}:\meta{upper e}}
Configures |\pgfmathprintnumber| to a standard algorithm. It chooses either |fixed| or |sci|, depending on the order of magnitude. Let $n=s \cdot m \cdot 10^e$ be the input number and $p$ the current precision. If $-p/2 \le e \le 4$, the number is displayed using |fixed| format. Otherwise, it is displayed using |sci| format. 

\begin{codeexample}[]
\pgfkeys{/pgf/number format/.cd,std,precision=2}
\pgfmathprintnumber{4.568}\hspace{1em}
\pgfmathprintnumber{5e-04}\hspace{1em}
\pgfmathprintnumber{0.1}\hspace{1em}
\pgfmathprintnumber{24415.98123}\hspace{1em}
\pgfmathprintnumber{123456.12345}
\end{codeexample}
The parameters can be customized using the optional integer argument(s): if $\text{\meta{lower e}} \le e \le \text{\meta{upper e}}$, the number is displayed in |fixed| format, otherwise in |sci| format. Note that \meta{lower e} should be negative for useful results. The precision used for scientific format can be adjusted with |sci precision| if necessary.

\end{keylist}

\begin{key}{/pgf/number format/int detect}
Configures |\pgfmathprintnumber| to detect integers automatically. If the input number is an integer, no period is displayed at all. If not, the scientific format is chosen.

\begin{codeexample}[]
\pgfkeys{/pgf/number format/.cd,int detect,precision=2}
\pgfmathprintnumber{15}\hspace{1em}
\pgfmathprintnumber{20}\hspace{1em}
\pgfmathprintnumber{20.4}\hspace{1em}
\pgfmathprintnumber{0.01}\hspace{1em}
\pgfmathprintnumber{0}
\end{codeexample}
\end{key}

\begin{command}{\pgfmathifisint\marg{number constant}\marg{true code}\marg{false code}}
	A command which does the same check as |int detect|, but it invokes \meta{true code} if the \meta{number constant} actually is an integer and the \meta{false code} if not.

	As a side--effect, |\pgfretval| will contain the parsed number, either in integer format or as parsed floating point number.


	The argument \meta{number constant} will be parsed with |\pgfmathfloatparsenumber|. 
\begin{codeexample}[]
15 \pgfmathifisint{15}{is an int: \pgfretval.}{is no int}\hspace{1em}
15.5 \pgfmathifisint{15.5}{is an int: \pgfretval.}{is no int}
\end{codeexample}
\end{command}

\begin{key}{/pgf/number format/int trunc}
Truncates every number to integers (discards any digit after the period).

\begin{codeexample}[]
\pgfkeys{/pgf/number format/.cd,int trunc}
\pgfmathprintnumber{4.568}\hspace{1em}
\pgfmathprintnumber{5e-04}\hspace{1em}
\pgfmathprintnumber{0.1}\hspace{1em}
\pgfmathprintnumber{24415.98123}\hspace{1em}
\pgfmathprintnumber{123456.12345}
\end{codeexample}
\end{key}

\begin{key}{/pgf/number format/frac}
Displays numbers as fractionals.

\begin{codeexample}[width=3cm]
\pgfkeys{/pgf/number format/frac}
\pgfmathprintnumber{0.333333333333333}\hspace{1em}
\pgfmathprintnumber{0.5}\hspace{1em}
\pgfmathprintnumber{2.133333333333325e-01}\hspace{1em}
\pgfmathprintnumber{0.12}\hspace{1em}
\pgfmathprintnumber{2.666666666666646e-02}\hspace{1em}
\pgfmathprintnumber{-1.333333333333334e-02}\hspace{1em}
\pgfmathprintnumber{7.200000000000000e-01}\hspace{1em}
\pgfmathprintnumber{6.666666666666667e-02}\hspace{1em}
\pgfmathprintnumber{1.333333333333333e-01}\hspace{1em}
\pgfmathprintnumber{-1.333333333333333e-02}\hspace{1em}
\pgfmathprintnumber{3.3333333}\hspace{1em}
\pgfmathprintnumber{1.2345}
\end{codeexample}

\begin{key}{/pgf/number format/frac TeX=\marg{\textbackslash macro} (initially \texttt{\textbackslash frac})}
	Allows to use a different implementation for |\frac| inside of the |frac| display type.
\end{key}
\begin{key}{/pgf/number format/frac shift=\marg{integer} (initially 4)}
	In case you experience problems because of stability problems, try experimenting with a different |frac shift|. 
	Higher shift values $k$ yield higher sensitivity to inaccurate data or inaccurate arithmetics.

	Technically, the following happens. If $r < 1$ is the fractional part of the mantissa, then a scale $i = 1/r \cdot 10^k$ is computed where $k$ is the shift; fractional parts of $i$ are neglected. The value $1/r$ is computed internally, its error is amplified.

	If you still experience stability problems, use |\usepackage{fp}| in your preamble. The |frac| style will then automatically employ the higher absolute precision of |fp| for the computation of $1/r$.
\end{key}
\end{key}

\begin{key}{/pgf/number format/precision=\marg{number}}
Sets the desired rounding precision for any display operation. For scientific format, this affects the mantisse.
\end{key}

\begin{key}{/pgf/number format/sci precision=\meta{number or empty} (initially empty)}
	Sets the desired rounding precision only for |sci| styles. 

	Use |sci precision={}| to restore the initial configuration (which uses the argument provided to |precision| for all number styles).
\end{key}

\subsection{Changing display styles}%
\label{sec:number:styles}%
You can change the way how numbers are displayed. For example, if you use the `\texttt{fixed}' style, the input number is rounded to the desired precision and the current fixed point display style is used to typeset the number. The same is applied to any other format: first, rounding routines are used to get the correct digits, afterwards a display style generates proper \TeX-code.

\begin{key}{/pgf/number format/set decimal separator=\marg{text}}
Assigns \marg{text} as decimal separator for any fixed point numbers (including the mantisse in sci format).

Use |\pgfkeysgetvalue{/pgf/number format/set decimal separator}\value| to get the current separator into |\value|.
\end{key}

\begin{stylekey}{/pgf/number format/dec sep=\marg{text}}
	Just another name for |set decimal separator|.
\end{stylekey}

\begin{key}{/pgf/number format/set thousands separator=\marg{text}}
Assigns \marg{text} as thousands separator for any fixed point numbers (including the mantisse in sci format).

\begin{codeexample}[]
\pgfkeys{/pgf/number format/.cd,
	fixed,
	fixed zerofill,
	precision=2,
	set thousands separator={}}
\pgfmathprintnumber{1234.56}
\end{codeexample}
\begin{codeexample}[]
\pgfkeys{/pgf/number format/.cd,
	fixed,
	fixed zerofill,
	precision=2,
	set thousands separator={}}
\pgfmathprintnumber{1234567890}
\end{codeexample}

\begin{codeexample}[]
\pgfkeys{/pgf/number format/.cd,
	fixed,
	fixed zerofill,
	precision=2,
	set thousands separator={.}}
\pgfmathprintnumber{1234567890}
\end{codeexample}
\begin{codeexample}[]
\pgfkeys{/pgf/number format/.cd,
	fixed,
	fixed zerofill,
	precision=2,
	set thousands separator={,}}
\pgfmathprintnumber{1234567890}
\end{codeexample}
\begin{codeexample}[]
\pgfkeys{/pgf/number format/.cd,
	fixed,
	fixed zerofill,
	precision=2,
	set thousands separator={{{,}}}}
\pgfmathprintnumber{1234567890}
\end{codeexample}
The last example employs commas and disables the default comma-spacing.

Use |\pgfkeysgetvalue{/pgf/number format/set thousands separator}\value| to get the current separator into |\value|.
\end{key}

\begin{stylekey}{/pgf/number format/1000 sep=\marg{text}}
	Just another name for |set thousands separator|.
\end{stylekey}

\begin{key}{/pgf/number format/min exponent for 1000 sep=\marg{number} (initially 0)}
	Defines the smalles exponent in scientific notation which is required to draw thousand separators. The exponent is the number of digits minus one, so $\meta{number}=4$ will use thousand separators starting with $1e4 = 10000$.
\begin{codeexample}[]
\pgfkeys{/pgf/number format/.cd,
	int detect,
	1000 sep={\,},
	min exponent for 1000 sep=0}
\pgfmathprintnumber{5000}; \pgfmathprintnumber{1000000}
\end{codeexample}

\begin{codeexample}[]
\pgfkeys{/pgf/number format/.cd,
	int detect,
	1000 sep={\,},
	min exponent for 1000 sep=4}
\pgfmathprintnumber{1000}; \pgfmathprintnumber{5000}
\end{codeexample}
\begin{codeexample}[]
\pgfkeys{/pgf/number format/.cd,
	int detect,
	1000 sep={\,},
	min exponent for 1000 sep=4}
\pgfmathprintnumber{10000}; \pgfmathprintnumber{1000000}
\end{codeexample}
\noindent A value of |0| disables this feature (negative values are ignored).
\end{key}

\begin{key}{/pgf/number format/use period}
A predefined style which installs periods `\texttt{.}' as decimal separators and commas `\texttt{,}' as thousands separators. This style is the default.

\begin{codeexample}[]
\pgfkeys{/pgf/number format/.cd,fixed,precision=2,use period}
\pgfmathprintnumber{12.3456}
\end{codeexample}
\begin{codeexample}[]
\pgfkeys{/pgf/number format/.cd,fixed,precision=2,use period}
\pgfmathprintnumber{1234.56}
\end{codeexample}
\end{key}

\begin{key}{/pgf/number format/use comma}
A predefined style which installs commas `\texttt{,}' as decimal separators and periods `\texttt{.}' as thousands separators.

\begin{codeexample}[]
\pgfkeys{/pgf/number format/.cd,fixed,precision=2,use comma}
\pgfmathprintnumber{12.3456}
\end{codeexample}
\begin{codeexample}[]
\pgfkeys{/pgf/number format/.cd,fixed,precision=2,use comma}
\pgfmathprintnumber{1234.56}
\end{codeexample}
\end{key}

\begin{key}{/pgf/number format/skip 0.=\marg{boolean} (initially false)}
	Configures whether numbers like $0.1$ shall be typeset as $.1$ or not.
\begin{codeexample}[]
\pgfkeys{/pgf/number format/.cd,
	fixed,
	fixed zerofill,precision=2,
	skip 0.}
\pgfmathprintnumber{0.56}
\end{codeexample}
\begin{codeexample}[]
\pgfkeys{/pgf/number format/.cd,
	fixed,
	fixed zerofill,precision=2,
	skip 0.=false}
\pgfmathprintnumber{0.56}
\end{codeexample}
\end{key}

\begin{key}{/pgf/number format/showpos=\marg{boolean} (initially false)}
	Enables or disables display of plus signs for non-negative numbers.
\begin{codeexample}[]
\pgfkeys{/pgf/number format/showpos}
\pgfmathprintnumber{12.345}
\end{codeexample}

\begin{codeexample}[]
\pgfkeys{/pgf/number format/showpos=false}
\pgfmathprintnumber{12.345}
\end{codeexample}

\begin{codeexample}[]
\pgfkeys{/pgf/number format/.cd,showpos,sci}
\pgfmathprintnumber{12.345}
\end{codeexample}
\end{key}

\begin{stylekey}{/pgf/number format/print sign=\marg{boolean}}
	A style which is simply an alias for |showpos=|\marg{boolean}.
\end{stylekey}

\begin{key}{/pgf/number format/sci 10e}
Uses $m \cdot 10^e$ for any number displayed in scientific format.

\begin{codeexample}[]
\pgfkeys{/pgf/number format/.cd,sci,sci 10e}
\pgfmathprintnumber{12.345}
\end{codeexample}
\end{key}

\begin{key}{/pgf/number format/sci 10\textasciicircum e}
The same as `|sci 10e|'.
\end{key}

\begin{key}{/pgf/number format/sci e}
Uses the `$1e{+}0$' format which is generated by common scientific tools for any number displayed in scientific format.

\begin{codeexample}[]
\pgfkeys{/pgf/number format/.cd,sci,sci e}
\pgfmathprintnumber{12.345}
\end{codeexample}
\end{key}

\begin{key}{/pgf/number format/sci E}
The same with an uppercase `\texttt{E}'.

\begin{codeexample}[]
\pgfkeys{/pgf/number format/.cd,sci,sci E}
\pgfmathprintnumber{12.345}
\end{codeexample}
\end{key}

\begin{key}{/pgf/number format/sci subscript}
Typesets the exponent as subscript for any number displayed in scientific format. This style requires very few space.

\begin{codeexample}[]
\pgfkeys{/pgf/number format/.cd,sci,sci subscript}
\pgfmathprintnumber{12.345}
\end{codeexample}
\end{key}

\begin{key}{/pgf/number format/sci superscript}
Typesets the exponent as superscript for any number displayed in scientific format. This style requires very few space.

\begin{codeexample}[]
\pgfkeys{/pgf/number format/.cd,sci,sci superscript}
\pgfmathprintnumber{12.345}
\end{codeexample}
\end{key}

\begin{key}{/pgf/number format/sci generic=\marg{keys}}
Allows to define an own number style for the scientific format. Here, \meta{keys} can be one of the following choices (omit the long key prefix):

\begin{key}{/pgf/number format/sci generic/mantisse sep=\marg{text} (initially empty)}
	Provides the separator between a mantisse and the exponent. It might be |\cdot|, for example,
\end{key}
\begin{key}{/pgf/number format/sci generic/exponent=\marg{text} (initially empty)}
	Provides text to format the exponent. The actual exponent is available as argument |#1| (see below).
\end{key}

\begin{codeexample}[]
\pgfkeys{
	/pgf/number format/.cd,
	sci,
	sci generic={mantisse sep=\times,exponent={10^{#1}}}}
\pgfmathprintnumber{12.345};
\pgfmathprintnumber{0.00012345}
\end{codeexample}
	The \meta{keys} can depend on three parameters, namely on |#1| which is the exponent, |#2| containing the flags entity of the floating point number and |#3| is the (unprocessed and unformatted) mantisse.

	Note that |sci generic| is \emph{not} suitable to modify the appearance of fixed point numbers, nor can it be used to format the mantisse (which is typeset like fixed point numbers). Use |dec sep|, |1000 sep| and |print sign| to customize the mantisse.
\end{key}


\begin{key}{/pgf/number format/@dec sep mark=\marg{text}}
	Will be placed right before the place where a decimal separator belongs to. However, \marg{text} will be inserted even if there is no decimal separator. It is intented as place-holder for auxiliary routines to find alignment positions.

	This key should never be used to change the decimal separator! Use |dec sep| instead. 
\end{key}

\begin{key}{/pgf/number format/@sci exponent mark=\marg{text}}
	Will be placed right before exponents in scientific notation. It is intented as place-holder for auxiliary routines to find alignment positions.

	This key should never be used to change the exponent!
\end{key}

\begin{key}{/pgf/number format/assume math mode=\marg{boolean} (default true)}
	Set this to |true| if you don't want any checks for math mode. The initial setting checks whether math mode is active using |\pgfutilensuremath| for each final number. 
	
	Use |assume math mode=true| if you know that math mode is active. In that case, the final number is typeset as-is, no further checking is performed.
\end{key}


\begin{stylekey}{/pgf/number format/verbatim}
	A style which configures the number printer to produce verbatim text output, i.e.\ it doesn't contain \TeX\ macros.
\begin{codeexample}[]
\pgfkeys{
	/pgf/fpu,
	/pgf/number format/.cd,
	sci,
	verbatim}
\pgfmathprintnumber{12.345};
\pgfmathprintnumber{0.00012345};
\pgfmathparse{exp(15)}
\pgfmathprintnumber{\pgfmathresult}
\end{codeexample}
	The style resets |1000 sep|, |dec sep|, |print sign|, |skip 0.| and sets |assume math mode|. Furthermore, it installs a |sci generic| format for verbatim output of scientific numbers.

	However, it will still respect |precision|, |fixed zerofill|, |sci zerofill| and the overall styles |fixed|, |sci|, |int detect| (and their variants). It might be useful if you intent to write output files.
\end{stylekey}
%--------------------------------------------------
% \subsubsection{Defining own display styles}
% You can define own display styles, although this may require some insight into \TeX-programming. Here are two examples:
% \begin{enumerate}
% 	\item A new fixed point display style: The following code defines a new style named `\texttt{my own fixed point style}' which uses $1{\cdot}00$ instead of $1.00$.
% \begin{lstlisting}
% \def\myfixedpointstyleimpl#1.#2\relax{%
% 	#1{\cdot}#2%
% }%
% \def\myfixedpointstyle#1{%
% 	\pgfutilensuremath{%
% 	\ifpgfmathfloatroundhasperiod
% 		\expandafter\myfixedpointstyleimpl#1\relax
% 	\else
% 		#1%
% 	\fi
% 	}%
% }
% \pgfkeys{/my own fixed point style/.code={%
% 	\let\pgfmathprintnumber@fixed@style=\myfixedpointstyle}
% }%
% \end{lstlisting}
% 	You only need to overwrite the macro \lstinline!\pgfmathprintnumber@fixed@style!. This macro takes one argument (the result of any numerical computations). The \TeX-boolean \lstinline!\ifpgfmathfloatroundhasperiod! is true if and only if the input number contains a period.
% 
% 	\item An example for a new scientific display style:
% \begin{lstlisting}
% % #1:
% % 		0 == '0' (the number is +- 0.0),
% % 		1 == '+', 
% % 		2 == '-',
% % 		3 == 'not a number'
% % 		4 == '+ infinity'
% % 		5 == '- infinity'
% % #2: the mantisse
% % #3: the exponent
% \def\myscistyle#1#2e#3\relax{%
% 	...
% }
% \pgfkeys{/my own sci style/.code={%
% 	\let\pgfmathfloatrounddisplaystyle=\myscistyle},
% }%
% \end{lstlisting}
% \end{enumerate}
%-------------------------------------------------- 

% Copyright 2008 by Till Tantau
%
% This file may be distributed and/or modified
%
% 1. under the LaTeX Project Public License and/or
% 2. under the GNU Free Documentation License.
%
% See the file doc/generic/pgf/licenses/LICENSE for more details.



\section{Object-Oriented Programming}

\label{section-oop}

This section describes the |oo| module.

\begin{pgfmodule}{oo}
  This module defines a relatively small set of \TeX\ commands for
  defining classes, methods, attributes and objects in the sense of
  object-oriented programming.
\end{pgfmodule}

In this chapter it is assumed that you are familiar with the basics
of a typical object-oriented programming language like Java, C++ or
Eiffel.



\subsection{Overview}

\TeX\ does not support object-oriented programming, presumably because
it was written at a time when this style of programming was not yet
``en vogue.'' When one is used to the object-oriented style of
thinking, some programming constructs in \TeX\ often seem overly
complicated. The object-oriented programming module of \pgfname\ may
help here. It is written completely using simple \TeX\ macros and is,
thus, perfectly portable. This also means, however, that it is not
particularly fast (but not too slow either), so you should use it only
for non-time-critical things.

Basically, the oo-system supports \emph{classes} (in the
object-oriented sense, this has nothing to do with \LaTeX-classes),
\emph{methods}, \emph{constructors}, \emph{attributes},
\emph{objects}, and \emph{object identities}. It (currently) does not
support either inheritance, overloading, destructors, or class nesting.

The first step is to define a class, using the macro |\pgfooclass|
(all normal macros in \pgfname's object-oriented system start with
|\pgfoo|). This macro gets the name of a class and in its body a
number of \emph{methods} are defined. These are defined using the
|\method| macro (which is defined only inside such a class definition)
and they look a bit like method definitions in, say, Java. Object
attributes are declared using the |\attribute| command, which is also
defined only inside a class definition.

Once a class has been defined, you can create objects of this
class. Objects are created using |\pgfoonew|. Such an object has many
characteristics of objects in a normal object-oriented programming
language: Each object has a \emph{unique identity}, so when you create
another object this object is completely distinct from all other
objects. Each object also has a set of private attributes, which may
change over time. Suppose, for instance, that we have a |point|
class. Then creating a new object (called an instance) of this class
would typically have an |x|-attribute and a |y|-attribute. These can
be changed over time. Creating another instance of the
|point| class creates another object with its own |x|- and
|y|-attributes.

Given an object, you can call a method for this object. Inside the
method the attributes of the object for which the method is being
called can be accessed.

The life of an object always ends with the end of the \TeX\ scope in
which it was created. However, changes to attribute values are not
local to scopes, so when you change an attribute anywhere, this change
persists till the end of the life of the object or until the attribute
is changed again.

\subsection{A Running Example: The Stamp Class}

As a running example we will develop a |stamp| class and |stamp|
objects. The idea is that a stamp object is able to ``stamp
something'' on a picture. This means that a stamp object has an
attribute storing the ``stamp text'' and there is a method that asks
the object to place this text somewhere on a
canvas. The method can be called repeatedly and there can be several
different stamp objects, each producing a different text. Stamp
objects can either be created dynamically when needed or a library
might define many such objects in an outer scope.

Such stamps are similar to many things present in \pgfname\ such as
arrow tips, patterns, or shadings and, indeed, these could all have
been implemented in this object-oriented fashion (which might have
been better, but the object-oriented subsystem is a fairly new
addition to \pgfname).


\subsection{Classes}

We start with the definition of the |stamp| class. This is done using
the |\pgfooclass| macro:

\begin{command}{\pgfooclass\marg{class name}\marg{body}}
  This command defines a class named \meta{class name}. The name of
  the class can contain spaces and most other characters, but no
  periods. So, valid class names are |MyClass| or |my class| or
  |Class_C++_emulation??1|.

  The \meta{body} is actually just executed, so any
  normal \TeX-code is permissible here. However, while the \meta{body} is
  being executed, the macros |\method| and |\attribute| are setup so
  that they can be used to define methods and attributes for this
  class (the original meanings are restored afterward).

  The definition of a class is local to the scope where the class has
  been defined.
\begin{codeexample}[code only]
\pgfooclass{stamp}{
  % This is the class stamp

  \attribute text;
  \attribute rotation angle=20;

  \method stamp(#1) { % The constructor
    ...
  }

  \method apply(#1,#2) { % Causes the stamp to be shown at coordinate (#1,#2)
    ...
  }
}

% We can now create objects of type "stamp"
\end{codeexample}
\end{command}

The \meta{body} of a class usually just consists of calls to the macros
|\attribute| and |\method|, which will be discussed in more detail in
later sections.



\subsection{Objects}

Once a class has been declared, we can start creating objects for this
class. For this the |\pgfoonew| command can be used, which has a
peculiar syntax:

\begin{command}{\pgfoonew\opt{\meta{object handle or attribute}|=|}|new |\meta{class
      name}|(|\meta{constructor arguments}|)|}
  Causes a new object to be created. The class of the object will be
  \meta{class name}, which must previously have been declared using
  |\pgfooclass|. Once the object has been created, the constructor
  method of the object will be called with the parameter list set to
  \meta{constructor arguments}.

  The resulting object is stored internally and its lifetime will
  end exactly at the end of the current scope.

  Here is an example in which three stamp objects are created.
\begin{codeexample}[code only]
\pgfoonew \firststamp  = new stamp()
\pgfoonew \secondstamp = new stamp()
{
  \pgfoonew \thirdstamp = new stamp()
  ...
}
% \thirdstamp no longer exists, but \firststamp and \secondstamp do
% even if you try to store \thirdstamp in a global variable, trying
% to access it will result in an error.
\end{codeexample}

  The optional \meta{object handle or attribute} can either be an
  \meta{object handle} or an \meta{attribute}. When an \meta{object
    handle} is give, it must be a normal \TeX\ macro name that will
  ``point'' to the object (handles are discussed in more detail in
  Section~\ref{section-identities}). You can use this macro to call
  methods of the object as discussed in the following section. When an
  \meta{attribute} is given, it must be given in curly braces (the
  curly braces are used to detect the presence of an attribute). In
  this case, a handle to the newly created object is stored in this
  attribute.
\begin{codeexample}[code only]
\pgfooclass{foo}
{
  \attribute stamp obj;
  \attribute another object;

  \method \foo() {
    \pgfoonew{stamp obj}=new stamp()
    \pgfoonew{another object}=new bar()
  }
  ...
}
\end{codeexample}
\end{command}

\begin{command}{\pgfoogc}
  This command causes the ``garbage collector'' to be invoked. The job
  of this garbage collector is to free the global \TeX-macros that are
  used by ``dead'' objects (objects whose life-time has ended). This
  macro is called automatically after every scope in which an object
  has been created, so you normally do not need to call this macro
  yourself.
\end{command}


\subsection{Methods}

Methods are defined inside the body of classes using the following
command:

\begin{command}{\method \meta{method name}|(|\meta{parameter list}|)|\marg{method body}}
  This macro, which is only defined inside a class definition, defines
  a new method named \meta{method name}. Just like class names, method
  names can contain spaces and other characters, so \meta{method names} like
  |put_stamp_here| or |put stamp here| are both legal.

  Three method names are special: First, a method having the same name
  as the class is called the \emph{constructor} of the class. The must be
  such a method, even if its body is empty. There are (currently) no
  destructors; objects simply become ``undefined'' at the
  end of the scope in which they have been created. The other two
  methods are called |get id| and |get handle|, which are always automatically
  defined and which you cannot redefine. They are discussed in
  Section~\ref{section-identities}.

  Overloading of methods is not possible, that is, it is illegal to
  have two methods inside a single class with the same name (despite
  possibly different parameter lists). However, two different classes
  may contain a method with the same name, that is, classes form
  namespaces for methods.

  The \meta{method name} must be followed by a \meta{parameter list}
  in parentheses, which must be present even when the \meta{parameter
    list} is empty. The \meta{parameter list} is actually a normal
  \TeX\ parameter list that will be matched against the parameters
  inside the parentheses upon method invocation and, thus, could be
  something like |#1#2 foo #3 bar.|, but a list like |#1,#2,#3| is
  more customary. By setting the parameter list to just |#1| and then
  calling, say,  |\pgfkeys{#1}| at the beginning of a method, you can
  implement Objective-C-like named parameters.

  When a method is called, the \meta{body} of
  the method will be executed. The main difference to a normal macro
  is that while the \meta{body} is executed a special macro called
  |\pgfoothis| is setup in such a way that it references that object
  for which the method is executed.
\end{command}

In order to call a method for an object, you first need to create the
object and you need a handle for this object. In order to invoke a
method for this object, a special syntax is used that is similar to
Java or C++ syntax:

\begin{pgfmanualentry}
  \pgfmanualentryheadline{\meta{object handle}|.|\meta{method name}|(|\meta{parameters}|)|}%
  \pgfmanualbody
  This causes the method \meta{method name} to be called for the
  object referenced by the \meta{object handle}. Naturally, the method
  is the one defined in the class of the object. The \meta{parameters}
  are matched against the parameters of the method and, then, the
  method body is executed. The execution of the method body is
  \emph{not} done inside a scope, so the effects of a method body
  persist.

\begin{codeexample}[code only]
\pgfooclass{stamp}{
  % This is the class stamp

  \method stamp() { % The constructor
  }

  \method apply(#1,#2) { % Causes the stamp to be shown at coordinate (#1,#2)
    % Draw the stamp:
    \node [rotate=20,font=\huge] at (#1,#2) {Passed};
  }
}

\pgfoonew \mystamp=new stamp()

\begin{tikzpicture}
  \mystamp.apply(1,2)
  \mystamp.apply(3,4)
\end{tikzpicture}
\end{codeexample}

  Inside a method, you can call other methods. If you have a handle
  for another object, you can simply call it in the manner described
  above. In order to call a method of the current object, you can use
  the special object handle |\pgfoothis|.

  \begin{command}{\pgfoothis}
    This object handle is defined only when a method is being
    executed. There, it is then set to point to the object for which
    the method is being called, which allows you to call another
    method for the same object.

\begin{codeexample}[code only]
\pgfooclass{stamp}{
  % This is the class stamp

  \method stamp() {}

  \method apply(#1,#2) {
    \pgfoothis.shift origin(#1,#2)

    % Draw the stamp:
    \node [rotate=20,font=\huge] {Passed};
  }

  % Private method:
  \method shift origin(#1,#2) {
    \tikzset{xshift=#1,yshift=#2}
  }
}
\end{codeexample}
  \end{command}
\end{pgfmanualentry}



\subsection{Attributes}

Every object has a set of attributes, which may change over
time. Attributes are declared using the |\attribute| command, which,
like the |\method| command, is defined only inside the scope of
|\pgfooclass|. Attributes can be modified (only) by methods. To take
the |stamp| example, an attribute of a |stamp| object might be the
text that should be stamped when the |apply| method is called.

When an attribute is changed, this change is \emph{not} local to the
current \TeX\ group. Changes will persist till the end
of the object's life or until the attribute is changed once more.

To declare an attribute you should use the |\attribute| command:
\begin{command}{\attribute \meta{attribute name}\opt{|=|\meta{initial
        value}}|;|}
  This command can only be given inside the body of an |\pgfooclass|
  command. It declares the attribute named \meta{attribute name}. This
  name, like method or class names, can be quite arbitrary, but should
  not contain periods. Valid names are |an_attribute?| or
  |my attribute|.

  You can optionally specify an \meta{initial value} for the
  attribute; if none is given, the empty string is used
  automatically. The initial value is the value that the attribute
  will have just after the object has been created and before the
  constructor is called.

\begin{codeexample}[code only]
\pgfooclass{stamp}{
  % This is the class stamp

  \attribute text;
  \attribute rotation angle = 20;

  \method stamp(#1) {
    \pgfooset{text}{#1} % Set the text
  }

  \method apply(#1,#2) {
    \pgfoothis.shift origin(#1,#2)

    % Draw the stamp:
    \node [rotate=\pgfoovalueof{rotation angle},font=\huge]
      {\pgfoovalueof{text}};
  }

  \method shift origin(#1,#2) { ... }

  \method set rotation (#1) {
    \pgfooset{rotation angle}{#1}
  }
}
\end{codeexample}
\end{command}

Attributes can be set and read only inside methods, it is not possible
to do so using an object handle. Spoken in terms of traditional
object-oriented programming, attributes are always private. You need
to define getter and setter methods if you wish to read or modify
attributes.

Reading and writing attributes is not done using the ``dot-notation''
that is used for method calls. This is mostly due to efficiency
reasons. Instead, a set of special macros is used, all of which can
\emph{only be used inside methods}.

\begin{command}{\pgfooset\marg{attribute}\marg{value}}
  Sets the \meta{attribute} of the current object to
  \meta{value}.
\begin{codeexample}[code only]
\method set rotation (#1) {
  \pgfooset{rotation angle}{#1}
}
\end{codeexample}
\end{command}

\begin{command}{\pgfoolet\marg{attribute}\marg{macro}}
  Sets the \meta{attribute} of the current value to the current value
  of \meta{macro} using \TeX's |\let| command.
\begin{codeexample}[code only]
\method foo () {
  \pgfoolet{my func}\myfunc
  % Changing \myfunc now has no effect on the value of attribute my func
}
\end{codeexample}
\end{command}

\begin{command}{\pgfoovalueof\marg{attribute}}
  Expands  (eventually) to the current value of \meta{attribute} of
  the current object.
\begin{codeexample}[code only]
\method apply(#1,#2) {
  \pgfoothis.shift origin(#1,#2)

  \node [rotate=\pgfoovalueof{rotation angle},font=\huge]
    {\pgfoovalueof{text}};
}
\end{codeexample}
\end{command}

\begin{command}{\pgfooget\marg{attribute}\marg{macro}}
  Reads the current value of \meta{attribute} and stores the result in
  \meta{macro}.
\begin{codeexample}[code only]
...
  \method get rotation (#1) {
    \pgfooget{rotation angle}{#1}
  }
...

\mystamp.get rotation(\therotation)
``\therotation'' is now ``20'' (or whatever).
\end{codeexample}
\end{command}


\subsection{Identities}
\label{section-identities}

Every object has a unique identity, which is simply an integer. It is
possible to retrieve the object id using the |get id| method (discussed
below), but normally you will not need to do so because the id itself
cannot be used to access an object. Rather, you access objects via
their methods and these, in turn, can only be called via object
handles.

Object handles can be created in four ways:
\begin{enumerate}
\item Calling |\pgfoonew|\meta{object handle}|=...| will cause
  \meta{object handle} to be a handle to the newly created object.
\item Using |\let| to create an alias of an existing object handle: If
  |\mystamp| is a handle, saying |\let\myotherstamp=\mystamp| creates
  a second handle to the same object.
\item |\pgfooobj|\marg{id} can be used as an object handle to the
  object with the given \meta{id}.
\item Using the |get handle| method to create a handle to a given object.
\end{enumerate}

Let us have a look at the last two methods.

\begin{command}{\pgfooobj\marg{id}}
  Provided that \meta{id} is the id of an existing object (an object
  whose life-time has not expired), calling this command yields a
  handle to this object. The handle can then be used to call methods:
\begin{codeexample}[code only]
% Create a new object:
\pgfoonew \mystamp=new stamp()

% Get the object's id and store it in \myid:
\mystamp.get id(\myid)

% The following two calls have the same effect:
\mystamp.apply(1,1)
\pgfooobj{\myid}.apply(1,1)
\end{codeexample}
\end{command}

The |get id| method can be used to retrieve the id of an object. This
method is predefined for every class and you should not try to define
a method of this name yourself.

\begin{predefinedmethod}{get id(\meta{macro})}
  Calling \meta{obj}|.get id(|\meta{macro}|)|  stores the id \meta{obj} in
  \meta{macro}. This is mainly useful when you wish to store an
  object for a longer time and you cannot guarantee that any handle
  that you happen to have for this object will be available later on.

  The only way to use the retrieved id later on is to call
  |\pgfooobj|.

  Different object that are alive (that are still within the scope in
  which they were created) will always have different ids, so you can
  use the id to test for equality of objects. However, after an object
  has been destroyed because its scope has ended the same id may be
  used again for newly created objects.

  Here is a typical application where you need to call this method:
  You wish to collect a list of objects for which you wish to call a
  specific method from time to time. For the collection process you
  wish to offer a macro called |\addtoobjectlist|, which takes an
  object handle as parameter. It is quite easy to store this handle
  somewhere, but a handle is, well, just a handle. Typically, shortly
  after the call to |\addtoobjectlist| the handle will no longer be
  valid or even exist, even though the object still exists. In this
  case, you wish to store the object id somewhere instead of the
  handle. Thus, for the object passed to |\addtoobjectlist| you call
  the |get id| method and store the resulting id, rather than the handle.
\end{predefinedmethod}

There is a second predefined methods, called |get handle|, which is also
used to create object handles.

\begin{predefinedmethod}{get handle(\marg{macro name})}
  Calling this method for an object will cause \meta{macro name} to
  become a handle to the given object. For any object handle |\obj| --
  other than |\pgfoothis| -- the following two have the same effect:
  \begin{enumerate}
  \item |\let|\meta{macro name}|=\obj|
  \item |\obj.get handle(|\meta{macro name}|)|
  \end{enumerate}

  The first method is simpler and faster. However, for |\pgfoothis|
  there is a difference: The call |\pgfoothis.get handle(|\meta{macro name}|)| will
  cause \meta{macro name} to be an object handle to the current
  object and will persist to be so even after the method is done. By
  comparison, |\let|\meta{macro name}|=\pgfoothis| causes |\obj| to be
  the same as the very special macro |\pgfoothis|, so |\obj| will
  always refer to the current object, which may change over time.
\end{predefinedmethod}




\subsection{The Signal Class}

\label{section-signals}

The object-oriented module predefines, in addition to the basic mechanism
for defining and using classes and object, one class: |signal|. This
class implements a so-called signal--slot mechanism.

\begin{ooclass}{signal}
  This class is used to implement a simple signal--slot
  mechanism. The idea is the following: Form time to time special
  things happen about which a number of objects need to be
  informed. Different things can happen and different object will be
  interested in these things. A |signal| object can be used to signal
  that such special things of a certain kind have happened. For
  example, on signal object might be used to signal the event that ``a
  page has been shipped out.'' Another signal might be used to signal
  that ``a figure is about to be typeset,'' and so on.

  Objects can ``tune in'' to signals. They do so by \emph{connecting}
  one of their methods (then called a \emph{slot}) to the
  signal. Then, whenever the signal is \emph{emitted}, the method of
  the connected object(s) get called. Different objects can connect
  different slots to the same signal as long as the argument lists
  will fit. For example, the object that is used to signal the ``end
  of page has been reached'' might emit signals that have, say, the
  box number in which the finished page can be found as a parameter
  (actually, the finished page is always in box 255). Then one object
  could connect a method |handle page(#1)| to this signal, another
  might connect the method |emergency action(#1)| to this signal, and
  so on.

  Currently, it is not possible to ``unregister'' or ``detach'' a slot
  from a signal, that is, once an object has been connect to a signal,
  it will continue to receive emissions of this signal till the end
  of the life-time of the signal. This is even true when the object no
  longer exists (but the signal does), so care must be taken that
  signal objects are always created before the objects that are
  listening to them.

  \begin{method}{signal()}
    The constructor does nothing.
  \end{method}

  \begin{method}{connect(\meta{object handle},\meta{method name})}
    This method gets an \meta{object handle} as parameter and a
    \meta{method name} of this object. It will queue the object-method
    pair in an internal list and each time the signal emits something,
    this object's method is called.

    Be careful not to pass |\pgfoothis| as \meta{object handle}. This
    would cause the signal object to connect to itself. Rather, if you
    wish to connect a signal to a method of the current object you
    first need to create an alias using the |get handle| method:
\begin{codeexample}[code only]
\pgfooclass{some class}{
  \method some class() {
    \pgfoothis.get handle(\me)
    \somesignal.connect(\me,foo)
    \anothersignal.connect(\me,bar)
  }
  \method foo () {}
  \method bar (#1,#2) {}
}
\pgfoonew \objA=new some class()
\pgfoonew \objB=new some class()
\end{codeexample}
  \end{method}

  \begin{method}{emit(\meta{arguments})}
    This method emits a signal to all connected slots. This means that
    for all objects that have previously been connected via a call of
    |connect|, the method (slot) that was specified during the call of
    |connect| is invoked with given \meta{arguments}.
\begin{codeexample}[code only]
\anothersignal.emit(1,2)
% will call \objA.bar(1,2) and \objB.bar(1,2)
\end{codeexample}
  \end{method}
\end{ooclass}


\subsection{Implementation Notes}

For the curious, here are some notes on how the oo-system is
implemented:

\begin{itemize}
\item There is an object id counter that gets incremented each time an
  object is created. However, this counter is local to the current
  scope, which means that it is reset at the end of each scope,
  corresponding to the fact that at the end of a scope all objects
  created in this scope become invalid. Newly created objects will
  then have the same id as ``deleted'' objects.
\item Attributes are stored globally. For each attribute of each
  object there is a macro whose name is composed of the object's id
  and the attribute name. Changes to object attributes are always
  global.
\item A call to the garbage collector causes a loop to be executed
  that tries to find objects whose object number is larger than the
  current maximum alive objects. The global attributes of these
  objects are then freed (set to |\relax|) by calling a special
  internal method of these (dead) objects.

  The garbage collector is automatically called after each group in
  which an object was created using |\aftergroup|.
\item When a method is called, before the method call some code is
  executed that sets a global counter storing the current object id to
  the object id of the object being called. After the method call some
  code is inserted that restores the global counter to its original
  value. This is done without scopes, so some tricky |\expandafter|
  magic is needed. Note that, because of this process, you cannot use
  commands like |\pgfutil@ifnextchar| at the end of a method.
\item An object handle contains just the code to setup and restore the
  current object number to the number of the object being called.
\end{itemize}



\part{The Basic Layer}

{\Large \emph{by Till Tantau}}


\bigskip
\noindent
\vskip1cm
\begin{codeexample}[graphic=white]
\begin{tikzpicture}
  \draw[gray,very thin] (-1.9,-1.9) grid (2.9,3.9)
          [step=0.25cm] (-1,-1) grid (1,1);
  \draw[blue] (1,-2.1) -- (1,4.1); % asymptote

  \draw[->] (-2,0) -- (3,0) node[right] {$x(t)$};
  \draw[->] (0,-2) -- (0,4) node[above] {$y(t)$};

  \foreach \pos in {-1,2}
    \draw[shift={(\pos,0)}] (0pt,2pt) -- (0pt,-2pt) node[below] {$\pos$};

  \foreach \pos in {-1,1,2,3}
    \draw[shift={(0,\pos)}] (2pt,0pt) -- (-2pt,0pt) node[left] {$\pos$};

  \fill (0,0) circle (0.064cm);
  \draw[thick,parametric,domain=0.4:1.5,samples=200]
    % The plot is reparameterised such that there are more samples
    % near the center.
    plot[id=asymptotic-example] function{(t*t*t)*sin(1/(t*t*t)),(t*t*t)*cos(1/(t*t*t))}
    node[right] {$\bigl(x(t),y(t)\bigr) = (t\sin \frac{1}{t}, t\cos \frac{1}{t})$};

  \fill[red] (0.63662,0) circle (2pt)
    node [below right,fill=white,yshift=-4pt] {$(\frac{2}{\pi},0)$};
\end{tikzpicture}
\end{codeexample}


% Copyright 2006 by Till Tantau
%
% This file may be distributed and/or modified
%
% 1. under the LaTeX Project Public License and/or
% 2. under the GNU Free Documentation License.
%
% See the file doc/generic/pgf/licenses/LICENSE for more details.


\section{Design Principles}

This section describes the basic layer of \pgfname. This layer is
built on top of the system layer. Whereas the system layer just
provides the absolute minimum for drawing graphics, the basic
layer provides numerous commands that make it possible to create
sophisticated graphics easily and also quickly.

The basic layer does not provide a convenient syntax for describing
graphics, which is left to frontends like \tikzname. For this reason, the
basic layer is typically used only by ``other programs.'' For example,
the \textsc{beamer} package uses the basic layer extensively, but does
not need a convenient input syntax. Rather, speed and flexibility are
needed when \textsc{beamer} creates graphics.

The following basic design principles underlie the basic layer:
\begin{enumerate}
\item Structuring into a core and modules.
\item Consistently named \TeX\ macros for all graphics commands.
\item Path-centered description of graphics.
\item Coordinate transformation system.
\end{enumerate}



\subsection{Core and Modules}

The basic layer consists of a \emph{core package}, called |pgfcore|,
which provides the most basic commands, and several
\emph{modules} like commands for plotting (in the |plot| module).
Modules are loaded using the |\usepgfmodule| command. 

If you say |\usepackage{pgf}| or |\input pgf.tex| or
|\usemodule[pgf]|, the |plot| and |shapes| modules are preloaded (as
well as the core and the system layer).



\subsection{Communicating with the Basic Layer via Macros}

In order to ``communicate'' with the basic layer you use long
sequences of commands that start with |\pgf|. You are only allowed to
give these commands inside a |{pgfpicture}| environment. (Note that
|{tikzpicture}| opens a |{pgfpicture}| internally, so you can freely
mix \pgfname\ commands and \tikzname\ commands inside a
|{tikzpicture}|.) It is possible to ``do other things'' between the
commands. For example, you might use one command to move to a certain
point, then have a complicated computation of the next point, and then
move there. 

\begin{codeexample}[]
\newdimen\myypos
\begin{pgfpicture}
  \pgfpathmoveto{\pgfpoint{0cm}{\myypos}}
  \pgfpathlineto{\pgfpoint{1cm}{\myypos}}
  \advance \myypos by 1cm
  \pgfpathlineto{\pgfpoint{1cm}{\myypos}}
  \pgfpathclose
  \pgfusepath{stroke}
\end{pgfpicture}
\end{codeexample}

The following naming conventions are used in the basic layer:

\begin{enumerate}
\item
  All commands and environments start with |pgf|.
\item
  All commands that specify a point (a coordinate) start with |\pgfpoint|.
\item
  All commands that extend the current path start with |\pgfpath|.
\item
  All commands that set/change a graphics parameter start with |\pgfset|.
\item
  All commands that use a previously declared object (like a path,
  image or shading) start with |\pgfuse|.
\item
  All commands having to do with coordinate transformations start with
  |\pgftransform|. 
\item
  All commands having to do with arrow tips start with |\pgfarrows|.
\item
  All commands for ``quickly'' extending or drawing a path start with
  |\pgfpathq| or |\pgfusepathq|.
\item
  All commands having to do with matrices start with |\pgfmatrix|.
\end{enumerate}


\subsection{Path-Centered Approach}

In \pgfname\ the most important entity is the \emph{path}. All
graphics are composed of numerous paths that can be stroked,
filled, shaded, or clipped against. Paths can be closed or open, they
can self-intersect and consist of unconnected parts.

Paths are first \emph{constructed} and then \emph{used}. In order to
construct a path, you can use commands starting with |\pgfpath|. Each
time such a command is called, the current path is extended in some
way.

Once a path has been completely constructed, you can use it using the
command |\pgfusepath|. Depending on the parameters given to this
command, the path will be stroked (drawn) or filled or subsequent
drawings will be clipped against this path.




\subsection{Coordinate Versus Canvas Transformations}

\label{section-design-transformations}

\pgfname\ provides two transformation systems: \pgfname's own
\emph{coordinate} transformation matrix and \pdf's or PostScript's
\emph{canvas} transformation matrix. These two systems are quite
different. Whereas a scaling by a factor of, say, $2$ of the canvas
causes \emph{everything} to be scaled by this factor (including
the thickness of lines and text), a scaling of two in the coordinate 
system causes only the \emph{coordinates} to be scaled, but not the
line width nor text.

By default, all transformations only apply to the coordinate
transformation system. However, using the command |\pgflowlevel|
it is possible to apply a transformation to the canvas.

Coordinate transformations are often preferable over canvas
transformations. Text and lines that are transformed using canvas 
transformations suffer from differing sizes and lines whose thickness 
differs depending on whether the line is horizontal or vertical. To
appreciate the difference, consider the following two ``circles'' both
of which have been scaled in the $x$-direction by a factor of $3$ and
by a factor of $0.5$ in the $y$-direction. The left circle uses a
canvas transformation, the right uses \pgfname's coordinate
transformation (some viewers will render the left graphic incorrectly
since they do no apply the low-level transformation the way they
should):  

\begin{tikzpicture}[line width=5pt]
  \useasboundingbox (-1.75,-1) rectangle (14,1);
  
  \begin{scope}
    \pgflowlevel{\pgftransformxscale{3}}
    \pgflowlevel{\pgftransformyscale{.5}}

    \draw (0,0) circle (0.5cm);
    \draw (.55cm,0pt) node[right] {canvas};
  \end{scope}
  \begin{scope}[xshift=9cm,xscale=3,yscale=.5]
    \draw (0,0) circle (0.5cm);
    \draw (.55cm,0pt) node[right] {coordinate};
  \end{scope}
\end{tikzpicture}


% Copyright 2006 by Till Tantau
%
% This file may be distributed and/or modified
%
% 1. under the LaTeX Project Public License and/or
% 2. under the GNU Free Documentation License.
%
% See the file doc/generic/pgf/licenses/LICENSE for more details.


\section[Hierarchical Structures: Package, Environments, Scopes, and Text]
{Hierarchical Structures:\\
  Package, Environments, Scopes, and Text}


\subsection{Overview}

\pgfname\ uses two kinds of hierarchical structuring: First, the
package itself is structured hierarchically, consisting of different
packages that are built on top of each other. Second, \pgfname\ allows you
to structure your graphics hierarchically using environments and scopes.

\subsubsection{The  Hierarchical Structure of the Package}

The \pgfname\ system consists of several layers:

\begin{description}
\item[System layer.]
  The lowest layer is called the \emph{system layer}, though it might
  also be called ``driver layer'' or perhaps ``backend layer.'' Its
  job is to provide an abstraction of the details of which driver
  is used to transform the |.dvi| file. The system layer is
  implemented by the package |pgfsys|, which will load appropriate
  driver files as needed.

  The system layer is documented in Part~\ref{part-system}.
\item[Basic layer.]
  The basic layer is loaded by the package |pgfcore| and subsequent
  use of the command |\usepgfmodule| to load additional modules of the
  basic layer.

  The basic layer is documented in the present part.
\item[Frontend layer.]
  The frontend layer is not loaded by a single package. Rather,
  different packages, like \tikzname\ or \textsc{pgfpict2e}, are
  different frontends to the basic layer.

  The \tikzname\ frontend is documented in Part~\ref{part-tikz}.
\end{description}

Each layer will automatically load the necessary files of the layers below
it.

In addition to the packages of these layers, there are also some
library packages. These packages provide additional definitions of
things like new arrow tips or new plot handlers.

The library packages are documented in Part~\ref{part-libraries}.




\subsubsection{The Hierarchical Structure of Graphics}

Graphics in \pgfname\ are typically structured
hierarchically. Hierarchical structuring can be used to identify
groups of graphical elements that are to be treated ``in the same
way.'' For example, you might group together a number of paths, all of
which are to be drawn in red. Then, when you decide later on that you
like them to be drawn in, say, blue, all you have to do is to change
the color once.

The general mechanism underlying hierarchical structuring is known as
\emph{scoping} in computer science. The idea is that all changes to
the general ``state'' of the graphic that are done inside a scope are
local to that scope. So, if you change the color inside a scope, this
does not affect the color used outside the scope. Likewise, when you
change the line width in a scope, the line width outside is not
changed, and so on.

There are different ways of starting and ending scopes of graphic
parameters. Unfortunately, these scopes are sometimes ``in conflict''
with each other and it is sometimes not immediately clear which scopes
apply. In essence, the following scoping mechanisms are available:

\begin{enumerate}
\item
  The ``outermost'' scope supported by \pgfname\ is the |{pgfpicture}|
  environment. All changes to the graphic state done inside a
  |{pgfpicture}| are local to that picture.

  In general, it is \emph{not} possible to set graphic parameters
  globally outside any |{pgfpicture}| environments. Thus, you can
  \emph{not} say |\pgfsetlinewidth{1pt}| at the beginning of your
  document to have a default line width of one point. Rather, you have
  to (re)set all graphic parameters inside each |{pgfpicture}|. (If
  this is too bothersome, try defining some macro that does the job
  for you.)
\item
  Inside a |{pgfpicture}| you can use a |{pgfscope}| environment to
  keep changes of the graphic state local to that environment.

  The effect of commands that change the graphic state are local to
  the current |{pgfscope}|, but not always to the current \TeX\
  group. Thus, if you open a \TeX\ group (some text in curly braces)
  inside a |{pgfscope}|, and if you change, for example, the dash
  pattern, the effect of this changed dash pattern will persist till
  the end of the |{pgfscope}|.

  Unfortunately, this is not always the case. \emph{Some} graphic
  parameters only persist till the end of the current \TeX\ group. For
  example, when you use |\pgfsetarrows| to set the arrow tip
  inside a \TeX\ group, the effect lasts only till the end of the
  current \TeX\ group.
\item
  Some graphic parameters are not scoped by |{pgfscope}| but
  ``already'' by \TeX\ groups. For example, the effect of coordinate
  transformation commands is always local to the current \TeX\
  group.

  Since every |{pgfscope}| automatically creates a \TeX\ group, all
  graphic parameters that are local to the current \TeX\ group are
  also local to the current |{pgfscope}|.
\item
  Some graphic parameters can only be scoped using \TeX\ groups, since
  in some situations it is not possible to introduce a
  |{pgfscope}|. For example, a path always has to be completely
  constructed and used in the same |{pgfscope}|. However, we might
  wish to have different coordinate transformations apply to different
  points on the path. In this case, we can use \TeX\ groups to keep
  the effect local, but we could not use |{pgfscope}|.
\item
  The |\pgftext| command can be used to create a scope in which \TeX\
  ``escapes back'' to normal \TeX\ mode. The text passed to the
  |\pgftext| is ``heavily guarded'' against having any effect on the
  scope in which it is used. For example, it is possible to use
  another  |{pgfpicture}| environment inside the argument of
  |\pgftext|.
\end{enumerate}


Most of the complications can be avoided if you stick to the following
rules:

\begin{itemize}
\item
  Give graphic commands only inside |{pgfpicture}| environments.
\item
  Use |{pgfscope}| to structure graphics.
\item
  Do not use \TeX\ groups inside graphics, \emph{except} for keeping
  the effect of coordinate transformations local.
\end{itemize}




\subsection{The Hierarchical Structure of the Package}

Before we come to the structuring commands provided by \pgfname\ to
structure your graphics, let us first have a look at the structure of
the package itself.

\subsubsection{The Core Package}

To use \pgfname, include the following package:

\begin{package}{pgfcore}
  This package loads the complete core of the ``basic layer'' of
  \pgfname, but not any modules. That is, it will load all of the
  commands described in the current part of this manual, but it will
  not load frontends like \tikzname. It will also load the system
  layer. To load additional modules, use the |\usepgfmodule| command
  explained below.
\end{package}

The following package is just a convenience.

\begin{package}{pgf}
  This package loads the |pgfcore| and the two modules |shapes| and
  |plot|.

  In \LaTeX, the package takes two options:
  \begin{packageoption}{draft}
    When this option is set, all images will be replaced by empty
    rectangles. This can speedup compilation.
  \end{packageoption}

  \begin{packageoption}{version=\meta{version}}
    Indicates that the commands of version \meta{version} need to be
    defined. If you set \meta{version} to |0.65|, then a large bunch of
    ``compatibility commands'' are loaded. If you set \meta{version} to
    |0.96|, then these compatibility commands will not be loaded.

    If this option is not given at all, then the commands of all
    versions are defined.
  \end{packageoption}
\end{package}



\subsubsection{The Modules}

\begin{command}{\usepgflibrary\marg{module names}}
  Once the core has been loaded, you can use this command to load
  further modules. The modules in the \meta{module names} list should
  be separated by commas. Instead of curly braces, you can also
  use square brackets, which is something Con\TeX t users will
  like. If you try to load a module a second time, nothing will
  happen.

  \example |\usepgfmodule{matrix,shapes}|

  What this command does is to load the file
  |pgfmodule|\meta{module}|.code.tex| for each \meta{module} in
  the list of \meta{module names}. Thus, to write your own module,
  all you need to do is to place a file of the appropriate name
  somewhere \TeX\ can find it. \LaTeX, plain \TeX, and Con\TeX t
  users can then use your library.
\end{command}

The following modules are available for use with |pgfcore|:

\begin{itemize}
\item The |plot| module provides commands for plotting functions.  The
  commands are explained in Section~\ref{section-plots}.
\item The |shapes| module provides commands for drawing shapes and
  nodes. These commands are explained in
  Section~\ref{section-shapes}.
\item The |decorations| module provides commands for adding
  decorations to paths. These commands are explained in
  Section~\ref{section-base-decorations}.
\item The |matrix| module provides the |\pgfmatrix| command. The
  commands are documented in Section~\ref{section-base-matrices}.
\end{itemize}



\subsubsection{The Library Packages}

There is a special command for loading library packages. The
difference between a library and module is the following: A library
just defines additional objects using the basic layer, whereas a
module adds completely new functionality. For instance, a decoration
library defines additional decorations, while a decoration module
defines the whole code for handling decorations.

\begin{command}{\usepgflibrary\marg{list of libraries}}
  Use this command to load further libraries. The list of libraries
  should contain the names of libraries separated by commas. Instead
  of curly braces, you can also use square brackets. If you try to
  load a library a second time, nothing will happen.

  \example |\usepgflibrary{arrows}|

  This command causes the file
  |pgflibrary|\meta{library}|.code.tex| to be loaded for each \meta{library} in
  the \meta{list of libraries}. This means that in order to write your
  own library file, place a file of the appropriate name somewhere
  where \TeX\ can find it. \LaTeX, plain \TeX, and Con\TeX t
  users can then use your library.

  You should also consider adding a \tikzname\ library that simply
  includes your \pgfname\ library.
\end{command}



\subsection{The Hierarchical Structure of the Graphics}

\subsubsection{The Main Environment}


Most, but not all, commands of the \pgfname\ package must be given
within a |{pgfpicture}| environment. The only commands that (must) be
given outside are commands having to do with including images (like
|\pgfuseimage|) and with inserting complete shadings (like
|\pgfuseshading|). However, just to keep life entertaining, the
|\pgfshadepath| command must be given \emph{inside} a |{pgfpicture}|
environment.

\begin{environment}{{pgfpicture}}
  This environment will insert a \TeX\ box containing the graphic drawn by
  the \meta{environment contents} at the current position.

  \medskip
  \textbf{The size of the bounding box.}
  The size of the box is determined in the following
  manner: While \pgfname\ parses the \meta{environment contents}, it
  keeps track of a bounding box for the graphic. Essentially, this
  bounding box is the smallest box that contains all coordinates
  mentioned in the graphics. Some coordinates may be ``mentioned'' by
  \pgfname\ itself; for example, when you add circle to the current
  path, the support points of the curve making up the circle are also
  ``mentioned'' despite the fact that you will not ``see'' them in
  your code.

  Once the \meta{environment contents} have been parsed completely, a
  \TeX\ box is created whose size is the size of the computed bounding
  box and this box is inserted at the current position.

\begin{codeexample}[]
Hello \begin{pgfpicture}
  \pgfpathrectangle{\pgfpointorigin}{\pgfpoint{2ex}{1ex}}
  \pgfusepath{stroke}
\end{pgfpicture} World!
\end{codeexample}

  Sometimes, you may need more fine-grained control over the size of
  the bounding box. For example, the computed bounding box may be too
  large or you intensionally wish the box to be ``too small.'' In
  these cases, you can use the command
  |\pgfusepath{use as bounding box}|, as described in
  Section~\ref{section-using-bb}.


  \medskip
  \textbf{The baseline of the bounding box.}
  When the box containing the graphic is inserted into the normal
  text, the baseline of the graphic is normally at the bottom of the
  graphic. For this reason, the following two sets of code lines have
  the same effect, despite the fact that the second graphic uses
  ``higher'' coordinates than the first:

\begin{codeexample}[]
Rectangles \begin{pgfpicture}
  \pgfpathrectangle{\pgfpointorigin}{\pgfpoint{2ex}{1ex}}
  \pgfusepath{stroke}
\end{pgfpicture} and \begin{pgfpicture}
  \pgfpathrectangle{\pgfpoint{0ex}{1ex}}{\pgfpoint{2ex}{1ex}}
  \pgfusepath{stroke}
\end{pgfpicture}.
\end{codeexample}

  You can change the baseline using the |\pgfsetbaseline| command, see
  below.

\begin{codeexample}[]
Rectangles \begin{pgfpicture}
  \pgfpathrectangle{\pgfpointorigin}{\pgfpoint{2ex}{1ex}}
  \pgfusepath{stroke}
  \pgfsetbaseline{0pt}
\end{pgfpicture} and \begin{pgfpicture}
  \pgfpathrectangle{\pgfpoint{0ex}{1ex}}{\pgfpoint{2ex}{1ex}}
  \pgfusepath{stroke}
  \pgfsetbaseline{0pt}
\end{pgfpicture}.
\end{codeexample}

  \medskip
  \textbf{Including text and images in a picture.}
  You cannot directly include text and images in a picture. Thus, you
  should \emph{not} simply write some text in a |{pgfpicture}| or use
  a command like |\includegraphics| or even |\pgfimage|. In all these
  cases, you need to place the text inside a |\pgftext| command. This
  will ``escape back'' to normal \TeX\ mode, see
  Section~\ref{section-text-command} for details.

  \medskip
  \textbf{Remembering a picture position for later reference.}
  After a picture has been typeset, its position on the page is
  normally forgotten by \pgfname\ and also by \TeX. This means that is
  not possible to reference a node in this picture later on. In
  particular, it is normally impossible to draw lines between nodes in
  different pictures automatically.

  In order to make \pgfname\ ``remember'' a picture, the \TeX-if
  |\||ifpgfrememberpicturepositiononpage| should be set to |true|. It
  is only important that this \TeX-if is |true| at the end of the
  |{pgfpicture}|-environment, so you can switch it on inside the
  environment. However, you can also just switch it on globally, then
  the positions of all pictures are remembered.

  There are several reasons why the remembering is not switched on by
  default. First, it does not work for all backend drivers (currently, it
  works only for pdf\TeX). Second, it requires two passes of \TeX\
  over the file; on the first pass all positions will be wrong. Third,
  for every remembered picture a line is added to the |.aux|-file,
  which may result in a large number of extra lines.

  Despite all these ``problems,'' for documents that are processed
  with pdf\TeX\ and in which there is only a small number of pictures
  (less than a hundred or so), you can switch on this option globally,
  it will not cause any significant slowing of \TeX.
\end{environment}

\begin{plainenvironment}{{pgfpicture}}
  The plain \TeX\ version of the environment. Note that in this
  version, also, a \TeX\ group is created around the environment.
\end{plainenvironment}

\begin{contextenvironment}{{pgfpicture}}
  This is the Con\TeX t version of the environment.
\end{contextenvironment}


{\let\ifpgfrememberpicturepositiononpage=\relax
\begin{command}{\ifpgfrememberpicturepositiononpage}
  Determines whether the position of pictures on the page should be
  recorded. The value of this \TeX-if at the end of a |{pgfpicture}|
  environment is important, not the value at the beginning.

  If this option is set to true of a picture, \pgfname\ will attempt
  to record the position of the picture on the page. (This attempt
  will fail with most drivers and when it works, it typically requires
  two runs of \TeX.) The position is not directly accessible. Rather,
  the nodes mechanism will use this position if you access a node from
  another picture. See Sections~\ref{section-cross-pictures-pgf}
  and~\ref{section-cross-picture-tikz} for more details.
\end{command}
}


\makeatletter
\begin{command}{\pgfsetbaseline\marg{dimension}}
  This command specifies a $y$-coordinate of the picture that should
  be used as the baseline of the whole picture. When a \pgfname\
  picture has been typeset completely, \pgfname\ must decide at which
  height the baseline of the picture should lie. Normally, the
  baseline is set to the $y$-coordinate of the bottom of the picture,
  but it is often desirable to use a different height.

\begin{codeexample}[]
Text \tikz{\pgfpathcircle{\pgfpointorigin}{1ex}\pgfusepath{stroke}},
     \tikz{\pgfsetbaseline{0pt}
          \pgfpathcircle{\pgfpointorigin}{1ex}\pgfusepath{stroke}},
     \tikz{\pgfsetbaseline{.5ex}
          \pgfpathcircle{\pgfpointorigin}{1ex}\pgfusepath{stroke}},
     \tikz{\pgfsetbaseline{-1ex}
          \pgfpathcircle{\pgfpointorigin}{1ex}\pgfusepath{stroke}}.
\end{codeexample}
\end{command}

\begin{command}{\pgfsetbaselinepointnow\marg{point}}
  This command specifies the baseline indirectly, namely as the
  $y$-coordinate that the given \meta{point} has when the command is
  called.
\end{command}

\begin{command}{\pgfsetbaselinepointlater\marg{point}}
  This command also specifies the baseline indirectly, but the
  $y$-coordinate of the given \meta{point} is only computed at the end
  of the picture.

\begin{codeexample}[]
Hello
\tikz{
  \pgfsetbaselinepointlater{\pgfpointanchor{X}{base}}
  % Note: no shape X, yet
  \node [cross out,draw] (X) {world.};
}
\end{codeexample}
\end{command}



\subsubsection{Graphic Scope Environments}

Inside a |{pgfpicture}| environment you can substructure your picture
using the following environment:

\begin{environment}{{pgfscope}}
  All changes to the graphic state done inside this environment are
  local to the environment. The graphic state includes the following:
  \begin{itemize}
  \item
    The line width.
  \item
    The stroke and fill colors.
  \item
    The dash pattern.
  \item
    The line join and cap.
  \item
    The miter limit.
  \item
    The canvas transformation matrix.
  \item
    The clipping path.
  \end{itemize}
  Other parameters may also influence how graphics are rendered, but they
  are \emph{not} part of the graphic state. For example, the arrow tip
  kind is not part of the graphic state and the effect of commands
  setting the arrow tip kind are local to the current \TeX\ group, not
  to the current |{pgfscope}|. However, since |{pgfscope}| starts and
  ends a \TeX\ group automatically, a |{pgfscope}| can be used to
  limit the effect of, say, commands that set the arrow tip kind.

\begin{codeexample}[]
\begin{pgfpicture}
  \begin{pgfscope}
    {
      \pgfsetlinewidth{2pt}
      \pgfpathrectangle{\pgfpointorigin}{\pgfpoint{2ex}{2ex}}
      \pgfusepath{stroke}
    }
    \pgfpathrectangle{\pgfpoint{3ex}{0ex}}{\pgfpoint{2ex}{2ex}}
    \pgfusepath{stroke}
  \end{pgfscope}
  \pgfpathrectangle{\pgfpoint{6ex}{0ex}}{\pgfpoint{2ex}{2ex}}
  \pgfusepath{stroke}
\end{pgfpicture}
\end{codeexample}

\begin{codeexample}[]
\begin{pgfpicture}
  \begin{pgfscope}
    {
      \pgfsetarrows{-to}
      \pgfpathmoveto{\pgfpointorigin}\pgfpathlineto{\pgfpoint{2ex}{2ex}}
      \pgfusepath{stroke}
    }
    \pgfpathmoveto{\pgfpoint{3ex}{0ex}}\pgfpathlineto{\pgfpoint{5ex}{2ex}}
    \pgfusepath{stroke}
  \end{pgfscope}
  \pgfpathmoveto{\pgfpoint{6ex}{0ex}}\pgfpathlineto{\pgfpoint{8ex}{2ex}}
  \pgfusepath{stroke}
\end{pgfpicture}
\end{codeexample}

  At the start of the scope, the current path must be empty, that is,
  you cannot open a scope while constructing a path.

  It is usually a good idea \emph{not} to introduce \TeX\ groups
  inside a |{pgfscope}| environment.
\end{environment}

\begin{plainenvironment}{{pgfscope}}
  Plain \TeX\ version of the |{pgfscope}| environment.
\end{plainenvironment}

\begin{contextenvironment}{{pgfscope}}
  This is the Con\TeX t version of the environment.
\end{contextenvironment}


The following scopes also encapsulate certain properties of the
graphic state. However, they are typically not used directly by the
user.

\begin{environment}{{pgfinterruptpath}}
  This environment can be used to temporarily interrupt the
  construction of the current path. The effect will be that the path
  currently under construction will be ``stored away'' and restored at
  the end of the environment. Inside the environment you can construct
  a new path and do something with it.

  An example application of this environment is the arrow tip
  caching. Suppose you ask \pgfname\ to use a specific arrow tip
  kind. When the arrow tip needs to be rendered for the first time,
  \pgfname\ will ``cache'' the path that makes up the arrow tip. To do
  so, it interrupts the current path construction and then protocols
  the path of the arrow tip. The |{pgfinterruptpath}| environment is
  used to ensure that this does not interfere with the path to which
  the arrow tips should be attached.

  This command does \emph{not} install a |{pgfscope}|. In particular,
  it does not call any |\pgfsys@| commands at all, which would,
  indeed, be dangerous in the middle of a path construction.
\end{environment}

\begin{plainenvironment}{{pgfinterruptpath}}
  Plain \TeX\ version of the environment.
\end{plainenvironment}

\begin{contextenvironment}{{pgfinterruptpath}}
  Con\TeX t version of the environment.
\end{contextenvironment}


\begin{environment}{{pgfinterruptpicture}}
  This environment can be used to temporarily interrupt a
  |{pgfpicture}|. However, the environment is intended only to be used
  at the beginning and end of a box that is (later) inserted into a
  |{pgfpicture}| using |\pgfqbox|. You cannot use this environment
  directly inside a |{pgfpicture}|.

\begin{codeexample}[]
\begin{pgfpicture}
  \pgfpathmoveto{\pgfpoint{0cm}{0cm}} % In the middle of path, now
  \newbox\mybox
  \setbox\mybox=\hbox{
    \begin{pgfinterruptpicture}
      Sub-\begin{pgfpicture} % a subpicture
        \pgfpathmoveto{\pgfpoint{1cm}{0cm}}
        \pgfpathlineto{\pgfpoint{1cm}{1cm}}
        \pgfusepath{stroke}
      \end{pgfpicture}-picture.
    \end{pgfinterruptpicture}
  }
  \pgfqbox{\mybox}%
  \pgfpathlineto{\pgfpoint{0cm}{1cm}}
  \pgfusepath{stroke}
\end{pgfpicture}\hskip3.9cm
\end{codeexample}
\end{environment}

\begin{plainenvironment}{{pgfinterruptpicture}}
  Plain \TeX\ version of the environment.
\end{plainenvironment}

\begin{contextenvironment}{{pgfinterruptpicture}}
  Con\TeX t version of the environment.
\end{contextenvironment}



\begin{environment}{{pgfinterruptboundingbox}}
  This environment temporarily interrupts the computation of the
  bounding box and sets up a new bounding box. At the beginning of the
  environment the old bounding box is saved and an empty bounding box
  is installed. After the environment the original bounding box is
  reinstalled as if nothing has happened.
\end{environment}

\begin{plainenvironment}{{pgfinterruptboundingbox}}
  Plain \TeX\ version of the environment.
\end{plainenvironment}

\begin{contextenvironment}{{pgfinterruptboundingbox}}
  Con\TeX t version of the environment.
\end{contextenvironment}


\subsubsection{Inserting Text and Images}

\label{section-text-command}

Often, you may wish to add normal \TeX\ text at a certain point inside
a |{pgfpicture}|. You cannot do so ``directly,'' that is, you cannot
simply write this text inside the |{pgfpicture}| environment. Rather,
you must pass the text as an argument to the |\pgftext| command.

You must \emph{also} use the |\pgftext| command to insert an image or
a shading into a |{pgfpicture}|.

\begin{command}{\pgftext\opt{\oarg{options}}\marg{text}}
  This command will typeset \meta{text} in normal \TeX\ mode and
  insert the resulting box into the |{pgfpicture}|. The bounding box
  of the graphic will be updated so that all of the text box is
  inside. By default, the text box is centered at the origin, but this
  can be changed either by giving appropriate \meta{options} or by
  applying an appropriate coordinate transformation beforehand.

  The \meta{text} may contain verbatim text. (In other words, the
  \meta{text} ``argument'' is not a normal argument, but is put in a
  box and some |\aftergroup| hackery is used to find the end of the
  box.)

  \pgfname's current (high-level) coordinate transformation is
  synchronized with the canvas transformation matrix temporarily
  when the text box is inserted. The effect is that if there is
  currently a high-level rotation of, say, 30 degrees, the \meta{text}
  will also be rotated by thirty degrees. If you do not want this
  effect, you have to (possibly temporarily) reset the high-level
  transformation matrix.

  The \meta{options} keys are used with the path |/pgf/text/|. The
  following keys are defined for this path:
  \begin{key}{/pgf/text/left}
    The key causes the text box to be placed such that its left
    border is on the origin.
\begin{codeexample}[]
\tikz{\draw[help lines] (-1,-.5) grid (1,.5);
     \pgftext[left] {lovely}}
\end{codeexample}
  \end{key}
  \begin{key}{/pgf/text/right}
    The key causes the text box to be placed such that its right
    border is on the origin.
\begin{codeexample}[]
\tikz{\draw[help lines] (-1,-.5) grid (1,.5);
     \pgftext[right] {lovely}}
\end{codeexample}
  \end{key}
  \begin{key}{/pgf/text/right}
    This key causes the text box to be placed such that its top is on the
    origin. This option can be used together with the |left| or
    |right| option.
\begin{codeexample}[]
\tikz{\draw[help lines] (-1,-.5) grid (1,.5);
     \pgftext[top] {lovely}}
\end{codeexample}
\begin{codeexample}[]
\tikz{\draw[help lines] (-1,-.5) grid (1,.5);
     \pgftext[top,right] {lovely}}
\end{codeexample}
  \end{key}
  \begin{key}{/pgf/text/bottom}
    This key causes the text box to be placed such that its bottom is
    on the origin.
\begin{codeexample}[]
\tikz{\draw[help lines] (-1,-.5) grid (1,.5);
     \pgftext[bottom] {lovely}}
\end{codeexample}
\begin{codeexample}[]
\tikz{\draw[help lines] (-1,-.5) grid (1,.5);
     \pgftext[bottom,right] {lovely}}
\end{codeexample}
  \end{key}
  \begin{key}{/pgf/text/base}
    This key causes the text box to be placed such that its baseline
    is on the origin.
\begin{codeexample}[]
\tikz{\draw[help lines] (-1,-.5) grid (1,.5);
     \pgftext[base] {lovely}}
\end{codeexample}
\begin{codeexample}[]
\tikz{\draw[help lines] (-1,-.5) grid (1,.5);
     \pgftext[base,right] {lovely}}
\end{codeexample}
  \end{key}
  \begin{key}{/pgf/text/at=\meta{point}}
    Translates the origin (that is, the point where the text is
    shown) to \meta{point}.
\begin{codeexample}[]
\tikz{\draw[help lines] (-1,-.5) grid (1,.5);
     \pgftext[base,at={\pgfpoint{1cm}{0cm}}] {lovely}}
\end{codeexample}
  \end{key}
  \begin{key}{/pgf/text/x=\meta{dimension}}
    Translates the origin by \meta{dimension} along the $x$-axis.
\begin{codeexample}[]
\tikz{\draw[help lines] (-1,-.5) grid (1,.5);
     \pgftext[base,x=1cm,y=-0.5cm] {lovely}}
\end{codeexample}
  \end{key}
  \begin{key}{/pgf/text/y=\meta{dimension}}
    This key works like the |x| option.
  \end{key}
  \begin{key}{/pgf/text/rotate=\meta{degree}}
    Rotates the coordinate system by \meta{degree}. This will also
    rotate the text box.
\begin{codeexample}[]
\tikz{\draw[help lines] (-1,-.5) grid (1,.5);
     \pgftext[base,x=1cm,y=-0.5cm,rotate=30] {lovely}}
\end{codeexample}
\end{key}

\end{command}


\subsection{Error Messages and Warnings}

Sometimes, a command inside \pgfname\ may fail. In this case, two
commands are useful to communicate with the author:

\begin{command}{\pgferror\marg{message}}
  Stops the processing of the current document and prints out the
  \meta{message}. In \LaTeX, this will be done using |\PackageError|,
  otherwise |\errmessage| is used directly.  
\end{command}

\begin{command}{\pgfwarning\marg{message}}
  Prints the \meta{message} on the output, but does not interrrupt the
  processing. In \LaTeX, this will be done using |\PackageWarning|,
  otherwise a write to stream $17$ is used.
\end{command}

% Copyright 2006 by Till Tantau
%
% This file may be distributed and/or modified
%
% 1. under the LaTeX Project Public License and/or
% 2. under the GNU Free Documentation License.
%
% See the file doc/generic/pgf/licenses/LICENSE for more details.


\section{Specifying Coordinates}

\label{section-points}

\subsection{Overview}

Most \pgfname\ commands expect you to provide the coordinates of a
\emph{point} (also called \emph{coordinate}) inside your
picture. Points are always ``local'' to your picture, that is, they
never refer to an absolute position on the page, but to a position
inside the current |{pgfpicture}| environment. To specify a coordinate
you can use commands that start with |\pgfpoint|.


\subsection{Basic Coordinate Commands}

The following commands are the most basic  for specifying a
coordinate.

\begin{command}{\pgfpoint\marg{x coordinate}\marg{y coordinate}}
  Yields a point location. The coordinates are given as \TeX\
  dimensions.

\begin{codeexample}[]
\begin{tikzpicture}
  \draw[help lines] (0,0) grid (3,2);
  \pgfpathcircle{\pgfpoint{1cm}{1cm}} {2pt}
  \pgfpathcircle{\pgfpoint{2cm}{5pt}} {2pt}
  \pgfpathcircle{\pgfpoint{0pt}{.5in}}{2pt}
  \pgfusepath{fill}
\end{tikzpicture}   
\end{codeexample}
\end{command}

\begin{command}{\pgfpointorigin}
  Yields the origin. Same as |\pgfpoint{0pt}{0pt}|.
\end{command}

\begin{command}{\pgfpointpolar\marg{degree}{\ttfamily\char`\{}\meta{radius}\opt{|/|\meta{y-radius}}{\ttfamily\char`\}}}
  Yields a point location given in polar coordinates. You can specify
  the angle only in degrees, radians are not supported, currently.

  If the optional \meta{y-radius} is given, the polar coordinate is
  actually a coordinate on an ellipse whose $x$-radius is given by
  \meta{radius} and whose $y$-radius is given by \meta{y-radius}.
\begin{codeexample}[]
\begin{tikzpicture}
  \draw[help lines] (0,0) grid (3,2);

  \foreach \angle in {0,10,...,90}
    {\pgfpathcircle{\pgfpointpolar{\angle}{1cm}}{2pt}}
  \pgfusepath{fill}
\end{tikzpicture}
\end{codeexample}
\begin{codeexample}[]
\begin{tikzpicture}
  \draw[help lines] (0,0) grid (3,2);

  \foreach \angle in {0,10,...,90}
    {\pgfpathcircle{\pgfpointpolar{\angle}{1cm/2cm}}{2pt}}
  \pgfusepath{fill}
\end{tikzpicture}   
\end{codeexample}
\end{command}



\subsection{Coordinates in the XY-Coordinate System}

Coordinates can also be specified as multiples of an $x$-vector and a
$y$-vector. Normally, the $x$-vector points one centimeter in the
$x$-direction and the $y$-vector points one centimeter in the
$y$-direction, but using the commands |\pgfsetxvec| and
|\pgfsetyvec| they can be changed. Note that the $x$- and
$y$-vector do not necessarily point ``horizontally'' and
``vertically.''

\begin{command}{\pgfpointxy\marg{$s_x$}\marg{$s_y$}}
  Yields a point that is situated at $s_x$ times the
  $x$-vector plus $s_y$ times the $y$-vector.
\begin{codeexample}[]
\begin{tikzpicture}
  \draw[help lines] (0,0) grid (3,2);
  \pgfpathmoveto{\pgfpointxy{1}{0}}
  \pgfpathlineto{\pgfpointxy{2}{2}}
  \pgfusepath{stroke}
\end{tikzpicture}   
\end{codeexample}
\end{command}


\begin{command}{\pgfsetxvec\marg{point}}
  Sets that current $x$-vector for usage in the $xyz$-coordinate
  system. 
  \example
\begin{codeexample}[]
\begin{tikzpicture}
  \draw[help lines] (0,0) grid (3,2);
  
  \pgfpathmoveto{\pgfpointxy{1}{0}}
  \pgfpathlineto{\pgfpointxy{2}{2}}
  \pgfusepath{stroke}

  \color{red}
  \pgfsetxvec{\pgfpoint{0.75cm}{0cm}}
  \pgfpathmoveto{\pgfpointxy{1}{0}}
  \pgfpathlineto{\pgfpointxy{2}{2}}
  \pgfusepath{stroke}
\end{tikzpicture}   
\end{codeexample}
\end{command}

\begin{command}{\pgfsetyvec\marg{point}}
  Works like |\pgfsetyvec|.
\end{command}



\begin{command}{\pgfpointpolarxy\marg{degree}{\ttfamily\char`\{}\meta{radius}\opt{|/|\meta{y-radius}}{\ttfamily\char`\}}}
  This command is similar to the |\pgfpointpolar| command, but the
  \meta{radius} is now a factor to be interpreted in the
  $xy$-coordinate system. This means that a degree of |0| is the same
  as the $x$-vector of the $xy$-coordinate  system times \meta{radius}
  and a degree of |90| is the $y$-vecotr times \meta{radius}. As for
  |\pgfpointpolar|, a \meta{radius} can also be a pair separated by a
  slash. In this case, the $x$- and $y$-vectors are multiplied by
  different factors.
\begin{codeexample}[]
\begin{tikzpicture}
  \draw[help lines] (0,0) grid (3,2);

  \begin{scope}[x={(1cm,-5mm)},y=1.5cm]
    \foreach \angle in {0,10,...,90}
      {\pgfpathcircle{\pgfpointpolarxy{\angle}{1}}{2pt}}
    \pgfusepath{fill}
  \end{scope}
\end{tikzpicture}
\end{codeexample}
\end{command}



\subsection{Three Dimensional Coordinates}

It is also possible to specify a point as a multiple of three vectors,
the $x$-, $y$-, and $z$-vector. This is useful for creating simple
three dimensional graphics.

\begin{command}{\pgfpointxyz\marg{$s_x$}\marg{$s_y$}\marg{$s_z$}}
  Yields a point that is situated at $s_x$ times the
  $x$-vector plus $s_y$ times the $y$-vector plus  $s_z$ times the
  $z$-vector.
\begin{codeexample}[]
\begin{pgfpicture}
  \pgfsetarrowsend{to}
  
  \pgfpathmoveto{\pgfpointorigin}
  \pgfpathlineto{\pgfpointxyz{0}{0}{1}}
  \pgfusepath{stroke}
  \pgfpathmoveto{\pgfpointorigin}
  \pgfpathlineto{\pgfpointxyz{0}{1}{0}}
  \pgfusepath{stroke}
  \pgfpathmoveto{\pgfpointorigin}
  \pgfpathlineto{\pgfpointxyz{1}{0}{0}}
  \pgfusepath{stroke}
\end{pgfpicture}
\end{codeexample}
\end{command}

\begin{command}{\pgfsetzvec\marg{point}}
  Works like |\pgfsetzvec|.
\end{command}

Inside the $xyz$-coordinate system, you can also specify points
using spherical and cylindrical coordinates.


\begin{command}{\pgfpointcylindrical\marg{degree}\marg{radius}\marg{height}}
  This command yields the same as
\begin{verbatim}
\pgfpointadd{\pgfpointpolarxy{degree}{radius}}{\pgfpointxyz{0}{0}{height}}
\end{verbatim}
\begin{codeexample}[]
\begin{tikzpicture}
  \draw [->] (0,0) -- (1,0,0) node [right] {$x$};
  \draw [->] (0,0) -- (0,1,0) node [above] {$y$};
  \draw [->] (0,0) -- (0,0,1) node [below left] {$z$};

  \pgfpathcircle{\pgfpointcylindrical{80}{1}{.5}}{2pt}
  \pgfusepath{fill}

  \draw[red] (0,0) -- (0,0,.5) -- +(80:1);
\end{tikzpicture}
\end{codeexample}
\end{command}

\begin{command}{\pgfpointspherical\marg{longitude}\marg{latitude}\marg{radius}}
  This command yields a point ``on the surface of the earth''
  specified by the \meta{longitude} and the \marg{latitude}. The
  radius of the earth is given by \meta{radius}. The equator lies in
  the $xy$-plane.
\begin{codeexample}[]
\begin{tikzpicture}
  \draw [->] (0,0) -- (1,0,0) node [right] {$x$};
  \draw [->] (0,0) -- (0,1,0) node [above] {$y$};
  \draw [->] (0,0) -- (0,0,1) node [below left] {$z$};

  \foreach \angle in {0,10,...,90}
    {\pgfpathcircle{\pgfpointspherical{\angle}{0}{1}}{2pt}}
  \pgfusepath{fill}

  \pgfsetcolor{red}
  \foreach \angle in {0,10,...,90}
    {\pgfpathcircle{\pgfpointspherical{80}{\angle}{1}}{2pt}}
  \pgfusepath{fill}
\end{tikzpicture}
\end{codeexample}
\end{command}



\subsection{Building Coordinates From Other Coordinates}

Many commands allow you to construct a coordinate in terms of other
coordinates.


\subsubsection{Basic Manipulations of Coordinates}

\begin{command}{\pgfpointadd\marg{$v_1$}\marg{$v_2$}}
  Returns the sum vector $\meta{$v_1$} + \meta{$v_2$}$.
\begin{codeexample}[]
\begin{tikzpicture}
  \draw[help lines] (0,0) grid (3,2);
  \pgfpathcircle{\pgfpointadd{\pgfpoint{1cm}{0cm}}{\pgfpoint{1cm}{1cm}}}{2pt}
  \pgfusepath{fill} 
\end{tikzpicture}
\end{codeexample}
\end{command}

\begin{command}{\pgfpointscale\marg{factor}\marg{coordinate}}
  Returns the vector $\meta{factor}\meta{coordinate}$.
\begin{codeexample}[]
\begin{tikzpicture}
  \draw[help lines] (0,0) grid (3,2);
  \pgfpathcircle{\pgfpointscale{1.5}{\pgfpoint{1cm}{0cm}}}{2pt}
  \pgfusepath{fill} 
\end{tikzpicture}
\end{codeexample}
\end{command}

\begin{command}{\pgfpointdiff\marg{start}\marg{end}}
  Returns the difference vector $\meta{end} - \meta{start}$.
\begin{codeexample}[]
\begin{tikzpicture}
  \draw[help lines] (0,0) grid (3,2);
  \pgfpathcircle{\pgfpointdiff{\pgfpoint{1cm}{0cm}}{\pgfpoint{1cm}{1cm}}}{2pt}
  \pgfusepath{fill} 
\end{tikzpicture}
\end{codeexample}
\end{command}


\begin{command}{\pgfpointnormalised\marg{point}}
  This command returns a normalized version of \meta{point}, that is,
  a vector of length 1pt pointing in the direction of \meta{point}. If
  \meta{point} is the $0$-vector or extremely short, a vector of
  length 1pt pointing upwards is returned.

  This command is \emph{not} implemented by calculating the length of
  the vector, but rather by calculating the angle of the vector and
  then using (something equivalent to) the |\pgfpointpolar|
  command. This ensures that the point will really have length 1pt,
  but it is not guaranteed that the vector will \emph{precisely} point
  in the direction of \meta{point} due to the fact that the polar
  tables are accurate only up to one degree. Normally, this is not a
  problem.
\begin{codeexample}[]
\begin{tikzpicture}
  \draw[help lines] (0,0) grid (3,2);
  \pgfpathcircle{\pgfpoint{2cm}{1cm}}{2pt}
  \pgfpathcircle{\pgfpointscale{20}
    {\pgfpointnormalised{\pgfpoint{2cm}{1cm}}}}{2pt}
  \pgfusepath{fill} 
\end{tikzpicture}
\end{codeexample}  
\end{command}


\subsubsection{Points Traveling along Lines and Curves}

\label{section-pointsattime}

The commands in this section allow you to specify points on a line or
a curve. Imaging a point ``traveling'' along a curve from some point
$p$ to another point $q$. At time $t=0$ the point is at $p$ and at
time $t=1$ it is at $q$ and at time, say, $t=1/2$ it is ``somewhere in
the middle.'' The exact location at time $t=1/2$ will not necessarily
be the ``halfway point,'' that is, the point whose distance on the
curve from $p$ and $q$ is equal. Rather, the exact location will
depend on the ``speed'' at which the point is traveling, which in
turn depends on the lengths of the support vectors in a complicated
manner. If you are interested in the details, please see a good book
on B�zier curves.



\begin{command}{\pgfpointlineattime\marg{time $t$}\marg{point $p$}\marg{point $q$}}
  Yields a point that is the $t$th fraction between $p$
  and~$q$, that is, $p + t(q-p)$. For $t=1/2$ this is the middle of
  $p$ and $q$.

\begin{codeexample}[]
\begin{tikzpicture}
  \draw[help lines] (0,0) grid (3,2);
  \pgfpathmoveto{\pgfpointorigin}
  \pgfpathlineto{\pgfpoint{2cm}{2cm}}
  \pgfusepath{stroke}
  \foreach \t in {0,0.25,...,1.25}
    {\pgftext[at=
      \pgfpointlineattime{\t}{\pgfpointorigin}{\pgfpoint{2cm}{2cm}}]{\t}}
\end{tikzpicture}    
\end{codeexample}
\end{command}

\begin{command}{\pgfpointlineatdistance\marg{distance}\marg{start point}\marg{end point}}
  Yields a point that is located \meta{distance} many units removed
  from the start point in the direction of the end point. In other
  words, this is the point that results if we travel \meta{distance}
  steps from \meta{start point} towards \meta{end point}.
  \example
\begin{codeexample}[]
\begin{tikzpicture}
  \draw[help lines] (0,0) grid (3,2);
  \pgfpathmoveto{\pgfpointorigin}
  \pgfpathlineto{\pgfpoint{3cm}{2cm}}
  \pgfusepath{stroke}
  \foreach \d in {0pt,20pt,40pt,70pt}
    {\pgftext[at=
      \pgfpointlineatdistance{\d}{\pgfpointorigin}{\pgfpoint{3cm}{2cm}}]{\d}}
\end{tikzpicture}    
\end{codeexample}
\end{command}

\begin{command}{\pgfpointcurveattime\marg{time $t$}\marg{point
      $p$}\marg{point $s_1$}\marg{point $s_2$}\marg{point $q$}} 
  Yields a point that is on the B�zier curve from $p$ to $q$ with the
  support points $s_1$ and $s_2$. The time $t$ is used to determine
  the location, where $t=0$ yields $p$ and $t=1$ yields $q$.

\begin{codeexample}[]
\begin{tikzpicture}
  \draw[help lines] (0,0) grid (3,2);
  \pgfpathmoveto{\pgfpointorigin}
  \pgfpathcurveto
    {\pgfpoint{0cm}{2cm}}{\pgfpoint{0cm}{2cm}}{\pgfpoint{3cm}{2cm}}
  \pgfusepath{stroke}
  \foreach \t in {0,0.25,0.5,0.75,1}
    {\pgftext[at=\pgfpointcurveattime{\t}{\pgfpointorigin}
                                         {\pgfpoint{0cm}{2cm}}
                                         {\pgfpoint{0cm}{2cm}}
                                         {\pgfpoint{3cm}{2cm}}]{\t}}
\end{tikzpicture}    
\end{codeexample}
\end{command}

\subsubsection{Points on Borders of Objects}

The following commands are useful for specifying a point that lies on
the border of special shapes. They are used, for example, by the shape
mechanism to determine border points of shapes.

\begin{command}{\pgfpointborderrectangle\marg{direction point}\marg{corner}}
  This command returns a point that lies on the intersection of a line
  starting at the origin and going towards the point \meta{direction
    point} and a rectangle whose center is in the origin and whose
  upper right corner is at \meta{corner}.

  The \meta{direction point} should have length ``about 1pt,'' but it
  will be normalized automatically. Nevertheless, the ``nearer'' the
  length is to 1pt, the less rounding errors.

\begin{codeexample}[]
\begin{tikzpicture}
  \draw[help lines] (0,0) grid (2,1.5);
  \pgfpathrectanglecorners{\pgfpoint{-1cm}{-1.25cm}}{\pgfpoint{1cm}{1.25cm}}
  \pgfusepath{stroke}

  \pgfpathcircle{\pgfpoint{5pt}{5pt}}{2pt}
  \pgfpathcircle{\pgfpoint{-10pt}{5pt}}{2pt}
  \pgfusepath{fill}
  \color{red}
  \pgfpathcircle{\pgfpointborderrectangle
    {\pgfpoint{5pt}{5pt}}{\pgfpoint{1cm}{1.25cm}}}{2pt}
  \pgfpathcircle{\pgfpointborderrectangle
    {\pgfpoint{-10pt}{5pt}}{\pgfpoint{1cm}{1.25cm}}}{2pt}
  \pgfusepath{fill}
\end{tikzpicture}    
\end{codeexample}
\end{command}


\begin{command}{\pgfpointborderellipse\marg{direction point}\marg{corner}}
  This command works like the corresponding command for rectangles,
  only this time the \meta{corner} is the corner of the bounding
  rectangle of an ellipse.

\begin{codeexample}[]
\begin{tikzpicture}
  \draw[help lines] (0,0) grid (2,1.5);
  \pgfpathellipse{\pgfpointorigin}{\pgfpoint{1cm}{0cm}}{\pgfpoint{0cm}{1.25cm}}
  \pgfusepath{stroke}

  \pgfpathcircle{\pgfpoint{5pt}{5pt}}{2pt}
  \pgfpathcircle{\pgfpoint{-10pt}{5pt}}{2pt}
  \pgfusepath{fill}
  \color{red}
  \pgfpathcircle{\pgfpointborderellipse
    {\pgfpoint{5pt}{5pt}}{\pgfpoint{1cm}{1.25cm}}}{2pt}
  \pgfpathcircle{\pgfpointborderellipse
    {\pgfpoint{-10pt}{5pt}}{\pgfpoint{1cm}{1.25cm}}}{2pt}
  \pgfusepath{fill}
\end{tikzpicture}    
\end{codeexample}
\end{command}


\subsubsection{Points on the Intersection of Lines}


\begin{command}{\pgfpointintersectionoflines\marg{$p$}\marg{$q$}\marg{$s$}\marg{$t$}}
  This command returns the intersection of a line going through $p$
  and $q$ and a line going through $s$ and $t$. If the lines do not
  intersection, an arithmetic overflow will occur.

\begin{codeexample}[]
\begin{tikzpicture}
  \draw[help lines] (0,0) grid (2,2);
  \draw (.5,0) -- (2,2);
  \draw (1,2) -- (2,0);
  \pgfpathcircle{%
    \pgfpointintersectionoflines
      {\pgfpointxy{.5}{0}}{\pgfpointxy{2}{2}}
      {\pgfpointxy{1}{2}}{\pgfpointxy{2}{0}}}
    {2pt}
  \pgfusepath{stroke}
\end{tikzpicture}    
\end{codeexample}
\end{command}

\subsection{Extracting Coordinates}

There are two commands that can be used to ``extract'' the $x$- or
$y$-coordinate of a coordinate. 

\begin{command}{\pgfextractx\marg{dimension}\marg{point}}
  Sets the \TeX-\meta{dimension} to the $x$-coordinate of the point.

\begin{codeexample}[code only]
\newdimen\mydim
\pgfextractx{\mydim}{\pgfpoint{2cm}{4pt}}
%% \mydim is now 2cm
\end{codeexample}
\end{command}

\begin{command}{\pgfextracty\marg{dimension}\marg{point}}
  Like |\pgfextractx|, except for the $y$-coordinate.
\end{command}




\subsection{Internals of How Point Commands Work}

As a normal user of \pgfname\ you do not need to read this section. It
is relevant only if you need to understand how the point commands work
internally. 

When a command like |\pgfpoint{1cm}{2pt}| is called, all that happens
is that the two \TeX-dimension variables |\pgf@x| and |\pgf@y| are set
to |1cm| and |2pt|, respectively. A command like |\pgfpathmoveto| that
takes a coordinate as parameter will just execute this parameter and
then use the values of |\pgf@x| and |\pgf@y| as the coordinates to
which it will move the pen on the current path.

since commands like |\pgfpointnormalised| modify other variables
besides |\pgf@x| and |\pgf@y| during the computation of the final values of
|\pgf@x| and |\pgf@y|, it is a good idea to enclose a call of a
command like |\pgfpoint| in a \TeX-scope and then make the changes of
|\pgf@x| and |\pgf@y| global as in the following example:
\begin{codeexample}[code only]
...
{ % open scope
  \pgfpointnormalised{\pgfpoint{1cm}{1cm}}
  \global\pgf@x=\pgf@x % make the change of \pgf@x persist past the scope
  \global\pgf@y=\pgf@y % make the change of \pgf@y persist past the scope
}
% \pgf@x and \pgf@y are now set correctly, all other variables are
% unchanged 
\end{codeexample}

\makeatletter
Since this situation arises very often, the macro |\pgf@process| can
be used to perform the above code:
\begin{command}{\pgf@process\marg{code}}
  Executes the \meta{code} in a scope and then makes |\pgf@x| and
  |\pgf@y| global.
\end{command}

Note that this macro is used often internally. For this reason, it is
not a good idea to keep anything important in the variables |\pgf@x|
and |\pgf@y| since they will be overwritten and changed
frequently. Instead, intermediate values can ge stored in the
\TeX-dimensions |\pgf@xa|, |\pgf@xb|, |\pgf@xc| and their
|y|-counterparts |\pgf@ya|, |\pgf@yb|, |pgf@yc|. For example, here is
the code of the command |\pgfpointadd|:
\begin{codeexample}[code only]
\def\pgfpointadd#1#2{%
  \pgf@process{#1}%
  \pgf@xa=\pgf@x%
  \pgf@ya=\pgf@y%
  \pgf@process{#2}%
  \advance\pgf@x by\pgf@xa%
  \advance\pgf@y by\pgf@ya}
\end{codeexample}



%%% Local Variables: 
%%% mode: latex
%%% TeX-master: "pgfmanual"
%%% End: 

% Copyright 2006 by Till Tantau
%
% This file may be distributed and/or modified
%
% 1. under the LaTeX Project Public License and/or
% 2. under the GNU Free Documentation License.
%
% See the file doc/generic/pgf/licenses/LICENSE for more details.


\section{Constructing Paths}

\subsection{Overview}

The ``basic entity of drawing'' in \pgfname\ is the \emph{path}. A
path consists of several parts, each of which is either a closed or
open curve. An open curve has a starting point and an end point and,
in between, consists of several \emph{segments}, each of which is
either a straight line or a B�zier curve. Here is an example of a
path (in red) consisting of two parts, one open, one closed:

\begin{codeexample}[]
\begin{tikzpicture}[scale=2]
  \draw[thick,red]
       (0,0) coordinate (a)
    -- coordinate (ab) (1,.5) coordinate (b)
    .. coordinate (bc) controls +(up:1cm) and +(left:1cm) .. (3,1)  coordinate (c)
       (0,1) -- (2,1) -- coordinate (x) (1,2) -- cycle;

  \draw (a)  node[below] {start part 1}
        (ab) node[below right] {straight segment}
        (b)  node[right] {end first segment}
        (c)  node[right] {end part 1}
        (x)  node[above right]  {part 2 (closed)};        
\end{tikzpicture}
\end{codeexample}

A path, by itself, has no ``effect,'' that is, it does not leave any
marks on the page. It is just a set of points on the plane. However,
you can \emph{use} a path in different ways. The most natural actions
are \emph{stroking} (also known as \emph{drawing}) and
\emph{filling}. Stroking can be imagined as picking up a pen of a
certain diameter and ``moving it along the path.'' Filling means that
everything ``inside'' the path is filled with a uniform
color. Naturally, the open parts of a path must first be closed before
a path can be filled.

In \pgfname, there are numerous commands for constructing paths, all
of which start with |\pgfpath|. There are also commands for
\emph{using} paths, though most operations can be performed by calling
|\pgfusepath| with an appropriate parameter.

As a side-effect, the path construction commands keep track of two
bounding boxes. One is the bounding box for the current path, the
other is a bounding box for all paths in the current picture. See
Section~\ref{section-bb} for more details.

Each path construction command extends the current path in some
way. The ``current path'' is a global entity that persists across
\TeX\ groups. Thus, between calls to the path construction commands
you can perform arbitrary computations and even open and closed \TeX\
groups. The current path only gets ``flushed'' when the |\pgfusepath|
command is called (or when the soft-path subsystem is used directly,
see Section~\ref{section-soft-paths}).

\subsection{The Move-To Path Operation}

The most basic operation is the move-to operation. It must be given at
the beginning of paths, though some path construction command (like
|\pgfpathrectangle|) generate move-tos implicitly. A move-to operation
can also be used to start a new part of a path. 

\begin{command}{\pgfpathmoveto\marg{coordinate}}
  This command expects a \pgfname-coordinate like |\pgfpointorigin| as
  its parameter. When the current path is empty, this operation will
  start the path at the given \meta{coordinate}. If a path has already
  been partly constructed, this command will end the current part of
  the path and start a new one.
\begin{codeexample}[]
\begin{pgfpicture}
  \pgfpathmoveto{\pgfpointorigin}
  \pgfpathlineto{\pgfpoint{1cm}{1cm}}
  \pgfpathlineto{\pgfpoint{2cm}{1cm}}
  \pgfpathlineto{\pgfpoint{3cm}{0.5cm}}
  \pgfpathlineto{\pgfpoint{3cm}{0cm}}
  \pgfsetfillcolor{examplefill}
  \pgfusepath{fill,stroke}
\end{pgfpicture}
\end{codeexample}
\begin{codeexample}[]
\begin{pgfpicture}
  \pgfpathmoveto{\pgfpointorigin}
  \pgfpathlineto{\pgfpoint{1cm}{1cm}}
  \pgfpathlineto{\pgfpoint{2cm}{1cm}}
  \pgfpathmoveto{\pgfpoint{2cm}{1cm}} % New part
  \pgfpathlineto{\pgfpoint{3cm}{0.5cm}}
  \pgfpathlineto{\pgfpoint{3cm}{0cm}}
  \pgfsetfillcolor{examplefill}
  \pgfusepath{fill,stroke}
\end{pgfpicture}
\end{codeexample}
  The command will apply the current coordinate transformation matrix
  to \meta{coordinate} before using it.

  The command will update the bounding box of the current path and
  picture, if necessary. 
\end{command}


\subsection{The Line-To Path Operation}

\begin{command}{\pgfpathlineto\marg{coordinate}}
  This command extends the current path in a straight line to the
  given \meta{coordinate}. If this command is given at the beginning
  of path without any other path construction command given before (in
  particular without a move-to operation), the \TeX\ file may compile
  without an error message, but a viewer application may display an
  error message when trying to render the picture. 
\begin{codeexample}[]
\begin{pgfpicture}
  \pgfpathmoveto{\pgfpointorigin}
  \pgfpathlineto{\pgfpoint{1cm}{1cm}}
  \pgfpathlineto{\pgfpoint{2cm}{1cm}}
  \pgfsetfillcolor{examplefill}
  \pgfusepath{fill,stroke}
\end{pgfpicture}
\end{codeexample}
  The command will apply the current coordinate transformation matrix
  to \meta{coordinate} before using it.

  The command will update the bounding box of the current path and
  picture, if necessary. 
\end{command}


\subsection{The Curve-To Path Operation}

\begin{command}{\pgfpathcurveto\marg{support 1}\marg{support 2}\marg{coordinate}}
  This command extends the current path with a B�zier curve from the
  last point of the path to  \meta{coordinate}. The \meta{support 1}
  and \meta{support 2} are the first and second support point of the
  B�zier curve. For more information on B�zier curve, please consult a
  standard textbook on computer graphics.

  Like the line-to command, this command may not be the first path
  construction command in a path.
\begin{codeexample}[]
\begin{pgfpicture}
  \pgfpathmoveto{\pgfpointorigin}
  \pgfpathcurveto
    {\pgfpoint{1cm}{1cm}}{\pgfpoint{2cm}{1cm}}{\pgfpoint{3cm}{0cm}}
  \pgfsetfillcolor{examplefill}
  \pgfusepath{fill,stroke}
\end{pgfpicture}
\end{codeexample}
  The command will apply the current coordinate transformation matrix
  to \meta{coordinate} before using it.

  The command will update the bounding box of the current path and
  picture, if necessary. However, the bounding box is simply made
  large enough such that it encompasses all of the support points and
  the \meta{coordinate}. This will guarantee that the curve is
  completely inside the bounding box, but the bounding box will
  typically be quite a bit too large. It is not clear (to me) how this 
  can be avoided without resorting to ``some serious math'' in order
  to calculate a precise bounding box. 
\end{command}


\subsection{The Close Path Operation}

\begin{command}{\pgfpathclose}
  This command closes the current part of the path by appending a
  straight line to the start point of the current part. Note that there
  \emph{is} a difference between closing a path and using the line-to
  operation to add a straight line to the start of the current
  path. The difference is demonstrated by the upper corners of the triangles
  in the following example: 
\begin{codeexample}[]
\begin{tikzpicture}
  \draw[help lines] (0,0) grid (3,2);
  \pgfsetlinewidth{5pt}
  \pgfpathmoveto{\pgfpoint{1cm}{1cm}}
  \pgfpathlineto{\pgfpoint{0cm}{-1cm}}
  \pgfpathlineto{\pgfpoint{1cm}{-1cm}}
  \pgfpathclose
  \pgfpathmoveto{\pgfpoint{2.5cm}{1cm}}
  \pgfpathlineto{\pgfpoint{1.5cm}{-1cm}}
  \pgfpathlineto{\pgfpoint{2.5cm}{-1cm}}
  \pgfpathlineto{\pgfpoint{2.5cm}{1cm}}
  \pgfusepath{stroke}
\end{tikzpicture}
\end{codeexample}
\end{command}


\subsection{Arc, Ellipse and Circle Path Operations}

The path construction commands that we have discussed up to now are
sufficient to create all paths that can be created ``at all.''
However, it is useful to have special commands to create certain
shapes, like circles, that arise often in practice.

In the following, the commands for adding (parts of) (transformed)
circles to a path are described.

\begin{command}{\pgfpatharc\marg{start angle}\marg{end
      angle}{\ttfamily\char`\{}\meta{radius}\opt{| and |\meta{y-radius}}{\ttfamily\char`\}}}
  This command appends a part of a circle (or an ellipse) to the current
  path. Imaging the curve between \meta{start angle} and \meta{end
    angle} on a circle of radius \meta{radius} (if $\meta{start angle}
  < \meta{end angle}$, the curve goes around the circle
  counterclockwise, otherwise clockwise). This curve is now moved such
  that the point where the curve starts is the previous last point of the
  path. Note that this command will \emph{not} start a new part of the
  path, which is important for example for filling purposes. 

\begin{codeexample}[]
\begin{tikzpicture}
  \draw[help lines] (0,0) grid (3,2);
  \pgfpathmoveto{\pgfpointorigin}
  \pgfpathlineto{\pgfpoint{0cm}{1cm}}
  \pgfpatharc{180}{90}{.5cm}
  \pgfpathlineto{\pgfpoint{3cm}{1.5cm}}
  \pgfpatharc{90}{-45}{.5cm}
  \pgfusepath{fill}
\end{tikzpicture}
\end{codeexample}

  Saying |\pgfpatharc{0}{360}{1cm}| ``nearly'' gives you a full
  circle. The ``nearly'' refers to the fact that the circle will not
  be closed. You can close it using |\pgfpathclose|.

  If the optional \meta{y-radius} is given, the \meta{radius} is the
  $x$-radius and the \meta{y-radius} the $y$-radius of the ellipse
  from which the curve is taken:

\begin{codeexample}[]
\begin{tikzpicture}
  \draw[help lines] (0,0) grid (3,2);
  \pgfpathmoveto{\pgfpointorigin}
  \pgfpatharc{180}{45}{2cm and 1cm}
  \pgfusepath{draw}
\end{tikzpicture}
\end{codeexample}

  The axes of the circle or ellipse from which the arc is ``taken''
  always point up and right. However, the current coordinate
  transformation matrix will have an effect on the arc. This can be
  used to, say, rotate an arc:

\begin{codeexample}[]
\begin{tikzpicture}
  \draw[help lines] (0,0) grid (3,2);
  \pgftransformrotate{30}
  \pgfpathmoveto{\pgfpointorigin}
  \pgfpatharc{180}{45}{2cm and 1cm}
  \pgfusepath{draw}
\end{tikzpicture}
\end{codeexample}

  The command will update the bounding box of the current path and
  picture, if necessary. Unless rotation or shearing transformations
  are applied, the bounding box will be tight.
\end{command}

\begin{command}{\pgfpatharcaxes\marg{start angle}\marg{end
      angle}\marg{first axis}\marg{second axis}}
  This command is similar to |\pgfpatharc|. The main difference is how
  the ellipse or circle is specified from which the arc is taken. The
  two parameters \meta{first axis} and \meta{second axis} are the
  $0^\circ$-axis and the $90^\circ$-axis of the ellipse from which the
  path is taken. Thus, |\pgfpatharc{0}{90}{1cm and 2cm}| has the same effect
  as
\begin{verbatim}
\pgfpatharcaxes{0}{90}{\pgfpoint{1cm}{0cm}}{\pgfpoint{0cm}{2cm}}
\end{verbatim}
\begin{codeexample}[]
\begin{tikzpicture}
  \draw[help lines] (0,0) grid (3,2);
  \draw (0,0) -- (2cm,5mm) (0,0) -- (0cm,1cm);
  
  \pgfpathmoveto{\pgfpoint{2cm}{5mm}}
  \pgfpatharcaxes{0}{90}{\pgfpoint{2cm}{5mm}}{\pgfpoint{0cm}{1cm}}
  \pgfusepath{draw}
\end{tikzpicture}
\end{codeexample}
\end{command}  


\begin{command}{\pgfpathellipse\marg{center}\marg{first
      axis}\marg{second axis}}
  The effect of this command is to append an ellipse to the current
  path (if the path is not empty, a new part is started). The
  ellipse's center will be \meta{center} and \meta{first axis} and
  \meta{second axis} are the axis \emph{vectors}. The same effect as
  this command can also be achieved using an appropriate sequence of
  move-to, arc, and close operations, but this command is easier and
  faster. 

\begin{codeexample}[]
\begin{tikzpicture}
  \draw[help lines] (0,0) grid (3,2);
  \pgfpathellipse{\pgfpoint{1cm}{0cm}}
                 {\pgfpoint{1.5cm}{0cm}}
                 {\pgfpoint{0cm}{1cm}}
  \pgfusepath{draw}
  \color{red}               
  \pgfpathellipse{\pgfpoint{1cm}{0cm}}
                 {\pgfpoint{1cm}{1cm}}
                 {\pgfpoint{-0.5cm}{0.5cm}}
  \pgfusepath{draw}
\end{tikzpicture}
\end{codeexample}

  The command will apply coordinate transformations to all coordinates
  of the ellipse. However, the coordinate transformations are applied
  only after the ellipse is ``finished conceptually.'' Thus, a
  transformation of 1cm to the right will simply shift the ellipse one
  centimeter to the right; it will not add 1cm to the $x$-coordinates
  of the two axis vectors.

  The command will update the bounding box of the current path and
  picture, if necessary. 
\end{command}

\begin{command}{\pgfpathcirlce\marg{center}\marg{radius}}
  A shorthand for |\pgfpathellipse| applied to \meta{center} and the
  two axis vectors $(\meta{radius},0)$ and $(0,\meta{radius})$. 
\end{command}


\subsection{Rectangle Path Operations}

Another shape that arises frequently is the rectangle. Two commands
can be used to add a rectangle to the current path. Both commands will
start a new part of the path.


\begin{command}{\pgfpathrectangle\marg{corner}\marg{diagonal vector}}
  Adds a rectangle to the path whose one corner is \meta{corner} and
  whose opposite corner is given by $\meta{corner} + \meta{diagonal
    vector}$.

\begin{codeexample}[]
\begin{tikzpicture}
  \draw[help lines] (0,0) grid (3,2);
  \pgfpathrectangle{\pgfpoint{1cm}{0cm}}{\pgfpoint{1.5cm}{1cm}}
  \pgfpathrectangle{\pgfpoint{1.5cm}{0.25cm}}{\pgfpoint{1.5cm}{1cm}}
  \pgfpathrectangle{\pgfpoint{2cm}{0.5cm}}{\pgfpoint{1.5cm}{1cm}}
  \pgfusepath{draw}
\end{tikzpicture}
\end{codeexample}
  The command will apply coordinate transformations and update the
  bounding boxes tightly.
\end{command}


\begin{command}{\pgfpathrectanglecorners\marg{corner}\marg{opposite corner}}
  Adds a rectangle to the path whose two opposing corners are
  \meta{corner} and \meta{opposite corner}.
\begin{codeexample}[]
\begin{tikzpicture}
  \draw[help lines] (0,0) grid (3,2);
  \pgfpathrectanglecorners{\pgfpoint{1cm}{0cm}}{\pgfpoint{1.5cm}{1cm}}
  \pgfusepath{draw}
\end{tikzpicture}
\end{codeexample}
  The command will apply coordinate transformations and update the
  bounding boxes tightly.
\end{command}



\subsection{The Grid Path Operation}

\begin{command}{\pgfpathgrid\oarg{options}\marg{lower left}\marg{upper right}}
  Appends a grid to the current path. That is, a (possibly large)
  number of parts are added to the path, each part consisting of a
  single horizontal or vertical straight line segment.

  Conceptually, the origin is part of the grid and the grid is clipped 
  to the rectangle specified by the \meta{lower left} and
  the \meta{upper right} corner. However, no clipping occurs (this
  command just adds parts to the current path). Rather, the points
  where the lines enter and leave the ``clipping area'' are computed
  and used to add simple lines to the current path.

  The following keys influence the grid:
  \begin{key}{/pgf/stepx=\meta{dimension} (initially 1cm)}
    The horizontal stepping.
  \end{key}
  \begin{key}{/pgf/stepy=\meta{dimension} (initially 1cm)}
    The vertical stepping.
  \end{key}
  \begin{key}{/pgf/step=\meta{vector}}
    Sets the horizontal stepping to the $x$-coordinate of
    \meta{vector} and the vertical stepping to its $y$-coordinate.
  \end{key}
\begin{codeexample}[]
\begin{pgfpicture}
  \pgfsetlinewidth{0.8pt}
  \pgfpathgrid[step={\pgfpoint{1cm}{1cm}}]
    {\pgfpoint{-3mm}{-3mm}}{\pgfpoint{33mm}{23mm}}
  \pgfusepath{stroke}
  \pgfsetlinewidth{0.4pt}
  \pgfpathgrid[stepx=1mm,stepy=1mm]
    {\pgfpoint{-1.5mm}{-1.5mm}}{\pgfpoint{31.5mm}{21.5mm}}
  \pgfusepath{stroke}
\end{pgfpicture}
\end{codeexample}
  The command will apply coordinate transformations and update the
  bounding boxes tightly. As for ellipses, the transformations are
  applied to the ``conceptually finished'' grid. 
\begin{codeexample}[]
\begin{pgfpicture}
  \pgftransformrotate{10}
  \pgfpathgrid[stepx=1mm,stepy=2mm]{\pgfpoint{0mm}{0mm}}{\pgfpoint{30mm}{30mm}}
  \pgfusepath{stroke}
\end{pgfpicture}
\end{codeexample}
\end{command}


\subsection{The Parabola Path Operation}

\begin{command}{\pgfpathparabola\marg{bend vector}\marg{end vector}}
  This command appends two half-parabolas to the  current path. The
  first starts at the current point and ends at the current point plus
  \meta{bend vector}. At his point, it has its bend. The second half
  parabola starts at that bend point and end at point that is given by
  the bend plus \meta{end vector}.

  If you set \meta{end vector} to the null vector, you append only a
  half parabola that goes from the current point to the bend; by
  setting \meta{bend vector} to the null vector, you append only a
  half parabola that goes to current point plus \meta{end vector} and
  has its bend at the current point.

  It is not possible to use this command to draw a part of a parabola
  that does not contain the bend.

\begin{codeexample}[]
\begin{pgfpicture}
  % Half-parabola going ``up and right''
  \pgfpathmoveto{\pgfpointorigin}
  \pgfpathparabola{\pgfpointorigin}{\pgfpoint{2cm}{4cm}}
  \color{red}
  \pgfusepath{stroke}

  % Half-parabola going ``down and right''
  \pgfpathmoveto{\pgfpointorigin}
  \pgfpathparabola{\pgfpoint{-2cm}{4cm}}{\pgfpointorigin}
  \color{blue}
  \pgfusepath{stroke}

  % Full parabola
  \pgfpathmoveto{\pgfpoint{-2cm}{2cm}}
  \pgfpathparabola{\pgfpoint{1cm}{-1cm}}{\pgfpoint{2cm}{4cm}}
  \color{orange}
  \pgfusepath{stroke}
\end{pgfpicture}
\end{codeexample}
  The command will apply coordinate transformations and update the
  bounding boxes.
\end{command}


\subsection{Sine and Cosine Path Operations}

Sine and cosine curves often need to be drawn and the following commands
may help with this. However, they only allow you to append sine and
cosine curves in intervals that are multiples of $\pi/2$.

\begin{command}{\pgfpathsine\marg{vector}}
  This command appends a sine curve in the interval $[0,\pi/2]$ to the
  current path. The sine curve is squeezed or stretched such that the
  curve starts at the current point and ends at the current point plus
  \meta{vector}.
\begin{codeexample}[]
\begin{tikzpicture}
  \draw[help lines] (0,0) grid (3,1);
  \pgfpathmoveto{\pgfpoint{1cm}{0cm}}
  \pgfpathsine{\pgfpoint{1cm}{1cm}}
  \pgfusepath{stroke}

  \color{red}
  \pgfpathmoveto{\pgfpoint{1cm}{0cm}}
  \pgfpathsine{\pgfpoint{-2cm}{-2cm}}
  \pgfusepath{stroke}
\end{tikzpicture}
\end{codeexample}
  The command will apply coordinate transformations and update the
  bounding boxes.  
\end{command}

\begin{command}{\pgfpathcosine\marg{vector}}
  This command appends a cosine curve in the interval $[0,\pi/2]$ to the
  current path. The curve is squeezed or stretched such that the
  curve starts at the current point and ends at the current point plus
  \meta{vector}. Using several sine and cosine operations in sequence
  allows you to produce a complete sine or cosine curve
\begin{codeexample}[]
\begin{pgfpicture}
  \pgfpathmoveto{\pgfpoint{0cm}{0cm}}
  \pgfpathsine{\pgfpoint{1cm}{1cm}}
  \pgfpathcosine{\pgfpoint{1cm}{-1cm}}
  \pgfpathsine{\pgfpoint{1cm}{-1cm}}
  \pgfpathcosine{\pgfpoint{1cm}{1cm}}
  \pgfsetfillcolor{examplefill}
  \pgfusepath{fill,stroke}
\end{pgfpicture}
\end{codeexample}
  The command will apply coordinate transformations and update the
  bounding boxes.  
\end{command}

\subsection{Drawing parts of curves}

There exist two commands to draw only part of a cubic B�zier curve:

\begin{command}{\pgfpathcurvebetweentime\marg{time $t_1$}\marg{time $t_2$}\marg{point p}\marg{point $s_1$}\marg{point $s_2$}\marg{point q}}

  This command draws the part of the curve described by $p$, $s_1$, 
  $s_2$ and $q$ between the times $t_1$ and $t_2$. A time value of 0 
  indicates the point $p$ and a time vaue of 1 indicates point $q$. 
  This command includes a moveto operation to the first point.

\begin{codeexample}[]
\begin{tikzpicture}
  \draw [thin] (0,0) .. controls (0,2) and (3,0) .. (3,2);
  \pgfpathcurvebetweentime{0.25}{0.9}{\pgfpointxy{0}{0}}{\pgfpointxy{0}{2}}
    {\pgfpointxy{3}{0}}{\pgfpointxy{3}{2}}
  \pgfsetstrokecolor{red}
  \pgfsetstrokeopacity{0.5}
  \pgfsetlinewidth{2pt}
  \pgfusepath{stroke}
\end{tikzpicture}
\end{codeexample}
\end{command}

\begin{command}{\pgfpathcurvebetweentime|*|\marg{time $t_1$}\marg{time $t_2$}\marg{point p}\marg{point $s_1$}\marg{point $s_2$}\marg{point q}}

  This command works like |\pgfpathcurvebetweentime|, except that a 
  moveto operation is \emph{not} made to the first point.

\end{command}

\subsection{Plot Path Operations}

There exist several commands for appending
plots to a path. These
commands are available through the module |plot|. They are
documented in Section~\ref{section-plots}.


\subsection{Rounded Corners}

Normally, when you connect two straight line segments or when you
connect two curves that end and start ``at different angles'' you get
``sharp corners'' between the lines or curves. In some cases it is
desirable to produce ``rounded corners'' instead. Thus, the lines
or curves should be shortened a bit and then connected by arcs.

\pgfname\ offers an easy way to achieve this effect, by calling the
following two commands.

\begin{command}{\pgfsetcornersarced\marg{point}}
  This command causes all subsequent corners to be replaced by little
  arcs. The effect of this command lasts till the end of the current
  \TeX\ scope.

  The \meta{point} dictates how large the corner arc will be. Consider
  a corner made by two lines $l$ and~$r$ and assume that the line $l$
  comes first on the path. The $x$-dimension of the \meta{point}
  decides by how much the line~$l$ will be shortened, the
  $y$-dimension of \meta{point} decides by how much the line $r$ will
  be shortened. Then, the shortened lines are connected by an arc.

\begin{codeexample}[]
\begin{tikzpicture}
  \draw[help lines] (0,0) grid (3,2);

  \pgfsetcornersarced{\pgfpoint{5mm}{5mm}}
  \pgfpathrectanglecorners{\pgfpointorigin}{\pgfpoint{3cm}{2cm}}
  \pgfusepath{stroke}
\end{tikzpicture}
\end{codeexample}

\begin{codeexample}[]
\begin{tikzpicture}
  \draw[help lines] (0,0) grid (3,2);

  \pgfsetcornersarced{\pgfpoint{10mm}{5mm}}
  % 10mm entering,
  % 5mm leaving.
  \pgfpathmoveto{\pgfpointorigin}
  \pgfpathlineto{\pgfpoint{0cm}{2cm}}
  \pgfpathlineto{\pgfpoint{3cm}{2cm}}
  \pgfpathcurveto
    {\pgfpoint{3cm}{0cm}}
    {\pgfpoint{2cm}{0cm}}
    {\pgfpoint{1cm}{0cm}}
  \pgfusepath{stroke}
\end{tikzpicture}
\end{codeexample}

  If the $x$- and $y$-coordinates of \meta{point} are the same and the
  corner is a right angle, you will get a perfect quarter circle
  (well, not quite perfect, but perfect up to six decimals). When the
  angle is not $90^\circ$, you only get a fair approximation.

  More or less ``all'' corners will be rounded, even the corner
  generated by a |\pgfpathclose| command. (The author is a bit proud
  of this feature.)
  
\begin{codeexample}[]
\begin{pgfpicture}
  \pgfsetcornersarced{\pgfpoint{4pt}{4pt}}
  \pgfpathmoveto{\pgfpointpolar{0}{1cm}}
  \pgfpathlineto{\pgfpointpolar{72}{1cm}}
  \pgfpathlineto{\pgfpointpolar{144}{1cm}}
  \pgfpathlineto{\pgfpointpolar{216}{1cm}}
  \pgfpathlineto{\pgfpointpolar{288}{1cm}}
  \pgfpathclose
  \pgfusepath{stroke}
\end{pgfpicture}
\end{codeexample}

  To return to normal (unrounded) corners, use
  |\pgfsetcornersarced{\pgfpointorigin}|.

  Note that the rounding will produce strange and undesirable effects
  if the lines at the corners are too short. In this case the
  shortening may cause the lines to ``suddenly extend over the other
  end'' which is rarely desirable. 
\end{command}




\subsection{Internal Tracking of Bounding Boxes for Paths and Pictures}

\label{section-bb}

\makeatletter

The path construction commands keep track of two bounding boxes: One
for the current path, which is reset whenever the path is used and
thereby flushed, and a bounding box for the current |{pgfpicture}|. 

The bounding boxes are not accessible by ``normal'' macros. Rather,
two sets of four dimension variables are used for this, all of which
contain the letter~|@|.

\begin{textoken}{\pgf@pathminx}
  The minimum $x$-coordinate ``mentioned'' in the current
  path. Initially, this is set to $16000$pt.
\end{textoken}

\begin{textoken}{\pgf@pathmaxx}
  The maximum $x$-coordinate ``mentioned'' in the current
  path. Initially, this is set to $-16000$pt.
\end{textoken}

\begin{textoken}{\pgf@pathminy}
  The minimum $y$-coordinate ``mentioned'' in the current
  path. Initially, this is set to $16000$pt.
\end{textoken}

\begin{textoken}{\pgf@pathmaxy}
  The maximum $y$-coordinate ``mentioned'' in the current
  path. Initially, this is set to $-16000$pt.
\end{textoken}

\begin{textoken}{\pgf@picminx}
  The minimum $x$-coordinate ``mentioned'' in the current
  picture. Initially, this is set to $16000$pt.
\end{textoken}

\begin{textoken}{\pgf@picmaxx}
  The maximum $x$-coordinate ``mentioned'' in the current
  picture. Initially, this is set to $-16000$pt.
\end{textoken}

\begin{textoken}{\pgf@picminy}
  The minimum $y$-coordinate ``mentioned'' in the current
  picture. Initially, this is set to $16000$pt.
\end{textoken}

\begin{textoken}{\pgf@picmaxy}
  The maximum $y$-coordinate ``mentioned'' in the current
  picture. Initially, this is set to $-16000$pt.
\end{textoken}


Each time a path construction command is called, the above variables
are (globally) updated. To facilitate this, you can use the following
command:

\begin{command}{\pgf@protocolsizes\marg{x-dimension}\marg{y-dimension}}
  Updates all of the above dimension in such a way that the point
  specified by the two arguments is inside both bounding boxes. For
  the picture's bounding box this updating occurs only if
  |\ifpgf@relevantforpicturesize| is true, see below.
\end{command}

For the bounding box of the picture it is not always desirable that
every path construction command affects this bounding box. For
example, if you have just used a clip command, you do not want anything
outside the clipping area to affect the bounding box. For this reason,
there exists a special ``\TeX\ if'' that (locally) decides whether
updating should be applied to the picture's bounding box. Clipping
will set this if to false, as will certain other commands.

\begin{command}{\pgf@relevantforpicturesizefalse}
  Suppresses updating of the picture's bounding box.
\end{command}

\begin{command}{\pgf@relevantforpicturesizetrue}
  Causes updating of the picture's bounding box.
\end{command}


 % Copyright 2008 by Till Tantau and Mark Wibrow
%
% This file may be distributed and/or modified
%
% 1. under the LaTeX Project Public License and/or
% 2. under the GNU Free Documentation License.
%
% See the file doc/generic/pgf/licenses/LICENSE for more details.

\section{Decorations}
\label{section-base-decorations}


\begin{pgfmodule}{decorations}
  The commands for creating decorations are defined in this
  module, so you need to load this module to use decorations. This
  module is automatically loaded by the different decoration
  libraries. 
\end{pgfmodule}


\subsection{Overview}

Decorations are a general way of creating graphics by ``moving along''
a path and, while doing so, either drawing something or constructing a
new path. This could be as simple as extending a path with a
``zigzagged'' line\ldots 

\begin{codeexample}[]
\tikz \draw decorate[decoration=zigzag] {(0,0) -- (3,0)};
\end{codeexample}
\ldots but could also be as complex as typesetting text along a path:
{\catcode`\|12
\begin{codeexample}[]
\tikz \path decorate [decoration={text along path,
     text={Some text along a path}}]
   { (0,2) .. controls (2,2) and (1,0) .. (3,0) };
\end{codeexample}
}

The workflow for using decorations is the following:
\begin{enumerate}
\item You define a decoration using the |\pgfdeclaredecoration|
  command. Different useful decorations are already declared in
  libraries like |decorations.shapes|.
\item You use normal path construction commands like |\pgfpathlineto|
  to construct a path. Let us call this path the
  \emph{to-be-decorated} path.
\item You place the path construction commands inside the environment
  |{pgfdecoration}|. This environment takes the name of a previously
  declared decoration as a parameter. It will then starting ``walking
  along'' the to-be-decorated path. As it does this, a special finite
  automaton called a \emph{decoration automaton} produces as its
  output new path construction commands (or even other outputs). These
  outputs replace the to-be-decorated path; indeed, after the
  to-be-decorated path has been fully walked along it is thrown away,
  only the output of the automaton persists.
\end{enumerate}

In the present section the process of how decoration automata work is
explained first. Then the command(s) for declaring decoration automata
and for using them are covered.



\subsection{Decoration Automata}

Decoration automata (and the closely related meta-decoration automata)
are a general concept for creating graphics ``along paths.'' For
straight lines, this idea was first proposed by Till Tantau in an
earlier version of \pgfname, the idea to extend this to arbitrary path
was proposed and implemented by Mark Wibrow. Further versitility is
provided by ``meta-decorations''. These are automata that decorate a
path with decorations. 

In the present subsection the different ideas underlying decoration
automata are presented.



\subsubsection{The Different Paths}

In order to prevent confusion with different types of path, such
as those that are extended, those that are decorated and those that 
are created, the following conventions will be used:

\begin{itemize}
\item 
  The \emph{preexisting} path refers to the current path in existence 
  before a decoration environment. (Possibly this path has been
  created by another decoration used earlier, but we will still call
  this path the preexisting path also in this case.)
\item
  The \emph{input} path refers to the to-be-decorated path that the
  decoration automaton moves along. The input path may consist of many
  line and curve input segments (for example, a circle or an ellipse
  consists of four curves). It is specified inside the decoration
  environment. 
\item
  The \emph{output} path refers to the path that the decoration 
  creates. Depending on the decoration, this path may or may not be
  empty (a decoration can also choose to use side-effects instead of
  producing an output path). The input path is always consumed by the
  decoration automaton, that is, it is no longer available in any way
  after the decoration automaton has finished. 
\end{itemize}

The effect of a decoration environment is the following: The input
path, which is specified inside the environment, is constructed and
stored. This process does not alter the preexisting path in any
way. Then the decoration automaton is started (as described later) and
it produces an output path (possibly empty). Whenever part of the
output path is produced, it is concatenated with the preexisting
path. After the environment, the current path will equal the original
preexisting path followed by the output path.

It is permissible that a decoration issues a |\pgfusepath|
command. As usual, this causes the current path to be
filled or stroked or some other action to be taken and the current
path is set to the empty path. As described above, when the decoration
automaton starts the current path is the preexisting path and as the
automaton progresses, the current path is constantly being extend by
the output path. This means that first time e |\pgfusepath| command is
used on a decoration, the preexisting path is part of the path this
command operates on; in subsequent calls only the part of the output
path constructed since the last |\pgfusepath| command will be used. 

You can use this mechanism to stroke or fill different part of the
output path in different colors, line widths, fills and shades; all 
within the same decoration. Alternatively, a decoration can choose to
produce no output path at all: the |text| decoration simply typesets
text along a path.


\subsubsection{Segments and States}

The most common use a decoration is to ``repeat something along a
path'' (for example, the |zigzag| decoration  
repeats \tikz\draw decorate[decoration=zigzag]
{(0,0)--(\pgfdecorationsegmentlength,0)}; along a path). However, it
not necessarily the case that only one thing be repeated: a decoration
can consist of different parts, or \emph{segments}, repeated in a
particular order. 

When you declare a decoration, you provide a description 
of how their different segments will be rendered. The description of
each segment should be given in a way as if the ``x-axis'' of the
segment is the tangent to the path at a particular point,
and that point is the origin of the segment.
Thus, for example, the segment of the |zigzag| decoration might be
defined using the following code: 
\begin{codeexample}[code only]
\pgfpathlineto{\pgfpoint{5pt}{5pt}}
\pgfpathlineto{\pgfpoint{15pt}{-5pt}}
\pgfpathlineto{\pgfpoint{20pt}{0pt}}
\end{codeexample}

\pgfname\ will ensure that an appropriate coordinate transformation
is in place when the segment is rendered such that
the segment actually points in the right direction. Also
subsequent segments will be transformed such that they are
``further along the path'' toward the end of the path.
All transformations are setup automatically.

Note that we did not use a |\pgfpathmoveto{\pgfpointorigin}| at the
beginning of the segment code. Doing so would subdivide the path into
numerous subpaths. Rather, we assume that the previous segment caused
the current point to be at the origin.

The width of a segment can (and must) be specified
explicitly. \pgfname\ will use this width to find out the start point
of the next segment and the correct rotation. The width the you
provide need not be the ``real'' width of the segment, which allows
decoration segments to overlap or to be spaced far apart. 

The |zigzag| decoration only has one segment that is repeated again and
again. However, we might also like to have \emph{different} segments
and use rules to describe which segment should be used where. For
example, we might have special segments at the start and at the end.

Decorations use a mechanism known in theoretical in computer science
as \emph{finite state automata} to describe which segment is used at a
particular point. The idea is the following: For the first segment we 
start in a special \emph{state} called the \emph{initial state}. In
this state, and also in all other state later, \pgfname\ first
computes how much space is left on the input path. That is, \pgfname\ keeps
track of the distance to the end of the input path. Attached to each state 
there is a set of rules of the following form: ``If the remaining 
distance on the input path is less than $x$, switch to state~$q$.''
\pgfname\ checks for each of these rules whether it applies and, if
so, immediately switches to state~$q$.

Only if none of the rules tell us to switch to another
state, \pgfname\ will execute the state's code. This code will
(typically) add a segment to the output path. In addition to the rules
there is also width parameter attached to each state. \pgfname\ then
translates the coordinate system by this width and reduces the
remaining distance on the input path. Then, \pgfname\ either stays in
the current state or switches to another state, depending on yet
another property attached of the state.

The whole process stops when a special state called |final| is
reached. The segment of this state is immediately added to the output
path (it is often empty, though) and the process ends.




\subsection{Declaring Decorations}

The following command is used to declare a decoration. Essentially,
this command describes the decoration automaton.


\begin{command}{\pgfdeclaredecoration\marg{name}\marg{initial
      state}\marg{states}}
  This command declares a new decoration called \meta{name}. The
  \meta{states} argument contains a description of the decoration
  automaton's states and the transitions between them. The
  \meta{initial state} is the state in which the automaton starts.

  When the automaton is later applied to an input path, it keeps track
  of a certain position on the input path. This current point
  will ``travel along the path,'' each time being moved along by a
  certain distance. This will also work if the path is not a straight
  line. That is, it is permissible that the path curves are veers at a
  sharp angle.  It is also permissible that while travelling along the
  input path the current input segment ends and a new input segment starts. In this
  case, the remaining distance on the first input segment is subtracted
  from the \meta{dimension} and then we travelled along the second
  input segment for the remaining distance. This input segment may also end
  early, in which case we travel along the next input segment, and so
  on. Note that it cannot happen that we travel past the end of the
  input path since this would have caused an immediate switch to
  the |final| state.

  Note note that the computation of the path lengths has only a low
  accuracy because of \TeX's small math capabilities. Do not
  expect high accuracy alignments when using decorations (unless the
  input path consists only of horizontal and vertical lines).

  The \meta{states} argument should consist of |\state| commands, one
  for each state of the decoration automaton. The |\state| command is
  defined only when the \meta{states} argument is executed.

  \begin{command}{\state\marg{name}\oarg{options}\marg{code}}
    This command declares a new state inside the current decoration
    automaton. The state is named \meta{name}.
    
    When the decoration automaton is in state \meta{name}, the following things
    happen:
    \begin{enumerate}
    \item
      The \meta{options} are parsed. This may lead, see below, to a 
      state switch. When this happens, the following steps are not
      executed. The \meta{options} are executed one after the other in
      the given order. If an option causes a state switch, the switch
      is immediate, even if later options might cause a different
      state switch.
    \item
      The \meta{code} is executed in a \TeX-group with the current
      transformation matrix setup in such a way that the origin is on
      the input path at the current point (the point at the distance
      travelled up to now) and the coordinate system is rotated in
      such a way that the positive $x$-axis points in the direction of
      the tangent to the input path at the current point, while the
      positive $y$-axis points to the left of this tangent.
      
      As described earlier, the \meta{code} can have two different
      effects: If it just contains path construction commands, the
      decoration will produce an output path, which is extends the
      preexisting path. Here is an example:

\begin{codeexample}[]
\pgfdeclaredecoration{example}{initial}
{
  \state{initial}[width=10pt]
  {
    \pgfpathlineto{\pgfpoint{0pt}{5pt}}
    \pgfpathlineto{\pgfpoint{5pt}{5pt}}
    \pgfpathlineto{\pgfpoint{5pt}{-5pt}}
    \pgfpathlineto{\pgfpoint{10pt}{-5pt}}
    \pgfpathlineto{\pgfpoint{10pt}{0pt}}
  }
  \state{final}
  {
    \pgfpathlineto{\pgfpointdecoratedpathlast}
  }
}
\tikz[decoration=example]
{
  \draw [decorate]     (0,0) -- (3,0);
  \draw [red,decorate] (0,0) to [out=45,in=135] (3,0);
}
\end{codeexample}

    Alternatively, the \meta{code} can also contain the
    |\pgfusepath| command. This will use the path in usual manner,
    where ``the path'' is the preexisting path plus a part of the
    output path for the first invocation and the different parts of
    the rest of the output path for the following invocation. Here is
    an example:
            
\begin{codeexample}[]
\pgfdeclaredecoration{stars}{initial}{
  \state{initial}[width=15pt]
  {
    \pgfmathparse{round(rnd*100)}
    \pgfsetfillcolor{yellow!\pgfmathresult!orange}
    \pgfsetstrokecolor{yellow!\pgfmathresult!red}
    \pgfnode{star}{center}{}{}{\pgfusepath{stroke,fill}}
  }
  \state{final}
  {
    \pgfpathmoveto{\pgfpointdecoratedpathlast}
  }
}
\tikz\path[decorate, decoration=stars, star point ratio=2, star points=5,
           inner sep=0, minimum size=rnd*10pt+2pt]            
  (0,0) .. controls (0,2)  and (3,2)  .. (3,0)
        .. controls (3,-3) and (0,0)  .. (0,-3)
        .. controls (0,-5) and (3,-5) .. (3,-3);
\end{codeexample}

    \item
      After the \meta{code} has been executed (possibly more than
      once, if the |repeat state| option is used), the state switches to
      whatever state has been specified inside the \meta{options}
      using the |next state| option. If no |next state| has been
      specified, the state stays the same.
    \end{enumerate}

    The \meta{options} are executed with the key path set to
    |/pgf/decoration automaton|. The following keys are defined:
    \begin{key}{/pgf/decoration automaton/switch if less than=\meta{dimension}| to |\meta{new state}}
      When this key is encountered, \pgfname\ checks whether the
      remaining distance to the end of the input path is less than
      \meta{dimension}. If so, an immediate state switch to \meta{new
        state} occurs.
    \end{key}
    \begin{key}{/pgf/decoration automaton/switch if input segment less than=\meta{dimension}| to |\meta{new state}}
      When this key is encountered, \pgfname\ checks whether the
      remaining distance to the end of the current input segment of the
      input path is less than \meta{dimension}. If so, an immediate
      state switch to \meta{new state} occurs.
    \end{key}
    \begin{key}{/pgf/decoration automaton/width=\meta{dimension}}
      First, this option causes an immediate switch to the
      state |final| if the remaining distance on the input path is
      less than \meta{dimension}. The effect is the same as if you had
      said |switch if less than=|\meta{dimension}| to final| just
      before the |width| option.

      If no switch occurs, this option tells \pgfname\ the width of
      the segment. The current point will travel along the input path
      (as described earlier)   by this distance.
    \end{key}
    \begin{key}{/pgf/decoration automaton/repeat state=\meta{repetitions} (initially 0)}
      Tells \pgfname\ how long the automaton stays ``normally'' in the
      current state. This count is reset to \meta{repetitions} each
      time one of the |switch if| keys causes a state switch. If no
      state switches occur, the \meta{code} is executed and the
      repetition counter is decreased. Then, there is once more a
      chance of a state change caused by any of the \meta{options}. If
      no repetition occurs, the \meta{code} is executed 
      once more and the counter is decreased once more. When the
      counter reaches zero, the \meta{code} is executed once more,
      but, then, a different state is entered, as specified by the
      |next state| option.

      Note, that the maximum number of times the state will be executed 
      is $\meta{repetitions}+1$.
    \end{key}
    \begin{key}{/pgf/decoration automaton/next state=\meta{new state}}
      After the \meta{code} for state has been executed for the last
      time, a state switch to \meta{new state} is performed. If this
      option is not given, the next state is the same as the current state.
    \end{key}

    \begin{key}{/pgf/decoration automaton/if input segment is closepath=\meta{options}}
      This key checks whether the current input segment is a closepath
      operation. If so, the \meta{options} get executed; otherwise
      nothing happens. You can use this option to handle a closepath
      in some special way, for instance, switching to a new state in
      which |\pgfpathclose| is executed.
    \end{key}

    \begin{key}{/pgf/decoration automaton/auto end on length=\meta{dimension}}
      This key is just included for convenience, it does nothing that
      cannot be achieved using the previous options. The effect is the
      following: If the remaingin input path's length is at most
      \meta{dimension}, the decorated path is ended with a straight
      line to the end of the input path and, possibly, it is closed,
      namely if the input path ended with a closepath
      operation. Otherwise, it is checked whether the current input
      segment is a closepath segment and whether the remaining
      distance on the current input segment is at most
      \meta{distance}. If so, the a closepath operation is used to
      close the decorated path and the automaton continues with the
      next subpath, remaining in the current state.

      In all other cases, nothing happens.
    \end{key}

    \begin{key}{/pgf/decoration automaton/auto corner on length=\meta{dimension}}
      This key has the following effect: Firstly, the \TeX-if
      |\ifpgfdecoratepathhascorners| is false, nothing
      happens. Otherwise, it is tested whether the remaining distance
      on the current input segment is at most \meta{dimension}. If so,
      a lineto operation is used to reach the end of this input
      segment and the automaton continues with the next input segment,
      but remains in the current state.

      The main idea behind this option is to avoid having decoration
      segments ``overshoot'' past a corner.
    \end{key}

    You may sometimes wish to do computations outside the
    transformational \TeX-group of the current segment,
    so that these results of these computations are available in the
    next state. For this, the following two options are useful:
    
    \begin{key}{/pgf/decoration automaton/persistent precomputation=\meta{precode}}
      If the \meta{code} of state is executed, the \meta{precode} is
      executed first and it executed outside the \TeX-group of the
      \meta{code}. Note that when the \meta{precode} is executed, the
      transformation matrix is not setup.
    \end{key}

    \begin{key}{/pgf/decoration automaton/persistent postcomputation=\meta{postcode}}
      Works like the |persistent precomputation| option, only the
      \meta{postcode} is executed after (and also outside) the
      \TeX-group of the main \meta{code}.
    \end{key}
    
    There are a number of macros and dimensions which may be useful
    inside a decoration automaton. The following macros are available:
    
    \begin{command}{\pgfdecoratedpathlength}
      The length of the input path. If the input path consists of
      several input segments, this number is the sum of the lengths of the
      input segments. 
    \end{command}
    
    \begin{command}{\pgfdecoratedinputsegmentlength}
      The length of the current input segment of the input path. ``Current
      input segment''  refers to the input segment on which the current point
      lies. 
    \end{command}
		
    \begin{command}{\pgfpointdecoratedpathlast}
      The final point of the input path.
    \end{command}
    
    \begin{command}{\pgfpointdecoratedinputsegmentlast}
      The final point of the current input segment of the input path.
    \end{command}
		
    \begin{command}{\pgfdecoratedangle}
      The angle of the tangent to the decorated path at the \emph{origin}
      of the current segment. The transformation matrix applied at 
      the beginning of a state includes a rotation equivalent to 
      this angle.
    \end{command}
		
    The following \TeX\ dimension registers are also available inside the 
    automaton:
    
    \begin{command}{\pgfdecoratedremainingdistance}
      The remaining distance on the input path.
    \end{command}
		
    \begin{command}{\pgfdecoratedcompleteddistance}
      The completed distance on the input path.
    \end{command}
    
    \begin{command}{\pgfdecoratedinputsegmentremainingdistance}
      The remaining distance on the current input segment of the input path.
    \end{command}
    
    \begin{command}{\pgfdecoratedinputsegmentcompleteddistance}
      The completed distance on the current input segment of the input path.
    \end{command}

    Further keys and macros are defined and used by the decoration
    libraries, see Section~\ref{section-library-decorations}.
    
    The following example shows how these options can be used:
\begin{codeexample}[]
\pgfdeclaredecoration{complicated example decoration}{initial}
{
  \state{initial}[width=5pt,next state=up]
  { \pgfpathlineto{\pgfpoint{5pt}{0pt}} }
  
  \state{up}[width=5pt,next state=down]
  {
    \ifdim\pgfdecoratedremainingdistance>\pgfdecoratedcompleteddistance
      % Growing
      \pgfpathlineto{\pgfpoint{0pt}{\pgfdecoratedcompleteddistance}}
      \pgfpathlineto{\pgfpoint{5pt}{\pgfdecoratedcompleteddistance}}
      \pgfpathlineto{\pgfpoint{5pt}{0pt}}
    \else
      % Shrinking
      \pgfpathlineto{\pgfpoint{0pt}{\pgfdecoratedremainingdistance}}
      \pgfpathlineto{\pgfpoint{5pt}{\pgfdecoratedremainingdistance}}
      \pgfpathlineto{\pgfpoint{5pt}{0pt}}
    \fi%      
  }
  \state{down}[width=5pt,next state=up]
  {
    \ifdim\pgfdecoratedremainingdistance>\pgfdecoratedcompleteddistance
      % Growing
      \pgfpathlineto{\pgfpoint{0pt}{-\pgfdecoratedcompleteddistance}}
      \pgfpathlineto{\pgfpoint{5pt}{-\pgfdecoratedcompleteddistance}}
      \pgfpathlineto{\pgfpoint{5pt}{0pt}}
    \else
      % Shrinking
      \pgfpathlineto{\pgfpoint{0pt}{-\pgfdecoratedremainingdistance}}
      \pgfpathlineto{\pgfpoint{5pt}{-\pgfdecoratedremainingdistance}}
      \pgfpathlineto{\pgfpoint{5pt}{0pt}}
    \fi%      
  }
  \state{final}
  {
    \pgfpathlineto{\pgfpointdecoratedpathlast}
  }
}
\begin{tikzpicture}[decoration=complicated example decoration]
  \draw decorate{ (0,0) -- (3,0)};
  \fill [red!50,rounded corners=2pt]
    decorate {(.5,-2) -- ++(2.5,-2.5)} -- (3,-5) -| (0,-2) -- cycle;
\end{tikzpicture}
\end{codeexample}
  \end{command}
\end{command}



\subsubsection{Predefined Decorations}

The three decorations |moveto|, |lineto|, and |curveto| are predefined
and ``always available.'' They are mostly useful in conjunction with
meta-decorations. They are documented in
Section~\ref{section-library-decorations} alongside the other
decorations.



\subsection{Using Decorations}

Once a decoration has been declared, it can be used. 

\begin{environment}{{pgfdecoration}\marg{decoration list}}
  The \meta{environment contents} should contain commands for creating
  an path. This path is the basis for the \emph{input paths}
  for the decorations in the \meta{decoration list}. In detail, the
  following happens: 
  \begin{enumerate}
  \item
    The preexisting unused path is saved.
  \item 
    The path commands specified in \meta{environment contents} are
    executed and this resulting path is saved. The path is then
    divided into different \emph{input paths} as follows:
    The format for each item in \marg{decoration list} is 
    \begin{quote}
      \marg{decoration}\marg{length}\opt{\marg{before code}\marg{after code}}
    \end{quote}
    The \meta{before code} and the \meta{after code} are optional. The
    input path is divided into input paths as follows: The first input
    path consists of the first lines of the path specified in the
    \meta{environment contents} until the \meta{length}  of the first
    element of the \meta{decoration list} is reached. If this length
    is reached in the middle of a line, the line is broken up at this
    exact position. Then the second input path has the \meta{length}
    of the second element in the \meta{decoration list} and consists
    of the lines making up the following \meta{length} part of the
    path in the \meta{environment contents}, and so on.

    If the lengths in the \meta{decoration list}
    do not add up to the total length of the path in the
    \meta{environment contents}, either some  decorations are dropped
    (if their lengths add up to more than the length of the
    \meta{environment contents}) or
    the input path is not fully used (if their lengths  add up to less).
  \item
    The preexisting path is reinstalled.
  \item
    The decoration automata move along the input paths, thus creating
    (and  possibly using) the output paths. These output paths extend
    (unless they are used) the current path.
  \end{enumerate}
	 
  Some important points should be noted regarding the use of this
  environment:
  
  \begin{itemize}
  \item
    If \meta{environment contents} does not begin with 
    |\pgfpathmoveto|,	he last known point on the preexisting path is 
    assumed as the starting point.
  \item
    All except the last of any sequence of consecutive move-to commands 
    in \meta{environment contents} are discarded.
  \item
    Any move-to commands at end of \meta{environment contents} are 
    ignored.
  \item
    Any close-path commands on the input path are interpreted as 
    straight lines.
    Internally somthing a little more complicated is going on,
    however, a closed path on the input path has no effect on the 
    output path, other than causing the automaton to travel in a 
    straight line towards the location of the last move-to command on 
    the input path.
  \item
    Although tangent computations for the input path work with the
    last point on the preexisting path, no automatic move-to
    operations are issued for the output path. 
    If an output path commences with a line-to or curve-to when the 
    existing path is empty, an appropriate move-to command should be 
    inserted before the decoration commences.
  \item
    If a decoration uses its own path, the first time this happens the
    preexisting path is part of the path that is used at this point.
  \end{itemize}

  When the decoration automata ``work on'' their respective input
  paths, before the automaton starts, \meta{before code} is
  executed. After the decoration automaton has finished, \meta{after
    code} is executed. 
        
\begin{codeexample}[]
\begin{tikzpicture}[decoration={segment length=5pt}]
  \draw [help lines] grid (3,2);
  \begin{pgfdecoration}{{curveto}{1cm},{zigzag}{2cm},{curveto}{1cm}}
    \pgfpathmoveto{\pgfpointorigin}
    \pgfpathcurveto
      {\pgfpoint{0cm}{2cm}}{\pgfpoint{3cm}{2cm}}{\pgfpoint{3cm}{0cm}}
  \end{pgfdecoration}
\pgfusepath{stroke}
\end{tikzpicture}
\end{codeexample}

  When the lengths are evaluated, the dimension
  |\pgfdecoratedremainingdistance| holds the remaining distance on
  the entire decorated path, and |\pgfdecoratedpathlength| holds the
  total length of the path. Thus, it is possible to specify lengths
  like |\pgfdecoratedpathlength/3|.

\begin{codeexample}[]
\begin{tikzpicture}[decoration={segment length=5pt}]
  \draw [help lines] grid (3,2);
  \begin{pgfdecoration}{
      {curveto}{\pgfdecoratedpathlength/3},
      {zigzag}{\pgfdecoratedpathlength/3},
      {curveto}{\pgfdecoratedremainingdistance}
    }
    \pgfpathmoveto{\pgfpointorigin}
    \pgfpathcurveto
      {\pgfpoint{0cm}{2cm}}{\pgfpoint{3cm}{2cm}}{\pgfpoint{3cm}{0cm}}
  \end{pgfdecoration}
  \pgfusepath{stroke}
\end{tikzpicture}
\end{codeexample}

  When \meta{before code} is executed, the following macro is useful:
  \begin{command}{\pgfpointdecoratedpathfirst}
    Returns the point corresponding to the start of the current
    input path.
  \end{command}
  When \meta{after code} is executed, the following macro can be used:
  \begin{command}{\pgfpointdecoratedpathlast}
    Returns the point corresponding to the end of the current input
    path.
  \end{command}
  This means that if decorations do not use their own path, it is 
  possible to do something with them and and continue from the
  correct place. 
	
\begin{codeexample}[]
\begin{tikzpicture}
  \draw [help lines] grid (3,2);
  \begin{pgfdecoration}{
      {curveto}{\pgfdecoratedpathlength/3}
      {}
      {
        \pgfusepath{stroke}
      },
      {zigzag}{\pgfdecoratedpathlength/3}
      {
        \pgfpathmoveto{\pgfpointdecoratedpathfirst}
        \pgfdecorationsegmentlength=5pt
      }
      {
        \pgfsetstrokecolor{red}
        \pgfusepath{stroke}
        \pgfpathmoveto{\pgfpointdecoratedpathlast}
        \pgfsetstrokecolor{black}
      },
      {curveto}{\pgfdecoratedremainingdistance}
    }
    \pgfpathmoveto{\pgfpointorigin}
    \pgfpathcurveto
      {\pgfpoint{0cm}{2cm}}{\pgfpoint{3cm}{2cm}}{\pgfpoint{3cm}{0cm}}
  \end{pgfdecoration}
  \pgfusepath{stroke}
\end{tikzpicture}
\end{codeexample}
	
  After the |{decoration}| environment has finished, the following 
  macros are available:
	
  \begin{command}{\pgfdecorateexistingpath}
    The preexisting path before the environment was entered.
  \end{command}
	
  \begin{command}{\pgfdecoratedpath}
    The (total) input path (that is, the path created by the environment contents).
  \end{command}
	
  \begin{command}{\pgfdecorationpath}
    The output path. If the path is used, this macro contains only the
    last unused part of the output path.
  \end{command}
	
  \begin{command}{\pgfpointdecoratedpathlast}
    The final point of the input path.
  \end{command}
  
  \begin{command}{\pgfpointdecorationpathlast}
    The final point of the output path.
  \end{command}

  The following style is executed each time a decoration is used. You
  may use it to setup default options for decorations.
  \begin{stylekey}{/pgf/every decoration (initially \normalfont empty)}
    This sytle is executed for every decoration.
  \end{stylekey}
\end{environment}

\begin{plainenvironment}{{pgfdecoration}\marg{name}}
  The plain-\TeX{} version of the |{pgfdecorate}| environment.
\end{plainenvironment}

\begin{contextenvironment}{{pgfdecoration}\marg{name}}
  The Con\TeX t version of the |{pgfdecoration}| environment.
\end{contextenvironment}

For convenience, the following macros provide a ``shorthand''
for decorations (internally, they all use the |{pgfdecoration}|
environment).

\begin{command}{\pgfdecoratepath\marg{name}\marg{path commands}}
  Decorate the path described by \meta{path commands} with the
  decoration \meta{name}. This is equivalent to
\begin{codeexample}[code only]
\pgfdecorate{{name}{\pgfdecoratedpathlength}
             {\pgfdecoratebeforecode}{\pgfdecorateaftercode}}
  // the path commands.
\endpgfdecorate    
\end{codeexample}
\end{command}

\begin{command}{\pgfdecoratecurrentpath\marg{name}}
  Decorate the preexisting path with the decoration \meta{name}.
\end{command}

Both the above commands use the current definitons of the following
macros:

\begin{command}{\pgfdecoratebeforecode}
  Code executed as \meta{before code}, see the description of
  |\pgfdecorate|. 
\end{command}

\begin{command}{\pgfdecorateaftercode}
  Code executed as \meta{after code}, see the description of
  |\pgfdecorate|. 
\end{command}

It may sometimes be useful to add an additional transformation
for each segment of a decoration. The following command allows 
you to define such a ``last minute transformation.''

\begin{command}{\pgfsetdecorationsegmenttransformation\marg{code}}
  The \meta{code} will be executed at the very beginning of each
  segment. Note when applying multiple decorations, this will
  be reset between decorations, so it needs to be specified for
  each segment.

\begin{codeexample}[]
\begin{tikzpicture}
  \draw [help lines] grid (3,2);
  \begin{pgfdecoration}{
      {curveto}{\pgfdecoratedpathlength/3},
      {zigzag}{\pgfdecoratedpathlength/3}
      {
        \pgfdecorationsegmentlength=5pt
        \pgfsetdecorationsegmenttransformation{\pgftransformyshift{.5cm}}
      },
      {curveto}{\pgfdecoratedremainingdistance}
    }
    \pgfpathmoveto{\pgfpointorigin}
    \pgfpathcurveto
      {\pgfpoint{0cm}{2cm}}{\pgfpoint{3cm}{2cm}}{\pgfpoint{3cm}{0cm}}
  \end{pgfdecoration}
  \pgfusepath{stroke}
\end{tikzpicture}
\end{codeexample}
\end{command}




\subsection{Meta-Decorations}

\label{section-base-meta-decorations}

A meta-decoration provides an alternative way to decorate a path with 
mutiple decorations. It is, in essence, an automaton that decorates
an input path with decoration automatons. In general, however, the end
effect is still that a path is decorated with other paths, and the input 
path should be thought of as being divided into sub-input-paths, each with 
their own decoration. Like ordinary decorations, before a
meta-decoration can be used it must be declared.

\subsubsection{Declaring Meta-Decorations}

\begin{command}{\pgfdeclaremetadecorate\marg{name}\marg{initial state}\marg{states}}

  This command declares a new meta-decoration called \meta{name}. The
  \meta{states} argument contains a description of the meta-decoration
  automaton's states and the transitions between them. The
  \meta{initial state} is the state in which the automaton starts.
  
  The |\state| command is similar to the one found in 
  decoration declarations, and takes the same form:
  
  \begin{command}{\state\marg{name}\oarg{options}\marg{code}}
    Declares the state \meta{name} inside the current meta-decoration
    automaton. Unlike decorations, states in meta-decorations are not
    executed within a group, which makes the persistent computation
    options superfluous. Consider using an initial state with
    |width=0pt| to do precalculations that could speed the execution
    of the meta-decoration. 
    
    The \meta{options} are executed with the key path set to
    |/pgf/meta-decorations automaton/|, and the following keys are defined for 
    this path: 
    
    \begin{key}{/pgf/meta-decoration automaton/switch if less than=\meta{dimension}| to |\meta{new state}}
      This causes \pgfname\ to check whether the
      remaining distance to the end of the input path is less than
      \meta{dimension}, and, if so, to immediately switch to the state 
      \meta{new state}. When this key is evaluated, the macro 
      |\pgfmetadecoratedpathlength| will be defined as the total length of 
      the decoration path, allowing for values such as
      |\pgfmetadecoratedpathlength/8|.
    \end{key}
    
    \begin{key}{/pgf/meta-decoration automaton/width=\meta{dimension}}
      As always, this option will cause an immediate switch to the
      state |final| if the remaining distance on the input path is less than
      \meta{dimension}. 

      Otherwise, this option tells \pgfname\ the width of the
      ``meta-segment'', that is, the length of the sub-input-path
      which the decoration automaton specified  in \meta{code} will decorate.
    \end{key}
    
    \begin{key}{/pgf/meta-decoration automaton/next state=\meta{new state}}
      After the code for a state has been executed, a state switch to
      \meta{new state} is performed. If this option is not given, the
      next state is the same as the current state.
    \end{key}
    
    The code in \meta{code} is quite different from the code in a 
    decoration state. In almost all cases only the following three
    macros will be required: 
    
    \begin{command}{\decoration\marg{name}}
      This sets the decoration for the current state to \meta{name}.
      If this command is omitted, the |moveto| decoration will be
      used.
    \end{command}
    
    \begin{command}{\beforedecoration\marg{before code}}
      Defines \marg{before code} as (typically) \pgfname{} commands to be
      executed before the decoration is applied to the current segment.
      This command can be omitted.
      If you wish to set up some decoration specific parameters 
      such as segment length, or segment amplitude, then they
      can be set in \meta{before code}.
    \end{command}
   
    \begin{command}{\afterdecoration\marg{after code}}
      Defines \marg{after code} as commands to be executed afer the 
      decoration has been applied to the current segment.
      This command can be omitted.
    \end{command}
    
    There are some macros that may be usedful when creating 
    meta-decorations (note that they are all macros):
    
    \begin{command}{\pgfpointmetadecoratedpathfirst}
      When the \meta{before code} is executed,
      this macro stores the first point on the current
      sub-input-path. 
    \end{command}

    \begin{command}{\pgfpointmetadecoratedpathlast}
      When the \meta{after code} is executed,
      this macro stores the last point on the current
      sub-input-path. 
    \end{command}
    
    \begin{command}{\pgfmetadecoratedpathlength}
      The entire length of the entire input path.
    \end{command}    
    
    \begin{command}{\pgfmetadecoratedcompleteddistance}
      The completed distance on the entire input path.
    \end{command}
    
    \begin{command}{\pgfmetadecoratedremainingdistance}
      The remaining distance on the entire input path.
    \end{command}
    
    \begin{command}{\pgfmetadecoratedinputsegmentcompleteddistance}
      The completed distance on the current input segment of the entire
      input path.
    \end{command}
    
    \begin{command}{\pgfmetadecoratedinputsegmentremainingdistance}
      The remaining distance on the current input segment of the entire
      input path.
    \end{command}
  \end{command}

  Here is a complete example of a meta-decoration:

\begin{codeexample}[]
\pgfdeclaremetadecoration{arrows}{initial}{
  \state{initial}[width=0pt, next state=arrow]
  {	
    \pgfmathdivide{100}{\pgfmetadecoratedpathlength}
    \let\factor\pgfmathresult
    \pgfsetlinewidth{1pt}
    \pgfset{/pgf/decoration/segment length=4pt}
  }
  \state{arrow}[
    switch if less than=\pgfmetadecorationsegmentlength to final,
    width=\pgfmetadecorationsegmentlength/3, 
    next state=zigzag]
  {
    \decoration{curveto}
    \beforedecoration
    {   
      \pgfmathparse{\pgfmetadecoratedcompleteddistance*\factor}
      \pgfsetcolor{red!\pgfmathresult!yellow}
      \pgfpathmoveto{\pgfpointmetadecoratedpathfirst}
    }	
  }
  \state{zigzag}[width=\pgfmetadecorationsegmentlength/3, next state=end arrow]
  {
  	\decoration{zigzag}
  }
  \state{end arrow}[width=\pgfmetadecorationsegmentlength/3, next state=move]
  {
    \decoration{curveto}
    \beforedecoration{\pgfpathmoveto{\pgfpointmetadecoratedpathfirst}}
    \afterdecoration
    {	
      \pgfsetarrowsend{to}
      \pgfusepath{stroke}	
    }
  }  
  \state{move}[width=\pgfmetadecorationsegmentlength/2, next state=arrow]{}
  \state{final}{}
}

\tikz\draw[decorate,decoration={arrows,meta-segment length=2cm}]
  (0,0) .. controls (0,2)   and (3,2)   .. (3,0)
        .. controls (3,-2)  and (0,-2)  .. (0,-4)
        .. controls (0,-6)  and (3,-6)  .. (3,-8)
        .. controls (3,-10) and (0,-10) .. (0,-8);
\end{codeexample}

\end{command}


\subsubsection{Predefined Meta-decorations}

There are no predefined meta-decorations loaded with \pgfname{}.


\subsubsection{Using Meta-Decorations}

Using meta-decorations is ``simpler'' than using decorations, because
you can only use one meta-decoration per path.

\begin{environment}{{pgfmetadecoration}\marg{name}}
  This environment decorates the input path described in 
  \meta{environment contents}, with the	meta-decoration \meta{name}.
\end{environment}

\begin{plainenvironment}{{pgfmetadecoration}\marg{name}}
  The plain \TeX{} version of the |{pgfmetadecoration}| environment.
\end{plainenvironment}

\begin{contextenvironment}{{pgfmetadecoration}\marg{name}}
  The Con\TeX t version of the |{pgfmetadecoration}| environment.
\end{contextenvironment}

% Copyright 2006 by Till Tantau
%
% This file may be distributed and/or modified
%
% 1. under the LaTeX Project Public License and/or
% 2. under the GNU Free Documentation License.
%
% See the file doc/generic/pgf/licenses/LICENSE for more details.


\section{Using Paths}

\subsection{Overview}

Once a path has been constructed, it can be \emph{used} in different
ways. For example, you can draw the path or fill it or use it for
clipping.

Numerous graph parameters influence how a path will be rendered. For
example, when you draw a path, the line width is important as well as
the dashing pattern. The options that govern how paths are rendered
can all be set with commands starting with |\pgfset|. \emph{All
  options that influence how a path is rendered always influence the
  complete path.} Thus, it is not possible to draw part of a path
using, say, a red color and drawing another part using a green
color. To achieve such an effect, you must use two paths.

In detail, paths can be used in the following ways:

\begin{enumerate}
\item
  You can \emph{stroke} (also known as \emph{draw}) a path.
\item
  You can add \emph{arrow tips} to a path.
\item
  You can \emph{fill} a path with a uniform color.
\item
  You can \emph{clip} subsequent renderings against the path.
\item
  You can \emph{shade} a path.
\item
  You can \emph{use the path as bounding box} for the whole picture.
\end{enumerate}
You can also perform any combination of the above, though it makes no
sense to fill and shade a path at the same time.

To perform (a combination of) the first four actions, you can use the
following command:
\begin{command}{\pgfusepath\marg{actions}}
  Applies the given \meta{actions} to the current path. Afterwards,
  the current path is (globally) empty. The following actions are
  possible:
  \begin{itemize}
  \item \declare{|fill|}
    fills the path. See Section~\ref{section-fill} for further details.
\begin{codeexample}[]
\begin{pgfpicture}
  \pgfpathmoveto{\pgfpointorigin}
  \pgfpathlineto{\pgfpoint{1cm}{1cm}}
  \pgfpathlineto{\pgfpoint{1cm}{0cm}}
  \pgfusepath{fill}
\end{pgfpicture}
\end{codeexample}
  \item \declare{|stroke|} strokes the path. See
    Section~\ref{section-stroke} for further details. 
\begin{codeexample}[]
\begin{pgfpicture}
  \pgfpathmoveto{\pgfpointorigin}
  \pgfpathlineto{\pgfpoint{1cm}{1cm}}
  \pgfpathlineto{\pgfpoint{1cm}{0cm}}
  \pgfusepath{stroke}
\end{pgfpicture}
\end{codeexample}
  \item \declare{|draw|} has the same effect as |stroke|.
  \item \declare{|clip|}
    clips all subsequent drawings against the path. Always supresses
    arrow tips. See Section~\ref{section-clip} for further details. 
\begin{codeexample}[]
\begin{pgfpicture}
  \pgfpathmoveto{\pgfpointorigin}
  \pgfpathlineto{\pgfpoint{1cm}{1cm}}
  \pgfpathlineto{\pgfpoint{1cm}{0cm}}
  \pgfusepath{stroke,clip}
  \pgfpathcircle{\pgfpoint{1cm}{1cm}}{0.5cm}
  \pgfusepath{fill}
\end{pgfpicture}
\end{codeexample}
  \item \declare{|discard|}
    discards the path, that is, it is not used at all. Giving this
    option (alone) has the same effect as giving an empty options
    list.
  \end{itemize}
  When more than one of the first three actions are given, they are
  applied in the above ordering, regardless of their ordering in
  \meta{actions}. Thus, |{stroke,fill}| and |{fill,stroke}| have the
  same effect. 
\end{command}

To shade a path, use the |\pgfshadepath| command, which is explained
in Section~\ref{section-shadings}.



\subsection{Stroking a Path}
\label{section-stroke}

When you use |\pgfusepath{stroke}| to stroke a path, several graphic
parameters influence how the path is drawn. The commands for setting
these parameters are explained in the following.

Note that all graphic parameters apply to the path as a whole, never
only to a part of it.

All graphic parameters are local to the current |{pgfscope}|, but they
persists past \TeX\ groups, \emph{except} for the interior rule
(even-odd or nonzero) and the arrow tip kinds. The latter graphic
parameters only persist till the end of the current \TeX\ group, but 
this may change in the future, so do not count on this.

\subsubsection{Graphic Parameter: Line Width}

\begin{command}{\pgfsetlinewidth\marg{line width}}
  This command sets the line width for subsequent strokes (in the
  current |pgfscope|). The line width is given as a normal \TeX\
  dimension like |0.4pt| or |1mm|.

\begin{codeexample}[]
\begin{pgfpicture}
  \pgfsetlinewidth{1mm}
  \pgfpathmoveto{\pgfpoint{0mm}{0mm}}
  \pgfpathlineto{\pgfpoint{2cm}{0mm}}
  \pgfusepath{stroke}
  \pgfsetlinewidth{2\pgflinewidth} % double in size
  \pgfpathmoveto{\pgfpoint{0mm}{5mm}}
  \pgfpathlineto{\pgfpoint{2cm}{5mm}}
  \pgfusepath{stroke}
\end{pgfpicture}
\end{codeexample}
\end{command}

\begin{textoken}{\pgflinewidth}
  You can access the current line width via the \TeX\ dimension
  |\pgflinewidth|. It will be set to the correct line width, that is,
  even when a \TeX\ group closed, the value will be correct since it
  is set globally, but when a |{pgfscope}| closes, the value is set to
  the correct value it had before the scope.
\end{textoken}


\subsubsection{Graphic Parameter: Caps and Joins}

\begin{command}{\pgfsetbuttcap}
  Sets the line cap to a butt cap. See Section~\ref{section-cap-joins}
  for an explanation of what this is.
\end{command}
\begin{command}{\pgfsetroundcap}
  Sets the line cap to a round cap. See again
  Section~\ref{section-cap-joins}.
\end{command}
\begin{command}{\pgfsetrectcap}
  Sets the line cap to a square cap. See again
  Section~\ref{section-cap-joins}. 
\end{command}
\begin{command}{\pgfsetroundjoin}
  Sets the line join to a round join. See again
  Section~\ref{section-cap-joins}. 
\end{command}
\begin{command}{\pgfsetbeveljoin}
  Sets the line join to a bevel join. See again
  Section~\ref{section-cap-joins}. 
\end{command}
\begin{command}{\pgfsetmiterjoin}
  Sets the line join to a miter join. See again
  Section~\ref{section-cap-joins}. 
\end{command}
\begin{command}{\pgfsetmiterlimit\marg{miter limit factor}}
  Sets the miter limit to  \meta{miter limit factor}. See again 
  Section~\ref{section-cap-joins}. 
\end{command}

\subsubsection{Graphic Parameter: Dashing}

\begin{command}{\pgfsetdash\marg{list of even length of dimensions}\marg{phase}}
  Sets the dashing of a line. The first entry in the list specifies
  the length of the first solid part of the list. The second entry
  specifies the length of the following gap. Then comes the length of
  the second solid part, following by the length of the second gap,
  and so on. The \meta{phase} specifies where the first solid part
  starts relative to the beginning of the line.

\begin{codeexample}[]
\begin{pgfpicture}
  \pgfsetdash{{0.5cm}{0.5cm}{0.1cm}{0.2cm}}{0cm}
  \pgfpathmoveto{\pgfpoint{0mm}{0mm}}
  \pgfpathlineto{\pgfpoint{2cm}{0mm}}
  \pgfusepath{stroke}
  \pgfsetdash{{0.5cm}{0.5cm}{0.1cm}{0.2cm}}{0.1cm}
  \pgfpathmoveto{\pgfpoint{0mm}{1mm}}
  \pgfpathlineto{\pgfpoint{2cm}{1mm}}
  \pgfusepath{stroke}
  \pgfsetdash{{0.5cm}{0.5cm}{0.1cm}{0.2cm}}{0.2cm}
  \pgfpathmoveto{\pgfpoint{0mm}{2mm}}
  \pgfpathlineto{\pgfpoint{2cm}{2mm}}
  \pgfusepath{stroke}
\end{pgfpicture}
\end{codeexample}

  Use |\pgfsetdash{}{0pt}| to get a solid dashing.
\end{command}

\subsubsection{Graphic Parameter: Stroke Color}

\begin{command}{\pgfsetstrokecolor\marg{color}}
  Sets the color used for stroking lines to \meta{color}, where
  \meta{color} is a \LaTeX\ color like |red| or |black!20!red|. Unlike
  the |\color| command, the effect of this command lasts till the end
  of the current |{pgfscope}| and not till the end of the current
  \TeX\ group.

  The color used for stroking may be different from the color used for
  filling. However, a |\color| command will always ``immediately
  override'' any special settings for the stroke and fill colors.

  In plain \TeX, this command will also work, but the problem of
  \emph{defining} a color arises. After all, plain \TeX\ does not
  provide \LaTeX\ colors. For this reason, \pgfname\ implements a
  minimalistic ``emulation'' of the |\definecolor|, |\colorlet|, and
  |\color| commands. Only gray-scale and rgb colors are supported. For
  most cases this turns out to be enough.

\begin{codeexample}[]
\begin{pgfpicture}
  \pgfsetlinewidth{1pt}
  \color{red}
  \pgfpathcircle{\pgfpoint{0cm}{0cm}}{3mm} \pgfusepath{fill,stroke}
  \pgfsetstrokecolor{black}
  \pgfpathcircle{\pgfpoint{1cm}{0cm}}{3mm} \pgfusepath{fill,stroke}
  \color{red}
  \pgfpathcircle{\pgfpoint{2cm}{0cm}}{3mm} \pgfusepath{fill,stroke}
\end{pgfpicture}
\end{codeexample}
\end{command}

\begin{command}{\pgfsetcolor\marg{color}}
  Sets both the stroke and fill color. The difference to the normal
  |\color| command is that the effect lasts till the end of the
  current |{pgfscope}|, not only till the end of the current \TeX\
  group. 
\end{command}


\subsubsection{Graphic Parameter: Stroke Opacity}

You can set the stroke opacity using |\pgfsetstrokeopacity|. This
command is described in Section~\ref{section-transparency}.


\subsubsection{Inner Lines}

When a path is stroked, it is possible to request that it is stroked
twice, the second time with a different line width and a different
color. This is a useful effect for creating ``double'' lines, for
instance by setting the line width to 2pt and stroking a black line
and then setting the inner line width to 1pt and stroking a white
line on the same path as the original path. This results in what looks
like two lines, each of thickness 0.5pt, spaced 1pt apart.

You may wonder why there is direct support for this ``double
stroking'' in the basic layer. After all, this effect is easy to
achieve ``by hand''. The main reason is that arrow tips must be
treated in a special manner when such ``double lines'' are
present. First, the order of actions is important: First, the (thick)
main line should be stroked, then the (thin) inner line, and only then
should the arrow tip be drawn. Second, the way an arrow tip looks
typically depends strongly on the width of the inner line, so the
arrow tip code, which is part of the basic layer, needs access to the
inner line thickness.

Two commands are used to set the inner line width and color.

\begin{command}{\pgfsetinnerlinewidth\marg{dimension}}
  This command sets the width of the inner line. Whenever a path is
  stroked (and only then), it will be stroked normally and, afterward,
  it is stroked once more with the color set to the inner line color
  and the line width set to \meta{dimension}.

  In case arrow tips are added to a path, the path is first stroked
  normally, then the inner line is stroked, and then the arrow tip is
  added. In case the main path is shortened because of the added arrow
  tip, this shortened path is double stroked, not the original path
  (which is exactly what you want).

  When the inner line width is set to 0pt, which is the default, no
  inner line is stroked at all (not even a line of width 0pt). So, in
  order to ``switch off'' double stroking, set \meta{dimension}
  to~|0pt|.

  The setting of the inner line width is local to the current \TeX\
  group and \emph{not} to the current \pgfname\ scope.

  Note that inner lines will \emph{not} be drawn for paths that are
  also used for clipping. However, this may change in the future, so
  you should not depend on this.

\begin{codeexample}[]
\begin{pgfpicture}
  \pgfpathmoveto{\pgfpointorigin}
  \pgfpathlineto{\pgfpoint{1cm}{1cm}}
  \pgfpathlineto{\pgfpoint{1cm}{0cm}}
  \pgfsetlinewidth{2pt}
  \pgfsetinnerlinewidth{1pt}
  \pgfusepath{stroke}
\end{pgfpicture}
\end{codeexample}  
\end{command}


\begin{command}{\pgfsetinnerstrokecolor\marg{color}}
  This command sets the \meta{color} that is to be used when the inner
  line is stroked. The effect of this command is also local to the
  current \TeX\ group.

\begin{codeexample}[]
\begin{pgfpicture}
  \pgfpathmoveto{\pgfpointorigin}
  \pgfpathlineto{\pgfpoint{1cm}{1cm}}
  \pgfpathlineto{\pgfpoint{1cm}{0cm}}
  \pgfsetlinewidth{2pt}
  \pgfsetinnerlinewidth{1pt}
  \pgfsetinnerstrokecolor{red!50}
  \pgfusepath{stroke}
\end{pgfpicture}
\end{codeexample}  
\end{command}


\subsection{Arrow Tips on a Path}
\label{section-tips}

After a path has been drawn, \pgfname\ can add arrow tips at the
ends, depending on how the |tips| key is set, on whether |stroke| or
|clip| arw used and on whether the path contains closed subpaths. The
exact rules when arrow tips are added are explained in
Section~\ref{section-arrow-tips-where}.

\begin{command}{\pgfsetarrowsstart\marg{start arrow tip specification}}
  Sets the arrow tip kind used at the start of a (possibly curved)
  path. The syntax of the \meta{start arrow specification} is detailed
  in Section~\ref{section-arrow-spec}. 

  To ``clear'' the start arrow, say |\pgfsetarrowsstart{}|.
\begin{codeexample}[]
\begin{pgfpicture}
  \pgfsetarrowsstart{Latex[length=10pt]}
  \pgfpathmoveto{\pgfpointorigin}
  \pgfpathlineto{\pgfpoint{1cm}{0cm}}
  \pgfusepath{stroke}
  \pgfsetarrowsstart{Computer Modern Rightarrow}
  \pgfpathmoveto{\pgfpoint{0cm}{2mm}}
  \pgfpathlineto{\pgfpoint{1cm}{2mm}}
  \pgfusepath{stroke}
\end{pgfpicture}
\end{codeexample}

  The effect of this command persists only till the end of the current
  \TeX\ scope.
\end{command}

\begin{command}{\pgfsetarrowsend\marg{end arrow tip specification}}
  Sets the arrow tip kind used at the end of a path.
\begin{codeexample}[]
\begin{pgfpicture}
  \pgfsetarrowsstart{Latex[length=10pt]}
  \pgfsetarrowsend{Computer Modern Rightarrow}
  \pgfpathmoveto{\pgfpointorigin}
  \pgfpathlineto{\pgfpoint{1cm}{0cm}}
  \pgfusepath{stroke}
\end{pgfpicture}
\end{codeexample}
\end{command}

\begin{command}{\pgfsetarrows\marg{argument}}
  The \meta{argument} can be of the form \meta{start arrow tip
    specifciation}|-|\meta{end arrow tip specification}. In this
  case, both the start and the end arrow specification are set:
\begin{codeexample}[]
\begin{pgfpicture}
  \pgfsetarrows{Latex[length=10pt]->>}
  \pgfpathmoveto{\pgfpointorigin}
  \pgfpathlineto{\pgfpoint{1cm}{0cm}}
  \pgfusepath{stroke}
\end{pgfpicture}
\end{codeexample}
  Alternatively, \meta{argument} can be of the form |[|\meta{arrow
    keys}|]|. In this case, the \meta{arrow keys} will be set for all
  arrow tips in the current scope, see Section~\ref{section-arrow-scopes}.
\end{command}


\begin{command}{\pgfsetshortenstart\marg{dimension}}
  This command will shortened the start of every stroked path by the
  given dimension. This shortening is done in addition to automatic
  shortening done by a start arrow, but it can be used even if no
  start arrow is given.

  It is usually better to use the |sep| key with arrow tips.

  This command is useful if you wish arrows or lines to ``stop shortly
  before'' a given point.
\begin{codeexample}[]
\begin{pgfpicture}
  \pgfpathcircle{\pgfpointorigin}{5mm}
  \pgfusepath{stroke}
  \pgfsetarrows{Latex-}
  \pgfsetshortenstart{4pt}
  \pgfpathmoveto{\pgfpoint{5mm}{0cm}} % would be on the circle
  \pgfpathlineto{\pgfpoint{2cm}{0cm}}
  \pgfusepath{stroke}
\end{pgfpicture}
\end{codeexample}
\end{command}
  
\begin{command}{\pgfsetshortenend\marg{dimension}}
  Works like |\pgfsetshortenstart|.
\end{command}



\subsection{Filling a Path}
\label{section-fill}

Filling a path means coloring every interior point of the path with
the current fill color. It is not always obvious whether a point is
``inside'' a  path when the path is self-intersecting and/or consists
or multiple parts. In this case either the nonzero winding number rule
or the even-odd crossing number rule is used to decide which points
lie ``inside.'' These rules are explained in
Section~\ref{section-rules}. 

\subsubsection{Graphic Parameter: Interior Rule}

You can set which rule is used using the following commands:

\begin{command}{\pgfseteorule}
  Dictates that the even-odd rule is used in subsequent fillings in
  the current \emph{\TeX\ scope}. Thus, for once, the effect of this
  command does not persist past the current \TeX\ scope.

\begin{codeexample}[]
\begin{pgfpicture}
  \pgfseteorule
  \pgfpathcircle{\pgfpoint{0mm}{0cm}}{7mm}
  \pgfpathcircle{\pgfpoint{5mm}{0cm}}{7mm}
  \pgfusepath{fill}
\end{pgfpicture}
\end{codeexample}
\end{command}

\begin{command}{\pgfsetnonzerorule}
  Dictates that the nonzero winding number rule is used in subsequent
  fillings in the current \TeX\ scope. This is the default.

\begin{codeexample}[]
\begin{pgfpicture}
  \pgfsetnonzerorule
  \pgfpathcircle{\pgfpoint{0mm}{0cm}}{7mm}
  \pgfpathcircle{\pgfpoint{5mm}{0cm}}{7mm}
  \pgfusepath{fill}
\end{pgfpicture}
\end{codeexample}
\end{command}

\subsubsection{Graphic Parameter: Filling Color}

\begin{command}{\pgfsetfillcolor\marg{color}}
  Sets the color used for filling paths to \meta{color}. Like the
  stroke color, the effect lasts only till the next use of |\color|. 
\end{command}


\subsubsection{Graphic Parameter: Fill Opacity}

You can set the fill opacity using |\pgfsetfillopacity|. This
command is described in Section~\ref{section-transparency}.

\subsection{Clipping a Path}
\label{section-clip}

When you add the |clip| option, the current path is used for
clipping subsequent drawings. The same rule as for filling is used to
decide whether a point is inside or outside the path, that is, either
the even-odd rule or the nonzero rule.

Clipping never enlarges the clipping area. Thus, when you clip against
a certain path and then clip again against another path, you clip
against the intersection of both.

The only way to enlarge the clipping path is to end the |{pgfscope}|
in which the clipping was done. At the end of a |{pgfscope}| the
clipping path that was in force at the beginning of the scope is
reinstalled. 

\subsection{Using a Path as a Bounding Box}
\label{section-using-bb}

When you add the |use as bounding box| option, the bounding box of the
picture will be enlarged such that the path in encompassed, but any
\emph{subsequent} paths of the current \TeX\ scope will not have any
effect on the size of the bounding box. Typically, you use this
command at the very beginning of a |{pgfpicture}| environment. Alternatively, you can use |\pgfresetboundingbox|, followed by |\pgfusepath{use as bounding box}| to overrule the picture's bounding box completely.

\begin{codeexample}[]
Left
\begin{pgfpicture}
  \pgfpathrectangle{\pgfpointorigin}{\pgfpoint{2ex}{1ex}}
  \pgfusepath{use as bounding box} % draws nothing

  \pgfpathcircle{\pgfpointorigin}{2ex}
  \pgfusepath{stroke}
\end{pgfpicture}
right.
\end{codeexample}


% Copyright 2013 by Till Tantau
%
% This file may be distributed and/or modified
%
% 1. under the LaTeX Project Public License and/or
% 2. under the GNU Free Documentation License.
%
% See the file doc/generic/pgf/licenses/LICENSE for more details.


\section{Defining New Arrow Tip Kinds}
\label{section-arrows}


\subsection{Overview}

In present section we have a look at how you can define new arrow tips
for use in \pgfname. The low-level commands for selecting which arrow
tips are to be used have already been described in
Section~\ref{section-tips}, the general syntax rules for using arrows
are detailed in Section~\ref{section-tikz-arrows}. Although
Section~\ref{section-tikz-arrows} describes the use of arrows in
\tikzname, in reality, \tikzname\ itself does not actually do anything
about arrow tips; all of the functionality is implemented on the
\pgfname\ level in the commands described in
Section~\ref{section-tikz-arrows}. Indeed, even the |/.tip| key
handler described in Section~\ref{section-tikz-arrows} is actually
implemented on the \pgfname\ layer.

What has \emph{not} yet been covered is how you can actually define a
complete new arrow tip. In \pgfname, arrows are ``meta-arrows'' in the
same way that fonts in \TeX\ are ``meta-fonts.'' When a meta-arrow is
resized, it is not simply scaled, but a possibly complicated
transformation is applied to the size.

A meta-font is not one particular font at a specific size with a
specific stroke width (and with a large number of other parameters
being fixed). Rather, it is a ``blueprint'' (actually, more like a
program) for generating such a font at a particular size and
width. This allows the designer of a meta-font to make sure that, say,
the font is somewhat thicker and wider at very small sizes. To
appreciate the difference: Compare the following texts: ``Berlin'' and
``\tikz{\node [scale=2,inner sep=0pt,outer sep=0pt]{\tiny
    Berlin};}''. The first is a ``normal'' text, the second is the tiny
version scaled by a factor of two. Obviously, the first look
better. Now, compare  ``\tikz{\node [scale=.5,inner sep=0pt,outer
  sep=0pt]{Berlin};}'' and ``{\tiny Berlin}''. This time, the normal
text was scaled down, while the second text is a ``normal'' tiny
text. The second text is easier to read.

\pgfname's meta-arrows work in a similar fashion: The shape of an
arrow tip can vary according to a great number of parameters, the line
width of the arrow tip being one of them. Thus, an arrow tip drawn at
a line width of 5pt will typically \emph{not} be five times as large
as an arrow tip of line width 1pt. Instead, the size of the arrow will
get bigger only slowly as the line width increases. 

To appreciate the difference, here are the |Latex| and
|Classical TikZ Rightarrow| arrows, as drawn by \pgfname\ at four different sizes:

\medskip
\begin{tikzpicture}[1/.tip=Latex, 2/.tip=Classical TikZ Rightarrow]
  \draw[-1, line width=0.1pt] (0pt,0ex) -- +(3,0)  node[thin,right] {line width is 0.1pt};
  \draw[-1, line width=0.4pt] (0pt,-2em) -- +(3,0) node[thin,right] {line width is 0.4pt};
  \draw[-1, line width=1.2pt] (0pt,-4em) -- +(3,0) node[thin,right] {line width is 1.2pt};
  \draw[-1, line width=5pt]   (0pt,-6em) -- +(3,0) node[thin,right] {line width is 5pt};

  \draw[-2, line width=0.1pt] (6cm,0ex) -- +(3,0)  node[thin,right] {line width is 0.1pt};
  \draw[-2, line width=0.4pt] (6cm,-2em) -- +(3,0) node[thin,right] {line width is 0.4pt};
  \draw[-2, line width=1.2pt] (6cm,-4em) -- +(3,0) node[thin,right] {line width is 1.2pt};
  \draw[-2, line width=5pt]   (6cm,-6em) -- +(3,0) node[thin,right] {line width is 5pt};
\end{tikzpicture}

\medskip
Here, by comparison, are the same arrows when they are simply ``resized'':

\medskip
\begin{tikzpicture}[1/.tip=Latex, 2/.tip=Classical TikZ Rightarrow]
  \draw[-{1[length=1pt]}, line width=0.1pt] (0pt,0ex) -- +(3,0)  node[thin,right] {line width is 0.1pt};
  \draw[-{1[length=4pt]}, line width=0.4pt] (0pt,-2em) -- +(3,0) node[thin,right] {line width is 0.4pt};
  \draw[-{1[length=12pt]}, line width=1.2pt] (0pt,-4em) -- +(3,0) node[thin,right] {line width is 1.2pt};
  \draw[-{1[length=32pt]}, line width=5pt]   (0pt,-6em) -- +(3,0) node[thin,right] {line width is 5pt};

  \draw[-{2[length=0.455pt]}, line width=0.1pt] (6cm,0ex) -- +(3,0)  node[thin,right] {line width is 0.1pt};
  \draw[-{2[length=1.82pt]}, line width=0.4pt] (6cm,-2em) -- +(3,0) node[thin,right] {line width is 0.4pt};
  \draw[-{2[length=5.46pt]}, line width=1.2pt] (6cm,-4em) -- +(3,0) node[thin,right] {line width is 1.2pt};
  \draw[-{2[length=14.56pt]}, line width=5pt]   (6cm,-6em) -- +(3,0) node[thin,right] {line width is 5pt};
\end{tikzpicture}

\bigskip
As can be seen, simple scaling produces arrow tips that are way too
large at larger sizes and way too small at smaller sizes.

In addition to the line width, other options may also influence the
appearance of an arrow tip. In particular, the width of the inner line
(the line used to create the effect of a double line) influences arrow
tips as well as other options that are specific to the arrow tip.


\subsection{Terminology}
\label{section-arrow-terminology}

Before we have a look at the exact commands used for defining arrow
tips, we need to fix some terminology. Consider the following drawing
of an arrow tip where the arrow tip is drawn transparently so that we
can see what is ``happening behind it'':

\begin{tikzpicture}
  \draw [red!50, ,line width=1cm] (0,0) -- (4,0);
  \path [tips, opacity=.25,line width=1cm, -{Stealth[black,line width=0pt,length=4cm, width=4cm, inset=1cm]}] (0,0) -- (6,0);
  
  \draw [->,thick] (1,0) -- (8,0) node [right] {$x$-axis};
  \draw [->,thick] (5,-2.25) -- (5,2.25) node [above] {$y$-axis};

  \foreach \i in {-3,-2,-1,1,2} \draw (\i+5,-1mm) -- (\i+5,1mm) node [above] {\small$\i$};
  \foreach \i in {-2,-1,1,2} \draw (49mm,\i) -- (51mm,\i) node [right] {\small$\i$};;
\end{tikzpicture}

I have also added a coordinate system. The code for drawing an arrow
tip always draws it in the way shown above: Pointing right along the
$x$-axis.

We will use the following terminology:
\begin{itemize}
\item The point where tip of the arrow ends is called the
  \emph{tip end}. It is at $(1,0)$ in our example and we always
  assume it to lie on the $x$-axis, so we just treat it as a distance,
  1 in this case. This is the position where the original path was
  supposed to end (so if the arrow tip had not been added to the red
  path, it would have ended here).
\item The \emph{back end} of the arrow is where a vertical line just
  to the left of the arrow intersects the $x$-axis. In our case, this
  is the point $(-3,0)$ and again we treat it as a distance, $-3$ in
  this case.
\item The \emph{line end} is the position where the path now
  ends. This should be a position inside the arrow head that gets
  ``covered'' by the path. Note that a path may have a round or a rect
  head and should still be covered. Clearly, necessary shortening of
  the path will be the difference between the tip end and the line
  end.
\item The \emph{visual back end} is the position where the path and
  the the arrow head ``meet last'' on the path. In our case, because
  of the inset, the visual back end is not the same as the back end:
  The arrow ends ``visually'' at $(-2,0)$. The difference between the
  back end and the visual back end is important when the arrow tip is
  flexed, see Section~\ref{section-arrow-flex} for an explanation of
  flexing.
\item There is also a \emph{visual tip end}, the counterpart of the
  visual back end for the front. In our case, the visual tip end and
  the tip end obviously coincide, but if we were to reverse the arrow
  tip, the visual tip end would be different form the tip end (while
  the visual back end would then coincide with the new back end).
\item There are four points that make up the \emph{convex hull} of the
  arrow tip: $(1,0)$, $(-3,2)$, and $(-3,-2)$.

  Normally, \pgfname\ automatically keeps track of a bounding box of
  everything you draw. However, since arrow tips are drawn so often,
  \pgfname\ caches the code needing for drawing arrow tips internally
  and because of this cache it cannot determine the size of the arrow
  tip just based on the drawing commands used for drawing the
  tip. Instead, a convex hull of the arrow tip must be explicitly
  provided in the definition.
\end{itemize}

When you design a new arrow tip, all of the above parameters must be
defined.

\subsection{Caching and Rendering of Arrows}

As a last preparation for the description of the commands for
declaring arrows, it is important to understand the exact process by
which \pgfname\ draws arrows.

\begin{enumerate}
\item First, you have to define an arrow tip kind using
  |\pgfdeclarearrow{name=foo,...|. This will tell \pgfname\ that  %}
  |foo| is now the name of an arrow tip. In particular, the 
  parser for arrow tip specifications will now treat |foo| as the name
  of an arrow tip and will not try to consider |f|, |o|, and |o| as
  the names of single-char shorthands.

  Other than storing the definitions in the declaration internally,
  this command has little other effect. In particular, no drawing or
  other processing takes place.
\item Now assume that at some point the arrow tip |foo| is actually
  used. In this case, certain options may have been set, for instance
  the user may have requested the arrow tip
  |foo[length=5pt,open]|. What happens next depends on whether it is
  the first time the arrow tip |foo| is used with \emph{these exact
    options} ornot.
\item Assume that is the first time |foo| is requested at a length of
  5pt and in an ``open'' version. \pgfname\ now retrieves the
  definition of the arrow tip kind that it stored in the first step
  and executes the so-called \emph{setup} code. When this code is
  executed, all the options will be in force (for
  instance, |\pgfarrowlength| will equal |5pt| in our case). The job
  of the setup code is two-fold: First, it needs to compute all of the
  parameters listed in Section~\ref{section-arrow-terminology}, that
  is, it has to compute where the tip end will lie in the arrow tip's
  coordinate system \emph{at the particular size of 5pt}, where the
  back end will be, where the convex hull points lie, and so
  on. Second, the setup code should precompute values that will be
  important for constructing the path of the arrow. In our example,
  there is little to do in this regard, but for more complicated
  arrows, all time-consuming preparations are done now.

  It is \emph{not} the job of the setup to actually draw the arrow
  tip, only to ``prepare'' this as much as possible.

  The setup code will always be executed only once for each arrow tip
  kind for a given set of options. Thus, when a user uses
  |foo[length=5pt,open]| once more later anywhere in the document, the
  setup code will not be executed again.
\item The next thing that happens is that we have a look at the
  \emph{drawin code} stored in the |code| field of the arrow. In our 
  example, the drawing code would consist of creating a filled path
  with four   straight segments.

  In most cases, what happens now is that the drawing code is executed in a
  special sandbox in which the low-level driver commands that do the
  actual drawing are intercepted and stored away in a so-called
  \emph{cache}. Once such a cache has been created, its contents will
  be reused whenever |foo[length=5pt,open]| is requested by a user and
  just like the setup code, the drawing code will not be executed
  again.

  There are, however, two cases in which the drawing code gets executed
  each time the arrow is used: First, an arrow tip kind can specify
  that this should always happen by saying |cachable=false| in its
  definition. This is necessary if the drawing code contains
  low-level drawing commands that cannot be intercepted such as a use
  of |\pgftext| for arrow tips that ``contain text.'' Second, when the
  |bend| option is used, the same arrow tip will look different each
  time it is used, namely in dependence on the exact curvature of the
  path to which it is added.

  Because the drawing code may be executed several times, while the
  setup code may not, we must find a way to ``communicate'' the
  values computed by the setup code to the drawing code. This is done
  by explicitly calling |\pgfarrowssave| inside the setup
  code. Whatever is ``saved'' in this way is restored each time before
  the drawing code is executed.
\end{enumerate}

As can be seen, the process is a bit involved, but it leads to
a reasonably fast arrow tip management.


\subsection{Declaring an Arrow Tip Kind}

\begin{command}{\pgfdeclarearrow\marg{config}}
  This command is both used to define a new arrow tip kind and to to
  declare a so-called shorthand. We have a look at the case that a
  complete new arrow tip kind is created and then have a look how the
  command can be used to create shorthands.
  
  \medskip
  \noindent\textbf{Defining a Complete New Arrow Tip Kind.}  
  The \meta{config} is a key--value list in which different keys are
  used to setup the to-be defined arrow. The following keys can be given:

  \begin{itemize}
  \item \declare{|name|}|=|\meta{name} or |name=|\meta{start name}|-|\meta{end
      name}

    This defines the name of the arrow tip. It is legal to define an
    arrow tip a second time, in this case the previous definition will
    be overwritten in the current \TeX\ scope. It is customary to use
    a name with an uppercase fist letter for a ``complete'' arrow tip
    kind. Short names and lower case names should be used for
    shorthands that change their meaning inside a document, while
    arrow tips with uppercase first letters should not be redefined.

    If the name contains a hyphen, the second syntax is assumed and
    everything before the hyphen will be the name used in start arrow
    specifications, while the text after the hyphen is the name used
    in end specifications.
  \item \declare{|parameters|}|=|\marg{list of macros}
    
    As explained earlier, an arrow tip typically needs to be redrawn
    each time an option like |length| or |inset| is changed. However,
    for some arrow tips, the |inset| has no influence, while for other
    it is important whether the arrow is reversed or not. (How keys
    like |length| actually set \TeX\ dimensions like |\pgfarrowlength|
    is explained in Section~\ref{section-arrow-options}.)

    The job of the |parameters| key is to specify which dependencies
    the arrow tip has. Everything that will influence any of the
    parameters computed in the setup code or used in the drawing code
    should be listed here.

    The \meta{list of macros} will be used inside a
    |\csname|-|\endcsname| pair and should expand to the current
    values of the relevant parameters have. For example, if the arrow
    tip depends on the current value of |\pgfarrowlength| and
    |\pgfarrowwidth| only, then \meta{list of macros} should be set to
    |\the\pgfarrowlength,\the\pgfarrowwidth|. (Actually, the comma is
    optional, the \meta{list of macros} does not really have to be a
    list, just something that can be expanded unambiuously.)
    
    Note that the line width (|\pgflinewidth|) and the inner line
    width (|\pgfinnerlinewidth|) are always parameters and need not be
    specified in the |parameters|.

    It is important to get this parameter right. Otherwise, arrow tips
    may look wrong because \pgfname\ thinks that it can reuse some
    code when, in reality, this code actually depends on a parameter
    not listed here.
    
  \item \declare{|setup code|}|=|\marg{code}

    When an arrow tip is used, the value stored in |parameters| is
    expanded and it is tested whether the result was encountered
    before. If not, the \meta{code} gets executed (only this
    once). The code can now do aribtrarily complicated computations
    the prepare the later drawing of the arrow tip. Also the
    \meta{code} must specify the different tip and back ends and the
    convex hull points. This is done by calling the following macros
    inside the \meta{code}:

    \begin{command}{\pgfarrowssettipend\marg{dimension}}
      When this command is called inside the setup code of an arrow
      tip, it specifies that the tip of the drawn arrow will end
      exactly at \meta{dimension}. For example, for our earlier
      example of the large arrow tip, where the tip end was at 1cm, we
      would call
\begin{codeexample}[code only]
\pgfarrowssettipend{1cm}        
\end{codeexample}
      Note that for efficience reasons, the \meta{dimension} is not
      passed through |\pgfmathsetlength|; rather what happens is that
      |\pgf@x=|\meta{dimension} gets executed. In particular, you can
      pack further computations into the \meta{dimension} by simply
      starting it with a number and then appending some code that
      modifies |\pgf@x|. Here is an example where instead of 1cm we
      use $1\mathrm{cm} - \frac12\mathrm{linewidth}$ as the tip end:
\begin{codeexample}[code only]
\pgfarrowssettipend{1cm\advance\pgf@x by-.5\pgflinewidth}        
\end{codeexample}
      If the command is not called at all inside the setup code, the
      tip end is set to |0pt|.
    \end{command}

    \begin{command}{\pgfarrowssetbackend\marg{dimension}}
      Works like the command for the tip end, only it sets the back
      end. In our example we would call
\begin{codeexample}[code only]
\pgfarrowssettipend{-3cm}        
\end{codeexample}
      Defaults to |0pt|.
    \end{command}

    \begin{command}{\pgfarrowssetlineend\marg{dimension}}
      Sets the line end, so in the example we have
      |\pgfarrowssettipend{-1cm}|. Default to |0pt|.
    \end{command}

    \begin{command}{\pgfarrowssetvisualbackend\marg{dimension}}
      Sets the visual back end, |\pgfarrowssetvisualbackend{-2cm}| in
      our example. Default to the value of the normal back end.
    \end{command}

    \begin{command}{\pgfarrowssetvisualtipend\marg{dimension}}
      Sets the visual tip end. Default to the value of the normal tip
      end and, thus, we need not set it in our example.
    \end{command}

    \begin{command}{\pgfarrowshullpoint\marg{x dimension}\marg{y dimension}}
      Adds a point to the convex hull of the arrow tip. As for the
      previous commands, no math parsing is done; instead \pgfname\
      says |\pgf@x=|\meta{x dimension} and then |\pgf@y=|\meta{y
        dimension}. Thus, both ``dimensions'' can contain code for
      advancing and thus modifying |\pgf@x| and |\pgf@y|.

      In our example we would write
\begin{codeexample}[code only]
\pgfarrowshullpoint{1cm}{0pt}
\pgfarrowshullpoint{-3cm}{2cm}
\pgfarrowshullpoint{-3cm}{-2cm}
\end{codeexample}
    \end{command}

    \begin{command}{\pgfarrowssave\marg{macro}}
      As explained earlier, the setup code needs to ``communicate''
      with the drawing code via ``saved values.'' This command get the
      name of a macro and will store the value this macro had
      internally. Then, each time drawing code is executed, the value
      of this macro will be restored.
    \end{command}

    \begin{command}{\pgfarrowssavethe\marg{register}}
      Works like |\pgfarrowssave|, only the parameter must be a
      register and the value |\the|\meta{register} will be saved.
      Typically, you will write something like
\begin{codeexample}[code only]
\pgfarrowssavethe{\pgfarrowlength}
\pgfarrowssavethe{\pgfarrowwidth}
\end{codeexample}
      To ensure that inside the drawing code the the dimension
      registers |\pgfarrowlength| and |\pgfarrowwidth| are setup with
      the values they had during the setup.
    \end{command}
    
  \item \declare{|drawing code|}|=|\marg{code}

    This code will be executed at least once for each setting of the
    parameters when the time arrow tip is actually drawn. Usually,
    this one execution will be all and the 
    low-level commands generated inside the \meta{code} will we stored
    in a special cache; but in some cases the \meta{code} gets
    executed each time the arrow tip is used, so do not assume
    anything about it. Inside the \meta{code}, you have access to all
    values that were saved in the setup code as well as to the line
    width.
    
    The \meta{code} should draw the arrow tip ``going right along the
    $x$-axis.'' \pgfname\ will take care of setting up a  canvas
    transformation beforehand to a rotation such that when the
    drawing is rendered, the arrow tip that is  actually drawn points
    in the direction of the line. Alternatively, when bending is
    switched on, even more complicated low-level transformations will
    be done automatically.
    
    The are some special considerations concerning the \meta{code}:
    \begin{itemize}
    \item
      In the \meta{code} you may \emph{not} use |\pgfusepath|
      since this would try to add arrow tips to the arrow tip and lead
      to a recursion. Use the ``quick'' versions |\pgfusepathqstroke|
      and so on instead, which never try to add arrow tips.
    \item
      If you stroke the path that you construct, you should first set
      the dashing to solid and set up fixed joins and caps, as
      needed. This will ensure that the arrow tip will always look the
      same.
    \item
      When the arrow tip code is executed, it is automatically put
      inside a low-level scope, so nothing will ``leak out'' from the
      scope.
    \item
      The high-level coordinate transformation matrix will be set to the
      identity matrix when the code is executed for the first time.
    \end{itemize}
    
  \item \declare{|cache|}|=|\meta{true or false}

    When set to |true|, which is the default, the \meta{code} will be
    executed only once for a partiular value of parameters and the
    low-level commands created by the drawing code (using the system
    layer protocol subsystem, see Section~\ref{section-protocols})
    will be cached and reused later on. However, when the drawing code
    contains ``uncachable'' code like a call to |\pgftext|, caching
    must be switched off by saying |cache=false|.

  \item \declare{|bending mode|}|=|\meta{mode}

    This key is important only when the |bend| option is used with an
    arrow, see Section~\ref{section-arrow-flex} for an introduction to
    this option. The |bend| option asks us to, well, bend the arrow
    head. For some arrow head this is not possible or leads to very
    strange drawings (for instance, when the |\pgftext| command is
    used) and then it is better to switch bending off for the arrow
    head (|flex| will then be used instead). To achieve this, set
    \meta{mode} to |none|.

    For most arrow tips it does, however, make sense to bend
    them. There are (at least) two different mathematical ways of
    doing so, see Section~\ref{section-library-curvilinear} for details. Which of
    these ways is use can be configured by setting \meta{mode} to
    either |orthogonal| or to |polar|. It is best to try simply try
    out both when designing an arrow tip to see which works
    better. Since |orthogonal| is quicker and often gives good oder
    even better results, it is the default. Some arrow tips, however,
    profit from saying |bending mode=polar|.

  \item \declare{|defaults|}|=|\meta{arrow keys}

    The \meta{arrow keys} allow you to configure the default values
    for the parameters on which an arrow tip depends. The \meta{arrow
      keys} will be executed first before any other arrow tip options
    are executed, see Section~\ref{section-arrow-scopes} for the exact
    sequence. Also see Section~\ref{section-arrow-options} below for
    more details on arrow options.
  \end{itemize}

  This concludes the description of the keys you provide for the
  declaration of an arrow. Let us now have a look at a simple example
  that uses these features: We want to define an arrow tip kind |foo|
  that produces the arrow tip we used as our running examlpe. However,
  to make things a bit more interesting, let us make it
  ``configurable'' insofar as the length of the arrow tip can be
  configured using the |length| option, which sets the
  |\pgfarrowlength|. By default, this length should be the gigantic
  4cm we say in the example, but uses should be able to set it to
  anything they like. We will not worry about the arrow width or
  insets, of arrow line width, or harpoons, or anything else in this
  example to keep it simple.

  Here is the code:
\begin{codeexample}[code only]
\pgfdeclarearrow{
  name = foo,
  parameters = { \the\pgfarrowlength },
  setup code = {
    % The different end values:
    \pgfarrowssettipend{.25\pgfarrowlength}
    \pgfarrowssetlineend{-.25\pgfarrowlength}
    \pgfarrowssetvisualbackend{-.5\pgfarrowlength}
    \pgfarrowssetbackend{-.75\pgfarrowlength}
    % The hull
    \pgfarrowshullpoint{.25\pgfarrowlength}{0pt}
    \pgfarrowshullpoint{-.75\pgfarrowlength}{.5\pgfarrowlength}
    \pgfarrowshullpoint{-.75\pgfarrowlength}{-.5\pgfarrowlength}
    % Saves: Only the length:
    \pgfarrowssavethe\pgfarrowlength
  },
  drawing code = {
    \pgfpathmoveto{.25\pgfarrowlength}{0pt}
    \pgfpathlineto{-.75\pgfarrowlength}{.5\pgfarrowlength}
    \pgfpathlineto{-.5\pgfarrowlength}{0pt}
    \pgfpathlineto{-.75\pgfarrowlength}{-.5\pgfarrowlength}
    \pgfpathclose
    \pgfusepathqfill
  },
  defaults = { length = 4cm }
}   
\end{codeexample}
  We can now use it:
\pgfdeclarearrow{
  name = foo,
  parameters = { \the\pgfarrowlength },
  setup code = {
    % The different end values:
    \pgfarrowssettipend{.25\pgfarrowlength}
    \pgfarrowssetlineend{-.25\pgfarrowlength}
    \pgfarrowssetvisualbackend{-.5\pgfarrowlength}
    \pgfarrowssetbackend{-.75\pgfarrowlength}
    % The hull
    \pgfarrowshullpoint{.25\pgfarrowlength}{0pt}
    \pgfarrowshullpoint{-.75\pgfarrowlength}{.5\pgfarrowlength}
    \pgfarrowshullpoint{-.75\pgfarrowlength}{-.5\pgfarrowlength}
    % Saves: Only the length:
    \pgfarrowssavethe\pgfarrowlength
  },
  drawing code = {
    \pgfpathmoveto{\pgfqpoint{.25\pgfarrowlength}{0pt}}
    \pgfpathlineto{\pgfqpoint{-.75\pgfarrowlength}{.5\pgfarrowlength}}
    \pgfpathlineto{\pgfqpoint{-.5\pgfarrowlength}{0pt}}
    \pgfpathlineto{\pgfqpoint{-.75\pgfarrowlength}{-.5\pgfarrowlength}}
    \pgfpathclose
    \pgfusepathqfill
  },
  defaults = { length = 4cm }
}    
\begin{codeexample}[]
\tikz \draw [-foo] (0,0) -- (8,0);    
\end{codeexample}
\begin{codeexample}[]
\tikz \draw [-{foo[length=2cm,bend]}] (0,0) to [bend left] (3,0);    
\end{codeexample}

  \medskip
  \noindent\textbf{Defining a Shorthand.}  
  The |\pgfdeclarearrow| command can also used to define
  \emph{shorthands}. This works as follows:
  \begin{itemize}
  \item First, you must provide a |name| just in the same way as when
    you define a full-flung new arrow tip kind.
  \item Second, instead of all of the other options listed above, you
    just use one more option:
    
    \smallskip
    \declare{|means|}|=|\meta{end arrow specification}
    
    This sets up things so that whenever \meta{name} is now used in an
    arrow specification, it will be replaced by the \meta{end arrow
      specification} (the problems resulting form the \meta{name}
    begin used in a start arrow  specification are taken care of
    automatically). See also Section~\ref{section-arrow-tip-macro} for
    details on the order in which options get executed in such cases.

    Note that the \meta{end arrow specification} will be executed
    immediately to build the so-called arrow option caches, a concept
    explored in more detail in
    Section~\ref{section-arrow-option-cache}. In practice, this has
    mainly two effects: First, all arrow tips referred to in the
    specification must already exist (at least as ``dummy''
    versions). Second, all dimensions mentioned in options of the
    \meta{end arrow specification} will be evaluated immediately. For
    instance, when you write
\begin{codeexample}[code only]
\pgfdeclarearrow{ name=foo, means = bar[length=2cm+\mydimen] }      
\end{codeexample}
    The value |2cm+\mydimen| is evaluated immediately. When |foo| is
    used later on and |\mydimen| has changed, this has no effect. 
  \end{itemize}
\end{command}



\subsection{Handling Arrow Options}

\label{section-arrow-options}

When you declare an arrow tip, your drawing code should take into
account the different arrow keys set for it (like the arrow tip length, 
width, or harpooning). The different arrow keys that are available
have been described in detail in Section~\ref{section-arrow-config}; but
how do we access the values set by an option like |length| or
|harpoon| or |bend| in the drawing code? In the present section we
have a look at how this works.


\subsubsection{Dimension Options}

Most arrow keys, like |length| or |width'|, simple set a \TeX\
dimension register to a certain value. For example, |length| sets the
value of the \TeX\ dimension register |\pgfarrowlength|. Note that
|length| takes several values as input with a complicated semantics as
explained for the |length| key on
page~\pageref{length-arrow-key}. All of these settings are not
important for the setup code: When it gets executed, the code behind
the |length| key will have computed a simple number that is stored
in |\pgfarrowlength|. Indeed, inside the setup code you do not have
access to the exact value given to the |length| key; just to the
final computed value.

The following \TeX\ dimensions are available to the setup code:

\begin{itemize}
\item |\pgfarrowslength|. It gets set by the arrow keys |length| and |angle|.
\item |\pgfarrowswidth|. It gets set by |width|, |width'|, and |angle|.
\item |\pgfarrowsinset|. It gets set by |inset| and |inset'|.
\item |\pgfarrowslinewidth|. It gets set by |line width| and |line width'|.
\end{itemize}

If your setup code depends on any of them, add them to the
|parameters| key of the arrow tip.


\subsubsection{True--False Options}

A number of arrow keys just do a yes/no switch, like |reversed|. All
of them setup a \TeX-if that you can access in the setup code:

\begin{itemize}
\item |\ifpgfarrowreversed| is setup by |reversed|.
\item |\ifpgfarrowswap| is setup by |swap| and also |right|.
\item |\ifpgfarrowharpoon| is setup by |harpoon| and also |left| and |right|.
\item |\ifpgfarrowroundcap| is set to true by |line cap=round| and set
  to false by |line cap=butt|. It also gets (re)set by  |round| and |sharp|.
\item |\ifpgfarrowroundjoin| is set to true by |line join=round| and set
  to false by |line join=miter|. It also gets (re)set by  |round| and |sharp|.
\item |\ifpgfarrowopen| is set to true by |fill=none| and by |open|
  (which is a shorthand for |fill=none|) and set to false by |color|
  and all other |fill=|\meta{color}.  
\end{itemize}

If you code depends on any of these, you must add them to
the |parameters| in such a way that the parameters are different when
the \TeX-if is set from when it is not set. An easy way to achieve
this is to write something like
\begin{codeexample}[code only]
  parameters = { \the\pgfarrowlength,...,
                 \ifpgfarrowharpoon h\fi\
                 \ifpgfarrowroundjoin j\fi}  
\end{codeexample}
In other words, for each set parameter on which the arrow tip depends,
a specific letter is added to the parameters, making them unique.

The first two of the above keys are a bit special: Reversing and swapping an
arrow tip can be done just by fiddling with the
transformation matrix: a reverse is a ``flip'' along the $y$-axis and
a swap is a flip along the $x$-axis. This is done automatically by
\pgfname.

Nevertheless, you may wish to modify you code in dependence
especially of the |reverse| key: When |\ifpgfarrowreverse| is true,
\pgfname\ will flip the coordinate system along the $y$-axis, will
negate all end values (like line end, tip end, and so on) and will
exchange the meaning of back end and tip end as well as of visual back
end and visual back end. Usually, this is exactly what one need;
\emph{except} that the line end may no longer be appropriate. After
all, the line end should be chosen so that it is completely covered by
the arrow. Now, when the arrow tip is open, a reversed arrow should no
longer have the line end near the old visual back end, but near to the
old visual tip end.

For these reasons, you may need to make the computation of the line
end dependent on whether the arrow is reversed or not. Note that when
you specify a different line end for a reversed arrow tip, the
transformation and inverting of the coordinate system will still be
done, meaning that if |reverse| is true, you need to specify a line
end in the ``old'' coordinate system that is at the position where,
after everything is inverted, it will be at the correct
position. Usually that means that if the |reverse| option is set, you
need to \emph{increase} the line end.


\subsubsection{Inaccessible Options}

There are some options that influence the way an arrow tip looks, but
that you cannot access inside the setup code. Handling these options
lies entirely with \pgfname. If you wish your setup code to handle
these options, you have to setup your own ``parallel'' options.

\begin{itemize}
\item |quick|, |flex|, |flex'|, and |bend| are all handled
  automatically. You can, however, set the |bending mode| to avoid
  bending of your arrow tip.
\item The colors set by |color| and |fill|. You can, however, access
  them indirectly, namely through the current stroke and fill colors.
\item |sep|
\end{itemize}


\subsubsection{Defining New Arrow Keys}
\label{section-arrow-option-cache}

The set of predefined options is already quite long and most arrow
tips will not need more than the predefined options. However,
sometimes an arrow tip may need to introduce a new special-purpose
option. For instance, suppose we wish to introcue a new fictive arrow
key |depth|. In such cases, you must do two things:

\begin{enumerate}
\item
  Introduce a new dimension register or macro that will hold the
  configuration value and which will be accessed by the setup
  code. The could be achieved by saying
\begin{codeexample}[code only]
\newdimen\pgfarrowdepth
\end{codeexample}
\item
  Introduce a new arrow key option |/pgf/arrow keys/depth| that allows
  users to configure the new macro or register.
\end{enumerate}

When an arrow is selected via for instance |foo[depth=5pt]|, the
key--value pairs between the square brackets are executed with the
path prefix |/pgf/arrow keys|. Thus, in the example, our depth key
would get executed. Thus, it is temping to write something like
\begin{codeexample}[code only]
\pgfkeys{/pgf/arrow keys/depth/.code = \pgfmathsetlength{\pgfarrowdepth}{#1}}
\end{codeexample}

Sadly, this will not work. The reason is that the is yet another level
of caching involved when \pgfname\ processes arrow tips: The option
cache! The problem is each time an arrow tip is used, even when the
drawing code of the arrow tip is nicely cached, we still need to
process the options in |foo[length=5pt]| to find out which version in
the cache we would like to access. To make matters worse, |foo| might
be a shorthand that calls other arrow tips, which add more options,
and so on. Unfortunately, executing keys is quite an expensive
operation (\pgfname's key--value parser is powerful, but that power
comes at a price). So, whenever possible, we do \emph{not} want the
key--value parser to be started.

For these reasons, when something like |foo[|\meta{options}|]| is 
encountered inside a shorthand, the \meta{options} are executed only
once. They should now setup the \emph{arrow option cache}, which is
some code that, when executed, should setup the values that the
\meta{options} configure. In our example, the |depth| key should add
something to the arrow option cache that sets |\pgfarrowdepth| to the
given value.

Adding something to the arrow option cache is done using the following
command:

\begin{command}{\pgfarrowsaddtooptions\marg{code}}
  This command should be called by keys with the prefix
  |/pgf/arrow keys| to add code to the arrow option cache. For our
  |depth| key example, we could use this key as follows:
\begin{codeexample}[code only]
\pgfkeys{/pgf/arrow keys/depth/.code=
  \pgfarrowsaddtooptions{\pgfmathsetlength{\pgfarrowdepth}{#1}}
\end{codeexample}
  Actually, this is still not optimal since the expensive
  |\pgfmathsetlength| command is now called each time an arrow tip is
  used with the |depth| option set. The trick is to do the expensive
  operation only once and then store only very quick code in the arrow
  option cache:
\begin{codeexample}[code only]
\pgfkeys{/pgf/arrow keys/depth/.code=
  \pgfmathsetlength{\somedimen}{#1}
  \pgfarrowsaddtooptions{\pgfarrowdepth=\somedimen} % buggy
\end{codeexample}
  The above code will not (yet) work since |\somedimen| will surely
  have a different value when the cache is executed. The trick is to
  use some |\expandafter|s:
\begin{codeexample}[code only]
\pgfkeys{/pgf/arrow keys/depth/.code=
  \pgfmathsetlength{\somedimen}{#1}
  \expandafter\pgfarrowsaddtooptions\expandafter{\expandafter\pgfarrowdepth\expandafter=\the\somedimen}
\end{codeexample}
\end{command}

\begin{command}{\pgfarrowsaddtolateoptions\marg{code}}
  This command works like |\pgfarrowsaddtooptions|, only the
  \meta{code} will be executed ``later'' than the code added by the
  normal version of the command. This is useful for keys that depend
  on the length of an arrow: Keys like |width'| want to define the
  arrow width as a multiple of the arrow length, but when the |width'|
  key is given, the length may not yet have been specified. By making
  the computation of the width a ``late'' option, we ensure that
  |\pgfarrowlength| will have been setup correctly.
\end{command}

If you define a new option that sets a dimensions and if that
dimension should change in accordance to the setting of either
|scale length| or |scale width|, you need to make \pgfname\ ``aware''
of this using the following key:

\begin{command}{\pgfarrowsaddtolengthscalelist\marg{dimension register}}
  Each time an arrow tip is used, the given \meta{dimension register}
  will be multiplied by the |scale length| factor prior to the actual
  drawing. You call this command only once in the preamble somewhere.
\end{command}

\begin{command}{\pgfarrowsaddtowidthscalelist\marg{dimension register}}
  Works like |\pgfarrowsaddtolengthscalelist|, only for width parameters.
\end{command}


\begin{command}{\pgfarrowsthreeparameters\marg{line-width dependent
      size specification}}
  This command is useful for parsing the values given to keys like
  |length| or |width| the expect a dimension followed optionally for
  some numbers. This command converts the \meta{line-width dependent
    size specification}, which may consist of one, two, or three
  numbers, into a triple of three numbers in curly braces, which gets
  stored in the macro |\pgfarrowstheparameters|. Here is an example,
  where |\showvalueofmacro| is used in this example to show the value
  stored in a macro:
  \makeatletter
  \def\showvalueofmacro#1{%
    \texttt{\expandafter\expandafter\expandafter\expandafter\expandafter\expandafter\expandafter\pgfutil@gobble\expandafter\expandafter\expandafter\string\expandafter\csname#1\endcsname}
  }
\begin{codeexample}[]
\pgfarrowsthreeparameters{2pt 1}
\showvalueofmacro\pgfarrowstheparameters 
\end{codeexample}
\end{command}


\begin{command}{\pgfarrowslinewidthdependent\marg{dimension}\marg{line
      width factor}\marg{outer factor}}
  This command take three parameters and does the ``line width
  dependent computation'' described on page~\pageref{length-arrow-key} 
  for the |length| key. The result is returned in |\pgf@x|.

  The idea is that you can setup line-width dependent keys like
  |length| or |width| using code like the following:
\begin{codeexample}[code only]
\pgfkeys{/pgf/arrow keys/depth/.code={%
  \pgfarrowsthreeparameters{#1}%
  \expandafter\pgfarrowsaddtolateoptions\expandafter{%
    \expandafter\pgfarrowslinewidthdependent\pgfarrowstheparameters% compute...
    \pgfarrowdepth\pgf@x% ... and store.
  }%
}    
\end{codeexample}
\end{command}

\begin{command}{\pgfarrowslengthdependent\marg{dimension}\marg{length factor}\marg{dummy}}
  This command take three parameters, of which the last one is
  ignored, and does the ``length dependent computation'' described for
  the |width'| and |inset'| keys. The result is returned in |\pgf@x|.

  You can setup length dependent keys using code like the following:
\begin{codeexample}[code only]
\pgfkeys{/pgf/arrow keys/depth'/.code={%
  \pgfarrowsthreeparameters{#1}%
  \expandafter\pgfarrowsaddtolateoptions\expandafter{%
    \expandafter\pgfarrowslengthdependent\pgfarrowstheparameters% compute...
    \pgfarrowdepth\pgf@x% ... and store.
  }%
}    
\end{codeexample}
\end{command}


%%% Local Variables:
%%% mode: latex
%%% TeX-master: "pgfmanual"
%%% End:

% Copyright 2006 by Till Tantau
%
% This file may be distributed and/or modified
%
% 1. under the LaTeX Project Public License and/or
% 2. under the GNU Free Documentation License.
%
% See the file doc/generic/pgf/licenses/LICENSE for more details.


\section{Nodes and Shapes}

\label{section-shapes}

This section describes the |pgfbaseshapes| package.

\begin{package}{pgfbaseshapes}
  This package defines commands both for creating nodes and for
  creating shapes. The package is loaded automatically by |pgf|, but
  you can load it manually if you have  only included |pgfcore|.  
\end{package}


\subsection{Overview}

\pgfname\ comes with a sophisticated set of commands for creating
\emph{nodes} and \emph{shapes}. A \emph{node} is a graphical object
that consists (typically) of (one or more) text labels and some
additional stroked or filled paths. Each node has a certain
\emph{shape}, which may be something simple like a |rectangle| or a
|circle|, but it may also be something complicated like a
|uml class diagram| (this shape is currently not implemented,
though). Different nodes that have the same shape may look quite
different, however, since shapes (need not) specify whether the shape
path is stroked or filled.


\subsubsection{Creating and Referencing Nodes}

You create a node by calling the macro |\pgfnode| or the more general
|\pgfmultipartnode|. These macro takes several parameters and draws
the requested shape at a certain position. In addition, it will
``remember'' the node's position within the current
|{pgfpicture}|. You can then, later on, refer to the 
node's position. Coordinate transformations are ``fully supported,''
which means that if you used coordinate transformations to shift or
rotate the shape of a node, the node's position will still be correctly
determined by \pgfname. This is \emph{not} the case if you use canvas
transformations, instead.


\subsubsection{Anchors}

An important property of a node or a shape in general are its
\emph{anchors}. Anchors are ``important'' positions in a shape. For
example, the |center| anchor lies at the center of a shape, the
|north| anchor is usually ``at the top, in the middle'' of a shape,
the |text| anchor is the lower left corner of the shape's text label
(if present), and so on.

Anchors are important both when you create a node and when you
reference it. When you create a node, you specify the node's
``position'' by asking \pgfname\ to place the shape in such a way that
a certain anchor lies at a certain point. For example, you might ask
that the node is placed such that the |north| anchor is at the
origin. This will effectively cause the node to be placed below the
origin.

When you reference a node, you always reference an anchor of the
node. For example, when you request the ``|north| anchor of the node
just placed'' you will get the origin. However, you can also request
the ``|south| anchor of this node,'' which will give you a point
somewhere below the origin. When a coordinate transformation was in
force at the time of creation of a node, all anchors are also
transformed accordingly.

\subsubsection{Layers of a Shape}

The simplest shape, the |coordinate|, has just one anchor, namely the
|center|, and a label (which is usually empty). More complicated
shapes like the |rectangle| shape also have a \emph{background
  path}. This is a \pgfname-path that is defined by the shape. The
shape does not prescribe what should happen with the path: When a node
is created this path may be stroked (resulting in a frame around the
label), filled (resulting in a background color for the text), or just
discarded.

Although most shapes consist just of a background path plus some label
text, when a shape is drawn, up to seven different layers are drawn:

\begin{enumerate}
\item
  The ``behind the background layer.'' Unlike the background path,
  which be used in different ways by different nodes, the graphic
  commands given for this layer will always stroke or
  always fill the path they construct. They might also insert some
  text that is ``behind everything.''
\item
  The background path layer. How this path is used depends on how the
  arguments of the |\pgfnode| command.
\item
  The ``before the background path layer.'' This layer works like the
  first one, only the commands of this layer are executed after the
  background path has been used (in whatever way the creator of the
  node chose).
\item
  The label layer. This layer inserts the node's text box(es).
\item
  The ``behind the foreground layer.'' This layer, like the
  first layer, once more contains graphic commands that are ``simply
  executed.''
\item
  The foreground path layer. This path is treated in the same way as the
  background path, only it is drawn only after the label text has been
  drawn.
\item
  The ``before the foreground layer.''
\end{enumerate}

Which of these layers are actually used depends on the shape.



\subsubsection{Node Parts}

A shape typically does not consist only of different background and
foreground paths, but it may also have text labels. Indeed, for many
shapes the text labels are the more important part of the shape.

Most shapes will have only one text label. In this case, this text
label is simply passed as a parameter to the |\pgfnode| command. When
the node is drawn, the text label is shifted around such that its
lower left corner is at the |text| anchor of the node.

More complicated shapes may have more than one text label. Nodes of
such shapes are called \emph{multipart nodes}. The different
\emph{node parts} are simply the different text labels. For example, a
|uml class| shape might have a |class name| part, a |method| part and
an |attributes| part. Indeed, single part nodes are a special case of
multipart nodes: They only have one part named |text|.

When a shape is declared, you must specify the node parts. There is a
simple command called |\nodeparts| that takes a list of the part names
as input. When you create a node of a multipart shape, for each part
of the node you must have setup a \TeX-box containing the text of the
part. For a part named |XYZ| you must setup the box
|\pgfnodepartXYZbox|. The box will be placed at the anchor |XYZ|. See
the description of |\pgfmultipartnode| for more details.


\subsection{Creating Nodes}

You create a node using on of the following commands:

\begin{command}{\pgfnode\marg{shape}\marg{anchor}\marg{label
      text}\marg{name}\marg{path usage command}} 
  This command creates a new node. The \meta{shape} of the node must
  have been declared previously using |\pgfdeclareshape|.

  The shape is shifted such that the \meta{anchor} is at the
  origin. In order to place the shape somewhere else, use the
  coordinate transformation prior to calling this command.

  The \meta{name} is a name for later reference. If no name is given,
  nothing will be ``saved'' for the node, it will just be drawn.

  The \meta{path usage command} is executed for the background and the
  foreground path (if the shape defines them).

\begin{codeexample}[]
\begin{tikzpicture}
  \draw[help lines] (0,0) grid (4,3);
  {
    \pgftransformshift{\pgfpoint{1cm}{1cm}}
    \pgfnode{rectangle}{north}{Hello World}{hellonode}{\pgfusepath{stroke}}
  }
  {
    \color{red!20}
    \pgftransformrotate{10}
    \pgftransformshift{\pgfpoint{3cm}{1cm}}
    \pgfnode{rectangle}{center}
      {\color{black}Hello World}{hellonode}{\pgfusepath{fill}}
  }
\end{tikzpicture}
\end{codeexample}

  As can be seen, all coordinate transformations are also applied to
  the text of the shape. Sometimes, it is desirable that the
  transformations are applied to the point where the shape will be
  anchored, but you do not wish the shape itself to the
  transformed. In this case, you should call
  |\pgftransformresetnontranslations| prior to calling the |\pgfnode|
  command. 

\begin{codeexample}[]
\begin{tikzpicture}
  \draw[help lines] (0,0) grid (4,3);
  {
    \color{red!20}
    \pgftransformrotate{10}
    \pgftransformshift{\pgfpoint{3cm}{1cm}}
    \pgftransformresetnontranslations
    \pgfnode{rectangle}{center}
      {\color{black}Hello World}{hellonode}{\pgfusepath{fill}}
  }
\end{tikzpicture}
\end{codeexample}

  The \meta{label text} is typeset inside the \TeX-box
  |\pgfnodeparttextbox|. This box is shown at the |text| anchor of the
  node, if the node has a |text| part. See the description of
  |\pgfmultipartnode| for details.
\end{command}

\begin{command}{\pgfmultipartnode\marg{shape}\marg{anchor}\marg{name}\marg{path
      usage command}}
  This command is the more general (and less user-friendly) version of
  the |\pgfnode| command. While the |\pgfnode| command can only be
  used for shapes that have a single part (which is the case for most
  shapes), this command can also be used with multi-part nodes.

  When this command is called, for each node part of the node you must
  have setup one \TeX-box. Suppose the shape has two parts: The |text|
  part and the |lower| part. Then, prior to calling
  |\pgfmultipartnode|, you must have setup the boxes
  |\pgfnodeparttextbox| and |\pgfnodepartlowerbox|. These boxes may 
  contain any \TeX-text. The shape code will then compute the
  positions of the shape's anchors based on the sizes of the these
  shapes. Finally, when the node is drawn, the boxes are placed at the
  anchor positions |text| and |lower|.

\begin{codeexample}[]
\setbox\pgfnodeparttextbox=\hbox{$q_1$}
\setbox\pgfnodepartlowerbox=\hbox{01}
\begin{pgfpicture}
  \pgfmultipartnode{circle split}{center}{my state}{\pgfusepath{stroke}}
\end{pgfpicture}
\end{codeexample}

  \emph{Note:\/} Be careful when using the |\setbox| command inside a
  |{pgfpicture}| command. You will have to use |\pgfinterruptpath| at
  the beginning of the box and |\endpgfinterruptpath| at the end of
  the box to make sure that the box is typeset correctly. In the above
  example this problem was sidestepped by moving the box construction
  outside the environment.

  \emph{Note:\/} It is not necessary to use |\newbox| for every node
  part name. Although you need a different box for each part of a
  single shape, two different shapes may very well use the same box
  even when the names of the parts are different. Suppose you have a
  |circle split| shape that has an |lower| part and you have a
  |uml class| shape that has a |methods| part. Then, in order to avoid
  exhausting \TeX's limited number of box registers, you can say
\begin{codeexample}[code only]
\newbox\pgfnodepartlowerbox
\let\pgfnodepartmethodsbox=\pgfnodepartlowerbox  
\end{codeexample}
  Also, when you have a node part name with spaces like |class name|,
  it may be useful to create an alias:
\begin{codeexample}[code only]
\newbox\mybox
\expandafter\let\csname pgfnodepartclass namebox\endcsname=\mybox
\end{codeexample}
\end{command}

There are a number of values that have an influence on the size of a
node. These parameters can be changed using the following commands:

\begin{command}{\pgfsetshapeminwidth\marg{dimension}}
  This command sets the macro \declare{|\pgfshapeminwidth|} to
  \meta{dimension}. This dimension is the \emph{recommended} minimum
  width of a shape. Thus, when a shape is drawn and when the shape's
  width would be smaller than \meta{dimension}, the shape's width is
  enlarged by adding some empty space.

  Note that this value is just a recommendation. A shape may choose to
  ignore the value of |\pgfshapeminwidth|.
  
\begin{codeexample}[]
\begin{tikzpicture}
  \draw[help lines] (-2,0) grid (2,1);

  \pgfsetshapeminwidth{3cm}
  \pgfnode{rectangle}{center}{Hello World}{}{\pgfusepath{stroke}}
\end{tikzpicture}
\end{codeexample}
\end{command}

\begin{command}{\pgfsetshapeminheight\marg{dimension}}
  Works like |\pgfsetshapeminwidth|.
\end{command}


\begin{command}{\pgfsetshapeinnerxsep\marg{dimension}}
  This command sets the macro \declare{|\pgfshapeinnerxsep|} to
  \meta{dimension}. This dimension is the \emph{recommended} horizontal
  inner separation between the label text and the background path. As
  before, this value is just a recommendation and a shape may choose
  to ignore the value of |\pgfshapeinnerxsep|.
  
\begin{codeexample}[]
\begin{tikzpicture}
  \draw[help lines] (-2,0) grid (2,1);

  \pgfsetshapeinnerxsep{1cm}
  \pgfnode{rectangle}{center}{Hello World}{}{\pgfusepath{stroke}}
\end{tikzpicture}
\end{codeexample}
\end{command}

\begin{command}{\pgfsetshapeinnerysep\marg{dimension}}
  Works like |\pgfsetshapeinnerysep|.
\end{command}



\begin{command}{\pgfsetshapeouterxsep\marg{dimension}}
  This command sets the macro \declare{|\pgfshapeouterxsep|} to
  \meta{dimension}. This dimension is the recommended horizontal
  outer separation between the background path and the ``outer
  anchors.'' For example, if \meta{dimension} is |1cm| then the
  |east| anchor will be 1cm to the right of the right border of the
  background path. 

  As before, this value is just a recommendation.
  
\begin{codeexample}[]
\begin{tikzpicture}
  \draw[help lines] (-2,0) grid (2,1);

  \pgfsetshapeouterxsep{.5cm}
  \pgfnode{rectangle}{center}{Hello World}{x}{\pgfusepath{stroke}}

  \pgfpathcircle{\pgfpointanchor{x}{north}}{2pt}
  \pgfpathcircle{\pgfpointanchor{x}{south}}{2pt}
  \pgfpathcircle{\pgfpointanchor{x}{east}}{2pt}
  \pgfpathcircle{\pgfpointanchor{x}{west}}{2pt}
  \pgfpathcircle{\pgfpointanchor{x}{north east}}{2pt}
  \pgfusepath{fill}
\end{tikzpicture}
\end{codeexample}
\end{command}

\begin{command}{\pgfsetshapeouterysep\marg{dimension}}
  Works like |\pgfsetshapeouterysep|.
\end{command}

\begin{command}{\pgfsetshapeaspect\marg{value}}
  This command sets the macro \declare{|\pgfshapeaspect|} to
  \meta{value}. Furthermore, \declare{|\pgfshapeaspectinverse|} is set
  to the reciprocal of \meta{value}. The aspect is a recommendation
  for the quotient of the width and the height of a shape.
\end{command}


\subsection{Using Anchors}

Each shape defines a set of anchors. We saw already that the anchors
are used when the shape is drawn: the shape is placed in such a way
that the given anchor is at the origin (which in turn is typically
translated somewhere else).

One has to look up the set of anchors of each shape, there is no
``default'' set of anchors, except for the |center| anchor, which
should always be present. Also, most shapes will declare anchors like
|north| or |east|, but this is not guaranteed.


\subsubsection{Referencing Anchors of Nodes in the Same Picture}

Once a node has been defined, you can refer to its anchors using the
following commands:

\begin{command}{\pgfpointanchor\marg{node}\marg{anchor}}
  This command is another ``point command'' like the commands
  described in Section~\ref{section-points}. It returns the coordinate
  of the given \meta{anchor} in the given \meta{node}. The command can
  be used in commands like |\pgfpathmoveto|.

\begin{codeexample}[]
\begin{pgfpicture}
  \pgftransformrotate{30}
  \pgfnode{rectangle}{center}{Hello World!}{x}{\pgfusepath{stroke}}

  \pgfpathcircle{\pgfpointanchor{x}{north}}{2pt}
  \pgfpathcircle{\pgfpointanchor{x}{south}}{2pt}
  \pgfpathcircle{\pgfpointanchor{x}{east}}{2pt}
  \pgfpathcircle{\pgfpointanchor{x}{west}}{2pt}
  \pgfpathcircle{\pgfpointanchor{x}{north east}}{2pt}
  \pgfusepath{fill}
\end{pgfpicture}
\end{codeexample}

  In the above example, you may have noticed something curious: The
  rotation transformation is still in force when the anchors are
  invoked, but it does not seem to have an effect. You might expect
  that the rotation should apply to the already rotated points once
  more.

  However, |\pgfpointanchor| returns a point that takes the current
  transformation matrix into account: \emph{The inverse transformation
    to the current coordinate transformation is applied to an anchor
    point before returning it.}

  This behavior may seem a bit strange, but you will find it very
  natural in most cases. If you really want to apply a transformation
  to an anchor point (for example, to ``shift it away'' a little bit),
  you have to invoke |\pgfpointanchor| without any transformations in
  force. Here is an example:

\makeatletter
\begin{codeexample}[]
\begin{pgfpicture}
  \pgftransformrotate{30}
  \pgfnode{rectangle}{center}{Hello World!}{x}{\pgfusepath{stroke}}

  {
    \pgftransformreset
    \pgfpointanchor{x}{east}
    \xdef\mycoordinate{\noexpand\pgfpoint{\the\pgf@x}{\the\pgf@y}}
  }
    
  \pgfpathcircle{\mycoordinate}{2pt}
  \pgfusepath{fill}
\end{pgfpicture}
\end{codeexample}

  A special situation arises when the \meta{node} lies in a picture
  different from the current picture. In this case, if you have not
  told \pgfname\ that the picture should be ``remembered,'' the
  \meta{node} will be treated as if it lied in the current
  picture. For example, if the \meta{node} was at position $(3,2)$ in
  the original picture, it is treated as if it lied at position
  $(3,2)$ in the current picture. However, if you have told \pgfname\
  to remember the picture position of the node's picture and also of
  the current picture,
  then |\pgfpointanchor| will return a coordinate that corresponds to
  the position of the node's anchor on the page, transformed into the
  current coordinate system. For examples and more details see
  Section~\ref{section-cross-pictures-pgf}. 
\end{command}

\begin{command}{\pgfpointshapeborder\marg{node}\marg{point}}
  This command returns the point on the border of the shape that lies
  on a straight line from the center of the node to \meta{point}. For
  complex shapes it is not guaranteed that this point will actually
  lie on the border, it may be on the border of a ``simplified''
  version of the shape.

\begin{codeexample}[]
\begin{pgfpicture}
  \begin{pgfscope}
    \pgftransformrotate{30}
    \pgfnode{rectangle}{center}{Hello World!}{x}{\pgfusepath{stroke}}
  \end{pgfscope}
  \pgfpathcircle{\pgfpointshapeborder{x}{\pgfpoint{2cm}{1cm}}}{2pt}
  \pgfpathcircle{\pgfpoint{2cm}{1cm}}{2pt}
  \pgfpathcircle{\pgfpointshapeborder{x}{\pgfpoint{-1cm}{1cm}}}{2pt}
  \pgfpathcircle{\pgfpoint{-1cm}{1cm}}{2pt}
  \pgfusepath{fill}
\end{pgfpicture}
\end{codeexample}
\end{command}


\subsubsection{Referencing Anchors of Nodes in Different Pictures}
\label{section-cross-pictures-pgf}

As a picture is typeset, \pgfname\ keeps track of the positions of all
nodes inside the picture. What \pgfname\ does not remember is the
position of the picture \emph{itself} on the page. Thus, if you define
a node in one picture and then try to reference this node while
another picture is typeset, \pgfname\ will only know the position of
the nodes that you try to typeset inside the original picture, but it
will not know where this picture lies. What is missing is the relative
positioning of the two pictures.

To overcome this problem, you need to tell \pgfname\ that it should
remember the position of pictures on a page. If these positions are
remembered, then \pgfname\ can compute the offset between the pictures
and make nodes in different pictures accessible.

Determining the positions of pictures on the page is, alas,
not-so-easy. Because of this, \pgfname\ does not do so
automatically. Rather, you have to proceed as follows:
\begin{enumerate}
\item You have to use a backend driver that supports position
  tracking. pdf\TeX\ is one such drivers, |dvips| currently is not.
\item You have to say |\pgfrememberpicturepositiononpagetrue|
  somewhere before or inside every picture
  \begin{itemize}
  \item in which you wish to reference a node and
  \item from which you wish to reference a node in another picture.
  \end{itemize}
  The second item is important since \pgfname\ does not only need to
  know the position of the picture in which the node you wish to
  reference lies, but it also needs to know where the current picture
  lies.
\item You typically have to run \TeX\ twice (depending on the backend
  driver) since the position information typically gets written into
  an external file on the first run and is available only on the
  second run.
\item You have to switch off automatic bounding bound
  computations. The reason is that the node in the other picture
  should not influence the size of the bouding box of the current
  picture. You should say |\pgfusepath{use as bounding box}| before
  using a coordinate in another picture.
\end{enumerate}



\subsection{Predefined Nodes}

There are several special nodes that are always defined and which you
should not attempt to redefine.

\begin{predefinednode}{current bounding box}
  This node is of shape |rectangle|. Unlike normal nodes, its size
  changes constantly and always reflects the size of the bounding box
  of the current picture. This means that, for instance, that
\begin{codeexample}[code only]
\pgfpointanchor{current bounding box}{south east}
\end{codeexample}
  returns the lower left corner of the bounding box of the current
  picture. 
\end{predefinednode}

\begin{predefinednode}{current path bounding box}
  This node is also of shape |rectangle|. Its size is the size of the
  bounding box of the current path.
\end{predefinednode}

\begin{predefinednode}{current page}
  This node is inside a virtual remembered picture. The size of this
  node is the size of the current page. This means that if you create
  a remembered picture and inside this picture you reference an anchor
  of this node, you reference an absolute position on the page. To
  demonstrate the effect, the following code puts some text in the
  lower left corner of the current page. Note that this works only if
  the backend driver supports it, otherwise the text is inserted right
  here.%
{%
\pgfrememberpicturepositiononpagetrue%
\begin{pgfpicture}
  \pgfusepath{use as bounding box}
  \pgftransformshift{\pgfpointanchor{current page}{south west}}
  \pgftransformshift{\pgfpoint{1cm}{1cm}}
  \pgftext[left,base]{
    \textcolor{red}{
      Text absolutely positioned in
      the lower left corner.}
  }
\end{pgfpicture}
}%
\begin{codeexample}[code only]
\pgfrememberpicturepositiononpagetrue
\begin{pgfpicture}
  \pgfusepath{use as bounding box}
  \pgftransformshift{\pgfpointanchor{current page}{south west}}
  \pgftransformshift{\pgfpoint{1cm}{1cm}}
  \pgftext[left,base]{
    \textcolor{red}{
      Text absolutely positioned in
      the lower left corner.}
  }
\end{pgfpicture}  
\end{codeexample}
\end{predefinednode}


\subsection{Declaring New Shapes}

Defining a shape is, unfortunately, a not-quite-trivial process. The
reason is that shapes need to be both very flexible (their size will
vary greatly according to circumstances) and they need to be
constructed reasonably ``fast.'' \pgfname\ must be able to handle
pictures with several hundreds of nodes and documents with thousands
of nodes in total. It would not do if \pgfname\ had to compute and
store, say, dozens of anchor positions for every node.


\subsubsection{What Must Be Defined For a Shape?}

In order to define a new shape, you must provide:
\begin{itemize}
\item
  a \emph{shape name},
\item
  code for computing the  \emph{saved anchors} and \emph{saved
    dimensions}, 
\item
  code for computing \emph{anchor} positions in terms of the saved anchors,
\item
  optionally code for the \emph{background path} and \emph{foreground path},
\item
  optionally code for \emph{things to be drawn before or behind} the
  background and foreground paths.
\item
  optionally a list of node parts.
\end{itemize}


\subsubsection{Normal Anchors Versus Saved Anchors}

Anchors  are special places in shape. For example, the |north east|
anchor, which is a normal anchor, lies at the upper right corner of
the  |rectangle| shape, as does |\northeast|, which is a saved
anchor. The difference is the following: \emph{saved anchors are 
  computed and stored for each node, anchors are only computed as
  needed.} The user only has access to the normal anchors, but a
normal anchor can just ``copy'' or ``pass through'' the location of a
saved anchor. 

The idea behind all this is that a shape can declare a very large
number of normal anchors, but when a node of this shape is created,
these anchors are not actually computed. However, this causes a
problem: When we wish to reference an anchor of a node at some later
time, we must still able to compute the position of the anchor. For 
this, we may need a lot of information: What was the transformation
matrix that was in force when the node was created? What was the size
of the text box? What were the values of the different separation
dimensions? And so on. 

To solve this problem, \pgfname\ will always compute the locations of
all \emph{saved anchors} and store these positions. Then, when an
normal anchor position is requested later on, the anchor position can
be given just from knowing where the locations of the saved anchors.

As an example, consider the |rectangle| shape. For this shape two
anchors are saved: The |\northeast| corner and the |\southwest|
corner. A normal anchor like |north west| can now easily be expressed
in terms of these coordinates: Take the $x$-position of the
|\southwest| point  and the $y$-position of the |\northeast| point. 
The |rectangle| shape currently defines 13 normal anchors, but needs
only two saved anchors. Adding new anchors like a  |south south east|
anchor would not increase the memory and computation requirements of
pictures. 

All anchors (both saved and normal) are specified in a local
\emph{shape coordinate space}. This is also true for the background
and foreground paths. The |\pgfnode| macro will automatically apply
appropriate transformations to the coordinates so that the shape is
shifted to the right anchor or otherwise transformed. 


\subsubsection{Command for Declaring New Shapes}

The following command declares a new shape:
\begin{command}{\pgfdeclareshape\marg{shape name}\marg{shape
      specification}}
  This command declares a new shape named \meta{shape name}. The shape
  name can later be used in commands like |\pgfnode|.

  The \meta{shape specification} is some \TeX\ code containing calls
  to special commands that are only defined inside the \meta{shape
    specification} (similarly to commands like |\draw| that are only
  available inside the |{tikzpicture}| environment).

  \example Here is the code of the |coordinate| shape:
\begin{codeexample}[code only]
\pgfdeclareshape{coordinate}
{
  \savedanchor\centerpoint{%
    \pgf@x=.5\wd\pgfnodeparttextbox%
    \pgf@y=.5\ht\pgfnodeparttextbox%
    \advance\pgf@y by -.5\dp\pgfnodeparttextbox%
  }
  \anchor{center}{\centerpoint}
  \anchorborder{\centerpoint}
}
\end{codeexample}

  The special commands are explained next. In the examples given for
  the special commands a new shape will be constructed, which we might
  call |simple rectangle|. It should behave like the normal rectangle
  shape, only without bothering about the fine details like inner and
  outer separations. The skeleton for the shape is the following.
\begin{codeexample}[code only]
\pgfdeclareshape{simple rectangle}{
  ...
}
\end{codeexample}

  \begin{command}{\nodeparts\marg{list of node parts}}
    This command declares which parts make up nodes of this shape. A
    \emph{node part} is a (possibly empty) text label that is drawn
    when a node of the shape is created.

    By default, a shape has just one node part called |text|. However,
    there can be several node parts. For example, the
    |circle split| shape has two parts: the |text| part, which
    shows that upper text, and a |lower| part, which shows the
    lower text. For the |circle split| shape the |\nodeparts| command
    was called with the argument |{text,lower}|.

    When a multipart node is created, the text labels are drawn in the
    sequences listed in the \meta{list of node parts}. For each node
    part there you must have declared one anchor and the \TeX-box of
    the part is placed at this anchor. For a node part called |XYZ|
    the \TeX-box |\pgfnodepartXYZbox| is placed at anchor |XYZ|.
  \end{command}

  \begin{command}{\savedanchor\marg{command}\marg{code}}
    This command declares a saved anchor. The argument \meta{command}
    should be a \TeX\ macro name like |\centerpoint|.

    The \meta{code} will be executed each time |\pgfnode| (or
    |\pgfmultipartnode|) is called to  create a node of the shape
    \meta{shape name}. When the \meta{code} 
    is executed, the \TeX-boxes of the node parts will contain the
    text labels of the node. Possibly, these box are void. For
    example, if there is just a |text| part, the node
    |\pgfnodeparttextbox| will be setup when the \meta{code} is
    executed. 

    The \meta{code} can use the width, height, and depth of the
    box(es) to compute the location of the saved anchor. In addition,
    the \meta{code} can take into account the values of dimensions like
    |\pgfshapeminwidth| or |\pgfshapeinnerxsep|. Furthermore, the
    \meta{code} can take into consideration the values of any further
    shape-specific variables that are set at the moment when
    |\pgfnode| is called.

    The net effect of the \meta{code} should be to set the two \TeX\
    dimensions |\pgf@x| and |\pgf@y|. One way to achieve this is to
    say |\pgfpoint{|\meta{x value}|}{|\meta{y value}|}| at the end of
    the \meta{code}, but you can also just set these variables.
    The values that |\pgf@x| and |\pgf@y| have after the code has been
    executed, let us call them $x$ and $y$, will be recorded and
    stored together with the node that is created by the command
    |\pgfnode|.

    The macro \meta{command} is defined to be
    |\pgfpoint{|$x$|}{|$y$|}|. However, the \meta{command} is only
    locally defined while anchor positions are being computed. Thus,
    it is possible to use very simple names for \meta{command}, like
    |\center| or |\a|, without causing a name-clash. (To be precise,
    very simple \meta{command} names will clash with existing names,
    but only locally inside the computation of anchor positions; and
    we do not need the normal |\center| command during these
    computations.)

    For our |simple rectangle| shape, we will need only one saved
    anchor: The upper right corner. The lower left corner could either
    be the origin or the ``mirrored'' upper right corner, depending on
    whether we want the text label to have its lower left corner at
    the origin or whether the text label should be centered on the
    origin. Either will be fine, for the final shape this will make no
    difference since the shape will be shifted anyway. So, let us
    assume that the text label is centered on the origin (this will be
    specified later on using the |text| anchor). We get 
    the following code for the upper right corner:
\begin{codeexample}[code only]
\savedanchor{\upperrightcorner}{
  \pgf@y=.5\ht\pgfnodeparttextbox % height of the box, ignoring the depth
  \pgf@x=.5\wd\pgfnodeparttextbox % width of the box
}
\end{codeexample}

    If we wanted to take, say, the |\pgfshapeminwidth| into account,
    we could use the following code:
    
\begin{codeexample}[code only]
\savedanchor{\upperrightcorner}{
  \pgf@y=.\ht\pgfnodeparttextbox % height of the box
  \pgf@x=.\wd\pgfnodeparttextbox % width of the box
  \setlength{\pgf@xa}{\pgfshapeminwidth}
  \ifdim\pgf@x<.5\pgf@xa
    \pgf@x=.5\pgf@xa
  \fi
}
\end{codeexample}
    Note that we could not have written |.5\pgfshapeminwidth| since
    the minium width is stored in a ``plain text macro,'' not as a
    real dimension. So if |\pgfshapeminwidth| depth were 
    2cm, writing |.5\pgfshapeminwidth| would yield the same as |.52cm|.

    In the ``real'' |rectangle| shape the code is somewhat more
    complex, but you get the basic idea.
  \end{command}  
  \begin{command}{\saveddimen\marg{command}\marg{code}}
    This command is similar to |\savedanchor|, only instead of setting
    \meta{command} to |\pgfpoint{|$x$|}{|$y$|}|, the \meta{command} is
    set just to (the value of) $x$.

    In the |simple rectangle| shape we might use a saved dimension to
    store the depth of the shape box.
  
\begin{codeexample}[code only]
\shapedimen{\depth}{
  \pgf@x=\dp\pgfnodeparttextbox 
}
\end{codeexample}
  \end{command}  
  \begin{command}{\anchor\marg{name}\marg{code}}
    This command declares an anchor named \meta{name}. Unlike for saved
    anchors, the \meta{code} will not be executed each time a node is
    declared. Rather, the \meta{code} is only executed when the anchor
    is specifically requested; either for anchoring the node during
    its creation or as a  position in the shape referenced later on.

    The \meta{name} is a quite arbitrary string that is not ``passed
    down'' to the system level. Thus, names like |south| or |1| or
    |::| would all be fine.

    A saved anchor is not automatically also a normal anchor. If you
    wish to give the users access to a saved anchor you must declare a
    normal anchor that just returns the position of the saved anchor.

    When the \meta{code} is executed, all saved anchor macros will be
    defined. Thus, you can reference them in your \meta{code}. The
    effect of the \meta{code} should be to set the values of |\pgf@x|
    and |\pgf@y| to the coordinates of the anchor.

    Let us consider some example for the |simple rectangle|
    shape. First, we would like to make the upper right corner
    publicly available, for example as |north east|:
    
\begin{codeexample}[code only]
\anchor{north east}{\upperrightcorner}
\end{codeexample}

    The |\upperrightcorner| macro will set |\pgf@x| and |\pgf@y| to
    the coordinates of the upper right corner. Thus, |\pgf@x| and
    |\pgf@y| will have exactly the right values at the end of the
    anchor's code.

    Next, let us define a |north west| anchor. For this anchor, we can
    negate the |\pgf@x| variable:
   
\begin{codeexample}[code only]
\anchor{north west}{
  \upperrightcorner
  \pgf@x=-\pgf@x
}
\end{codeexample}

    Finally, it is a good idea to always define a |center| anchor,
    which will be the default location for a shape.

\begin{codeexample}[code only]
\anchor{center}{\pgfpointorigin}
\end{codeexample}

    You might wonder whether we should not take into consideration
    that the node is not placed at the origin, but has been shifted
    somewhere. However, the anchor positions are always specified in
    the shape's ``private'' coordinate system. The ``outer''
    transformation that has been applied to the shape upon its
    creation is applied automatically to the coordinates returned by
    the anchor's \meta{code}.

    Out |simple rectangle| only has one text label (node
    part) called |text|. This is the default situation, so we need not
    do anything. For the |text| node part we must setup a |text|
    anchor. This   anchor is used upon creation of a node to determine
    the lower left  corner of the text label (within the private
    coordinate system of the shape). By default, the |text| anchor is
    at the origin, but you may change this. For example, we would say
\begin{codeexample}[code only]
\anchor{text}{%
  \upperrightcorner%
  \pgf@x=-\pgf@x%
  \pgf@y=-\pgf@y%
}
\end{codeexample}
    to center the text label on the origin in the shape coordinate
    space. Note that we could \emph{not} have written the following:
    
\begin{codeexample}[code only]
\anchor{text}{\pgfpoint{-.5\wd\pgfnodeparttextbox}{-.5\ht\pgfnodeparttextbox}}
\end{codeexample}
    Do you see why this is wrong? The problem is that the box
    |\pgfnodeparttextbox| will most likely not have the correct size
    when the anchor is computed. After all, the anchor position might
    be recomputed at a time when several other nodes have been created. 

    If a shape has several node parts, we would have to define an
    anchor for each part.    
  \end{command}  
  \begin{command}{\anchorborder\marg{code}}
    A \emph{border anchor} is an anchor point on the border of the
    shape. What exactly is considered as the ``border'' of the shape
    depends on the shape.

    When the user request a point on the border of the shape using the
    |\pgfpointshapeborder| command, the \meta{code} will be executed
    to discern this point. When the execution of  the \meta{code}
    starts, the dimensions |\pgf@x| and |\pgf@y| will have been set to
    a location $p$ in the shape's coordinate system. It is now the job of
    the \meta{code} to setup |\pgf@x| and |\pgf@y| such that they
    specify the point on the shape's border that lies on a straight
    line from the shape's center to the point $p$. Usually, this is a
    somewhat complicated computation, involving many case distinctions
    and some basic math.

    For our |simple rectangle| we must compute a point on the border
    of a rectangle whose one corner is the origin (ignoring the depth
    for simplicity) and whose other corner is |\upperrightcorner|. The
    following code might be used:
\begin{codeexample}[code only]
\anchorborder{%
  % Call a function that computes a border point. Since this
  % function will modify dimensions like \pgf@x, we must move them to
  % other dimensions.
  \@tempdima=\pgf@x
  \@tempdimb=\pgf@y
  \pgfpointborderrectangle{\pgfpoint{\@tempdima}{\@tempdimb}}{\upperrightcorner}
}
\end{codeexample}
  \end{command}  
  \begin{command}{\backgroundpath\marg{code}}
    This command specifies the path that ``makes up'' the background
    of the shape. Note that the shape cannot prescribe what is going
    to happen with the path: It might be drawn, shaded, filled, or
    even thrown away. If you want to specify that something should
    ``always'' happen when this shape is drawn (for example, if the
    shape is a stop-sign, we \emph{always} want it to be filled with a
    red color), you can use commands like |\beforebackgroundpath|,
    explained below.

    When the \meta{code} is executed, all saved anchors will be in
    effect. The \meta{code} should contain path construction
    commands.

    For our |simple rectangle|, the following code might be used:
\begin{codeexample}[code only]
\backgroundpath{
  \pgfpathrectanglecorners
    {\upperrightcorner}
    {\pgfpointscale{-1}{\upperrightcorner}}
}  
\end{codeexample}
    As the name suggests, the background path is used ``behind'' the
    text labels. Thus, this path is used first, then the text labels are
    drawn, possibly obscuring part of the path.
  \end{command}  
  \begin{command}{\foregroundpath\marg{code}}
    This command works like |\backgroundpath|, only it is invoked
    after the text labels have been drawn. This means that this path can
    possibly obscure (part of) the text labels.
  \end{command}  
  \begin{command}{\behindbackgroundpath\marg{code}}
    Unlike the previous two commands, \meta{code} should not only
    construct a path, it should also use this path in whatever way is
    appropriate. For example, the \meta{code} might fill some area
    with a uniform color.

    Whatever the \meta{code} does, it does it first. This means that
    any drawing done by \meta{code} will be even behind the background
    path.

    Note that the \meta{code} is protected with a |{pgfscope}|.
  \end{command}  
  \begin{command}{\beforebackgroundpath\marg{code}}
    This command works like |\behindbackgroundpath|, only the
    \meta{code} is executed after the background path has been used,
    but before the texts label are drawn.
  \end{command}  
  \begin{command}{\behindforegroundpath\marg{code}}
    The \meta{code} is executed after the text labels have been drawn,
    but before the foreground path is used.
  \end{command}  
  \begin{command}{\beforeforegroundpath\marg{code}}
    This \meta{code} is executed at the very end.
  \end{command}  
  \begin{command}{\inheritsavedanchors|[from=|\marg{another shape name}|]|}
    This command allows you to inherit the code for saved anchors from
    \meta{another shape name}. The idea is that if you wish to create
    a new shape that is just a small modification of a another shape,
    you can recycle the code used for \meta{another shape name}.

    The effect of this command is the same as if you had called
    |\savedanchor| and |\saveddimen| for each saved anchor or saved
    dimension declared in \meta{another shape name}. Thus, it is not
    possible to ``selectively'' inherit only some saved anchors, you
    always have to inherit all saved anchors from another
    shape. However, you can inherit the saved anchors of more than one
    shape by calling this command several times.
  \end{command}  
  \begin{command}{\inheritbehindbackgroundpath|[from=|\marg{another shape name}|]|}
    This command can be used to inherit the code used for the
    drawings behind the background path from \meta{another shape name}. 
  \end{command}  
  \begin{command}{\inheritbackgroundpath|[from=|\marg{another shape name}|]|}
    Inherits the background path code from \meta{another shape name}.
  \end{command}  
  \begin{command}{\inheritbeforebackgroundpath|[from=|\marg{another shape name}|]|}
    Inherits the before background path code from \meta{another shape name}.
  \end{command}  
  \begin{command}{\inheritbehindforegroundpath|[from=|\marg{another shape name}|]|}
    Inherits the behind foreground path code from \meta{another shape name}.
  \end{command}  
  \begin{command}{\inheritforegroundpath|[from=|\marg{another shape name}|]|}
    Inherits the foreground path code from \meta{another shape name}.
  \end{command}  
  \begin{command}{\inheritbeforeforegroundpath|[from=|\marg{another shape name}|]|}
    Inherits the before foreground path code from \meta{another shape name}.
  \end{command}  
  \begin{command}{\inheritanchor|[from=|\marg{another shape name}|]|\marg{name}}
    Inherits the code of one specific anchor named \meta{name} from
    \meta{another shape name}. Thus, unlike saved anchors, which must
    be inherited collectively, normal anchors can and must be
    inherited individually.
  \end{command}  
  \begin{command}{\inheritanchorborder|[from=|\marg{another shape name}|]|}
    Inherits the border anchor code from \meta{another shape name}.
  \end{command}

  The following example shows how a shape can be defined that relies
  heavily on inheritance:
\makeatletter
\begin{codeexample}[]
\pgfdeclareshape{document}{
  \inheritsavedanchors[from=rectangle] % this is nearly a rectangle
  \inheritanchorborder[from=rectangle]
  \inheritanchor[from=rectangle]{center}
  \inheritanchor[from=rectangle]{north}
  \inheritanchor[from=rectangle]{south}
  \inheritanchor[from=rectangle]{west}
  \inheritanchor[from=rectangle]{east}
  % ... and possibly more
  \backgroundpath{% this is new
    % store lower right in xa/ya and upper right in xb/yb
    \southwest \pgf@xa=\pgf@x \pgf@ya=\pgf@y
    \northeast \pgf@xb=\pgf@x \pgf@yb=\pgf@y
    % compute corner of ``flipped page''
    \pgf@xc=\pgf@xb \advance\pgf@xc by-5pt % this should be a parameter
    \pgf@yc=\pgf@yb \advance\pgf@yc by-5pt
    % construct main path
    \pgfpathmoveto{\pgfpoint{\pgf@xa}{\pgf@ya}}
    \pgfpathlineto{\pgfpoint{\pgf@xa}{\pgf@yb}}
    \pgfpathlineto{\pgfpoint{\pgf@xc}{\pgf@yb}}
    \pgfpathlineto{\pgfpoint{\pgf@xb}{\pgf@yc}}
    \pgfpathlineto{\pgfpoint{\pgf@xb}{\pgf@ya}}
    \pgfpathclose
    % add little corner
    \pgfpathmoveto{\pgfpoint{\pgf@xc}{\pgf@yb}}
    \pgfpathlineto{\pgfpoint{\pgf@xc}{\pgf@yc}}
    \pgfpathlineto{\pgfpoint{\pgf@xb}{\pgf@yc}}
    \pgfpathlineto{\pgfpoint{\pgf@xc}{\pgf@yc}}
 }
}\hskip-1.2cm
\begin{tikzpicture}
  \node[shade,draw,shape=document,inner sep=2ex] (x) {Remark};
  \node[fill=examplefill,draw,ellipse,double]
    at ([shift=(-80:3cm)]x) (y) {Use Case};

  \draw[dashed] (x) -- (y);  
\end{tikzpicture}
\end{codeexample}
  
\end{command}




\subsection{Predefined Shapes}

\begin{shape}{coordinate}
  The |coordinate| is mainly intended to be used to store locations
  using the node mechanism. This shape does not have any background
  path and options like |draw| have no effect on it. Also, it does not
  have any node parts, so no text is drawn when this shape is used.

  \tikzname\ handles this shape in a special way, see
  Section~\ref{section-tikz-coordinate-shape}. 
\end{shape}

\begin{shape}{rectangle}
  This shape is a rectangle tightly fitting the text box. Use inner or
  outer separation to increase the distance between the text box and
  the border and the anchors. The following figure shows the anchors
  defined by this shape; the anchors |10| and |130| are example of border
  anchors. 
\begin{codeexample}[]
\Huge
\begin{tikzpicture}
  \node[name=s,shape=rectangle,style=shape example] {Rectangle\vrule width 1pt height 2cm};
  \foreach \anchor/\placement in
    {north west/above left, north/above, north east/above right, 
     west/left, center/above, east/right, 
     mid west/right, mid/above, mid east/left, 
     base west/left, base/below, base east/right, 
     south west/below left, south/below, south east/below right, 
     text/left, 10/right, 130/above}
    \draw[shift=(s.\anchor)] plot[mark=x] coordinates {(0,0)}
      node[\placement] {\scriptsize\texttt{(s.\anchor)}};
\end{tikzpicture}
\end{codeexample}
\end{shape}

\begin{shape}{circle}
  This shape is a circle tightly fitting the text box.
\begin{codeexample}[]
\Huge
\begin{tikzpicture}
  \node[name=s,shape=circle,style=shape example] {Circle\vrule width 1pt height 2cm};
  \foreach \anchor/\placement in
    {north west/above left, north/above, north east/above right, 
     west/left, center/above, east/right, 
     mid west/right, mid/above, mid east/left, 
     base west/left, base/below, base east/right, 
     south west/below left, south/below, south east/below right, 
     text/left, 10/right, 130/above}
     \draw[shift=(s.\anchor)] plot[mark=x] coordinates {(0,0)}
       node[\placement] {\scriptsize\texttt{(s.\anchor)}};
\end{tikzpicture}
\end{codeexample}
\end{shape}



%%% Local Variables: 
%%% mode: latex
%%% TeX-master: "pgfmanual"
%%% End: 

% Copyright 2006 by Till Tantau
%
% This file may be distributed and/or modified
%
% 1. under the LaTeX Project Public License and/or
% 2. under the GNU Free Documentation License.
%
% See the file doc/generic/pgf/licenses/LICENSE for more details.


\section{Matrices}

\label{section-base-matrices}

\begin{pgfmodule}{matrix}
  The present section documents the commands of this module.
\end{pgfmodule}

\subsection{Overview}

Matrices are a mechanism for aligning several so-called cell pictures
horizontally and vertically. The resulting alignment is placed in a
normal node and the command for creating matrices, |\pgfmatrix|, takes
options very similar to the |\pgfnode| command.

In the following, the basic idea behind the alignment mechanism is
explained first. Then the command |\pgfmatrix| is explained. At the
end of the section, additional ways of modifying the width of columns
and rows are discussed.


\subsection{Cell Pictures and Their Alignment}

A matrix consists of rows of \emph{cells}. Cells are separated using
the special command |\pgfmatrixnextcell|, rows are ended using the
command |\pgfmatrixendrow| (the command |\\| is set up to mean the same
as |\pgfmatrixendrow| by default). Each cell contains a \emph{cell
  picture}, although cell pictures are not complete pictures as they
lack layers. However, each cell picture has its own bounding box like a
normal picture does. These bounding boxes are important for the
alignment as explained in the following.

Each cell picture will have an origin somewhere in the picture (or
even outside the picture). The position of these origins are important
for the alignment: On each row the origins will be on the same
horizontal line and for each column the origins will also be on the
same vertical line. These two requirements mean that the cell pictures
may need to be shifted around so that the origins wind up on the same
lines. The top of a row is given by the top of the cell picture whose
bounding box's maximum $y$-position is largest. Similarly, the bottom
of a row is given by the bottom of the cell picture whose bounding
box's minimum $y$-position is the most negative. Similarly, the left
end of a row is given by the left end of the cell whose bounding box's
$x$-position is the most negative; and similarly for the right end of
a row.

\begin{codeexample}[]
\begin{tikzpicture}[x=3mm,y=3mm,fill=blue!50]
  \def\atorig#1{\node[black] at (0,0) {\tiny #1};}

  \pgfmatrix{rectangle}{center}{mymatrix}
    {\pgfusepath{}}{\pgfpointorigin}{}
    {
      \fill (0,-3)  rectangle (1,1);\atorig1 \pgfmatrixnextcell
      \fill (-1,0)  rectangle (1,1);\atorig2 \pgfmatrixnextcell
      \fill (-1,-2) rectangle (0,0);\atorig3 \pgfmatrixnextcell
      \fill (-1,-1) rectangle (0,3);\atorig4 \\
      \fill (-1,0)  rectangle (4,1);\atorig5 \pgfmatrixnextcell
      \fill (0,-1)  rectangle (1,1);\atorig6 \pgfmatrixnextcell
      \fill (0,0)   rectangle (1,4);\atorig7 \pgfmatrixnextcell
      \fill (-1,-1) rectangle (0,0);\atorig8 \\
    }
\end{tikzpicture}
\end{codeexample}


\subsection{The Matrix Command}

All matrices are typeset using the following command:

\begin{command}{\pgfmatrix\marg{shape}\marg{anchor}\marg{name}%
    \marg{usage}\marg{shift}\marg{pre-code}\marg{matrix cells}}

  This command creates a node that contains a matrix. The name of the
  node is \meta{name}, its shape is \meta{shape} and the node is
  anchored at \meta{anchor}.

  The \meta{matrix cell} parameter contains the cells of the
  matrix. In each cell drawing commands may be given, which create a
  so-called cell picture. For each cell picture a bounding box is
  computed and the cells are aligned according to the rules outlined
  in the previous section.

  The resulting matrix is used as the |text| box of the node. As for a
  normal node, the \meta{usage} commands are applied, so that the
  path(s) of the resulting node is (are) stroked or filled or whatever.

  \medskip
  \textbf{Specifying the cells and rows.\ }
  Even though this command uses |\halign| internally, there are two
  special rules for indicating cells:
  \begin{enumerate}
  \item Cells in the same row must be separated using the macro
    |\pgfmatrixnextcell| rather than |&|. Using |&| will result in an
    error message.

    However, you can make |&| an active character and have it expand
    to |\pgfmatrixnextcell|. This way, it will ``look'' as if |&| is
    used.
  \item Rows are ended using the command |\pgfmatrixendrow|, but |\\|
    is set up to mean the same by default. However, some environments
    like |{minipage}| redefine |\\|, so it is good to have
    |\pgfmatrixendrow| as a ``fallback.''
  \item Every row \emph{including the last row} must be ended using
    the command |\\| or |\pgfmatrixendrow|.
  \end{enumerate}

  Both |\pgfmatrixnextcell| and |\pgfmatrixendrow| (and, thus, also
  |\\|) take an optional argument as explained in the
  Section~\ref{section-matrix-spacing}

\begin{codeexample}[]
\begin{tikzpicture}
  \pgfmatrix{rectangle}{center}{mymatrix}
    {\pgfusepath{}}{\pgfpointorigin}{}
    {
      \node {a}; \pgfmatrixnextcell \node {b}; \pgfmatrixendrow
      \node {c}; \pgfmatrixnextcell \node {d}; \pgfmatrixendrow
    }
\end{tikzpicture}
\end{codeexample}

  \medskip
  \textbf{Anchoring matrices at nodes inside the matrix.\ }
  The parameter \meta{shift} is an additional negative shift for the
  node. Normally, such a shift could be given beforehand (that is, the
  shift could be preapplied to the current transformation
  matrix). However, when \meta{shift} is evaluated, you can refer to
  \emph{temporary} positions of nodes inside the matrix. In detail,
  the following happens: When the matrix has been typeset, all nodes
  in the matrix temporarily get assigned their positions in the matrix
  box. The origin of this coordinate system is at the left baseline
  end of the matrix box, which corresponds to the |text| anchor. The
  position \meta{shift} is then interpreted inside this coordinate
  system and then used for shifting.

  This allows you to use the parameter \meta{shift} in the following
  way: If you use |text| as the \meta{anchor} and specify
  |\pgfpointanchor{inner node}{some anchor}| for the parameter
  \meta{shift}, where |inner node| is a node that
  is created in the matrix, then the whole matrix will be shifted such
  that |inner node.some anchor| lies at the origin of the whole
  picture.

  \medskip
  \textbf{Rotations and scaling.\ }
  The matrix node is never rotated or shifted, because the current
  coordinate transformation matrix is reset (except for the
  translational part) at the beginning of |\pgfmatrix|. This is
  intentional and will not change in the future. If you need to rotate
  the matrix, you must install an appropriate canvas transformation
  yourself.

  However, nodes and stuff inside the cell pictures can be rotated and
  scaled normally.

  \medskip
  \textbf{Callbacks.\ }
  At the beginning and at the end of each cell the special macros
  |\pgfmatrixbegincode|, |\pgfmatrixendcode| and possibly
  |\pgfmatrixemptycode| are called. The effect is explained in
  Section~\ref{section-matrix-callbacks}.

  \medskip
  \textbf{Executing extra code.\ }
  The parameter \meta{pre-code} is executed at the beginning of the
  outermost \TeX-group enclosing the matrix node. It is inside this
  \TeX-group, but outside the matrix itself. It can be used
  for different purposes:
  \begin{enumerate}
  \item It can be used to simplify the next cell macro. For example,
    saying |\let\&=\pgfmatrixnextcell| allows you to use |\&| instead
    of |\pgfmatrixnextcell|. You can also set the catcode of |&| to
    active.
  \item It can be used to issue an |\aftergroup| command. This allows
    you to regain control after the |\pgfmatrix| command. (If you do
    not know the |\aftergroup| command, you are probably blessed with
    a simple and happy life.)
  \end{enumerate}

  \medskip
  \textbf{Special considerations concerning macro expansion.\ }
  As said before, the matrix is typeset using |\halign|
  internally. This command does a lot of strange and magic things like
  expanding the first macro of every cell in a most unusual
  manner. Here are some effects you may wish to be aware of:
  \begin{itemize}
  \item It is not necessary to actually mention |\pgfmatrixnextcell|
    or |\pgfmatrixendrow| inside the \meta{matrix cells}. It suffices
    that the  macros inside \meta{matrix cells} expand to these macros
    sooner or later.
  \item In particular, you can define clever macros that insert
    columns and rows as needed for special effects.
  \end{itemize}
\end{command}


\subsection{Row and Column Spacing}
\label{section-matrix-spacing}

It is possible to control the space between columns and rows rather
detailedly. Two commands are important for the row spacing and two
commands for the column spacing.

\begin{command}{\pgfsetmatrixcolumnsep\marg{sep list}}
  This macro sets the default separation list for columns. The details of the
  format of this list are explained in the description of the next command.
\end{command}

\begin{command}{\pgfmatrixnextcell\opt{\oarg{additional sep list}}}
  This command has two purposes: First, it is used to separate
  cells. Second, by providing the optional argument \meta{additional
    sep list} you can modify the spacing between the columns that are
  separated by this command.

  The optional \meta{additional sep list} may only be provided when
  the |\pgfmatrixnextcell| command starts a new column. Normally, this
  will only be the case in the first row, but sometimes a later row
  has more elements than the first row. In this case, the
  |\pgfmatrixnextcell| commands that start the new columns in the
  later row may also have the optional argument. Once a column has
  been started, subsequent uses of this optional argument for the
  column have no effect.

  To determine the space between the two columns that are separated by
  |\pgfmatrixnextcell|, the following algorithm is executed:
  \begin{enumerate}
  \item Both the default separation list (as set up by
    |\pgfsetmatrixcolumnsep|) and the \meta{additional sep list} are
    processed, in this order. If the \meta{additional sep list}
    argument is missing, only the default separation list is
    processed.
  \item Both lists may contain dimensions, separated by commas, as
    well as occurrences of the keywords |between origins| and
    |between borders|.
  \item All dimensions occurring in either list are added together to
    arrive at a dimension $d$.
  \item The last occurrence of either of the keywords is located. If
    neither keyword is present, we proceed as if |between borders|
    were present.
  \end{enumerate}
  At the end of the algorithm, a dimension $d$ has been computed and
  one of the two \emph{modes} |between borders| and |between origins|
  has been determined. Depending on which mode has been determined,
  the following happens:
  \begin{itemize}
  \item For the |between borders| mode, an additional horizontal space
    of $d$ is added between the two columns. Note that $d$ may be
    negative.
  \item For the |between origins| mode, the spacing between the two
    columns is computed differently: Recall that the origins of the
    cell pictures in both pictures lie on two vertical lines. The
    spacing between the two columns is set up such that the horizontal
    distance between these two lines is exactly $d$.

    This mode may only be used between columns \emph{already
      introduced in the first row}.
  \end{itemize}
  All of the above rules boil down to the following effects:
  \begin{itemize}
  \item A default spacing between columns should be set up using
    |\pgfsetmatrixcolumnsep|. For example, you might say
    |\pgfsetmatrixcolumnsep{5pt}| to have columns spaced apart by
    |5pt|. You could say
\begin{verbatim}
\pgfsetmatrixcolumnsep{1cm,between origins}
\end{verbatim}
    to specify that horizontal space between the origins of cell
    pictures in adjacent columns should be 1cm by default --
    regardless of the actual size of the cell pictures.
  \item You can now use the optional argument of |\pgfmatrixnextcell|
    to locally overrule the spacing between two columns. By saying
    |\pgfmatrixnextcell[5pt]| you \emph{add} 5pt to the space between
    of the two columns, regardless of the mode.

    You can also (locally) change the spacing mode for these two
    columns. For example, even if the normal spacing mode is
    |between origins|, you can say
\begin{verbatim}
\pgfmatrixnextcell[5pt,between borders]
\end{verbatim}
    to locally change the mode for these columns to
    |between borders|.
  \end{itemize}

    \begin{codeexample}[]
\begin{tikzpicture}[every node/.style=draw]
  \pgfsetmatrixcolumnsep{1mm}
  \pgfmatrix{rectangle}{center}{mymatrix}
    {\pgfusepath{}}{\pgfpointorigin}{\let\&=\pgfmatrixnextcell}
  {
    \node {8}; \&[2mm] \node{1}; \&[-1mm] \node {6}; \\
    \node {3}; \&      \node{5}; \&       \node {7}; \\
    \node {4}; \&      \node{9}; \&       \node {2}; \\
  }
\end{tikzpicture}
    \end{codeexample}
    \begin{codeexample}[]
\begin{tikzpicture}[every node/.style=draw]
  \pgfsetmatrixcolumnsep{1mm}
  \pgfmatrix{rectangle}{center}{mymatrix}
    {\pgfusepath{}}{\pgfpointorigin}{\let\&=\pgfmatrixnextcell}
  {
    \node {8}; \&[2mm] \node(a){1}; \&[1cm,between origins] \node(b){6}; \\
    \node {3}; \&      \node   {5}; \&                      \node   {7}; \\
    \node {4}; \&      \node   {9}; \&                      \node   {2}; \\
  }
  \draw [<->,red,thick,every node/.style=] (a.center) -- (b.center)
        node [above,midway] {11mm};
\end{tikzpicture}
    \end{codeexample}
    \begin{codeexample}[]
\begin{tikzpicture}[every node/.style=draw]
  \pgfsetmatrixcolumnsep{1cm,between origins}
  \pgfmatrix{rectangle}{center}{mymatrix}
    {\pgfusepath{}}{\pgfpointorigin}{\let\&=\pgfmatrixnextcell}
  {
    \node (a) {8}; \& \node (b) {1}; \&[between borders] \node (c) {6}; \\
    \node     {3}; \& \node     {5}; \&                  \node     {7}; \\
    \node     {4}; \& \node     {9}; \&                  \node     {2}; \\
  }
  \begin{scope}[every node/.style=]
    \draw [<->,red,thick] (a.center) -- (b.center) node [above,midway] {10mm};
    \draw [<->,red,thick] (b.east) -- (c.west) node [above,midway]
    {10mm};
  \end{scope}
\end{tikzpicture}
    \end{codeexample}
\end{command}

The mechanism for the between-row-spacing is the same, only the
commands are called differently.


\begin{command}{\pgfsetmatrixrowsep\marg{sep list}}
  This macro sets the default separation list for rows.
\end{command}



\begin{command}{\pgfmatrixendrow\opt{\oarg{additional sep list}}}
  This command ends a line. The optional \meta{additional sep list} is
  used to determine the spacing between the row being ended and the
  next row. The modes and the computation of $d$ is done in the same
  way as for columns. For the last row the optional argument has no
  effect.

  Inside matrices (and only there) the command |\\| is set up to mean
  the same as this command.
\end{command}


\subsection{Callbacks}
\label{section-matrix-callbacks}

There are three macros that get called at the beginning and end of
cells. By redefining these macros, which are empty by default, you can
change the appearance of cells in a very general manner.

\begin{command}{\pgfmatrixemptycode}
  This macro is executed for empty cells. This means that \pgfname\
  uses some macro magic to determine whether a cell is empty (it
  immediately ends with |\pgfmatrixemptycode| or |\pgfmatrixendrow|)
  and, if so, put this macro inside the cell.
  \begin{codeexample}[]
\begin{tikzpicture}
  \def\pgfmatrixemptycode{\node{empty};}
  \pgfmatrix{rectangle}{center}{mymatrix}
    {\pgfusepath{}}{\pgfpointorigin}{\let\&=\pgfmatrixnextcell}
  {
    \node {a}; \&           \& \node {b}; \\
               \& \node{c}; \& \node {d}; \& \\
  }
\end{tikzpicture}
  \end{codeexample}
  As can be seen, the macro is not executed for empty cells at the end
  of row when columns are added only later on.
\end{command}


\begin{command}{\pgfmatrixbegincode}
  This macro is executed at the beginning of non-empty
  cells. Correspondingly, |\pgfmatrixendcode| is added at
  the end of every non-empty cell.
  \begin{codeexample}[]
\begin{tikzpicture}
  \def\pgfmatrixbegincode{\node[draw]\bgroup}
  \def\pgfmatrixendcode{\egroup;}
  \pgfmatrix{rectangle}{center}{mymatrix}
    {\pgfusepath{}}{\pgfpointorigin}{\let\&=\pgfmatrixnextcell}
  {
    a \& b \& c \\
    d \&   \& e \\
  }
\end{tikzpicture}
  \end{codeexample}
  Note that between |\pgfmatrixbegincode| and |\pgfmatrixendcode|
  there will \emph{not} only be the contents of the cell. Rather,
  \pgfname\ will add some (invisible) commands for book-keeping
  purposes that involve |\let| and |\gdef|. In particular, it is not a
  good idea to have |\pgfmatrixbegincode| end with |\csname| and
  |\pgfmatrixendcode| start with |\endcsname|.
\end{command}

\begin{command}{\pgfmatrixendcode}
  See the explanation above.
\end{command}


The following two counters allow you to access the current row and
current column in a callback:

\begin{command}{\pgfmatrixcurrentrow}
  This counter stores the current row of the current cell of the
  matrix. Do not even think about changing this counter.
\end{command}

\begin{command}{\pgfmatrixcurrentcolumn}
  This counter stores the current column of the current cell of the
  matrix.
\end{command}

%%% Local Variables:
%%% mode: latex
%%% TeX-master: "pgfmanual"
%%% End:

% Copyright 2006 by Till Tantau
%
% This file may be distributed and/or modified
%
% 1. under the LaTeX Project Public License and/or
% 2. under the GNU Free Documentation License.
%
% See the file doc/generic/pgf/licenses/LICENSE for more details.

\section{Coordinate and Canvas Transformations}

\subsection{Overview}

\pgfname\ offers two different ways of scaling, shifting, and rotating
(these operations are generally known as \emph{transformations})
graphics: You can apply \emph{coordinate transformations} to all
coordinates and you can apply \emph{canvas transformations} to the
canvas on which you draw. (The names ``coordinate'' and ``canvas''
transformations are not standard, I introduce them only for the
purposes of this manual.) 

The difference is the following:

\begin{itemize}
\item
  As the name ``coordinate transformation'' suggests, coordinate
  transformations apply only to coordinates. For example, when you
  specify a coordinate like |\pgfpoint{1cm}{2cm}| and you wish to
  ``use'' this coordinate---for example as an argument to a
  |\pgfpathmoveto| command---then the coordinate transformation matrix
  is applied to the coordinate, resulting in a new
  coordinate. Continuing the example, if the current coordinate
  transformation is ``scale by a factor of two,'' the coordinate
  |\pgfpoint{1cm}{2cm}| actually designates the point
  $(2\mathrm{cm},4\mathrm{cm})$. 

  Note that coordinate transformations apply \emph{only} to
  coordinates. They do not apply to, say, line width or shadings or
  text.
\item
  The effect of a ``canvas transformation'' like ``scale by a factor
  of two'' can be imagined as follows: You first draw your picture on
  a ``rubber canvas'' normally. Then, once you are done, the whole
  canvas is transformed, in this case stretched by a factor of
  two. In the resulting image \emph{everything} will be larger: Text,
  lines, coordinates, and shadings.
\end{itemize}

In many cases, it is preferable that you use coordinate
transformations and not canvas transformations. When canvas
transformations are used, \pgfname\ looses track of the coordinates of
nodes and shapes. Also, canvas transformations often cause undesirable
effects like changing text size. For these reasons, \pgfname\ makes it
easy to setup the coordinate transformation, but a bit harder to
change the canvas transformation.


\subsection{Coordinate Transformations}

\subsubsection{How PGF Keeps Track of the Coordinate Transformation
  Matrix}

\pgfname\ has an internal coordinate transformation matrix. This
matrix is applied to coordinates ``in certain situations.'' This means
that the matrix is not always applied to every coordinate ``no matter
what.'' Rather, \pgfname\ tries to be reasonably smart at when and how
this matrix should be applied. The most prominent examples are the
path construction commands, which apply the coordinate transformation
matrix to their inputs.

The coordinate transformation matrix consists of four numbers $a$,
$b$, $c$, and $d$, and two dimensions $s$ and $t$. When the coordinate
transformation matrix is applied to a coordinate $(x,y)$ the new
coordinate $(ax+by+s,cx+dy+t)$ results. For more details on how
transformation matrices work in general, please see, for example, the
\textsc{pdf} or PostScript reference or a textbook on computer
graphics.

The coordinate transformation matrix is equal to the identity matrix
at the beginning. More precisely, $a=1$, $b=0$, $c=0$, $d=1$,
$s=0\mathrm{pt}$, and $t=0\mathrm{pt}$.

The different coordinate transformation commands will modify the
matrix by concatenating it with another transformation matrix. This
way the effect of applying several transformation commands will
\emph{accumulate}.

The coordinate transformation matrix is local to the current \TeX\
group (unlike the canvas transformation matrix, which is local to the
current |{pgfscope}|). Thus, the effect of adding a coordinate
transformation to the coordinate transformation matrix will last only
till the end of the current \TeX\ group.




\subsubsection{Commands for Relative Coordinate Transformations}

The following commands add a basic coordinate transformation to the
current coordinate transformation matrix. For all commands, the
transformation is applied \emph{in addition} to any previous
coordinate transformations.

\begin{command}{\pgftransformshift\marg{point}}
  Shifts coordinates by \meta{point}.
\begin{codeexample}[]
\begin{tikzpicture}
  \draw[help lines] (0,0) grid (3,2);
  \draw      (0,0) -- (2,1) -- (1,0);
  \pgftransformshift{\pgfpoint{1cm}{1cm}}
  \draw[red] (0,0) -- (2,1) -- (1,0);
\end{tikzpicture}
\end{codeexample}
\end{command}

\begin{command}{\pgftransformxshift\marg{dimensions}}
  Shifts coordinates by \meta{dimension} along the $x$-axis.
\begin{codeexample}[]
\begin{tikzpicture}
  \draw[help lines] (0,0) grid (3,2);
  \draw      (0,0) -- (2,1) -- (1,0);
  \pgftransformxshift{.5cm}
  \draw[red] (0,0) -- (2,1) -- (1,0);
\end{tikzpicture}
\end{codeexample}
\end{command}

\begin{command}{\pgftransformyshift\marg{dimensions}}
  Like |\pgftransformxshift|, only for the $y$-axis.
\end{command}

\begin{command}{\pgftransformscale\marg{factor}}
  Scales coordinates by \meta{factor}.
\begin{codeexample}[]
\begin{tikzpicture}
  \draw[help lines] (0,0) grid (3,2);
  \draw      (0,0) -- (2,1) -- (1,0);
  \pgftransformscale{.75}
  \draw[red] (0,0) -- (2,1) -- (1,0);
\end{tikzpicture}
\end{codeexample}
\end{command}

\begin{command}{\pgftransformxscale\marg{factor}}
  Scales coordinates by \meta{factor} in the $x$-direction.
\begin{codeexample}[]
\begin{tikzpicture}
  \draw[help lines] (0,0) grid (3,2);
  \draw      (0,0) -- (2,1) -- (1,0);
  \pgftransformxscale{.75}
  \draw[red] (0,0) -- (2,1) -- (1,0);
\end{tikzpicture}
\end{codeexample}
\end{command}


\begin{command}{\pgftransformyscale\marg{factor}}
  Like |\pgftransformxscale|, only for the $y$-axis.
\end{command}


\begin{command}{\pgftransformxslant\marg{factor}}
  Slants coordinates by \meta{factor} in the $x$-direction. Here, a
  factor of |1| means $45^\circ$.
\begin{codeexample}[]
\begin{tikzpicture}
  \draw[help lines] (0,0) grid (3,2);
  \draw      (0,0) -- (2,1) -- (1,0);
  \pgftransformxslant{.5}
  \draw[red] (0,0) -- (2,1) -- (1,0);
\end{tikzpicture}
\end{codeexample}
\end{command}


\begin{command}{\pgftransformyslant\marg{factor}}
  Slants coordinates by \meta{factor} in the $y$-direction.
\begin{codeexample}[]
\begin{tikzpicture}
  \draw[help lines] (0,0) grid (3,2);
  \draw      (0,0) -- (2,1) -- (1,0);
  \pgftransformyslant{-1}
  \draw[red] (0,0) -- (2,1) -- (1,0);
\end{tikzpicture}
\end{codeexample}
\end{command}

  

\begin{command}{\pgftransformrotate\marg{degrees}}
  Rotates coordinates counterclockwise by \meta{degrees}.
\begin{codeexample}[]
\begin{tikzpicture}
  \draw[help lines] (0,0) grid (3,2);
  \draw      (0,0) -- (2,1) -- (1,0);
  \pgftransformrotate{30}
  \draw[red] (0,0) -- (2,1) -- (1,0);
\end{tikzpicture}
\end{codeexample}
\end{command}

  

\begin{command}{\pgftransformtriangle\marg{a}\marg{b}\marg{c}}
  This command transforms the coordinate system in such a way that the
  triangle given by the points \meta{a}, \meta{b} and \meta{c} lies at
  the coordinates $(0,0)$, $(1\mathrm{pt},0\mathrm{pt})$ and
  $(0\mathrm{pt},1\mathrm{pt})$. 
\begin{codeexample}[]
\begin{tikzpicture}
  \draw[help lines] (0,0) grid (3,2);
  \pgftransformtriangle
  {\pgfpoint{1cm}{0cm}}
  {\pgfpoint{0cm}{2cm}}
  {\pgfpoint{3cm}{1cm}}
  
  \draw (0,0) -- (1pt,0pt) -- (0pt,1pt) -- cycle;
\end{tikzpicture}
\end{codeexample}
\end{command}

  
\begin{command}{\pgftransformcm\marg{a}\marg{b}\marg{c}\marg{d}\marg{point}}
  Applies the transformation matrix given by $a$, $b$, $c$, and $d$
  and the shift \meta{point} to coordinates (in addition to any
  previous transformations already in force).
\begin{codeexample}[]
\begin{tikzpicture}
  \draw[help lines] (0,0) grid (3,2);
  \draw      (0,0) -- (2,1) -- (1,0);
  \pgftransformcm{1}{1}{0}{1}{\pgfpoint{.25cm}{.25cm}}
  \draw[red] (0,0) -- (2,1) -- (1,0);
\end{tikzpicture}
\end{codeexample}
\end{command}

  
\begin{command}{\pgftransformarrow\marg{start}\marg{end}}
  Shift coordinates to the end of the line going from \meta{start} 
  to \meta{end} with the correct rotation. 
\begin{codeexample}[]
\begin{tikzpicture}
  \draw[help lines] (0,0) grid (3,2);
  \draw      (0,0) -- (3,1);
  \pgftransformarrow{\pgfpointorigin}{\pgfpoint{3cm}{1cm}}
  \pgftext{tip}
\end{tikzpicture}
\end{codeexample}
\end{command}

  
\begin{command}{\pgftransformlineattime\marg{time}\marg{start}\marg{end}}
  Shifts coordinates by a specific point on a line at a specific
  time. The point by which the coordinate is shifted is calculated by
  calling |\pgfpointlineattime|, see
  Section~\ref{section-pointsattime}.

  In addition to shifting the coordinate, a rotation \emph{may} also
  be applied. Whether this is the case depends on whether the \TeX\ if
  |\ifpgfslopedattime| is set to true or not.
\begin{codeexample}[]
\begin{tikzpicture}
  \draw[help lines] (0,0) grid (3,2);
  \draw      (0,0) -- (2,1);
  \pgftransformlineattime{.25}{\pgfpointorigin}{\pgfpoint{2cm}{1cm}}
  \pgftext{Hi!}
\end{tikzpicture}
\end{codeexample}
\begin{codeexample}[]
\begin{tikzpicture}
  \draw[help lines] (0,0) grid (3,2);
  \draw      (0,0) -- (2,1);
  \pgfslopedattimetrue
  \pgftransformlineattime{.25}{\pgfpointorigin}{\pgfpoint{2cm}{1cm}}
  \pgftext{Hi!}
\end{tikzpicture}
\end{codeexample}
  If |\ifpgfslopedattime| is true, another \TeX\ |\if| is important:
  |\ifpgfallowupsidedowattime|. If this is false, \pgfname\ will
  ensure that the rotation is done in such a way that text is never
  ``upside down.''

  There is another \TeX\ if that influences this command. If you set
  |\ifpgfresetnontranslationattime| to true, then, between
  shifting the coordinate and (possibly) rotating/sloping the
  coordinate, the command |\pgftransformresetnontranslations| is
  called. See the description of this command for details.
\begin{codeexample}[]
\begin{tikzpicture}
  \draw[help lines] (0,0) grid (3,2);
  \pgftransformscale{1.5}
  \draw      (0,0) -- (2,1);
  \pgfslopedattimetrue
  \pgfresetnontranslationattimefalse
  \pgftransformlineattime{.25}{\pgfpointorigin}{\pgfpoint{2cm}{1cm}}
  \pgftext{Hi!}
\end{tikzpicture}
\end{codeexample}
\begin{codeexample}[]
\begin{tikzpicture}
  \draw[help lines] (0,0) grid (3,2);
  \pgftransformscale{1.5}
  \draw      (0,0) -- (2,1);
  \pgfslopedattimetrue
  \pgfresetnontranslationattimetrue
  \pgftransformlineattime{.25}{\pgfpointorigin}{\pgfpoint{2cm}{1cm}}
  \pgftext{Hi!}
\end{tikzpicture}
\end{codeexample}
\end{command}


\begin{command}{\pgftransformcurveattime\marg{time}\marg{start}\marg{first
      support}\marg{second support}\marg{end}}
  Shifts coordinates by a specific point on a curve at a specific
  time, see  Section~\ref{section-pointsattime} once more.

  As for the line-at-time transformation command, |\ifpgfslopedattime|
  decides whether an additional rotation should be applied. Again, the
  value of |\ifpgfallowupsidedowattime| is also considered. 
\begin{codeexample}[]
\begin{tikzpicture}
  \draw[help lines] (0,0) grid (3,2);
  \draw      (0,0) .. controls (0,2) and (1,2) .. (2,1);
  \pgftransformcurveattime{.25}{\pgfpointorigin}
    {\pgfpoint{0cm}{2cm}}{\pgfpoint{1cm}{2cm}}{\pgfpoint{2cm}{1cm}}
  \pgftext{Hi!}
\end{tikzpicture}
\end{codeexample}
\begin{codeexample}[]
\begin{tikzpicture}
  \draw[help lines] (0,0) grid (3,2);
  \draw      (0,0) .. controls (0,2) and (1,2) .. (2,1);
  \pgfslopedattimetrue
  \pgftransformcurveattime{.25}{\pgfpointorigin}
    {\pgfpoint{0cm}{2cm}}{\pgfpoint{1cm}{2cm}}{\pgfpoint{2cm}{1cm}}
  \pgftext{Hi!}
\end{tikzpicture}
\end{codeexample}
  The value of |\ifpgfresetnontranslationsattime| is also taken into account.
\end{command}


{
  \let\ifpgfslopedattime=\relax
  \begin{textoken}{\ifpgfslopedattime}
    Decides whether the ``at time'' transformation commands also
    rotate coordinates or not.
  \end{textoken}
}
{
  \let\ifpgfallowupsidedowattime=\relax
  \begin{textoken}{\ifpgfallowupsidedowattime}
    Decides whether the ``at time'' transformation commands should
    allow the rotation be down in such a way that ``upside-down text''
    can result.
  \end{textoken}
}
{
  \let\ifpgfresetnontranslationsattime=\relax
  \begin{textoken}{\ifpgfresetnontranslationsattime}
    Decides whether the ``at time'' transformation commands should
    reset the non-translations between shifting and rotating.
  \end{textoken}
}


\subsubsection{Commands for Absolute Coordinate Transformations}

The coordinate transformation commands introduced up to now are always
applied in addition to any previous transformations. In contrast, the
commands presented in the following can be used to change the
transformation matrix ``absolutely.'' Note that this is, in general,
dangerous and will often produce unexpected effects. You should use
these commands only if you really know what you are doing.

\begin{command}{\pgftransformreset}
  Resets the coordinate transformation matrix to the identity
  matrix. Thus, once this command is given no transformations are
  applied till the end of the scope.
\begin{codeexample}[]
\begin{tikzpicture}
  \draw[help lines] (0,0) grid (3,2);
  \pgftransformrotate{30}
  \draw      (0,0) -- (2,1) -- (1,0);
  \pgftransformreset
  \draw[red] (0,0) -- (2,1) -- (1,0);
\end{tikzpicture}
\end{codeexample}
\end{command}


\begin{command}{\pgftransformresetnontranslations}
  This command sets the $a$, $b$, $c$, and $d$ part of the coordinate
  transformation matrix to $a=1$, $b=0$, $c=0$, and $d=1$. However,
  the current shifting of the matrix is not modified.

  The effect of this command is that any rotation/scaling/slanting is
  undone in the current \TeX\ group, but the origin is not ``moved
  back.''

  This command is mostly useful directly before a |\pgftext| command
  to ensure that the text is not scaled or rotated.
\begin{codeexample}[]
\begin{tikzpicture}
  \draw[help lines] (0,0) grid (3,2);
  \pgftransformscale{2}
  \pgftransformrotate{30}
  \pgftransformxshift{1cm}
  {\color{red}\pgftext{rotated}}
  \pgftransformresetnontranslations
  \pgftext{shifted only}
\end{tikzpicture}
\end{codeexample}
\end{command}


\begin{command}{\pgftransforminvert}
  Replaces the coordinate transformation matrix by a coordinate
  transformation matrix that ``exactly undoes the original
  transformation.'' For example, if the original transformation was
  ``scale by 2 and then shift right by 1cm'' the new one is ``shift
  left by 1cm and then scale by $1/2$.''

  This command will produce an error if the determinant of
  the matrix is too small, that is, if the matrix is near-singular.
\begin{codeexample}[]
\begin{tikzpicture}
  \draw[help lines] (0,0) grid (3,2);
  \pgftransformrotate{30}
  \draw      (0,0) -- (2,1) -- (1,0);
  \pgftransforminvert
  \draw[red] (0,0) -- (2,1) -- (1,0);
\end{tikzpicture}
\end{codeexample}
\end{command}

\subsubsection{Saving and Restoring the Coordinate Transformation
  Matrix}

There are two commands for saving and restoring coordinate
transformation matrices.

\begin{command}{\pgfgettransform\marg{macro}}
  This command will (locally) define \meta{macro} to a representation
  of the current coordinate transformation matrix. This matrix can
  later on be reinstalled using |\pgfsettransform|.
\end{command}


\begin{command}{\pgfsettransform\marg{macro}}
  Reinstalls a coordinate transformation matrix that was previously
  saved using |\pgfgettransform|.
\end{command}



\subsection{Canvas Transformations}

The canvas transformation matrix is not managed by \pgfname, but by
the output format like \pdf\ or PostScript. All the \pgfname\ does is
to call appropriate low-level |\pgfsys@| commands to change the canvas
transformation matrix.

Unlike coordinate transformations, canvas transformations apply to
``everything,'' including images, text, shadings, line thickness, and
so on. The idea is that a canvas transformation really stretches and
deforms the canvas after the graphic is finished.

Unlike coordinate transformations, canvas transformations are local to
the current |{pgfscope}|, not to the current \TeX\ group. This is due
to the fact that they are managed by the backend driver, not by \TeX\
or \pgfname.

Unlike the coordinate transformation matrix, it is not possible to
``reset'' the canvas transformation matrix. The only way to change it
is to concatenate it with another canvas transformation matrix or to
end the current |{pgfscope}|.

Unlike coordinate transformations, \pgfname\ does not ``keep track''
of canvas transformations. In particular, it will not be able to
correctly save the coordinates of shapes or nodes when a canvas
transformation is used.

\pgfname\ does not offer a whole set of special commands for modifying
the canvas transformation matrix. Instead, different commands allow
you to concatenate the canvas transformation matrix with a coordinate
transformation matrix (and there are numerous commands for specifying
a coordinate transformation, see the previous section).

\begin{command}{\pgflowlevelsynccm}
  This command concatenates the canvas transformation matrix with the
  current coordinate transformation matrix. Afterward, the coordinate
  transformation matrix is reset.

  The effect of this command is to ``synchronize'' the coordinate
  transformation matrix and the canvas transformation matrix. All
  transformations that were previously applied by the coordinate
  transformations matrix are now applied by the canvas transformation
  matrix.

\begin{codeexample}[]
\begin{tikzpicture}
  \draw[help lines] (0,0) grid (3,2);
  \pgfsetlinewidth{1pt}
  \pgftransformscale{5}
  \draw      (0,0) -- (0.4,.2);
  \pgftransformxshift{0.2cm}
  \pgflowlevelsynccm
  \draw[red] (0,0) -- (0.4,.2);
\end{tikzpicture}
\end{codeexample}
\end{command}


\begin{command}{\pgflowlevel\marg{transformation code}}
  This command concatenates the canvas transformation matrix with the
  coordinate transformation specified by \meta{transformation code}.

\begin{codeexample}[]
\begin{tikzpicture}
  \draw[help lines] (0,0) grid (3,2);
  \pgfsetlinewidth{1pt}
  \pgflowlevel{\pgftransformscale{5}}
  \draw      (0,0) -- (0.4,.2);
\end{tikzpicture}
\end{codeexample}
\end{command}


\begin{command}{\pgflowlevelobj\marg{transformation code}\marg{code}}
  This command creates a local |{pgfscope}|. Inside this scope,
  |\pgflowlevel| is first called with the argument
  \meta{transformation code}, then the \meta{code} is inserted. 

\begin{codeexample}[]
\begin{tikzpicture}
  \draw[help lines] (0,0) grid (3,2);
  \pgfsetlinewidth{1pt}
  \pgflowlevelobj{\pgftransformscale{5}}    {\draw (0,0) -- (0.4,.2);}
  \pgflowlevelobj{\pgftransformxshift{-1cm}}{\draw (0,0) -- (0.4,.2);}
\end{tikzpicture}
\end{codeexample}
\end{command}


\begin{environment}{{pgflowlevelscope}\marg{transformation code}}
  This environment first surrounds the \meta{environment contents} by
  a |{pgfscope}|. Then it calls |\pgflowlevel| with the argument
  \meta{transformation code}.

\begin{codeexample}[]
\begin{tikzpicture}
  \draw[help lines] (0,0) grid (3,2);
  \pgfsetlinewidth{1pt}
  \begin{pgflowlevelscope}{\pgftransformscale{5}}
    \draw (0,0) -- (0.4,.2);
  \end{pgflowlevelscope}
  \begin{pgflowlevelscope}{\pgftransformxshift{-1cm}}
    \draw (0,0) -- (0.4,.2);
  \end{pgflowlevelscope}
\end{tikzpicture}
\end{codeexample}
\end{environment}


\begin{plainenvironment}{{pgflowlevelscope}\marg{transformation code}}
  Plain \TeX\ version of the environment.
\end{plainenvironment}

\begin{contextenvironment}{{pgflowlevelscope}\marg{transformation code}}
  Con\TeX t version of the environment.
\end{contextenvironment}




%%% Local Variables: 
%%% mode: latex
%%% TeX-master: "pgfmanual"
%%% End: 

% Copyright 2003 by Till Tantau <tantau@cs.tu-berlin.de>.
%
% This program can be redistributed and/or modified under the terms
% of the LaTeX Project Public License Distributed from CTAN
% archives in directory macros/latex/base/lppl.txt.


\section{Patterns}

\label{section-patterns}

\begin{package}{pgfbasepattterns}
  This package provides commands for declaring and using patterns. The
  package is loaded automatically by 
  |pgf|, but you can load it manually if you have only included
  |pgfcore|.   
\end{package}



\subsection{Overview}

There are many ways of filling a path. First, you can fill it using a
solid color and this is also the fasted method. Second, you can also
fill it using a shading, which means that the color changes smoothly
between two (or more) different colors. Third, you can fill it using a
tiling pattern and it is explained in the following how this is done.

A tiling pattern can be imagined as a rectangular tile (hence the
name) on which a small picture is painted. There is not a single tile,
but (conceptually) an infinite number of tiles, all showing the same 
picture, and these tiles are arranged horizontally and vertically to
fill the plane. When you use a tiling pattern to fill a path, what
happens is that the path clips out a ``window'' through which we see
part of this infinite plane.

Patterns come in two versions: \emph{inherently colored patterns} and
\emph{form-only patterns}. (These are often called ``color patterns''
and ``uncolored patterns,'' but these names are misleading since
uncolored patterns do have a color and the color changes. As I said, 
the name is misleading\dots) An inherently colored pattern is just a
colored tile like, say, a red star with a black outline. A form-only
pattern can be imagined as a tile that is a kind of rubber stamp. When
this pattern is used, the stamp is used to print copies of the stamp
picture onto the plane, but we can use a different stamp color each
time we use a form-only pattern.

\pgfname\ provides a special support for patterns. You can declare a
pattern and then use it very much like a fill color. \pgfname\
directly maps patterns to the pattern facilities of the underlying
graphic languages (PostScript, \textsc{pdf}, and \textsc{svg}). This
means that filling a path using a pattern will be nearly as fast as if
you used a uniform color.

There are a number of pitfalls and restrictions when using
patterns. First, once a pattern has been declared, you cannot change
it anymore. In particular, it is not possible to enlarge it or change
the line width. Such flexibility would require that the repeating of
the pattern were not done by the graphic language, but on the
\pgfname\ level. This would make patterns orders of magnitude slower
to produce and to render.

Second, the phase of patterns is not well-defined, that is, it is not
clear where origin of the ``first'' tile is. To be more precise,
PostScript and \textsc{pdf} on the one hand and \textsc{svg} on the
other hand define the origin differently. PostScript and \textsc{pdf}
define a fixed origin that is independent of where the path lies. This
has the highly desirable effect that if you use the same pattern to
fill multiple paths, this has the same effect as if you used the
pattern to will a single path that is the union of all the paths. By
comparison, \textsc{svg} uses the upper-left (?) corner of the path to
be filled as the origin. However, the \textsc{svg} specification is a
bit vague on this question.


\subsection{Declaring a Pattern}

Before a pattern can be used, it must be declared. The following
command is used for this:

\begin{command}{\pgfdeclarepatternformonly%
    \marg{name}%
    \marg{lower left}%
    \marg{upper right}%
    \marg{tile size}%
    \marg{code}}
  This command declares a new form-only pattern. The \marg{name} is a
  name for later reference. The two parameters \marg{lower left} and
  \marg{upper right} must describe a bounding box that is large enough
  to encompass the complete tile.

  The size of a tile is given by \meta{tile size}, that is, a tile is
  a rectangle whose lower left   corner is the origin and whose upper
  right corner is given by \meta{tile size}. This might make you
  wonder why the second and third parameters are needed. First, the
  bounding box might be smaller than the tile size if the tile is
  larger than the picture on the tile. Second, the bounding box might
  be bigger, in which case the picture will ``bleed'' over the tile.

  The \meta{code} should be \pgfname\ code than can be protocolled. It
  should not contain any color code.

  
\begin{codeexample}[]
\pgfdeclarepatternformonly{stars}
{\pgfpointorigin}{\pgfpoint{1cm}{1cm}}
{\pgfpoint{1cm}{1cm}}
{
  \pgftransformshift{\pgfpoint{.5cm}{.5cm}}
  \pgfpathmoveto{\pgfpointpolar{0}{4mm}}
  \pgfpathlineto{\pgfpointpolar{144}{4mm}}
  \pgfpathlineto{\pgfpointpolar{288}{4mm}}
  \pgfpathlineto{\pgfpointpolar{72}{4mm}}
  \pgfpathlineto{\pgfpointpolar{216}{4mm}}
  \pgfpathclose%
  \pgfusepath{fill}
}
\begin{tikzpicture}
  \filldraw[pattern=stars] (0,0)   rectangle (1.5,2);  
  \filldraw[pattern=stars,pattern color=red]
                           (1.5,0) rectangle (3,2);  
\end{tikzpicture}
\end{codeexample}
\end{command}

\begin{command}{\pgfdeclarepatterninherentlycolored
    \marg{name}
    \marg{lower left}
    \marg{upper right}
    \marg{tile size}
    \marg{code}}
  This command works like |\pgfdeclarepatternuncolored|, only the
  pattern will have an inherent color. To set the color, you should
  use \pgfname's color commands, not the |\color| command, since this
  fill not be protocolled.
  
\begin{codeexample}[]
\pgfdeclarepatterninherentlycolored{green stars}
{\pgfpointorigin}{\pgfpoint{1cm}{1cm}}
{\pgfpoint{1cm}{1cm}}
{
  \pgfsetfillcolor{green!50!black}
  \pgftransformshift{\pgfpoint{.5cm}{.5cm}}
  \pgfpathmoveto{\pgfpointpolar{0}{4mm}}
  \pgfpathlineto{\pgfpointpolar{144}{4mm}}
  \pgfpathlineto{\pgfpointpolar{288}{4mm}}
  \pgfpathlineto{\pgfpointpolar{72}{4mm}}
  \pgfpathlineto{\pgfpointpolar{216}{4mm}}
  \pgfpathclose%
  \pgfusepath{stroke,fill}
}
\begin{tikzpicture}
  \filldraw[pattern=green stars] (0,0) rectangle (3,2);  
\end{tikzpicture}
\end{codeexample}
\end{command}


\subsection{Setting a Pattern}

Once a pattern has been declared, it can be used.

\begin{command}{\pgfsetfillpattern\marg{name}\marg{color}}
  This command specifies that paths that are filled should be filled
  with the ``color'' by the pattern \meta{name}. For an inherently
  colored pattern, the \meta{color} parameter is ignored. For
  form-only patterns, the \meta{color} parameter specified the color
  to be used for the pattern.
\begin{codeexample}[]
\begin{tikzpicture}
  \pgfsetfillpattern{stars}{red}
  \filldraw (0,0) rectangle (1.5,2);  

  \pgfsetfillpattern{green stars}{red}
  \filldraw (1.5,0) rectangle (3,2);  
\end{tikzpicture}
\end{codeexample} 
\end{command}



%%% Local Variables: 
%%% mode: latex
%%% TeX-master: "pgfmanual"
%%% End: 

% Copyright 2003 by Till Tantau <tantau@cs.tu-berlin.de>.
%
% This program can be redistributed and/or modified under the terms
% of the LaTeX Project Public License Distributed from CTAN
% archives in directory macros/latex/base/lppl.txt.


\section{Declaring and Using Images}
\label{section-images}


This section describes the |pgfbaseimage| package.

\begin{package}{pgfbaseimage}
  This package offers an abstraction of the image inclusion
  process. It is loaded automatically by |pgf|, but you can load it
  manually if you have  only included |pgfcore|.  
\end{package}

\subsection{Overview}

To be quite frank, \LaTeX's |\includegraphics| is designed better than
|pgfbaseimage|. For this reason, \emph{I recommend that you use the
  standard image inclusion mechanism of your format}. Thus, \LaTeX\
users are encouraged to use |\includegraphics| to include images.

However, there are reasons why you might need to use the image
inclusion facilities of \pgfname:
\begin{itemize}
\item
  There is no standard image inclusion mechanism in your format. For
  example, plain \TeX\ does not have one, so \pgfname's inclusion
  mechanism is ``better than nothing.''

  However, this applies only to the |pdftex| backend. For all other
  backends, \pgfname\ currently maps its commands back to the |graphicx|
  package. Thus, in plain \TeX, this does not really help. It might be
  a good idea to fix this in the future such that \pgfname\ becomes
  independent of \LaTeX, thereby providing a uniform image abstraction
  for all formats. 
\item
  You wish to use masking. This is a feature that is only supported by
  \pgfname, though I hope that someone will implement this also for
  the graphics package in \LaTeX\ in the future.
\end{itemize}

Whatever your choice, you can still use the usual image inclusion
facilities of the |graphics| package.

The general approach taken by \pgfname\ to including an image is the
following: First, |\pgfdeclareimage| declares the
image. This must be done prior to the first use of the image. Once you
have declared an image, you can insert it into the text using
|\pgfuseimage|. The advantage of this two-phase approach is that, at
least for \textsc{pdf}, the image data will only be included once in the
file. This can drastically reduce the file size if you use an image
repeatedly, for example in an overlay. However, there is also a
command called |\pgfimage| that declares and then immediately uses the
image.

To speedup the compilation, you may wish to use the following class
option:
\begin{packageoption}{draft}
  In draft mode boxes showing the image name replace the
  images. It is checked whether the image files exist, but they are
  not read. If either height or width is not given, 1cm is used
  instead. 
\end{packageoption}

\subsection{Declaring an Image}

\begin{command}{\pgfdeclareimage\oarg{options}\marg{image
      name}\marg{filename}}
  Declares an image, but does not paint anything. To draw the image,
  use |\pgfuseimage{|\meta{image name}|}|. The \meta{filename} may not
  have an extension.  For \textsc{pdf}, the extensions |.pdf|, |.jpg|,
  and |.png| will automatically tried. For PostScript, the extensions
  |.eps|, |.epsi|, and |.ps| will be tried. 

  The following options are possible:
  \begin{itemize}
  \item
    \declare{|height=|\meta{dimension}} sets the height of the
    image. If the width is not specified simultaneously, the aspect
    ratio of the image is kept.
  \item
    \declare{|width=|\meta{dimension}} sets the width of the
    image. If the height is not specified simultaneously, the aspect
    ratio of the image is kept.
  \item
    \declare{|page=|\meta{page number}} selects a given page number
    from a multipage document. Specifying this option will have the
    following effect: first, \pgfname\ tries to find a file named
    \begin{quote}
      \meta{filename}|.page|\meta{page number}|.|\meta{extension}
    \end{quote}
    If such a file is found, it will be used instead of the originally
    specified filename. If not, \pgfname\ inserts the image stored in
    \meta{filename}|.|\meta{extension} and if a recent version of
    |pdflatex| is used, only the selected page is inserted. For older
    versions of |pdflatex| and for |dvips| the complete document is
    inserted and a warning is printed.    
  \item
    \declare{|interpolate=|\meta{true or false}} selects whether the
    image should ``smoothed'' when zoomed. False by default.
  \item
    \declare{|mask=|\meta{mask name}} selects a transparency mask. The
    mask must previously be declared using |\pgfdeclaremask| (see
    below). This option only has an effect for |pdf|. Not all viewers
    support masking. 
  \end{itemize}

\begin{codeexample}[code only]
\pgfdeclareimage[interpolate=true,height=1cm]{image1}{pgf-tu-logo}
\pgfdeclareimage[interpolate=true,width=1cm,height=1cm]{image2}{pgf-tu-logo}
\pgfdeclareimage[interpolate=true,height=1cm]{image3}{pgf-tu-logo}
\end{codeexample}
\end{command}


\begin{command}{\pgfaliasimage\marg{new image name}\marg{existing image name}}
  The \marg{existing image name} is ``cloned'' and the \marg{new image
    name} can now be used whenever original image is used. This
  command is useful for creating aliases for alternate extensions
  and for accessing the last image inserted using |\pgfimage|.

  \example |\pgfaliasimage{image.!30!white}{image.!25!white}|
\end{command}


\subsection{Using an Image}

\begin{command}{\pgfuseimage\marg{image name}}
  Inserts a previously declared image into the \emph{normal text}. If
  you wish to use it in a |{pgfpicture}| environment, you must put a
  |\pgftext| around it.

  If the macro |\pgfalternateextension| expands to some nonempty
  \meta{alternate extension}, \pgfname\ will first try to use the image
  names \meta{image name}|.|\meta{alternate extension}. If this
  image is not defined, \pgfname\ will next check whether \meta{alternate
    extension} contains a |!| character. If so, everything up to this
  exclamation mark and including it is deleted from \meta{alternate
    extension} and the \pgfname\ again tries to use the image \meta{image
    name}|.|\meta{alternate extension}. This is repeated until
  \meta{alternate extension} no longer contains a~|!|. Then the
  original image is used.

  The |xxcolor| package sets the alternate extension to the current
  color mixin. 

\begin{codeexample}[]
\pgfdeclareimage[interpolate=true,width=1cm,height=1cm]{image1}{pgf-tu-logo}
\pgfdeclareimage[interpolate=true,width=1cm]{image2}{pgf-tu-logo}
\pgfdeclareimage[interpolate=true,height=1cm]{image3}{pgf-tu-logo}
\begin{pgfpicture}
  \pgftext[at=\pgfpoint{1cm}{5cm},left,base]{\pgfuseimage{image1}}
  \pgftext[at=\pgfpoint{1cm}{3cm},left,base]{\pgfuseimage{image2}}
  \pgftext[at=\pgfpoint{1cm}{1cm},left,base]{\pgfuseimage{image3}}

  \pgfpathrectangle{\pgfpoint{1cm}{5cm}}{\pgfpoint{1cm}{1cm}}
  \pgfpathrectangle{\pgfpoint{1cm}{3cm}}{\pgfpoint{1cm}{1cm}}
  \pgfpathrectangle{\pgfpoint{1cm}{1cm}}{\pgfpoint{1cm}{1cm}}
  \pgfusepath{stroke}
\end{pgfpicture}
\end{codeexample}

  The following example demonstrates the effect of using
  |\pgfuseimage| inside a color mixin environment.

\begin{codeexample}[]
\pgfdeclareimage[interpolate=true,width=1cm,height=1cm]
  {image1.!25!white}{pgf-tu-logo.25}
\pgfdeclareimage[interpolate=true,width=1cm]
  {image2.25!white}{pgf-tu-logo.25}
\pgfdeclareimage[interpolate=true,height=1cm]
  {image3.white}{pgf-tu-logo.25}
\begin{colormixin}{25!white}
\begin{pgfpicture}
  \pgftext[at=\pgfpoint{1cm}{5cm},left,base]{\pgfuseimage{image1}}
  \pgftext[at=\pgfpoint{1cm}{3cm},left,base]{\pgfuseimage{image2}}
  \pgftext[at=\pgfpoint{1cm}{1cm},left,base]{\pgfuseimage{image3}}

  \pgfpathrectangle{\pgfpoint{1cm}{5cm}}{\pgfpoint{1cm}{1cm}}
  \pgfpathrectangle{\pgfpoint{1cm}{3cm}}{\pgfpoint{1cm}{1cm}}
  \pgfpathrectangle{\pgfpoint{1cm}{1cm}}{\pgfpoint{1cm}{1cm}}
  \pgfusepath{stroke}
\end{pgfpicture}
\end{colormixin}
\end{codeexample}
\end{command}

\begin{command}{\pgfalternateextension}
  You should redefine this command to install a different alternate
  extension.

  \example |\def\pgfalternateextension{!25!white}|
\end{command}


\begin{command}{\pgfimage\oarg{options}\marg{filename}}
  Declares the image under the name |pgflastimage| and
  immediately uses it. You can ``save'' the image for later usage by
  invoking |\pgfaliasimage| on |pgflastimage|.
  
\begin{codeexample}[]
\begin{colormixin}{25!white}
\begin{pgfpicture}
  \pgftext[at=\pgfpoint{1cm}{5cm},left,base]
    {\pgfimage[interpolate=true,width=1cm,height=1cm]{pgf-tu-logo}}
  \pgftext[at=\pgfpoint{1cm}{3cm},left,base]
    {\pgfimage[interpolate=true,width=1cm]{pgf-tu-logo}}
  \pgftext[at=\pgfpoint{1cm}{1cm},left,base]
    {\pgfimage[interpolate=true,height=1cm]{pgf-tu-logo}}

  \pgfpathrectangle{\pgfpoint{1cm}{5cm}}{\pgfpoint{1cm}{1cm}}
  \pgfpathrectangle{\pgfpoint{1cm}{3cm}}{\pgfpoint{1cm}{1cm}}
  \pgfpathrectangle{\pgfpoint{1cm}{1cm}}{\pgfpoint{1cm}{1cm}}
  \pgfusepath{stroke}
\end{pgfpicture}
\end{colormixin}
\end{codeexample}
\end{command}



\subsection{Masking an Image}


\begin{command}{\pgfdeclaremask\oarg{options}\marg{mask  name}\marg{filename}}
  Declares a transparency mask named \meta{mask name} (called a
  \emph{soft mask} in the \textsc{pdf} specification). This mask is
  read from the file \meta{filename}. This file should contain a
  grayscale image that is as large as the actual image. A white
  pixel in the mask will correspond to ``transparent,'' a black pixel
  to ``solid,'' and gray values correspond to intermediate values. The
  mask must have a single ``color channel.'' This means that the
  mask must be a ``real'' grayscale image, not an \textsc{rgb}-image
  in which all \textsc{rgb}-triples happen to have the same
  components.

  You can only mask images the are in a ``pixel format.'' These are
  |.jpg| and |.png|.  You cannot mask |.pdf| images in this way. Also,
  again, the mask file and the image file must have the same size.

  The following options may be given:
  \begin{itemize}
  \item |matte=|\marg{color components} sets the so-called
    \emph{matte} of the actual image (strangely, this has to be
    specified together with the mask, not with the image itself). The
    matte is the color that has been used to preblend the image. For
    example, if the image has been preblended with a red background,
    then \meta{color components} should be set to |{1 0 0}|. The
    default is |{1 1 1}|, which is white in the rgb model.

    The matte is specified in terms of the parent's image color
    space. Thus, if the parent is a grayscale image, the matte has to
    be set to |{1}|.
  \end{itemize}
  \example
\begin{codeexample}[]
%% Draw a large colorful background
\pgfdeclarehorizontalshading{colorful}{5cm}{color(0cm)=(red);
color(2cm)=(green); color(4cm)=(blue); color(6cm)=(red);
color(8cm)=(green); color(10cm)=(blue); color(12cm)=(red);
color(14cm)=(green)}
\hbox{\pgfuseshading{colorful}\hskip-14cm\hskip1cm
\pgfimage[height=4cm]{pgf-apple}\hskip1cm
\pgfimage[height=4cm]{pgf-apple.mask}\hskip1cm
\pgfdeclaremask{mymask}{pgf-apple.mask}
\pgfimage[mask=mymask,height=4cm,interpolate=true]{pgf-apple}}
\end{codeexample}
\end{command}

%%% Local Variables: 
%%% mode: latex
%%% TeX-master: "pgfmanual"
%%% End: 

% Copyright 2007 by Till Tantau
%
% This file may be distributed and/or modified
%
% 1. under the LaTeX Project Public License and/or
% 2. under the GNU Free Documentation License.
%
% See the file doc/generic/pgf/licenses/LICENSE for more details.


\section{Externalizing Graphics}
\label{section-external}


\subsection{Overview}

There are two fundamentally different ways of inserting graphics into
a \TeX-document. First, you can create a graphic using some external
program like |xfig| or |InDesign| and then include this graphic in
your text. This is done using commands like |\includegraphics| or
|\pgfimage|. In this case, the graphic file contains all the low-level
graphic commands that describe the picture. When such a file is
included, all \TeX\ has to worry about is the size of the picture; the
internals of the picture are unknown to \TeX\ and it does not care
about them.

The second method of creating graphics is to use a special package
that transforms \TeX-commands like |\draw| or |\psline| into
appropriate low-level graphic commands. In this case, \TeX\ has to do
all the hard work of ``typesetting'' the picture and if a picture has
a complicated internal structure this may take a lot of time.

While \pgfname\ was created to facilitate the second method of
creating pictures, there are two main reasons why you may need to
employ the first method of image-inclusion, nevertheless:

\begin{enumerate}
\item Typesetting a picture using \TeX\ can be a very time-consuming
  process. If \TeX\ needs a minute to typeset a picture, you do not
  want to wait this minute when you re\TeX\ your document after having
  changed a single comma.
\item Some users, especially journal editors, may not be able to
  process files that contain \pgfname\ commands -- for the simple
  reason that the systems of many publishing houses do not have
  \pgfname\ installed. 
\end{enumerate}

In both cases, the solution is to ``extract'' or ``externalize''
pictures that would normally be typeset every time a document is \TeX
ed. Once the pictures have been extracted into separate graphics
files, these graphic files can be reinserted into the text using the
first method.

Extracting a graphic from a file is not as easy as it may sound at
first since \TeX\ cannot write parts of its output into different
files and a bit of trickery is needed. The following macros
simplify the workflow: 

\begin{enumerate}
\item You have to tell \pgfname\ which files will be used for which
  pictures. To do so, you enclose each picture that you wish to be
  ``externalized'' in a pair of |\beginpgfgraphicnamed| and
  |\endpgfgraphicnamed| macros.
\item The next step is to generate the extracted graphics. For this
  you run \TeX\ with the |\jobname| set to the graphic file's
  name. This will cause |\pgfname| to behave in a very special way:
  All of your document will simply be thrown away, \emph{except} for
  the single graphic having the same name as the current jobname.
\item After you have run \TeX\ once for each graphic that your wish to
  externalize, you can rerun \TeX\ on your document normally. This
  will have the following effect: Each time a |\beginpgfgraphicnamed|
  is encountered, \pgfname\ checks whether a graphic file of the given
  name exists (if you did step 2, it will). If this graphic file
  exists, it will be input and the text till the corresponding
  |\endpgfgraphicnamed| will be ignored.
\end{enumerate}

In the rest of this section, the above workflow is explained in more
detail.


\subsection{Workflow Step 1: Naming Graphics}

In order to put each graphic in an external file, you first need to
tell \pgfname\ the names of these files.

\begin{command}{\beginpgfgraphicnamed\marg{file name prefix}}
  This command indicates that everything up to
  the next call of |\endpgfgraphicnamed| is part of a graphic that
  should be placed in a file named \meta{file name
    prefix}|.|\meta{suffix}, where the \meta{suffix} depends on your
  backend driver. Typically, \meta{suffix} will be |dvi| or |pdf|.

  Here is a typical example of how this command is used:
\begin{codeexample}[code only]
% In file main.tex:
...
As we see in Figure~\ref{fig1}, the world is flat.
\begin{figure}
  \beginpgfgraphicnamed{graphic-of-flat-world}
  \begin{tikzpicture}
    \fill (0,0) circle (1cm);
  \end{tikzpicture}
  \endpgfgraphicnamed
  \caption{The flat world.}
  \label{fig1}
\end{figure}
\end{codeexample}

  Each graphic that is be externalized should have a unique name. Note
  that this name will be used as the name of a file in the file
  system, so it should not contain any funny characters.

  This command can have three different effects:
  \begin{enumerate}
  \item The easiest situation arises if there does not yet exist a
    graphic file called \meta{file name  prefix}|.|\meta{suffix},
    where the \meta{suffix} is one of the suffixes understood by your
    current backend driver (so |pdf| or |jpg| if you use |pdftex|,
    |eps| if you use |dvips|, and so on). In this case, both this
    command and the |\endpgfgraphicnamed| command simply have no
    effect. 
  \item A more complex situation arises when a graphic file named
    \meta{file name  prefix}|.|\meta{suffix} \emph{does} exist. In
    this case, this graphic file is included using the
    |\includegraphics| command. Furthermore, the text between
    |\beginpgfgraphicnamed| and |\endpgfgraphicnamed| is ignored.

    When the text is ``ignored,'' what actually happens is that all
    text up to the next occurrence  of |\endpgfgraphicnamed| is thrown
    away without any macro expansion. This means, in particular, that
    (a) you cannot put |\endpgfgraphicnamed| inside a macro and (b)
    the macros used in the graphics need not be defined at all when
    the graphic file is included.
  \item The most complex behaviour arises when current the |\jobname|
    equals the \meta{file name prefix} and, furthermore, the
    a \emph{real job name} has been declared. The behaviour for this
    case is explained later.
  \end{enumerate}

  Note that the |\beginpgfgraphicnamed| does not really have any
  effect until you have generated the graphic files named. Till then,
  this command is simply ignored. Also, if you delete the graphics
  file later on, the graphics are typeset normally once more.
\end{command}

\begin{command}{\endpgfgraphicnamed}
  This command just marks the end of the graphic that should be
  externalized.
\end{command}


\subsection{Workflow Step 2: Generating the External Graphics}

We have now indicated all the graphics for which we would like graphic
files to be generated. In order to generate the files, you now need to
modify the |\jobname| appropriately. This is done in two steps:

\begin{enumerate}
\item You use the following command to tell \pgfname\ the real name of
  your |.tex| file:
  \begin{command}{\pgfrealjobname\marg{name}}
    Tells \pgfname\ the real name of your job. For instance, if you
    have a file called |survey.tex| that contains two graphics that
    you wish to be called |survey-graphic1| and |survey-graphic2|,
    then you should write the following.
\begin{codeexample}[code only]
% This is file survey.tex
\documentclass{article}
...
\usepackage{tikz}
\pgfrealjobname{survey}
\end{codeexample}
  \end{command}
\item  You run \TeX\ with the |\jobname| set to the name of
the graphic for which you need an external graphic to be generated.
To set the |\jobname|, you use the |--jobname=| option of \TeX:

\begin{codeexample}[code only]
bash> latex --jobname=survey-graphic1 survey.tex
\end{codeexample}
\end{enumerate}

The following things will now happen:
\begin{enumerate}
\item |\pgfrealjobname| notices that the |\jobname|
  is not the ``real'' jobname and, thus, must be the name of a graphic
  that is to be put in an external file.
\item At the beginning of the document, \pgfname\ changes the
  definition of \TeX's internal |\shipout| macro. The new shipout
  macro simply throws away the output. This means that the document is
  typeset normally, but no output is produced.
\item When the |\beginpgfgraphicnamed{|\meta{name}|}| command is
  encountered where the \meta{name} is the same as the current
  |\jobname|, then a \TeX-box is started and \meta{everything} up to the
  following |\endpgfgraphicnamed| command is stored inside this box.

  Note that, typically, \meta{everything} will contain just a single
  |{tikzpicture}| or |{pgfpicture}| environment. However, this need
  not be the case, you use, say, a |{pspicture}| environment as
  \meta{everything} or even just some normal \TeX-text.  
\item At the |\endpgfgraphicnamed|, the box \emph{is} shipped out
  using the original |\shipout| command. Thus, unlike everything else,
  the contents of the graphic is made part of the output.
\item When the box containing the graphic is shipped out, the paper
  size is modified such that it exactly equal to the height and width
  of the box. 
\end{enumerate}

The net effect of everything described above is that the two
commands
\begin{codeexample}[code only]
bash> latex --jobname=survey-graphic1 survey.tex
bash> dvips survey-graphic1
\end{codeexample}
\noindent produce a file called |survey-graphic1.ps| that consists of a single
page that contains exactly the graphic produced by the code between
|\beginpgfgraphicnamed{survey-graphic1}| and
|\endpgfgraphicnamed|. Furthermore, the size of this single page is
exactly the size of the graphic.

If you use pdf\TeX, producing the graphic is even simpler:
\begin{codeexample}[code only]
bash> pdflatex --jobname=survey-graphic1 survey.tex
\end{codeexample}
\noindent produces the single-page |pdf|-file |survey-graphic1.pdf|.

\subsection{Workflow Step 3: Including the External Graphics}

Once you have produced all the pictures in the text, including them
into the main document is easy: Simply run \TeX\ again without any
modification of the |\jobname|. In this case the
|\pgfrealjobname| command will notice that the main file is, indeed,
the main file. The main file will then be typeset normally and the
|\beginpgfgraphicnamed| commands also behave normally, which means
that they will try to include the generated graphic files -- which is
exactly what you want.

Suppose that you wish to send your survey to a journal that does not
have \pgfname\ installed. In this case, you now have all the necessary
external graphics, but you still need \pgfname\ to automatically
include them instead of the executing the picture code! One way to
solve this problem is to simply delete all of the \pgfname\ or
\tikzname\ code from your |survey.tex| and instead insert appropriate
|\includegraphics| commands ``by hand.'' However, there is a better
way: You input the file |pgfexternal.tex|.

\begin{filedescription}{pgfexternal.tex}
  This file defines the command |\beginpgfgraphicnamed| and causes it
  to have the following effect: It includes the graphic file given as
  a parameter to it and then gobbles everything up to
  |\endpgfgraphicnamed|.

  Since |\beginpgfgraphicnamed| does not do macro expansion as it
  searches for |\endpgfgraphicnamed|, it is not necessary to actually
  include the packages necessary for \emph{creating} the graphics. 
  So the idea is that you comment out things like |\usepackage{tikz}|
  and instead say |\input pgfexternal.tex|.

  Indeed, the contents of this file is simply the following line:
\begin{codeexample}[code only]
\long\def\beginpgfgraphicnamed#1#2\endpgfgraphicnamed{\includegraphics{#1}}
\end{codeexample}

  Instead of |\input pgfexternal.tex| you could also include this line
  in your main file. 
\end{filedescription}

As a final remark, note that the |baseline| option does not work directly
with pictures written to an external graphic file. The simple
reason is that there is no way to store this baseline information in
an external graphic file. To allow the |baseline| option (or any \TeX\ construction
with non-zero depth), the baseline information is stored into a separate file.
This file is named \marg{image file}|.dpth| and contains something like |5pt|.

So, if you need baseline information, you will have to keep the external graphic file
together with its~|.dpth| file. Furthermore, the short command in |\input pgfexternal.tex|
is no longer enough because it ignores any baseline information. You will need to use
|\input pgfexternalwithdepth.tex| instead. It has about a dozen lines but can be used in
the same way as |pgfexternal.tex|.


\subsection{A Complete Example}

Let us now have a look at a simple, but complete example. We start out
with a normal file called |survey.tex| that has the following
contents:
\begin{codeexample}[code only]
% This is the file survey.tex
\documentclass{article}

\usepackage{graphics}
\usepackage{tikz}

\begin{document}
In the following figure, we see a circle:
\begin{tikzpicture}
  \fill (0,0) circle (10pt);
\end{tikzpicture}

By comparison, in this figure we see a rectangle:
\begin{tikzpicture}
  \fill (0,0) rectangle (10pt,10pt);
\end{tikzpicture}
\end{document}
\end{codeexample}

Now our editor tells us that the publisher will need all figures to be
provided in separate PostScript or |.pdf|-files. For this, we 
enclose all figures in |...graphicnamed|-pairs and we add a call to
the |\pgfrealjobname| macro:
\begin{codeexample}[code only]
% This is the file survey.tex
\documentclass{article}

\usepackage{graphics}
\usepackage{tikz}
\pgfrealjobname{survey}

\begin{document}
In the following figure, we see a circle:
\beginpgfgraphicnamed{survey-f1}
\begin{tikzpicture}
  \fill (0,0) circle (10pt);
\end{tikzpicture}
\endpgfgraphicnamed

By comparison, in this figure we see a rectangle:
\beginpgfgraphicnamed{survey-f2}
\begin{tikzpicture}
  \fill (0,0) rectangle (10pt,10pt);
\end{tikzpicture}
\endpgfgraphicnamed
\end{document}
\end{codeexample}

After these changes, typesetting the file will still yield the same
output as it did before -- after all, we have not yet created any
external graphics.

To create the external graphics, we run |pdflatex| twice, once for
each graphic:
\begin{codeexample}[code only]
bash> pdflatex --jobname=survey-f1 survey.tex
This is pdfTeX, Version 3.141592-1.40.3 (Web2C 7.5.6)
entering extended mode
(./survey.tex
LaTeX2e <2005/12/01>
...
) [1] (./survey-f1.aux) )
Output written on survey-f1.pdf (1 page, 1016 bytes).
Transcript written on survey-f1.log.
\end{codeexample}

\begin{codeexample}[code only]
bash> pdflatex --jobname=survey-f2 survey.tex
This is pdfTeX, Version 3.141592-1.40.3 (Web2C 7.5.6)
entering extended mode
(./survey.tex
LaTeX2e <2005/12/01>
...
(./survey-f2.aux) )
Output written on survey-f2.pdf (1 page, 1002 bytes).
Transcript written on survey-f2.log.
\end{codeexample}

We can now send the two generated graphics (|survey-f1.pdf| and
|survey-f2.pdf|) to the editor. However, the publisher cannot use our
|survey.tex| file, yet. The reason is that it contains the command
|\usepackage{tikz}| and they do not have \pgfname\ installed.

Thus, we modify the main file |survey.tex| as follows:
\begin{codeexample}[code only]
% This is the file survey.tex
\documentclass{article}

\usepackage{graphics}
\input pgfexternal.tex
% \usepackage{tikz}
% \pgfrealjobname{survey}

\begin{document}
In the following figure, we see a circle:
\beginpgfgraphicnamed{survey-f1}
\begin{tikzpicture}
  \fill (0,0) circle (10pt);
\end{tikzpicture}
\endpgfgraphicnamed

By comparison, in this figure we see a rectangle:
\beginpgfgraphicnamed{survey-f2}
\begin{tikzpicture}
  \fill (0,0) rectangle (10pt,10pt);
\end{tikzpicture}
\endpgfgraphicnamed
\end{document}
\end{codeexample}
If we now run pdf\LaTeX, then, indeed, \pgfname\ is no longer needed:
\begin{codeexample}[code only]
bash> pdflatex survey.tex
This is pdfTeX, Version 3.141592-1.40.3 (Web2C 7.5.6)
entering extended mode
(./survey.tex
LaTeX2e <2005/12/01>
Babel <v3.8h> and hyphenation patterns for english, ..., loaded.
(/usr/local/gwTeX/texmf.texlive/tex/latex/base/article.cls
Document Class: article 2005/09/16 v1.4f Standard LaTeX document class
(/usr/local/gwTeX/texmf.texlive/tex/latex/base/size10.clo))
(/usr/local/gwTeX/texmf.texlive/tex/latex/graphics/graphics.sty
(/usr/local/gwTeX/texmf.texlive/tex/latex/graphics/trig.sty)
(/usr/local/gwTeX/texmf.texlive/tex/latex/config/graphics.cfg)
(/usr/local/gwTeX/texmf.texlive/tex/latex/pdftex-def/pdftex.def))
(/Users/tantau/Library/texmf/tex/generic/pgf/generic/pgf/utilities/pgfexternal.
tex) (./survey.aux)
(/usr/local/gwTeX/texmf.texlive/tex/context/base/supp-pdf.tex
[Loading MPS to PDF converter (version 2006.09.02).]
) <survey-f1.pdf, id=1, 23.33318pt x 19.99973pt> <use survey-f1.pdf>
<survey-f2.pdf, id=2, 13.33382pt x 10.00037pt> <use survey-f2.pdf> [1{/Users/ta
ntau/Library/texmf/fonts/map/pdftex/updmap/pdftex.map} <./survey-f1.pdf> <./sur
vey-f2.pdf>] (./survey.aux) )</usr/local/gwTeX/texmf.texlive/fonts/type1/bluesk
y/cm/cmr10.pfb>
Output written on survey.pdf (1 page, 10006 bytes).
Transcript written on survey.log.
\end{codeexample}

To our editor, we send the following files:
\begin{itemize}
\item The last |survey.tex| shown above.
\item The graphic file |survey-f1.pdf|.
\item The graphic file |survey-f2.pdf|.
\item The file |pgfexternal.tex|, whose contents is simply
\begin{codeexample}[code only]
\long\def\beginpgfgraphicnamed#1#2\endpgfgraphicnamed{\includegraphics{#1}}
\end{codeexample}
  (Alternatively, we can also directly add this line to our
  |survey.tex| file).
\end{itemize}

%%% Local Variables: 
%%% mode: latex
%%% TeX-master: "pgfmanual"
%%% End: 

% Copyright 2006 by Till Tantau
%
% This file may be distributed and/or modified
%
% 1. under the LaTeX Project Public License and/or
% 2. under the GNU Free Documentation License.
%
% See the file doc/generic/pgf/licenses/LICENSE for more details.


\section{Creating Plots}

\label{section-plots}

This section describes the |pgfbaseplot| package.

\begin{package}{pgfbaseplot}
  This package provides a set of commands that are intended to make it
  reasonably easy to plot functions using \pgfname. It is loaded
  automatically by |pgf|, but you can load it manually if you have
  only included |pgfcore|.  
\end{package}

\subsection{Overview}

There are different reasons for using \pgfname\ for creating plots
rather than some more powerful program such as \textsc{gnuplot} or
\textsc{mathematica}, as discussed in
Section~\ref{section-why-pgname-for-plots}. So, let us assume that --
for whatever reason -- you wish to use \pgfname\ for generating a plot.

\pgfname\ (conceptually) uses a two-stage process for generating
plots. First, a \emph{plot stream} must be produced. This stream
consists (more or less) of a large number of coordinates. Second a 
\emph{plot handler} is applied to the stream. A plot handler ``does
something'' with the stream. The standard handler will issue
line-to operations to the coordinates in the stream. However, a
handler might also try to issue appropriate curve-to operations in
order to smooth the curve. A handler may even do something else
entirely, like writing each coordinate to another stream, thereby
duplicating the original stream.

Both for the creation of streams and the handling of streams different
sets of commands exist. The commands for creating streams start with
|\pgfplotstream|, the commands for setting the handler start with
|\pgfplothandler|.



\subsection{Generating Plot Streams}

\subsubsection{Basic Building Blocks of Plot Streams}
A \emph{plot stream} is a (long) sequence of the following three
commands:
\begin{enumerate}
\item
  |\pgfplotstreamstart|,
\item
  |\pgfplotstreampoint|, and
\item
  |\pgfplotstreamend|.
\end{enumerate}
Between calls of these commands arbitrary other code may be
called. Obviously, the stream should start with the first command and
end with the last command. Here is an example of a plot stream:
\begin{codeexample}[code only]
\pgfplotstreamstart
\pgfplotstreampoint{\pgfpoint{1cm}{1cm}}
\newdimen\mydim
\mydim=2cm
\pgfplotstreampoint{\pgfpoint{\mydim}{2cm}}
\advance \mydim by 3cm
\pgfplotstreampoint{\pgfpoint{\mydim}{2cm}}
\pgfplotstreamend
\end{codeexample}

\begin{command}{\pgfplotstreamstart}
  This command signals that a plot stream starts. The effect of this
  command is to call the internal command |\pgf@plotstreamstart|,
  which is set by the current plot handler to do whatever needs to be
  done at the beginning of the plot.
\end{command}

\begin{command}{\pgfplotstreampoint\marg{point}}
  This command adds a \meta{point} to the current plot stream. The
  effect of this command is to call the internal command |\pgf@plotstreampoint|,
  which is also set by the current plot handler. This command should
  now ``handle'' the point in some sensible way. For example, a
  line-to command might be issued for the point.
\end{command}

\begin{command}{\pgfplotstreamend}
  This command signals that a plot stream ends. It calls
  |\pgf@plotstreamend|, which should now do any necessary ``cleanup.''
\end{command}

Note that plot streams are not buffered, that is, the different points
are handled immediately. However, using the recording handler, it is
possible to record a stream.

\subsubsection{Commands That Generate Plot Streams}

Plot streams can be created ``by hand'' as in the earlier
example. However, most of the time the coordinates will be produced
internally by some command. For example, the |\pgfplotxyfile| reads a
file and converts it into a plot stream.

\begin{command}{\pgfplotxyfile\marg{filename}}
  This command will try to open the file \meta{filename}. If this
  succeeds, it will convert the file contents into a plot stream as
  follows: A |\pgfplotstreamstart| is issued. Then, each nonempty line
  of the file should start with two numbers separated by a space, such
  as |0.1 1| or |100 -.3|. Anything following the numbers is ignored.

  Each pair \meta{x} and \meta{y} of numbers is converted into one
  plot stream point in the xy-coordinate system. Thus, a line like
\begin{codeexample}[code only]
2 -5 some text
\end{codeexample}
  is turned into 
\begin{codeexample}[code only]
\pgfplotstreampoint{\pgfpointxy{2}{-5}}
\end{codeexample}

  The two characters |%| and |#| are also allowed in a file and they
  are both treated as comment characters. Thus, a line starting with
  either of them is empty and, hence, ignored.

  When the file has been read completely, |\pgfplotstreamend| is
  called. 
\end{command}


\begin{command}{\pgfplotxyzfile\marg{filename}}
  This command works like |\pgfplotxyfile|, only \emph{three} numbers
  are expected on each non-empty line. They are converted into points
  in the xyz-coordinate system. Consider, the following file:
\begin{codeexample}[code only]
% Some comments
# more comments
2 -5  1 first entry
2 -.2 2 second entry
2 -5  2 third entry
\end{codeexample}
  It is turned into the following stream:
\begin{codeexample}[code only]
\pgfplotstreamstart
\pgfplotstreampoint{\pgfpointxyz{2}{-5}{1}}
\pgfplotstreampoint{\pgfpointxyz{2}{-.2}{2}}
\pgfplotstreampoint{\pgfpointxyz{2}{-5}{2}}
\pgfplotstreamend
\end{codeexample}
\end{command}


Currently, there is no command that can decide automatically whether
the xy-coordinate system should be used or whether the xyz-system
should be used. However, it would not be terribly difficult to write a
``smart file reader'' that parses coordinate files a bit more
intelligently. 


\begin{command}{\pgfplotfunction\marg{variable}\marg{sample list}\marg{point}} 
  This command will produce coordinates by iterating the
  \meta{variable} over all values in \meta{sample list}, which should
  be a list in the |\foreach| syntax. For each value of
  \meta{variable}, the \meta{point} is evaluated and the resulting
  coordinate is inserted into the plot stream.

\begin{codeexample}[]
\begin{tikzpicture}[x=3.8cm/360]
  \pgfplothandlerlineto
  \pgfplotfunction{\x}{0,5,...,360}{\pgfpointxy{\x}{sin(\x)+sin(3*\x)}}
  \pgfusepath{stroke}  
\end{tikzpicture}
\end{codeexample}

\begin{codeexample}[]
\begin{tikzpicture}[y=3cm/360]
  \pgfplothandlerlineto
  \pgfplotfunction{\y}{0,5,...,360}{\pgfpointxyz{sin(2*\y)}{\y}{cos(2*\y)}}
  \pgfusepath{stroke}  
\end{tikzpicture}
\end{codeexample}

  Be warnded that if the expressions that need to evaluated for each
  point are complex, then this command can be very slow.
\end{command}



\begin{command}{\pgfplotgnuplot\oarg{prefix}\marg{function}}
  This command will ``try'' to call the \textsc{gnuplot} program to
  generate the coordinates of the \meta{function}. In detail, the
  following happens:

  This command works with two files: \meta{prefix}|.gnuplot| and
  \meta{prefix}|.table|.  If the optional argument \meta{prefix} is
  not given, it is set to |\jobname|.

  Let us start with the situation where none of these files
  exists. Then \pgfname\ will first generate the file
  \meta{prefix}|.gnuplot|. In this file it writes
\begin{codeexample}[code only]
set terminal table; set output "#1.table"; set format "%.5f"
\end{codeexample}
  where |#1| is replaced by \meta{prefix}. Then, in a second line, it
  writes the text \meta{function}.

  Next, \pgfname\ will try to invoke the program |gnuplot| with the
  argument \meta{prefix}|.gnuplot|. This call may or may not succeed,
  depending on whether the |\write18| mechanism (also known as
  shell escape) is switched on and whether the |gnuplot| program is
  available.

  Assuming that the call succeeded, the next step is to invoke
  |\pgfplotxyfile| on the file \meta{prefix}|.table|; which is exactly
  the file that has just been created by |gnuplot|.
  
\begin{codeexample}[]
\begin{tikzpicture}
  \draw[help lines] (0,-1) grid (4,1);
  \pgfplothandlerlineto
  \pgfplotgnuplot[plots/pgfplotgnuplot-example]{plot [x=0:3.5] x*sin(x)}
  \pgfusepath{stroke}
\end{tikzpicture}
\end{codeexample}

  The more difficult situation arises when the |.gnuplot| file exists,
  which will be the case on the second run of \TeX\ on the \TeX\
  file. In this case \pgfname\ will read this file and check whether
  it contains exactly what \pgfname\ ``would have written'' into
  this file. If this is not the case, the file contents is overwritten
  with what ``should be there'' and, as above, |gnuplot| is invoked to
  generate a new |.table| file. However, if the file contents is ``as
  expected,'' the external |gnuplot| program is \emph{not}
  called. Instead, the \meta{prefix}|.table| file is immediately
  read.

  As explained in Section~\ref{section-tikz-gnuplot}, the net effect
  of the above mechanism is that |gnuplot| is called as little as
  possible and that when you pass along the |.gnuplot| and |.table|
  files with your |.tex| file to someone else, that person can
  \TeX\ the |.tex| file without having |gnuplot| installed and without
  having the |\write18| mechanism switched on.
\end{command}



\subsection{Plot Handlers}

\label{section-plot-handlers}

A \emph{plot handler}  prescribes what ``should be done'' with a
plot stream. You must set the plot handler before the stream starts.
The following commands install the most basic plot handlers; more plot
handlers are defined in the file |pgflibraryplothandlers|, which is
documented in Section~\ref{section-library-plothandlers}.

All plot handlers work by setting redefining the following three
macros: |\pgf@plotstreamstart|, |\pgf@plotstreampoint|, and
|\pgf@plotstreamend|.

\begin{command}{\pgfplothandlerlineto}
  This handler will issue a |\pgfpathlineto| command for each point of
  the plot, \emph{except} possibly for the first. What happens with
  the first point can be specified using the two commands described
  below.

\begin{codeexample}[]
\begin{pgfpicture}
  \pgfpathmoveto{\pgfpointorigin}
  \pgfplothandlerlineto
  \pgfplotstreamstart
  \pgfplotstreampoint{\pgfpoint{1cm}{0cm}}
  \pgfplotstreampoint{\pgfpoint{2cm}{1cm}}
  \pgfplotstreampoint{\pgfpoint{3cm}{2cm}}
  \pgfplotstreampoint{\pgfpoint{1cm}{2cm}}
  \pgfplotstreamend
  \pgfusepath{stroke}
\end{pgfpicture}
\end{codeexample}
\end{command}

\begin{command}{\pgfsetmovetofirstplotpoint}
  Specifies that the line-to plot handler (and also some other plot 
  handlers) should issue a move-to command for the
  first point of the plot instead of a line-to. This will start a new
  part of the current path, which is not always, but often,
  desirable. This is the default.
\end{command}

\begin{command}{\pgfsetlinetofirstplotpoint}
  Specifies that  plot handlers should issue a line-to command for the
  first point of the plot.

\begin{codeexample}[]
\begin{pgfpicture}
  \pgfpathmoveto{\pgfpointorigin}
  \pgfsetlinetofirstplotpoint
  \pgfplothandlerlineto
  \pgfplotstreamstart
  \pgfplotstreampoint{\pgfpoint{1cm}{0cm}}
  \pgfplotstreampoint{\pgfpoint{2cm}{1cm}}
  \pgfplotstreampoint{\pgfpoint{3cm}{2cm}}
  \pgfplotstreampoint{\pgfpoint{1cm}{2cm}}
  \pgfplotstreamend
  \pgfusepath{stroke}
\end{pgfpicture}
\end{codeexample}
\end{command}

\begin{command}{\pgfplothandlerdiscard}
  This handler will simply throw away the stream.
\end{command}

\begin{command}{\pgfplothandlerrecord\marg{macro}}
  When this handler is installed, each time a plot stream command is
  called, this command will be appended to \meta{macros}. Thus, at
  the end of the stream, \meta{macro} will contain all the
  commands that were issued on the stream. You can then install
  another handler and invoke \meta{macro} to ``replay'' the stream
  (possibly many times).
 
\begin{codeexample}[]
\begin{pgfpicture}
  \pgfplothandlerrecord{\mystream}
  \pgfplotstreamstart
  \pgfplotstreampoint{\pgfpoint{1cm}{0cm}}
  \pgfplotstreampoint{\pgfpoint{2cm}{1cm}}
  \pgfplotstreampoint{\pgfpoint{3cm}{1cm}}
  \pgfplotstreampoint{\pgfpoint{1cm}{2cm}}
  \pgfplotstreamend
  \pgfplothandlerlineto
  \mystream
  \pgfplothandlerclosedcurve
  \mystream
  \pgfusepath{stroke}
\end{pgfpicture}
\end{codeexample} 
\end{command}

%%% Local Variables: 
%%% mode: latex
%%% TeX-master: "pgfmanual"
%%% End: 

% Copyright 2006 by Till Tantau
%
% This file may be distributed and/or modified
%
% 1. under the LaTeX Project Public License and/or
% 2. under the GNU Free Documentation License.
%
% See the file doc/generic/pgf/licenses/LICENSE for more details.


\section{Layered Graphics}

\label{section-layers}

\subsection{Overview}

\pgfname\ provides a layering mechanism for composing graphics from
multiple layers. (This mechanism is not to be confused with the
conceptual ``software layers'' the \pgfname\ system is composed of.)
Layers are often used in graphic programs. The idea is that you can
draw on the different layers in any order. So you might start drawing
something on the ``background'' layer, then something on the
``foreground'' layer, then something on the ``middle'' layer, and then
something on the background layer once more, and so on. At the end, no
matter in which ordering you drew on the different layers, the layers
are ``stacked on top of each other'' in a fixed ordering to produce
the final picture. Thus, anything drawn on the middle layer would come
on top of everything of the background layer.

Normally, you do not need to use different layers since you will have
little trouble ``ordering'' your graphic commands in such a way that
layers are superfluous. However, in certain situations you only
``know'' what you should draw behind something else after the
``something else'' has been drawn.

For example, suppose you wish to draw a yellow background behind your
picture. The background should be as large as the bounding box of the
picture, plus a little border. If you know the size of the bounding box
of the picture at its beginning, this is easy to accomplish. However,
in general this is not the case and you need to create a
``background'' layer in addition to the standard ``main'' layer. Then,
at the end of the picture, when the bounding box has been established,
you can add a rectangle of the appropriate size to the picture.



\subsection{Declaring Layers}

In \pgfname\ layers are referenced using names. The standard layer,
which is a bit special in certain ways, is called |main|. If nothing
else is specified, all graphic commands are added to the |main|
layer. You can declare a new layer using the following command:

\begin{command}{\pgfdeclarelayer\marg{name}}
  This command declares a layer named \meta{name} for later
  use. Mainly, this will set up some internal bookkeeping.
\end{command}

The next step toward using a layer is to tell \pgfname\ which layers
will be part of the actual picture and which will be their
ordering. Thus, it is possible to have more layers declared than are
actually used.

\begin{command}{\pgfsetlayers\marg{layer list}}
  This command tells \pgfname\ which layers will be used in
  pictures. They are stacked on top of each other in the order
  given. The layer |main| should always be part of the list. Here is
  an example:
\begin{codeexample}[code only]
\pgfdeclarelayer{background}
\pgfdeclarelayer{foreground}  
\pgfsetlayers{background,main,foreground}
\end{codeexample}

  The command can be given outside of any picture 
  or inside of a |pgfpicture| environment.
\end{command}


\subsection{Using Layers}

Once the layers of your picture have been declared, you can start to
``fill'' them. As said before, all graphics commands are normally
added to the |main| layer. Using the |{pgfonlayer}| environment, you
can tell \pgfname\ that certain commands should, instead, be added to
the given layer.

\begin{environment}{{pgfonlayer}\marg{layer name}}
  The whole \meta{environment contents} is added to the layer with the
  name \meta{layer name}. This environment can be used anywhere inside
  a picture. Thus, even if it is used inside a |{pgfscope}| or a \TeX\
  group, the contents will still be added to the ``whole'' picture.
  Using this environment multiple times inside the same picture will
  cause the \meta{environment contents} to accumulate.

  \emph{Note:} You can \emph{not} add anything to the |main| layer
  using this environment. The only way to add anything to the main
  layer is to give graphic commands outside all |{pgfonlayer}|
  environments. 

\begin{codeexample}[]
\pgfdeclarelayer{background layer}
\pgfdeclarelayer{foreground layer}
\pgfsetlayers{background layer,main,foreground layer}
\begin{tikzpicture}
  % On main layer:
  \fill[blue] (0,0) circle (1cm);
  
  \begin{pgfonlayer}{background layer}
    \fill[yellow] (-1,-1) rectangle (1,1);
  \end{pgfonlayer}
  
  \begin{pgfonlayer}{foreground layer}
    \node[white] {foreground};
  \end{pgfonlayer}
  
  \begin{pgfonlayer}{background layer}
    \fill[black] (-.8,-.8) rectangle (.8,.8);
  \end{pgfonlayer}

  % On main layer again:
  \fill[blue!50] (-.5,-1) rectangle (.5,1);
\end{tikzpicture}
\end{codeexample}
\end{environment}

\begin{plainenvironment}{{pgfonlayer}\marg{layer name}}
  This is the plain \TeX\ version of the environment.
\end{plainenvironment}

\begin{contextenvironment}{{pgfonlayer}\marg{layer name}}
  This is the Con\TeX t version of the environment.
\end{contextenvironment}






%%% Local Variables: 
%%% mode: latex
%%% TeX-master: "pgfmanual"
%%% End: 

% Copyright 2006 by Till Tantau
%
% This file may be distributed and/or modified
%
% 1. under the LaTeX Project Public License and/or
% 2. under the GNU Free Documentation License.
%
% See the file doc/generic/pgf/licenses/LICENSE for more details.


\section{Declaring and Using Shadings}

\label{section-shadings}

\subsection{Overview}

A shading is an area in which the color changes smoothly between different
colors. Similarly to an image, a shading must first be declared before
it can be used. Also similarly to an image, a shading is put into a
\TeX-box. Hence, in order to include a shading in a |{pgfpicture}|,
you have to use |\pgftext| around it.

There are different kinds of shadings: horizontal, vertical, radial,
and functional shadings. However, you can rotate and clip shadings
like any other graphics object, which allows you to create more
complicated shadings. Horizontal shadings could be created by rotating
a vertical shading by 90 degrees, but explicit commands for creating both
horizontal and vertical shadings are included for convenience.

Once you have declared a shading, you can insert it into text using
the command |\pgfuseshading|. This command cannot be used directly in
a |{pgfpicture}|, you have to put a |\pgftext| around it. The second
command for using shadings, |\pgfshadepath|, on the other hand, can
only be used  inside |{pgfpicture}| environments. It will ``fill'' the
current path with the shading.

A horizontal shading is a horizontal bar of a certain height whose
color changes smoothly. You must at least specify the colors at the
left and at the right end of the bar, but you can also add color
specifications for points in between. For example, suppose you
which to create a bar that is red at the left end, green in the
middle, and blue at the end. Suppose you would like the bar to be 4cm
long. This could be specified as follows:
\begin{codeexample}[code only]
rgb(0cm)=(1,0,0); rgb(2cm)=(0,1,0); rgb(4cm)=(0,0,1)
\end{codeexample}
This line means that at 0cm (the left end) of the bar, the color
should be red, which has red-green-blue (rgb) components (1,0,0). At
2cm, the bar should be green, and at 4cm it should be blue.
Instead of |rgb|, you can currently also specify |gray| as
color model, in which case only one value is needed, or |color|,
in which case you must provide the name of a color in parentheses. In
a color specification the individual specifications must 
be separated using a semicolon, which may be followed by a whitespace
(like a space or a newline). Individual specifications must be given
in increasing order. 


\subsection{Declaring Shadings}

\begin{command}{\pgfdeclarehorizontalshading\oarg{color list}\marg{shading
      name}\marg{shading height}\marg{color specification}}
  Declares a horizontal shading named \meta{shading name} of the specified
  \meta{height} with the specified colors. The length of the bar is
  deduced automatically from the maximum dimension in the specification.

\begin{codeexample}[]
\pgfdeclarehorizontalshading{myshadingA}
  {1cm}{rgb(0cm)=(1,0,0); color(2cm)=(green); color(4cm)=(blue)}
\pgfuseshading{myshadingA}
\end{codeexample}

  The effect of the \meta{color list}, which is a
  comma-separated list of colors, is the following: Normally, when
  this list is empty, once a shading has been declared, it becomes
  ``frozen.'' This means that even if you change a color that was used
  in the declaration of the shading later on, the shading will not
  change. By specifying a \meta{color list} you can specify
  that the shading should be recalculated whenever one of the colors
  listed in the list changes (this includes effects like color
  mixins). Thus, when you specify a \meta{color list},
  whenever the shading is used, \pgfname\ first converts the colors in the
  list to \textsc{rgb} triples using the current values of the
  colors and taking any mixins and blends into account. If the
  resulting \textsc{rgb} triples have not yet been   used, a new
  shading is internally created and used. Note that if the 
  option \meta{color list} is used, then no shading is created until
  the first use of |\pgfuseshading|. In particular, the colors
  mentioned in the shading need not be defined when the declaration is
  given.

  When a shading is recalculated because of a change in the
  colors mentioned in \meta{color list}, the complete shading
  is recalculated. Thus even colors not mentioned in the list will be
  used with their current values, not with the values they had upon
  declaration.
  
\begin{codeexample}[]
\pgfdeclarehorizontalshading[mycolor]{myshadingB}
  {1cm}{rgb(0cm)=(1,0,0); color(2cm)=(mycolor)}
\colorlet{mycolor}{green}
\pgfuseshading{myshadingB}
\colorlet{mycolor}{blue}
\pgfuseshading{myshadingB}
\end{codeexample}
\end{command}


\begin{command}{\pgfdeclareverticalshading\oarg{color list}\marg{shading
      name}\marg{shading width}\marg{color specification}}
   Declares a vertical shading named \meta{shading name} of the
   specified \meta{width}. The height of the bar is deduced
   automatically. The effect of \meta{color list} is the same as for
   horizontal shadings.

\begin{codeexample}[]
\pgfdeclareverticalshading{myshadingC}
  {4cm}{rgb(0cm)=(1,0,0); rgb(1.5cm)=(0,1,0); rgb(2cm)=(0,0,1)}
\pgfuseshading{myshadingC}
\end{codeexample}
\end{command}


\begin{command}{\pgfdeclareradialshading\oarg{color list}\marg{shading
      name}\marg{center point}\marg{color specification}}
  Declares an radial shading. A radial shading is a circle whose inner
  color changes as specified by the color specification. Assuming that
  the center of the shading is at the origin, the color of the center
  will be the color specified for 0cm and the color of the border of
  the circle will be the color for the maximum dimension given in
  the \meta{color specified}. This maximum will also be the radius of
  the circle. If the \meta{center point} is not at the 
  origin, the whole shading inside the circle (whose size remains
  exactly the same) will be distorted such that the given center now
  has the color specified for 0cm. The effect of \meta{color list} is
  the same as for horizontal shadings. 

\begin{codeexample}[]  
\pgfdeclareradialshading{sphere}{\pgfpoint{0.5cm}{0.5cm}}%
  {rgb(0cm)=(0.9,0,0);
   rgb(0.7cm)=(0.7,0,0);
   rgb(1cm)=(0.5,0,0);
   rgb(1.05cm)=(1,1,1)}
\pgfuseshading{sphere}
\end{codeexample}
\end{command}


\begin{command}{\pgfdeclarefunctionalshading\oarg{color list}\marg{shading
      name}\marg{lower left corner}\marg{upper right corner}\marg{init code}\marg{type 4 function}}
  \emph{Warning: These shadings are the least portable of all and they
  put the heaviest burden of the renderer. They are slow and,
  possibly, will not print correctly!}

  This command creates a \emph{functional shading}. For such a
  shading, the color of each point is calculated by calling a function
  that gets the coordinates of the point as input and yields the
  color as an output. Note that the function is evaluated by the 
  \emph{renderer}, not by \pgfname\ or \TeX or someone else at
  compile-time. This means that the evaluation of this function has to
  be done \emph{extremely quickly} and the funciton should be
  \emph{very simple}. For this reason, only a very restricted set of
  operations are possible in the function and functions should be
  kept small. Any errors in the function will only be noticed by the
  renderer.

  The syntax for specifying functions is the following: You use a
  simplified form of a subset of the PostScript language. This subset
  will be understood by the PDF-renderer (yes, PDF-renderers do
  have a basic understanding of PostScript) and also by PostScript
  renders. This subset is detailed in Seciton 3.9.4 of the
  PDF-specification (version 1.7). In essence, the specificaiton
  states that these functions may contain ``expressions involving
  integers, real numbers, and boolean values only. There are no
  composite data structures such as strings or arrays, no procedures,
  and no variables or names.'' The allowed operators are (exactly) the
  following: \texttt{abs}, \texttt{add}, \texttt{atan},
  \texttt{ceiling}, \texttt{cos}, \texttt{cvi}, \texttt{cvr},
  \texttt{div}, \texttt{exp}, \texttt{floor}, \texttt{idiv},
  \texttt{ln}, \texttt{log}, \texttt{mod}, \texttt{mul}, \texttt{neg},
  \texttt{round}, \texttt{sin}, \texttt{sqrt}, \texttt{sub},
  \texttt{truncate}, \texttt{and}, \texttt{bitshift}, \texttt{eq},
  \texttt{false}, \texttt{ge}, \texttt{gt}, \texttt{le}, \texttt{lt},
  \texttt{ne}, \texttt{not}, \texttt{or}, \texttt{true}, \texttt{xor},
  \texttt{if}, \texttt{ifelse}, \texttt{copy}, \texttt{dup},
  \texttt{exch}, \texttt{index}, \texttt{pop}.

  When the function is evaluated, the top two stack elements are the
  coordinates of the point for which the color should be computed. The
  coordinates are dimensionless and given in big points, so for the
  coordinate $(50bp, 72.27pt)$ the top two stack elements would be
  \texttt{50.0} and \texttt{72.0}. Ohterwise, the (virtual) stack is
  empty (or should be treated as if it were empty). The function
  should then replace these two values by three values, representing
  the red, green, and blue color of the point. The numbers should be
  real values, not integers since Apple's PDF renderer is broken in
  this regard (use cvr at the end if necessary).

  Conceptually, the function will be evaluated once for each point of
  the rectangle \meta{lower left corner} to \meta{upper right corner},
  which should be a \pgfname-point expression like
  |\pgfpoint{100bp}{100bp}|. A renderer may choose to evaluate the
  function at less points, but, in principle, the function will be
  evaluated for each pixel independently. 

  Because of the rather difficult PostScript syntax, use this macro
  only \emph{if you know what you are doing} (or if you are
  advanterous, of course). 

  As for other shadings, the optional \meta{color list} is used to
  determine whether a shading needs to be recalculated when a color
  has changed.

  The \meta{init code} is executed each time a shading is
  (re)calculated. Typically, it will contain code to extract
  coordinates from colors (see below).

  Inside the PostScript function \meta{type 4 function} you cannot use
  colors directly. Rather, you must push the color components on the
  stack. For this, it is useful to call |\pgfshadecolorrgb| in the
  \meta{init code}. The macro takes a color name as input and stores
  the color's red/green/blue components real numbers between 0.0 and
  1.0 separated by spaces (which is exactly what you need if you want
  to push it on a stack) in a macro. You can then use this macro in
  the argument \meta{type 4 function}.

\begin{codeexample}[]
\pgfdeclarefunctionalshading{twospots}
    {\pgfpointorigin}{\pgfpoint{4cm}{4cm}}{}{
  % Save coordinates for later
  2 copy
  % Compute distance from (40bp,45bp), with x doubled
  45 sub dup mul exch
  40 sub dup mul 0.5 mul add sqrt
  % expontial decay
  dup mul neg 1.0005 exch exp 1.0 exch sub
  % Compute distance form (70bp,70bp) from stored coordiante, scaled
  3 1 roll
  70 sub dup mul .5 mul exch
  70 sub dup mul add sqrt
  % Decay
  dup mul neg 1.002 exch exp 1.0 exch sub
  % red component
  1.0 3 1 roll
}
\pgfuseshading{twospots}
\end{codeexample}

\begin{codeexample}[]
\pgfdeclarefunctionalshading[mycol]{sweep}{\pgfpoint{-1cm}{-1cm}}
{\pgfpoint{1cm}{1cm}}{\pgfshadecolortorgb{mycol}{\myrgb}}{
  2 copy        % whirl
  atan
  3 1 roll
  dup mul exch
  dup mul add sqrt
  30 mul
  1 index add
  sin
  1 add 2 div  
  dup
  \myrgb        % push mycol
  5 4 roll      % multiply all components by calculated value
  mul
  3 1 roll
  3 index
  mul
  3 1 roll
  4 3 roll
  mul
  3 1 roll
}
\colorlet{mycol}{white}%
\pgfuseshading{sweep}%
\colorlet{mycol}{red}%
\pgfuseshading{sweep}
\end{codeexample}
\end{command}


\subsection{Using Shadings}
\label{section-shading-a-path}

\begin{command}{\pgfuseshading\marg{shading name}}
  Inserts a previously declared shading into the text. If you wish to
  use it in a |pgfpicture| environment, you should put a |\pgfbox|
  around it.
  
\begin{codeexample}[]
\begin{pgfpicture}
  \pgfdeclareverticalshading{myshadingD}
    {20pt}{color(0pt)=(red); color(20pt)=(blue)}
  \pgftext[at=\pgfpoint{1cm}{0cm}]  {\pgfuseshading{myshadingD}}
  \pgftext[at=\pgfpoint{2cm}{0.5cm}]{\pgfuseshading{myshadingD}}
\end{pgfpicture}
\end{codeexample}
\end{command}

\begin{command}{\pgfshadepath\marg{shading name}\marg{angle}}
  This command must be used inside a |{pgfpicture}| environment. The
  effect is a bit complex, so let us go over it step by step.

  First, \pgfname\ will setup a local scope.

  Second, it uses the current path to clip everything inside this
  scope. However, the current path is once more available after the
  scope, so it can be used, for example, to stroke it.

  Now, the \meta{shading name} should be a shading whose width and
  height are 100\,bp, that is, 100 big points. \pgfname\ has a look at
  the bounding box of the current path. This bounding box is computed
  automatically when a path is computed; however, it can sometimes be
  (quite a bit) too large, especially when complicated curves are
  involved. 

  Inside the scope, the low-level transformation matrix is modified.
  The center of the shading is translated (moved) such that it lies on
  the center of the bounding box of the path. The low-level coordinate
  system is also scaled such that the shading ``covers'' the shading (the 
  details are a bit more complex, see below). Then, the coordinate
  system is rotated by \meta{angle}. Finally, if the macro 
  |\pgfsetadditionalshadetransform| has been used, an additional
  transformation is applied. 

  After everything has been set up, the shading is inserted. Due to
  the transformations and clippings, the effect will be that  the
  shading seems to ``fill'' the path.

  If both the path and the shadings were always rectangles and if
  rotation were never involved, it would be easy to scale shadings
  such they always cover the path. However, when a vertical shading is
  rotated, it must obviously be ``magnified'' so that it
  still covers the path. Things get worse when the path is not a
  rectangle itself.

  For these reasons, things work slightly differently ``in reality.''
  The shading is scaled and translated such that the
  the point $(50\mathrm{bp},50\mathrm{bp})$, which is the middle of
  the shading, is at the middle of the path and such that the the
  point $(25\mathrm{bp},25\mathrm{bp})$ is at the lower left corner of
  the path and that  $(75\mathrm{bp},75\mathrm{bp})$  is at upper
  right corner.

  In other words, only the center quarter of the shading will actually
  ``survive the clipping'' if the path is a rectangle. If the path is
  not a rectangle, but, say, a circle, even less is seen of the
  shading. Here is an example that demonstrates this effect:

\begin{codeexample}[]
\pgfdeclareverticalshading{myshadingE}{100bp}    
 {color(0bp)=(red); color(25bp)=(green);  color(75bp)=(blue);  color(100bp)=(black)}
\pgfuseshading{myshadingE} 
\hskip 1cm
\begin{pgfpicture}
  \pgfpathrectangle{\pgfpointorigin}{\pgfpoint{2cm}{1cm}}
  \pgfshadepath{myshadingE}{0}
  \pgfusepath{stroke}
  \pgfpathrectangle{\pgfpoint{3cm}{0cm}}{\pgfpoint{1cm}{2cm}}
  \pgfshadepath{myshadingE}{0}
  \pgfusepath{stroke}
  \pgfpathrectangle{\pgfpoint{5cm}{0cm}}{\pgfpoint{2cm}{2cm}}
  \pgfshadepath{myshadingE}{45}
  \pgfusepath{stroke}
  \pgfpathcircle{\pgfpoint{9cm}{1cm}}{1cm}
  \pgfshadepath{myshadingE}{45}
  \pgfusepath{stroke}
\end{pgfpicture}
\end{codeexample}

  As can be seen above in the last case, the ``hidden'' part of the
  shading actually \emph{can} become visible if the shading is
  rotated. The reason is that it is scaled as if no rotation took
  place, then the rotation is done.

  The following graphics show which part of the shading are actually
  shown: 

\begin{codeexample}[]
\pgfdeclareverticalshading{myshadingF}{100bp}    
 {color(0bp)=(red); color(25bp)=(green);  color(75bp)=(blue);  color(100bp)=(black)}
\begin{tikzpicture}
  \draw (50bp,50bp) node {\pgfuseshading{myshadingF}};
  \draw[white,thick] (25bp,25bp) rectangle (75bp,75bp);
  \draw (50bp,0bp) node[below] {first two applications};

  \begin{scope}[xshift=5cm]
    \draw (50bp,50bp) node{\pgfuseshading{myshadingF}};
    \draw[rotate around={45:(50bp,50bp)},white,thick] (25bp,25bp) rectangle (75bp,75bp);
    \draw (50bp,0bp) node[below] {third application};
  \end{scope}

  \begin{scope}[xshift=10cm]
    \draw (50bp,50bp) node{\pgfuseshading{myshadingF}};
    \draw[white,thick] (50bp,50bp) circle (25bp);
    \draw (50bp,0bp) node[below] {fourth application};
  \end{scope}
\end{tikzpicture}
\end{codeexample}
  
  An advantage of this approach is that when you rotate a radial
  shading, no distortion is introduced:

\begin{codeexample}[]
\pgfdeclareradialshading{ballshading}{\pgfpoint{-10bp}{10bp}}
 {color(0bp)=(red!15!white); color(9bp)=(red!75!white);
 color(18bp)=(red!70!black); color(25bp)=(red!50!black); color(50bp)=(black)}
\pgfuseshading{ballshading}
\hskip 1cm
\begin{pgfpicture}
  \pgfpathrectangle{\pgfpointorigin}{\pgfpoint{1cm}{1cm}}
  \pgfshadepath{ballshading}{0}
  \pgfusepath{}
  \pgfpathcircle{\pgfpoint{3cm}{0cm}}{1cm}
  \pgfshadepath{ballshading}{0}
  \pgfusepath{}
  \pgfpathcircle{\pgfpoint{6cm}{0cm}}{1cm}
  \pgfshadepath{ballshading}{45}
  \pgfusepath{}
\end{pgfpicture}
\end{codeexample}

  If you specify a rotation of $90^\circ$
  and if the path is not a square, but an elongated rectangle,  the
  ``desired'' effect results: The shading will exactly vary between
  the colors at the 25bp and 75bp boundaries. Here is an example:
  
\begin{codeexample}[]
\pgfdeclareverticalshading{myshadingG}{100bp}    
 {color(0bp)=(red); color(25bp)=(green);  color(75bp)=(blue);  color(100bp)=(black)}
\begin{pgfpicture}
  \pgfpathrectangle{\pgfpointorigin}{\pgfpoint{2cm}{1cm}}
  \pgfshadepath{myshadingG}{0}
  \pgfusepath{stroke}
  \pgfpathrectangle{\pgfpoint{3cm}{0cm}}{\pgfpoint{2cm}{1cm}}
  \pgfshadepath{myshadingG}{90}
  \pgfusepath{stroke}
  \pgfpathrectangle{\pgfpoint{6cm}{0cm}}{\pgfpoint{2cm}{1cm}}
  \pgfshadepath{myshadingG}{45}
  \pgfusepath{stroke}
\end{pgfpicture}
\end{codeexample}


  As a final example, let us define a ``rainbow spectrum'' shading for
  use with \tikzname.
\begin{codeexample}[]
\pgfdeclareverticalshading{rainbow}{100bp}
 {color(0bp)=(red); color(25bp)=(red); color(35bp)=(yellow);
  color(45bp)=(green); color(55bp)=(cyan); color(65bp)=(blue);
  color(75bp)=(violet); color(100bp)=(violet)}
\begin{tikzpicture}[shading=rainbow]
  \shade (0,0) rectangle node[white] {\textsc{pride}} (2,1);
  \shade[shading angle=90] (3,0) rectangle +(1,2);
\end{tikzpicture}
\end{codeexample}

  Note that rainbow shadings are \emph{way} to colorful in almost all
  applications. 
\end{command}

\begin{command}{\pgfsetadditionalshadetransform\marg{transformation}}
    This command allows you to specify an additional transformation
    that should be applied to shadings when the |\pgfshadepath|
    command is used. The \meta{transformation} should be
    transformation code like |\pgftransformrotate{20}|.  
\end{command}

%%% Local Variables: 
%%% mode: latex
%%% TeX-master: "pgfmanual"
%%% End: 

% Copyright 2006 by Till Tantau
%
% This file may be distributed and/or modified
%
% 1. under the LaTeX Project Public License and/or
% 2. under the GNU Free Documentation License.
%
% See the file doc/generic/pgf/licenses/LICENSE for more details.


\section{Transparency}

\label{section-transparency}


For an introduction to the notion of transparency, fadings, and
transparency groups, please consult Section~\ref{section-tikz-transparency}.


\subsection{Specifying a Uniform Opacity}

Specifying a stroke and/or fill opacity is quite easy.

\begin{command}{\pgfsetstrokeopacity\marg{value}}
  Sets the opacity of stroking operations. The \meta{value} should be
  a number between |0| and |1|, where |1| means ``fully opaque'' and
  |0| means ``fully transparent.'' A value like |0.5| will cause paths
  to be stroked in a semitransparent way.

\begin{codeexample}[]
\begin{pgfpicture}
  \pgfsetlinewidth{5mm}
  \color{red}
  \pgfpathcircle{\pgfpoint{0cm}{0cm}}{10mm} \pgfusepath{stroke}
  \color{black}
  \pgfsetstrokeopacity{0.5}
  \pgfpathcircle{\pgfpoint{1cm}{0cm}}{10mm} \pgfusepath{stroke}
\end{pgfpicture}
\end{codeexample}
\end{command}


\begin{command}{\pgfsetfillopacity\marg{value}}
  Sets the opacity of filling operations. As for stroking, the
  \meta{value} should be a number between |0| and~|1|.

  The ``filling transparency'' will also be used for text and images.

\begin{codeexample}[]
\begin{tikzpicture}
  \pgfsetfillopacity{0.5}
  \fill[red]   (90:1cm)  circle (11mm);
  \fill[green] (210:1cm) circle (11mm);
  \fill[blue]  (-30:1cm) circle (11mm);
\end{tikzpicture}
\end{codeexample}
\end{command}

Note the following effect: If you setup a certain opacity for stroking
or filling and you stroke or fill the same area twice, the effect
accumulates:

\begin{codeexample}[]
\begin{tikzpicture}
  \pgfsetfillopacity{0.5}
  \fill[red] (0,0) circle (1);
  \fill[red] (1,0) circle (1);
\end{tikzpicture}
\end{codeexample}

Often, this is exactly what you intend, but not always. You can use
transparency groups, see the end of this section, to change this.


\subsection{Specifying a Fading}

The method used by \pgfname\ for specifying fadings is quite
general: You ``paint'' the fading using any of the standard graphics
commands. In more detail: You create a normal picture, which may even
contain text, image, and shadings. Then, you create a fading based on
this picture. For this, the \emph{luminosity} of each pixel of the
picture is analyzed (the brighter the pixel, the higher the luminosity
-- a black pixel has luminosity $0$, a white pixel has luminosity $1$,
a gray pixel has some intermediate value as does a red pixel). Then,
when the fading is used, the luminosity of the pixel determines the
opacity of the fading at that position. Positions in the fading where
the picture was black will be completely transparent, positions where
the picture was white will be completely opaque. Positions that have
not been painted at all in the picture are always completely
transparent.


\begin{command}{\pgfdeclarefading\marg{name}\marg{contents}}
  This command declare a fading named \meta{name} for later use. The
  ``picture'' on which the fading is based is given by the
  \meta{contents}. This \meta{contents} is normally typeset in a \TeX\
  box. The resulting box is then used as the ``picture.'' In
  particular, inside the \meta{contents} you must explicitly open a
  |{pgfpicture}| environment if you wish to use \pgfname\ commands.

  Let's start with an easy example. Our first fading picture is just
  some text:
\begin{codeexample}[]
\pgfdeclarefading{fading1}{\color{white}Ti\emph{k}Z}
\begin{tikzpicture}
  \fill [black!20] (0,0) rectangle (2,2);
  \fill [black!30] (0,0) arc (180:0:1);
  \pgfsetfading{fading1}{\pgftransformshift{\pgfpoint{1cm}{1cm}}}
  \fill [red] (0,0) rectangle (2,2);
\end{tikzpicture}
\end{codeexample}
  What's happening here? The ``fading picture'' is mostly transparent,
  except for the pixels that are part of the word Ti\emph{k}Z. Now,
  these pixels are \emph{white} and, thus, have a high
  luminosity. This in turn means that these pixels of the fading will
  be highly opaque. For this reason, only those pixels of the big red
  rectangle ``shine through'' that are at the positions of these
  opaque pixels.

  It is somewhat counter-intuitive that the white pixels in a fading
  picture are opaque in a fading. For this reason, the color
  |pgftransparent| is defined to be the same as |black|. This allows
  one to write |pgftransparent| for completely transparent parts of a
  fading picture and |pgftransparent!0| for the opaque parts and
  things like |pgftransparent!20| for parts that are 20\%
  transparent.

  Furthermore, the color |pgftransparent!0| (which is the same as
  white and which corresponds to completely opaque) is installed at
  the beginning of a fading picture. Thus, in the above example the
  |\color{white}| was not really necessary.

  Next, let us create a fading that gets more and more transparent as
  we go from left to right. For this, we put a shading inside the
  fading picture that has the color |pgftransparent!0| at the
  left-hand side and the color |pgftransparent!100| at the right-hand
  side.
\begin{codeexample}[]
\pgfdeclarefading{fading2}
{\tikz \shade[left color=pgftransparent!0,
              right color=pgftransparent!100] (0,0) rectangle (2,2);}
\begin{tikzpicture}
  \fill [black!20] (0,0) rectangle (2,2);
  \fill [black!30] (0,0) arc (180:0:1);
  \pgfsetfading{fading2}{\pgftransformshift{\pgfpoint{1cm}{1cm}}}
  \fill [red] (0,0) rectangle (2,2);
\end{tikzpicture}
\end{codeexample}

  In our final example, we create a fading that is based on a radial
  shading.
\begin{codeexample}[]
\pgfdeclareradialshading{myshading}{\pgfpointorigin}
{
  color(0mm)=(pgftransparent!0);
  color(5mm)=(pgftransparent!0);
  color(8mm)=(pgftransparent!100);
  color(15mm)=(pgftransparent!100)
}
\pgfdeclarefading{fading3}{\pgfuseshading{myshading}}
\begin{tikzpicture}
  \fill [black!20] (0,0) rectangle (2,2);
  \fill [black!30] (0,0) arc (180:0:1);
  \pgfsetfading{fading3}{\pgftransformshift{\pgfpoint{1cm}{1cm}}}
  \fill [red] (0,0) rectangle (2,2);
\end{tikzpicture}
\end{codeexample}
\end{command}

After having declared a fading, we can use it. As for shadings, there
are two different commands for using fadings:

\begin{command}{\pgfsetfading\marg{name}\marg{transformations}}
  This command sets the graphic state parameter ``fading'' to a
  previously defined fading \meta{name}. This graphic state works like
  other graphic states, that is, is persists till the end of the
  current scope or until a different transparency setting is chosen.

  When the fading is installed, it will be centered on the origin with
  its natural size. Anything outside the fading pictures's original
  bounding box will be transparent and, thus, the fading effectively
  clips against this bounding box.

  The \meta{transformations} are applied to the fading before it is
  used. They contain normal \pgfname\ transformation commands like
  |\pgftransformshift|. You can also scale the fading using this
  command. Note, however, that the transformation needs to be inverted
  internally, which may result in inaccuracies and the following
  graphics may be slightly distorted if you use a strong
  \meta{transformation}.
\begin{codeexample}[]
\pgfdeclarefading{fading2}
{\tikz \shade[left color=pgftransparent!0,
              right color=pgftransparent!100] (0,0) rectangle (2,2);}
\begin{tikzpicture}
  \fill [black!20] (0,0) rectangle (2,2);
  \fill [black!30] (0,0) arc (180:0:1);
  \pgfsetfading{fading2}{}
  \fill [red] (0,0) rectangle (2,2);
\end{tikzpicture}
\end{codeexample}
\begin{codeexample}[]
\begin{tikzpicture}
  \fill [black!20] (0,0) rectangle (2,2);
  \fill [black!30] (0,0) arc (180:0:1);
  \pgfsetfading{fading2}{\pgftransformshift{\pgfpoint{1cm}{1cm}}
                         \pgftransformrotate{20}}
  \fill [red] (0,0) rectangle (2,2);
\end{tikzpicture}
\end{codeexample}
\end{command}

\begin{command}{\pgfsetfadingforcurrentpath\marg{name}\marg{transformations}}
  This command works like |\pgfsetfading|, but the fading is scaled
  are transformed according to the following rules:
  \begin{enumerate}
  \item
    If the current path is empty, the command has the same effect as
    |\pgfsetfading|.
  \item
    Otherwise it is assumed that the fading has a size of 100bp times
    100bp.
  \item
    The fading is resized and shifted (using appropriate
    transformations) such that the position
    $(25\mathrm{bp},25\mathrm{bp})$ lies at the lower-left corner of
    the current path and the position $(75\mathrm{bp},75\mathrm{bp})$
    lies at the upper-right corner of the current path.
  \end{enumerate}
  Note that these rules are the same as the ones used in
  |\pgfshadepath| for shadings. After these transformations, the
  \meta{transformations} are executed (typically a rotation).
\begin{codeexample}[]
\pgfdeclarehorizontalshading{shading}{100bp}
{ color(0pt)=(transparent!0);    color(25bp)=(transparent!0);
  color(75bp)=(transparent!100); color(100bp)=(transparent!100)}

\pgfdeclarefading{fading}{\pgfuseshading{shading}}

\begin{tikzpicture}
  \fill [black!20] (0,0) rectangle (2,2);
  \fill [black!30] (0,0) arc (180:0:1);

  \pgfpathrectangle{\pgfpointorigin}{\pgfpoint{2cm}{1cm}}
  \pgfsetfadingforcurrentpath{fading}{}
  \pgfusepath{discard}

  \fill [red] (0,0) rectangle (2,1);

  \pgfpathrectangle{\pgfpoint{0cm}{1cm}}{\pgfpoint{2cm}{1cm}}
  \pgfsetfadingforcurrentpath{fading}{\pgftransformrotate{90}}
  \pgfusepath{discard}

  \fill [red] (0,1) rectangle (2,2);
\end{tikzpicture}
\end{codeexample}

\end{command}

\subsection{Transparency Groups}

Transparency groups are declared using the following commands.

\begin{environment}{{pgftransparencygroup}}
  This environment should only be used inside a |{pgfpicture}|. It has
  the following effect:
  \begin{enumerate}
  \item The \meta{environment contents} is stroked/filled
    ``ignoring any outside transparency.'' This means, all previous
    transparency settings are ignored (you can still set transparency
    inside the group, but never mind). This means that if in the
    \meta{environment contents} you stroke a pixel three times in
    black, it is just black. Stroking it white afterwards yields a
    white pixel, and so on.
  \item When the group is finished, it is painted as a whole. The
    \emph{fill} transparency settings are now applied to the resulting
    picture. For instance, the pixel that has been painted three times
    in black and once in white is just white at the end, so this white
    color will be blended with whatever is ``behind'' the group on the
    page.
  \end{enumerate}

  Note that, depending on the driver, \pgfname\ may have to guess the
  size of the contents of the transparency group (because such a group
  is put in an XForm in \textsc{pdf} and a bounding box must be
  supplied). \pgfname\ will use normally use the size of the picture's
  bounding box at the end of the transparency group plus a safety
  margin of 1cm. Under normal circumstances, this will work nicely
  since the picture's bounding box contains everything
  anyway. However, if you have switched off the picture size tracking
  or if you are using canvas transformations, you may have to make
  sure that the bounding box is big enough. The trick is to locally
  create a picture that is ``large enough'' and then insert this
  picture into the main picture while ignoring the size. The following
  example shows how this is done:


{\tikzexternaldisable
\begin{codeexample}[]
\begin{tikzpicture}
  \draw [help lines] (0,0) grid (2,2);

  % Stuff outside the picture, but still in a transparency group.
  \node [left,overlay] at (0,1) {
    \begin{tikzpicture}
      \pgfsetfillopacity{0.5}
      \pgftransparencygroup
      \node at (2,0) [forbidden sign,line width=2ex,draw=red,fill=white]
        {Smoking};
      \endpgftransparencygroup
    \end{tikzpicture}
  };
\end{tikzpicture}
\end{codeexample}
}%


\begin{plainenvironment}{{pgftransparencygroup}}
  Plain \TeX\ version of the |{pgftransparencygroup}| environment.
\end{plainenvironment}

\begin{contextenvironment}{{pgftransparencygroup}}
  This is the Con\TeX t version of the environment.
\end{contextenvironment}

\end{environment}



%%% Local Variables:
%%% mode: latex
%%% TeX-master: "pgfmanual"
%%% End:

% Copyright 2008 by Christian Feuersaenger
%
% This file may be distributed and/or modified
%
% 1. under the LaTeX Project Public License and/or
% 2. under the GNU Free Documentation License.
%
% See the file doc/generic/pgf/licenses/LICENSE for more details.


\section{Adding libraries to \pgfname: temporary registers}

\label{section-internal-registers}

This section is intended for those who like to write libraries to extend \pgfname. Of course, this requires a good deal of knowledge about \TeX-programming and the structure of the \pgfname\ basic layer. Besides, one will encounter the need of temporary variables and, especially, temporary \TeX\ registers. This section describes how to use a set of pre-allocated temporary registers of the basic layer without needing to allocate more of them.

A part of these internals are already mentioned in section~\ref{section-internal-pointcmds}, but the basic layer provides more temporaries than |\pgf@x| and |\pgf@y|.

\begin{internallist}[dimen register]{\pgf@x,\pgf@y}
	These registers are used to process point coordinates in the basic layer of \pgfname, see section~\ref{section-internal-pointcmds}. After a |\pgfpoint|$\dotsc$ command, they contain the final $x$ and $y$ coordinate, respectively.

	The values of |\pgf@x| and |\pgf@y| are set \emph{globally} in contrast to other available \pgfname\ registers. You should never assume anything about their value unless the context defines them explicitly.


	Please prefer the |\pgf@xa|, |\pgf@xb|, $\dotsc$ registers for temporary dimen registers unless you are writing point coordinate commands.
\end{internallist}

\begin{internallist}[dimen register]{
	\pgf@xa,
	\pgf@xb,
	\pgf@xc,
	\pgf@ya,
	\pgf@yb,
	\pgf@yc}
	Temporary registers for \TeX\ dimensions which can be modified freely. Just make sure changes occur only within \TeX\ groups. 

	\paragraph{Attention:}
	\pgfname\ uses these registers to perform path operations. For reasons of efficiency, path commands do not always guard them. As a consequence, the code
\begin{codeexample}[code only]
\pgfpointadd{\pgfpoint{\pgf@xa}{\pgf@ya}}{\pgfpoint{\pgf@xb}{\pgf@yb}}
\end{codeexample}
\noindent
may fail: Inside |\pgfpointadd|, the |\pgf@xa| and friend registers might be  
modified. In particular, it might happen that |\pgf@xb| is changed  
before |\pgfpoint{\pgf@xb}{\pgf@yb}| is evaluated. The right thing to do  
would be to first expand everything using |\edef| and process the values afterwards,
resulting in unnecessary expensive operations. Of course, one can avoid this by simply 
looking into the source code of |\pgfpointadd| to see which registers are used.
\end{internallist}

\begin{internallist}[dimen register]{\pgfutil@tempdima,\pgfutil@tempdimb}
	Further multi-purpose temporary dimen registers. For \LaTeX, these registers are already allocated as |\@tempdima| and |\@tempdimb| and are simply |\let| to the |\pgfutil@|$\dotsc$ names.
\end{internallist}

\begin{internallist}[count register]{
	\c@pgf@counta,
	\c@pgf@countb,
	\c@pgf@countc,
	\c@pgf@countd}
	These multiple-purpose count registers are used throughout \pgfname\ to perform integer computations. Feel free to use them as well, just make sure changes are scoped by local \TeX\ groups.
\end{internallist}

\begin{internallist}[openout handle]{\w@pgf@writea}
	An |\openout| handle which is used to generate complete output files within locally scoped parts of \pgfname\ (for example, to interact with |gnuplot|). You should always use |\immediate| in front of output operations involving |\w@pgf@writea| and you should always close the file before returning from your code.
\begin{codeexample}[code only]
\immediate\openout\w@pgf@writea=myfile.dat
\immediate\write\w@pgf@writea{...}%
\immediate\write\w@pgf@writea{...}%
\immediate\closeout\w@pgf@writea%
\end{codeexample}
\end{internallist}

\begin{internallist}[openin handle]{\r@pgf@reada}
	An |\openin| handle which is used to read files within locally scoped parts of \pgfname, for example to check if a file exists or to read data files. You should always use |\immediate| in front of output operations involving |\w@pgf@writea| and you should always close the file before returning from your code.
\begin{codeexample}[code only]
\immediate\openin\r@pgf@reada=myfile.dat
% do something with \macro
\ifeof\r@pgf@reada
    % end of file or it doesn't exist
\else
    % loop or whatever
    \immediate\read\r@pgf@reada to\macro
    ...
\fi
\immediate\closein\r@pgf@reada
\end{codeexample}
\end{internallist}

\begin{internallist}[box]{\pgfutil@tempboxa}
	A box for temporary use inside of local \TeX\ scopes. For \LaTeX, this box is the same as the already pre-allocated |\@tempboxa|.
\end{internallist}


% Copyright 2006 by Till Tantau
%
% This file may be distributed and/or modified
%
% 1. under the LaTeX Project Public License and/or
% 2. under the GNU Free Documentation License.
%
% See the file doc/generic/pgf/licenses/LICENSE for more details.


\section{Quick Commands}

This section explains the ``quick'' commands of \pgfname. These
commands are executed more quickly than the normal commands of
\pgfname, but offer less functionality. You should use these commands
only if you either have a very large number of commands that need to
be processed or if you expect your commands to be executed very often.


\subsection{Quick Coordiante Commands}

\begin{command}{\pgfqpoint\marg{x}\marg{y}}
  This command does the same as |\pgfpoint|, but \meta{x} and \meta{y}
  must be simple dimensions like |1pt| or |1cm|. Things like |2ex| or
  |2cm+1pt| are not allowed.
\end{command}

\subsection{Quick Path Construction Commands}

The difference between the quick and the normal path commands is that
the quick path commands
\begin{itemize}
\item
  do not keep track of the bounding boxes,
\item
  do not allow you to arc corners,
\item
  do not apply coordinate transformations.
\end{itemize}

However, they do use the soft-path subsystem (see
Section~\ref{section-soft-paths} for details), which allows you to mix
quick and normal path commands arbitrarily.

All quick path construction commands start with |\pgfpathq|.

\begin{command}{\pgfpathqmoveto\marg{x dimension}\marg{y dimension}}
  Either starts a path or starts a new part of a path at the coordinate
  $(\meta{x dimension},\meta{y dimension})$. The coordinate is
  \emph{not} transformed by the current coordinate transformation
  matrix. However, any low-level transformations apply.

\begin{codeexample}[]
\begin{tikzpicture}
  \draw[help lines] (0,0) grid (3,2);
  \pgftransformxshift{1cm}
  \pgfpathqmoveto{0pt}{0pt} % not transformed
  \pgfpathqlineto{1cm}{1cm} % not transformed
  \pgfpathlineto{\pgfpoint{2cm}{0cm}}
  \pgfusepath{stroke}
\end{tikzpicture}
\end{codeexample}
\end{command}

\begin{command}{\pgfpathqlineto\marg{x dimension}\marg{y dimension}}
  The quick version of the line-to operation.
\end{command}

\begin{command}{\pgfpathqcurveto\marg{$s^1_x$}\marg{$s^1_y$}\marg{$s^2_x$}\marg{$s^2_y$}\marg{$t_x$}\marg{$t_y$}}
  The quick version of the curve-to operation. The first support point
  is $(s^1_x,s^1_y)$, the second support point is  $(s^2_x,s^2_y)$,
  and the target is $(t_x,t_y)$.
 
\begin{codeexample}[]
\begin{tikzpicture}
  \draw[help lines] (0,0) grid (3,2);
  \pgfpathqmoveto{0pt}{0pt}
  \pgfpathqcurveto{1cm}{1cm}{2cm}{1cm}{3cm}{0cm}
  \pgfusepath{stroke}
\end{tikzpicture}
\end{codeexample}
\end{command}

\begin{command}{\pgfpathqcircle\marg{radius}}
  Adds a radius around the origin of the given \meta{radius}. This
  command is orders of magnitude faster than
  |\pgfcircle{\pgfpointorigin}{|\meta{radius}|}|. 
 
\begin{codeexample}[]
\begin{tikzpicture}
  \draw[help lines] (0,0) grid (1,1);
  \pgfpathqcircle{10pt}
  \pgfsetfillcolor{examplefill}
  \pgfusepath{stroke,fill}
\end{tikzpicture}
\end{codeexample}
\end{command}



\subsection{Quick Path Usage Commands}

The quick path usage commands perform similar tasks as |\pgfusepath|,
but they
\begin{itemize}
\item
  do not add arrows,
\item
  do not modify the path in any way, in particular,
\item
  ends are not shortened,
\item
  corners are not replaced by arcs.
\end{itemize}

Note that you \emph{have to} use the quick versions in the code of
arrow tip definitions since, inside these definition, you obviously do
not want arrows to be drawn.

\begin{command}{\pgfusepathqstroke}
  Strokes the path without further ado. No arrows are drawn, no
  corners are arced.

\begin{codeexample}[]
\begin{pgfpicture}
  \pgfpathqcircle{5pt}
  \pgfusepathqstroke
\end{pgfpicture}
\end{codeexample}
\end{command}

\begin{command}{\pgfusepathqfill}
  Fills the path without further ado.
\end{command}

\begin{command}{\pgfusepathqfillstroke}
  Fills and then strokes the path without further ado.
\end{command}

\begin{command}{\pgfusepathqclip}
  Clips all subsequent drawings against the current path. The path is
  not processed.
\end{command}


\subsection{Quick Text Box Commands}

\begin{command}{\pgfqbox\marg{box number}}
  This command inserts a \TeX\ box into a |{pgfpicture}| by
  ``escaping'' to \TeX, inserting the box number \meta{box number} at
  the origin, and then returning to the typesetting the picture.
\end{command}

\begin{command}{\pgfqboxsynced\marg{box number}}
  This command works similarly to the |\pgfqbox| command. However,
  before inserting the text in \meta{box number}, the current
  coordinate transformation matrix is applied to the current canvas
  transformation matrix (is it ``synced'' with this matrix, hence the
  name).

  Thus, this command basically has the same effect as if you first
  called |\pgflowlevelsynccm| followed by |\pgfqbox|. However, this
  command will use |\hskip| and |\raise| commands for the
  ``translational part'' of the coordinate transformation matrix,
  instead of adding the translational part to the current
  canvas transformation matrix directly. Both methods have the same
  effect (box \meta{box number} is translated where it should), but
  the method used by |\pgfqboxsynced| ensures that hyperlinks are
  placed correctly. Note that scaling and rotation will not (cannot,
  even) apply to hyperlinks.
\end{command}

%%% Local Variables: 
%%% mode: latex
%%% TeX-master: "pgfmanual"
%%% End: 





\part{The System Layer}
\label{part-system}

{\Large \emph{by Till Tantau}}


\bigskip
\noindent
This part describes the low-level interface of \pgfname, called the
\emph{system layer}. This interface provides a complete abstraction of
the internals of the underlying drivers.

Unless you intend to port \pgfname\ to another driver or unless you intend
to write your own optimized frontend, you need not read this part.

In the following it is assumed that you are familiar with the basic
workings of the |graphics| package and that you know what
\TeX-drivers are and how they work.

\vskip1cm
\begin{codeexample}[graphic=white]
\begin{tikzpicture}
  [shorten >=1pt,->,
   vertex/.style={circle,fill=black!25,minimum size=17pt,inner sep=0pt}]

  \foreach \name/\x in {s/1, 2/2, 3/3, 4/4, 15/11, 16/12, 17/13, 18/14, 19/15, t/16}
    \node[vertex] (G-\name) at (\x,0) {$\name$};

  \foreach \name/\angle/\text in {P-1/234/5, P-2/162/6, P-3/90/7, P-4/18/8, P-5/-54/9}
    \node[vertex,xshift=6cm,yshift=.5cm] (\name) at (\angle:1cm) {$\text$};

  \foreach \name/\angle/\text in {Q-1/234/10, Q-2/162/11, Q-3/90/12, Q-4/18/13, Q-5/-54/14}
    \node[vertex,xshift=9cm,yshift=.5cm] (\name) at (\angle:1cm) {$\text$};

  \foreach \from/\to in {s/2,2/3,3/4,3/4,15/16,16/17,17/18,18/19,19/t}
    \draw (G-\from) -- (G-\to);

  \foreach \from/\to in {1/2,2/3,3/4,4/5,5/1,1/3,2/4,3/5,4/1,5/2}
    { \draw (P-\from) -- (P-\to); \draw (Q-\from) -- (Q-\to); }

  \draw (G-3) .. controls +(-30:2cm) and +(-150:1cm) .. (Q-1);
  \draw (Q-5) -- (G-15);
\end{tikzpicture}
\end{codeexample}

% Copyright 2003 by Till Tantau <tantau@cs.tu-berlin.de>.
%
% This program can be redistributed and/or modified under the terms
% of the LaTeX Project Public License Distributed from CTAN
% archives in directory macros/latex/base/lppl.txt.

\section{Design of the System Layer}

\makeatletter


\subsection{Driver Files}
\label{section-pgfsys}

The \pgfname\ system layer mainly consists of a large number of
commands starting with |\pgfsys@|. These commands will be called
\emph{system commands} in the following. The higher layers
``interface'' with the system layer by calling these commands. The
higher layers should never use |\special| commands directly or even
check whether |\pdfoutput| is defined. Instead, all drawing requests
should be ``channeled'' through the system commands. 

The system layer is loaded and setup by the following package:

\begin{package}{pgfsys}
  This file provides ``default implementations'' of all system
  commands, but most simply produce a warning that they are not
  implemented. The actual implementations of the system commands for a
  particular driver like, say, |pdftex| reside in files called
  |pgfsys-xxxx.sty|, where |xxxx| is the driver name. These will be
  called \emph{driver files} in the following.

  When |pgfsys.sty| is loaded, it will try to determine which driver
  is used by loading |pgf.cfg|. This file should setup the macro
  |\pgfsysdriver| appropriately. The, |pgfsys.sty| will input the
  appropriate |pgfsys-|\meta{drivername}|.sty|. 
\end{package}

\begin{command}{\pgfsysdriver}
  This macro should expand to the name of the driver to be used by
  |pgfsys|. The default from |pgf.cfg| is |pgfsys-\Gin@driver|. This
  is very likely to be correct if you are using \LaTeX. For plain
  \TeX, the macro will be set to |pgfsys-pdftex.def| if |pdftex| is
  used and to |pgfsys-dvips.def| otherwise.
\end{command}

\begin{filedescription}{pgf.cfg}
  This file should setup the command |\pgfsysdriver| correctly. If
  |\pgfsysdriver| is already set to some value, the driver normally
  should not change it. Otherwise, it should make a ``good guess'' at
  which driver will be appropriate.
\end{filedescription}


The currently supported backend drivers are discussed in
Section~\ref{section-drivers}. 


\subsection{Common Definition Files}

Some drivers share many |\pgfsys@| commands. For the reason, files
defining these ``common'' commands are available. These files are
\emph{not} usable alone.

\begin{filedescription}{pgfsys-common-postscript}
  This file defines some |\pgfsys@| commands so that they produce
  appropriate PostScript code.
\end{filedescription}

\begin{filedescription}{pgfsys-common-pdf}
  This file defines some |\pgfsys@| commands so that they produce
  appropriate \textsc{pdf} code.
\end{filedescription}


%%% Local Variables: 
%%% mode: latex
%%% TeX-master: "pgfmanual"
%%% End: 

% Copyright 2006 by Till Tantau
%
% This file may be distributed and/or modified
%
% 1. under the LaTeX Project Public License and/or
% 2. under the GNU Free Documentation License.
%
% See the file doc/generic/pgf/licenses/LICENSE for more details.


\section{Commands of the System Layer}

\makeatletter

\subsection{Beginning and Ending a Stream of System Commands}

A ``user'' of the \pgfname\ system layer (like the basic layer or a
frontend) will interface with the system layer by calling a stream of
commands starting with |\pgfsys@|. From the system layer's point of
view, these commands form a long stream. Between calls to the system
layer, control goes back to the user.

The driver files implement system layer commands by inserting
|\special| commands that implement the desired operation. For example,
|\pgfsys@stroke| will be mapped to |\special{pdf: S}| by the driver
file for |pdftex|.

For many drivers, when such a stream of specials starts, it is
necessary to install an appropriate transformation and perhaps perform
some more bureaucratic tasks. For this reason, every stream will start
with a |\pgfsys@beginpicture| and will end with a corresponding ending
command.

\begin{command}{\pgfsys@beginpicture}
  Called at the beginning of a |{pgfpicture}|. This command should
  ``setup things.''

  Most drivers will need to implement this command.
\end{command}

\begin{command}{\pgfsys@endpicture}
  Called at the end of a pgfpicture. 

  Most drivers will need to implement this command.
\end{command}

\begin{command}{\pgfsys@typesetpicturebox\marg{box}}
  Called \emph{after} a |{pgfpicture}| has been typeset. The picture
  will have been put in box \meta{box}. This command should insert the
  box into the normal text. The box \meta{box} will still be a ``raw''
  box that contains only the |\special|'s that make up the description
  of the picture. The  job of this command is to resize and shift
  \meta{box} according to the  baseline shift and the size of the
  box. 

  This command has a default implementation and need not be
  implemented by a driver file.
\end{command}

\begin{command}{\pgfsys@beginpurepicture}
  This version of the |\pgfsys@beginpicture| picture command can be
  used for pictures that are guaranteed not to contain any escaped
  boxes (see below). In this case, a driver might provide a more
  compact version of the command. 
  
  This command has a default implementation and need not be
  implemented by a driver file.
\end{command}

\begin{command}{\pgfsys@endpurepicture}
  Called at the end of a ``pure'' |{pgfpicture}|.
  
  This command has a default implementation and need not be
  implemented by a driver file.
\end{command}

Inside a stream it is sometimes necessary to ``escape'' back into
normal typesetting mode; for example to insert some normal text, but
with all of the current transformations and clippings being in
force. For this escaping, the following command is used:

\begin{command}{\pgfsys@hbox\marg{box number}}
  Called to insert a (horizontal) TeX box inside a
  |{pgfpicture}|.
  
  Most drivers will need to (re-)implement this command.
\end{command}

\begin{command}{\pgfsys@hboxsynced\marg{box number}}
  Called to insert a (horizontal) TeX box inside a
  |{pgfpicture}|, but with the current coordiante transformation
  matrix synced with the canvas transformation matrix. 

  This command should do the same as if you used
  |\pgflowlevelsynccm| followed by |\pgfsys@hbox|. However, the default
  implementation of this command will use a ``TeX-translation'' for
  the translation part of the transformation matrix. This will ensure
  that hyperlinks ``survive'' at least translations. On the other
  hand, a driver may choose to revert to a simpler
  implementation. This is done, for example, for the \textsc{svg}
  implementation, where a \TeX-translation makes no sense.
\end{command}



\subsection{Path Construction System Commands}

\begin{command}{\pgfsys@moveto\marg{x}\marg{y}}
  This command is used to start a path at a specific point
  $(x,y)$ or to move the current point of the current path to  $(x,y)$
  without drawing anything upon stroking (the current path is
  ``interrupted'').

  Both \meta{x} and \meta{y} are given as \TeX\ dimensions. It is the
  driver's job to transform these to the coordinate system of the
  backend. Typically, this means converting the \TeX\ dimension into a
  dimensionless multiple of $\frac{1}{72}\mathrm{in}$. The function
  |\pgf@sys@bp| helps with this conversion.

  \example Draw a line from $(10\mathrm{pt},10\mathrm{pt})$ to the
  origin of the picture. 
\begin{codeexample}[code only]
\pgfsys@moveto{10pt}{10pt}
\pgfsys@lineto{0pt}{0pt}
\pgfsys@stroke
\end{codeexample}

  This command is protocoled, see Section~\ref{section-protocols}.
\end{command}


\begin{command}{\pgfsys@lineto\marg{x}\marg{y}}
  Continue the current path to $(x,y)$ with
  a straight line.

  This command is protocoled, see Section~\ref{section-protocols}.
\end{command}


\begin{command}{\pgfsys@curveto\marg{$x_1$}\marg{$y_1$}\marg{$x_2$}\marg{$y_2$}\marg{$x_3$}\marg{$y_3$}}
  Continue the current path to $(x_3,y_3)$
  with a B�zier curve that has the two control points  $(x_1,y_1)$ and  $(x_2,y_2)$.

  \example Draw a good approximation of a quarter circle:
\begin{codeexample}[code only]
\pgfsys@moveto{10pt}{0pt}
\pgfsys@curveto{10pt}{5.55pt}{5.55pt}{10pt}{0pt}{10pt}
\pgfsys@stroke
\end{codeexample}

  This command is protocoled, see Section~\ref{section-protocols}.
\end{command}


\begin{command}{\pgfsys@rect\marg{x}\marg{y}\marg{width}\marg{height}}
  Append a rectangle to the current path whose lower left corner is
  at $(x,y)$ and whose width and height in
  big points are  given by \meta{width} and \meta{height}.

  This command can be ``mapped back'' to |\pgfsys@moveto| and
  |\pgfsys@lineto| commands, but it is included since \pdf\ has a
  special, quick version of this command. 

  This command is protocoled, see Section~\ref{section-protocols}.
\end{command}


\begin{command}{\pgfsys@closepath}
  Close the current path. This results in joining the current point of
  the path with the point specified by the last |\pgfsys@moveto|
  operation. Typically, this is preferable over using |\pgfsys@lineto|
  to the last point specified by a |\pgfsys@moveto|, since the line
  starting at this point and the line ending at this point will be
  smoothly joined by |\pgfsys@closepath|.

  \example Consider
\begin{codeexample}[code only]
\pgfsys@moveto{0pt}{0pt}
\pgfsys@lineto{10bp}{10bp}
\pgfsys@lineto{0bp}{10bp}
\pgfsys@closepath
\pgfsys@stroke
\end{codeexample}
  and
\begin{codeexample}[code only]
\pgfsys@moveto{0bp}{0bp}
\pgfsys@lineto{10bp}{10bp}
\pgfsys@lineto{0bp}{10bp}
\pgfsys@lineto{0bp}{0bp}
\pgfsys@stroke
\end{codeexample}
  
  The difference between the above will be that in the second triangle
  the corner at the origin will be wrong; it will just be the overlay
  of two lines going in different directions, not a sharp pointed
  corner.

  This command is protocoled, see Section~\ref{section-protocols}.
\end{command}




\subsection{Canvas Transformation System Commands}

\begin{command}{\pgfsys@transformcm\marg{a}\marg{b}\marg{c}\marg{d}\marg{e}\marg{f}}
  Perform a concatenation of the canvas transformation matrix with the
  matrix given by the values \meta{a} to \meta{f}, see the \pdf\ or
  PostScript manual for details. The values \meta{a} to \meta{d} are
  dimensionless factors, \meta{e} and \meta{f} are \TeX\ dimensions 

  \example |\pgfsys@transformcm{1}{0}{0}{1}{1cm}{1cm}|.

  This command is protocoled, see Section~\ref{section-protocols}.  
\end{command}


\begin{command}{\pgfsys@transformshift\marg{x displacement}\marg{y displacement}}
  This command will change the origin of the canvas to $(x,y)$.

  This command has a default implementation and need not be
  implemented by a driver file.

  This command is protocoled, see Section~\ref{section-protocols}.
\end{command}

\begin{command}{\pgfsys@transformxyscale\marg{x scale}\marg{y scale}}
  This command will scale the canvas (and  everything that is drawn)
  by a factor of \meta{x scale} in the $x$-direction and \meta{y
    scale} in the  $y$-direction. Note that this applies to
  everything, including  lines. So a scaled line will have a different
  width and may even have a different width when going along the
  $x$-axis and when going along the $y$-axis, if the scaling is
  different in these directions. Usually, you do not want this.

  This command has a default implementation and need not be
  implemented by a driver file.

  This command is protocoled, see Section~\ref{section-protocols}.
\end{command}


\subsection{Stroking, Filling, and Clipping System Commands}

\begin{command}{\pgfsys@stroke}
  Stroke the current path (as if it were drawn with a pen). A number
  of graphic state parameters influence this, which can be
  set using appropriate system commands described later.

  \begin{description}
  \item[Line width]
    The ``thickness'' of the line. A width of 0 is the thinnest width
    renderable on the device. On a high-resolution printer this may
    become invisible and should be avoided. A good choice is 0.4pt,
    which is the default.

  \item[Stroke color]
    This special color is used for stroking. If it is not set, the
    current color is used.
 
  \item[Cap]
    The cap describes how the endings of lines are drawn. A round cap
    adds a little half circle to these endings. A butt cap ends the
    lines exactly at the end (or start) point without anything
    added. A rectangular cap ends the lines like the butt cap, but the
    lines protrude over the endpoint by the line thickness. (See also
    the \pdf\ manual.) If the path has been closed, no cap
    is drawn.
 
  \item[Join]
    This describes how a bend (a join) in a path is rendered. A round
    join draws bends using small arcs. A bevel join just draws the two
    lines and then fills the join minimally so that it becomes
    convex. A miter join extends the lines so that they form a single
    sharp corner, but only up to a certain miter limit. (See the \pdf\
    manual once more.)
 
  \item[Dash]
    The line may be dashed according to a dashing pattern.
 
  \item[Clipping area]
    If a clipping area is established, only those parts of the path
    that are inside the clipping area will be drawn.
  \end{description}
  
  In addition to stroking a path, the path may also be used for
  clipping after it has been stroked. This will happen if the
  |\pgfsys@clipnext| is used prior to this command, see there for
  details.

  This command is protocoled, see Section~\ref{section-protocols}.
\end{command}


\begin{command}{\pgfsys@closestroke}
  This command should have the same effect as first closing the path
  and then stroking it.

  This command has a default implementation and need not be
  implemented by a driver file.

  This command is protocoled, see Section~\ref{section-protocols}.
\end{command}


\begin{command}{\pgfsys@fill}
  This command fills the area surrounded by the current path. If the
  path has not yet been closed, it is closed prior to filling. The
  path itself is not stroked. For self-intersecting paths or paths
  consisting of multiple parts, the nonzero winding number rule is
  used to determine whether a point is inside or outside the
  path, except if |\ifpgfsys@eorule| holds -- in which case the
  even-odd rule should be used. (See the \pdf\ or PostScript manual
  for details.)  
 
  The following graphic state parameters influence the filling:
 
  \begin{description}
  \item[Interior rule]
    If |\ifpgfsys@eorule| is set, the even-odd rule is used, otherwise
    the non-zero winding number rule.
 
  \item[Fill color]
    If the fill color is not especially set, the current color is
    used. 
 
  \item[Clipping area]
    If a clipping area is established, only those parts of the filling
    area that are inside the clipping area will be drawn.
  \end{description}

  In addition to filling the path, the path will also be used for
  clipping if |\pgfsys@clipnext| is used prior to this command.

  This command is protocoled, see Section~\ref{section-protocols}.
\end{command}

\begin{command}{\pgfsys@fillstroke}
  First, the path is filled, then the path is stroked. If the fill and
  stroke colors are the same (or if they are not specified and the
  current color is used), this yields almost the same as a
  |\pgfsys@fill|. However, due to the line thickness of the stroked
  path, the fill-stroked area will be slightly larger.

  In addition to stroking and filling the path, the path will also be
  used for clipping if |\pgfsys@clipnext| is used prior to this command.

  This command is protocoled, see Section~\ref{section-protocols}.
\end{command}


\begin{command}{\pgfsys@discardpath}
 Normally, this command should ``throw away'' the current path.
 However, after |\pgfsys@clipnext| has been called, the current path
 should subsequently be used for clipping. See |\pgfsys@clipnext| for 
 details. 

  This command is protocoled, see Section~\ref{section-protocols}.
\end{command}


\begin{command}{\pgfsys@clipnext}
  This command should be issued after a path has been constructed, but
  before it has been stroked and/or filled or discarded. When the
  command is used, the next stroking/filling/discarding command will
  first be executed normally. Then, afterwards, the just-used path
  will be used for subsequent clipping. If there has already been a
  clipping region, this region is intersected with the new clipping
  path (the clipping cannot get bigger). The nonzero winding number
  rule is used to determine whether a point is inside or outside the
  clipping area or the even-odd rule, depending on whether
  |\ifpgfsys@eorule| holds.
\end{command}




\subsection{Graphic State Option System Commands}

\begin{command}{\pgfsys@setlinewidth\marg{width}}
  Sets the width of lines, when stroked, to \meta{width}, which must
  be a \TeX\ dimension.

  This command is protocoled, see Section~\ref{section-protocols}.
\end{command}

\begin{command}{\pgfsys@buttcap}
  Sets the cap to a butt cap. See |\pgfsys@stroke|.

  This command is protocoled, see Section~\ref{section-protocols}.
\end{command}

\begin{command}{\pgfsys@roundcap}
  Sets the cap to a round cap. See |\pgfsys@stroke|.

  This command is protocoled, see Section~\ref{section-protocols}.
\end{command}

\begin{command}{\pgfsys@rectcap}
  Sets the cap to a rectangular cap. See |\pgfsys@stroke|.

  This command is protocoled, see Section~\ref{section-protocols}.
\end{command}

\begin{command}{\pgfsys@miterjoin}
  Sets the join to a miter join. See |\pgfsys@stroke|.

  This command is protocoled, see Section~\ref{section-protocols}.
\end{command}

\begin{command}{\pgfsys@setmiterlimit\marg{factor}}
  Sets the miter limit of lines to \meta{factor}. See
  the \pdf\ or PostScript for details on what the miter limit is.

  This command is protocoled, see Section~\ref{section-protocols}.
\end{command}

\begin{command}{\pgfsys@roundjoin}
  Sets the join to a round join. See |\pgfsys@stroke|.

  This command is protocoled, see Section~\ref{section-protocols}.
\end{command}

\begin{command}{\pgfsys@beveljoin}
  Sets the join to a bevel join. See |\pgfsys@stroke|.

  This command is protocoled, see Section~\ref{section-protocols}.
\end{command}

\begin{command}{\pgfsys@setdash\marg{pattern}\marg{phase}}
  Sets the dashing patter. \meta{pattern} should be a list of \TeX\
  dimensions lengths separated by commas. \meta{phase} should be a
  single dimension.

  \example |\pgfsys@setdash{3pt,3pt}{0pt}|
 
  The list of values in \meta{pattern} is used to determine the
  lengths of the ``on'' phases of the dashing and of the ``off''
  phases. For example, if \meta{pattern} is |3bp,4bp|, then the dashing
  pattern is ``3bp on followed by 4bp off, followed by 3bp on,
  followed by 4bp off, and so on.'' A pattern of |.5pt,4pt,3pt,1.5pt| means
  ``.5pt on, 4pt off, 3pt on, 1.5pt off, .5pt on, \dots'' If the
  number of entries is odd, the last one is used twice, so |3pt| means
  ``3pt on, 3pt off, 3pt on, 3pt off, \dots'' An empty list 
  means  ``always on.''
 
  The second argument determines the ``phase'' of the pattern. For
  example, for a pattern of |3bp,4bp| and a phase of |1bp|, the pattern
  would start: ``2bp on, 4bp off, 3bp on, 4bp off, 3bp on, 4bp off,
  \dots''

  This command is protocoled, see Section~\ref{section-protocols}.
\end{command}

{\let\ifpgfsys@eorule=\relax
\begin{command}{\ifpgfsys@eorule}
  Determines whether the even odd rule is used for filling and
  clipping or not.
\end{command}
}


\subsection{Color System Commands}

The \pgfname\ system layer provides a number of system commands for
setting colors. These command coexist with commands from the |color|
and |xcolor| package, which perform similar functions. However, the
|color| package does not support having two different colors for
stroking and filling, which is a useful feature that is supported by
\pgfname. For this reason, the \pgfname\ system layer offers commands for
setting these colors separatedly. Also, plain \TeX\ profits from the
fact that \pgfname\ can set colors.

For \pdf, implementing these color commands is easy since \pdf\ 
supports different stroking and filling colors directly. For
PostScript, a more complicated approach is needed in which the colors
need to be stored in special PostScript variables that are set
whenever a stroking or a filling operation is done.

\begin{command}{\pgfsys@color@rgb\marg{red}\marg{green}\marg{blue}}
  Sets the color used for stroking and filling operations to the given 
  red/green/blue tuple (numbers between 0 and 1).

  This command is protocoled, see Section~\ref{section-protocols}.
\end{command}

\begin{command}{\pgfsys@color@rgb@stroke\marg{red}\marg{green}\marg{blue}}
  Sets the color used for stroking operations to the given
  red/green/blue tuple (numbers between 0 and 1). 

  \example Make stroked text dark red: |\pgfsys@color@rgb@stroke{0.5}{0}{0}|

  The special stroking color is only used if the stroking color has
  been set since the last |\color| or |\pgfsys@color@xxx|
  command. Thus, each |\color| command will reset both the stroking
  and filling colors by calling |\pgfsys@color@reset|. 

  This command is protocoled, see Section~\ref{section-protocols}.
\end{command}

\begin{command}{\pgfsys@color@rgb@fill\marg{red}\marg{green}\marg{blue}}
  Sets the color used for filling operations to the given
  red/green/blue tuple (numbers between 0 and 1). This color may be
  different from the stroking color.

  This command is protocoled, see Section~\ref{section-protocols}.
\end{command}

\begin{command}{\pgfsys@color@cmyk\marg{cyan}\marg{magenta}\marg{yellow}\marg{black}}
  Sets the color used for stroking and filling operations to the given
  cymk tuple (numbers between 0 and 1). 

  This command is protocoled, see Section~\ref{section-protocols}.
\end{command}

\begin{command}{\pgfsys@color@cmyk@stroke\marg{cyan}\marg{magenta}\marg{yellow}\marg{black}}
  Sets the color used for stroking operations to the given cymk tuple
  (numbers between 0 and 1). 

  This command is protocoled, see Section~\ref{section-protocols}.
\end{command}

\begin{command}{\pgfsys@color@cmyk@fill\marg{cyan}\marg{magenta}\marg{yellow}\marg{black}}
  Sets the color used for filling operations to the given cymk tuple
  (numbers between 0 and 1). 

  This command is protocoled, see Section~\ref{section-protocols}.
\end{command}

\begin{command}{\pgfsys@color@cmy\marg{cyan}\marg{magenta}\marg{yellow}}
  Sets the color used for stroking and filling operations to the given
  cym tuple (numbers between 0 and 1). 

  This command is protocoled, see Section~\ref{section-protocols}.
\end{command}

\begin{command}{\pgfsys@color@cmy@stroke\marg{cyan}\marg{magenta}\marg{yellow}}
  Sets the color used for stroking operations to the given cym tuple
  (numbers between 0 and 1). 

  This command is protocoled, see Section~\ref{section-protocols}.
\end{command}

\begin{command}{\pgfsys@color@cmy@fill\marg{cyan}\marg{magenta}\marg{yellow}}
  Sets the color used for filling operations to the given cym tuple
  (numbers between 0 and 1). 

  This command is protocoled, see Section~\ref{section-protocols}.
\end{command}

\begin{command}{\pgfsys@color@gray\marg{black}}
  Sets the color used for stroking and filling operations to the given
  black value, where 0 means black and 1 means white.

  This command is protocoled, see Section~\ref{section-protocols}.
\end{command}

\begin{command}{\pgfsys@color@gray@stroke\marg{black}}
  Sets the color used for stroking operations to the given black value,
  where 0 means black and 1 means white.

  This command is protocoled, see Section~\ref{section-protocols}.
\end{command}

\begin{command}{\pgfsys@color@gray@fill\marg{black}}
  Sets the color used for filling operations to the given black value,
  where 0 means black and 1 means white.

  This command is protocoled, see Section~\ref{section-protocols}.
\end{command}

\begin{command}{\pgfsys@color@reset}
  This command will be called when the |\color| command is used. It
  should purge any internal settings of stroking and filling
  color. After this call, till the next use of a command like
  |\pgfsys@color@rgb@fill|, the current color installed by the
  |\color| command should be used.

  If the \TeX-if |\pgfsys@color@reset@inorder| is set to true, this
  command may ``assume'' that any call to a color command that sets
  the fill or stroke color came ``before'' the call to this command
  and may try to optimize the output accordingly.

  An example of an incorrect ``out of order'' call would be using
  |\pgfsys@color@reset| at the beginning of a box that is constructed
  using |\setbox|. Then, when the box is constructed, no special fill
  or stroke color might be in force. However, when the box is later on
  inserted at some point, a special fill color might already have been
  set. In this case, this command is not guaranteed to reset the color
  correctly. 
\end{command}

\begin{command}{\pgfsys@color@reset@inordertrue}
  Sets the optimized ``in order'' version of the color resetting. This
  is the default.
\end{command}

\begin{command}{\pgfsys@color@reset@inorderfalse}
  Switches off the optimized color resetting. 
\end{command}

\begin{command}{\pgfsys@color@unstacked\marg{\LaTeX\ color}}
  This slightly obscure command causes the color stack to be
  tricked. When called, this command should set the current color to
  \meta{\LaTeX\ color} without causing any change in the color stack.

  \example |\pgfsys@color@unstacked{red}|
\end{command}




\subsection{Pattern System Commands}


\begin{command}{\pgfsys@declarepattern
    \marg{name}\marg{$x_1$}\marg{$y_1$}\marg{$x_2$}\marg{$y_2$}
    \marg{$x$ step}\marg{$y$ step}\marg{code}\marg{flag}}
  This command declares a new colored or uncolored pattern, depending
  on whether \meta{flag} is |0|, which means uncolored, or |1|, which
  means colored. Uncolored patterns have no inherent color, the color
  is provided when they are set. Colored patters have an inherent
  color.

  The \meta{name} is a name for later use when the pattern is to be
  shown. The pairs $(x_1,y_1)$ and $(x_2,y_2)$ must describe a
  bounding box of the pattern \meta{code}.

  The tiling step of the pattern is given by \meta{$x$ step} and
  \meta{$y$ step}.

  \example
\begin{codeexample}[code only]
\pgfsys@declarepattern{hori}{-.5pt}{0pt}{.5pt}{3pt}{3pt}{3pt}
{\pgfsys@moveto{0pt}{0pt}\pgfsys@lineto{0pt}{3pt}\pgfsys@stroke}
{0}
\end{codeexample}
\end{command}

\begin{command}{\pgfsys@setpatternuncolored\marg{name}\marg{red}\marg{green}\marg{blue}}
  Sets the fill color to the pattern named \meta{name}. This pattern
  must previously have been declared with \meta{flag} set to |0|. The
  color of the pattern is given in the parameters \meta{red},
  \meta{green}, and \meta{blue} in the usual way.

  The fill color ``pattern''  will persist till the next color command
  that modifies the fill color.
\end{command}

\begin{command}{\pgfsys@setpatterncolored\marg{name}}
  Sets the fill color to the pattern named \meta{name}. This pattern 
  must have been declared with the |1| flag.
\end{command}



\subsection{Scoping System Commands}

The scoping commands are used to keep changes of the graphics state
local.

\begin{command}{\pgfsys@beginscope}
  Saves the current graphic state on a graphic state stack. All
  changes to the graphic state parameters mentioned for |\pgfsys@stroke|
  and |\pgfsys@fill| will be local to the current graphic state and 
  the old values will be restored after |\pgfsys@endscope| is used.
 
  \emph{Warning:} \pdf\ and PostScript differ with respect to the
  question of whether the current path is part of the graphic state or
  not. For this reason, you should never use this command unless the
  path is currently empty. For example, it might be a good idea to use 
  |\pgfsys@discardpath| prior to calling this command. 

  This command is protocoled, see Section~\ref{section-protocols}.
\end{command}

\begin{command}{\pgfsys@endscope}
  Restores the last saved graphic state.

  This command is protocoled, see Section~\ref{section-protocols}.
\end{command}







\subsection{Image System Commands}

The system layer provides some commands for image inclusion.

\begin{command}{\pgfsys@imagesuffixlist}
  This macro should expand to a list of suffixes, separated by `:',
  that will be tried when searching for an image.

  \example |\def\pgfsys@imagesuffixlist{eps:epsi:ps}|
\end{command}


\begin{command}{\pgfsys@defineimage}
  Called, when an image should be defined. 
 
  This command does not take any parameters. Instead, certain macros
  will be preinstalled with appropriate values when this command is
  invoked. These are:
 
  \begin{itemize}
  \item\declare{|\pgf@filename|}
    File name of the image to be defined.

  \item\declare{|\pgf@imagewidth|}
    Will be set to the desired (scaled) width of the image.

  \item\declare{|\pgf@imageheight|}
    Will be set to the desired (scaled) height of the image.
 
    If this macro and also the height macro are empty, the image
    should have its ``natural'' size.
 
    If exactly only of them is specified, the undefined value the
    image is scaled so that the aspect ratio is kept.
 
    If both are set, the image is scaled in both directions
    independently, possibly changing the aspect ratio.
  \end{itemize}
 
  The following macros presumable mostly make sense for drivers that
  can handle \pdf: 

  \begin{itemize}
  \item \declare{|\pgf@imagepage|}
    The desired page number to be extracted from a multi-page
    ``image.''

  \item\declare{|\pgf@imagemask|}
    If set, it will be set to |/SMask x 0 R| where |x| is the \pdf\ 
    object number of a soft mask to be applied to the image.

  \item\declare{|\pgf@imageinterpolate|}
    If set, it will be set to |/Interpolate true| or
    |/Interpolate false|, indicating whether the image should be
    interpolated in \pdf. 
  \end{itemize}
 
  The command should now setup the macro |\pgf@image| such that calling
  this macro will result in typesetting the image. Thus, |\pgf@image| is
  the ``return value'' of the command.

  This command has a default implementation and need not be
  implemented by a driver file.
\end{command}



\subsection{Shading System Commands}


\begin{command}{\pgfsys@horishading\marg{name}\marg{height}\marg{specification}}
  Declares a horizontal shading for later use. The effect of this
  command should be the definition of a macro called |\@pgfshading|\meta{name}|!|
  (or |\csname @pdfshading|\meta{name}|!\endcsname|, to be
  precise). When invoked, this new macro should insert a shading at
  the current position. 
 
  \meta{name} is the name of the shading, which is also used in the
  output macro name. \meta{height} is the height of the shading and
  must be given as a TeX dimension like |2cm| or
  |10pt|. \meta{specification} is a shading color 
  specification as specified in Section~\ref{section-shadings}. The
  shading specification implicitly fixes the width of the shading. 
 
  When |\@pgfshading|\meta{name}|!| is invoked, it should insert a box
  of height \meta{height} and the width implicit in the shading
  declaration. 
\end{command}


\begin{command}{\pgfsys@vertshading\marg{name}\marg{width}\marg{specification}}
  Like the horizontal version, only for vertical shadings. This time,
  the height of the shading is implicit in \meta{specification} and
  the width is given as \meta{width}.
\end{command}

\begin{command}{\pgfsys@radialshading\marg{name}\marg{starting point}\marg{specification}}
  Declares a radial shading. Like the previous macros, this command
  should setup the macro |\@pgfshading|\meta{name}|!|, which upon
  invocation should insert a radial shading whose size is implicit in
  \meta{specification}.

  The parameter \meta{starting point} is a \pgfname\ point
  specifying the inner starting point of the shading.
\end{command}


\begin{command}{\pgfsys@functionalshading\marg{name}\marg{lower left
      corner}\meta{upper right corner}\marg{type 4 function}}
 Declares a shading using a PostScript-like function that provides a
 color for each point. Like the previous macros, this command
 should setup the macro |\@pgfshading|\meta{name}|!| so that it will
 produce a box containing the desired shading.

 Parameter \meta{name} is the name of the shading. Parameter
 \meta{type 4 function} is a
 Postscript-like function (type 4 function of the PDF specification)
 as described in Section 3.9.4 of the PDF Specification version 1.7.
 Parameters \meta{lower left corner} and \meta{upper right corner} are
 \pgfname\ points that specifies the lower left and upper right
 corners of the shading.

 When \meta{type 4 function} is evaluated, the coordinate of the current
 point will be on the (virtual) PostScript stack in bp units. After
 the function has been evaluated, the stack should consist of three
 numbers (not integers! -- the Apple PDF renderer is broken in this
 regard, so add cvr's at the end if needed) that represent the red,
 green, and blue components of the color.

 A buggy function will result is \emph{totally unpredictable chaos} during
 rendering. 
\end{command}



\subsection{Transparency System Commands}

\begin{command}{\pgfsys@stroke@opacity\marg{value}}
  Sets the opacity of stroking operations.
\end{command}

\begin{command}{\pgfsys@fill@opacity\marg{value}}
  Sets the opacity of filling operations.
\end{command}

\begin{command}{\pgfsys@transparencygroupfrombox\marg{box}}
  This takes a TeX box and converts it into a transparency
  group. This means that any transparency settings apply to the box as
  a whole. For instance, if a box contains two overlapping black
  circles and you draw the box and, thus, the two cirlces normally
  with 50\% transparency, then the overlap will be darker than the
  rest. By comparison, if the circles are part of a transparency
  group, the overlap will get the same color as the rest.
\end{command}

\begin{command}{\pgfsys@fadingfrombox\marg{name}\marg{box}}
  Declares the fading \meta{name}. The \meta{box} is a \TeX-box. Its
  contents luminosity determines the opacity of the resulting
  fading. This means that the lighter a pixel inside the box, the more
  opaque the fading will be at this position.
\end{command}

\begin{command}{\pgfsys@usefading\meta{name}\marg{a}\marg{b}\marg{c}\marg{d}\marg{e}\marg{f}}
  Installs a previously declared fading \meta{name} in the current
  graphics state. Afterwards, all drawings will be masked by the
  fading. The fading should be centered on the origin and have its
  original size, except that the parameters \meta{a} to \meta{f}
  specify a transformation matrix that should be applied additionally
  to the fading before it is installed. The transformtion should not
  apply to the following graphics, however.
\end{command}


\begin{command}{\pgfsys@definemask}
  This command declares a fading (known as a soft mask in this
  context) based on an image and for usage with images. It
  works similar to |\pgfsys@defineimage|: Certain macros are set when
  the command is called. The result should be to set the macro
  |\pgf@mask| to a pdf object count that can subsequently be used as a
  transparency mask. The following macros will be set when this command is
  invoked: 
 
  \begin{itemize}
  \item \declare{|\pgf@filename|}
    File name of the mask to be defined.

  \item \declare{|\pgf@maskmatte|}
    The so-called matte of the mask (see the \pdf\ documentation for
    details). The matte is a color specification consisting of 1, 3 or
    4 numbers between 0 and 1. The number of numbers depends on the
    number of color channels in the image (not in the mask!). It will
    be assumed that the image has been preblended with this color.
  \end{itemize}
\end{command}




\subsection{Reusable Objects System Commands}

\begin{command}{\pgfsys@invoke\marg{literals}}
  This command gets protocoled literals and should insert them into
  the |.pdf| or |.dvi| file using an appropriate |\special|.
\end{command}

\begin{command}{\pgfsys@defobject\marg{name}\marg{lower
      left}\marg{upper right}\marg{code}}
  Declares an object for later use. The idea is that the object can be
  precached in some way and then be rendered more quickly when used
  several times. For example, an arrow head might be defined and
  prerendered in this way.
 
  The parameter \meta{name} is the name for later use. \meta{lower
  left} and \meta{upper right} are \pgfname\ points specifying a bounding
  box for the object. \meta{code} is the code for the object. The code
  should not be too fancy.

  This command has a default implementation and need not be
  implemented by a driver file.
\end{command}

\begin{command}{\pgfsys@useobject\marg{name}\marg{extra code}}
  Renders a previously declared object. The first parameter is the
  name of the the object. The second parameter is extra code that
  should be executed right \emph{before} the object is
  rendered. Typically, this will be some transformation code.

  This command has a default implementation and need not be
  implemented by a driver file.
\end{command}


\subsection{Invisibility System Commands}

All drawing or stroking or text rendering between calls of the
following commands should be suppressed. A similar effect can be
achieved by clipping against an empty region, but the following
commands do not open a graphics scope and can be opened and closed
``orthogonally'' to other scopes.

\begin{command}{\pgfsys@begininvisible}
  Between this command and the closing |\pgfsys@endinvisible| all
  output should be suppressed. Nothing should be drawn at all, which
  includes all paths, images and shadings. However, no groups (neither
  \TeX\ groups nor graphic state groups) should be opened by this
  command. 

  This command has a default implementation and need not be
  implemented by a driver file.

  This command is protocoled, see Section~\ref{section-protocols}.
\end{command}
  
\begin{command}{\pgfsys@endinvisible}
  Ends the invisibility section, unless invisibility blocks have been
  nested. In this case, only the ``last'' one restores visibility.

  This command has a default implementation and need not be
  implemented by a driver file.

  This command is protocoled, see Section~\ref{section-protocols}.
\end{command}


\subsection{Position Tracking Commands}

The following commands are used to determine the position of text on a
page. This is a rather complicated process in general since at the
moment when the text is read by \TeX\ the final position cannot be
determined, yet. For example, the text might be put in a box which is
later put in the headline or perhaps in the footline or perhaps even
on a different page.

For these reasons, position tracking is typically a two-stage
process. In a first stage you indicate that a certain position is of
interest by \emph{marking} it. This will (depending on the details of
the backend driver) cause page coordinates or this position to be
written to a |.aux| file when the page is shipped. Possibly, the
position might also be determined at an even later stage. Then, on a
second run of \TeX, the position is read from the |.aux| file and can
be used.

\begin{command}{\pgfsys@markposition\marg{name}}
  Marks a position on the page. This command should be given while
  normal typesetting is done such as in
\begin{codeexample}[code only]
The value of $x$ is \pgfsys@markposition{here}important.
\end{codeexample}
  It causes the position |here| to be saved when the page is shipped
  out.
\end{command}

\begin{command}{\pgfsys@getposition\marg{name}\marg{macro}}
  This command retrieves a position that has been marked on an earlier
  run of \TeX\ on the current file. The \meta{macro} must be a macro
  name such as |\mymarco|. It will redefined such that it is
  \begin{itemize}
  \item either just |\relax| or
  \item a |\pgfpoint...| command.
  \end{itemize}
  The first case will happen when the position has not been marked at
  all or when the file is typeset for the first time, when the
  coordinates are not yet available.

  In the second case, executing \meta{macro} yields the position on
  the page that is to be interpreted as follows: A coordinate like
  |\pgfpoint{2cm}{3cm}| means ``2cm to the right and 3cm up from the
  origin of the page.'' The position of the origin of the page is not
  guaranteed to be at the lower left corner, it is only guaranteed
  that all pictures on a page use the same origin.

  To determine the lower left corner of a page, you can call
  |\pgfsys@getposition| with \meta{name} set to the special name
  |pgfpageorigin|. By shifting all positions by the amount returned by
  this call you can position things absolutely on a page.

  \example Referencing a point or the page:
\begin{codeexample}[code only]
The value of $x$ is \pgfsys@markposition{here}important.

Lots of text.

\hbox{\pgfsys@markposition{myorigin}%
\begin{pgfpicture}
  % Switch of size protocol
  \pgfpathmoveto{\pgfpointorigin}
  \pgfusepath{use as bounding box}

  \pgfsys@getposition{here}{\hereposition}
  \pgfsys@getposition{myorigin}{\thispictureposition}

  \pgftransformshift{\pgfpointscale{-1}{\thispictureposition}}
  \pgftransformshift{\hereposition}

  \pgfpathcircle{\pgfpointorigin}{1cm}
  \pgfusepath{draw}
\end{pgfpicture}}
\end{codeexample}
\end{command}


\subsection{Internal Conversion Commands}

The system commands take \TeX\ dimensions as input, but the dimensions
that have to be inserted into \pdf\ and PostScript files need to be
dimensionless values that are interpreted as multiples of
$\frac{1}{72}\mathrm{in}$. For example, the \TeX\ dimension $2bp$
should be inserted as |2| into a \pdf\ file and the \TeX\ dimension
$10\mathrm{pt}$ as |9.9626401|. To make this conversion easier, the following
command may be useful:

\begin{command}{\pgf@sys@bp\marg{dimension}}
  Inserts how many multiples of $\frac{1}{72}\mathrm{in}$ the
  \meta{dimension} is into the current protocol stream (buffered).

  \example |\pgf@sys@bp{\pgf@x}| or |\pgf@sys@bp{1cm}|.
\end{command}

Note that this command is \emph{not} a system command that can/needs
to be overwritten by a driver. 

%%% Local Variables: 
%%% mode: latex
%%% TeX-master: "pgfmanual"
%%% End: 

% Copyright 2006 by Till Tantau
%
% This file may be distributed and/or modified
%
% 1. under the LaTeX Project Public License and/or
% 2. under the GNU Free Documentation License.
%
% See the file doc/generic/pgf/licenses/LICENSE for more details.



\section{The Soft Path Subsystem}

\label{section-soft-paths}

\makeatletter


This section describes a set of commands for creating \emph{soft
  paths} as opposed to the commands of the previous section, which
created \emph{hard paths}. A soft path is a path that can still be
``changed'' or ``molded.'' Once you (or the \pgfname\ system) is
satisfied with a soft path, it is turned into a hard path, which can
be inserted into the resulting |.pdf| or |.ps| file.

Note that the commands described in this section are ``high-level'' in
the sense that they are not implemented in driver files, but rather
directly by the \pgfname-system layer. For this reason, the commands for
creating soft paths do not start with |\pgfsys@|, but rather with
|\pgfsyssoftpath@|. On the other hand, as a user you will never use
these commands directly, so they are described as part of the
low-level interface. 



\subsection{Path Creation Process}

When the user writes a command like |\draw (0bp,0bp) -- (10bp,0bp);|
quite a lot happens behind the scenes:
\begin{enumerate}
\item
  The frontend command is translated by \tikzname\ into commands
  of the basic layer. In essence, the command is translated to
  something like
\begin{codeexample}[code only]
\pgfpathmoveto{\pgfpoint{0bp}{0bp}}
\pgfpathlineto{\pgfpoint{10bp}{0bp}}
\pgfusepath{stroke}
\end{codeexample}
\item
  The |\pgfpathxxxx| command do \emph{not} directly call ``hard''
  commands like |\pgfsys@xxxx|. Instead, the command |\pgfpathmoveto|
  invokes a special command called |\pgfsyssoftpath@moveto| and
  |\pgfpathlineto| invokes |\pgfsyssoftpath@lineto|. 

  The |\pgfsyssoftpath@xxxx| commands, which are described below,
  construct a soft path. Each time such a command is used, special
  tokens are added to the end of an internal macro that stores the
  soft path currently being constructed. 
\item
  When the |\pgfusepath| is encountered, the soft path stored in
  the internal macro is ``invoked.'' Only now does a special macro
  iterate over the soft path. For each line-to or move-to
  operation on this path it calls an appropriate |\pgfsys@moveto| or
  |\pgfsys@lineto| in order to, finally, create the desired hard path,
  namely, the string of literals in the |.pdf| or |.ps| file.
\item
  After the path has been invoked, |\pgfsys@stroke| is called to
  insert the literal for stroking the path.
\end{enumerate}

Why such a complicated process? Why not have |\pgfpathlineto| directly
call |\pgfsys@lineto| and be done with it? There are two reasons:
\begin{enumerate}
\item
  The \pdf\ specification requires that a path is not interrupted by
  any non-path-construction commands. Thus, the following code will
  result in a corrupted |.pdf|:
\begin{codeexample}[code only]
\pgfsys@moveto{0}{0}
\pgfsys@setlinewidth{1}
\pgfsys@lineto{10}{0}
\pgfsys@stroke
\end{codeexample}
  Such corrupt code is \emph{tolerated} by most viewers, but not
  always. It is much better to create only (reasonably) legal code.
\item
  A soft path can still be changed, while a hard path is fixed. For
  example, one can still change the starting and end points of a soft
  path or do optimizations on it. Such transformations are not possible
  on hard paths.
\end{enumerate}


\subsection{Starting and Ending a Soft Path}

No special action must be taken in order to start the creation of a
soft path. Rather, each time a command like |\pgfsyssoftpath@lineto|
is called, a special token is added to the (global) current soft path
being constructed.

However, you can access and change the current soft path. In this way,
it is possible to store a soft path, to manipulate it, or to invoke
it.

\begin{command}{\pgfsyssoftpath@getcurrentpath\marg{macro name}}
  This command will store the current soft path in \meta{macro name}.
\end{command}

\begin{command}{\pgfsyssoftpath@setcurrentpath\marg{macro name}}
  This command will set the current soft path to be the path stored in
  \meta{macro name}. This macro should store a path that has
  previously been extracted using the |\pgfsyssoftpath@getcurrentpath|
  command and has possibly been modified subsequently.
\end{command}

\begin{command}{\pgfsyssoftpath@invokecurrentpath}
  This command will turn the current soft path in a ``hard'' path. To
  do so, it iterates over the soft path and calls an appropriate
  |\pgfsys@xxxx| command for each element of the path. Note that the
  current soft path is \emph{not changed} by this command. Thus, in
  order to start a new soft path after the old one has been invoked
  and is no longer needed, you need to set the current soft path to be
  empty. This may seems strange, but it is often useful to immediately
  use the last soft path again.
\end{command}

\begin{command}{\pgfsyssoftpath@flushcurrentpath}
  This command will invoke the current soft path and then set it to be
  empty. 
\end{command}



\subsection{Soft Path Creation Commands}

\begin{command}{\pgfsyssoftpath@moveto\marg{x}\marg{y}}
  This command appends a ``move-to'' segment to the current soft
  path. The coordinates \meta{x} and \meta{y} are given as normal
  \TeX\ dimensions.

  \example One way to draw a line:
\begin{codeexample}[code only]
\pgfsyssoftpath@moveto{0pt}{0pt}
\pgfsyssoftpath@lineto{10pt}{10pt}
\pgfsyssoftpath@flushcurrentpath
\pgfsys@stroke
\end{codeexample}
\end{command}

\begin{command}{\pgfsyssoftpath@lineto\marg{x}\marg{y}}
  Appends a ``line-to'' segment to the current soft path. 
\end{command}

\begin{command}{\pgfsyssoftpath@curveto\marg{a}\marg{b}\marg{c}\marg{d}\marg{x}\marg{y}}
  Appends a ``curve-to'' segment to the current soft path with controls
  $(a,b)$ and $(c,d)$.
\end{command}

\begin{command}{\pgfsyssoftpath@rect\marg{lower left x}\marg{lower left y}\marg{width}\marg{height}}
  Appends a rectangle segment to the current soft path. 
\end{command}

\begin{command}{\pgfsyssoftpath@closepath}
  Appends a ``close-path'' segment to the current soft path. 
\end{command}




\subsection{The Soft Path Data Structure}

A soft path is stored in a standardized way, which makes it possible to
modify it before it becomes ``hard.'' Basically, a soft path is a long
sequence of triples. Each triple starts with a \emph{token} that
identifies what is going on. This token is followed by two dimensions in
braces. For example, the following is a soft path that means ``the
path starts at $(0\mathrm{bp}, 0\mathrm{bp})$ and then
continues in a straight line to $(10\mathrm{bp},
0\mathrm{bp})$.''

\begin{codeexample}[code only]
\pgfsyssoftpath@movetotoken{0bp}{0bp}\pgfsyssoftpath@linetotoken{10bp}{0bp}
\end{codeexample}

A curve-to is hard to express in this way since we need six numbers to
express it, not two. For this reasons, a curve-to is expressed using
three triples as follows: The command
\begin{codeexample}[code only]
\pgfsyssoftpath@curveto{1bp}{2bp}{3bp}{4bp}{5bp}{6bp}
\end{codeexample}
\noindent
results in the following three triples:
\begin{codeexample}[code only]
\pgfsyssoftpath@curvetosupportatoken{1bp}{2bp}
\pgfsyssoftpath@curvetosupportbtoken{3bp}{4bp}
\pgfsyssoftpath@curvetotoken{5bp}{6bp}
\end{codeexample}

These three triples must always ``remain together.'' Thus, a lonely
|supportbtoken| is forbidden.

In details, the following tokens exist:
\begin{itemize}
\item
  \declare{|\pgfsyssoftpath@movetotoken|} indicates a move-to
  operation. The two following numbers indicate the position to which
  the current point should be moved.
\item
  \declare{|\pgfsyssoftpath@linetotoken|} indicates a line-to
  operation. 
\item
  \declare{|\pgfsyssoftpath@curvetosupportatoken|} indicates the first
  control point of a curve-to operation. The triple must be followed
  by a |\pgfsyssoftpath@curvetosupportbtoken|.
\item
  \declare{|\pgfsyssoftpath@curvetosupportbtoken|} indicates the second
  control point of a curve-to operation. The triple must be followed
  by a |\pgfsyssoftpath@curvetotoken|.
\item
  \declare{|\pgfsyssoftpath@curvetotoken|} indicates the target
  of a curve-to operation.
\item
  \declare{|\pgfsyssoftpath@rectcornertoken|} indicates the corner of
  a rectangle on the soft path. The triple must be followed
  by a |\pgfsyssoftpath@rectsizetoken|.
\item
  \declare{|\pgfsyssoftpath@rectsizetoken|} indicates the size of
  a rectangle on the soft path.
\item
  \declare{|\pgfsyssoftpath@closepath|} indicates that the subpath
  begun with the last move-to operation should be closed. The parameter
  numbers are currently not important, but if set to anything
  different from |{0pt}{0pt}|, they should be set to the coordinate of
  the original move-to operation to which the path ``returns'' now.
\end{itemize}





% Copyright 2006 by Till Tantau
%
% This file may be distributed and/or modified
%
% 1. under the LaTeX Project Public License and/or
% 2. under the GNU Free Documentation License.
%
% See the file doc/generic/pgf/licenses/LICENSE for more details.

\section{The Protocol Subsystem}

\label{section-protocols}

\makeatletter

This section describes commands for \emph{protocolling} literal text
created by \pgfname. The idea is that some literal text, like the string
of commands used to draw an arrow head, will be used over and over
again in a picture. It is then much more efficient to compute the
necessary literal text just once and to quickly insert it ``in a
single sweep.''

When protocolling is ``switched on,'' there is a ``current protocol''
to which literal text gets appended. Once all commands that needed to
be protocolled have been issued, the protocol can be obtained and
stored using |\pgfsysprotocol@getcurrentprotocol|. At any point, the
current protocol can be changed using a corresponding setting
command. Finally, |\pgfsysprotocol@invokecurrentprotocol| is used to
insert the protocolled commands into the |.pdf| or |.dvi| file.

Only those |\pgfsys@| commands can be protocolled that use the
command |\pgfsysprotocol@literal| internally. For example, the
definition of |\pgfsys@moveto| in |pgfsys-common-pdf.def| is
\begin{codeexample}[code only]
\def\pgfsys@moveto#1#2{\pgfsysprotocol@literal{#1 #2 m}}
\end{codeexample}
All ``normal'' system-level commands can be protocolled. However,
commands for creating or invoking shadings, images, or whole pictures
require special |\special|'s and cannot be protocolled.

\begin{command}{\pgfsysprotocol@literalbuffered\marg{literal text}}
  Adds the \meta{literal text} to the current protocol, after it has
  been ``|\edef|ed.'' This command will always protocol.
\end{command}

\begin{command}{\pgfsysprotocol@literal\marg{literal text}}
  First calls |\pgfsysprotocol@literalbuffered| on \meta{literal
    text}. Then, if protocolling is currently switched off, the
  \meta{literal text} is passed on to |\pgfsys@invoke|.
\end{command}

\begin{command}{\pgfsysprotocol@bufferedtrue}
  Turns on protocolling. All subsequent calls of
  |\pgfsysprotocol@literal| will append their argument to the current
  protocol.
\end{command}

\begin{command}{\pgfsysprotocol@bufferedfalse}
  Turns off protocolling. Subsequent calls of
  |\pgfsysprotocol@literal| directly insert their argument into the
  current |.pdf| or |.ps|.

  Note that if the current protocol is not empty when protocolling is
  switched off, the next call to |\pgfsysprotocol@literal| will first
  flush the current protocol, that is, insert it into the file.
\end{command}

\begin{command}{\pgfsysprotocol@getcurrentprotocol\marg{macro name}}
  Stores the current protocol in \meta{macro name} for later use.
\end{command}

\begin{command}{\pgfsysprotocol@setcurrentprotocol\marg{macro name}}
  Sets the current protocol to \meta{macro name}.
\end{command}

\begin{command}{\pgfsysprotocol@invokecurrentprotocol}
  Inserts the text stored in the current protocol into the |.pdf| or
  |.dvi| file. This does \emph{not} change the current protocol.
\end{command}

\begin{command}{\pgfsysprotocol@flushcurrentprotocol}
  First inserts the current protocol, then sets the current protocol
  to the empty string.
\end{command}


%%% Local Variables:
%%% mode: latex
%%% TeX-master: "pgfmanual"
%%% End:




\part{References and Index}

\vskip1cm
\begin{codeexample}[graphic=white]
\begin{tikzpicture}
  \draw[line width=0.3cm,color=red!30,line cap=round,line join=round] (0,0)--(2,0)--(2,5);
  \draw[help lines] (-2.5,-2.5) grid (5.5,7.5);
  \draw[very thick] (1,-1)--(-1,-1)--(-1,1)--(0,1)--(0,0)--
    (1,0)--(1,-1)--(3,-1)--(3,2)--(2,2)--(2,3)--(3,3)--
    (3,5)--(1,5)--(1,4)--(0,4)--(0,6)--(1,6)--(1,5)
    (3,3)--(4,3)--(4,5)--(3,5)--(3,6)
    (3,-1)--(4,-1);
  \draw[below left] (0,0) node(s){$s$};
  \draw[below left] (2,5) node(t){$t$};
  \fill (0,0) circle (0.06cm) (2,5) circle (0.06cm);
  \draw[->,rounded corners=0.2cm,shorten >=2pt]
    (1.5,0.5)-- ++(0,-1)-- ++(1,0)-- ++(0,2)-- ++(-1,0)-- ++(0,2)-- ++(1,0)--
    ++(0,1)-- ++(-1,0)-- ++(0,-1)-- ++(-2,0)-- ++(0,3)-- ++(2,0)-- ++(0,-1)--
    ++(1,0)-- ++(0,1)-- ++(1,0)-- ++(0,-1)-- ++(1,0)-- ++(0,-3)-- ++(-2,0)--
    ++(1,0)-- ++(0,-3)-- ++(1,0)-- ++(0,-1)-- ++(-6,0)-- ++(0,3)-- ++(2,0)--
    ++(0,-1)-- ++(1,0);
\end{tikzpicture}
\end{codeexample}

\printindex

%\typeout{Examples: \the\codeexamplecount}%
\end{document}



%%% Local Variables:
%%% mode: latex
%%% TeX-master: "~/pgf/doc/generic/pgf/version-for-luatex/en/pgfmanual"
%%% coding: iso-latin-1-unix
%%% End:
